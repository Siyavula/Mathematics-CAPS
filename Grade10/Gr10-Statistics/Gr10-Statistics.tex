\documentclass[a4paper,11pt]{report}

\usepackage{amsmath,amsfonts,amssymb}
\usepackage{graphics}
\usepackage{graphicx}
\usepackage{booktabs}
\usepackage{tabularx}
\usepackage{array}
\usepackage{icomma}
\usepackage{tikz, ifthen}
\usetikzlibrary{arrows,shapes,backgrounds,patterns,decorations.pathreplacing,decorations.pathmorphing}

% <Temporary FHSST definitions>
\def\Definition#1#2{\paragraph{Definition:} #1 --- #2}

\def\Identity#1#2{\paragraph{Identity:} #1 --- #2}

\newenvironment{wex}[3]%
{\rule{\linewidth}{0.5mm}
\textbf{Worked example:} #1

\textbf{Question:} #2

\textbf{Solution:} #3}%
{\rule{\linewidth}{0.5mm}}

\newcommand{\westep}[1]{\paragraph{Step:} #1}

\newenvironment{exercises}[1]%
{\rule{\linewidth}{0.5mm}
\textbf{Exercises:} #1\\}%
{\rule{\linewidth}{0.5mm}}

\newenvironment{eocexercises}[1]%
{\section{End of chapter exercises: #1}}%
{}
% </Temporary FHSST definitions>

% <Temporary spacing changes>
\setlength\parindent{0pt}
\setlength\parskip{0.5\baselineskip}
\setlength\textwidth{15cm}
\setlength\textheight{23cm}
\setlength\oddsidemargin{0.5cm}
\setlength\topmargin{0cm}
% </Temporary spacing changes>

\begin{document}
\chapter{Statistics}

\section{Introduction}

Statistics: Summarise data

How?
  Go from many numbers (individual data) to a few.

Why?
  Easier to interpret
  Want to highlight important aspects of the data
  Figure
     Side-by-side images of two datasets that are from same
     distribution but individual datapoints are different. Make point
     that the summary is closer to what we're interested in than the
     individual datapoints.

Challenges

  Need to determine which aspects are important and which ones are not
  since we're throwing away information. Don't want to throw away
  important information.

\section{Measures of central tendency}

\subsection{Mean}

Also known as average. 

\subsection{Median}

The media is literally in the middle of the data. Take a list of
numbers, sort it, find the one that sits in the middle.

\subsection{Mode}
Cannot always find the mode in ungrouped data (continuous in particular)

Need to make the point that the most probable value isn't necessarily
typical.

\subsection{Comparison}

\section{Grouping data}

Why bother... good question. Binning is generally a bad
idea. Sometimes we are given binned data. E.g. marks. General form:
range + count for each bin.

Histograms

Now estimating measures of central tendency changes because we've
already thrown away some information. The best we can do is to assume
that values are distributed uniformly in a bin. Visualise. This leads
to the following formulae for mean, median, mode.

\section{Dispersion}
The centre is not the only or event necessarily most interesting
statistic. We also want to know how data are spread around the centre.

Give example of datasets with same mean but different dispersions.

Range as (a very poor) measure of dispersion. Extremely sensitive to
outliers! Range is inter-percentile range between 0th and 100th
percentile (i.e. min and max).

\section{Percentiles}

These give a more complete picture. Some measures of central tendency
and dispersion are special cases of percentiles.

There are many percentiles, so we need to specify which one. They are
generally between 0 and 100. We have already seen the one in the
middle, the 50th percentile, the median.

For percentiles: same process as median, just different distances from
the edges.

FIGURE: Show ordered data set again, with 10th percentiles.

0th percentile is the minimum value
100th percentile is the maximum value
50th is value in the middle

\subsection{Quartiles}
Quartiles are special case of percentiles

Interquartile range is a measure of dispersion

Semi-interquartile range is half interquartile range (how silly do
define this).

\subsection{Five number summary}

Visualisation in the box and whisker plot.

-----

Need summary figure of different statistics -- preferably to highlight
how they are the same and how they are different. Classic example:
mode v mean v median.

Need example of something that doesn't follow a bell-shape distribution.

\end{document}
