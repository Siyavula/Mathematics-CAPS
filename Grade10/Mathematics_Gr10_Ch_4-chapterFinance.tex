         \chapter{Finance and Growth}
    \setcounter{figure}{1}
    \setcounter{subfigure}{1}
    \label{5925cb6120ab0c0f7c78bd2516b027ff}
%          \section{ Introduction and simple interest}
%     \nopagebreak
%             \label{m39332} $ \hspace{-5pt}\begin{array}{cccccccccccc}   \includegraphics[width=0.75cm]{col11306.imgs/summary_fullmarks.png} &   \end{array} $ \hspace{2 pt}\raisebox{-5 pt}{} {(section shortcode: MG10027 )} \par 
%     
%     
%     
%     
%     
%     
%   
%     \label{m39332*cid2}
%             \subsection{ Introduction}
%             \nopagebreak
      \label{m39332*cid4}
            \section{ Being Interested in Interest}
            \nopagebreak
      \label{m39332*id66349}Should you ever find yourself stuck with a mathematics question on a television quiz show, you will probably wish you had remembered how many even prime numbers there are between 1 and 100 for the sake of R1~000~000. And who does not want to be a millionaire, right?\par 
      \label{m39332*id66356}Welcome to the Grade 10 Finance Chapter, where we apply maths skills to everyday financial situations that you are likely to face both now and along your journey to purchasing your first private jet.\par 
      \label{m39332*id66361}If you master the techniques in this chapter, you will grasp the concept of \textsl{compound interest}, and how it can ruin your fortunes if you have credit card debt, or make you millions if you successfully invest your hard-earned money. You will also understand the effects of fluctuating exchange rates, and its impact on your spending power during your overseas holidays!\par \label{m39332*eip-810}Before we begin this chapter it is worth noting that the vast majority of countries use a decimal currency system. This simply means that countries use a currency system that works with powers of ten, for example in South Africa we have 100 (10 squared) cents in a rand. In America there are 100 cents in a dollar. Another way of saying this is that the country has one basic unit of currency and a sub-unit which is a power of 10 of the major unit. This means that, if we ignore the effect of exchange rates, we can essentially substitute rands for dollars or rands for pounds.\par 
      \label{m39332*id69176}If you had R1~000, you could either keep it in your wallet, or deposit it in a bank account. If it stayed in your wallet, you could spend it any time you wanted. If the bank looked after it for you, then they could spend it, with the plan of making profit from it. The bank usually ``pays" you to deposit it into an account, as a way of encouraging you to bank it with them, This payment is like a reward, which provides you with a reason to leave it with the bank for a while, rather than keeping the money in your wallet.\par 
      \label{m39332*id69184}We call this reward "interest".\par 
      \label{m39332*id69188}If you deposit money into a bank account, you are effectively lending money to the bank - and you can expect to receive interest in return. Similarly, if you borrow money from a bank (or from a department store, or a car dealership, for example) then you can expect to have to pay interest on the loan. That is the price of borrowing money.\par 
      \label{m39332*id69194}The concept is simple, yet it is core to the world of finance. Accountants, actuaries and bankers, for example, could spend their entire working career dealing with the effects of interest on financial matters.\par 
      \label{m39332*id69199}In this chapter you will be introduced to the concept of financial mathematics - and given the tools to cope with even advanced concepts and problems.\par 
\label{m39332*notfhsst!!!underscore!!!id901}
\begin{tabular}{cc}
	   \hspace*{-50pt}\raisebox{-8 mm}{ \includegraphics[width=0.5in]{col11306.imgs/pstip2.png}  }& 
	\begin{minipage}{0.85\textwidth}
	\begin{note}
      {tip: }Interest
	\end{note}
	\end{minipage}
	\end{tabular}
	\par
      \label{m39332*id69209}The concepts in this chapter are simple - we are just looking at the same idea, but from many different angles. The best way to learn from this chapter is to do the examples yourself, as you work your way through. Do not just take our word for it!\par 
    \label{m39332*cid5}
            \section{ Simple Interest}
            \nopagebreak
\par
            \label{m39332*fhsst!!!underscore!!!id906}\begin{definition}
	  \begin{tabular*}{15 cm}{m{15 mm}m{}}
	\hspace*{-50pt}  \includegraphics[width=0.5in]{col11306.imgs/psflag2.png}   & \Definition{   \label{id2476740}\textbf{ Simple Interest }} { \label{m39332*meaningfhsst!!!underscore!!!id906}
      \label{m39332*id69230}Simple interest is where you earn interest on the initial amount that you invested, but not interest on interest. \par 
       } 
      \end{tabular*}
      \end{definition}
      \label{m39332*id69242}As an easy example of simple interest, consider how much you will get by investing R1~000 for 1 year with a bank that pays you 5\% simple interest. At the end of the year, you will get an interest of:\par 
      \label{m39332*id69248}\nopagebreak\noindent{}\settowidth{\mymathboxwidth}{\begin{equation}
    \begin{array}{ccc}\hfill \mathrm{Interest}& =& R1\phantom{\rule{3.33333pt}{0ex}}000\ensuremath{\times}5\%\hfill \\ & =& R1\phantom{\rule{3.33333pt}{0ex}}000\ensuremath{\times}\frac{5}{100}\hfill \\ & =& R1\phantom{\rule{3.33333pt}{0ex}}000\ensuremath{\times}0,05\hfill \\ & =& R50\hfill \end{array}\tag{3.1}
      \end{equation}
    }
    \typeout{Columnwidth = \the\columnwidth}\typeout{math as usual width = \the\mymathboxwidth}
    \ifthenelse{\lengthtest{\mymathboxwidth < \columnwidth}}{% if the math fits, do it again, for real
    \begin{equation}
    \begin{array}{ccc}\hfill \mathrm{Interest}& =& R1\phantom{\rule{3.33333pt}{0ex}}000\ensuremath{\times}5\%\hfill \\ & =& R1\phantom{\rule{3.33333pt}{0ex}}000\ensuremath{\times}\frac{5}{100}\hfill \\ & =& R1\phantom{\rule{3.33333pt}{0ex}}000\ensuremath{\times}0,05\hfill \\ & =& R50\hfill \end{array}\tag{3.1}
      \end{equation}
    }{% else, if it doesn't fit
    \setlength{\mymathboxwidth}{\columnwidth}
      \addtolength{\mymathboxwidth}{-48pt}
    \par\vspace{12pt}\noindent\begin{minipage}{\columnwidth}
    \parbox[t]{\mymathboxwidth}{\large\begin{math}
    \mathrm{Interest}=R1\phantom{\rule{3.33333pt}{0ex}}000\ensuremath{\times}5\%=R1\phantom{\rule{3.33333pt}{0ex}}000\ensuremath{\times}\frac{5}{100}=R1\phantom{\rule{3.33333pt}{0ex}}000\ensuremath{\times}0,05=R50\end{math}}\hfill
    \parbox[t]{48pt}{\raggedleft 
    (3.1)}
    \end{minipage}\vspace{12pt}\par
    }% end of conditional for this bit of math
    \typeout{math as usual width = \the\mymathboxwidth}
      \label{m39332*id69370}So, with an ``opening balance" of R1~000 at the start of the year, your ``closing balance" at the end of the year will therefore be:\par 
      \label{m39332*id69376}\nopagebreak\noindent{}\settowidth{\mymathboxwidth}{\begin{equation}
    \begin{array}{ccc}\hfill \mathrm{Closing\; Balance}& =& \mathrm{Opening\; Balance}+\mathrm{Interest}\hfill \\ & =& \mathrm{R}1\phantom{\rule{3.33333pt}{0ex}}000+\mathrm{R}50\hfill \\ & =& \mathrm{R}1\phantom{\rule{3.33333pt}{0ex}}050\hfill \end{array}\tag{3.2}
      \end{equation}
    }
    \typeout{Columnwidth = \the\columnwidth}\typeout{math as usual width = \the\mymathboxwidth}
    \ifthenelse{\lengthtest{\mymathboxwidth < \columnwidth}}{% if the math fits, do it again, for real
    \begin{equation}
    \begin{array}{ccc}\hfill \mathrm{Closing\; Balance}& =& \mathrm{Opening\; Balance}+\mathrm{Interest}\hfill \\ & =& \mathrm{R}1\phantom{\rule{3.33333pt}{0ex}}000+\mathrm{R}50\hfill \\ & =& \mathrm{R}1\phantom{\rule{3.33333pt}{0ex}}050\hfill \end{array}\tag{3.2}
      \end{equation}
    }{% else, if it doesn't fit
    \setlength{\mymathboxwidth}{\columnwidth}
      \addtolength{\mymathboxwidth}{-48pt}
    \par\vspace{12pt}\noindent\begin{minipage}{\columnwidth}
    \parbox[t]{\mymathboxwidth}{\large\begin{math}
    \mathrm{Closing\; Balance}=\mathrm{Opening\; Balance}+\mathrm{Interest}=\mathrm{R}1\phantom{\rule{3.33333pt}{0ex}}000+\mathrm{R}50=\mathrm{R}1\phantom{\rule{3.33333pt}{0ex}}050\end{math}}\hfill
    \parbox[t]{48pt}{\raggedleft 
    (3.2)}
    \end{minipage}\vspace{12pt}\par
    }% end of conditional for this bit of math
    \typeout{math as usual width = \the\mymathboxwidth}
      \label{m39332*id69469}We sometimes call the opening balance in financial calculations the \textsl{Principal}, which is abbreviated as \begin{math}P\end{math} (R1~000 in the example). The interest rate is usually labelled \begin{math}i\end{math} (5\% in the example), and the interest amount (in Rand terms) is labelled \begin{math}I\end{math} (R50 in the example).\par 
      \label{m39332*id69507}So we can see that:\par 
      \label{m39332*uid28}\nopagebreak\noindent{}
        \settowidth{\mymathboxwidth}{\begin{equation}
    I=P\ensuremath{\times}i\tag{3.3}
      \end{equation}
    }
    \typeout{Columnwidth = \the\columnwidth}\typeout{math as usual width = \the\mymathboxwidth}
    \ifthenelse{\lengthtest{\mymathboxwidth < \columnwidth}}{% if the math fits, do it again, for real
    \begin{equation}
    I=P\ensuremath{\times}i\tag{3.3}
      \end{equation}
    }{% else, if it doesn't fit
    \setlength{\mymathboxwidth}{\columnwidth}
      \addtolength{\mymathboxwidth}{-48pt}
    \par\vspace{12pt}\noindent\begin{minipage}{\columnwidth}
    \parbox[t]{\mymathboxwidth}{\large\begin{math}
    I=P\ensuremath{\times}i\end{math}}\hfill
    \parbox[t]{48pt}{\raggedleft 
    (3.3)}
    \end{minipage}\vspace{12pt}\par
    }% end of conditional for this bit of math
    \typeout{math as usual width = \the\mymathboxwidth}
      \label{m39332*id69536}and\par 
      \label{m39332*id69539}\nopagebreak\noindent{}\settowidth{\mymathboxwidth}{\begin{equation}
    \begin{array}{ccc}\hfill \mathrm{Closing\; Balance}& =& \mathrm{Opening\; Balance}+\mathrm{Interest}\hfill \\ & =& P+I\hfill \\ & =& P+\left(P\ensuremath{\times}i\right)\hfill \\ & =& P\left(1+i\right)\hfill \end{array}\tag{3.4}
      \end{equation}
    }
    \typeout{Columnwidth = \the\columnwidth}\typeout{math as usual width = \the\mymathboxwidth}
    \ifthenelse{\lengthtest{\mymathboxwidth < \columnwidth}}{% if the math fits, do it again, for real
    \begin{equation}
    \begin{array}{ccc}\hfill \mathrm{Closing\; Balance}& =& \mathrm{Opening\; Balance}+\mathrm{Interest}\hfill \\ & =& P+I\hfill \\ & =& P+\left(P\ensuremath{\times}i\right)\hfill \\ & =& P\left(1+i\right)\hfill \end{array}\tag{3.4}
      \end{equation}
    }{% else, if it doesn't fit
    \setlength{\mymathboxwidth}{\columnwidth}
      \addtolength{\mymathboxwidth}{-48pt}
    \par\vspace{12pt}\noindent\begin{minipage}{\columnwidth}
    \parbox[t]{\mymathboxwidth}{\large\begin{math}
    \mathrm{Closing\; Balance}=\mathrm{Opening\; Balance}+\mathrm{Interest}=P+I=P+\left(P\ensuremath{\times}i\right)=P\left(1+i\right)\end{math}}\hfill
    \parbox[t]{48pt}{\raggedleft 
    (3.4)}
    \end{minipage}\vspace{12pt}\par
    }% end of conditional for this bit of math
    \typeout{math as usual width = \the\mymathboxwidth}
      \label{m39332*id69642}This is how you calculate simple interest. It is not a complicated formula, which is just as well because you are going to see a lot of it!\par 
      \label{m39332*uid29}
            \subsection{ Simple Interest Monthly Calculations}
            \nopagebreak
        \label{m39332*id69658}You might be wondering to yourself:\par 
        \label{m39332*id69662}\begin{enumerate}[noitemsep, label=\textbf{\arabic*}. ] 
            \label{m39332*uid30}\item how much interest will you be paid if you only leave the money in the account for 3 months, or
\label{m39332*uid31}\item what if you leave it there for 3 years?
\end{enumerate}
        \label{m39332*id69691}
It is actually quite simple - which is why they call it \textbf{Simple Interest}.\par 
        \label{m39332*id69703}\begin{enumerate}[noitemsep, label=\textbf{\arabic*}. ] 
            \label{m39332*uid32}\item Three months is 1/4 of a year, so you would only get 1/4 of a full year's interest, which is: \begin{math}1/4\ensuremath{\times}\left(P\ensuremath{\times}i\right)\end{math}. The closing balance would therefore be:
\label{m39332*id69748}\nopagebreak\noindent{}\settowidth{\mymathboxwidth}{\begin{equation}
    \begin{array}{ccc}\hfill \mathrm{Closing\; Balance}& =& P+1/4\ensuremath{\times}\left(P\ensuremath{\times}i\right)\hfill \\ & =& P\left(1+\left(1/4\right)i\right)\hfill \end{array}\tag{3.5}
      \end{equation}
    }
    \typeout{Columnwidth = \the\columnwidth}\typeout{math as usual width = \the\mymathboxwidth}
    \ifthenelse{\lengthtest{\mymathboxwidth < \columnwidth}}{% if the math fits, do it again, for real
    \begin{equation}
    \begin{array}{ccc}\hfill \mathrm{Closing\; Balance}& =& P+1/4\ensuremath{\times}\left(P\ensuremath{\times}i\right)\hfill \\ & =& P\left(1+\left(1/4\right)i\right)\hfill \end{array}\tag{3.5}
      \end{equation}
    }{% else, if it doesn't fit
    \setlength{\mymathboxwidth}{\columnwidth}
      \addtolength{\mymathboxwidth}{-48pt}
    \par\vspace{12pt}\noindent\begin{minipage}{\columnwidth}
    \parbox[t]{\mymathboxwidth}{\large\begin{math}
    \mathrm{Closing\; Balance}=P+1/4\ensuremath{\times}\left(P\ensuremath{\times}i\right)=P\left(1+\left(1/4\right)i\right)\end{math}}\hfill
    \parbox[t]{48pt}{\raggedleft 
    (3.5)}
    \end{minipage}\vspace{12pt}\par
    }% end of conditional for this bit of math
    \typeout{math as usual width = \the\mymathboxwidth}
    \label{m39332*uid33}\item For 3 years, you would get three years' worth of interest, being: \begin{math}3\ensuremath{\times}\left(P\ensuremath{\times}i\right)\end{math}. The closing balance at the end of the three year period would be:
\label{m39332*id69871}\nopagebreak\noindent{}\settowidth{\mymathboxwidth}{\begin{equation}
    \begin{array}{ccc}\hfill \mathrm{Closing\; Balance}& =& P+3\ensuremath{\times}\left(P\ensuremath{\times}i\right)\hfill \\ & =& P\ensuremath{\times}\left(1+\left(3\right)i\right)\hfill \end{array}\tag{3.6}
      \end{equation}
    }
    \typeout{Columnwidth = \the\columnwidth}\typeout{math as usual width = \the\mymathboxwidth}
    \ifthenelse{\lengthtest{\mymathboxwidth < \columnwidth}}{% if the math fits, do it again, for real
    \begin{equation}
    \begin{array}{ccc}\hfill \mathrm{Closing\; Balance}& =& P+3\ensuremath{\times}\left(P\ensuremath{\times}i\right)\hfill \\ & =& P\ensuremath{\times}\left(1+\left(3\right)i\right)\hfill \end{array}\tag{3.6}
      \end{equation}
    }{% else, if it doesn't fit
    \setlength{\mymathboxwidth}{\columnwidth}
      \addtolength{\mymathboxwidth}{-48pt}
    \par\vspace{12pt}\noindent\begin{minipage}{\columnwidth}
    \parbox[t]{\mymathboxwidth}{\large\begin{math}
    \mathrm{Closing\; Balance}=P+3\ensuremath{\times}\left(P\ensuremath{\times}i\right)=P\ensuremath{\times}\left(1+\left(3\right)i\right)\end{math}}\hfill
    \parbox[t]{48pt}{\raggedleft 
    (3.6)}
    \end{minipage}\vspace{12pt}\par
    }% end of conditional for this bit of math
    \typeout{math as usual width = \the\mymathboxwidth}
    \end{enumerate}
        \label{m39332*id69952}If you look carefully at the similarities between the two answers above, we can generalise the result. If you invest your money (\begin{math}P\end{math}) in an account which pays a rate of interest (\begin{math}i\end{math}) for a period of time (\begin{math}n\end{math} years), then, using the symbol \begin{math}A\end{math} for the Closing Balance:\par 
        \label{m39332*uid34}\nopagebreak\noindent{}
          \settowidth{\mymathboxwidth}{\begin{equation}
    A=P\left(1+i\ensuremath{\cdot}n\right)\tag{3.7}
      \end{equation}
    }
    \typeout{Columnwidth = \the\columnwidth}\typeout{math as usual width = \the\mymathboxwidth}
    \ifthenelse{\lengthtest{\mymathboxwidth < \columnwidth}}{% if the math fits, do it again, for real
    \begin{equation}
    A=P\left(1+i\ensuremath{\cdot}n\right)\tag{3.7}
      \end{equation}
    }{% else, if it doesn't fit
    \setlength{\mymathboxwidth}{\columnwidth}
      \addtolength{\mymathboxwidth}{-48pt}
    \par\vspace{12pt}\noindent\begin{minipage}{\columnwidth}
    \parbox[t]{\mymathboxwidth}{\large\begin{math}
    A=P\left(1+i\ensuremath{\cdot}n\right)\end{math}}\hfill
    \parbox[t]{48pt}{\raggedleft 
    (3.7)}
    \end{minipage}\vspace{12pt}\par
    }% end of conditional for this bit of math
    \typeout{math as usual width = \the\mymathboxwidth}
        \label{m39332*id70029}As we have seen, this works when \begin{math}n\end{math} is a fraction of a year and also when \begin{math}n\end{math} covers several years.\par 
\label{m39332*notfhsst!!!underscore!!!id1166}
\begin{tabular}{cc}
	\hspace*{-50pt}\raisebox{-8 mm}{\hspace{-0.2in}\includegraphics[width=0.75in]{col11306.imgs/psfact2.png} } & 
	\begin{minipage}{0.85\textwidth}
	\begin{note}
      {important: }Annual Rates means Yearly rates. and p.a.(per annum) = per year
	\end{note}
	\end{minipage}
	\end{tabular}
	\par
\label{m39332*secfhsst!!!underscore!!!id1168}\vspace{.5cm} 
      \noindent
      \hspace*{-30pt}\includegraphics[width=0.5in]{col11306.imgs/pspencil2.png}   \raisebox{25mm}{   
      \begin{mdframed}[linewidth=4, leftmargin=40, rightmargin=40]  
      \begin{exercise}
    \noindent\textbf{Exercise 3.1:  Simple Interest }
        \label{m39332*probfhsst!!!underscore!!!id1169}
        \label{m39332*id70072}If I deposit R1~000 into a special bank account which pays a Simple Interest of 7\% for 3 years, how much will I get back at the end of this term? \par 
        \vspace{5pt}
        \label{m39332*solfhsst!!!underscore!!!id1172}\noindent\textbf{Solution to Exercise } \label{m39332*listfhsst!!!underscore!!!id1172}\begin{enumerate}[noitemsep, label=\textbf{Step} \textbf{\arabic*}. ] 
            \leftskip=20pt\rightskip=\leftskip\item  
        \label{m39332*id70098}\begin{itemize}[noitemsep]
            \leftskip=20pt\rightskip=\leftskip\label{m39332*uid35}\item opening balance, \begin{math}P=\mathrm{R}1\phantom{\rule{3.33333pt}{0ex}}000\end{math}\label{m39332*uid36}\item interest rate, \begin{math}i=7\%\end{math}\label{m39332*uid37}\item period of time, \begin{math}n=3\phantom{\rule{3.33333pt}{0ex}}\mathrm{years}\end{math}\end{itemize}
        \label{m39332*id70196}We are required to find the closing balance (A).\par 
        \item  
        \label{m39332*id70204}We know from (3.7) that:\par 
        \label{m39332*id70212}\nopagebreak\noindent{}\settowidth{\mymathboxwidth}{\begin{equation}
    \mathrm{A}=\mathrm{P}\left(1+\mathrm{i}\ensuremath{\cdot}\mathrm{n}\right)\tag{3.8}
      \end{equation}
    }
    \typeout{Columnwidth = \the\columnwidth}\typeout{math as usual width = \the\mymathboxwidth}
    \ifthenelse{\lengthtest{\mymathboxwidth < \columnwidth}}{% if the math fits, do it again, for real
    \begin{equation}
    \mathrm{A}=\mathrm{P}\left(1+\mathrm{i}\ensuremath{\cdot}\mathrm{n}\right)\tag{3.8}
      \end{equation}
    }{% else, if it doesn't fit
    \setlength{\mymathboxwidth}{\columnwidth}
      \addtolength{\mymathboxwidth}{-48pt}
    \par\vspace{12pt}\noindent\begin{minipage}{\columnwidth}
    \parbox[t]{\mymathboxwidth}{\large\begin{math}
    \mathrm{A}=\mathrm{P}\left(1+\mathrm{i}\ensuremath{\cdot}\mathrm{n}\right)\end{math}}\hfill
    \parbox[t]{48pt}{\raggedleft 
    (3.8)}
    \end{minipage}\vspace{12pt}\par
    }% end of conditional for this bit of math
    \typeout{math as usual width = \the\mymathboxwidth}
        \item  
        \label{m39332*id70268}\nopagebreak\noindent{}
          \settowidth{\mymathboxwidth}{\begin{equation}
    \begin{array}{ccc}\hfill \mathrm{A}& =& P\left(1+i\ensuremath{\cdot}n\right)\hfill \\ & =& R1\phantom{\rule{3.33333pt}{0ex}}000\left(1+3\ensuremath{\times}7\%\right)\hfill \\ & =& R1\phantom{\rule{3.33333pt}{0ex}}210\hfill \end{array}\tag{3.9}
      \end{equation}
    }
    \typeout{Columnwidth = \the\columnwidth}\typeout{math as usual width = \the\mymathboxwidth}
    \ifthenelse{\lengthtest{\mymathboxwidth < \columnwidth}}{% if the math fits, do it again, for real
    \begin{equation}
    \begin{array}{ccc}\hfill \mathrm{A}& =& P\left(1+i\ensuremath{\cdot}n\right)\hfill \\ & =& R1\phantom{\rule{3.33333pt}{0ex}}000\left(1+3\ensuremath{\times}7\%\right)\hfill \\ & =& R1\phantom{\rule{3.33333pt}{0ex}}210\hfill \end{array}\tag{3.9}
      \end{equation}
    }{% else, if it doesn't fit
    \setlength{\mymathboxwidth}{\columnwidth}
      \addtolength{\mymathboxwidth}{-48pt}
    \par\vspace{12pt}\noindent\begin{minipage}{\columnwidth}
    \parbox[t]{\mymathboxwidth}{\large\begin{math}
    \mathrm{A}=P\left(1+i\ensuremath{\cdot}n\right)=R1\phantom{\rule{3.33333pt}{0ex}}000\left(1+3\ensuremath{\times}7\%\right)=R1\phantom{\rule{3.33333pt}{0ex}}210\end{math}}\hfill
    \parbox[t]{48pt}{\raggedleft 
    (3.9)}
    \end{minipage}\vspace{12pt}\par
    }% end of conditional for this bit of math
    \typeout{math as usual width = \the\mymathboxwidth}
        \item  
        \label{m39332*id70374}The closing balance after 3 years of saving R1~000 at an interest rate of 7\% is R1~210. \par 
        \end{enumerate}
    \end{exercise}
    \end{mdframed}
    }
    \noindent
\label{m39332*secfhsst!!!underscore!!!id1272}\vspace{.5cm} 
      \noindent
      \hspace*{-30pt}\includegraphics[width=0.5in]{col11306.imgs/pspencil2.png}   \raisebox{25mm}{   
      \begin{mdframed}[linewidth=4, leftmargin=40, rightmargin=40]  
      \begin{exercise}
    \noindent\textbf{Exercise 3.2:  Calculating \begin{math}n\end{math} }
        \label{m39332*probfhsst!!!underscore!!!id1273}
        \label{m39332*id70413}If I deposit R30~000 into a special bank account which pays a Simple Interest of 7.5\%, for how many years must I invest this amount to generate R45~000? \par 
        \vspace{5pt}
        \label{m39332*solfhsst!!!underscore!!!id1276}\noindent\textbf{Solution to Exercise } \label{m39332*listfhsst!!!underscore!!!id1276}\begin{enumerate}[noitemsep, label=\textbf{Step} \textbf{\arabic*}. ] 
            \leftskip=20pt\rightskip=\leftskip\item  
        \label{m39332*id70439}\begin{itemize}[noitemsep]
            \leftskip=20pt\rightskip=\leftskip\label{m39332*uid38}\item opening balance, \begin{math}P=\mathrm{R}30\phantom{\rule{3.33333pt}{0ex}}000\end{math}\label{m39332*uid39}\item interest rate, \begin{math}i=7,5\%\end{math}\label{m39332*uid40}\item closing balance, \begin{math}A=\mathrm{R}45\phantom{\rule{3.33333pt}{0ex}}000\end{math}\end{itemize}
        \label{m39332*id70545}We are required to find the number of years.\par 
        \item  
        \label{m39332*id70552}We know from (3.7) that:\par 
        \label{m39332*id70560}\nopagebreak\noindent{}\settowidth{\mymathboxwidth}{\begin{equation}
    A=\mathrm{P}\left(1+\mathrm{i}\ensuremath{\cdot}\mathrm{n}\right)\tag{3.10}
      \end{equation}
    }
    \typeout{Columnwidth = \the\columnwidth}\typeout{math as usual width = \the\mymathboxwidth}
    \ifthenelse{\lengthtest{\mymathboxwidth < \columnwidth}}{% if the math fits, do it again, for real
    \begin{equation}
    A=\mathrm{P}\left(1+\mathrm{i}\ensuremath{\cdot}\mathrm{n}\right)\tag{3.10}
      \end{equation}
    }{% else, if it doesn't fit
    \setlength{\mymathboxwidth}{\columnwidth}
      \addtolength{\mymathboxwidth}{-48pt}
    \par\vspace{12pt}\noindent\begin{minipage}{\columnwidth}
    \parbox[t]{\mymathboxwidth}{\large\begin{math}
    A=\mathrm{P}\left(1+\mathrm{i}\ensuremath{\cdot}\mathrm{n}\right)\end{math}}\hfill
    \parbox[t]{48pt}{\raggedleft 
    (3.10)}
    \end{minipage}\vspace{12pt}\par
    }% end of conditional for this bit of math
    \typeout{math as usual width = \the\mymathboxwidth}
        \item  
        \label{m39332*id70614}\nopagebreak\noindent{}\settowidth{\mymathboxwidth}{\begin{equation}
    \begin{array}{ccc}\hfill A& =& P\left(1+i\ensuremath{\cdot}n\right)\hfill \\ \hfill R45\phantom{\rule{3.33333pt}{0ex}}000& =& R30\phantom{\rule{3.33333pt}{0ex}}000\left(1+n\ensuremath{\times}7,5\%\right)\hfill \\ \hfill \left(1+0,075\ensuremath{\times}n\right)& =& \frac{45000}{30000}\hfill \\ \hfill 0,075\ensuremath{\times}n& =& 1,5-1\hfill \\ \hfill n& =& \frac{0,5}{0,075}\hfill \\ \hfill n& =& 6,6666667\hfill \\ \hfill n& =& 6\phantom{\rule{1mm}{0ex}}\mathrm{years}\phantom{\rule{1mm}{0ex}}8\phantom{\rule{1mm}{0ex}}\mathrm{months}\hfill \end{array}\tag{3.11}
      \end{equation}
    }
    \typeout{Columnwidth = \the\columnwidth}\typeout{math as usual width = \the\mymathboxwidth}
    \ifthenelse{\lengthtest{\mymathboxwidth < \columnwidth}}{% if the math fits, do it again, for real
    \begin{equation}
    \begin{array}{ccc}\hfill A& =& P\left(1+i\ensuremath{\cdot}n\right)\hfill \\ \hfill R45\phantom{\rule{3.33333pt}{0ex}}000& =& R30\phantom{\rule{3.33333pt}{0ex}}000\left(1+n\ensuremath{\times}7,5\%\right)\hfill \\ \hfill \left(1+0,075\ensuremath{\times}n\right)& =& \frac{45000}{30000}\hfill \\ \hfill 0,075\ensuremath{\times}n& =& 1,5-1\hfill \\ \hfill n& =& \frac{0,5}{0,075}\hfill \\ \hfill n& =& 6,6666667\hfill \\ \hfill n& =& 6\phantom{\rule{1mm}{0ex}}\mathrm{years}\phantom{\rule{1mm}{0ex}}8\phantom{\rule{1mm}{0ex}}\mathrm{months}\hfill \end{array}\tag{3.11}
      \end{equation}
    }{% else, if it doesn't fit
    \setlength{\mymathboxwidth}{\columnwidth}
      \addtolength{\mymathboxwidth}{-48pt}
    \par\vspace{12pt}\noindent\begin{minipage}{\columnwidth}
    \parbox[t]{\mymathboxwidth}{\large\begin{math}
    A=P\left(1+i\ensuremath{\cdot}n\right)R45\phantom{\rule{3.33333pt}{0ex}}000=R30\phantom{\rule{3.33333pt}{0ex}}000\left(1+n\ensuremath{\times}7,5\%\right)\left(1+0,075\ensuremath{\times}n\right)=\frac{45000}{30000}0,075\ensuremath{\times}n=1,5-1n=\frac{0,5}{0,075}n=6,6666667n=6\phantom{\rule{1mm}{0ex}}\mathrm{years}\phantom{\rule{1mm}{0ex}}8\phantom{\rule{1mm}{0ex}}\mathrm{months}\end{math}}\hfill
    \parbox[t]{48pt}{\raggedleft 
    (3.11)}
    \end{minipage}\vspace{12pt}\par
    }% end of conditional for this bit of math
    \typeout{math as usual width = \the\mymathboxwidth}
        \item  
        \label{m39332*id70878}The period is 6 years and 8 months for R30~000 to generate R45~000 at a simple interest rate of 7,5\%. If we were asked for the nearest whole number of years, we would have to invest the money for 7 years.\par 
        \end{enumerate}
    \end{exercise}
    \end{mdframed}
    }
    \noindent
      \label{m39332*uid41}
            \subsection{ Other Applications of the Simple Interest Formula}
            \nopagebreak
\par
            \label{m39332*secfhsst!!!underscore!!!id1464}\vspace{.5cm} 
      \noindent
      \hspace*{-30pt}\includegraphics[width=0.5in]{col11306.imgs/pspencil2.png}   \raisebox{25mm}{   
      \begin{mdframed}[linewidth=4, leftmargin=40, rightmargin=40]  
      \begin{exercise}
    \noindent\textbf{Exercise 3.3:  Hire-Purchase }
        \label{m39332*probfhsst!!!underscore!!!id1465}
        \label{m39332*id70919}Troy is keen to buy an additional hard drive for his laptop advertised for R 2~500 on the internet. There is an option of paying a 10\% deposit then making 24 monthly payments using a hire-purchase agreement where interest is calculated at 7,5\% p.a. simple interest. Calculate what Troy's monthly payments will be. \par 
        \vspace{5pt}
        \label{m39332*solfhsst!!!underscore!!!id1468}\noindent\textbf{Solution to Exercise } \label{m39332*listfhsst!!!underscore!!!id1468}\begin{enumerate}[noitemsep, label=\textbf{Step} \textbf{\arabic*}. ] 
            \leftskip=20pt\rightskip=\leftskip\item  
        \label{m39332*id70948}A new opening balance is required, as the 10\% deposit is paid in cash.\par 
        \label{m39332*id70952}\begin{itemize}[noitemsep]
            \leftskip=20pt\rightskip=\leftskip\label{m39332*uid42}\item 10\% of R 2~500 = R250
\label{m39332*uid43}\item new opening balance, \begin{math}P=\mathrm{R}2\phantom{\rule{3.33333pt}{0ex}}500-\mathrm{R}250=\mathrm{R}2\phantom{\rule{3.33333pt}{0ex}}250\end{math}\label{m39332*uid44}\item interest rate, \begin{math}i=7,5\%\end{math}\label{m39332*uid45}\item period of time, \begin{math}n=2\phantom{\rule{3.33333pt}{0ex}}\mathrm{years}\end{math}\end{itemize}
        \label{m39332*id71089}
We are required to find the closing balance (A) and then the monthly payments.\par 
        \item  
        \label{m39332*id71099}We know from (3.7) that:\par 
        \label{m39332*id71107}\nopagebreak\noindent{}\settowidth{\mymathboxwidth}{\begin{equation}
    \mathrm{A}=\mathrm{P}\left(1+\mathrm{i}\ensuremath{\cdot}\mathrm{n}\right)\tag{3.12}
      \end{equation}
    }
    \typeout{Columnwidth = \the\columnwidth}\typeout{math as usual width = \the\mymathboxwidth}
    \ifthenelse{\lengthtest{\mymathboxwidth < \columnwidth}}{% if the math fits, do it again, for real
    \begin{equation}
    \mathrm{A}=\mathrm{P}\left(1+\mathrm{i}\ensuremath{\cdot}\mathrm{n}\right)\tag{3.12}
      \end{equation}
    }{% else, if it doesn't fit
    \setlength{\mymathboxwidth}{\columnwidth}
      \addtolength{\mymathboxwidth}{-48pt}
    \par\vspace{12pt}\noindent\begin{minipage}{\columnwidth}
    \parbox[t]{\mymathboxwidth}{\large\begin{math}
    \mathrm{A}=\mathrm{P}\left(1+\mathrm{i}\ensuremath{\cdot}\mathrm{n}\right)\end{math}}\hfill
    \parbox[t]{48pt}{\raggedleft 
    (3.12)}
    \end{minipage}\vspace{12pt}\par
    }% end of conditional for this bit of math
    \typeout{math as usual width = \the\mymathboxwidth}
        \item  
        \label{m39332*id71164}\nopagebreak\noindent{}\settowidth{\mymathboxwidth}{\begin{equation}
    \begin{array}{ccc}\hfill \mathrm{A}& =& P\left(1+i\ensuremath{\cdot}n\right)\hfill \\ & =& R2\phantom{\rule{3.33333pt}{0ex}}250\left(1+\left(2\ensuremath{\times}7,5\%\right)\right)\hfill \\ & =& R2\phantom{\rule{3.33333pt}{0ex}}587,50\hfill \\ \hfill \mathrm{Monthly\; payment}& =& 2587,50÷24\hfill \\ & =& R107,81\hfill \end{array}\tag{3.13}
      \end{equation}
    }
    \typeout{Columnwidth = \the\columnwidth}\typeout{math as usual width = \the\mymathboxwidth}
    \ifthenelse{\lengthtest{\mymathboxwidth < \columnwidth}}{% if the math fits, do it again, for real
    \begin{equation}
    \begin{array}{ccc}\hfill \mathrm{A}& =& P\left(1+i\ensuremath{\cdot}n\right)\hfill \\ & =& R2\phantom{\rule{3.33333pt}{0ex}}250\left(1+\left(2\ensuremath{\times}7,5\%\right)\right)\hfill \\ & =& R2\phantom{\rule{3.33333pt}{0ex}}587,50\hfill \\ \hfill \mathrm{Monthly\; payment}& =& 2587,50÷24\hfill \\ & =& R107,81\hfill \end{array}\tag{3.13}
      \end{equation}
    }{% else, if it doesn't fit
    \setlength{\mymathboxwidth}{\columnwidth}
      \addtolength{\mymathboxwidth}{-48pt}
    \par\vspace{12pt}\noindent\begin{minipage}{\columnwidth}
    \parbox[t]{\mymathboxwidth}{\large\begin{math}
    \mathrm{A}=P\left(1+i\ensuremath{\cdot}n\right)=R2\phantom{\rule{3.33333pt}{0ex}}250\left(1+\left(2\ensuremath{\times}7,5\%\right)\right)=R2\phantom{\rule{3.33333pt}{0ex}}587,50\mathrm{Monthly\; payment}=2587,50÷24=R107,81\end{math}}\hfill
    \parbox[t]{48pt}{\raggedleft 
    (3.13)}
    \end{minipage}\vspace{12pt}\par
    }% end of conditional for this bit of math
    \typeout{math as usual width = \the\mymathboxwidth}
        \item  
        \label{m39332*id71325}Troy's monthly payments = R 107,81
 \par 
        \end{enumerate}
    \end{exercise}
    \end{mdframed}
    }
    \noindent
        \label{m39332*id71341}Many items become less valuable as they are used and age. For example, you pay less for a second hand car than a new car of the same model. The older a car is the less you pay for it. The reduction in value with time can be due purely to wear and tear from usage but also to the development of new technology that makes the item obsolete, for example, new computers that are released force down the value of older models. The term we use to descrive the decrease in value of items with time is \textsl{depreciation}.\par 
        \label{m39332*id71354}Depreciation, like interest can be calculated on an annual basis and is often done with a rate or percentage change per year. It is like ''negative'' interest. The simplest way to do depreciation is to assume a constant rate per year, which we will call simple depreciation. There are more complicated models for depreciation but we won't deal with them here.\par 
\label{m39332*secfhsst!!!underscore!!!id1613}\vspace{.5cm} 
      \noindent
      \hspace*{-30pt}\includegraphics[width=0.5in]{col11306.imgs/pspencil2.png}   \raisebox{25mm}{   
      \begin{mdframed}[linewidth=4, leftmargin=40, rightmargin=40]  
      \begin{exercise}
    \noindent\textbf{Exercise 3.4:  Depreciation }
        \label{m39332*probfhsst!!!underscore!!!id1614}
        \label{m39332*id71376}Seven years ago, Tjad's drum kit cost him R12~500. It has now been valued at R2~300. What rate of simple depreciation does this represent ? \par 
        \vspace{5pt}
        \label{m39332*solfhsst!!!underscore!!!id1617}\noindent\textbf{Solution to Exercise } \label{m39332*listfhsst!!!underscore!!!id1617}\begin{enumerate}[noitemsep, label=\textbf{Step} \textbf{\arabic*}. ] 
            \leftskip=20pt\rightskip=\leftskip\item  
        \label{m39332*id71402}\begin{itemize}[noitemsep]
            \leftskip=20pt\rightskip=\leftskip\label{m39332*uid46}\item opening balance, \begin{math}P=\mathrm{R}12\phantom{\rule{3.33333pt}{0ex}}500\end{math}\label{m39332*uid47}\item period of time, \begin{math}n=7\phantom{\rule{3.33333pt}{0ex}}\mathrm{years}\end{math}\label{m39332*uid48}\item closing balance, \begin{math}A=\mathrm{R}2\phantom{\rule{3.33333pt}{0ex}}300\end{math}\end{itemize}
        \label{m39332*id71507}We are required to find the interest rate(\begin{math}i\end{math}).\par 
        \item  
        \label{m39332*id71526}We know from (3.7) that:\par 
        \label{m39332*id71534}\nopagebreak\noindent{}\settowidth{\mymathboxwidth}{\begin{equation}
    \mathrm{A}=\mathrm{P}\left(1+\mathrm{i}\ensuremath{\cdot}\mathrm{n}\right)\tag{3.14}
      \end{equation}
    }
    \typeout{Columnwidth = \the\columnwidth}\typeout{math as usual width = \the\mymathboxwidth}
    \ifthenelse{\lengthtest{\mymathboxwidth < \columnwidth}}{% if the math fits, do it again, for real
    \begin{equation}
    \mathrm{A}=\mathrm{P}\left(1+\mathrm{i}\ensuremath{\cdot}\mathrm{n}\right)\tag{3.14}
      \end{equation}
    }{% else, if it doesn't fit
    \setlength{\mymathboxwidth}{\columnwidth}
      \addtolength{\mymathboxwidth}{-48pt}
    \par\vspace{12pt}\noindent\begin{minipage}{\columnwidth}
    \parbox[t]{\mymathboxwidth}{\large\begin{math}
    \mathrm{A}=\mathrm{P}\left(1+\mathrm{i}\ensuremath{\cdot}\mathrm{n}\right)\end{math}}\hfill
    \parbox[t]{48pt}{\raggedleft 
    (3.14)}
    \end{minipage}\vspace{12pt}\par
    }% end of conditional for this bit of math
    \typeout{math as usual width = \the\mymathboxwidth}
        \label{m39332*id71586}Therefore, for \textbf{depreciation} the formula will change to:\par 
        \label{m39332*id71598}\nopagebreak\noindent{}\settowidth{\mymathboxwidth}{\begin{equation}
    \mathrm{A}=\mathrm{P}\left(1-\mathrm{i}\ensuremath{\cdot}\mathrm{n}\right)\tag{3.15}
      \end{equation}
    }
    \typeout{Columnwidth = \the\columnwidth}\typeout{math as usual width = \the\mymathboxwidth}
    \ifthenelse{\lengthtest{\mymathboxwidth < \columnwidth}}{% if the math fits, do it again, for real
    \begin{equation}
    \mathrm{A}=\mathrm{P}\left(1-\mathrm{i}\ensuremath{\cdot}\mathrm{n}\right)\tag{3.15}
      \end{equation}
    }{% else, if it doesn't fit
    \setlength{\mymathboxwidth}{\columnwidth}
      \addtolength{\mymathboxwidth}{-48pt}
    \par\vspace{12pt}\noindent\begin{minipage}{\columnwidth}
    \parbox[t]{\mymathboxwidth}{\large\begin{math}
    \mathrm{A}=\mathrm{P}\left(1-\mathrm{i}\ensuremath{\cdot}\mathrm{n}\right)\end{math}}\hfill
    \parbox[t]{48pt}{\raggedleft 
    (3.15)}
    \end{minipage}\vspace{12pt}\par
    }% end of conditional for this bit of math
    \typeout{math as usual width = \the\mymathboxwidth}
        \item  
        \label{m39332*id71654}\nopagebreak\noindent{}
          \settowidth{\mymathboxwidth}{\begin{equation}
    \begin{array}{ccc}\hfill \mathrm{A}& =& P\left(1-i\ensuremath{\cdot}n\right)\hfill \\ \hfill R2\phantom{\rule{3.33333pt}{0ex}}300& =& R12\phantom{\rule{3.33333pt}{0ex}}500\left(1-7\ensuremath{\times}i\right)\hfill \\ \hfill i& =& 0,11657...\hfill \end{array}\tag{3.16}
      \end{equation}
    }
    \typeout{Columnwidth = \the\columnwidth}\typeout{math as usual width = \the\mymathboxwidth}
    \ifthenelse{\lengthtest{\mymathboxwidth < \columnwidth}}{% if the math fits, do it again, for real
    \begin{equation}
    \begin{array}{ccc}\hfill \mathrm{A}& =& P\left(1-i\ensuremath{\cdot}n\right)\hfill \\ \hfill R2\phantom{\rule{3.33333pt}{0ex}}300& =& R12\phantom{\rule{3.33333pt}{0ex}}500\left(1-7\ensuremath{\times}i\right)\hfill \\ \hfill i& =& 0,11657...\hfill \end{array}\tag{3.16}
      \end{equation}
    }{% else, if it doesn't fit
    \setlength{\mymathboxwidth}{\columnwidth}
      \addtolength{\mymathboxwidth}{-48pt}
    \par\vspace{12pt}\noindent\begin{minipage}{\columnwidth}
    \parbox[t]{\mymathboxwidth}{\large\begin{math}
    \mathrm{A}=P\left(1-i\ensuremath{\cdot}n\right)R2\phantom{\rule{3.33333pt}{0ex}}300=R12\phantom{\rule{3.33333pt}{0ex}}500\left(1-7\ensuremath{\times}i\right)i=0,11657...\end{math}}\hfill
    \parbox[t]{48pt}{\raggedleft 
    (3.16)}
    \end{minipage}\vspace{12pt}\par
    }% end of conditional for this bit of math
    \typeout{math as usual width = \the\mymathboxwidth}
        \item  
        \label{m39332*id71779}Therefore the rate of depreciation is \begin{math}11,66\%\end{math}
 \par 
        \end{enumerate}
    \end{exercise}
    \end{mdframed}
    }
    \noindent
\label{m39332*secfhsst!!!underscore!!!id1753}
            \subsection{Exercises: Simple Interest }
            \nopagebreak
            \label{m39332*id71818}\begin{enumerate}[noitemsep, label=\textbf{\arabic*}. ] 
            \label{m39332*uid49}\item An amount of R3 500 is invested in a savings account which pays simple interest at a rate of 7,5\% per annum. Calculate the balance accumulated by the end of 2 years.\newline
\label{m39332*uid50}\item Calculate the simple interest for the following problems.
\label{m39332*id71847}\begin{enumerate}[noitemsep, label=\textbf{\alph*}. ] 
            \label{m39332*uid51}\item A loan of R300 at a rate of 8\% for l year.
\label{m39332*uid52}\item An investment of R225 at a rate of 12,5\% for 6 years.
\end{enumerate}
\label{m39332*uid53}\item I made a deposit of R5~000 in the bank for my 5 year old son's 21st birthday. I have given him the amount of R 18~000 on his birthday. At what rate was the money invested, if simple interest was calculated ?\newline
\label{m39332*uid54}\item Bongani buys a dining room table costing R 8~500 on Hire Purchase. He is charged simple interest at 17,5\% per annum over 3 years.
\label{m39332*id71906}\begin{enumerate}[noitemsep, label=\textbf{\alph*}. ] 
            \label{m39332*uid55}\item How much will Bongani pay in total ?
\label{m39332*uid56}\item How much interest does he pay ?
\label{m39332*uid57}\item What is his monthly installment ?
\end{enumerate}
\end{enumerate}
  \label{m39332**end}
\par \raisebox{-5 pt}{\includegraphics[width=0.5cm]{col11306.imgs/summary_www.png}} Find the answers with the shortcodes:
 \par \begin{tabular}[h]{cccccc}
 (1.) lcT  &  (2.) lcb  &  (3.) lcj  &  (4.) lcD  & \end{tabular}
         \section{ Compound interest}
    \nopagebreak
            \label{m39334} $ \hspace{-5pt}\begin{array}{cccccccccccc}   \includegraphics[width=0.75cm]{col11306.imgs/summary_fullmarks.png} &   \end{array} $ \hspace{2 pt}\raisebox{-5 pt}{} {(section shortcode: MG10028 )} \par 
%     \label{m39334*cid6}
%             \subsection{ Compound Interest}
%             \nopagebreak
%             
      \label{m39334*id71967}To explain the concept of compound interest, the following example is discussed:\par 
\label{m39334*secfhsst!!!underscore!!!id1776}\vspace{.5cm} 
      \noindent
      \hspace*{-30pt}\includegraphics[width=0.5in]{col11306.imgs/pspencil2.png}   \raisebox{25mm}{   
      \begin{mdframed}[linewidth=4, leftmargin=40, rightmargin=40]  
      \begin{exercise}
    \noindent\textbf{Exercise 3.5:  Using Simple Interest to lead to the concept Compound Interest }
      \label{m39334*probfhsst!!!underscore!!!id1777}
      \label{m39334*id71984}I deposit R1~000 into a special bank account which pays a Simple Interest of 7\%. What if I empty the bank account after a year, and then take the principal and the interest and invest it back into the same account again. Then I take it all out at the end of the second year, and then put it all back in again? And then I take it all out at the end of 3 years? \par 
      \vspace{5pt}
      \label{m39334*solfhsst!!!underscore!!!id1780}\noindent\textbf{Solution to Exercise } \label{m39334*listfhsst!!!underscore!!!id1780}\begin{enumerate}[noitemsep, label=\textbf{Step} \textbf{\arabic*}. ] 
            \leftskip=20pt\rightskip=\leftskip\item  
      \label{m39334*id72013}\begin{itemize}[noitemsep]
            \leftskip=20pt\rightskip=\leftskip\label{m39334*uid58}\item opening balance, \begin{math}P=\mathrm{R}1\phantom{\rule{3.33333pt}{0ex}}000\end{math}\label{m39334*uid59}\item interest rate, \begin{math}i=7\%\end{math}\label{m39334*uid60}\item period of time, \begin{math}1\phantom{\rule{3.33333pt}{0ex}}\mathrm{year}\end{math} at a time, for 3 years
\end{itemize}
      \label{m39334*id72108}We are required to find the closing balance at the end of three years.\par 
      \item  
      \label{m39334*id72115}We know that:\par 
      \label{m39334*id72119}\nopagebreak\noindent{}\settowidth{\mymathboxwidth}{\begin{equation}
    \mathrm{A}=\mathrm{P}\left(1+\mathrm{i}\ensuremath{\cdot}\mathrm{n}\right)\tag{3.17}
      \end{equation}
    }
    \typeout{Columnwidth = \the\columnwidth}\typeout{math as usual width = \the\mymathboxwidth}
    \ifthenelse{\lengthtest{\mymathboxwidth < \columnwidth}}{% if the math fits, do it again, for real
    \begin{equation}
    \mathrm{A}=\mathrm{P}\left(1+\mathrm{i}\ensuremath{\cdot}\mathrm{n}\right)\tag{3.17}
      \end{equation}
    }{% else, if it doesn't fit
    \setlength{\mymathboxwidth}{\columnwidth}
      \addtolength{\mymathboxwidth}{-48pt}
    \par\vspace{12pt}\noindent\begin{minipage}{\columnwidth}
    \parbox[t]{\mymathboxwidth}{\large\begin{math}
    \mathrm{A}=\mathrm{P}\left(1+\mathrm{i}\ensuremath{\cdot}\mathrm{n}\right)\end{math}}\hfill
    \parbox[t]{48pt}{\raggedleft 
    (3.17)}
    \end{minipage}\vspace{12pt}\par
    }% end of conditional for this bit of math
    \typeout{math as usual width = \the\mymathboxwidth}
      \item  
      \label{m39334*id72165}\nopagebreak\noindent{}\settowidth{\mymathboxwidth}{\begin{equation}
    \begin{array}{ccc}\hfill \mathrm{A}& =& P\left(1+i\ensuremath{\cdot}n\right)\hfill \\ & =& \mathrm{R}1\phantom{\rule{3.33333pt}{0ex}}000\left(1+1\ensuremath{\times}7\%\right)\hfill \\ & =& \mathrm{R}1\phantom{\rule{3.33333pt}{0ex}}070\hfill \end{array}\tag{3.18}
      \end{equation}
    }
    \typeout{Columnwidth = \the\columnwidth}\typeout{math as usual width = \the\mymathboxwidth}
    \ifthenelse{\lengthtest{\mymathboxwidth < \columnwidth}}{% if the math fits, do it again, for real
    \begin{equation}
    \begin{array}{ccc}\hfill \mathrm{A}& =& P\left(1+i\ensuremath{\cdot}n\right)\hfill \\ & =& \mathrm{R}1\phantom{\rule{3.33333pt}{0ex}}000\left(1+1\ensuremath{\times}7\%\right)\hfill \\ & =& \mathrm{R}1\phantom{\rule{3.33333pt}{0ex}}070\hfill \end{array}\tag{3.18}
      \end{equation}
    }{% else, if it doesn't fit
    \setlength{\mymathboxwidth}{\columnwidth}
      \addtolength{\mymathboxwidth}{-48pt}
    \par\vspace{12pt}\noindent\begin{minipage}{\columnwidth}
    \parbox[t]{\mymathboxwidth}{\large\begin{math}
    \mathrm{A}=P\left(1+i\ensuremath{\cdot}n\right)=\mathrm{R}1\phantom{\rule{3.33333pt}{0ex}}000\left(1+1\ensuremath{\times}7\%\right)=\mathrm{R}1\phantom{\rule{3.33333pt}{0ex}}070\end{math}}\hfill
    \parbox[t]{48pt}{\raggedleft 
    (3.18)}
    \end{minipage}\vspace{12pt}\par
    }% end of conditional for this bit of math
    \typeout{math as usual width = \the\mymathboxwidth}
      \item  
      \label{m39334*id72277}After the first year, we withdraw all the money and re-deposit it. The opening balance for the second year is therefore \begin{math}\mathrm{R}1\phantom{\rule{3.33333pt}{0ex}}070\end{math}, because this is the balance after the first year.\par 
      \label{m39334*id72302}\nopagebreak\noindent{}\settowidth{\mymathboxwidth}{\begin{equation}
    \begin{array}{ccc}\hfill \mathrm{A}& =& P\left(1+i\ensuremath{\cdot}n\right)\hfill \\ & =& \mathrm{R}1\phantom{\rule{3.33333pt}{0ex}}070\left(1+1\ensuremath{\times}7\%\right)\hfill \\ & =& \mathrm{R}1\phantom{\rule{3.33333pt}{0ex}}144,90\hfill \end{array}\tag{3.19}
      \end{equation}
    }
    \typeout{Columnwidth = \the\columnwidth}\typeout{math as usual width = \the\mymathboxwidth}
    \ifthenelse{\lengthtest{\mymathboxwidth < \columnwidth}}{% if the math fits, do it again, for real
    \begin{equation}
    \begin{array}{ccc}\hfill \mathrm{A}& =& P\left(1+i\ensuremath{\cdot}n\right)\hfill \\ & =& \mathrm{R}1\phantom{\rule{3.33333pt}{0ex}}070\left(1+1\ensuremath{\times}7\%\right)\hfill \\ & =& \mathrm{R}1\phantom{\rule{3.33333pt}{0ex}}144,90\hfill \end{array}\tag{3.19}
      \end{equation}
    }{% else, if it doesn't fit
    \setlength{\mymathboxwidth}{\columnwidth}
      \addtolength{\mymathboxwidth}{-48pt}
    \par\vspace{12pt}\noindent\begin{minipage}{\columnwidth}
    \parbox[t]{\mymathboxwidth}{\large\begin{math}
    \mathrm{A}=P\left(1+i\ensuremath{\cdot}n\right)=\mathrm{R}1\phantom{\rule{3.33333pt}{0ex}}070\left(1+1\ensuremath{\times}7\%\right)=\mathrm{R}1\phantom{\rule{3.33333pt}{0ex}}144,90\end{math}}\hfill
    \parbox[t]{48pt}{\raggedleft 
    (3.19)}
    \end{minipage}\vspace{12pt}\par
    }% end of conditional for this bit of math
    \typeout{math as usual width = \the\mymathboxwidth}
      \item  
      \label{m39334*id72418}After the second year, we withdraw all the money and re-deposit it. The opening balance for the third year is therefore \begin{math}\mathrm{R}1\phantom{\rule{3.33333pt}{0ex}}144,90\end{math}, because this is the balance after the first year.\par 
      \label{m39334*id72447}\nopagebreak\noindent{}\settowidth{\mymathboxwidth}{\begin{equation}
    \begin{array}{ccc}\hfill \mathrm{A}& =& P\left(1+i\ensuremath{\cdot}n\right)\hfill \\ & =& \mathrm{R}1\phantom{\rule{3.33333pt}{0ex}}144,90\left(1+1\ensuremath{\times}7\%\right)\hfill \\ & =& \mathrm{R}1\phantom{\rule{3.33333pt}{0ex}}225,04\hfill \end{array}\tag{3.20}
      \end{equation}
    }
    \typeout{Columnwidth = \the\columnwidth}\typeout{math as usual width = \the\mymathboxwidth}
    \ifthenelse{\lengthtest{\mymathboxwidth < \columnwidth}}{% if the math fits, do it again, for real
    \begin{equation}
    \begin{array}{ccc}\hfill \mathrm{A}& =& P\left(1+i\ensuremath{\cdot}n\right)\hfill \\ & =& \mathrm{R}1\phantom{\rule{3.33333pt}{0ex}}144,90\left(1+1\ensuremath{\times}7\%\right)\hfill \\ & =& \mathrm{R}1\phantom{\rule{3.33333pt}{0ex}}225,04\hfill \end{array}\tag{3.20}
      \end{equation}
    }{% else, if it doesn't fit
    \setlength{\mymathboxwidth}{\columnwidth}
      \addtolength{\mymathboxwidth}{-48pt}
    \par\vspace{12pt}\noindent\begin{minipage}{\columnwidth}
    \parbox[t]{\mymathboxwidth}{\large\begin{math}
    \mathrm{A}=P\left(1+i\ensuremath{\cdot}n\right)=\mathrm{R}1\phantom{\rule{3.33333pt}{0ex}}144,90\left(1+1\ensuremath{\times}7\%\right)=\mathrm{R}1\phantom{\rule{3.33333pt}{0ex}}225,04\end{math}}\hfill
    \parbox[t]{48pt}{\raggedleft 
    (3.20)}
    \end{minipage}\vspace{12pt}\par
    }% end of conditional for this bit of math
    \typeout{math as usual width = \the\mymathboxwidth}
      \item  
      \label{m39334*id72567}The closing balance after withdrawing all the money and re-depositing each year for 3 years of saving R1~000 at an interest rate of 7\% is R1~225,04. \par 
      \end{enumerate}
    \end{exercise}
    \end{mdframed}
    }
    \noindent
      \label{m39334*id72585}In the two worked examples using simple interest ( and Exercise~3.5 ( Using Simple Interest to lead to the concept Compound Interest )), we have basically the same problem because \begin{math}P\end{math}=R1~000, \begin{math}i\end{math}=7\% and \begin{math}n\end{math}=3 years for both problems. Except in the second situation, we end up with R1 225,04 which is more than R1 210 from the first example. What has changed?\par 
      \label{m39334*id72618}In the first example I earned R70 interest each year - the same in the first, second and third year. But in the second situation, when I took the money out and then re-invested it, I was actually earning interest in the second year on my interest (R70) from the first year. (And interest on the interest on my interest in the third year!)\par 
      \label{m39334*id72624}This more realistically reflects what happens in the real world, and is known as Compound Interest. It is this concept which underlies just about everything we do - so we will look at it more closely next.\par 
\label{m39334*fhsst!!!underscore!!!id2023}\begin{definition}
	  \begin{tabular*}{15 cm}{m{15 mm}m{}}
	\hspace*{-50pt}  \includegraphics[width=0.5in]{col11306.imgs/psflag2.png}   & \Definition{   \label{id2481335}\textbf{ Compound Interest }} { \label{m39334*meaningfhsst!!!underscore!!!id2023}
      \label{m39334*id72635}Compound interest is the interest payable on the principal and its accumulated interest. \par 
       } 
      \end{tabular*}
      \end{definition}
      \label{m39334*id72647}Compound interest is a double-edged sword, though - great if you are earning interest on cash you have invested, but more serious if you are stuck having to pay interest on money you have borrowed!\par 
      \label{m39334*id72652}In the same way that we developed a formula for Simple Interest, let us find one for Compound Interest.\par 
      \label{m39334*id72656}If our opening balance is \begin{math}P\end{math} and we have an interest rate of \begin{math}i\end{math} then, the closing balance at the end of the first year is:\par 
      \label{m39334*id72678}\nopagebreak\noindent{}\settowidth{\mymathboxwidth}{\begin{equation}
    \mathrm{Closing\; Balance\; after}\phantom{\rule{2pt}{0ex}} 1 \phantom{\rule{2pt}{0ex}}\mathrm{year}=\mathrm{P}\left(1+\mathrm{i}\right)\tag{3.21}
      \end{equation}
    }
    \typeout{Columnwidth = \the\columnwidth}\typeout{math as usual width = \the\mymathboxwidth}
    \ifthenelse{\lengthtest{\mymathboxwidth < \columnwidth}}{% if the math fits, do it again, for real
    \begin{equation}
    \mathrm{Closing\; Balance\; after}\phantom{\rule{2pt}{0ex}} 1 \phantom{\rule{2pt}{0ex}}\mathrm{year}=\mathrm{P}\left(1+\mathrm{i}\right)\tag{3.21}
      \end{equation}
    }{% else, if it doesn't fit
    \setlength{\mymathboxwidth}{\columnwidth}
      \addtolength{\mymathboxwidth}{-48pt}
    \par\vspace{12pt}\noindent\begin{minipage}{\columnwidth}
    \parbox[t]{\mymathboxwidth}{\large\begin{math}
    \mathrm{Closing\; Balance\; after}\phantom{\rule{2pt}{0ex}} 1 \phantom{\rule{2pt}{0ex}}\mathrm{year}=\mathrm{P}\left(1+\mathrm{i}\right)\end{math}}\hfill
    \parbox[t]{48pt}{\raggedleft 
    (3.21)}
    \end{minipage}\vspace{12pt}\par
    }% end of conditional for this bit of math
    \typeout{math as usual width = \the\mymathboxwidth}
      \label{m39334*id72719}This is the same as Simple Interest because it only covers a single year. Then, if we take that out and re-invest it for another year - just as you saw us doing in the worked example above - then the balance after the second year will be:\par 
      \label{m39334*id72727}\nopagebreak\noindent{}\settowidth{\mymathboxwidth}{\begin{equation}
    \begin{array}{ccc}\hfill \mathrm{Closing\; Balance\; after}\phantom{\rule{2pt}{0ex}}2\phantom{\rule{2pt}{0ex}}\mathrm{years}& =& \left[P\left(1+i\right)\right]\ensuremath{\times}\left(1+i\right)\hfill \\ & =& P{\left(1+i\right)}^{2}\hfill \end{array}\tag{3.22}
      \end{equation}
    }
    \typeout{Columnwidth = \the\columnwidth}\typeout{math as usual width = \the\mymathboxwidth}
    \ifthenelse{\lengthtest{\mymathboxwidth < \columnwidth}}{% if the math fits, do it again, for real
    \begin{equation}
    \begin{array}{ccc}\hfill \mathrm{Closing\; Balance\; after}\phantom{\rule{2pt}{0ex}}2\phantom{\rule{2pt}{0ex}}\mathrm{years}& =& \left[P\left(1+i\right)\right]\ensuremath{\times}\left(1+i\right)\hfill \\ & =& P{\left(1+i\right)}^{2}\hfill \end{array}\tag{3.22}
      \end{equation}
    }{% else, if it doesn't fit
    \setlength{\mymathboxwidth}{\columnwidth}
      \addtolength{\mymathboxwidth}{-48pt}
    \par\vspace{12pt}\noindent\begin{minipage}{\columnwidth}
    \parbox[t]{\mymathboxwidth}{\large\begin{math}
    \mathrm{Closing\; Balance\; after}\phantom{\rule{2pt}{0ex}}2\phantom{\rule{2pt}{0ex}}\mathrm{years}=\left[P\left(1+i\right)\right]\ensuremath{\times}\left(1+i\right)=P{\left(1+i\right)}^{2}\end{math}}\hfill
    \parbox[t]{48pt}{\raggedleft 
    (3.22)}
    \end{minipage}\vspace{12pt}\par
    }% end of conditional for this bit of math
    \typeout{math as usual width = \the\mymathboxwidth}
      \label{m39334*id72823}And if we take that money out, then invest it for another year, the balance becomes:\par 
      \label{m39334*id72827}\nopagebreak\noindent{}\settowidth{\mymathboxwidth}{\begin{equation}
    \begin{array}{ccc}\hfill \mathrm{Closing\; Balance\; after}\phantom{\rule{2pt}{0ex}}3\phantom{\rule{2pt}{0ex}}\mathrm{years}& =& \left[P{\left(1+i\right)}^{2}\right]\ensuremath{\times}\left(1+i\right)\hfill \\ & =& P{\left(1+i\right)}^{3}\hfill \end{array}\tag{3.23}
      \end{equation}
    }
    \typeout{Columnwidth = \the\columnwidth}\typeout{math as usual width = \the\mymathboxwidth}
    \ifthenelse{\lengthtest{\mymathboxwidth < \columnwidth}}{% if the math fits, do it again, for real
    \begin{equation}
    \begin{array}{ccc}\hfill \mathrm{Closing\; Balance\; after}\phantom{\rule{2pt}{0ex}}3\phantom{\rule{2pt}{0ex}}\mathrm{years}& =& \left[P{\left(1+i\right)}^{2}\right]\ensuremath{\times}\left(1+i\right)\hfill \\ & =& P{\left(1+i\right)}^{3}\hfill \end{array}\tag{3.23}
      \end{equation}
    }{% else, if it doesn't fit
    \setlength{\mymathboxwidth}{\columnwidth}
      \addtolength{\mymathboxwidth}{-48pt}
    \par\vspace{12pt}\noindent\begin{minipage}{\columnwidth}
    \parbox[t]{\mymathboxwidth}{\large\begin{math}
    \mathrm{Closing\; Balance\; after}\phantom{\rule{2pt}{0ex}}3\phantom{\rule{2pt}{0ex}}\mathrm{years}=\left[P{\left(1+i\right)}^{2}\right]\ensuremath{\times}\left(1+i\right)=P{\left(1+i\right)}^{3}\end{math}}\hfill
    \parbox[t]{48pt}{\raggedleft 
    (3.23)}
    \end{minipage}\vspace{12pt}\par
    }% end of conditional for this bit of math
    \typeout{math as usual width = \the\mymathboxwidth}
      \label{m39334*id72931}We can see that the power of the term \begin{math}\left(1+i\right)\end{math} is the same as the number of years. Therefore,\par 
      \label{m39334*uid61}\nopagebreak\noindent{}\settowidth{\mymathboxwidth}{\begin{equation}
    \mathrm{Closing\; Balance\; after}\phantom{\rule{2pt}{0ex}}n\phantom{\rule{2pt}{0ex}}\mathrm{years}=\mathrm{P}{\left(1+\mathrm{i}\right)}^{\mathrm{n}}\tag{3.24}
      \end{equation}
    }
    \typeout{Columnwidth = \the\columnwidth}\typeout{math as usual width = \the\mymathboxwidth}
    \ifthenelse{\lengthtest{\mymathboxwidth < \columnwidth}}{% if the math fits, do it again, for real
    \begin{equation}
    \mathrm{Closing\; Balance\; after}\phantom{\rule{2pt}{0ex}}n\phantom{\rule{2pt}{0ex}}\mathrm{years}=\mathrm{P}{\left(1+\mathrm{i}\right)}^{\mathrm{n}}\tag{3.24}
      \end{equation}
    }{% else, if it doesn't fit
    \setlength{\mymathboxwidth}{\columnwidth}
      \addtolength{\mymathboxwidth}{-48pt}
    \par\vspace{12pt}\noindent\begin{minipage}{\columnwidth}
    \parbox[t]{\mymathboxwidth}{\large\begin{math}
    \mathrm{Closing\; Balance\; after}\phantom{\rule{2pt}{0ex}}n\phantom{\rule{2pt}{0ex}}\mathrm{years}=\mathrm{P}{\left(1+\mathrm{i}\right)}^{\mathrm{n}}\end{math}}\hfill
    \parbox[t]{48pt}{\raggedleft 
    (3.24)}
    \end{minipage}\vspace{12pt}\par
    }% end of conditional for this bit of math
    \typeout{math as usual width = \the\mymathboxwidth}
%       \label{m39334*uid62}
%             \subsubsection{ Fractions add up to the Whole}
%             \nopagebreak
%             
%         
%         \label{m39334*id73019}It is easy to show that this formula works even when \begin{math}n\end{math} is a fraction of a year. For example, let us invest the money for 1 month, then for 4 months, then for 7 months.\par 
%         \label{m39334*id73033}\nopagebreak\noindent{}\settowidth{\mymathboxwidth}{\begin{equation}
%     \begin{array}{ccc}\hfill \mathrm{Closing\; Balance\; after}\phantom{\rule{2pt}{0ex}}1\phantom{\rule{2pt}{0ex}}\mathrm{month}& =& P{\left(1+i\right)}^{\frac{1}{12}}\hfill \\ \hfill \mathrm{Closing\; Balance\; after}\phantom{\rule{2pt}{0ex}}5\phantom{\rule{2pt}{0ex}}\mathrm{months}& =& \mathrm{Closing\; Balance\; after}\phantom{\rule{2pt}{0ex}}1\phantom{\rule{2pt}{0ex}}\mathrm{month\; invested\; for}\phantom{\rule{2pt}{0ex}}4\phantom{\rule{2pt}{0ex}}\mathrm{months\; more}\hfill \\ & =& \left[P{\left(1+i\right)}^{\frac{1}{12}}\right]{\left(1+i\right)}^{\frac{4}{12}}\hfill \\ & =& P{\left(1+i\right)}^{\frac{1}{12}+\frac{4}{12}}\hfill \\ & =& P{\left(1+i\right)}^{\frac{5}{12}}\hfill \\ \hfill \mathrm{Closing\; Balance\; after}\phantom{\rule{2pt}{0ex}}12\phantom{\rule{2pt}{0ex}}\mathrm{months}& =& \mathrm{Closing\; Balance\; after}\phantom{\rule{2pt}{0ex}}5\phantom{\rule{2pt}{0ex}}\mathrm{months\; invested\; for}\phantom{\rule{2pt}{0ex}}7\phantom{\rule{2pt}{0ex}}\mathrm{months\; more}\hfill \\ & =& \left[P{\left(1+i\right)}^{\frac{5}{12}}\right]{\left(1+i\right)}^{\frac{7}{12}}\hfill \\ & =& P{\left(1+i\right)}^{\frac{5}{12}+\frac{7}{12}}\hfill \\ & =& P{\left(1+i\right)}^{\frac{12}{12}}\hfill \\ & =& P{\left(1+i\right)}^{1}\hfill \end{array}\tag{3.25}
%       \end{equation}
%     }
%     \typeout{Columnwidth = \the\columnwidth}\typeout{math as usual width = \the\mymathboxwidth}
%     \ifthenelse{\lengthtest{\mymathboxwidth < \columnwidth}}{% if the math fits, do it again, for real
%     \begin{equation}
%     \begin{array}{ccc}\hfill \mathrm{Closing\; Balance\; after}\phantom{\rule{2pt}{0ex}}1\phantom{\rule{2pt}{0ex}}\mathrm{month}& =& P{\left(1+i\right)}^{\frac{1}{12}}\hfill \\ \hfill \mathrm{Closing\; Balance\; after}\phantom{\rule{2pt}{0ex}}5\phantom{\rule{2pt}{0ex}}\mathrm{months}& =& \mathrm{Closing\; Balance\; after}\phantom{\rule{2pt}{0ex}}1\phantom{\rule{2pt}{0ex}}\mathrm{month\; invested\; for}\phantom{\rule{2pt}{0ex}}4\phantom{\rule{2pt}{0ex}}\mathrm{months\; more}\hfill \\ & =& \left[P{\left(1+i\right)}^{\frac{1}{12}}\right]{\left(1+i\right)}^{\frac{4}{12}}\hfill \\ & =& P{\left(1+i\right)}^{\frac{1}{12}+\frac{4}{12}}\hfill \\ & =& P{\left(1+i\right)}^{\frac{5}{12}}\hfill \\ \hfill \mathrm{Closing\; Balance\; after}\phantom{\rule{2pt}{0ex}}12\phantom{\rule{2pt}{0ex}}\mathrm{months}& =& \mathrm{Closing\; Balance\; after}\phantom{\rule{2pt}{0ex}}5\phantom{\rule{2pt}{0ex}}\mathrm{months\; invested\; for}\phantom{\rule{2pt}{0ex}}7\phantom{\rule{2pt}{0ex}}\mathrm{months\; more}\hfill \\ & =& \left[P{\left(1+i\right)}^{\frac{5}{12}}\right]{\left(1+i\right)}^{\frac{7}{12}}\hfill \\ & =& P{\left(1+i\right)}^{\frac{5}{12}+\frac{7}{12}}\hfill \\ & =& P{\left(1+i\right)}^{\frac{12}{12}}\hfill \\ & =& P{\left(1+i\right)}^{1}\hfill \end{array}\tag{3.25}
%       \end{equation}
%     }{% else, if it doesn't fit
%     \setlength{\mymathboxwidth}{\columnwidth}
%       \addtolength{\mymathboxwidth}{-48pt}
%     \par\vspace{12pt}\noindent\begin{minipage}{\columnwidth}
%     \parbox[t]{\mymathboxwidth}{\large\begin{math}
%     \mathrm{Closing\; Balance\; after}\phantom{\rule{2pt}{0ex}}1\phantom{\rule{2pt}{0ex}}\mathrm{month}=P{\left(1+i\right)}^{\frac{1}{12}}\mathrm{Closing\; Balance\; after}\phantom{\rule{2pt}{0ex}}5\phantom{\rule{2pt}{0ex}}\mathrm{months}=\mathrm{Closing\; Balance\; after}\phantom{\rule{2pt}{0ex}}1\phantom{\rule{2pt}{0ex}}\mathrm{month\; invested\; for}\phantom{\rule{2pt}{0ex}}4\phantom{\rule{2pt}{0ex}}\mathrm{months\; more}=\left[P{\left(1+i\right)}^{\frac{1}{12}}\right]{\left(1+i\right)}^{\frac{4}{12}}=P{\left(1+i\right)}^{\frac{1}{12}+\frac{4}{12}}=P{\left(1+i\right)}^{\frac{5}{12}}\mathrm{Closing\; Balance\; after}\phantom{\rule{2pt}{0ex}}12\phantom{\rule{2pt}{0ex}}\mathrm{months}=\mathrm{Closing\; Balance\; after}\phantom{\rule{2pt}{0ex}}5\phantom{\rule{2pt}{0ex}}\mathrm{months\; invested\; for}\phantom{\rule{2pt}{0ex}}7\phantom{\rule{2pt}{0ex}}\mathrm{months\; more}=\left[P{\left(1+i\right)}^{\frac{5}{12}}\right]{\left(1+i\right)}^{\frac{7}{12}}=P{\left(1+i\right)}^{\frac{5}{12}+\frac{7}{12}}=P{\left(1+i\right)}^{\frac{12}{12}}=P{\left(1+i\right)}^{1}\end{math}}\hfill
%     \parbox[t]{48pt}{\raggedleft 
%     (3.25)}
%     \end{minipage}\vspace{12pt}\par
%     }% end of conditional for this bit of math
%     \typeout{math as usual width = \the\mymathboxwidth}
%     
%         
%         \label{m39334*id73464}which is the same as investing the money for a year.\par 
%         \label{m39334*id73470}Look carefully at the long equation above. It is not as complicated as it looks! All we are doing is taking the opening amount (\begin{math}P\end{math}), then adding interest for just 1 month. Then we are taking that new balance and adding interest for a further 4 months, and then finally we are taking the new balance after a total of 5 months, and adding interest for 7 more months. Take a look again, and check how easy it really is.\par 
%         \label{m39334*id73487}Does the final formula look familiar? Correct - it is the same result as you would get for simply investing \begin{math}P\end{math} for one full year. This is exactly what we would expect, because:\par 
%         \label{m39334*id73500}1 month + 4 months + 7 months = 12 months, which is a year.\par 
%         \label{m39334*id73506}Can you see that? Do not move on until you have understood this point.\par 
%       
      \label{m39334*uid63}
            \subsection{ The Power of Compound Interest}
            \nopagebreak
        \label{m39334*id73519}To see how important this ``interest on interest" is, we shall compare the difference in closing balances for money earning simple interest and money earning compound interest. Consider an amount of R10~000 that you have to invest for 10 years, and assume we can earn interest of 9\%. How much would that be worth after 10 years?\par 
        \label{m39334*id73528}The closing balance for the money earning simple interest is:\par 
        \label{m39334*id73532}\nopagebreak\noindent{}\settowidth{\mymathboxwidth}{\begin{equation}
    \begin{array}{ccc}\hfill \mathrm{A}& =& P\left(1+i\ensuremath{\cdot}n\right)\hfill \\ & =& \mathrm{R}10\phantom{\rule{3.33333pt}{0ex}}000\left(1+9\%\ensuremath{\times}10\right)\hfill \\ & =& \mathrm{R}19\phantom{\rule{3.33333pt}{0ex}}000\hfill \end{array}\tag{3.26}
      \end{equation}
    }
    \typeout{Columnwidth = \the\columnwidth}\typeout{math as usual width = \the\mymathboxwidth}
    \ifthenelse{\lengthtest{\mymathboxwidth < \columnwidth}}{% if the math fits, do it again, for real
    \begin{equation}
    \begin{array}{ccc}\hfill \mathrm{A}& =& P\left(1+i\ensuremath{\cdot}n\right)\hfill \\ & =& \mathrm{R}10\phantom{\rule{3.33333pt}{0ex}}000\left(1+9\%\ensuremath{\times}10\right)\hfill \\ & =& \mathrm{R}19\phantom{\rule{3.33333pt}{0ex}}000\hfill \end{array}\tag{3.26}
      \end{equation}
    }{% else, if it doesn't fit
    \setlength{\mymathboxwidth}{\columnwidth}
      \addtolength{\mymathboxwidth}{-48pt}
    \par\vspace{12pt}\noindent\begin{minipage}{\columnwidth}
    \parbox[t]{\mymathboxwidth}{\large\begin{math}
    \mathrm{A}=P\left(1+i\ensuremath{\cdot}n\right)=\mathrm{R}10\phantom{\rule{3.33333pt}{0ex}}000\left(1+9\%\ensuremath{\times}10\right)=\mathrm{R}19\phantom{\rule{3.33333pt}{0ex}}000\end{math}}\hfill
    \parbox[t]{48pt}{\raggedleft 
    (3.26)}
    \end{minipage}\vspace{12pt}\par
    }% end of conditional for this bit of math
    \typeout{math as usual width = \the\mymathboxwidth}
        \label{m39334*id73640}The closing balance for the money earning compound interest is:\par 
        \label{m39334*id73644}\nopagebreak\noindent{}\settowidth{\mymathboxwidth}{\begin{equation}
    \begin{array}{ccc}\hfill \mathrm{A}& =& P{\left(1+i\right)}^{n}\hfill \\ & =& \mathrm{R}10\phantom{\rule{3.33333pt}{0ex}}000{\left(1+9\%\right)}^{10}\hfill \\ & =& \mathrm{R}23\phantom{\rule{3.33333pt}{0ex}}673,64\hfill \end{array}\tag{3.27}
      \end{equation}
    }
    \typeout{Columnwidth = \the\columnwidth}\typeout{math as usual width = \the\mymathboxwidth}
    \ifthenelse{\lengthtest{\mymathboxwidth < \columnwidth}}{% if the math fits, do it again, for real
    \begin{equation}
    \begin{array}{ccc}\hfill \mathrm{A}& =& P{\left(1+i\right)}^{n}\hfill \\ & =& \mathrm{R}10\phantom{\rule{3.33333pt}{0ex}}000{\left(1+9\%\right)}^{10}\hfill \\ & =& \mathrm{R}23\phantom{\rule{3.33333pt}{0ex}}673,64\hfill \end{array}\tag{3.27}
      \end{equation}
    }{% else, if it doesn't fit
    \setlength{\mymathboxwidth}{\columnwidth}
      \addtolength{\mymathboxwidth}{-48pt}
    \par\vspace{12pt}\noindent\begin{minipage}{\columnwidth}
    \parbox[t]{\mymathboxwidth}{\large\begin{math}
    \mathrm{A}=P{\left(1+i\right)}^{n}=\mathrm{R}10\phantom{\rule{3.33333pt}{0ex}}000{\left(1+9\%\right)}^{10}=\mathrm{R}23\phantom{\rule{3.33333pt}{0ex}}673,64\end{math}}\hfill
    \parbox[t]{48pt}{\raggedleft 
    (3.27)}
    \end{minipage}\vspace{12pt}\par
    }% end of conditional for this bit of math
    \typeout{math as usual width = \the\mymathboxwidth}
        \label{m39334*id73760}So next time someone talks about the ``magic of compound interest", not only will you know what they mean - but you will be able to prove it mathematically yourself!\par 
        \label{m39334*id73766}Again, keep in mind that this is good news and bad news. When you are earning interest on money you have invested, compound interest helps that amount to increase exponentially. But if you have borrowed money, the build up of the amount you owe will grow exponentially too.\par 
% \label{m39334*secfhsst!!!underscore!!!id2645}\vspace{.5cm} 
%       
%       \noindent
%       \hspace*{-30pt}\includegraphics[width=0.5in]{col11306.imgs/pspencil2.png}   \raisebox{25mm}{   
%       \begin{mdframed}[linewidth=4, leftmargin=40, rightmargin=40]  
%       \begin{exercise}
%     \noindent\textbf{Exercise 3.6:  Taking out a Loan }
%         \label{m39334*probfhsst!!!underscore!!!id2646}
%         \label{m39334*id73785}Mr Lowe wants to take out a loan of R 350~000.
% He does not want to pay back more than R625~000 altogether on the loan. If the
% interest rate he is offered is 13\%, over what period should he take the loan. \par 
%         \vspace{5pt}
%         \label{m39334*solfhsst!!!underscore!!!id2651}\noindent\textbf{Solution to Exercise } \label{m39334*listfhsst!!!underscore!!!id2651}\begin{enumerate}[noitemsep, label=\textbf{Step} \textbf{\arabic*}. ] 
%             \leftskip=20pt\rightskip=\leftskip\item  
%         \label{m39334*id73812}\begin{itemize}[noitemsep]
%             \leftskip=20pt\rightskip=\leftskip\label{m39334*uid64}\item opening balance, \begin{math}P=\mathrm{R}350\phantom{\rule{3.33333pt}{0ex}}000\end{math}\label{m39334*uid65}\item closing balance, \begin{math}A=\mathrm{R}625\phantom{\rule{3.33333pt}{0ex}}000\end{math}\label{m39334*uid66}\item interest rate, \begin{math}i=13\%\phantom{\rule{3.33333pt}{0ex}}\mathrm{per}\mathrm{year}\end{math}\end{itemize}
%         
%         \label{m39334*id73922}
% We are required to find the time period(\begin{math}n\end{math}).\par 
%         \item  
%         \label{m39334*id73942}We know from (3.24) that:\par 
%         \label{m39334*id73949}\nopagebreak\noindent{}\settowidth{\mymathboxwidth}{\begin{equation}
%     \mathrm{A}=\mathrm{P}{\left(1+\mathrm{i}\right)}^{\mathrm{n}}\tag{3.28}
%       \end{equation}
%     }
%     \typeout{Columnwidth = \the\columnwidth}\typeout{math as usual width = \the\mymathboxwidth}
%     \ifthenelse{\lengthtest{\mymathboxwidth < \columnwidth}}{% if the math fits, do it again, for real
%     \begin{equation}
%     \mathrm{A}=\mathrm{P}{\left(1+\mathrm{i}\right)}^{\mathrm{n}}\tag{3.28}
%       \end{equation}
%     }{% else, if it doesn't fit
%     \setlength{\mymathboxwidth}{\columnwidth}
%       \addtolength{\mymathboxwidth}{-48pt}
%     \par\vspace{12pt}\noindent\begin{minipage}{\columnwidth}
%     \parbox[t]{\mymathboxwidth}{\large\begin{math}
%     \mathrm{A}=\mathrm{P}{\left(1+\mathrm{i}\right)}^{\mathrm{n}}\end{math}}\hfill
%     \parbox[t]{48pt}{\raggedleft 
%     (3.28)}
%     \end{minipage}\vspace{12pt}\par
%     }% end of conditional for this bit of math
%     \typeout{math as usual width = \the\mymathboxwidth}
%     
%         
%         \label{m39334*id74003}We need to find \begin{math}n\end{math}.\par 
%         \label{m39334*id74019}Therefore we convert the formula to:\par 
%         \label{m39334*id74024}\nopagebreak\noindent{}
%           \settowidth{\mymathboxwidth}{\begin{equation}
%     \frac{\mathrm{A}}{\mathrm{P}}={\left(1+i\right)}^{n}\tag{3.29}
%       \end{equation}
%     }
%     \typeout{Columnwidth = \the\columnwidth}\typeout{math as usual width = \the\mymathboxwidth}
%     \ifthenelse{\lengthtest{\mymathboxwidth < \columnwidth}}{% if the math fits, do it again, for real
%     \begin{equation}
%     \frac{\mathrm{A}}{\mathrm{P}}={\left(1+i\right)}^{n}\tag{3.29}
%       \end{equation}
%     }{% else, if it doesn't fit
%     \setlength{\mymathboxwidth}{\columnwidth}
%       \addtolength{\mymathboxwidth}{-48pt}
%     \par\vspace{12pt}\noindent\begin{minipage}{\columnwidth}
%     \parbox[t]{\mymathboxwidth}{\large\begin{math}
%     \frac{\mathrm{A}}{\mathrm{P}}={\left(1+i\right)}^{n}\end{math}}\hfill
%     \parbox[t]{48pt}{\raggedleft 
%     (3.29)}
%     \end{minipage}\vspace{12pt}\par
%     }% end of conditional for this bit of math
%     \typeout{math as usual width = \the\mymathboxwidth}
%     
%         
%         \label{m39334*id74064}and then find \begin{math}n\end{math} by trial and error.\par 
%         \item  
%         \label{m39334*id74083}\nopagebreak\noindent{}\settowidth{\mymathboxwidth}{\begin{equation}
%     \begin{array}{ccc}\hfill \frac{\mathrm{A}}{\mathrm{P}}& =& {\left(1+i\right)}^{n}\hfill \\ \hfill \frac{625000}{350000}& =& {\left(1+0,13\right)}^{n}\hfill \\ \hfill 1,785...& =& {\left(1,13\right)}^{n}\hfill \\ & & \\ \hfill \mathrm{Try}\phantom{\rule{2pt}{0ex}}n& =& 3:{\left(1,13\right)}^{3}=1,44...\hfill \\ \hfill \mathrm{Try}\phantom{\rule{2pt}{0ex}}n& =& 4:{\left(1,13\right)}^{4}=1,63...\hfill \\ \hfill \mathrm{Try}\phantom{\rule{2pt}{0ex}}n& =& 5:{\left(1,13\right)}^{5}=1,84...\hfill \end{array}\tag{3.30}
%       \end{equation}
%     }
%     \typeout{Columnwidth = \the\columnwidth}\typeout{math as usual width = \the\mymathboxwidth}
%     \ifthenelse{\lengthtest{\mymathboxwidth < \columnwidth}}{% if the math fits, do it again, for real
%     \begin{equation}
%     \begin{array}{ccc}\hfill \frac{\mathrm{A}}{\mathrm{P}}& =& {\left(1+i\right)}^{n}\hfill \\ \hfill \frac{625000}{350000}& =& {\left(1+0,13\right)}^{n}\hfill \\ \hfill 1,785...& =& {\left(1,13\right)}^{n}\hfill \\ & & \\ \hfill \mathrm{Try}\phantom{\rule{2pt}{0ex}}n& =& 3:{\left(1,13\right)}^{3}=1,44...\hfill \\ \hfill \mathrm{Try}\phantom{\rule{2pt}{0ex}}n& =& 4:{\left(1,13\right)}^{4}=1,63...\hfill \\ \hfill \mathrm{Try}\phantom{\rule{2pt}{0ex}}n& =& 5:{\left(1,13\right)}^{5}=1,84...\hfill \end{array}\tag{3.30}
%       \end{equation}
%     }{% else, if it doesn't fit
%     \setlength{\mymathboxwidth}{\columnwidth}
%       \addtolength{\mymathboxwidth}{-48pt}
%     \par\vspace{12pt}\noindent\begin{minipage}{\columnwidth}
%     \parbox[t]{\mymathboxwidth}{\large\begin{math}
%     \frac{\mathrm{A}}{\mathrm{P}}={\left(1+i\right)}^{n}\frac{625000}{350000}={\left(1+0,13\right)}^{n}1,785...={\left(1,13\right)}^{n}\mathrm{Try}\phantom{\rule{2pt}{0ex}}n=3:{\left(1,13\right)}^{3}=1,44...\mathrm{Try}\phantom{\rule{2pt}{0ex}}n=4:{\left(1,13\right)}^{4}=1,63...\mathrm{Try}\phantom{\rule{2pt}{0ex}}n=5:{\left(1,13\right)}^{5}=1,84...\end{math}}\hfill
%     \parbox[t]{48pt}{\raggedleft 
%     (3.30)}
%     \end{minipage}\vspace{12pt}\par
%     }% end of conditional for this bit of math
%     \typeout{math as usual width = \the\mymathboxwidth}
%     
%         
%         \item  
%         Mr Lowe should take the loan over four years (If he took the loan over five years, he would end up paying more than he wants to.)
%         \end{enumerate}
%          
% 
%     \end{exercise}
%     \end{mdframed}
%     }
%     \noindent
%   
      \label{m39334*uid67}
            \subsection{ Other Applications of Compound Growth}
            \nopagebreak
\label{m39334*eip-431}The following two examples show how we can take the formula for compound interest and apply it to real life problems involving compound growth or compound decrease.\par \label{m39334*secfhsst!!!underscore!!!id2915}\vspace{.5cm} 
      \noindent
      \hspace*{-30pt}\includegraphics[width=0.5in]{col11306.imgs/pspencil2.png}   \raisebox{25mm}{   
      \begin{mdframed}[linewidth=4, leftmargin=40, rightmargin=40]  
      \begin{exercise}
    \noindent\textbf{Exercise 3.7:  Population Growth }
        \label{m39334*probfhsst!!!underscore!!!id2916}
        \label{m39334*id74449}South Africa's population is increasing by 2,5\% per year. If the current population is 43 million, how many more people will there be in South Africa in two years' time ? \par 
        \vspace{5pt}
        \label{m39334*solfhsst!!!underscore!!!id2919}\noindent\textbf{Solution to Exercise } \label{m39334*listfhsst!!!underscore!!!id2919}\begin{enumerate}[noitemsep, label=\textbf{Step} \textbf{\arabic*}. ] 
            \leftskip=20pt\rightskip=\leftskip\item  
        \label{m39334*id74474}\begin{itemize}[noitemsep]
            \leftskip=20pt\rightskip=\leftskip\label{m39334*uid68}\item initial value (opening balance), \begin{math}P=43\phantom{\rule{3.33333pt}{0ex}}000\phantom{\rule{3.33333pt}{0ex}}000\end{math}\label{m39334*uid69}\item period of time, \begin{math}n=2\phantom{\rule{3.33333pt}{0ex}}\mathrm{year}\end{math}\label{m39334*uid70}\item rate of increase, \begin{math}i=2,5\%\phantom{\rule{3.33333pt}{0ex}}\mathrm{per\; year}\end{math}\end{itemize}
        \label{m39334*id74586}We are required to find the final value (closing balance \begin{math}A\end{math}).\par 
        \item  
        \label{m39334*id74605}We know from (3.24) that:\par 
        \label{m39334*id74612}\nopagebreak\noindent{}
          \settowidth{\mymathboxwidth}{\begin{equation}
    A=P{\left(1+i\right)}^{n}\tag{3.31}
      \end{equation}
    }
    \typeout{Columnwidth = \the\columnwidth}\typeout{math as usual width = \the\mymathboxwidth}
    \ifthenelse{\lengthtest{\mymathboxwidth < \columnwidth}}{% if the math fits, do it again, for real
    \begin{equation}
    A=P{\left(1+i\right)}^{n}\tag{3.31}
      \end{equation}
    }{% else, if it doesn't fit
    \setlength{\mymathboxwidth}{\columnwidth}
      \addtolength{\mymathboxwidth}{-48pt}
    \par\vspace{12pt}\noindent\begin{minipage}{\columnwidth}
    \parbox[t]{\mymathboxwidth}{\large\begin{math}
    A=P{\left(1+i\right)}^{n}\end{math}}\hfill
    \parbox[t]{48pt}{\raggedleft 
    (3.31)}
    \end{minipage}\vspace{12pt}\par
    }% end of conditional for this bit of math
    \typeout{math as usual width = \the\mymathboxwidth}
        \item  
        \label{m39334*id74650}\nopagebreak\noindent{}
          \settowidth{\mymathboxwidth}{\begin{equation}
    \begin{array}{ccc}\hfill \mathrm{A}& =& P{\left(1+i\right)}^{n}\hfill \\ & =& 43\phantom{\rule{3.33333pt}{0ex}}000\phantom{\rule{3.33333pt}{0ex}}000{\left(1+0,025\right)}^{2}\hfill \\ & =& 45\phantom{\rule{3.33333pt}{0ex}}176\phantom{\rule{3.33333pt}{0ex}}875\hfill \end{array}\tag{3.32}
      \end{equation}
    }
    \typeout{Columnwidth = \the\columnwidth}\typeout{math as usual width = \the\mymathboxwidth}
    \ifthenelse{\lengthtest{\mymathboxwidth < \columnwidth}}{% if the math fits, do it again, for real
    \begin{equation}
    \begin{array}{ccc}\hfill \mathrm{A}& =& P{\left(1+i\right)}^{n}\hfill \\ & =& 43\phantom{\rule{3.33333pt}{0ex}}000\phantom{\rule{3.33333pt}{0ex}}000{\left(1+0,025\right)}^{2}\hfill \\ & =& 45\phantom{\rule{3.33333pt}{0ex}}176\phantom{\rule{3.33333pt}{0ex}}875\hfill \end{array}\tag{3.32}
      \end{equation}
    }{% else, if it doesn't fit
    \setlength{\mymathboxwidth}{\columnwidth}
      \addtolength{\mymathboxwidth}{-48pt}
    \par\vspace{12pt}\noindent\begin{minipage}{\columnwidth}
    \parbox[t]{\mymathboxwidth}{\large\begin{math}
    \mathrm{A}=P{\left(1+i\right)}^{n}=43\phantom{\rule{3.33333pt}{0ex}}000\phantom{\rule{3.33333pt}{0ex}}000{\left(1+0,025\right)}^{2}=45\phantom{\rule{3.33333pt}{0ex}}176\phantom{\rule{3.33333pt}{0ex}}875\end{math}}\hfill
    \parbox[t]{48pt}{\raggedleft 
    (3.32)}
    \end{minipage}\vspace{12pt}\par
    }% end of conditional for this bit of math
    \typeout{math as usual width = \the\mymathboxwidth}
        \item  
        \label{m39334*id74768}There will be \begin{math}45\phantom{\rule{3.33333pt}{0ex}}176\phantom{\rule{3.33333pt}{0ex}}875-43\phantom{\rule{3.33333pt}{0ex}}000\phantom{\rule{3.33333pt}{0ex}}000=2\phantom{\rule{3.33333pt}{0ex}}176\phantom{\rule{3.33333pt}{0ex}}875\end{math} more people in 2 years' time
 \par 
        \end{enumerate}
    \end{exercise}
    \end{mdframed}
    }
    \noindent
\label{m39334*secfhsst!!!underscore!!!id3026}\vspace{.5cm} 
      \noindent
      \hspace*{-30pt}\includegraphics[width=0.5in]{col11306.imgs/pspencil2.png}   \raisebox{25mm}{   
      \begin{mdframed}[linewidth=4, leftmargin=40, rightmargin=40]  
      \begin{exercise}
    \noindent\textbf{Exercise 3.8:  Compound Decrease }
        \label{m39334*probfhsst!!!underscore!!!id3027}
        \label{m39334*id74847}A swimming pool is being treated for a build-up of algae. Initially, \begin{math}50{m}^{2}\end{math} of the pool is covered by algae. With each day of treatment, the algae reduces by 5\%. What area is covered by algae after 30 days of treatment ? \par 
        \vspace{5pt}
        \label{m39334*solfhsst!!!underscore!!!id3030}\noindent\textbf{Solution to Exercise } \label{m39334*listfhsst!!!underscore!!!id3030}\begin{enumerate}[noitemsep, label=\textbf{Step} \textbf{\arabic*}. ] 
            \leftskip=20pt\rightskip=\leftskip\item  
        \label{m39334*id74890}\begin{itemize}[noitemsep]
            \leftskip=20pt\rightskip=\leftskip\label{m39334*uid71}\item Starting amount (opening balance), \begin{math}P=50{\mathrm{m}}^{2}\end{math}\label{m39334*uid72}\item period of time, \begin{math}n=30\phantom{\rule{3.33333pt}{0ex}}\mathrm{days}\end{math}\label{m39334*uid73}\item rate of decrease, \begin{math}i=5\%\phantom{\rule{3.33333pt}{0ex}}\mathrm{per\; day}\end{math}\end{itemize}
        \label{m39334*id74996}We are required to find the final area covered by algae (closing balance \begin{math}A\end{math}).\par 
        \item  
        \label{m39334*id75016}We know from (3.24) that:\par 
        \label{m39334*id75023}\nopagebreak\noindent{}\settowidth{\mymathboxwidth}{\begin{equation}
    \mathrm{A}=\mathrm{P}{\left(1+\mathrm{i}\right)}^{\mathrm{n}}\tag{3.33}
      \end{equation}
    }
    \typeout{Columnwidth = \the\columnwidth}\typeout{math as usual width = \the\mymathboxwidth}
    \ifthenelse{\lengthtest{\mymathboxwidth < \columnwidth}}{% if the math fits, do it again, for real
    \begin{equation}
    \mathrm{A}=\mathrm{P}{\left(1+\mathrm{i}\right)}^{\mathrm{n}}\tag{3.33}
      \end{equation}
    }{% else, if it doesn't fit
    \setlength{\mymathboxwidth}{\columnwidth}
      \addtolength{\mymathboxwidth}{-48pt}
    \par\vspace{12pt}\noindent\begin{minipage}{\columnwidth}
    \parbox[t]{\mymathboxwidth}{\large\begin{math}
    \mathrm{A}=\mathrm{P}{\left(1+\mathrm{i}\right)}^{\mathrm{n}}\end{math}}\hfill
    \parbox[t]{48pt}{\raggedleft 
    (3.33)}
    \end{minipage}\vspace{12pt}\par
    }% end of conditional for this bit of math
    \typeout{math as usual width = \the\mymathboxwidth}
        \label{m39334*id75077}But this is compound \textbf{decrease} so we can use the formula:\par 
        \label{m39334*id75089}\nopagebreak\noindent{}\settowidth{\mymathboxwidth}{\begin{equation}
    \mathrm{A}=\mathrm{P}{\left(1-\mathrm{i}\right)}^{\mathrm{n}}\tag{3.34}
      \end{equation}
    }
    \typeout{Columnwidth = \the\columnwidth}\typeout{math as usual width = \the\mymathboxwidth}
    \ifthenelse{\lengthtest{\mymathboxwidth < \columnwidth}}{% if the math fits, do it again, for real
    \begin{equation}
    \mathrm{A}=\mathrm{P}{\left(1-\mathrm{i}\right)}^{\mathrm{n}}\tag{3.34}
      \end{equation}
    }{% else, if it doesn't fit
    \setlength{\mymathboxwidth}{\columnwidth}
      \addtolength{\mymathboxwidth}{-48pt}
    \par\vspace{12pt}\noindent\begin{minipage}{\columnwidth}
    \parbox[t]{\mymathboxwidth}{\large\begin{math}
    \mathrm{A}=\mathrm{P}{\left(1-\mathrm{i}\right)}^{\mathrm{n}}\end{math}}\hfill
    \parbox[t]{48pt}{\raggedleft 
    (3.34)}
    \end{minipage}\vspace{12pt}\par
    }% end of conditional for this bit of math
    \typeout{math as usual width = \the\mymathboxwidth}
        \item  
        \label{m39334*id75148}\nopagebreak\noindent{}
          \settowidth{\mymathboxwidth}{\begin{equation}
    \begin{array}{ccc}\hfill \mathrm{A}& =& P{\left(1-i\right)}^{n}\hfill \\ & =& 50{\left(1-0,05\right)}^{30}\hfill \\ & =& 10,73{m}^{2}\hfill \end{array}\tag{3.35}
      \end{equation}
    }
    \typeout{Columnwidth = \the\columnwidth}\typeout{math as usual width = \the\mymathboxwidth}
    \ifthenelse{\lengthtest{\mymathboxwidth < \columnwidth}}{% if the math fits, do it again, for real
    \begin{equation}
    \begin{array}{ccc}\hfill \mathrm{A}& =& P{\left(1-i\right)}^{n}\hfill \\ & =& 50{\left(1-0,05\right)}^{30}\hfill \\ & =& 10,73{m}^{2}\hfill \end{array}\tag{3.35}
      \end{equation}
    }{% else, if it doesn't fit
    \setlength{\mymathboxwidth}{\columnwidth}
      \addtolength{\mymathboxwidth}{-48pt}
    \par\vspace{12pt}\noindent\begin{minipage}{\columnwidth}
    \parbox[t]{\mymathboxwidth}{\large\begin{math}
    \mathrm{A}=P{\left(1-i\right)}^{n}=50{\left(1-0,05\right)}^{30}=10,73{m}^{2}\end{math}}\hfill
    \parbox[t]{48pt}{\raggedleft 
    (3.35)}
    \end{minipage}\vspace{12pt}\par
    }% end of conditional for this bit of math
    \typeout{math as usual width = \the\mymathboxwidth}
        \item  
        \label{m39334*id75256}Therefore the area still covered with algae is \begin{math}10,73{m}^{2}\end{math}
 \par 
        \end{enumerate}
    \end{exercise}
    \end{mdframed}
    }
    \noindent
\label{m39334*secfhsst!!!underscore!!!id3169}
            \subsection{Exercises: Compound Interest }
            \nopagebreak
            \label{m39334*id75300}\begin{enumerate}[noitemsep, label=\textbf{\arabic*}. ] 
            \label{m39334*uid74}\item An amount of R3 500 is invested in a savings account which pays compound interest at a rate of 7,5\% per annum. Calculate the balance accumulated by the end of 2 years.\newline
\label{m39334*uid75}\item If the average rate of inflation for the past few years was 7,3\% and your water and electricity account is R 1~425 on average, what would you expect to pay in 6 years time ?\newline
\label{m39334*uid76}\item Shrek wants to invest some money at 11\% per annum compound interest. How much money (to the nearest rand) should he invest if he wants to reach a sum of R 100~000 in five year's time ?\newline
\end{enumerate}
  \label{m39334*eip-523}The next section on exchange rates is included for completeness. However, you should know about fluctuating exchange rates and the impact that this has on imports and exports. Fluctuating exchange rates lead to things like increases in the cost of petrol. You can read more about this in Fluctuating exchange rates.\par 
  \label{m39334**end}
\par \raisebox{-5 pt}{\includegraphics[width=0.5cm]{col11306.imgs/summary_www.png}} Find the answers with the shortcodes:
 \par \begin{tabular}[h]{cccccc}
 (1.) lcW  &  (2.) lcZ  &  (3.) lcB  & \end{tabular}
         \section{ Foreign exchange rates}
    \nopagebreak
            \label{m39335} $ \hspace{-5pt}\begin{array}{cccccccccccc}   \includegraphics[width=0.75cm]{col11306.imgs/summary_fullmarks.png} &   \includegraphics[width=0.75cm]{col11306.imgs/summary_video.png} &   \end{array} $ \hspace{2 pt}\raisebox{-5 pt}{} {(section shortcode: MG10029 )} \par 
% \label{m39335*cid3}
%             \subsection{ Foreign Exchange Rates - (Not in CAPS, included for completeness)}
%             \nopagebreak
%             
%       \label{m39335*id66720}Is \$500 ("500 US dollars") per person per night a good deal on a hotel in New York City? The first question you will ask is ``How much is that worth in Rands?". A quick call to the local bank or a search on the Internet (for example on http://www.x-rates.com\footnote{http://www.x-rates.com}
%         ) for the Dollar/Rand exchange rate will give you a basis for assessing the price.\par 
%       \label{m39335*id66734}A foreign exchange rate is nothing more than the price of one currency in terms of another. For example, the exchange rate of 6,18 Rands/US Dollars means that \$1 costs R6,18. In other words, if you have \$1 you could sell it for R6,18 - or if you wanted \$1 you would have to pay R6,18 for it.\par 
%       \label{m39335*id66739}But what drives exchange rates, and what causes exchange rates to change? And how does this affect you anyway? This section looks at answering these questions.\par 
%       \label{m39335*uid1}
%             \subsubsection{ How much is R1 really worth?}
%             \nopagebreak
%             
%         
%         \label{m39335*id66752}We can quote the price of a currency in terms of any other currency, for example, we can quote the Japanese Yen in term of the Indian Rupee. The US Dollar (USD), British Pound Sterling (GBP) and the Euro (EUR) are, however, the most common used market standards. You will notice that the financial news will report the South African Rand exchange rate in terms of these three major currencies.\par 
%         
%     % \textbf{m39335*uid2}\par
%     
%     % how many colspecs?  3
%           % name: cnx:colspec
%             % colnum: 1
%             % colwidth: 10*
%             % latex-name: columna
%             % colname: 
%             % align/tgroup-align/default: //left
%             % -------------------------
%             % name: cnx:colspec
%             % colnum: 2
%             % colwidth: 10*
%             % latex-name: columnb
%             % colname: 
%             % align/tgroup-align/default: //left
%             % -------------------------
%             % name: cnx:colspec
%             % colnum: 3
%             % colwidth: 10*
%             % latex-name: columnc
%             % colname: 
%             % align/tgroup-align/default: //left
%             % -------------------------
%       
%     
%     \setlength\mytablespace{6\tabcolsep}
%     \addtolength\mytablespace{4\arrayrulewidth}
%     \setlength\mytablewidth{\linewidth}
%         
%     
%     \setlength\mytableroom{\mytablewidth}
%     \addtolength\mytableroom{-\mytablespace}
%     
%     \setlength\myfixedwidth{0pt}
%     \setlength\mystarwidth{\mytableroom}
%         \addtolength\mystarwidth{-\myfixedwidth}
%         \divide\mystarwidth 30
%         
%     
%       % ----- Begin capturing width of table in LR mode woof
%       \settowidth{\mytableboxwidth}{\begin{tabular}[t]{|l|l|l|}\hline
%     % count in rowspan-info-nodeset: 3
%     % align/colidx: left,1
%     
%     % rowcount: '0' | start: 'false' | colidx: '1'
%     
%         % Formatting a regular cell and recurring on the next sibling
%         
%                   \textbf{Currency}
%                  &
%       % align/colidx: left,2
%     
%     % rowcount: '0' | start: 'false' | colidx: '2'
%     
%         % Formatting a regular cell and recurring on the next sibling
%         
%                   \textbf{Abbreviation}
%                  &
%       % align/colidx: left,3
%     
%     % rowcount: '0' | start: 'false' | colidx: '3'
%     
%         % Formatting a regular cell and recurring on the next sibling
%         
%                   \textbf{Symbol}
%                 % make-rowspan-placeholders
%     % rowspan info: col1 '0' | 'false' | '' || col2 '0' | 'false' | '' || col3 '0' | 'false' | ''
%      \tabularnewline\cline{1-1}\cline{2-2}\cline{3-3}
%       %--------------------------------------------------------------------
%     % align/colidx: left,1
%     
%     % rowcount: '0' | start: 'false' | colidx: '1'
%     
%         % Formatting a regular cell and recurring on the next sibling
%         South African Rand &
%       % align/colidx: left,2
%     
%     % rowcount: '0' | start: 'false' | colidx: '2'
%     
%         % Formatting a regular cell and recurring on the next sibling
%         ZAR &
%       % align/colidx: left,3
%     
%     % rowcount: '0' | start: 'false' | colidx: '3'
%     
%         % Formatting a regular cell and recurring on the next sibling
%         R% make-rowspan-placeholders
%     % rowspan info: col1 '0' | 'false' | '' || col2 '0' | 'false' | '' || col3 '0' | 'false' | ''
%      \tabularnewline\cline{1-1}\cline{2-2}\cline{3-3}
%       %--------------------------------------------------------------------
%     % align/colidx: left,1
%     
%     % rowcount: '0' | start: 'false' | colidx: '1'
%     
%         % Formatting a regular cell and recurring on the next sibling
%         United States Dollar &
%       % align/colidx: left,2
%     
%     % rowcount: '0' | start: 'false' | colidx: '2'
%     
%         % Formatting a regular cell and recurring on the next sibling
%         USD &
%       % align/colidx: left,3
%     
%     % rowcount: '0' | start: 'false' | colidx: '3'
%     
%         % Formatting a regular cell and recurring on the next sibling
%         \$% make-rowspan-placeholders
%     % rowspan info: col1 '0' | 'false' | '' || col2 '0' | 'false' | '' || col3 '0' | 'false' | ''
%      \tabularnewline\cline{1-1}\cline{2-2}\cline{3-3}
%       %--------------------------------------------------------------------
%     % align/colidx: left,1
%     
%     % rowcount: '0' | start: 'false' | colidx: '1'
%     
%         % Formatting a regular cell and recurring on the next sibling
%         British Pounds Sterling &
%       % align/colidx: left,2
%     
%     % rowcount: '0' | start: 'false' | colidx: '2'
%     
%         % Formatting a regular cell and recurring on the next sibling
%         GBP &
%       % align/colidx: left,3
%     
%     % rowcount: '0' | start: 'false' | colidx: '3'
%     
%         % Formatting a regular cell and recurring on the next sibling
%         £% make-rowspan-placeholders
%     % rowspan info: col1 '0' | 'false' | '' || col2 '0' | 'false' | '' || col3 '0' | 'false' | ''
%      \tabularnewline\cline{1-1}\cline{2-2}\cline{3-3}
%       %--------------------------------------------------------------------
%     \end{tabular}} % end mytableboxwidth set%       
%       % ----- End capturing width of table in LR mode
%     
%         % ----- LR or paragraph mode: must test
%         % ----- Begin capturing height of table
%         \settoheight{\mytableboxheight}{\begin{tabular}[t]{|l|l|l|}\hline
%     % count in rowspan-info-nodeset: 3
%     % align/colidx: left,1
%     
%     % rowcount: '0' | start: 'false' | colidx: '1'
%     
%         % Formatting a regular cell and recurring on the next sibling
%         
%                   \textbf{Currency}
%                  &
%       % align/colidx: left,2
%     
%     % rowcount: '0' | start: 'false' | colidx: '2'
%     
%         % Formatting a regular cell and recurring on the next sibling
%         
%                   \textbf{Abbreviation}
%                  &
%       % align/colidx: left,3
%     
%     % rowcount: '0' | start: 'false' | colidx: '3'
%     
%         % Formatting a regular cell and recurring on the next sibling
%         
%                   \textbf{Symbol}
%                 % make-rowspan-placeholders
%     % rowspan info: col1 '0' | 'false' | '' || col2 '0' | 'false' | '' || col3 '0' | 'false' | ''
%      \tabularnewline\cline{1-1}\cline{2-2}\cline{3-3}
%       %--------------------------------------------------------------------
%     % align/colidx: left,1
%     
%     % rowcount: '0' | start: 'false' | colidx: '1'
%     
%         % Formatting a regular cell and recurring on the next sibling
%         South African Rand &
%       % align/colidx: left,2
%     
%     % rowcount: '0' | start: 'false' | colidx: '2'
%     
%         % Formatting a regular cell and recurring on the next sibling
%         ZAR &
%       % align/colidx: left,3
%     
%     % rowcount: '0' | start: 'false' | colidx: '3'
%     
%         % Formatting a regular cell and recurring on the next sibling
%         R% make-rowspan-placeholders
%     % rowspan info: col1 '0' | 'false' | '' || col2 '0' | 'false' | '' || col3 '0' | 'false' | ''
%      \tabularnewline\cline{1-1}\cline{2-2}\cline{3-3}
%       %--------------------------------------------------------------------
%     % align/colidx: left,1
%     
%     % rowcount: '0' | start: 'false' | colidx: '1'
%     
%         % Formatting a regular cell and recurring on the next sibling
%         United States Dollar &
%       % align/colidx: left,2
%     
%     % rowcount: '0' | start: 'false' | colidx: '2'
%     
%         % Formatting a regular cell and recurring on the next sibling
%         USD &
%       % align/colidx: left,3
%     
%     % rowcount: '0' | start: 'false' | colidx: '3'
%     
%         % Formatting a regular cell and recurring on the next sibling
%         \$% make-rowspan-placeholders
%     % rowspan info: col1 '0' | 'false' | '' || col2 '0' | 'false' | '' || col3 '0' | 'false' | ''
%      \tabularnewline\cline{1-1}\cline{2-2}\cline{3-3}
%       %--------------------------------------------------------------------
%     % align/colidx: left,1
%     
%     % rowcount: '0' | start: 'false' | colidx: '1'
%     
%         % Formatting a regular cell and recurring on the next sibling
%         British Pounds Sterling &
%       % align/colidx: left,2
%     
%     % rowcount: '0' | start: 'false' | colidx: '2'
%     
%         % Formatting a regular cell and recurring on the next sibling
%         GBP &
%       % align/colidx: left,3
%     
%     % rowcount: '0' | start: 'false' | colidx: '3'
%     
%         % Formatting a regular cell and recurring on the next sibling
%         £% make-rowspan-placeholders
%     % rowspan info: col1 '0' | 'false' | '' || col2 '0' | 'false' | '' || col3 '0' | 'false' | ''
%      \tabularnewline\cline{1-1}\cline{2-2}\cline{3-3}
%       %--------------------------------------------------------------------
%     \end{tabular}} % end mytableboxheight set
%         \settodepth{\mytableboxdepth}{\begin{tabular}[t]{|l|l|l|}\hline
%     % count in rowspan-info-nodeset: 3
%     % align/colidx: left,1
%     
%     % rowcount: '0' | start: 'false' | colidx: '1'
%     
%         % Formatting a regular cell and recurring on the next sibling
%         
%                   \textbf{Currency}
%                  &
%       % align/colidx: left,2
%     
%     % rowcount: '0' | start: 'false' | colidx: '2'
%     
%         % Formatting a regular cell and recurring on the next sibling
%         
%                   \textbf{Abbreviation}
%                  &
%       % align/colidx: left,3
%     
%     % rowcount: '0' | start: 'false' | colidx: '3'
%     
%         % Formatting a regular cell and recurring on the next sibling
%         
%                   \textbf{Symbol}
%                 % make-rowspan-placeholders
%     % rowspan info: col1 '0' | 'false' | '' || col2 '0' | 'false' | '' || col3 '0' | 'false' | ''
%      \tabularnewline\cline{1-1}\cline{2-2}\cline{3-3}
%       %--------------------------------------------------------------------
%     % align/colidx: left,1
%     
%     % rowcount: '0' | start: 'false' | colidx: '1'
%     
%         % Formatting a regular cell and recurring on the next sibling
%         South African Rand &
%       % align/colidx: left,2
%     
%     % rowcount: '0' | start: 'false' | colidx: '2'
%     
%         % Formatting a regular cell and recurring on the next sibling
%         ZAR &
%       % align/colidx: left,3
%     
%     % rowcount: '0' | start: 'false' | colidx: '3'
%     
%         % Formatting a regular cell and recurring on the next sibling
%         R% make-rowspan-placeholders
%     % rowspan info: col1 '0' | 'false' | '' || col2 '0' | 'false' | '' || col3 '0' | 'false' | ''
%      \tabularnewline\cline{1-1}\cline{2-2}\cline{3-3}
%       %--------------------------------------------------------------------
%     % align/colidx: left,1
%     
%     % rowcount: '0' | start: 'false' | colidx: '1'
%     
%         % Formatting a regular cell and recurring on the next sibling
%         United States Dollar &
%       % align/colidx: left,2
%     
%     % rowcount: '0' | start: 'false' | colidx: '2'
%     
%         % Formatting a regular cell and recurring on the next sibling
%         USD &
%       % align/colidx: left,3
%     
%     % rowcount: '0' | start: 'false' | colidx: '3'
%     
%         % Formatting a regular cell and recurring on the next sibling
%         \$% make-rowspan-placeholders
%     % rowspan info: col1 '0' | 'false' | '' || col2 '0' | 'false' | '' || col3 '0' | 'false' | ''
%      \tabularnewline\cline{1-1}\cline{2-2}\cline{3-3}
%       %--------------------------------------------------------------------
%     % align/colidx: left,1
%     
%     % rowcount: '0' | start: 'false' | colidx: '1'
%     
%         % Formatting a regular cell and recurring on the next sibling
%         British Pounds Sterling &
%       % align/colidx: left,2
%     
%     % rowcount: '0' | start: 'false' | colidx: '2'
%     
%         % Formatting a regular cell and recurring on the next sibling
%         GBP &
%       % align/colidx: left,3
%     
%     % rowcount: '0' | start: 'false' | colidx: '3'
%     
%         % Formatting a regular cell and recurring on the next sibling
%         £% make-rowspan-placeholders
%     % rowspan info: col1 '0' | 'false' | '' || col2 '0' | 'false' | '' || col3 '0' | 'false' | ''
%      \tabularnewline\cline{1-1}\cline{2-2}\cline{3-3}
%       %--------------------------------------------------------------------
%     \end{tabular}} % end mytableboxdepth set
%         \addtolength{\mytableboxheight}{\mytableboxdepth}
%         % ----- End capturing height of table%         
%         \ifthenelse{\mytableboxwidth<\textwidth}{% the table fits in LR mode
%           \addtolength{\mytableboxwidth}{-\mytablespace}
%           \typeout{textheight: \the\textheight}
%           \typeout{mytableboxheight: \the\mytableboxheight}
%           \typeout{textwidth: \the\textwidth}
%           \typeout{mytableboxwidth: \the\mytableboxwidth}
%           \ifthenelse{\mytableboxheight<\textheight}{%
%         
%     % \begin{table}[H]
%     % \\ '' '0'
%     
%         \begin{center}
%       
%       \label{m39335*uid2}
%       
%     \noindent
%     \begin{tabular}[t]{|l|l|l|}\hline
%     % count in rowspan-info-nodeset: 3
%     % align/colidx: left,1
%     
%     % rowcount: '0' | start: 'false' | colidx: '1'
%     
%         % Formatting a regular cell and recurring on the next sibling
%         
%                   \textbf{Currency}
%                  &
%       % align/colidx: left,2
%     
%     % rowcount: '0' | start: 'false' | colidx: '2'
%     
%         % Formatting a regular cell and recurring on the next sibling
%         
%                   \textbf{Abbreviation}
%                  &
%       % align/colidx: left,3
%     
%     % rowcount: '0' | start: 'false' | colidx: '3'
%     
%         % Formatting a regular cell and recurring on the next sibling
%         
%                   \textbf{Symbol}
%                 % make-rowspan-placeholders
%     % rowspan info: col1 '0' | 'false' | '' || col2 '0' | 'false' | '' || col3 '0' | 'false' | ''
%      \tabularnewline\cline{1-1}\cline{2-2}\cline{3-3}
%       %--------------------------------------------------------------------
%     % align/colidx: left,1
%     
%     % rowcount: '0' | start: 'false' | colidx: '1'
%     
%         % Formatting a regular cell and recurring on the next sibling
%         South African Rand &
%       % align/colidx: left,2
%     
%     % rowcount: '0' | start: 'false' | colidx: '2'
%     
%         % Formatting a regular cell and recurring on the next sibling
%         ZAR &
%       % align/colidx: left,3
%     
%     % rowcount: '0' | start: 'false' | colidx: '3'
%     
%         % Formatting a regular cell and recurring on the next sibling
%         R% make-rowspan-placeholders
%     % rowspan info: col1 '0' | 'false' | '' || col2 '0' | 'false' | '' || col3 '0' | 'false' | ''
%      \tabularnewline\cline{1-1}\cline{2-2}\cline{3-3}
%       %--------------------------------------------------------------------
%     % align/colidx: left,1
%     
%     % rowcount: '0' | start: 'false' | colidx: '1'
%     
%         % Formatting a regular cell and recurring on the next sibling
%         United States Dollar &
%       % align/colidx: left,2
%     
%     % rowcount: '0' | start: 'false' | colidx: '2'
%     
%         % Formatting a regular cell and recurring on the next sibling
%         USD &
%       % align/colidx: left,3
%     
%     % rowcount: '0' | start: 'false' | colidx: '3'
%     
%         % Formatting a regular cell and recurring on the next sibling
%         \$% make-rowspan-placeholders
%     % rowspan info: col1 '0' | 'false' | '' || col2 '0' | 'false' | '' || col3 '0' | 'false' | ''
%      \tabularnewline\cline{1-1}\cline{2-2}\cline{3-3}
%       %--------------------------------------------------------------------
%     % align/colidx: left,1
%     
%     % rowcount: '0' | start: 'false' | colidx: '1'
%     
%         % Formatting a regular cell and recurring on the next sibling
%         British Pounds Sterling &
%       % align/colidx: left,2
%     
%     % rowcount: '0' | start: 'false' | colidx: '2'
%     
%         % Formatting a regular cell and recurring on the next sibling
%         GBP &
%       % align/colidx: left,3
%     
%     % rowcount: '0' | start: 'false' | colidx: '3'
%     
%         % Formatting a regular cell and recurring on the next sibling
%         £% make-rowspan-placeholders
%     % rowspan info: col1 '0' | 'false' | '' || col2 '0' | 'false' | '' || col3 '0' | 'false' | ''
%      \tabularnewline\cline{1-1}\cline{2-2}\cline{3-3}
%       %--------------------------------------------------------------------
%     \end{tabular}
%       \end{center}
%     \begin{center}{\small\bfseries Table 3.1}: Abbreviations and symbols for some common currencies.\end{center}
%     %\end{table}
%     %     
%           }{ % else
%         
%     % \begin{table}[H]
%     % \\ '' '0'
%     
%         \begin{center}
%       
%       \label{m39335*uid2}
%       
%     \noindent
%     \tabletail{%
%         \hline
%         \multicolumn{3}{|p{\mytableboxwidth}|}{\raggedleft \small \sl continued on next page}\\
%         \hline
%       }
%       \tablelasttail{}
%       \begin{xtabular}[t]{|l|l|l|}\hline
%     % count in rowspan-info-nodeset: 3
%     % align/colidx: left,1
%     
%     % rowcount: '0' | start: 'false' | colidx: '1'
%     
%         % Formatting a regular cell and recurring on the next sibling
%         
%                   \textbf{Currency}
%                  &
%       % align/colidx: left,2
%     
%     % rowcount: '0' | start: 'false' | colidx: '2'
%     
%         % Formatting a regular cell and recurring on the next sibling
%         
%                   \textbf{Abbreviation}
%                  &
%       % align/colidx: left,3
%     
%     % rowcount: '0' | start: 'false' | colidx: '3'
%     
%         % Formatting a regular cell and recurring on the next sibling
%         
%                   \textbf{Symbol}
%                 % make-rowspan-placeholders
%     % rowspan info: col1 '0' | 'false' | '' || col2 '0' | 'false' | '' || col3 '0' | 'false' | ''
%      \tabularnewline\cline{1-1}\cline{2-2}\cline{3-3}
%       %--------------------------------------------------------------------
%     % align/colidx: left,1
%     
%     % rowcount: '0' | start: 'false' | colidx: '1'
%     
%         % Formatting a regular cell and recurring on the next sibling
%         South African Rand &
%       % align/colidx: left,2
%     
%     % rowcount: '0' | start: 'false' | colidx: '2'
%     
%         % Formatting a regular cell and recurring on the next sibling
%         ZAR &
%       % align/colidx: left,3
%     
%     % rowcount: '0' | start: 'false' | colidx: '3'
%     
%         % Formatting a regular cell and recurring on the next sibling
%         R% make-rowspan-placeholders
%     % rowspan info: col1 '0' | 'false' | '' || col2 '0' | 'false' | '' || col3 '0' | 'false' | ''
%      \tabularnewline\cline{1-1}\cline{2-2}\cline{3-3}
%       %--------------------------------------------------------------------
%     % align/colidx: left,1
%     
%     % rowcount: '0' | start: 'false' | colidx: '1'
%     
%         % Formatting a regular cell and recurring on the next sibling
%         United States Dollar &
%       % align/colidx: left,2
%     
%     % rowcount: '0' | start: 'false' | colidx: '2'
%     
%         % Formatting a regular cell and recurring on the next sibling
%         USD &
%       % align/colidx: left,3
%     
%     % rowcount: '0' | start: 'false' | colidx: '3'
%     
%         % Formatting a regular cell and recurring on the next sibling
%         \$% make-rowspan-placeholders
%     % rowspan info: col1 '0' | 'false' | '' || col2 '0' | 'false' | '' || col3 '0' | 'false' | ''
%      \tabularnewline\cline{1-1}\cline{2-2}\cline{3-3}
%       %--------------------------------------------------------------------
%     % align/colidx: left,1
%     
%     % rowcount: '0' | start: 'false' | colidx: '1'
%     
%         % Formatting a regular cell and recurring on the next sibling
%         British Pounds Sterling &
%       % align/colidx: left,2
%     
%     % rowcount: '0' | start: 'false' | colidx: '2'
%     
%         % Formatting a regular cell and recurring on the next sibling
%         GBP &
%       % align/colidx: left,3
%     
%     % rowcount: '0' | start: 'false' | colidx: '3'
%     
%         % Formatting a regular cell and recurring on the next sibling
%         £% make-rowspan-placeholders
%     % rowspan info: col1 '0' | 'false' | '' || col2 '0' | 'false' | '' || col3 '0' | 'false' | ''
%      \tabularnewline\cline{1-1}\cline{2-2}\cline{3-3}
%       %--------------------------------------------------------------------
%     \end{xtabular}
%       \end{center}
%     \begin{center}{\small\bfseries Table 3.1}: Abbreviations and symbols for some common currencies.\end{center}
%     %\end{table}
%     %     
%           } % 
%         }{% else
%         % typeset the table in paragraph mode
%         % ----- Begin capturing height of table
%         \settoheight{\mytableboxheight}{\begin{tabular*}{\mytablewidth}[t]{|p{10\mystarwidth}|p{10\mystarwidth}|p{10\mystarwidth}|}\hline
%     % count in rowspan-info-nodeset: 3
%     % align/colidx: left,1
%     
%     % rowcount: '0' | start: 'false' | colidx: '1'
%     
%         % Formatting a regular cell and recurring on the next sibling
%         
%                   \textbf{Currency}
%                  &
%       % align/colidx: left,2
%     
%     % rowcount: '0' | start: 'false' | colidx: '2'
%     
%         % Formatting a regular cell and recurring on the next sibling
%         
%                   \textbf{Abbreviation}
%                  &
%       % align/colidx: left,3
%     
%     % rowcount: '0' | start: 'false' | colidx: '3'
%     
%         % Formatting a regular cell and recurring on the next sibling
%         
%                   \textbf{Symbol}
%                 % make-rowspan-placeholders
%     % rowspan info: col1 '0' | 'false' | '' || col2 '0' | 'false' | '' || col3 '0' | 'false' | ''
%      \tabularnewline\cline{1-1}\cline{2-2}\cline{3-3}
%       %--------------------------------------------------------------------
%     % align/colidx: left,1
%     
%     % rowcount: '0' | start: 'false' | colidx: '1'
%     
%         % Formatting a regular cell and recurring on the next sibling
%         South African Rand &
%       % align/colidx: left,2
%     
%     % rowcount: '0' | start: 'false' | colidx: '2'
%     
%         % Formatting a regular cell and recurring on the next sibling
%         ZAR &
%       % align/colidx: left,3
%     
%     % rowcount: '0' | start: 'false' | colidx: '3'
%     
%         % Formatting a regular cell and recurring on the next sibling
%         R% make-rowspan-placeholders
%     % rowspan info: col1 '0' | 'false' | '' || col2 '0' | 'false' | '' || col3 '0' | 'false' | ''
%      \tabularnewline\cline{1-1}\cline{2-2}\cline{3-3}
%       %--------------------------------------------------------------------
%     % align/colidx: left,1
%     
%     % rowcount: '0' | start: 'false' | colidx: '1'
%     
%         % Formatting a regular cell and recurring on the next sibling
%         United States Dollar &
%       % align/colidx: left,2
%     
%     % rowcount: '0' | start: 'false' | colidx: '2'
%     
%         % Formatting a regular cell and recurring on the next sibling
%         USD &
%       % align/colidx: left,3
%     
%     % rowcount: '0' | start: 'false' | colidx: '3'
%     
%         % Formatting a regular cell and recurring on the next sibling
%         \$% make-rowspan-placeholders
%     % rowspan info: col1 '0' | 'false' | '' || col2 '0' | 'false' | '' || col3 '0' | 'false' | ''
%      \tabularnewline\cline{1-1}\cline{2-2}\cline{3-3}
%       %--------------------------------------------------------------------
%     % align/colidx: left,1
%     
%     % rowcount: '0' | start: 'false' | colidx: '1'
%     
%         % Formatting a regular cell and recurring on the next sibling
%         British Pounds Sterling &
%       % align/colidx: left,2
%     
%     % rowcount: '0' | start: 'false' | colidx: '2'
%     
%         % Formatting a regular cell and recurring on the next sibling
%         GBP &
%       % align/colidx: left,3
%     
%     % rowcount: '0' | start: 'false' | colidx: '3'
%     
%         % Formatting a regular cell and recurring on the next sibling
%         £% make-rowspan-placeholders
%     % rowspan info: col1 '0' | 'false' | '' || col2 '0' | 'false' | '' || col3 '0' | 'false' | ''
%      \tabularnewline\cline{1-1}\cline{2-2}\cline{3-3}
%       %--------------------------------------------------------------------
%     \end{tabular*}} % end mytableboxheight set
%         \settodepth{\mytableboxdepth}{\begin{tabular*}{\mytablewidth}[t]{|p{10\mystarwidth}|p{10\mystarwidth}|p{10\mystarwidth}|}\hline
%     % count in rowspan-info-nodeset: 3
%     % align/colidx: left,1
%     
%     % rowcount: '0' | start: 'false' | colidx: '1'
%     
%         % Formatting a regular cell and recurring on the next sibling
%         
%                   \textbf{Currency}
%                  &
%       % align/colidx: left,2
%     
%     % rowcount: '0' | start: 'false' | colidx: '2'
%     
%         % Formatting a regular cell and recurring on the next sibling
%         
%                   \textbf{Abbreviation}
%                  &
%       % align/colidx: left,3
%     
%     % rowcount: '0' | start: 'false' | colidx: '3'
%     
%         % Formatting a regular cell and recurring on the next sibling
%         
%                   \textbf{Symbol}
%                 % make-rowspan-placeholders
%     % rowspan info: col1 '0' | 'false' | '' || col2 '0' | 'false' | '' || col3 '0' | 'false' | ''
%      \tabularnewline\cline{1-1}\cline{2-2}\cline{3-3}
%       %--------------------------------------------------------------------
%     % align/colidx: left,1
%     
%     % rowcount: '0' | start: 'false' | colidx: '1'
%     
%         % Formatting a regular cell and recurring on the next sibling
%         South African Rand &
%       % align/colidx: left,2
%     
%     % rowcount: '0' | start: 'false' | colidx: '2'
%     
%         % Formatting a regular cell and recurring on the next sibling
%         ZAR &
%       % align/colidx: left,3
%     
%     % rowcount: '0' | start: 'false' | colidx: '3'
%     
%         % Formatting a regular cell and recurring on the next sibling
%         R% make-rowspan-placeholders
%     % rowspan info: col1 '0' | 'false' | '' || col2 '0' | 'false' | '' || col3 '0' | 'false' | ''
%      \tabularnewline\cline{1-1}\cline{2-2}\cline{3-3}
%       %--------------------------------------------------------------------
%     % align/colidx: left,1
%     
%     % rowcount: '0' | start: 'false' | colidx: '1'
%     
%         % Formatting a regular cell and recurring on the next sibling
%         United States Dollar &
%       % align/colidx: left,2
%     
%     % rowcount: '0' | start: 'false' | colidx: '2'
%     
%         % Formatting a regular cell and recurring on the next sibling
%         USD &
%       % align/colidx: left,3
%     
%     % rowcount: '0' | start: 'false' | colidx: '3'
%     
%         % Formatting a regular cell and recurring on the next sibling
%         \$% make-rowspan-placeholders
%     % rowspan info: col1 '0' | 'false' | '' || col2 '0' | 'false' | '' || col3 '0' | 'false' | ''
%      \tabularnewline\cline{1-1}\cline{2-2}\cline{3-3}
%       %--------------------------------------------------------------------
%     % align/colidx: left,1
%     
%     % rowcount: '0' | start: 'false' | colidx: '1'
%     
%         % Formatting a regular cell and recurring on the next sibling
%         British Pounds Sterling &
%       % align/colidx: left,2
%     
%     % rowcount: '0' | start: 'false' | colidx: '2'
%     
%         % Formatting a regular cell and recurring on the next sibling
%         GBP &
%       % align/colidx: left,3
%     
%     % rowcount: '0' | start: 'false' | colidx: '3'
%     
%         % Formatting a regular cell and recurring on the next sibling
%         £% make-rowspan-placeholders
%     % rowspan info: col1 '0' | 'false' | '' || col2 '0' | 'false' | '' || col3 '0' | 'false' | ''
%      \tabularnewline\cline{1-1}\cline{2-2}\cline{3-3}
%       %--------------------------------------------------------------------
%     \end{tabular*}} % end mytableboxdepth set
%         \addtolength{\mytableboxheight}{\mytableboxdepth}
%         % ----- End capturing height of table
%         \typeout{textheight: \the\textheight}
%         \typeout{mytableboxheight: \the\mytableboxheight}
%         \typeout{table too wide, outputting in para mode}
%         
%     % \begin{table}[H]
%     % \\ '' '0'
%     
%         \begin{center}
%       
%       \label{m39335*uid2}
%       
%     \noindent
%     \tabletail{%
%         \hline
%         \multicolumn{3}{|p{\mytableroom}|}{\raggedleft \small \sl continued on next page}\\
%         \hline
%       }
%       \tablelasttail{}
%       \begin{xtabular*}{\mytablewidth}[t]{|p{10\mystarwidth}|p{10\mystarwidth}|p{10\mystarwidth}|}\hline
%     % count in rowspan-info-nodeset: 3
%     % align/colidx: left,1
%     
%     % rowcount: '0' | start: 'false' | colidx: '1'
%     
%         % Formatting a regular cell and recurring on the next sibling
%         
%                   \textbf{Currency}
%                  &
%       % align/colidx: left,2
%     
%     % rowcount: '0' | start: 'false' | colidx: '2'
%     
%         % Formatting a regular cell and recurring on the next sibling
%         
%                   \textbf{Abbreviation}
%                  &
%       % align/colidx: left,3
%     
%     % rowcount: '0' | start: 'false' | colidx: '3'
%     
%         % Formatting a regular cell and recurring on the next sibling
%         
%                   \textbf{Symbol}
%                 % make-rowspan-placeholders
%     % rowspan info: col1 '0' | 'false' | '' || col2 '0' | 'false' | '' || col3 '0' | 'false' | ''
%      \tabularnewline\cline{1-1}\cline{2-2}\cline{3-3}
%       %--------------------------------------------------------------------
%     % align/colidx: left,1
%     
%     % rowcount: '0' | start: 'false' | colidx: '1'
%     
%         % Formatting a regular cell and recurring on the next sibling
%         South African Rand &
%       % align/colidx: left,2
%     
%     % rowcount: '0' | start: 'false' | colidx: '2'
%     
%         % Formatting a regular cell and recurring on the next sibling
%         ZAR &
%       % align/colidx: left,3
%     
%     % rowcount: '0' | start: 'false' | colidx: '3'
%     
%         % Formatting a regular cell and recurring on the next sibling
%         R% make-rowspan-placeholders
%     % rowspan info: col1 '0' | 'false' | '' || col2 '0' | 'false' | '' || col3 '0' | 'false' | ''
%      \tabularnewline\cline{1-1}\cline{2-2}\cline{3-3}
%       %--------------------------------------------------------------------
%     % align/colidx: left,1
%     
%     % rowcount: '0' | start: 'false' | colidx: '1'
%     
%         % Formatting a regular cell and recurring on the next sibling
%         United States Dollar &
%       % align/colidx: left,2
%     
%     % rowcount: '0' | start: 'false' | colidx: '2'
%     
%         % Formatting a regular cell and recurring on the next sibling
%         USD &
%       % align/colidx: left,3
%     
%     % rowcount: '0' | start: 'false' | colidx: '3'
%     
%         % Formatting a regular cell and recurring on the next sibling
%         \$% make-rowspan-placeholders
%     % rowspan info: col1 '0' | 'false' | '' || col2 '0' | 'false' | '' || col3 '0' | 'false' | ''
%      \tabularnewline\cline{1-1}\cline{2-2}\cline{3-3}
%       %--------------------------------------------------------------------
%     % align/colidx: left,1
%     
%     % rowcount: '0' | start: 'false' | colidx: '1'
%     
%         % Formatting a regular cell and recurring on the next sibling
%         British Pounds Sterling &
%       % align/colidx: left,2
%     
%     % rowcount: '0' | start: 'false' | colidx: '2'
%     
%         % Formatting a regular cell and recurring on the next sibling
%         GBP &
%       % align/colidx: left,3
%     
%     % rowcount: '0' | start: 'false' | colidx: '3'
%     
%         % Formatting a regular cell and recurring on the next sibling
%         £% make-rowspan-placeholders
%     % rowspan info: col1 '0' | 'false' | '' || col2 '0' | 'false' | '' || col3 '0' | 'false' | ''
%      \tabularnewline\cline{1-1}\cline{2-2}\cline{3-3}
%       %--------------------------------------------------------------------
%     \end{xtabular*}
%       \end{center}
%     \begin{center}{\small\bfseries Table 3.1}: Abbreviations and symbols for some common currencies.\end{center}
%     %\end{table}
%     %     
%         }% ending lr/para test clause
%       
%     \par
%   
%         \label{m39335*id66888}So the South African Rand, noted ZAR, could be quoted on a certain date as 6,07040 ZAR per USD (i.e. \$1,00 costs R6,07040), or 12,2374 ZAR per GBP. So if I wanted to spend \$1~000 on a holiday in the United States of America, this would cost me R6~070,40; and if I wanted £1~000 for a weekend in London it would cost me R12~237,40.\par 
%         \label{m39335*id66897}This seems obvious, but let us see how we calculated those numbers:
% The rate is given as ZAR per USD, or ZAR/USD such that \$1,00 buys R6,0704. Therefore, we need to multiply by 1~000 to get the number of Rands per \$1~000.\par 
%         \label{m39335*id66904}Mathematically,\par 
%         \label{m39335*id66909}\nopagebreak\noindent{}\settowidth{\mymathboxwidth}{\begin{equation}
%     \begin{array}{ccc}\hfill \mathrm{\$}1,00& =& \mathrm{R}6,0740\hfill \\ \hfill \therefore \phantom{\rule{3.33333pt}{0ex}}1\phantom{\rule{3.33333pt}{0ex}}000\phantom{\rule{3.33333pt}{0ex}}\ensuremath{\times}\phantom{\rule{3.33333pt}{0ex}}\mathrm{\$}1,00& =& 1\phantom{\rule{3.33333pt}{0ex}}000\phantom{\rule{3.33333pt}{0ex}}\ensuremath{\times}\phantom{\rule{3.33333pt}{0ex}}\mathrm{R}6,0740\hfill \\ & =& \mathrm{R}6\phantom{\rule{3.33333pt}{0ex}}074,00\hfill \end{array}\tag{3.36}
%       \end{equation}
%     }
%     \typeout{Columnwidth = \the\columnwidth}\typeout{math as usual width = \the\mymathboxwidth}
%     \ifthenelse{\lengthtest{\mymathboxwidth < \columnwidth}}{% if the math fits, do it again, for real
%     \begin{equation}
%     \begin{array}{ccc}\hfill \mathrm{\$}1,00& =& \mathrm{R}6,0740\hfill \\ \hfill \therefore \phantom{\rule{3.33333pt}{0ex}}1\phantom{\rule{3.33333pt}{0ex}}000\phantom{\rule{3.33333pt}{0ex}}\ensuremath{\times}\phantom{\rule{3.33333pt}{0ex}}\mathrm{\$}1,00& =& 1\phantom{\rule{3.33333pt}{0ex}}000\phantom{\rule{3.33333pt}{0ex}}\ensuremath{\times}\phantom{\rule{3.33333pt}{0ex}}\mathrm{R}6,0740\hfill \\ & =& \mathrm{R}6\phantom{\rule{3.33333pt}{0ex}}074,00\hfill \end{array}\tag{3.36}
%       \end{equation}
%     }{% else, if it doesn't fit
%     \setlength{\mymathboxwidth}{\columnwidth}
%       \addtolength{\mymathboxwidth}{-48pt}
%     \par\vspace{12pt}\noindent\begin{minipage}{\columnwidth}
%     \parbox[t]{\mymathboxwidth}{\large\begin{math}
%     \mathrm{\$}1,00=\mathrm{R}6,0740\therefore \phantom{\rule{3.33333pt}{0ex}}1\phantom{\rule{3.33333pt}{0ex}}000\phantom{\rule{3.33333pt}{0ex}}\ensuremath{\times}\phantom{\rule{3.33333pt}{0ex}}\mathrm{\$}1,00=1\phantom{\rule{3.33333pt}{0ex}}000\phantom{\rule{3.33333pt}{0ex}}\ensuremath{\times}\phantom{\rule{3.33333pt}{0ex}}\mathrm{R}6,0740=\mathrm{R}6\phantom{\rule{3.33333pt}{0ex}}074,00\end{math}}\hfill
%     \parbox[t]{48pt}{\raggedleft 
%     (3.36)}
%     \end{minipage}\vspace{12pt}\par
%     }% end of conditional for this bit of math
%     \typeout{math as usual width = \the\mymathboxwidth}
%     
%         
%         \label{m39335*id67034}as expected.\par 
%         \label{m39335*id67039}What if you have saved R10~000 for spending money for the same trip and you wanted to use this to buy USD? How many USD could you get for this? Our rate is in ZAR/USD but we want to know how many USD we can get for our ZAR. This is easy. We know how much \$1,00 costs in terms of Rands.\par 
%         \label{m39335*id67049}\nopagebreak\noindent{}
%           \settowidth{\mymathboxwidth}{\begin{equation}
%     \begin{array}{ccc}\hfill \mathrm{\$}1,00& =& \mathrm{R}6,0740\hfill \\ \hfill \therefore \frac{\mathrm{\$}1,00}{6,0740}& =& \frac{\mathrm{R}6,0740}{6,0740}\hfill \\ \hfill \mathrm{\$}\frac{1,00}{6,0740}& =& \mathrm{R}1,00\hfill \\ \hfill \mathrm{R}1,00& =& \mathrm{\$}\frac{1,00}{6,0740}\hfill \\ & =& \mathrm{\$}0,164636\hfill \end{array}\tag{3.37}
%       \end{equation}
%     }
%     \typeout{Columnwidth = \the\columnwidth}\typeout{math as usual width = \the\mymathboxwidth}
%     \ifthenelse{\lengthtest{\mymathboxwidth < \columnwidth}}{% if the math fits, do it again, for real
%     \begin{equation}
%     \begin{array}{ccc}\hfill \mathrm{\$}1,00& =& \mathrm{R}6,0740\hfill \\ \hfill \therefore \frac{\mathrm{\$}1,00}{6,0740}& =& \frac{\mathrm{R}6,0740}{6,0740}\hfill \\ \hfill \mathrm{\$}\frac{1,00}{6,0740}& =& \mathrm{R}1,00\hfill \\ \hfill \mathrm{R}1,00& =& \mathrm{\$}\frac{1,00}{6,0740}\hfill \\ & =& \mathrm{\$}0,164636\hfill \end{array}\tag{3.37}
%       \end{equation}
%     }{% else, if it doesn't fit
%     \setlength{\mymathboxwidth}{\columnwidth}
%       \addtolength{\mymathboxwidth}{-48pt}
%     \par\vspace{12pt}\noindent\begin{minipage}{\columnwidth}
%     \parbox[t]{\mymathboxwidth}{\large\begin{math}
%     \mathrm{\$}1,00=\mathrm{R}6,0740\therefore \frac{\mathrm{\$}1,00}{6,0740}=\frac{\mathrm{R}6,0740}{6,0740}\mathrm{\$}\frac{1,00}{6,0740}=\mathrm{R}1,00\mathrm{R}1,00=\mathrm{\$}\frac{1,00}{6,0740}=\mathrm{\$}0,164636\end{math}}\hfill
%     \parbox[t]{48pt}{\raggedleft 
%     (3.37)}
%     \end{minipage}\vspace{12pt}\par
%     }% end of conditional for this bit of math
%     \typeout{math as usual width = \the\mymathboxwidth}
%     
%         
%         \label{m39335*id67253}As we can see, the final answer is simply the reciprocal of the ZAR/USD rate. Therefore, for R10~000 will get:\par 
%         \label{m39335*id67260}\nopagebreak\noindent{}\settowidth{\mymathboxwidth}{\begin{equation}
%     \begin{array}{ccc}\hfill \mathrm{R}1,00& =& \mathrm{\$}\frac{1,00}{6,0740}\hfill \\ \hfill \therefore \phantom{\rule{3.33333pt}{0ex}}10\phantom{\rule{3.33333pt}{0ex}}000\ensuremath{\times}\mathrm{R}1,00& =& 10\phantom{\rule{3.33333pt}{0ex}}000\ensuremath{\times}\mathrm{\$}\frac{1,00}{6,0740}\hfill \\ & =& \mathrm{\$}1\phantom{\rule{3.33333pt}{0ex}}646,36\hfill \end{array}\tag{3.38}
%       \end{equation}
%     }
%     \typeout{Columnwidth = \the\columnwidth}\typeout{math as usual width = \the\mymathboxwidth}
%     \ifthenelse{\lengthtest{\mymathboxwidth < \columnwidth}}{% if the math fits, do it again, for real
%     \begin{equation}
%     \begin{array}{ccc}\hfill \mathrm{R}1,00& =& \mathrm{\$}\frac{1,00}{6,0740}\hfill \\ \hfill \therefore \phantom{\rule{3.33333pt}{0ex}}10\phantom{\rule{3.33333pt}{0ex}}000\ensuremath{\times}\mathrm{R}1,00& =& 10\phantom{\rule{3.33333pt}{0ex}}000\ensuremath{\times}\mathrm{\$}\frac{1,00}{6,0740}\hfill \\ & =& \mathrm{\$}1\phantom{\rule{3.33333pt}{0ex}}646,36\hfill \end{array}\tag{3.38}
%       \end{equation}
%     }{% else, if it doesn't fit
%     \setlength{\mymathboxwidth}{\columnwidth}
%       \addtolength{\mymathboxwidth}{-48pt}
%     \par\vspace{12pt}\noindent\begin{minipage}{\columnwidth}
%     \parbox[t]{\mymathboxwidth}{\large\begin{math}
%     \mathrm{R}1,00=\mathrm{\$}\frac{1,00}{6,0740}\therefore \phantom{\rule{3.33333pt}{0ex}}10\phantom{\rule{3.33333pt}{0ex}}000\ensuremath{\times}\mathrm{R}1,00=10\phantom{\rule{3.33333pt}{0ex}}000\ensuremath{\times}\mathrm{\$}\frac{1,00}{6,0740}=\mathrm{\$}1\phantom{\rule{3.33333pt}{0ex}}646,36\end{math}}\hfill
%     \parbox[t]{48pt}{\raggedleft 
%     (3.38)}
%     \end{minipage}\vspace{12pt}\par
%     }% end of conditional for this bit of math
%     \typeout{math as usual width = \the\mymathboxwidth}
%     
%         
%         \label{m39335*id67402}We can check the answer as follows:\par 
%         \label{m39335*id67408}\nopagebreak\noindent{}\settowidth{\mymathboxwidth}{\begin{equation}
%     \begin{array}{ccc}\hfill \mathrm{\$}1,00& =& \mathrm{R}6,0740\hfill \\ \hfill \therefore \phantom{\rule{3.33333pt}{0ex}}1\phantom{\rule{3.33333pt}{0ex}}646,36\ensuremath{\times}\mathrm{\$}1,00& =& 1\phantom{\rule{3.33333pt}{0ex}}646,36\ensuremath{\times}\mathrm{R}6,0740\hfill \\ & =& \mathrm{R}10\phantom{\rule{3.33333pt}{0ex}}000,00\hfill \end{array}\tag{3.39}
%       \end{equation}
%     }
%     \typeout{Columnwidth = \the\columnwidth}\typeout{math as usual width = \the\mymathboxwidth}
%     \ifthenelse{\lengthtest{\mymathboxwidth < \columnwidth}}{% if the math fits, do it again, for real
%     \begin{equation}
%     \begin{array}{ccc}\hfill \mathrm{\$}1,00& =& \mathrm{R}6,0740\hfill \\ \hfill \therefore \phantom{\rule{3.33333pt}{0ex}}1\phantom{\rule{3.33333pt}{0ex}}646,36\ensuremath{\times}\mathrm{\$}1,00& =& 1\phantom{\rule{3.33333pt}{0ex}}646,36\ensuremath{\times}\mathrm{R}6,0740\hfill \\ & =& \mathrm{R}10\phantom{\rule{3.33333pt}{0ex}}000,00\hfill \end{array}\tag{3.39}
%       \end{equation}
%     }{% else, if it doesn't fit
%     \setlength{\mymathboxwidth}{\columnwidth}
%       \addtolength{\mymathboxwidth}{-48pt}
%     \par\vspace{12pt}\noindent\begin{minipage}{\columnwidth}
%     \parbox[t]{\mymathboxwidth}{\large\begin{math}
%     \mathrm{\$}1,00=\mathrm{R}6,0740\therefore \phantom{\rule{3.33333pt}{0ex}}1\phantom{\rule{3.33333pt}{0ex}}646,36\ensuremath{\times}\mathrm{\$}1,00=1\phantom{\rule{3.33333pt}{0ex}}646,36\ensuremath{\times}\mathrm{R}6,0740=\mathrm{R}10\phantom{\rule{3.33333pt}{0ex}}000,00\end{math}}\hfill
%     \parbox[t]{48pt}{\raggedleft 
%     (3.39)}
%     \end{minipage}\vspace{12pt}\par
%     }% end of conditional for this bit of math
%     \typeout{math as usual width = \the\mymathboxwidth}
%     
%         
%         \label{m39335*uid3}
%             \subsubsection{ Six of one and half a dozen of the other}
%             \nopagebreak
%             
%           
%           \label{m39335*id67548}So we have two different ways of expressing the same exchange rate: Rands per Dollar (ZAR/USD) and Dollar per Rands (USD/ZAR). Both exchange rates mean the same thing and express the value of one currency in terms of another. You can easily work out one from the other - they are just the reciprocals of the other.\par 
%           \label{m39335*id67554}If the South African Rand is our domestic (or home) currency, we call the ZAR/USD rate a ``direct" rate, and we call a USD/ZAR rate an ``indirect" rate.\par 
%           \label{m39335*id67560}In general, a direct rate is an exchange rate that is expressed as units of home currency per units of foreign currency, i.e., \begin{math}\frac{\mathrm{Domestic\; Currency}}{\mathrm{Foreign\; Currency}}\end{math}.\par 
%           \label{m39335*id67583}The Rand exchange rates that we see on the news are usually expressed as direct rates, for example you might see:\par 
%           
%     % \textbf{m39335*uid4}\par
%     
%     % how many colspecs?  2
%           % name: cnx:colspec
%             % colnum: 1
%             % colwidth: 10*
%             % latex-name: columna
%             % colname: 
%             % align/tgroup-align/default: //left
%             % -------------------------
%             % name: cnx:colspec
%             % colnum: 2
%             % colwidth: 10*
%             % latex-name: columnb
%             % colname: 
%             % align/tgroup-align/default: //left
%             % -------------------------
%       
%     
%     \setlength\mytablespace{4\tabcolsep}
%     \addtolength\mytablespace{3\arrayrulewidth}
%     \setlength\mytablewidth{\linewidth}
%         
%     
%     \setlength\mytableroom{\mytablewidth}
%     \addtolength\mytableroom{-\mytablespace}
%     
%     \setlength\myfixedwidth{0pt}
%     \setlength\mystarwidth{\mytableroom}
%         \addtolength\mystarwidth{-\myfixedwidth}
%         \divide\mystarwidth 20
%         
%     
%       % ----- Begin capturing width of table in LR mode woof
%       \settowidth{\mytableboxwidth}{\begin{tabular}[t]{|l|l|}\hline
%     % count in rowspan-info-nodeset: 2
%     % align/colidx: left,1
%     
%     % rowcount: '0' | start: 'false' | colidx: '1'
%     
%         % Formatting a regular cell and recurring on the next sibling
%         
%                     \textbf{Currency Abbreviation}
%                    &
%       % align/colidx: left,2
%     
%     % rowcount: '0' | start: 'false' | colidx: '2'
%     
%         % Formatting a regular cell and recurring on the next sibling
%         
%                     \textbf{Exchange Rates}
%                   % make-rowspan-placeholders
%     % rowspan info: col1 '0' | 'false' | '' || col2 '0' | 'false' | ''
%      \tabularnewline\cline{1-1}\cline{2-2}
%       %--------------------------------------------------------------------
%     % align/colidx: left,1
%     
%     % rowcount: '0' | start: 'false' | colidx: '1'
%     
%         % Formatting a regular cell and recurring on the next sibling
%         1 USD &
%       % align/colidx: left,2
%     
%     % rowcount: '0' | start: 'false' | colidx: '2'
%     
%         % Formatting a regular cell and recurring on the next sibling
%         R6,9556% make-rowspan-placeholders
%     % rowspan info: col1 '0' | 'false' | '' || col2 '0' | 'false' | ''
%      \tabularnewline\cline{1-1}\cline{2-2}
%       %--------------------------------------------------------------------
%     % align/colidx: left,1
%     
%     % rowcount: '0' | start: 'false' | colidx: '1'
%     
%         % Formatting a regular cell and recurring on the next sibling
%         1 GBP &
%       % align/colidx: left,2
%     
%     % rowcount: '0' | start: 'false' | colidx: '2'
%     
%         % Formatting a regular cell and recurring on the next sibling
%         R13,6628% make-rowspan-placeholders
%     % rowspan info: col1 '0' | 'false' | '' || col2 '0' | 'false' | ''
%      \tabularnewline\cline{1-1}\cline{2-2}
%       %--------------------------------------------------------------------
%     % align/colidx: left,1
%     
%     % rowcount: '0' | start: 'false' | colidx: '1'
%     
%         % Formatting a regular cell and recurring on the next sibling
%         1 EUR &
%       % align/colidx: left,2
%     
%     % rowcount: '0' | start: 'false' | colidx: '2'
%     
%         % Formatting a regular cell and recurring on the next sibling
%         R9,1954% make-rowspan-placeholders
%     % rowspan info: col1 '0' | 'false' | '' || col2 '0' | 'false' | ''
%      \tabularnewline\cline{1-1}\cline{2-2}
%       %--------------------------------------------------------------------
%     \end{tabular}} % end mytableboxwidth set%       
%       % ----- End capturing width of table in LR mode
%     
%         % ----- LR or paragraph mode: must test
%         % ----- Begin capturing height of table
%         \settoheight{\mytableboxheight}{\begin{tabular}[t]{|l|l|}\hline
%     % count in rowspan-info-nodeset: 2
%     % align/colidx: left,1
%     
%     % rowcount: '0' | start: 'false' | colidx: '1'
%     
%         % Formatting a regular cell and recurring on the next sibling
%         
%                     \textbf{Currency Abbreviation}
%                    &
%       % align/colidx: left,2
%     
%     % rowcount: '0' | start: 'false' | colidx: '2'
%     
%         % Formatting a regular cell and recurring on the next sibling
%         
%                     \textbf{Exchange Rates}
%                   % make-rowspan-placeholders
%     % rowspan info: col1 '0' | 'false' | '' || col2 '0' | 'false' | ''
%      \tabularnewline\cline{1-1}\cline{2-2}
%       %--------------------------------------------------------------------
%     % align/colidx: left,1
%     
%     % rowcount: '0' | start: 'false' | colidx: '1'
%     
%         % Formatting a regular cell and recurring on the next sibling
%         1 USD &
%       % align/colidx: left,2
%     
%     % rowcount: '0' | start: 'false' | colidx: '2'
%     
%         % Formatting a regular cell and recurring on the next sibling
%         R6,9556% make-rowspan-placeholders
%     % rowspan info: col1 '0' | 'false' | '' || col2 '0' | 'false' | ''
%      \tabularnewline\cline{1-1}\cline{2-2}
%       %--------------------------------------------------------------------
%     % align/colidx: left,1
%     
%     % rowcount: '0' | start: 'false' | colidx: '1'
%     
%         % Formatting a regular cell and recurring on the next sibling
%         1 GBP &
%       % align/colidx: left,2
%     
%     % rowcount: '0' | start: 'false' | colidx: '2'
%     
%         % Formatting a regular cell and recurring on the next sibling
%         R13,6628% make-rowspan-placeholders
%     % rowspan info: col1 '0' | 'false' | '' || col2 '0' | 'false' | ''
%      \tabularnewline\cline{1-1}\cline{2-2}
%       %--------------------------------------------------------------------
%     % align/colidx: left,1
%     
%     % rowcount: '0' | start: 'false' | colidx: '1'
%     
%         % Formatting a regular cell and recurring on the next sibling
%         1 EUR &
%       % align/colidx: left,2
%     
%     % rowcount: '0' | start: 'false' | colidx: '2'
%     
%         % Formatting a regular cell and recurring on the next sibling
%         R9,1954% make-rowspan-placeholders
%     % rowspan info: col1 '0' | 'false' | '' || col2 '0' | 'false' | ''
%      \tabularnewline\cline{1-1}\cline{2-2}
%       %--------------------------------------------------------------------
%     \end{tabular}} % end mytableboxheight set
%         \settodepth{\mytableboxdepth}{\begin{tabular}[t]{|l|l|}\hline
%     % count in rowspan-info-nodeset: 2
%     % align/colidx: left,1
%     
%     % rowcount: '0' | start: 'false' | colidx: '1'
%     
%         % Formatting a regular cell and recurring on the next sibling
%         
%                     \textbf{Currency Abbreviation}
%                    &
%       % align/colidx: left,2
%     
%     % rowcount: '0' | start: 'false' | colidx: '2'
%     
%         % Formatting a regular cell and recurring on the next sibling
%         
%                     \textbf{Exchange Rates}
%                   % make-rowspan-placeholders
%     % rowspan info: col1 '0' | 'false' | '' || col2 '0' | 'false' | ''
%      \tabularnewline\cline{1-1}\cline{2-2}
%       %--------------------------------------------------------------------
%     % align/colidx: left,1
%     
%     % rowcount: '0' | start: 'false' | colidx: '1'
%     
%         % Formatting a regular cell and recurring on the next sibling
%         1 USD &
%       % align/colidx: left,2
%     
%     % rowcount: '0' | start: 'false' | colidx: '2'
%     
%         % Formatting a regular cell and recurring on the next sibling
%         R6,9556% make-rowspan-placeholders
%     % rowspan info: col1 '0' | 'false' | '' || col2 '0' | 'false' | ''
%      \tabularnewline\cline{1-1}\cline{2-2}
%       %--------------------------------------------------------------------
%     % align/colidx: left,1
%     
%     % rowcount: '0' | start: 'false' | colidx: '1'
%     
%         % Formatting a regular cell and recurring on the next sibling
%         1 GBP &
%       % align/colidx: left,2
%     
%     % rowcount: '0' | start: 'false' | colidx: '2'
%     
%         % Formatting a regular cell and recurring on the next sibling
%         R13,6628% make-rowspan-placeholders
%     % rowspan info: col1 '0' | 'false' | '' || col2 '0' | 'false' | ''
%      \tabularnewline\cline{1-1}\cline{2-2}
%       %--------------------------------------------------------------------
%     % align/colidx: left,1
%     
%     % rowcount: '0' | start: 'false' | colidx: '1'
%     
%         % Formatting a regular cell and recurring on the next sibling
%         1 EUR &
%       % align/colidx: left,2
%     
%     % rowcount: '0' | start: 'false' | colidx: '2'
%     
%         % Formatting a regular cell and recurring on the next sibling
%         R9,1954% make-rowspan-placeholders
%     % rowspan info: col1 '0' | 'false' | '' || col2 '0' | 'false' | ''
%      \tabularnewline\cline{1-1}\cline{2-2}
%       %--------------------------------------------------------------------
%     \end{tabular}} % end mytableboxdepth set
%         \addtolength{\mytableboxheight}{\mytableboxdepth}
%         % ----- End capturing height of table%         
%         \ifthenelse{\mytableboxwidth<\textwidth}{% the table fits in LR mode
%           \addtolength{\mytableboxwidth}{-\mytablespace}
%           \typeout{textheight: \the\textheight}
%           \typeout{mytableboxheight: \the\mytableboxheight}
%           \typeout{textwidth: \the\textwidth}
%           \typeout{mytableboxwidth: \the\mytableboxwidth}
%           \ifthenelse{\mytableboxheight<\textheight}{%
%         
%     % \begin{table}[H]
%     % \\ '' '0'
%     
%         \begin{center}
%       
%       \label{m39335*uid4}
%       
%     \noindent
%     \begin{tabular}[t]{|l|l|}\hline
%     % count in rowspan-info-nodeset: 2
%     % align/colidx: left,1
%     
%     % rowcount: '0' | start: 'false' | colidx: '1'
%     
%         % Formatting a regular cell and recurring on the next sibling
%         
%                     \textbf{Currency Abbreviation}
%                    &
%       % align/colidx: left,2
%     
%     % rowcount: '0' | start: 'false' | colidx: '2'
%     
%         % Formatting a regular cell and recurring on the next sibling
%         
%                     \textbf{Exchange Rates}
%                   % make-rowspan-placeholders
%     % rowspan info: col1 '0' | 'false' | '' || col2 '0' | 'false' | ''
%      \tabularnewline\cline{1-1}\cline{2-2}
%       %--------------------------------------------------------------------
%     % align/colidx: left,1
%     
%     % rowcount: '0' | start: 'false' | colidx: '1'
%     
%         % Formatting a regular cell and recurring on the next sibling
%         1 USD &
%       % align/colidx: left,2
%     
%     % rowcount: '0' | start: 'false' | colidx: '2'
%     
%         % Formatting a regular cell and recurring on the next sibling
%         R6,9556% make-rowspan-placeholders
%     % rowspan info: col1 '0' | 'false' | '' || col2 '0' | 'false' | ''
%      \tabularnewline\cline{1-1}\cline{2-2}
%       %--------------------------------------------------------------------
%     % align/colidx: left,1
%     
%     % rowcount: '0' | start: 'false' | colidx: '1'
%     
%         % Formatting a regular cell and recurring on the next sibling
%         1 GBP &
%       % align/colidx: left,2
%     
%     % rowcount: '0' | start: 'false' | colidx: '2'
%     
%         % Formatting a regular cell and recurring on the next sibling
%         R13,6628% make-rowspan-placeholders
%     % rowspan info: col1 '0' | 'false' | '' || col2 '0' | 'false' | ''
%      \tabularnewline\cline{1-1}\cline{2-2}
%       %--------------------------------------------------------------------
%     % align/colidx: left,1
%     
%     % rowcount: '0' | start: 'false' | colidx: '1'
%     
%         % Formatting a regular cell and recurring on the next sibling
%         1 EUR &
%       % align/colidx: left,2
%     
%     % rowcount: '0' | start: 'false' | colidx: '2'
%     
%         % Formatting a regular cell and recurring on the next sibling
%         R9,1954% make-rowspan-placeholders
%     % rowspan info: col1 '0' | 'false' | '' || col2 '0' | 'false' | ''
%      \tabularnewline\cline{1-1}\cline{2-2}
%       %--------------------------------------------------------------------
%     \end{tabular}
%       \end{center}
%     \begin{center}{\small\bfseries Table 3.2}: Examples of exchange rates\end{center}
%     %\end{table}
%     %     
%           }{ % else
%         
%     % \begin{table}[H]
%     % \\ '' '0'
%     
%         \begin{center}
%       
%       \label{m39335*uid4}
%       
%     \noindent
%     \tabletail{%
%         \hline
%         \multicolumn{2}{|p{\mytableboxwidth}|}{\raggedleft \small \sl continued on next page}\\
%         \hline
%       }
%       \tablelasttail{}
%       \begin{xtabular}[t]{|l|l|}\hline
%     % count in rowspan-info-nodeset: 2
%     % align/colidx: left,1
%     
%     % rowcount: '0' | start: 'false' | colidx: '1'
%     
%         % Formatting a regular cell and recurring on the next sibling
%         
%                     \textbf{Currency Abbreviation}
%                    &
%       % align/colidx: left,2
%     
%     % rowcount: '0' | start: 'false' | colidx: '2'
%     
%         % Formatting a regular cell and recurring on the next sibling
%         
%                     \textbf{Exchange Rates}
%                   % make-rowspan-placeholders
%     % rowspan info: col1 '0' | 'false' | '' || col2 '0' | 'false' | ''
%      \tabularnewline\cline{1-1}\cline{2-2}
%       %--------------------------------------------------------------------
%     % align/colidx: left,1
%     
%     % rowcount: '0' | start: 'false' | colidx: '1'
%     
%         % Formatting a regular cell and recurring on the next sibling
%         1 USD &
%       % align/colidx: left,2
%     
%     % rowcount: '0' | start: 'false' | colidx: '2'
%     
%         % Formatting a regular cell and recurring on the next sibling
%         R6,9556% make-rowspan-placeholders
%     % rowspan info: col1 '0' | 'false' | '' || col2 '0' | 'false' | ''
%      \tabularnewline\cline{1-1}\cline{2-2}
%       %--------------------------------------------------------------------
%     % align/colidx: left,1
%     
%     % rowcount: '0' | start: 'false' | colidx: '1'
%     
%         % Formatting a regular cell and recurring on the next sibling
%         1 GBP &
%       % align/colidx: left,2
%     
%     % rowcount: '0' | start: 'false' | colidx: '2'
%     
%         % Formatting a regular cell and recurring on the next sibling
%         R13,6628% make-rowspan-placeholders
%     % rowspan info: col1 '0' | 'false' | '' || col2 '0' | 'false' | ''
%      \tabularnewline\cline{1-1}\cline{2-2}
%       %--------------------------------------------------------------------
%     % align/colidx: left,1
%     
%     % rowcount: '0' | start: 'false' | colidx: '1'
%     
%         % Formatting a regular cell and recurring on the next sibling
%         1 EUR &
%       % align/colidx: left,2
%     
%     % rowcount: '0' | start: 'false' | colidx: '2'
%     
%         % Formatting a regular cell and recurring on the next sibling
%         R9,1954% make-rowspan-placeholders
%     % rowspan info: col1 '0' | 'false' | '' || col2 '0' | 'false' | ''
%      \tabularnewline\cline{1-1}\cline{2-2}
%       %--------------------------------------------------------------------
%     \end{xtabular}
%       \end{center}
%     \begin{center}{\small\bfseries Table 3.2}: Examples of exchange rates\end{center}
%     %\end{table}
%     %     
%           } % 
%         }{% else
%         % typeset the table in paragraph mode
%         % ----- Begin capturing height of table
%         \settoheight{\mytableboxheight}{\begin{tabular*}{\mytablewidth}[t]{|p{10\mystarwidth}|p{10\mystarwidth}|}\hline
%     % count in rowspan-info-nodeset: 2
%     % align/colidx: left,1
%     
%     % rowcount: '0' | start: 'false' | colidx: '1'
%     
%         % Formatting a regular cell and recurring on the next sibling
%         
%                     \textbf{Currency Abbreviation}
%                    &
%       % align/colidx: left,2
%     
%     % rowcount: '0' | start: 'false' | colidx: '2'
%     
%         % Formatting a regular cell and recurring on the next sibling
%         
%                     \textbf{Exchange Rates}
%                   % make-rowspan-placeholders
%     % rowspan info: col1 '0' | 'false' | '' || col2 '0' | 'false' | ''
%      \tabularnewline\cline{1-1}\cline{2-2}
%       %--------------------------------------------------------------------
%     % align/colidx: left,1
%     
%     % rowcount: '0' | start: 'false' | colidx: '1'
%     
%         % Formatting a regular cell and recurring on the next sibling
%         1 USD &
%       % align/colidx: left,2
%     
%     % rowcount: '0' | start: 'false' | colidx: '2'
%     
%         % Formatting a regular cell and recurring on the next sibling
%         R6,9556% make-rowspan-placeholders
%     % rowspan info: col1 '0' | 'false' | '' || col2 '0' | 'false' | ''
%      \tabularnewline\cline{1-1}\cline{2-2}
%       %--------------------------------------------------------------------
%     % align/colidx: left,1
%     
%     % rowcount: '0' | start: 'false' | colidx: '1'
%     
%         % Formatting a regular cell and recurring on the next sibling
%         1 GBP &
%       % align/colidx: left,2
%     
%     % rowcount: '0' | start: 'false' | colidx: '2'
%     
%         % Formatting a regular cell and recurring on the next sibling
%         R13,6628% make-rowspan-placeholders
%     % rowspan info: col1 '0' | 'false' | '' || col2 '0' | 'false' | ''
%      \tabularnewline\cline{1-1}\cline{2-2}
%       %--------------------------------------------------------------------
%     % align/colidx: left,1
%     
%     % rowcount: '0' | start: 'false' | colidx: '1'
%     
%         % Formatting a regular cell and recurring on the next sibling
%         1 EUR &
%       % align/colidx: left,2
%     
%     % rowcount: '0' | start: 'false' | colidx: '2'
%     
%         % Formatting a regular cell and recurring on the next sibling
%         R9,1954% make-rowspan-placeholders
%     % rowspan info: col1 '0' | 'false' | '' || col2 '0' | 'false' | ''
%      \tabularnewline\cline{1-1}\cline{2-2}
%       %--------------------------------------------------------------------
%     \end{tabular*}} % end mytableboxheight set
%         \settodepth{\mytableboxdepth}{\begin{tabular*}{\mytablewidth}[t]{|p{10\mystarwidth}|p{10\mystarwidth}|}\hline
%     % count in rowspan-info-nodeset: 2
%     % align/colidx: left,1
%     
%     % rowcount: '0' | start: 'false' | colidx: '1'
%     
%         % Formatting a regular cell and recurring on the next sibling
%         
%                     \textbf{Currency Abbreviation}
%                    &
%       % align/colidx: left,2
%     
%     % rowcount: '0' | start: 'false' | colidx: '2'
%     
%         % Formatting a regular cell and recurring on the next sibling
%         
%                     \textbf{Exchange Rates}
%                   % make-rowspan-placeholders
%     % rowspan info: col1 '0' | 'false' | '' || col2 '0' | 'false' | ''
%      \tabularnewline\cline{1-1}\cline{2-2}
%       %--------------------------------------------------------------------
%     % align/colidx: left,1
%     
%     % rowcount: '0' | start: 'false' | colidx: '1'
%     
%         % Formatting a regular cell and recurring on the next sibling
%         1 USD &
%       % align/colidx: left,2
%     
%     % rowcount: '0' | start: 'false' | colidx: '2'
%     
%         % Formatting a regular cell and recurring on the next sibling
%         R6,9556% make-rowspan-placeholders
%     % rowspan info: col1 '0' | 'false' | '' || col2 '0' | 'false' | ''
%      \tabularnewline\cline{1-1}\cline{2-2}
%       %--------------------------------------------------------------------
%     % align/colidx: left,1
%     
%     % rowcount: '0' | start: 'false' | colidx: '1'
%     
%         % Formatting a regular cell and recurring on the next sibling
%         1 GBP &
%       % align/colidx: left,2
%     
%     % rowcount: '0' | start: 'false' | colidx: '2'
%     
%         % Formatting a regular cell and recurring on the next sibling
%         R13,6628% make-rowspan-placeholders
%     % rowspan info: col1 '0' | 'false' | '' || col2 '0' | 'false' | ''
%      \tabularnewline\cline{1-1}\cline{2-2}
%       %--------------------------------------------------------------------
%     % align/colidx: left,1
%     
%     % rowcount: '0' | start: 'false' | colidx: '1'
%     
%         % Formatting a regular cell and recurring on the next sibling
%         1 EUR &
%       % align/colidx: left,2
%     
%     % rowcount: '0' | start: 'false' | colidx: '2'
%     
%         % Formatting a regular cell and recurring on the next sibling
%         R9,1954% make-rowspan-placeholders
%     % rowspan info: col1 '0' | 'false' | '' || col2 '0' | 'false' | ''
%      \tabularnewline\cline{1-1}\cline{2-2}
%       %--------------------------------------------------------------------
%     \end{tabular*}} % end mytableboxdepth set
%         \addtolength{\mytableboxheight}{\mytableboxdepth}
%         % ----- End capturing height of table
%         \typeout{textheight: \the\textheight}
%         \typeout{mytableboxheight: \the\mytableboxheight}
%         \typeout{table too wide, outputting in para mode}
%         
%     % \begin{table}[H]
%     % \\ '' '0'
%     
%         \begin{center}
%       
%       \label{m39335*uid4}
%       
%     \noindent
%     \tabletail{%
%         \hline
%         \multicolumn{2}{|p{\mytableroom}|}{\raggedleft \small \sl continued on next page}\\
%         \hline
%       }
%       \tablelasttail{}
%       \begin{xtabular*}{\mytablewidth}[t]{|p{10\mystarwidth}|p{10\mystarwidth}|}\hline
%     % count in rowspan-info-nodeset: 2
%     % align/colidx: left,1
%     
%     % rowcount: '0' | start: 'false' | colidx: '1'
%     
%         % Formatting a regular cell and recurring on the next sibling
%         
%                     \textbf{Currency Abbreviation}
%                    &
%       % align/colidx: left,2
%     
%     % rowcount: '0' | start: 'false' | colidx: '2'
%     
%         % Formatting a regular cell and recurring on the next sibling
%         
%                     \textbf{Exchange Rates}
%                   % make-rowspan-placeholders
%     % rowspan info: col1 '0' | 'false' | '' || col2 '0' | 'false' | ''
%      \tabularnewline\cline{1-1}\cline{2-2}
%       %--------------------------------------------------------------------
%     % align/colidx: left,1
%     
%     % rowcount: '0' | start: 'false' | colidx: '1'
%     
%         % Formatting a regular cell and recurring on the next sibling
%         1 USD &
%       % align/colidx: left,2
%     
%     % rowcount: '0' | start: 'false' | colidx: '2'
%     
%         % Formatting a regular cell and recurring on the next sibling
%         R6,9556% make-rowspan-placeholders
%     % rowspan info: col1 '0' | 'false' | '' || col2 '0' | 'false' | ''
%      \tabularnewline\cline{1-1}\cline{2-2}
%       %--------------------------------------------------------------------
%     % align/colidx: left,1
%     
%     % rowcount: '0' | start: 'false' | colidx: '1'
%     
%         % Formatting a regular cell and recurring on the next sibling
%         1 GBP &
%       % align/colidx: left,2
%     
%     % rowcount: '0' | start: 'false' | colidx: '2'
%     
%         % Formatting a regular cell and recurring on the next sibling
%         R13,6628% make-rowspan-placeholders
%     % rowspan info: col1 '0' | 'false' | '' || col2 '0' | 'false' | ''
%      \tabularnewline\cline{1-1}\cline{2-2}
%       %--------------------------------------------------------------------
%     % align/colidx: left,1
%     
%     % rowcount: '0' | start: 'false' | colidx: '1'
%     
%         % Formatting a regular cell and recurring on the next sibling
%         1 EUR &
%       % align/colidx: left,2
%     
%     % rowcount: '0' | start: 'false' | colidx: '2'
%     
%         % Formatting a regular cell and recurring on the next sibling
%         R9,1954% make-rowspan-placeholders
%     % rowspan info: col1 '0' | 'false' | '' || col2 '0' | 'false' | ''
%      \tabularnewline\cline{1-1}\cline{2-2}
%       %--------------------------------------------------------------------
%     \end{xtabular*}
%       \end{center}
%     \begin{center}{\small\bfseries Table 3.2}: Examples of exchange rates\end{center}
%     %\end{table}
%     %     
%         }% ending lr/para test clause
%       
%     \par
%   
%           \label{m39335*id67685}The exchange rate is just the price of each of the Foreign Currencies (USD, GBP and EUR) in terms of our domestic currency, Rands.\par 
%           \label{m39335*id67689}An indirect rate is an exchange rate expressed as units of foreign currency per units of home currency, i.e. \begin{math}\frac{\mathrm{Foreign\; Currency}}{\mathrm{Domestic\; Currency}}\end{math}.\par 
%           \label{m39335*id67712}Defining exchange rates as direct or indirect depends on which currency is defined as the domestic currency. The domestic currency for an American investor would be USD which is the South African investor's foreign currency. So direct rates, from the perspective of the American investor (USD/ZAR), would be the same as the indirect rate from the perspective of the South Africa investor.\par 
%         
%         \label{m39335*uid5}
%             \subsubsection{ Terminology}
%             \nopagebreak
%             
%           
%           \label{m39335*id67731}Since exchange rates are simply prices of currencies, movements in exchange rates means that the price or value of the currency has changed. The price of petrol changes all the time, so does the price of gold, and currency prices also move up and down all the time.\par 
%           \label{m39335*id67737}If the Rand exchange rate moved from say R6,71 per USD to R6,50 per USD, what does this mean? Well, it means that \$1 would now cost only R6,50 instead of R6,71. The Dollar is now cheaper to buy, and we say that the Dollar has depreciated (or weakened) against the Rand. Alternatively we could say that the Rand has appreciated (or strengthened) against the Dollar.\par 
%           \label{m39335*id67743}What if we were looking at indirect exchange rates, and the exchange rate moved from \$0,149 per ZAR (=\begin{math}\frac{1}{6,71}\end{math}) to \$0,1538 per ZAR (=\begin{math}\frac{1}{6,50}\end{math}).\par 
%           \label{m39335*id67782}Well now we can see that the R1,00 cost \$0,149 at the start, and then cost \$0,1538 at the end. The Rand has become more expensive (in terms of Dollars), and again we can say that the Rand has appreciated.\par 
%           \label{m39335*id67787}Regardless of which exchange rate is used, we still come to the same conclusions.\par 
%           \label{m39335*id67791}In general,\par 
%           \label{m39335*id67794}\begin{itemize}[noitemsep]
%             \label{m39335*uid6}\item for direct exchange rates, the home currency will appreciate (depreciate) if the exchange rate falls (rises)
% \label{m39335*uid7}\item For indirect exchange rates, the home currency will appreciate (depreciate) if the exchange rate rises (falls)
% \end{itemize}
%         
%           \label{m39335*id67823}As with just about everything in this chapter, do not get caught up in memorising these formulae - doing so is only going to get confusing. Think about what you have and what you want - and it should be quite clear how to get the correct answer.\par 
% \label{m39335*secfhsst!!!underscore!!!id536}
%             \subsubsection{  Discussion : Foreign Exchange Rates }
%             \nopagebreak
%             
%           \label{m39335*id67836}In groups of 5, discuss:\par 
%           \label{m39335*id67841}\begin{enumerate}[noitemsep, label=\textbf{\arabic*}. ] 
%             \label{m39335*uid8}\item Why might we need to know exchange rates?
% \label{m39335*uid9}\item What happens if one country's currency falls drastically vs another country's currency?
% \label{m39335*uid10}\item When might you use exchange rates?
% \end{enumerate}
%         
%           
% 
%         
%       
%       \label{m39335*uid11}
%             \subsubsection{ Cross Currency Exchange Rates - (not in CAPS, included for completeness)}
%             \nopagebreak
%             
%         
%         \label{m39335*id67898}We know that exchange rates are the value of one currency expressed in terms of another currency, and we can quote exchange rates against any other currency. The Rand exchange rates we see on the news are usually expressed against the major currencies, USD, GBP and EUR.\par 
%         \label{m39335*id67904}So if for example, the Rand exchange rates were given as 6,71 ZAR/USD and 12,71 ZAR/GBP, does this tell us anything about the exchange rate between USD and GBP?\par 
%         \label{m39335*id67908}Well I know that if \$1 will buy me R6,71, and if £1.00 will buy me R12,71, then surely the GBP is stronger than the USD because you will get more Rands for one unit of the currency, and we can work out the USD/GBP exchange rate as follows:\par 
%         \label{m39335*id67916}Before we plug in any numbers, how can we get a USD/GBP exchange rate from the ZAR/USD and ZAR/GBP exchange rates?\par 
%         \label{m39335*id67920}Well,\par 
%         \label{m39335*id67923}\nopagebreak\noindent{}
%           \settowidth{\mymathboxwidth}{\begin{equation}
%     \mathrm{USD}/\mathrm{GBP}=\mathrm{USD}/\mathrm{ZAR}\ensuremath{\times}\mathrm{ZAR}/\mathrm{GBP}.\tag{3.40}
%       \end{equation}
%     }
%     \typeout{Columnwidth = \the\columnwidth}\typeout{math as usual width = \the\mymathboxwidth}
%     \ifthenelse{\lengthtest{\mymathboxwidth < \columnwidth}}{% if the math fits, do it again, for real
%     \begin{equation}
%     \mathrm{USD}/\mathrm{GBP}=\mathrm{USD}/\mathrm{ZAR}\ensuremath{\times}\mathrm{ZAR}/\mathrm{GBP}.\tag{3.40}
%       \end{equation}
%     }{% else, if it doesn't fit
%     \setlength{\mymathboxwidth}{\columnwidth}
%       \addtolength{\mymathboxwidth}{-48pt}
%     \par\vspace{12pt}\noindent\begin{minipage}{\columnwidth}
%     \parbox[t]{\mymathboxwidth}{\large\begin{math}
%     \mathrm{USD}/\mathrm{GBP}=\mathrm{USD}/\mathrm{ZAR}\ensuremath{\times}\mathrm{ZAR}/\mathrm{GBP}.\end{math}}\hfill
%     \parbox[t]{48pt}{\raggedleft 
%     (3.40)}
%     \end{minipage}\vspace{12pt}\par
%     }% end of conditional for this bit of math
%     \typeout{math as usual width = \the\mymathboxwidth}
%     
%         
%         \label{m39335*id67961}Note that the ZAR in the numerator will cancel out with the ZAR in the denominator, and we are left with the USD/GBP exchange rate.\par 
%         \label{m39335*id67965}Although we do not have the USD/ZAR exchange rate, we know that this is just the reciprocal of the ZAR/USD exchange rate.\par 
%         \label{m39335*id67970}\nopagebreak\noindent{}
%           \settowidth{\mymathboxwidth}{\begin{equation}
%     \mathrm{USD}/\mathrm{ZAR}=\frac{1}{\mathrm{ZAR}/\mathrm{USD}}\tag{3.41}
%       \end{equation}
%     }
%     \typeout{Columnwidth = \the\columnwidth}\typeout{math as usual width = \the\mymathboxwidth}
%     \ifthenelse{\lengthtest{\mymathboxwidth < \columnwidth}}{% if the math fits, do it again, for real
%     \begin{equation}
%     \mathrm{USD}/\mathrm{ZAR}=\frac{1}{\mathrm{ZAR}/\mathrm{USD}}\tag{3.41}
%       \end{equation}
%     }{% else, if it doesn't fit
%     \setlength{\mymathboxwidth}{\columnwidth}
%       \addtolength{\mymathboxwidth}{-48pt}
%     \par\vspace{12pt}\noindent\begin{minipage}{\columnwidth}
%     \parbox[t]{\mymathboxwidth}{\large\begin{math}
%     \mathrm{USD}/\mathrm{ZAR}=\frac{1}{\mathrm{ZAR}/\mathrm{USD}}\end{math}}\hfill
%     \parbox[t]{48pt}{\raggedleft 
%     (3.41)}
%     \end{minipage}\vspace{12pt}\par
%     }% end of conditional for this bit of math
%     \typeout{math as usual width = \the\mymathboxwidth}
%     
%         
%         \label{m39335*id68000}Now plugging in the numbers, we get:\par 
%         \label{m39335*id68003}\nopagebreak\noindent{}
%           \settowidth{\mymathboxwidth}{\begin{equation}
%     \begin{array}{ccc}\hfill \mathrm{USD}/\mathrm{GBP}& =& \mathrm{USD}/\mathrm{ZAR}\ensuremath{\times}\mathrm{ZAR}/\mathrm{GBP}\hfill \\ & =& \frac{1}{\mathrm{ZAR}/\mathrm{USD}}\ensuremath{\times}\mathrm{ZAR}/\mathrm{GBP}\hfill \\ & =& \frac{1}{6,71}\ensuremath{\times}12,71\hfill \\ & =& 1,894\hfill \end{array}\tag{3.42}
%       \end{equation}
%     }
%     \typeout{Columnwidth = \the\columnwidth}\typeout{math as usual width = \the\mymathboxwidth}
%     \ifthenelse{\lengthtest{\mymathboxwidth < \columnwidth}}{% if the math fits, do it again, for real
%     \begin{equation}
%     \begin{array}{ccc}\hfill \mathrm{USD}/\mathrm{GBP}& =& \mathrm{USD}/\mathrm{ZAR}\ensuremath{\times}\mathrm{ZAR}/\mathrm{GBP}\hfill \\ & =& \frac{1}{\mathrm{ZAR}/\mathrm{USD}}\ensuremath{\times}\mathrm{ZAR}/\mathrm{GBP}\hfill \\ & =& \frac{1}{6,71}\ensuremath{\times}12,71\hfill \\ & =& 1,894\hfill \end{array}\tag{3.42}
%       \end{equation}
%     }{% else, if it doesn't fit
%     \setlength{\mymathboxwidth}{\columnwidth}
%       \addtolength{\mymathboxwidth}{-48pt}
%     \par\vspace{12pt}\noindent\begin{minipage}{\columnwidth}
%     \parbox[t]{\mymathboxwidth}{\large\begin{math}
%     \mathrm{USD}/\mathrm{GBP}=\mathrm{USD}/\mathrm{ZAR}\ensuremath{\times}\mathrm{ZAR}/\mathrm{GBP}=\frac{1}{\mathrm{ZAR}/\mathrm{USD}}\ensuremath{\times}\mathrm{ZAR}/\mathrm{GBP}=\frac{1}{6,71}\ensuremath{\times}12,71=1,894\end{math}}\hfill
%     \parbox[t]{48pt}{\raggedleft 
%     (3.42)}
%     \end{minipage}\vspace{12pt}\par
%     }% end of conditional for this bit of math
%     \typeout{math as usual width = \the\mymathboxwidth}
%     
%         
% \label{m39335*notfhsst!!!underscore!!!id696}
% \begin{tabular}{cc}
% 	   \hspace*{-50pt}\raisebox{-8 mm}{ \includegraphics[width=0.5in]{col11306.imgs/pstip2.png}  }& 
% 
% 	\begin{minipage}{0.85\textwidth}
% 	\begin{note}
%       {tip: }Sometimes you will see exchange rates in the real world that do not appear
% to work exactly like this. This is usually because some financial institutions
% add other costs to the exchange rates, which alter the results. However, if you
% could remove the effect of those extra costs, the numbers would balance again.
% 	\end{note}
% 	\end{minipage}
% 	\end{tabular}
% 	\par
%       
% \label{m39335*secfhsst!!!underscore!!!id700}\vspace{.5cm} 
%       
%       \noindent
%       \hspace*{-30pt}\includegraphics[width=0.5in]{col11306.imgs/pspencil2.png}   \raisebox{25mm}{   
%       \begin{mdframed}[linewidth=4, leftmargin=40, rightmargin=40]  
%       \begin{exercise}
%     \noindent\textbf{Exercise 3.9:  Cross Exchange Rates }
%         \label{m39335*probfhsst!!!underscore!!!id701}
%         \label{m39335*id68151}If \$1 = R 6,40, and £1 = R11,58 what is the \$/£ exchange rate (i.e. the number of US\$ per £)? \par 
%         \vspace{5pt}
%         \label{m39335*solfhsst!!!underscore!!!id704}\noindent\textbf{Solution to Exercise } \label{m39335*listfhsst!!!underscore!!!id704}\begin{enumerate}[noitemsep, label=\textbf{Step} \textbf{\arabic*}. ] 
%             \leftskip=20pt\rightskip=\leftskip\item  
%         \label{m39335*id68176}The following are given:\par 
%         \label{m39335*id68180}\begin{itemize}[noitemsep]
%             \leftskip=20pt\rightskip=\leftskip\label{m39335*uid12}\item ZAR/USD rate = R6,40
% \label{m39335*uid13}\item ZAR/GBP rate = R11,58
% \end{itemize}
%         
%         \label{m39335*id68208}The following is required:\par 
%         \label{m39335*id68212}\begin{itemize}[noitemsep]
%             \leftskip=20pt\rightskip=\leftskip\label{m39335*uid14}\item USD/GBP rate
% \end{itemize}
%         
%         \item  
%         \label{m39335*id68233}We know that:\par 
%         \label{m39335*id68236}\nopagebreak\noindent{}
%           \settowidth{\mymathboxwidth}{\begin{equation}
%     \mathrm{USD}/\mathrm{GBP}=\mathrm{USD}/\mathrm{ZAR}\ensuremath{\times}\mathrm{ZAR}/\mathrm{GBP}.\tag{3.43}
%       \end{equation}
%     }
%     \typeout{Columnwidth = \the\columnwidth}\typeout{math as usual width = \the\mymathboxwidth}
%     \ifthenelse{\lengthtest{\mymathboxwidth < \columnwidth}}{% if the math fits, do it again, for real
%     \begin{equation}
%     \mathrm{USD}/\mathrm{GBP}=\mathrm{USD}/\mathrm{ZAR}\ensuremath{\times}\mathrm{ZAR}/\mathrm{GBP}.\tag{3.43}
%       \end{equation}
%     }{% else, if it doesn't fit
%     \setlength{\mymathboxwidth}{\columnwidth}
%       \addtolength{\mymathboxwidth}{-48pt}
%     \par\vspace{12pt}\noindent\begin{minipage}{\columnwidth}
%     \parbox[t]{\mymathboxwidth}{\large\begin{math}
%     \mathrm{USD}/\mathrm{GBP}=\mathrm{USD}/\mathrm{ZAR}\ensuremath{\times}\mathrm{ZAR}/\mathrm{GBP}.\end{math}}\hfill
%     \parbox[t]{48pt}{\raggedleft 
%     (3.43)}
%     \end{minipage}\vspace{12pt}\par
%     }% end of conditional for this bit of math
%     \typeout{math as usual width = \the\mymathboxwidth}
%     
%         
%         \item  
%         \label{m39335*id68280}\nopagebreak\noindent{}
%           \settowidth{\mymathboxwidth}{\begin{equation}
%     \begin{array}{ccc}\hfill \mathrm{USD}/\mathrm{GBP}& =& \mathrm{USD}/\mathrm{ZAR}\ensuremath{\times}\mathrm{ZAR}/\mathrm{GBP}\hfill \\ & =& \frac{1}{\mathrm{ZAR}/\mathrm{USD}}\ensuremath{\times}\mathrm{ZAR}/\mathrm{GBP}\hfill \\ & =& \frac{1}{6,40}\ensuremath{\times}11,58\hfill \\ & =& 1,8094\hfill \end{array}\tag{3.44}
%       \end{equation}
%     }
%     \typeout{Columnwidth = \the\columnwidth}\typeout{math as usual width = \the\mymathboxwidth}
%     \ifthenelse{\lengthtest{\mymathboxwidth < \columnwidth}}{% if the math fits, do it again, for real
%     \begin{equation}
%     \begin{array}{ccc}\hfill \mathrm{USD}/\mathrm{GBP}& =& \mathrm{USD}/\mathrm{ZAR}\ensuremath{\times}\mathrm{ZAR}/\mathrm{GBP}\hfill \\ & =& \frac{1}{\mathrm{ZAR}/\mathrm{USD}}\ensuremath{\times}\mathrm{ZAR}/\mathrm{GBP}\hfill \\ & =& \frac{1}{6,40}\ensuremath{\times}11,58\hfill \\ & =& 1,8094\hfill \end{array}\tag{3.44}
%       \end{equation}
%     }{% else, if it doesn't fit
%     \setlength{\mymathboxwidth}{\columnwidth}
%       \addtolength{\mymathboxwidth}{-48pt}
%     \par\vspace{12pt}\noindent\begin{minipage}{\columnwidth}
%     \parbox[t]{\mymathboxwidth}{\large\begin{math}
%     \mathrm{USD}/\mathrm{GBP}=\mathrm{USD}/\mathrm{ZAR}\ensuremath{\times}\mathrm{ZAR}/\mathrm{GBP}=\frac{1}{\mathrm{ZAR}/\mathrm{USD}}\ensuremath{\times}\mathrm{ZAR}/\mathrm{GBP}=\frac{1}{6,40}\ensuremath{\times}11,58=1,8094\end{math}}\hfill
%     \parbox[t]{48pt}{\raggedleft 
%     (3.44)}
%     \end{minipage}\vspace{12pt}\par
%     }% end of conditional for this bit of math
%     \typeout{math as usual width = \the\mymathboxwidth}
%     
%         
%         \item  
%         \label{m39335*id68414}\$ 1,8094 can be bought for £1. \par 
%         \end{enumerate}
%          
% 
%     \end{exercise}
%     \end{mdframed}
%     }
%     \noindent
%   
% \label{m39335*secfhsst!!!underscore!!!id841}
%             \subsubsection{  Investigation : Cross Exchange Rates - Alternative Method }
%             \nopagebreak
%             
%         \label{m39335*id68438}If \$1 = R
% 6,40, and £1 = R11,58 what is the \$/£ exchange rate (i.e. the
% number of US\$ per £)?\par 
%         \label{m39335*id68445}
%           \textbf{Overview of problem}
%         \par 
%         \label{m39335*id68454}You need the \$/£ exchange rate, in other words how many dollars must you pay for a pound. So you need £1. From the given information we know that it would cost you R11,58 to buy £1 and that \$ 1 = R6,40.\par 
%         \label{m39335*id68463}Use this information to:\par 
%         \label{m39335*id68466}\begin{enumerate}[noitemsep, label=\textbf{\arabic*}. ] 
%             \label{m39335*uid15}\item calculate how much R1 is worth in \$.
% \label{m39335*uid16}\item calculate how much R11,58 is worth in \$.
% \end{enumerate}
%         
%         \label{m39335*id68496}Do you get the same answer as in the worked example? \par 
      \label{m39335*uid17}
            \subsection{ Fluctuating exchange rates}
            \nopagebreak
        \label{m39335*id68515}If everyone wants to buy houses in a certain suburb, then house prices are going to go up - because the buyers will be competing to buy those houses. If there is a suburb where all residents want to move out, then there are lots of sellers and this will cause house prices in the area to fall - because the buyers would not have to struggle as much to find an eager seller.\par 
        \label{m39335*id68522}This is all about supply and demand, which is a very important section in the study of Economics. You can think about this is many different contexts, like stamp-collecting for example. If there is a stamp that lots of people want (high demand) and few people own (low supply) then that stamp is going to be expensive.\par 
        \label{m39335*id68528}And if you are starting to wonder why this is relevant - think about currencies. If you are going to visit London, then you have Rands but you need to ``buy" Pounds. The exchange rate is the price you have to pay to buy those Pounds.\par 
        \label{m39335*id68535}Think about a time where lots of South Africans are visiting the United Kingdom, and other South Africans are importing goods from the United Kingdom. That means there are lots of Rands (high supply) trying to buy Pounds. Pounds will start to become more expensive (compare this to the house price example at the start of this section if you are not convinced), and the exchange rate will change. In other words, for R1~000 you will get fewer Pounds than you would have before the exchange rate moved.\par 
        \label{m39335*id68547}Another context which might be useful for you to understand this: consider what would happen if people in other countries felt that South Africa was becoming a great place to live, and that more people were wanting to invest in South Africa - whether in properties, businesses - or just buying more goods from South Africa. There would be a greater demand for Rands - and the ``price of the Rand" would go up. In other words, people would need to use more Dollars, or Pounds, or Euros ... to buy the same amount of Rands. This is seen as a movement in exchange rates.\par 
        \label{m39335*id68560}Although it really does come down to supply and demand, it is interesting to think about what factors might affect the supply (people wanting to ``sell" a particular currency) and the demand (people trying to ``buy" another currency). This is covered in detail in the study of Economics, but let us look at some of the basic issues here.\par 
        \label{m39335*id68568}There are various factors which affect exchange rates, some of which have more economic rationale than others:\par 
        \label{m39335*id68573}\begin{itemize}[noitemsep]
            \label{m39335*uid18}\item economic factors (such as inflation figures, interest rates, trade deficit information, monetary policy and fiscal policy)
\label{m39335*uid19}\item political factors (such as uncertain political environment, or political unrest)
\label{m39335*uid20}\item market sentiments and market behaviour (for example if foreign exchange markets perceived a currency to be overvalued and starting selling the currency, this would cause the currency to fall in value - a self-fulfilling expectation).
\end{itemize}
\label{m39335*id7324}The exchange rate also influences the price we pay for certain goods. All countries import certain goods and export other goods. For example, South Africa has a lot of minerals (gold, platinum, etc.) that the rest of the world wants. So South Africa exports these minerals to the world for a certain price. The exchange rate at the time of export influences how much we can get for the minerals. In the same way, any goods that are imported are also influenced by the exchange rate. The price of petrol is a good example of something that is affected by the exchange rate. 
\par 
\label{m39335*secfhsst!!!underscore!!!id876}
            \subsection{Exercise: Foreign Exchange }
            \nopagebreak
            \label{m39335*id68624}\begin{enumerate}[noitemsep, label=\textbf{\arabic*}. ] 
            \label{m39335*uid21}\item I want to buy an IPOD that costs £100, with the exchange rate currently at
\begin{math}£1=R14\end{math}. I believe the exchange rate will reach \begin{math}R12\end{math} in a month.
\label{m39335*id68677}\begin{enumerate}[noitemsep, label=\textbf{\alph*}. ] 
            \label{m39335*uid22}\item How much will the MP3 player cost in Rands, if I buy it now?
\label{m39335*uid23}\item How much will I save if the exchange rate drops to \begin{math}R12\end{math}?
\label{m39335*uid24}\item How much will I lose if the exchange rate moves to \begin{math}R15\end{math}?
\end{enumerate}
        \label{m39335*uid25}\item Study the following exchange rate table:
    % \textbf{m39335*id68755}\par
    % how many colspecs?  3
          % name: cnx:colspec
            % colnum: 1
            % colwidth: 10*
            % latex-name: columna
            % colname: 
            % align/tgroup-align/default: //left
            % -------------------------
            % name: cnx:colspec
            % colnum: 2
            % colwidth: 10*
            % latex-name: columnb
            % colname: 
            % align/tgroup-align/default: //left
            % -------------------------
            % name: cnx:colspec
            % colnum: 3
            % colwidth: 10*
            % latex-name: columnc
            % colname: 
            % align/tgroup-align/default: //left
            % -------------------------
    \setlength\mytablespace{6\tabcolsep}
    \addtolength\mytablespace{4\arrayrulewidth}
    \setlength\mytablewidth{\linewidth}
    \setlength\mytableroom{\mytablewidth}
    \addtolength\mytableroom{-\mytablespace}
    \setlength\myfixedwidth{0pt}
    \setlength\mystarwidth{\mytableroom}
        \addtolength\mystarwidth{-\myfixedwidth}
        \divide\mystarwidth 30
      % ----- Begin capturing width of table in LR mode woof
      \settowidth{\mytableboxwidth}{\begin{tabular}[t]{|l|l|l|}\hline
    % count in rowspan-info-nodeset: 3
    % align/colidx: left,1
    % rowcount: '0' | start: 'false' | colidx: '1'
        % Formatting a regular cell and recurring on the next sibling
        Country &
      % align/colidx: left,2
    % rowcount: '0' | start: 'false' | colidx: '2'
        % Formatting a regular cell and recurring on the next sibling
        Currency &
      % align/colidx: left,3
    % rowcount: '0' | start: 'false' | colidx: '3'
        % Formatting a regular cell and recurring on the next sibling
        Exchange Rate% make-rowspan-placeholders
    % rowspan info: col1 '0' | 'false' | '' || col2 '0' | 'false' | '' || col3 '0' | 'false' | ''
     \tabularnewline\cline{1-1}\cline{2-2}\cline{3-3}
      %--------------------------------------------------------------------
    % align/colidx: left,1
    % rowcount: '0' | start: 'false' | colidx: '1'
        % Formatting a regular cell and recurring on the next sibling
        United Kingdom (UK) &
      % align/colidx: left,2
    % rowcount: '0' | start: 'false' | colidx: '2'
        % Formatting a regular cell and recurring on the next sibling
        Pounds(£) &
      % align/colidx: left,3
    % rowcount: '0' | start: 'false' | colidx: '3'
        % Formatting a regular cell and recurring on the next sibling
        \begin{math}R14,13\end{math}% make-rowspan-placeholders
    % rowspan info: col1 '0' | 'false' | '' || col2 '0' | 'false' | '' || col3 '0' | 'false' | ''
     \tabularnewline\cline{1-1}\cline{2-2}\cline{3-3}
      %--------------------------------------------------------------------
    % align/colidx: left,1
    % rowcount: '0' | start: 'false' | colidx: '1'
        % Formatting a regular cell and recurring on the next sibling
        United States (USA) &
      % align/colidx: left,2
    % rowcount: '0' | start: 'false' | colidx: '2'
        % Formatting a regular cell and recurring on the next sibling
        Dollars (\$) &
      % align/colidx: left,3
    % rowcount: '0' | start: 'false' | colidx: '3'
        % Formatting a regular cell and recurring on the next sibling
        \begin{math}R7,04\end{math}% make-rowspan-placeholders
    % rowspan info: col1 '0' | 'false' | '' || col2 '0' | 'false' | '' || col3 '0' | 'false' | ''
     \tabularnewline\cline{1-1}\cline{2-2}\cline{3-3}
      %--------------------------------------------------------------------
    \end{tabular}} % end mytableboxwidth set      
      % ----- End capturing width of table in LR mode
        % ----- LR or paragraph mode: must test
        % ----- Begin capturing height of table
        \settoheight{\mytableboxheight}{\begin{tabular}[t]{|l|l|l|}\hline
    % count in rowspan-info-nodeset: 3
    % align/colidx: left,1
    % rowcount: '0' | start: 'false' | colidx: '1'
        % Formatting a regular cell and recurring on the next sibling
        Country &
      % align/colidx: left,2
    % rowcount: '0' | start: 'false' | colidx: '2'
        % Formatting a regular cell and recurring on the next sibling
        Currency &
      % align/colidx: left,3
    % rowcount: '0' | start: 'false' | colidx: '3'
        % Formatting a regular cell and recurring on the next sibling
        Exchange Rate% make-rowspan-placeholders
    % rowspan info: col1 '0' | 'false' | '' || col2 '0' | 'false' | '' || col3 '0' | 'false' | ''
     \tabularnewline\cline{1-1}\cline{2-2}\cline{3-3}
      %--------------------------------------------------------------------
    % align/colidx: left,1
    % rowcount: '0' | start: 'false' | colidx: '1'
        % Formatting a regular cell and recurring on the next sibling
        United Kingdom (UK) &
      % align/colidx: left,2
    % rowcount: '0' | start: 'false' | colidx: '2'
        % Formatting a regular cell and recurring on the next sibling
        Pounds(£) &
      % align/colidx: left,3
    % rowcount: '0' | start: 'false' | colidx: '3'
        % Formatting a regular cell and recurring on the next sibling
        \begin{math}R14,13\end{math}% make-rowspan-placeholders
    % rowspan info: col1 '0' | 'false' | '' || col2 '0' | 'false' | '' || col3 '0' | 'false' | ''
     \tabularnewline\cline{1-1}\cline{2-2}\cline{3-3}
      %--------------------------------------------------------------------
    % align/colidx: left,1
    % rowcount: '0' | start: 'false' | colidx: '1'
        % Formatting a regular cell and recurring on the next sibling
        United States (USA) &
      % align/colidx: left,2
    % rowcount: '0' | start: 'false' | colidx: '2'
        % Formatting a regular cell and recurring on the next sibling
        Dollars (\$) &
      % align/colidx: left,3
    % rowcount: '0' | start: 'false' | colidx: '3'
        % Formatting a regular cell and recurring on the next sibling
        \begin{math}R7,04\end{math}% make-rowspan-placeholders
    % rowspan info: col1 '0' | 'false' | '' || col2 '0' | 'false' | '' || col3 '0' | 'false' | ''
     \tabularnewline\cline{1-1}\cline{2-2}\cline{3-3}
      %--------------------------------------------------------------------
    \end{tabular}} % end mytableboxheight set
        \settodepth{\mytableboxdepth}{\begin{tabular}[t]{|l|l|l|}\hline
    % count in rowspan-info-nodeset: 3
    % align/colidx: left,1
    % rowcount: '0' | start: 'false' | colidx: '1'
        % Formatting a regular cell and recurring on the next sibling
        Country &
      % align/colidx: left,2
    % rowcount: '0' | start: 'false' | colidx: '2'
        % Formatting a regular cell and recurring on the next sibling
        Currency &
      % align/colidx: left,3
    % rowcount: '0' | start: 'false' | colidx: '3'
        % Formatting a regular cell and recurring on the next sibling
        Exchange Rate% make-rowspan-placeholders
    % rowspan info: col1 '0' | 'false' | '' || col2 '0' | 'false' | '' || col3 '0' | 'false' | ''
     \tabularnewline\cline{1-1}\cline{2-2}\cline{3-3}
      %--------------------------------------------------------------------
    % align/colidx: left,1
    % rowcount: '0' | start: 'false' | colidx: '1'
        % Formatting a regular cell and recurring on the next sibling
        United Kingdom (UK) &
      % align/colidx: left,2
    % rowcount: '0' | start: 'false' | colidx: '2'
        % Formatting a regular cell and recurring on the next sibling
        Pounds(£) &
      % align/colidx: left,3
    % rowcount: '0' | start: 'false' | colidx: '3'
        % Formatting a regular cell and recurring on the next sibling
        \begin{math}R14,13\end{math}% make-rowspan-placeholders
    % rowspan info: col1 '0' | 'false' | '' || col2 '0' | 'false' | '' || col3 '0' | 'false' | ''
     \tabularnewline\cline{1-1}\cline{2-2}\cline{3-3}
      %--------------------------------------------------------------------
    % align/colidx: left,1
    % rowcount: '0' | start: 'false' | colidx: '1'
        % Formatting a regular cell and recurring on the next sibling
        United States (USA) &
      % align/colidx: left,2
    % rowcount: '0' | start: 'false' | colidx: '2'
        % Formatting a regular cell and recurring on the next sibling
        Dollars (\$) &
      % align/colidx: left,3
    % rowcount: '0' | start: 'false' | colidx: '3'
        % Formatting a regular cell and recurring on the next sibling
        \begin{math}R7,04\end{math}% make-rowspan-placeholders
    % rowspan info: col1 '0' | 'false' | '' || col2 '0' | 'false' | '' || col3 '0' | 'false' | ''
     \tabularnewline\cline{1-1}\cline{2-2}\cline{3-3}
      %--------------------------------------------------------------------
    \end{tabular}} % end mytableboxdepth set
        \addtolength{\mytableboxheight}{\mytableboxdepth}
        % ----- End capturing height of table        
        \ifthenelse{\mytableboxwidth<\textwidth}{% the table fits in LR mode
          \addtolength{\mytableboxwidth}{-\mytablespace}
          \typeout{textheight: \the\textheight}
          \typeout{mytableboxheight: \the\mytableboxheight}
          \typeout{textwidth: \the\textwidth}
          \typeout{mytableboxwidth: \the\mytableboxwidth}
          \ifthenelse{\mytableboxheight<\textheight}{%
    % \begin{table}[H]
    % \\ 'id2866031' '1'
        \begin{center}
      \label{m39335*id68755}
    \noindent
    \begin{tabular}[t]{|l|l|l|}\hline
    % count in rowspan-info-nodeset: 3
    % align/colidx: left,1
    % rowcount: '0' | start: 'false' | colidx: '1'
        % Formatting a regular cell and recurring on the next sibling
        Country &
      % align/colidx: left,2
    % rowcount: '0' | start: 'false' | colidx: '2'
        % Formatting a regular cell and recurring on the next sibling
        Currency &
      % align/colidx: left,3
    % rowcount: '0' | start: 'false' | colidx: '3'
        % Formatting a regular cell and recurring on the next sibling
        Exchange Rate% make-rowspan-placeholders
    % rowspan info: col1 '0' | 'false' | '' || col2 '0' | 'false' | '' || col3 '0' | 'false' | ''
     \tabularnewline\cline{1-1}\cline{2-2}\cline{3-3}
      %--------------------------------------------------------------------
    % align/colidx: left,1
    % rowcount: '0' | start: 'false' | colidx: '1'
        % Formatting a regular cell and recurring on the next sibling
        United Kingdom (UK) &
      % align/colidx: left,2
    % rowcount: '0' | start: 'false' | colidx: '2'
        % Formatting a regular cell and recurring on the next sibling
        Pounds(£) &
      % align/colidx: left,3
    % rowcount: '0' | start: 'false' | colidx: '3'
        % Formatting a regular cell and recurring on the next sibling
        \begin{math}R14,13\end{math}% make-rowspan-placeholders
    % rowspan info: col1 '0' | 'false' | '' || col2 '0' | 'false' | '' || col3 '0' | 'false' | ''
     \tabularnewline\cline{1-1}\cline{2-2}\cline{3-3}
      %--------------------------------------------------------------------
    % align/colidx: left,1
    % rowcount: '0' | start: 'false' | colidx: '1'
        % Formatting a regular cell and recurring on the next sibling
        United States (USA) &
      % align/colidx: left,2
    % rowcount: '0' | start: 'false' | colidx: '2'
        % Formatting a regular cell and recurring on the next sibling
        Dollars (\$) &
      % align/colidx: left,3
    % rowcount: '0' | start: 'false' | colidx: '3'
        % Formatting a regular cell and recurring on the next sibling
        \begin{math}R7,04\end{math}% make-rowspan-placeholders
    % rowspan info: col1 '0' | 'false' | '' || col2 '0' | 'false' | '' || col3 '0' | 'false' | ''
     \tabularnewline\cline{1-1}\cline{2-2}\cline{3-3}
      %--------------------------------------------------------------------
    \end{tabular}
      \end{center}
    \begin{center}{\small\bfseries Table 3.3}\end{center}
    %\end{table}
          }{ % else
    % \begin{table}[H]
    % \\ 'id2866031' '1'
        \begin{center}
      \label{m39335*id68755}
    \noindent
    \tabletail{%
        \hline
        \multicolumn{3}{|p{\mytableboxwidth}|}{\raggedleft \small \sl continued on next page}\\
        \hline
      }
      \tablelasttail{}
      \begin{xtabular}[t]{|l|l|l|}\hline
    % count in rowspan-info-nodeset: 3
    % align/colidx: left,1
    % rowcount: '0' | start: 'false' | colidx: '1'
        % Formatting a regular cell and recurring on the next sibling
        Country &
      % align/colidx: left,2
    % rowcount: '0' | start: 'false' | colidx: '2'
        % Formatting a regular cell and recurring on the next sibling
        Currency &
      % align/colidx: left,3
    % rowcount: '0' | start: 'false' | colidx: '3'
        % Formatting a regular cell and recurring on the next sibling
        Exchange Rate% make-rowspan-placeholders
    % rowspan info: col1 '0' | 'false' | '' || col2 '0' | 'false' | '' || col3 '0' | 'false' | ''
     \tabularnewline\cline{1-1}\cline{2-2}\cline{3-3}
      %--------------------------------------------------------------------
    % align/colidx: left,1
    % rowcount: '0' | start: 'false' | colidx: '1'
        % Formatting a regular cell and recurring on the next sibling
        United Kingdom (UK) &
      % align/colidx: left,2
    % rowcount: '0' | start: 'false' | colidx: '2'
        % Formatting a regular cell and recurring on the next sibling
        Pounds(£) &
      % align/colidx: left,3
    % rowcount: '0' | start: 'false' | colidx: '3'
        % Formatting a regular cell and recurring on the next sibling
        \begin{math}R14,13\end{math}% make-rowspan-placeholders
    % rowspan info: col1 '0' | 'false' | '' || col2 '0' | 'false' | '' || col3 '0' | 'false' | ''
     \tabularnewline\cline{1-1}\cline{2-2}\cline{3-3}
      %--------------------------------------------------------------------
    % align/colidx: left,1
    % rowcount: '0' | start: 'false' | colidx: '1'
        % Formatting a regular cell and recurring on the next sibling
        United States (USA) &
      % align/colidx: left,2
    % rowcount: '0' | start: 'false' | colidx: '2'
        % Formatting a regular cell and recurring on the next sibling
        Dollars (\$) &
      % align/colidx: left,3
    % rowcount: '0' | start: 'false' | colidx: '3'
        % Formatting a regular cell and recurring on the next sibling
        \begin{math}R7,04\end{math}% make-rowspan-placeholders
    % rowspan info: col1 '0' | 'false' | '' || col2 '0' | 'false' | '' || col3 '0' | 'false' | ''
     \tabularnewline\cline{1-1}\cline{2-2}\cline{3-3}
      %--------------------------------------------------------------------
    \end{xtabular}
      \end{center}
    \begin{center}{\small\bfseries Table 3.3}\end{center}
    %\end{table}
          } % 
        }{% else
        % typeset the table in paragraph mode
        % ----- Begin capturing height of table
        \settoheight{\mytableboxheight}{\begin{tabular*}{\mytablewidth}[t]{|p{10\mystarwidth}|p{10\mystarwidth}|p{10\mystarwidth}|}\hline
    % count in rowspan-info-nodeset: 3
    % align/colidx: left,1
    % rowcount: '0' | start: 'false' | colidx: '1'
        % Formatting a regular cell and recurring on the next sibling
        Country &
      % align/colidx: left,2
    % rowcount: '0' | start: 'false' | colidx: '2'
        % Formatting a regular cell and recurring on the next sibling
        Currency &
      % align/colidx: left,3
    % rowcount: '0' | start: 'false' | colidx: '3'
        % Formatting a regular cell and recurring on the next sibling
        Exchange Rate% make-rowspan-placeholders
    % rowspan info: col1 '0' | 'false' | '' || col2 '0' | 'false' | '' || col3 '0' | 'false' | ''
     \tabularnewline\cline{1-1}\cline{2-2}\cline{3-3}
      %--------------------------------------------------------------------
    % align/colidx: left,1
    % rowcount: '0' | start: 'false' | colidx: '1'
        % Formatting a regular cell and recurring on the next sibling
        United Kingdom (UK) &
      % align/colidx: left,2
    % rowcount: '0' | start: 'false' | colidx: '2'
        % Formatting a regular cell and recurring on the next sibling
        Pounds(£) &
      % align/colidx: left,3
    % rowcount: '0' | start: 'false' | colidx: '3'
        % Formatting a regular cell and recurring on the next sibling
        \begin{math}R14,13\end{math}% make-rowspan-placeholders
    % rowspan info: col1 '0' | 'false' | '' || col2 '0' | 'false' | '' || col3 '0' | 'false' | ''
     \tabularnewline\cline{1-1}\cline{2-2}\cline{3-3}
      %--------------------------------------------------------------------
    % align/colidx: left,1
    % rowcount: '0' | start: 'false' | colidx: '1'
        % Formatting a regular cell and recurring on the next sibling
        United States (USA) &
      % align/colidx: left,2
    % rowcount: '0' | start: 'false' | colidx: '2'
        % Formatting a regular cell and recurring on the next sibling
        Dollars (\$) &
      % align/colidx: left,3
    % rowcount: '0' | start: 'false' | colidx: '3'
        % Formatting a regular cell and recurring on the next sibling
        \begin{math}R7,04\end{math}% make-rowspan-placeholders
    % rowspan info: col1 '0' | 'false' | '' || col2 '0' | 'false' | '' || col3 '0' | 'false' | ''
     \tabularnewline\cline{1-1}\cline{2-2}\cline{3-3}
      %--------------------------------------------------------------------
    \end{tabular*}} % end mytableboxheight set
        \settodepth{\mytableboxdepth}{\begin{tabular*}{\mytablewidth}[t]{|p{10\mystarwidth}|p{10\mystarwidth}|p{10\mystarwidth}|}\hline
    % count in rowspan-info-nodeset: 3
    % align/colidx: left,1
    % rowcount: '0' | start: 'false' | colidx: '1'
        % Formatting a regular cell and recurring on the next sibling
        Country &
      % align/colidx: left,2
    % rowcount: '0' | start: 'false' | colidx: '2'
        % Formatting a regular cell and recurring on the next sibling
        Currency &
      % align/colidx: left,3
    % rowcount: '0' | start: 'false' | colidx: '3'
        % Formatting a regular cell and recurring on the next sibling
        Exchange Rate% make-rowspan-placeholders
    % rowspan info: col1 '0' | 'false' | '' || col2 '0' | 'false' | '' || col3 '0' | 'false' | ''
     \tabularnewline\cline{1-1}\cline{2-2}\cline{3-3}
      %--------------------------------------------------------------------
    % align/colidx: left,1
    % rowcount: '0' | start: 'false' | colidx: '1'
        % Formatting a regular cell and recurring on the next sibling
        United Kingdom (UK) &
      % align/colidx: left,2
    % rowcount: '0' | start: 'false' | colidx: '2'
        % Formatting a regular cell and recurring on the next sibling
        Pounds(£) &
      % align/colidx: left,3
    % rowcount: '0' | start: 'false' | colidx: '3'
        % Formatting a regular cell and recurring on the next sibling
        \begin{math}R14,13\end{math}% make-rowspan-placeholders
    % rowspan info: col1 '0' | 'false' | '' || col2 '0' | 'false' | '' || col3 '0' | 'false' | ''
     \tabularnewline\cline{1-1}\cline{2-2}\cline{3-3}
      %--------------------------------------------------------------------
    % align/colidx: left,1
    % rowcount: '0' | start: 'false' | colidx: '1'
        % Formatting a regular cell and recurring on the next sibling
        United States (USA) &
      % align/colidx: left,2
    % rowcount: '0' | start: 'false' | colidx: '2'
        % Formatting a regular cell and recurring on the next sibling
        Dollars (\$) &
      % align/colidx: left,3
    % rowcount: '0' | start: 'false' | colidx: '3'
        % Formatting a regular cell and recurring on the next sibling
        \begin{math}R7,04\end{math}% make-rowspan-placeholders
    % rowspan info: col1 '0' | 'false' | '' || col2 '0' | 'false' | '' || col3 '0' | 'false' | ''
     \tabularnewline\cline{1-1}\cline{2-2}\cline{3-3}
      %--------------------------------------------------------------------
    \end{tabular*}} % end mytableboxdepth set
        \addtolength{\mytableboxheight}{\mytableboxdepth}
        % ----- End capturing height of table
        \typeout{textheight: \the\textheight}
        \typeout{mytableboxheight: \the\mytableboxheight}
        \typeout{table too wide, outputting in para mode}
    % \begin{table}[H]
    % \\ 'id2866031' '1'
        \begin{center}
      \label{m39335*id68755}
    \noindent
    \tabletail{%
        \hline
        \multicolumn{3}{|p{\mytableroom}|}{\raggedleft \small \sl continued on next page}\\
        \hline
      }
      \tablelasttail{}
      \begin{xtabular*}{\mytablewidth}[t]{|p{10\mystarwidth}|p{10\mystarwidth}|p{10\mystarwidth}|}\hline
    % count in rowspan-info-nodeset: 3
    % align/colidx: left,1
    % rowcount: '0' | start: 'false' | colidx: '1'
        % Formatting a regular cell and recurring on the next sibling
        Country &
      % align/colidx: left,2
    % rowcount: '0' | start: 'false' | colidx: '2'
        % Formatting a regular cell and recurring on the next sibling
        Currency &
      % align/colidx: left,3
    % rowcount: '0' | start: 'false' | colidx: '3'
        % Formatting a regular cell and recurring on the next sibling
        Exchange Rate% make-rowspan-placeholders
    % rowspan info: col1 '0' | 'false' | '' || col2 '0' | 'false' | '' || col3 '0' | 'false' | ''
     \tabularnewline\cline{1-1}\cline{2-2}\cline{3-3}
      %--------------------------------------------------------------------
    % align/colidx: left,1
    % rowcount: '0' | start: 'false' | colidx: '1'
        % Formatting a regular cell and recurring on the next sibling
        United Kingdom (UK) &
      % align/colidx: left,2
    % rowcount: '0' | start: 'false' | colidx: '2'
        % Formatting a regular cell and recurring on the next sibling
        Pounds(£) &
      % align/colidx: left,3
    % rowcount: '0' | start: 'false' | colidx: '3'
        % Formatting a regular cell and recurring on the next sibling
        \begin{math}R14,13\end{math}% make-rowspan-placeholders
    % rowspan info: col1 '0' | 'false' | '' || col2 '0' | 'false' | '' || col3 '0' | 'false' | ''
     \tabularnewline\cline{1-1}\cline{2-2}\cline{3-3}
      %--------------------------------------------------------------------
    % align/colidx: left,1
    % rowcount: '0' | start: 'false' | colidx: '1'
        % Formatting a regular cell and recurring on the next sibling
        United States (USA) &
      % align/colidx: left,2
    % rowcount: '0' | start: 'false' | colidx: '2'
        % Formatting a regular cell and recurring on the next sibling
        Dollars (\$) &
      % align/colidx: left,3
    % rowcount: '0' | start: 'false' | colidx: '3'
        % Formatting a regular cell and recurring on the next sibling
        \begin{math}R7,04\end{math}% make-rowspan-placeholders
    % rowspan info: col1 '0' | 'false' | '' || col2 '0' | 'false' | '' || col3 '0' | 'false' | ''
     \tabularnewline\cline{1-1}\cline{2-2}\cline{3-3}
      %--------------------------------------------------------------------
    \end{xtabular*}
      \end{center}
    \begin{center}{\small\bfseries Table 3.3}\end{center}
    %\end{table}
        }% ending lr/para test clause
    \par
  \label{m39335*id68888}\begin{enumerate}[noitemsep, label=\textbf{\alph*}. ] 
            \label{m39335*uid26}\item In South Africa the cost of a new Honda Civic is \begin{math}R173\phantom{\rule{3.33333pt}{0ex}}400\end{math}. In England the same vehicle costs \begin{math}£12\phantom{\rule{3.33333pt}{0ex}}200\end{math} and in the USA \$ \begin{math}21\phantom{\rule{3.33333pt}{0ex}}900\end{math}. In which country is the car the cheapest when you compare the prices converted to South African Rand ?
\label{m39335*uid27}\item Sollie and Arinda are waiters in a South African restaurant attracting many tourists from abroad. Sollie gets a \begin{math}£6\end{math} tip from a tourist and Arinda gets \$ 12.
How many South African Rand did each one get ?
\end{enumerate}
        \end{enumerate}
    \label{m39335*cid7}
\par \raisebox{-5 pt}{\includegraphics[width=0.5cm]{col11306.imgs/summary_www.png}} Find the answers with the shortcodes:
 \par \begin{tabular}[h]{cccccc}
 (1.) lc4  &  (2.) lc2  & \end{tabular}
            \section{ Summary}
            \nopagebreak
            \label{m39335*eip-923}\begin{itemize}[noitemsep]
            \item There are two types of interest: simple and compound.\item The following table summarises the key definitions that are used in both simple and compound interest.
    % \textbf{m39335*id75382}\par
    % how many colspecs?  2
          % name: cnx:colspec
            % colnum: 1
            % colwidth: 10*
            % latex-name: columna
            % colname: 
            % align/tgroup-align/default: //left
            % -------------------------
            % name: cnx:colspec
            % colnum: 2
            % colwidth: 10*
            % latex-name: columnb
            % colname: 
            % align/tgroup-align/default: //left
            % -------------------------
    \setlength\mytablespace{4\tabcolsep}
    \addtolength\mytablespace{3\arrayrulewidth}
    \setlength\mytablewidth{\linewidth}
    \setlength\mytableroom{\mytablewidth}
    \addtolength\mytableroom{-\mytablespace}
    \setlength\myfixedwidth{0pt}
    \setlength\mystarwidth{\mytableroom}
        \addtolength\mystarwidth{-\myfixedwidth}
        \divide\mystarwidth 20
      % ----- Begin capturing width of table in LR mode woof
      \settowidth{\mytableboxwidth}{\begin{tabular}[t]{|l|l|}\hline
    % count in rowspan-info-nodeset: 2
    % align/colidx: left,1
    % rowcount: '0' | start: 'false' | colidx: '1'
        % Formatting a regular cell and recurring on the next sibling
                  \begin{math}P\end{math}
                 &
      % align/colidx: left,2
    % rowcount: '0' | start: 'false' | colidx: '2'
        % Formatting a regular cell and recurring on the next sibling
        Principal (the amount of money at the starting point of the calculation)% make-rowspan-placeholders
    % rowspan info: col1 '0' | 'false' | '' || col2 '0' | 'false' | ''
     \tabularnewline\cline{1-1}\cline{2-2}
      %--------------------------------------------------------------------
    % align/colidx: left,1
    % rowcount: '0' | start: 'false' | colidx: '1'
        % Formatting a regular cell and recurring on the next sibling
                  \begin{math}A\end{math}
                 &
      % align/colidx: left,2
    % rowcount: '0' | start: 'false' | colidx: '2'
        % Formatting a regular cell and recurring on the next sibling
        Closing balance (the amount of money at the ending point of the calculation)% make-rowspan-placeholders
    % rowspan info: col1 '0' | 'false' | '' || col2 '0' | 'false' | ''
     \tabularnewline\cline{1-1}\cline{2-2}
      %--------------------------------------------------------------------
    % align/colidx: left,1
    % rowcount: '0' | start: 'false' | colidx: '1'
        % Formatting a regular cell and recurring on the next sibling
                  \begin{math}i\end{math}
                 &
      % align/colidx: left,2
    % rowcount: '0' | start: 'false' | colidx: '2'
        % Formatting a regular cell and recurring on the next sibling
        interest rate, normally the effective rate per annum% make-rowspan-placeholders
    % rowspan info: col1 '0' | 'false' | '' || col2 '0' | 'false' | ''
     \tabularnewline\cline{1-1}\cline{2-2}
      %--------------------------------------------------------------------
    % align/colidx: left,1
    % rowcount: '0' | start: 'false' | colidx: '1'
        % Formatting a regular cell and recurring on the next sibling
                  \begin{math}n\end{math}
                 &
      % align/colidx: left,2
    % rowcount: '0' | start: 'false' | colidx: '2'
        % Formatting a regular cell and recurring on the next sibling
        period for which the investment is made% make-rowspan-placeholders
    % rowspan info: col1 '0' | 'false' | '' || col2 '0' | 'false' | ''
     \tabularnewline\cline{1-1}\cline{2-2}
      %--------------------------------------------------------------------
    \end{tabular}} % end mytableboxwidth set      
      % ----- End capturing width of table in LR mode
        % ----- LR or paragraph mode: must test
        % ----- Begin capturing height of table
        \settoheight{\mytableboxheight}{\begin{tabular}[t]{|l|l|}\hline
    % count in rowspan-info-nodeset: 2
    % align/colidx: left,1
    % rowcount: '0' | start: 'false' | colidx: '1'
        % Formatting a regular cell and recurring on the next sibling
                  \begin{math}P\end{math}
                 &
      % align/colidx: left,2
    % rowcount: '0' | start: 'false' | colidx: '2'
        % Formatting a regular cell and recurring on the next sibling
        Principal (the amount of money at the starting point of the calculation)% make-rowspan-placeholders
    % rowspan info: col1 '0' | 'false' | '' || col2 '0' | 'false' | ''
     \tabularnewline\cline{1-1}\cline{2-2}
      %--------------------------------------------------------------------
    % align/colidx: left,1
    % rowcount: '0' | start: 'false' | colidx: '1'
        % Formatting a regular cell and recurring on the next sibling
                  \begin{math}A\end{math}
                 &
      % align/colidx: left,2
    % rowcount: '0' | start: 'false' | colidx: '2'
        % Formatting a regular cell and recurring on the next sibling
        Closing balance (the amount of money at the ending point of the calculation)% make-rowspan-placeholders
    % rowspan info: col1 '0' | 'false' | '' || col2 '0' | 'false' | ''
     \tabularnewline\cline{1-1}\cline{2-2}
      %--------------------------------------------------------------------
    % align/colidx: left,1
    % rowcount: '0' | start: 'false' | colidx: '1'
        % Formatting a regular cell and recurring on the next sibling
                  \begin{math}i\end{math}
                 &
      % align/colidx: left,2
    % rowcount: '0' | start: 'false' | colidx: '2'
        % Formatting a regular cell and recurring on the next sibling
        interest rate, normally the effective rate per annum% make-rowspan-placeholders
    % rowspan info: col1 '0' | 'false' | '' || col2 '0' | 'false' | ''
     \tabularnewline\cline{1-1}\cline{2-2}
      %--------------------------------------------------------------------
    % align/colidx: left,1
    % rowcount: '0' | start: 'false' | colidx: '1'
        % Formatting a regular cell and recurring on the next sibling
                  \begin{math}n\end{math}
                 &
      % align/colidx: left,2
    % rowcount: '0' | start: 'false' | colidx: '2'
        % Formatting a regular cell and recurring on the next sibling
        period for which the investment is made% make-rowspan-placeholders
    % rowspan info: col1 '0' | 'false' | '' || col2 '0' | 'false' | ''
     \tabularnewline\cline{1-1}\cline{2-2}
      %--------------------------------------------------------------------
    \end{tabular}} % end mytableboxheight set
        \settodepth{\mytableboxdepth}{\begin{tabular}[t]{|l|l|}\hline
    % count in rowspan-info-nodeset: 2
    % align/colidx: left,1
    % rowcount: '0' | start: 'false' | colidx: '1'
        % Formatting a regular cell and recurring on the next sibling
                  \begin{math}P\end{math}
                 &
      % align/colidx: left,2
    % rowcount: '0' | start: 'false' | colidx: '2'
        % Formatting a regular cell and recurring on the next sibling
        Principal (the amount of money at the starting point of the calculation)% make-rowspan-placeholders
    % rowspan info: col1 '0' | 'false' | '' || col2 '0' | 'false' | ''
     \tabularnewline\cline{1-1}\cline{2-2}
      %--------------------------------------------------------------------
    % align/colidx: left,1
    % rowcount: '0' | start: 'false' | colidx: '1'
        % Formatting a regular cell and recurring on the next sibling
                  \begin{math}A\end{math}
                 &
      % align/colidx: left,2
    % rowcount: '0' | start: 'false' | colidx: '2'
        % Formatting a regular cell and recurring on the next sibling
        Closing balance (the amount of money at the ending point of the calculation)% make-rowspan-placeholders
    % rowspan info: col1 '0' | 'false' | '' || col2 '0' | 'false' | ''
     \tabularnewline\cline{1-1}\cline{2-2}
      %--------------------------------------------------------------------
    % align/colidx: left,1
    % rowcount: '0' | start: 'false' | colidx: '1'
        % Formatting a regular cell and recurring on the next sibling
                  \begin{math}i\end{math}
                 &
      % align/colidx: left,2
    % rowcount: '0' | start: 'false' | colidx: '2'
        % Formatting a regular cell and recurring on the next sibling
        interest rate, normally the effective rate per annum% make-rowspan-placeholders
    % rowspan info: col1 '0' | 'false' | '' || col2 '0' | 'false' | ''
     \tabularnewline\cline{1-1}\cline{2-2}
      %--------------------------------------------------------------------
    % align/colidx: left,1
    % rowcount: '0' | start: 'false' | colidx: '1'
        % Formatting a regular cell and recurring on the next sibling
                  \begin{math}n\end{math}
                 &
      % align/colidx: left,2
    % rowcount: '0' | start: 'false' | colidx: '2'
        % Formatting a regular cell and recurring on the next sibling
        period for which the investment is made% make-rowspan-placeholders
    % rowspan info: col1 '0' | 'false' | '' || col2 '0' | 'false' | ''
     \tabularnewline\cline{1-1}\cline{2-2}
      %--------------------------------------------------------------------
    \end{tabular}} % end mytableboxdepth set
        \addtolength{\mytableboxheight}{\mytableboxdepth}
        % ----- End capturing height of table        
        \ifthenelse{\mytableboxwidth<\textwidth}{% the table fits in LR mode
          \addtolength{\mytableboxwidth}{-\mytablespace}
          \typeout{textheight: \the\textheight}
          \typeout{mytableboxheight: \the\mytableboxheight}
          \typeout{textwidth: \the\textwidth}
          \typeout{mytableboxwidth: \the\mytableboxwidth}
          \ifthenelse{\mytableboxheight<\textheight}{%
    % \begin{table}[H]
    % \\ 'id2866211' '1'
        \begin{center}
      \label{m39335*id75382}
    \noindent
    \begin{tabular}[t]{|l|l|}\hline
    % count in rowspan-info-nodeset: 2
    % align/colidx: left,1
    % rowcount: '0' | start: 'false' | colidx: '1'
        % Formatting a regular cell and recurring on the next sibling
                  \begin{math}P\end{math}
                 &
      % align/colidx: left,2
    % rowcount: '0' | start: 'false' | colidx: '2'
        % Formatting a regular cell and recurring on the next sibling
        Principal (the amount of money at the starting point of the calculation)% make-rowspan-placeholders
    % rowspan info: col1 '0' | 'false' | '' || col2 '0' | 'false' | ''
     \tabularnewline\cline{1-1}\cline{2-2}
      %--------------------------------------------------------------------
    % align/colidx: left,1
    % rowcount: '0' | start: 'false' | colidx: '1'
        % Formatting a regular cell and recurring on the next sibling
                  \begin{math}A\end{math}
                 &
      % align/colidx: left,2
    % rowcount: '0' | start: 'false' | colidx: '2'
        % Formatting a regular cell and recurring on the next sibling
        Closing balance (the amount of money at the ending point of the calculation)% make-rowspan-placeholders
    % rowspan info: col1 '0' | 'false' | '' || col2 '0' | 'false' | ''
     \tabularnewline\cline{1-1}\cline{2-2}
      %--------------------------------------------------------------------
    % align/colidx: left,1
    % rowcount: '0' | start: 'false' | colidx: '1'
        % Formatting a regular cell and recurring on the next sibling
                  \begin{math}i\end{math}
                 &
      % align/colidx: left,2
    % rowcount: '0' | start: 'false' | colidx: '2'
        % Formatting a regular cell and recurring on the next sibling
        interest rate, normally the effective rate per annum% make-rowspan-placeholders
    % rowspan info: col1 '0' | 'false' | '' || col2 '0' | 'false' | ''
     \tabularnewline\cline{1-1}\cline{2-2}
      %--------------------------------------------------------------------
    % align/colidx: left,1
    % rowcount: '0' | start: 'false' | colidx: '1'
        % Formatting a regular cell and recurring on the next sibling
                  \begin{math}n\end{math}
                 &
      % align/colidx: left,2
    % rowcount: '0' | start: 'false' | colidx: '2'
        % Formatting a regular cell and recurring on the next sibling
        period for which the investment is made% make-rowspan-placeholders
    % rowspan info: col1 '0' | 'false' | '' || col2 '0' | 'false' | ''
     \tabularnewline\cline{1-1}\cline{2-2}
      %--------------------------------------------------------------------
    \end{tabular}
      \end{center}
    \begin{center}{\small\bfseries Table 3.4}\end{center}
    %\end{table}
          }{ % else
    % \begin{table}[H]
    % \\ 'id2866211' '1'
        \begin{center}
      \label{m39335*id75382}
    \noindent
    \tabletail{%
        \hline
        \multicolumn{2}{|p{\mytableboxwidth}|}{\raggedleft \small \sl continued on next page}\\
        \hline
      }
      \tablelasttail{}
      \begin{xtabular}[t]{|l|l|}\hline
    % count in rowspan-info-nodeset: 2
    % align/colidx: left,1
    % rowcount: '0' | start: 'false' | colidx: '1'
        % Formatting a regular cell and recurring on the next sibling
                  \begin{math}P\end{math}
                 &
      % align/colidx: left,2
    % rowcount: '0' | start: 'false' | colidx: '2'
        % Formatting a regular cell and recurring on the next sibling
        Principal (the amount of money at the starting point of the calculation)% make-rowspan-placeholders
    % rowspan info: col1 '0' | 'false' | '' || col2 '0' | 'false' | ''
     \tabularnewline\cline{1-1}\cline{2-2}
      %--------------------------------------------------------------------
    % align/colidx: left,1
    % rowcount: '0' | start: 'false' | colidx: '1'
        % Formatting a regular cell and recurring on the next sibling
                  \begin{math}A\end{math}
                 &
      % align/colidx: left,2
    % rowcount: '0' | start: 'false' | colidx: '2'
        % Formatting a regular cell and recurring on the next sibling
        Closing balance (the amount of money at the ending point of the calculation)% make-rowspan-placeholders
    % rowspan info: col1 '0' | 'false' | '' || col2 '0' | 'false' | ''
     \tabularnewline\cline{1-1}\cline{2-2}
      %--------------------------------------------------------------------
    % align/colidx: left,1
    % rowcount: '0' | start: 'false' | colidx: '1'
        % Formatting a regular cell and recurring on the next sibling
                  \begin{math}i\end{math}
                 &
      % align/colidx: left,2
    % rowcount: '0' | start: 'false' | colidx: '2'
        % Formatting a regular cell and recurring on the next sibling
        interest rate, normally the effective rate per annum% make-rowspan-placeholders
    % rowspan info: col1 '0' | 'false' | '' || col2 '0' | 'false' | ''
     \tabularnewline\cline{1-1}\cline{2-2}
      %--------------------------------------------------------------------
    % align/colidx: left,1
    % rowcount: '0' | start: 'false' | colidx: '1'
        % Formatting a regular cell and recurring on the next sibling
                  \begin{math}n\end{math}
                 &
      % align/colidx: left,2
    % rowcount: '0' | start: 'false' | colidx: '2'
        % Formatting a regular cell and recurring on the next sibling
        period for which the investment is made% make-rowspan-placeholders
    % rowspan info: col1 '0' | 'false' | '' || col2 '0' | 'false' | ''
     \tabularnewline\cline{1-1}\cline{2-2}
      %--------------------------------------------------------------------
    \end{xtabular}
      \end{center}
    \begin{center}{\small\bfseries Table 3.4}\end{center}
    %\end{table}
          } % 
        }{% else
        % typeset the table in paragraph mode
        % ----- Begin capturing height of table
        \settoheight{\mytableboxheight}{\begin{tabular*}{\mytablewidth}[t]{|p{10\mystarwidth}|p{10\mystarwidth}|}\hline
    % count in rowspan-info-nodeset: 2
    % align/colidx: left,1
    % rowcount: '0' | start: 'false' | colidx: '1'
        % Formatting a regular cell and recurring on the next sibling
                  \begin{math}P\end{math}
                 &
      % align/colidx: left,2
    % rowcount: '0' | start: 'false' | colidx: '2'
        % Formatting a regular cell and recurring on the next sibling
        Principal (the amount of money at the starting point of the calculation)% make-rowspan-placeholders
    % rowspan info: col1 '0' | 'false' | '' || col2 '0' | 'false' | ''
     \tabularnewline\cline{1-1}\cline{2-2}
      %--------------------------------------------------------------------
    % align/colidx: left,1
    % rowcount: '0' | start: 'false' | colidx: '1'
        % Formatting a regular cell and recurring on the next sibling
                  \begin{math}A\end{math}
                 &
      % align/colidx: left,2
    % rowcount: '0' | start: 'false' | colidx: '2'
        % Formatting a regular cell and recurring on the next sibling
        Closing balance (the amount of money at the ending point of the calculation)% make-rowspan-placeholders
    % rowspan info: col1 '0' | 'false' | '' || col2 '0' | 'false' | ''
     \tabularnewline\cline{1-1}\cline{2-2}
      %--------------------------------------------------------------------
    % align/colidx: left,1
    % rowcount: '0' | start: 'false' | colidx: '1'
        % Formatting a regular cell and recurring on the next sibling
                  \begin{math}i\end{math}
                 &
      % align/colidx: left,2
    % rowcount: '0' | start: 'false' | colidx: '2'
        % Formatting a regular cell and recurring on the next sibling
        interest rate, normally the effective rate per annum% make-rowspan-placeholders
    % rowspan info: col1 '0' | 'false' | '' || col2 '0' | 'false' | ''
     \tabularnewline\cline{1-1}\cline{2-2}
      %--------------------------------------------------------------------
    % align/colidx: left,1
    % rowcount: '0' | start: 'false' | colidx: '1'
        % Formatting a regular cell and recurring on the next sibling
                  \begin{math}n\end{math}
                 &
      % align/colidx: left,2
    % rowcount: '0' | start: 'false' | colidx: '2'
        % Formatting a regular cell and recurring on the next sibling
        period for which the investment is made% make-rowspan-placeholders
    % rowspan info: col1 '0' | 'false' | '' || col2 '0' | 'false' | ''
     \tabularnewline\cline{1-1}\cline{2-2}
      %--------------------------------------------------------------------
    \end{tabular*}} % end mytableboxheight set
        \settodepth{\mytableboxdepth}{\begin{tabular*}{\mytablewidth}[t]{|p{10\mystarwidth}|p{10\mystarwidth}|}\hline
    % count in rowspan-info-nodeset: 2
    % align/colidx: left,1
    % rowcount: '0' | start: 'false' | colidx: '1'
        % Formatting a regular cell and recurring on the next sibling
                  \begin{math}P\end{math}
                 &
      % align/colidx: left,2
    % rowcount: '0' | start: 'false' | colidx: '2'
        % Formatting a regular cell and recurring on the next sibling
        Principal (the amount of money at the starting point of the calculation)% make-rowspan-placeholders
    % rowspan info: col1 '0' | 'false' | '' || col2 '0' | 'false' | ''
     \tabularnewline\cline{1-1}\cline{2-2}
      %--------------------------------------------------------------------
    % align/colidx: left,1
    % rowcount: '0' | start: 'false' | colidx: '1'
        % Formatting a regular cell and recurring on the next sibling
                  \begin{math}A\end{math}
                 &
      % align/colidx: left,2
    % rowcount: '0' | start: 'false' | colidx: '2'
        % Formatting a regular cell and recurring on the next sibling
        Closing balance (the amount of money at the ending point of the calculation)% make-rowspan-placeholders
    % rowspan info: col1 '0' | 'false' | '' || col2 '0' | 'false' | ''
     \tabularnewline\cline{1-1}\cline{2-2}
      %--------------------------------------------------------------------
    % align/colidx: left,1
    % rowcount: '0' | start: 'false' | colidx: '1'
        % Formatting a regular cell and recurring on the next sibling
                  \begin{math}i\end{math}
                 &
      % align/colidx: left,2
    % rowcount: '0' | start: 'false' | colidx: '2'
        % Formatting a regular cell and recurring on the next sibling
        interest rate, normally the effective rate per annum% make-rowspan-placeholders
    % rowspan info: col1 '0' | 'false' | '' || col2 '0' | 'false' | ''
     \tabularnewline\cline{1-1}\cline{2-2}
      %--------------------------------------------------------------------
    % align/colidx: left,1
    % rowcount: '0' | start: 'false' | colidx: '1'
        % Formatting a regular cell and recurring on the next sibling
                  \begin{math}n\end{math}
                 &
      % align/colidx: left,2
    % rowcount: '0' | start: 'false' | colidx: '2'
        % Formatting a regular cell and recurring on the next sibling
        period for which the investment is made% make-rowspan-placeholders
    % rowspan info: col1 '0' | 'false' | '' || col2 '0' | 'false' | ''
     \tabularnewline\cline{1-1}\cline{2-2}
      %--------------------------------------------------------------------
    \end{tabular*}} % end mytableboxdepth set
        \addtolength{\mytableboxheight}{\mytableboxdepth}
        % ----- End capturing height of table
        \typeout{textheight: \the\textheight}
        \typeout{mytableboxheight: \the\mytableboxheight}
        \typeout{table too wide, outputting in para mode}
    % \begin{table}[H]
    % \\ 'id2866211' '1'
        \begin{center}
      \label{m39335*id75382}
    \noindent
    \tabletail{%
        \hline
        \multicolumn{2}{|p{\mytableroom}|}{\raggedleft \small \sl continued on next page}\\
        \hline
      }
      \tablelasttail{}
      \begin{xtabular*}{\mytablewidth}[t]{|p{10\mystarwidth}|p{10\mystarwidth}|}\hline
    % count in rowspan-info-nodeset: 2
    % align/colidx: left,1
    % rowcount: '0' | start: 'false' | colidx: '1'
        % Formatting a regular cell and recurring on the next sibling
                  \begin{math}P\end{math}
                 &
      % align/colidx: left,2
    % rowcount: '0' | start: 'false' | colidx: '2'
        % Formatting a regular cell and recurring on the next sibling
        Principal (the amount of money at the starting point of the calculation)% make-rowspan-placeholders
    % rowspan info: col1 '0' | 'false' | '' || col2 '0' | 'false' | ''
     \tabularnewline\cline{1-1}\cline{2-2}
      %--------------------------------------------------------------------
    % align/colidx: left,1
    % rowcount: '0' | start: 'false' | colidx: '1'
        % Formatting a regular cell and recurring on the next sibling
                  \begin{math}A\end{math}
                 &
      % align/colidx: left,2
    % rowcount: '0' | start: 'false' | colidx: '2'
        % Formatting a regular cell and recurring on the next sibling
        Closing balance (the amount of money at the ending point of the calculation)% make-rowspan-placeholders
    % rowspan info: col1 '0' | 'false' | '' || col2 '0' | 'false' | ''
     \tabularnewline\cline{1-1}\cline{2-2}
      %--------------------------------------------------------------------
    % align/colidx: left,1
    % rowcount: '0' | start: 'false' | colidx: '1'
        % Formatting a regular cell and recurring on the next sibling
                  \begin{math}i\end{math}
                 &
      % align/colidx: left,2
    % rowcount: '0' | start: 'false' | colidx: '2'
        % Formatting a regular cell and recurring on the next sibling
        interest rate, normally the effective rate per annum% make-rowspan-placeholders
    % rowspan info: col1 '0' | 'false' | '' || col2 '0' | 'false' | ''
     \tabularnewline\cline{1-1}\cline{2-2}
      %--------------------------------------------------------------------
    % align/colidx: left,1
    % rowcount: '0' | start: 'false' | colidx: '1'
        % Formatting a regular cell and recurring on the next sibling
                  \begin{math}n\end{math}
                 &
      % align/colidx: left,2
    % rowcount: '0' | start: 'false' | colidx: '2'
        % Formatting a regular cell and recurring on the next sibling
        period for which the investment is made% make-rowspan-placeholders
    % rowspan info: col1 '0' | 'false' | '' || col2 '0' | 'false' | ''
     \tabularnewline\cline{1-1}\cline{2-2}
      %--------------------------------------------------------------------
    \end{xtabular*}
      \end{center}
    \begin{center}{\small\bfseries Table 3.4}\end{center}
    %\end{table}
        }% ending lr/para test clause
    \par
\item For simple interest we use:
        \label{m39335*id75453}\nopagebreak\noindent{}\settowidth{\mymathboxwidth}{\begin{equation}
    \mathrm{A}=P\left(1+i\ensuremath{\cdot}n\right)\tag{3.45}
      \end{equation}
    }
    \typeout{Columnwidth = \the\columnwidth}\typeout{math as usual width = \the\mymathboxwidth}
    \ifthenelse{\lengthtest{\mymathboxwidth < \columnwidth}}{% if the math fits, do it again, for real
    \begin{equation}
    \mathrm{A}=P\left(1+i\ensuremath{\cdot}n\right)\tag{3.45}
      \end{equation}
    }{% else, if it doesn't fit
    \setlength{\mymathboxwidth}{\columnwidth}
      \addtolength{\mymathboxwidth}{-48pt}
    \par\vspace{12pt}\noindent\begin{minipage}{\columnwidth}
    \parbox[t]{\mymathboxwidth}{\large\begin{math}
    \mathrm{A}=P\left(1+i\ensuremath{\cdot}n\right)\end{math}}\hfill
    \parbox[t]{48pt}{\raggedleft 
    (3.45)}
    \end{minipage}\vspace{12pt}\par
    }% end of conditional for this bit of math
    \typeout{math as usual width = \the\mymathboxwidth}
        \item  For compound interest we use:
        \label{m39335*id75538}\nopagebreak\noindent{}\settowidth{\mymathboxwidth}{\begin{equation}
    \mathrm{A}=P{\left(1+i\right)}^{n}\tag{3.46}
      \end{equation}
    }
    \typeout{Columnwidth = \the\columnwidth}\typeout{math as usual width = \the\mymathboxwidth}
    \ifthenelse{\lengthtest{\mymathboxwidth < \columnwidth}}{% if the math fits, do it again, for real
    \begin{equation}
    \mathrm{A}=P{\left(1+i\right)}^{n}\tag{3.46}
      \end{equation}
    }{% else, if it doesn't fit
    \setlength{\mymathboxwidth}{\columnwidth}
      \addtolength{\mymathboxwidth}{-48pt}
    \par\vspace{12pt}\noindent\begin{minipage}{\columnwidth}
    \parbox[t]{\mymathboxwidth}{\large\begin{math}
    \mathrm{A}=P{\left(1+i\right)}^{n}\end{math}}\hfill
    \parbox[t]{48pt}{\raggedleft 
    (3.46)}
    \end{minipage}\vspace{12pt}\par
    }% end of conditional for this bit of math
    \typeout{math as usual width = \the\mymathboxwidth}
        \item The formulae for simple and compound interest can be applied to many everyday problems.\item A foreign exchange rate is the price of one currency in terms of another.\end{itemize}
\label{m39335*notfhsst!!!underscore!!!id3321}
\begin{tabular}{cc}
	   \hspace*{-50pt}\raisebox{-8 mm}{ \includegraphics[width=0.5in]{col11306.imgs/pstip2.png}  }& 
	\begin{minipage}{0.85\textwidth}
	\begin{note}
      {tip: }Always keep the interest and the time period in the same units of time (e.g. both in years, or both in months etc.).
	\end{note}
	\end{minipage}
	\end{tabular}
	\par
     \label{m39335*eip-595}The following three videos provide a summary of how to calculate interest. Take note that although the examples are done using dollars, we can use the fact that dollars are a decimal currency and so are interchangeable (ignoring the exchange rate) with rands. This is what is done in the subtitles.\par \label{m39335*eip-209}
    \setcounter{subfigure}{0}
	\begin{figure}[H] % horizontal\label{m39335*circuits-1}
    \textnormal{Khan academy video on interest - 1}\vspace{.1in} \nopagebreak
  \label{m39335*yt-media1}\label{m39335*yt-video1}
            \raisebox{-5 pt}{ \includegraphics[width=0.5cm]{col11306.imgs/summary_www.png}} { (Video:  MG10030 )}
      \vspace{2pt}
    \vspace{.1in}
 \end{figure}       \par 
    \label{m39335*eip-20912}
    \setcounter{subfigure}{0}
	\begin{figure}[H] % horizontal\label{m39335*interest-2}
    \textnormal{Khan academy video on interest - 2}\vspace{.1in} \nopagebreak
  \label{m39335*yt-media2}\label{m39335*yt-video2}
            \raisebox{-5 pt}{ \includegraphics[width=0.5cm]{col11306.imgs/summary_www.png}} { (Video:  MG10031 )}
      \vspace{2pt}
    \vspace{.1in}
 \end{figure}       \par 
\label{m39335*eip-20913}Note in this video that at the very end the rule of 72 is mentioned. You will not be using this rule, but will rather be using trial and error to solve the problem posed. 
    \setcounter{subfigure}{0}
	\begin{figure}[H] % horizontal\label{m39335*interest-3}
    \textnormal{Khan academy video on interest - 3}\vspace{.1in} \nopagebreak
  \label{m39335*yt-media3}\label{m39335*yt-video3}
            \raisebox{-5 pt}{ \includegraphics[width=0.5cm]{col11306.imgs/summary_www.png}} { (Video:  MG10032 )}
      \vspace{2pt}
    \vspace{.1in}
 \end{figure}       \par 
    \label{m39335*cid8}
            \section{ End of Chapter Exercises}
            \nopagebreak
            \label{m39335*id75641}\begin{enumerate}[noitemsep, label=\textbf{\arabic*}. ] 
            \label{m39335*uid79}\item You are going on holiday to Europe. Your hotel will cost 200 euros per night. How much will you need in Rands to cover your hotel bill, if the exchange rate is 1 euro = R9,20?\newline
\label{m39335*uid80}\item Calculate how much you will earn if you invested R500 for 1 year at the following interest rates:
\label{m39335*id75671}\begin{enumerate}[noitemsep, label=\textbf{\alph*}. ] 
            \label{m39335*uid81}\item 6,85\% simple interest.
\label{m39335*uid82}\item 4,00\% compound interest.
\end{enumerate}
\label{m39335*uid83}\item Bianca has R1 450 to invest for 3 years. Bank A offers a savings account which pays simple interest at a rate of 11\% per annum, whereas Bank B offers a savings account paying compound interest at a rate of 10,5\% per annum. Which account would leave Bianca with the highest accumulated balance at the end of the 3 year period?\newline
\label{m39335*uid84}\item How much simple interest is payable on a loan of R2~000 for a year, if the interest rate is 10\%?\newline
\label{m39335*uid85}\item How much compound interest is payable on a loan of R2~000 for a year, if the interest rate is 10\%?\newline
\label{m39335*uid86}\item Discuss:
\label{m39335*id75754}\begin{enumerate}[noitemsep, label=\textbf{\alph*}. ] 
            \label{m39335*uid87}\item Which type of interest would you like to use if you are the borrower?
\label{m39335*uid88}\item Which type of interest would you like to use if you were the banker?
\end{enumerate}
\label{m39335*uid89}\item Calculate the compound interest for the following problems.
\label{m39335*id75796}\begin{enumerate}[noitemsep, label=\textbf{\alph*}. ] 
            \label{m39335*uid90}\item A R2~000 loan for 2 years at 5\%.
\label{m39335*uid91}\item A R1~500 investment for 3 years at 6\%.
\label{m39335*uid92}\item An R800 loan for l year at 16\%.
\end{enumerate}
\label{m39335*uid93}\item If the exchange rate for 100 Yen = R 6,2287 and 1 Australian Doller (AUD) = R 5,1094 , determine the exchange rate between the Australian Dollar and the Japanese Yen.\newline
\label{m39335*uid94}\item Bonnie bought a stove for R 3~750. After 3 years she had finished paying for it and the R 956,25 interest that was charged for hire-purchase. Determine the rate of simple interest that was charged.\newline
\end{enumerate}
  \label{m39335**end}
  \label{5925cb6120ab0c0f7c78bd2516b027ff**end}
\par \raisebox{-5 pt}{\includegraphics[width=0.5cm]{col11306.imgs/summary_www.png}} Find the answers with the shortcodes:
 \par \begin{tabular}[h]{cccccc}
 (1.) lcK  &  (2.) lck  &  (3.) lc0  &  (4.) lc8  &  (5.) lc9  &  (6.) lcX  &  (7.) lcI  &  (8.) lc5  &  (9.) lcN  & \end{tabular}
