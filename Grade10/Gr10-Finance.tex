\chapter{Finance and growth}

\section{Being interested in interest}

% Welcome to the Grade 10 Finance Chapter, where we apply mathematics skills to everyday financial situations.\par

If you had R~$1 000$, you could either keep it in your piggy bank, or deposit it into a bank account. If you deposit the
money into a bank account, you are effectively lending money to the bank and as a result, you can expect to receive
interest in return. Similarly, if you borrow money from a bank, then you can expect to pay interest on the loan.
Interest is charged at a percentage of the money owed over the period of time it takes to pay back the loan, meaning
the longer the loan exists, the more interest will have to be paid on it.\par

The concept is simple, yet it is core to the world of finance. Accountants, actuaries and bankers can spend their
entire working career dealing with the effects of interest on financial matters.\par

\section{Simple interest}
\Definition{Simple interest}{Simple interest is where you earn interest on the initial amount that you invested, but not interest on interest.}
  
As an easy example of simple interest, consider how much you will get by investing R~$1~000$ for 1 year with a bank that pays you $5\%$ simple interest. At the end of the year, you will get an interest of:\par
\begin{align*}
    \mbox{Interest} &= \text{R}~1~000 \times 5\%\\
    &= \text{R}~1~000 \times \frac{5}{100}\\
    &= \text{R}~1~000 \times 0,05\\
    &= \text{R}~50
\end{align*}

So, with an ``opening balance" of R~$1~000$ at the start of the year, your ``closing balance" at the end of the year will therefore be:\par 
\begin{align*}
    \mbox{Closing balance} &= \mbox{Opening balance + Interest}\\
    &= \text{R}~1~000 + \text{R}~50\\
    &= \text{R}~1~050
\end{align*}

We sometimes call the opening balance in financial calculations the \textsl{Principal}, which is abbreviated as $P$ (R~$1~000$ in the example). The interest rate is usually labelled $i$ ($5\%$ in the example), and the interest amount (in Rand terms) is labelled $I$ (R~$50$ in the example).\par 

So we can see that:
        
\begin{align*}
    I = P \times i
\end{align*}

and,

\begin{align*}
    \mbox{Closing balance} &= \mbox{Opening balance + Interest} \nonumber\\
    &= P + I \nonumber\\
    &= P + (P \times i)\nonumber\\
    &= P(1 + i)
\end{align*}

\Tip{Remember that percentage is a number written over the denominator of $100$. In other words, $12\%$ can be written as $\frac{12}{100}$ --- and thus $12\%$ can be changed to the decimal $0,12$.}


The above calculations give us a good idea of what the simple interest formula looks like, however, the example talks
about an investment that only lasts one year. If the investment or loan is over a longer period, we need to take this
into account. We use the symbol ‘$n$’ to indicate time period and this value must always be written in years.\par

The formula used to calculate simple interest is:
\Tip{Annual Rates means yearly rates, and p.a.(per annum) = per year.}

\Identity{Simple Interest}{
    \begin{eqnarray*}
	\label{FG:SI}
	A &=& P (1 + i . n)\\
	\text{Where:~} \nonumber\\
	A &=& \text{accumulated amount} \nonumber\\
	P &=& \text{principal amount} \nonumber\\
	i &=& \text{interest written as decimal} \nonumber\\
	n &=& \text{number of years} \nonumber
    \end{eqnarray*}
}




\begin{wex}{Calculating interest on a deposit}{
    If I deposit R~$1 000$ into a special bank account which pays a simple interest rate of $7\%$ p.a. for 3 years, how much will be in the account at the end of the term?}{

    \westep{Write down your known values}
    \begin{align*}
	A &= ~?\\
	P &= 1~000\\
	i &= 0,07\\
	n &= 3
    \end{align*}
    
    \westep{Write down the formula}
    \begin{align*}
	A &= P(1 + i . n)
    \end{align*}

    \westep{Substitute the values and solve for the remaining variable}
    \begin{align*}
	A &= 1~000(1 + 0,07 \times 3)\\
	  &= 1~210
    \end{align*}

    \westep{Explain your answer}
    At the end of 3 years, I will have R~$1~210$ in the bank account.
    }
\end{wex}


\begin{wex}{Calculating interest on a loan}{
    Sarah borrows R~$5~000$ from her neighbour at an agreed simple interest rate of $12,5\%$ p.a. She will pay back the loan in one lump sum at the end of 2 years. How much will she have to pay her neighbour?}{

    \westep{Write down the known variables}
    \begin{align*}
	A &= ~?\\
	P &= 5~000\\
	i &= 0,125\\
	n &= 2
    \end{align*}

    \westep{Write down the formula}
    \begin{align*}
	A &= P(1 + i . n)
    \end{align*}

    \westep{Substitute the values and solve for the remaining variable}
    \begin{align*}
	A &= 5~000(1 + 0,125 \times 2)\\
	  &= 6~250
    \end{align*}

    \westep{Explain your answer}
    At the end of 2 years, Sarah will pay her neighbour R~$6~250$.
    }
\end{wex}


We can use the simple interest formula to discover pieces of missing information. For instance, if we have an amount of money that we want to invest for a set amount of time to achieve a certain outcome, we can rearrange the variables to solve for the interest rate we need to invest the money at. The same principles apply to find the length of time we would need to invest the money for, if we knew the principal and accumulated amount and the interest rate.


\begin{wex}{Determining the investment period to achieve a goal amount}{
    If I deposit R~$30~000$ into a bank account which pays a simple interest rate of $7,5\%$ p.a., for how many years must I invest to generate R~$45~000$?}{

    \westep{Write down the known variables}
    \begin{align*}
	A &= 45~000\\
	P &= 30~000\\
	i &= 0,075\\
	n &= ~?
    \end{align*}

    \westep{Write down the formula}
    \begin{align*}
	A &= P(1 + i . n)
    \end{align*}

    \westep{Substitute the values and solve for the remaining variable}
    \begin{align*}
	45~000 &= 30~000(1 + 0,075 \times n)\\
	\frac{45~000}{30~000} &= 1 + 0,075 \times n \\[5pt]
	\frac{45~000}{30~000} -1 &= 0,075 \times n\\[5pt]
	\frac{(\frac{45~000}{30~000}) -1}{0,075} &= n\\
	n &= 6,6
    \end{align*}

    \westep{Explain your answer}
    It will take 6 years 8 months to make R~$45~000$ from R~$30~000$ at a simple interest rate of $7,5\%$.
    }
\end{wex}


\Tip{To get a more accurate answer, try to do all your calculations on the calculator in one go. This will prevent rounding off errors from influencing your final answer.}


\begin{wex}{Calculating the interest rate to achieve the desired growth}{
    At what simple interest rate should I invest if I want to grow R~$2~500$ to R~$4~000$ in 5 years?}{

    \westep{Write down the known variables}
    \begin{align*}
	A &= 4~000\\
	P &= 2~500\\
	i &= ~?\\
	n &= 5
    \end{align*}

    \westep{Write down the formula}
    \begin{align*}
	A &= P(1 + i . n)
    \end{align*}

    \westep{Substitute the values and solve for the remaining variable}
    \begin{align*}
	4~000 &= 2~500(1 + i \times 5)\\
	\frac{4~000}{2~500} &= 1 + i \times 5\\[5pt]
	\frac{4~000}{2~500} - 1&= i \times 5\\[5pt]
	\frac{(\frac{4~000}{2~500}) - 1}{5} &= i\\
	i &= 0,12
    \end{align*}

    \westep{Explain your answer}
    A simple interest rate of $12\%$ will be needed when investing R~$2~500$ for 5 years to become R~$4~000$.
    }
\end{wex}


\subsection{When the time period is not in years}

Often someone is not able to invest money for a full year, or they want to take out a loan for a shorter period. How do we calculate the accumulated amount if the time period is not in years?\par

You need to remember that in every year there are 12 months. If I wanted to invest money for just three months, $n$ would be equal to $\frac{3}{12}$. Similarly, if I wanted to take out a loan for a period of 7 months, $n = \frac{7}{12}$.\\


\begin{exercises}{}{
    \begin{enumerate}[itemsep=6pt, label=\textbf{\arabic*}.]
	\item An amount of R~$3~500$ is invested in a savings account which pays simple interest at a rate of $7,5\%$ per annum. Calculate the balance accumulated by the end of 2 years.

	\item Calculate the accumulated amount in the following situations:
	\begin{enumerate}[noitemsep, label=\textbf{(\alph*)} ]
	    \item A loan of R~$300$ at a rate of $8\%$ for 1 year.

	    \item An investment of R~$2~250$ at a rate of $12,5\%$p.a. for 6 years.
	\end{enumerate}

	\item Sally wanted to calculate the number of years she needed to invest R~$1~000$ for in order to accumulate R~$2~500$. She has been offered a simple interest rate of $8,2\%$ p.a. How many years will it take for the money to grow to R~$2~500$?

	\item I made a deposit of R~$5~000$ in the bank for my 5 year old son’s 21st birthday. I have given him the amount of R~$18~000$ on his birthday. At what rate was the money invested, if simple interest was calculated?\\
    \end{enumerate}

    Find the answers with the short codes:\\
    \begin{tabularx}{\textwidth}{ XXXX }
	(1)	&	(2)	&	(3)	&	(4)\\
    \end{tabularx}
}
\end{exercises}


\subsection{Hire purchase}

As a general rule, it’s a bad idea to buy things on credit. Buying things on credit means that you have had to borrow money to pay for the object, meaning you will have to pay more for it due to the interest on the loan. That being said, occasionally there are appliances, such as a fridge, that are very difficult to live without. Most people don’t have the cash up front to purchase such items, so they buy it on a hire purchase agreement.\par

A hire purchase agreement is a financial agreement between the shop and the customer on how the customer will pay for the desired product. The interest on a hire purchase loan is always charged at a simple interest rate and only charged on the amount owing. Most agreements require that a deposit is paid before you can take home the product. The principal amount of the loan is therefore the cash price minus the deposit. The accumulated loan will be worked out using the number of years the loan is needed for. The total loan amount is then divided into monthly payments over the period of the loan.


\begin{wex}{Hire purchase}{
    Troy is keen to buy an additional hard drive for his laptop advertised for R~$2~500$ on the internet. There is an option of paying a $10\%$ deposit then making 24 monthly payments using a hire purchase agreement where interest is calculated at $7,5\%$ p.a. simple interest. Calculate what Troy’s monthly payments will be.}{

    \westep{A new opening balance is required, as the $10\%$ deposit is paid in cash.}
    \begin{itemize}
	\item $10\%$ of R~$2~500$ $=$ R~$250$\\
	\item new opening balance, $P$ $=$ R~$2~500$ $−$ R~$250$ $=$ R~$2~250$\\
	\item interest rate, $i = 7,5\%$\\
	\item period of time, $n = 2$ years
    \end{itemize}

    We are required to find the closing balance ($A$) and then the monthly payments.

    \westep{We know from \ref{FG:SI} that:}
    \begin{align*}
	    A &= P(1 + i . n)
    \end{align*}

    \westep{Substitute the values and solve for the remaining variable}
    \begin{align*}
	A &= 2~250(1 + 0,075 \times 2)\\
	  &= 2~587,50\\
    \end{align*}

    \westep{Calculate the monthly repayments on the hire purchase agreement}
    \begin{align*}
	\text{Monthly payment} &= \frac{2~587,50}{24}\\
			&= \text{R}~107,81
    \end{align*}

    \westep{Explain your answer}
    Troy's monthly payments $=$ R~$107,81$.
}
\end{wex}


Occasionally the shop will add a monthly insurance premium to the monthly instalments. This insurance premium will be an amount of money paid monthly and will buy you more time between a missed payment and possible repossession of the product.


\begin{wex}{Hire purchase with extra conditions}{
    Cassidy desperately wants to buy a tv and as a result, decides to buy one on a hire purchase agreement. The chosen TV’s cash price is R~$5~500$. She will pay it off over 54 months at an interest rate of $21\%$p.a. An insurance premium of R~$12,50$ is added to every monthly payment. How much are her monthly payments?}{

    \westep{Write down the known variables}
    \begin{eqnarray*}
	A &=& ?\\
	P &=& 5~500\\
	i &=& 0,21\\
	n &=& \frac{54}{12} = 4,5
    \end{eqnarray*}
(The question says nothing about a deposit, therefore we assume that Cassidy did not pay one)

    \westep{Write down the formula}
    \begin{align*}
	    A &= P(1 + i . n)
    \end{align*}

    \westep{Substitute the values and solve for the remaining variable}
    \begin{align*}
	A &= 5~500(1 + 0,21 \times 4,5)\\
	  &= 10~697,50\\
    \end{align*}

    \westep{Calculate the monthly repayments on the hire purchase agreement}
    \begin{align*}
	\text{Monthly payment} &= \frac{10~697,50}{54}\\
			&= 198,10
    \end{align*}

    \westep{Add the insurance premium}
    \begin{align*}
	198,10 + 12,50 &= 210,60
    \end{align*}

    \westep{Explain your answer}
    Cassidy will pay R~$210,60$ monthly for 54 months until her TV is paid off.
}
\end{wex}


\Tip{You are expected to know that hire purchase is charged at a simple interest rate. When you are asked a hire purchase question in a test, don’t forget to always use the simple interest formula.}


\begin{exercises}{}{
    \begin{enumerate}[label=\textbf{\arabic*}.]
	\item Vanessa wants to buy a fridge on a hire purchase agreement. The cash price of the fridge is R~$4~500$. She is required to put down a deposit of $15\%$ and pay the remaining loan amount off over 24 months at an interest rate of $12\%$ p.a.
	\begin{enumerate}[noitemsep, label=\textbf{(\alph*)} ]
	    \item What is the principal loan amount?
	    \item What is the accumulated loan amount?
	    \item What are Vanessa’s monthly repayments?
	    \item What is the total amount she has paid for the fridge?
	\end{enumerate}


	\item Bongani buys a dining room table costing R~$8~500$ on a hire purchase agreement. He is charged an interest rate of $17,5\%$p.a. over 3 years.
	\begin{enumerate}[noitemsep, label=\textbf{(\alph*)} ]
	    \item How much will Bongani pay in total?
	    \item How much interest does he pay?
	    \item What is his monthly instalment?
	\end{enumerate}

	\item A lounge suite is advertised on tv, to be paid off over 36 months at R~$150$ a month.
	\begin{enumerate}[noitemsep, label=\textbf{(\alph*)} ]
	    \item Assuming that no deposit is needed, how much will the buyer pay for the lounge suite once it has been paid off?
	    \item If the interest rate is $9\%$ p.a, what is the cash price of the suite?\\
	\end{enumerate}
    \end{enumerate}

    Find the answers with the short codes:\\
    \begin{tabularx}{\textwidth}{ XXX }
	(1)	&	(2)	&	(3)\\
    \end{tabularx}
}
\end{exercises}


\section{Compound interest}

Compound interest allows us to earn interest on our interest. In simple interest, you only every earn interest on your original investment, but in compound interest, you are able to earn interest both on your original investment and the interest earned on it.\par

Compound interest is a great concept if you are investing, however, when taking out a loan, you will be paying more if it is calculated on a compound interest rate rather than simple.


\Definition{Compound interest}{Compound interest is the interest payable on the principal and its accumulated interest.}


In the same way that we developed a formula for Simple Interest, let us find one for Compound Interest.\par

If our opening balance is $P$ and we have an interest rate of $i$ then, the closing balance at the end of the first year is:
\begin{eqnarray*}
    \text{Closing balance after 1 year} = P(1 + i)
\end{eqnarray*}

This is the same as simple interest because it only covers a single year. Then, if we take that out and re-invest it for another year --- just as you saw us doing in the worked example above --- then the balance after the second year will be:
\begin{eqnarray*}
    \text{Closing balance after 2 years} &=& [P(1 + i)] \times (1 + i)\\
    &=& P(1 + i)^2
\end{eqnarray*}

And if we take that money out, then invest it for another year, the balance becomes:

\begin{eqnarray*}
    \text{Closing balance after 3 years} &=& [P(1 + i)^2] \times (1 + i)\\
    &=& P(1 + i)^3
\end{eqnarray*}

We can see that the power of the term $(1 + i)$ is the same as the number of years. Therefore the formula used to calculate compound interest can be written as:


\Identity{Compound interest}{
    \begin{eqnarray*}
	\label{FG:CI}
	A &=& P(1 + i)^n\\
	\text{Where:~} \nonumber\\
	A &=& \text{accumulated amount} \nonumber\\
	P &=& \text{principal amount} \nonumber\\
	i &=& \text{interest written as decimal} \nonumber\\
	n &=& \text{number of years} \nonumber
    \end{eqnarray*}
}


\begin{wex}{Compound interest}{
    Mpho wants to invest R~$30~000$ into an account that offers a compound interest rate of $6\%$p.a. How much money will be in the account at the end of 4 years?}{
    
    \westep{Write down the known variables}
    \begin{eqnarray*}
	A &=& ?\\
	P &=& 30~000\\
	i &=& 0,06\\
	n &=& 4
    \end{eqnarray*}

    \westep{Write down the formula}
    \begin{align*}
	A &= P(1 + i)^n
    \end{align*}

    \westep{Substitute the values and solve for the remaining variable}
    \begin{align*}
	A &= 30~000(1 + 0,06)^4\\
	  &= 37~874,31
    \end{align*}

    \westep{Explain your answer}
    Mpho will have R~$37~874,31$ in the account at the end of 4 years.
    }
\end{wex}


\begin{wex}{Calculating the compound interest rate to achieve the desired growth}{
    Charlie has been given R~$5~000$ for his sixteenth birthday. Rather than spending it, he has decided to invest it so that on his eighteenth birthday, he can put a deposit of R~$10~000$ on a car. What compound interest rate should he be looking for?}{
    
    \westep{Write down the known variables}
    \begin{eqnarray*}
	A &=& 10~000\\
	P &=& 5~000\\
	i &=& ?\\
	n &=& 2
    \end{eqnarray*}

    \westep{Write down the formula}
    \begin{align*}
	A &= P(1 + i)^n
    \end{align*}

    \westep{Substitute the values and solve for the remaining variable}
    \begin{align*}
	10~000 &= 5~000(1 + i)^2\\
	\frac{10~000}{5~000}&= (1 +i)^2\\[5pt]
	\sqrt[]{\frac{10~000}{5~000}} &= 1 + i\\[5pt]
	\sqrt[]{\frac{10~000}{5~000}} - 1 &= i\\
	i &= 0,4142135624
    \end{align*}

    \westep{Explain your answer}
    Charlie needs to find an account that offers a compound interest rate of $41,42\%$.
    }
\end{wex}


\subsection{The power of compound interest}

To see how important this “interest on interest" is, we shall compare the difference in closing balances for money earning simple interest and money earning compound interest. Consider an amount of R~$10~000$ that you have to invest for 10 years, and assume we can earn interest of $9\%$. How much would that be worth after 10 years?\par

The closing balance for the money earning simple interest is:
\begin{align*}
    A &= P(1 + i . n)\\
      &= 10~000(1 + 0,09 \times 10)\\
      &= \mbox{R}~19~000\\
\end{align*}

The closing balance for the money earning compound interest is:
\begin{align*}
    A &= P(1 + i)^n\\
      &= 10~000(1 + 0,09)^10\\
      &= \mbox{R}~23~673,64
\end{align*}

If we graph the growth of the money, it’s easier to see how it increases in just a straight line in simple interest but exponentially in compound interest:
%There's a weird anomaly on all these graphs near the origin - ''23truetrue,``. wtf?!
\begin{figure}[H]
    \begin{center}
      \scalebox{0.8}{
	\begin{pspicture}(-5,-2)(7,8)
	    \psset{yunit=0.75,xunit=1}
	    \psgrid[subgriddiv=1,griddots=10,gridlabels=0](0,0)(10.4,10)
	    \psaxes[arrows=-, dx=1, Dx=1, dy=1, Dy=2000](0,0)(0,0)(10.4,10)
% 	    \psline[linecolor=red](0,5)(10,9.5)
	    \psplot[linecolor=red]{0}{10}{  x 0.09 mul 1 add 5 mul}
	\end{pspicture}}\\
	\begin{caption*}The growth of money in a simple interest account over 10 years.\end{caption*}
	\label{FG:fig:SI10}
    \end{center}
\end{figure}

\begin{figure}[H]
    \begin{center}
\scalebox{0.8}{
	\begin{pspicture}(-5,-2)(7,8)
	    \psset{yunit=0.75,xunit=1}
	    \psgrid[subgriddiv=1,griddots=10,gridlabels=0](0,0)(10.4,12.3)
	    \psaxes[arrows=-, dx=1, Dx=1, dy=1, Dy=2000](0,0)(0,0)(10.4,12.3)
	    \psplot[linecolor=red]{0}{10}{1.09 x exp 5 mul}
	\end{pspicture}}\\
	\begin{caption*}The growth of money in a compound interest account over 10 years.\end{caption*}
	\label{FG:fig:CI10}
    \end{center}
\end{figure}

It’s easier to see the vast difference in growth if we extend the time period to 50 years:
\begin{figure}[H]
    \begin{center}
\scalebox{0.8}{
	\begin{pspicture}(-2,-2)(7,8)
	    \psset{yunit=0.75,xunit=0.65}
	    \psgrid[subgriddiv=1,griddots=10,gridlabels=0](0,0)(20,5)
	    \psaxes[arrows=-, dx=2, Dx=5, dy=1, Dy=50000](0,0)(0,0)(20,5)
% 	    \psline[linecolor=red](0,0.2)(20,1.1)
	    \psplot[linecolor=red]{0}{20}{x 0.09 mul 2.5 mul 1 add 0.2 mul} 
	\end{pspicture}}\\
	\begin{caption*}The growth of money in a simple interest account over 50 years.\end{caption*}
	\label{FG:fig:SI10}
    \end{center}
\end{figure}

\begin{figure}[H]
    \begin{center}
\scalebox{0.8}{
	\begin{pspicture}(-2,-2)(7,8)
	    \psset{yunit=0.75,xunit=0.65}
	    \psgrid[subgriddiv=1,griddots=10,gridlabels=0](0,0)(20,15)
	    \psaxes[arrows=-, dx=2, Dx=5, dy=1, Dy=50000](0,0)(0,0)(20,15)
	    \psplot[linecolor=red]{0}{20}{1.09 x 2.5 mul exp 0.2 mul}
	\end{pspicture}}\\
	\begin{caption*}The growth of money in a compound interest account over 50 years.\end{caption*}
	\label{FG:fig:CI10}
    \end{center}
\end{figure}

Again, keep in mind that this is good news and bad news. When you are earning interest on money you have invested, compound interest helps that amount to increase exponentially. But if you have borrowed money, the build up of the amount you owe will grow exponentially too.


\begin{exercises}{}{
    \begin{enumerate}[label=\textbf{\arabic*}.]
	\item An amount of R~$3~500$ is invested in a savings account which pays a compound interest rate of $7,5\%$ p.a. Calculate the balance accumulated by the end of 2 years.

	\item Morgan invests R~$5~000$ into an account which pays out a lump sum at the end of 5 years. If he gets R~$7~500$ at the end of the period, what compound interest rate did the bank offer him?

	\item Shrek wants to invest some money at a compound interest rate of $11\%$ p.a. How much money (to the nearest Rand) should be invested if he wants to reach a sum of R~$100~000$ in five years time?\\
    \end{enumerate}

    Find the answers with the short codes:\\
    \begin{tabularx}{\textwidth}{ XXX }
	(1)	&	(2)	&	(3)\\
    \end{tabularx}
}
\end{exercises}


\subsection{Inflation}

You may have heard many older people recalling times when things were cheaper. You may remember buying things as a child and be shocked at the increase in price. There are many factors that influence the change in price of an item, one of them being inflation.\par

Inflation is the average percentage that all goods increase by every year. As this increase happens every year, it is calculated using the compound interest formula.


\begin{wex}{Calculating future cost based on inflation}
    {Milk costs R~$14$ for two litres. How much will it cost in 4 years time if the inflation rate is $9\%$p.a.?}{
    
    \westep{Write down the known variables}
    \begin{eqnarray*}
	A &=& ?\\
	P &=& 14\\
	i &=& 0,09\\
	n &=& 4
    \end{eqnarray*}

    \westep{Write down the formula}
    \begin{align*}
	A &= P(1 + i)^n
    \end{align*}

    \westep{Substitute the values and solve for the remaining variable}
    \begin{align*}
	A &= 14(1 + 0,09)^4\\
	  &= 19,76
    \end{align*}

    \westep{Explain your answer}
    In four years time, milk will cost R~$19,76$.
    }
\end{wex}


\begin{wex}{Calculating past cost based on inflation}
    {A box of chocolates costs R~$55$ today. How much did it cost 3 years ago if the average rate of inflation was $11\%$ p.a.?}{
    
    \westep{Write down the known variables}
    \begin{eqnarray*}
	A &=& 55\\
	P &=& ?\\
	i &=& 0,11\\
	n &=& 3
    \end{eqnarray*}

    \westep{Write down the formula}
    \begin{align*}
	A &= P(1 + i)^n
    \end{align*}

    \westep{Substitute the values and solve for the remaining variable}
    \begin{align*}
	55 &= P(1 + 0,11)^3\\
	\frac{55}{(1 + 0,11)^3} &= P\\
	P  &= 40,22
    \end{align*}

    \westep{Explain your answer}
    Three years ago, a box of chocolates would have cost R~$40,22$.
    }
\end{wex}


\subsection{Population growth}

How many people are your grandparents responsible for bringing into this world? Maybe they just had one child (your mom or dad) or maybe they had a dozen. How many of your parent’s siblings then had children of their own? How many of your cousins have already started having babies? A family tree typically starts with two people who have offspring. These people then have more babies which in turn procreate further. The family tree increases exponentially as every person who is born has the ability to be a part of the creation process of another. For this reason we calculate population growth using the compound interest formula.


\begin{wex}{Population growth}
    {If the current population of Johannesburg is $3~888~180$, and the average rate of population growth in South Africa is $2,1\%$ p.a., what can city planners expect the population of Johannesburg to be in 10 years?}{
    
    \westep{Write down the known variables}
    \begin{eqnarray*}
	A &=& ?\\
	P &=& 3~888~180\\
	i &=& 0,021\\
	n &=& 10
    \end{eqnarray*}

    \westep{Write down the formula}
    \begin{align*}
	A &= P(1 + i)^n
    \end{align*}

    \westep{Substitute the values and solve for the remaining variable}
    \begin{align*}
	A &= 3~888~180(1 + 0,021)^{10}\\
	  &= 4~786~342,614
    \end{align*}

    \westep{Explain your answer}
    City planners can expect a population of $4~786~343$ in Johannesburg in ten years time.
    }
\end{wex}


\begin{exercises}{}{
    \begin{enumerate}[label=\textbf{\arabic*}.]
	\item If the average rate of inflation for the past few years was $7,3\%$ and your water and electricity account is R~$1~425$ on average, what would you expect to pay in 6 years time?

	\item The price of pop corn and coke at the movies is now R~$60$. If the average rate of inflation is $9,2\%$, what was the price of pop corn and coke 5 years ago?

	\item A small town in Ohio, USA is experience a huge increase in births. If the average growth rate of the population is $16\%$ p.a, how many babies will be born to the 1600 residents in the next 2 years?\\
    \end{enumerate}

    Find the answers with the short codes:\\
    \begin{tabularx}{\textwidth}{ XXX }
	(1)	&	(2)	&	(3)\\
    \end{tabularx}
}
\end{exercises}



\subsection{Foreign exchange rates}

Different countries have developed their own currencies through the years. In England, a Big Mac from McDonalds will cost you £~$4$, in South Africa it will cost you R~$20$ and in Norway it will cost you $48$~kr. In all three of those places, you will get the same meal, however, some will be more expensive than others. At the time of writing this paragraph, £~$1$ $=$ R~$12,41$ and $1$~kr $=$ R~$1,37$, this means that the Big Mac would cost a South African who was travelling to England R~$49,64$ and one who was travelling to Norway R~$65,76$.\par

Exchange rates affect a lot more than just the price of a Big Mac. The price of oil will go up when the South African Rand gets weaker. This is because when the Rand is weaker, we can buy less of other currencies with the same amount of money.\par

A currency gets stronger when money is invested into the country. When we buy products that are made in South Africa, we are investing in South African business and keeping the money in the country. When we buy products imported from other countries, we are investing money into those countries and as a result, the Rand will weaken. Remember, the more South African products we buy, the greater demand there will be for them, meaning that more jobs will open for South Africans. Local is lekker!


\begin{wex}{Foreign exchange rates}
    {Saba wants to travel to see family in Spain. She has been given R~$10~000$ spending money. How many Euros can she buy if the exchange rate is currently €~$1$ $=$ R~$10,68$?}{
    
    Saba can buy €~$936,33$ with R~$10~000$.
}
\end{wex}


\begin{exercises}{}
{
    \begin{enumerate}[itemsep=6pt, label=\textbf{\arabic*}.]
	\item I want to buy an iPod that costs £~$100$, with the exchange rate currently at £~$1$ $=$ R~$14$. I believe the exchange rate will reach R~$12$ in a month.
	\begin{enumerate}[noitemsep, label=\textbf{(\alph*)} ]
	    \item How much will the MP3 player cost in Rands, if I buy it now?
	    \item How much will I save if the exchange rate drops to R~$12$?
	    \item How much will I lose if the exchange rate moves to R~$15$?
	\end{enumerate}

	\item Study the following exchange rate table:
	\begin{center}
	    \begin{tabular}{ |l|l|l| }
		\hline
		\textbf{Country}	&	\textbf{Currency}	&	\textbf{Exchange Rate}\\ \hline
		United Kingdom (UK)	&	Pounds (£)	&	R~$14,13$\\ \hline
		United States (USA)	&	Dollars (\$)	&	R~$7,04$\\ \hline
	    \end{tabular}
	\end{center}
	\\
	\begin{enumerate}[noitemsep, label=\textbf{(\alph*)} ]
	    \item In South Africa the cost of a new Honda Civic is R~$173~400$. In England the same vehicle costs £~$12~200$ and in the USA \$~$21~900$. In which country is the car the cheapest when you compare the prices converted to South African Rand?

	    \item Sollie and Arinda are waiters in a South African restaurant attracting many tourists from abroad. Sollie gets a £~$6$ tip from a tourist and Arinda gets \$~$12$. How many South African Rand did each one get?
	\end{enumerate}
    \end{enumerate}

    Find the answers with the short codes:\\
    \begin{tabularx}{\textwidth}{ XX }
	(1)	&	(2)\\
    \end{tabularx}
}
\end{exercises}


\summary

\begin{itemize}
    \item There are two types of interest rates that can be charged:\\
    
% \framebox{
    \begin{tabularx}{\textwidth}{ XX }
	Simple	&	Compound\\
	$A = P (1 + i . n)$	&	$A = P(1 + i)^n$\\
    \end{tabularx}
    \par
    Where:
    \begin{eqnarray*}
% 	\text{Where:~}\\
	A &=& \text{accumulated amount}\\
	P &=& \text{principal amount}\\
	i &=& \text{interest written as decimal}\\
	n &=& \text{number of years}
    \end{eqnarray*}
% }

    \item Hire purchase loan repayments are calculated using the simple interest formula on the remaining loan amount. Monthly repayments are calculated by dividing the accumulated amount by the number of months the loan needs to be paid off in.

    \item Population growth and inflation are calculated using the compound interest formula.

    \item Foreign exchange rate is the price of one currency in terms of another.
\end{itemize}


The following three videos provide a summary of how to calculate interest. Take note that although the examples are done using dollars, we can use the fact that dollars are a decimal currency and so are interchangeable (ignoring the exchange rate) with rands. This is what is done in the subtitles.\par 

\begin{figure}[H]
    \textnormal{Khan academy video on interest - 1}
    \vspace{.1in}
    \nopagebreak
    \raisebox{-5 pt}{ \includegraphics[width=0.5cm]{col11306.imgs/summary_www.png}} { (Video:  MG10030 )}
 \end{figure}

\begin{figure}[H] % horizontal\label{m39335*interest-2}
    \textnormal{Khan academy video on interest - 2}
    \vspace{.1in}
    \nopagebreak
    \raisebox{-5 pt}{ \includegraphics[width=0.5cm]{col11306.imgs/summary_www.png}} { (Video:  MG10031 )}
\end{figure}

Note in this video that at the very end the rule of 72 is mentioned. You will not be using this rule, but will rather be using trial and error to solve the problem posed.

\begin{figure}[H] % horizontal\label{m39335*interest-3}
    \textnormal{Khan academy video on interest - 3}
    \vspace{.1in}
    \nopagebreak
    \raisebox{-5 pt}{ \includegraphics[width=0.5cm]{col11306.imgs/summary_www.png}} { (Video:  MG10032 )}
\end{figure}


\begin{eocexercises}{}
    \begin{enumerate}[label=\textbf{\arabic*}.]
	\item You are going on holiday to Europe. Your hotel will cost €~$200$ per night. How much will you need in Rands to cover your hotel bill, if the exchange rate is €~$1$ = R~$9,20$?

	\item Calculate how much you will earn if you invested R~$500$ for 1 year at the following interest rates:
	\begin{enumerate}[noitemsep, label=\textbf{(\alph*)} ]
	    \item $6,85\%$ simple interest.
	    \item $4,00\%$ compound interest.
	\end{enumerate}

	\item Bianca has R~$1~450$ to invest for 3 years. Bank A offers a savings account which pays simple interest at a rate of $11\%$ per annum, whereas Bank B offers a savings account paying compound interest at a rate of $10,5\%$ per annum. Which account would leave Bianca with the highest accumulated balance at the end of the 3 year period?

	\item How much simple interest is payable on a loan of R~$2~000$ for a year, if the interest rate is $10\%$?

	\item How much compound interest is payable on a loan of R~$2~000$ for a year, if the interest rate is $10\%$?

	\item Discuss:
	\begin{enumerate}[noitemsep, label=\textbf{(\alph*)} ]
	    \item Which type of interest would you like to use if you are the borrower?

	    \item Which type of interest would you like to use if you were the banker?
	\end{enumerate}

	\item Calculate the compound interest for the following problems.
	\begin{enumerate}[noitemsep, label=\textbf{(\alph*)} ]
	    \item A R~$2~000$ loan for 2 years at $5\%$.
	    \item A R~$1~500$ investment for 3 years at $6\%$.
	    \item An R~$800$ loan for 1 year at $16\%$.
	\end{enumerate}

	\item If the exchange rate for ¥~$100$ = R~$6,2287$ (¥ = Yen) and 1 Australian Dollar (AUD) = R~$5,1094$, determine the exchange rate between the Australian Dollar and the Japanese Yen.

	\item Bonnie bought a stove for R~$3~750$. After 3 years she had finished paying for it and the R~$956,25$ interest that was charged for hire purchase. Determine the rate of simple interest that was charged.
    \end{enumerate}

    Find the answers with the short codes:\\
    \begin{tabularx}{\textwidth}{ XXXXXXXXX }
	(1) & (2) & (3) & (4) & (5) & (6) & (7) & (8) & (9)\\
    \end{tabularx}
\end{eocexercises}
