\chapter{Trigonometry}
\setcounter{figure}{1}
\setcounter{subfigure}{1}

Trigonometry (pronounced: trig-oh-nom-eh-tree) deals with the relationship between the angles and the sides of a right-angled triangle. We will learn about trigonometric functions, which form the basis of trigonometry.\par 

\section{Trigonometry is useful}
\nopagebreak
There are many applications of trigonometry. Of particular value is the technique of triangulation, which is used in astronomy to measure the distance to nearby stars, in geography to measure distances between landmarks, and in satellite navigation systems. GPSs (global positioning systems) would not be possible without trigonometry. Other fields which make use of trigonometry music theory, acoustics, optics, analysis of financial markets, electronics, probability theory, statistics, biology, medical imaging (CAT scans and ultrasound), pharmacy, chemistry, cryptology, meteorology, oceanography, land surveying, architecture, phonetics, engineering, computer graphics and game development.\par 

\section{Similarity of triangles}
\nopagebreak
If $\triangle ABC$ is similar to $ \triangle DEF$, then this is written as:\par 

\begin{equation*}
\triangle ABC||| \triangle DEF
\end{equation*}

\setcounter{subfigure}{0}
\begin{figure}[H] % horizontal\label{m39405*id78218}% \label{m39405*secfhsst!!!underscore!!!id95}
%             \subsection{  Discussion : Uses of Trigonometry }
%             \nopagebreak
%             
%       \label{m39405*id78126}Select one of the fields that uses trigonometry from the list given above and write a 1-page report describing \textsl{how} trigonometry is used in the field that you chose. \par 
\begin{center}
\begin{pspicture}(-1,-0.6)(6.6,3)
%\psgrid[gridcolor=gray]
\rput(1.2,0){\pstTriangle(0,0){A}(2; 25){B}(3; 125){C}}
\rput(5.2,0){\pstTriangle[unit=0.5](0,0){D}(2; 25){E}(3; 125){F}}
\end{pspicture}
\end{center}   
\end{figure}   
\par 
Then, it is possible to deduce ratios between corresponding sides of the two similar triangles:\par 


\begin{equation*}
\begin{array}{ccl}\hfill \frac{AB}{BC}& =& \frac{DE}{EF}\hfill \\ \hfill \frac{AB}{AC}& =& \frac{DE}{DF}\hfill \\ \hfill \frac{AC}{BC}& =& \frac{DF}{EF}\hfill \\ \hfill \frac{AB}{DE}& =& \frac{BC}{EF}=\frac{AC}{DF}\hfill \end{array}
\end{equation*}
Another important fact about similar triangles $ABC$ and $DEF$ is that the angle at vertex $A$ is equal to the angle at vertex $D$, the angle at $B$ is equal to the angle at $E$, and the angle at $C$ is equal to the angle at $F$.\par 


\begin{equation*}
\begin{array}{ccc}\hfill \hat A& =& \hat D\hfill \\ \hfill \hat B& =& \hat E\hfill \\ \hfill \hat C& =& \hat F\hfill \end{array}
\end{equation*}

\begin{Investigation}{Ratios of similar triangles}
\nopagebreak
Draw three similar triangles of different sizes, but each with $\hat{A}={30}^{\circ }$; $\hat{B}={90}^{\circ }$ and $\hat{C}={60}^{\circ }$. Measure angles and lengths very accurately in order to fill in the table (round answers to one decimal place):\par 

\setcounter{subfigure}{0}
\scalebox{1} % Change this value to rescale the drawing.
{
\begin{pspicture}(0,-2.3817186)(8.567187,2.3817186)
\psline[linewidth=0.04cm](0.301875,-1.9001563)(1.942836,-1.9001563)
\psline[linewidth=0.04cm](1.942836,-1.9001563)(1.942836,0.84772635)
\psline[linewidth=0.04cm](0.301875,-1.9001563)(1.942836,0.8604187)
\psline[linewidth=0.04cm](2.641875,-1.9201562)(4.641875,-1.9201562)
\psline[linewidth=0.04cm](4.641875,-1.9201562)(4.641875,1.5438437)
\psline[linewidth=0.04cm](2.641875,-1.9201562)(4.641875,1.5598438)
\psline[linewidth=0.04cm](5.521875,-1.9201562)(7.801521,-1.9201562)
\psline[linewidth=0.04cm](7.801521,-1.9201562)(7.801521,1.9617095)
\psline[linewidth=0.04cm](5.521875,-1.9201562)(7.801521,1.9796396)
\usefont{T1}{ptm}{m}{n}
\rput(0.26546875,-2.1101563){$C$}
\usefont{T1}{ptm}{m}{n}
\rput(2.8054688,-2.1901562){$C'$}
\usefont{T1}{ptm}{m}{n}
\rput(5.6254687,-2.1901562){$C''$}
\usefont{T1}{ptm}{m}{n}
\rput(2.0454688,-2.1301563){$B$}
\usefont{T1}{ptm}{m}{n}
\rput(4.7654686,-2.1901562){$B'$}
\usefont{T1}{ptm}{m}{n}
\rput(8.105469,-2.2101562){$B''$}
\usefont{T1}{ptm}{m}{n}
\rput(2.1254687,1.0898438){$A$}
\usefont{T1}{ptm}{m}{n}
\rput(4.865469,1.8098438){$A'$}
\usefont{T1}{ptm}{m}{n}
\rput(8.105469,2.1698437){$A''$}
\psframe[linewidth=0.04,dimen=outer](1.961875,-1.5601562)(1.601875,-1.9201562)
\psframe[linewidth=0.04,dimen=outer](4.661875,-1.5801562)(4.301875,-1.9401562)
\psframe[linewidth=0.04,dimen=outer](7.801875,-1.5801562)(7.441875,-1.9401562)
\usefont{T1}{ptm}{m}{n}
\rput(1.64,-0.25){$30^{\circ}$}
% \pscircle[linewidth=0.0060,dimen=outer](1.811875,0.02984375){0.03}
\usefont{T1}{ptm}{m}{n}
\rput(0.9,-1.6){$60^{\circ}$}
% \pscircle[linewidth=0.0060,dimen=outer](0.951875,-1.6101563){0.03}
\usefont{T1}{ptm}{m}{n}
\rput(3.3,-1.6){$60^{\circ}$}
% \pscircle[linewidth=0.0060,dimen=outer](3.291875,-1.6301563){0.03}
\usefont{T1}{ptm}{m}{n}
\rput(6.2,-1.6){$60^{\circ}$}
% \pscircle[linewidth=0.0060,dimen=outer](6.131875,-1.6301563){0.03}
\usefont{T1}{ptm}{m}{n}
\rput(4.3173437,0.4){$30^{\circ}$}
% \pscircle[linewidth=0.0060,dimen=outer](4.531875,0.64984375){0.03}
\usefont{T1}{ptm}{m}{n}
\rput(7.4973435,0.9){$30^{\circ}$}
% \pscircle[linewidth=0.0060,dimen=outer](7.711875,1.1498437){0.03}
\end{pspicture} 
}      
\par 
% \textbf{m39405*id78604}\par
\begin{table}[H]
% \begin{table}[H]
% \\ '' '0'
\begin{center}

\noindent

\begin{tabular}{|l|l|l|}\hline
% My position: 0
% my spanname: 
% my ct of spanspec: 0
% my column-count: 3
\multicolumn{3}{|c|}{Dividing lengths of sides (ratios)}
\\ \hline
%--------------------------------------------------------------------
$\frac{AB}{BC}=\phantom{\rule{42.67912pt}{0ex}}$
&
$\frac{AB}{AC}=\phantom{\rule{42.67912pt}{0ex}}$
&
$\frac{CB}{AC}=\phantom{\rule{42.67912pt}{0ex}}$
% make-rowspan-placeholders
\\ \hline
%--------------------------------------------------------------------
$\frac{A'B'}{B'C'}=\phantom{\rule{42.67912pt}{0ex}}$
&
$\frac{A'B'}{A'C'}=\phantom{\rule{42.67912pt}{0ex}}$
&
$\frac{C'B'}{A'C'}=\phantom{\rule{42.67912pt}{0ex}}$
% make-rowspan-placeholders
\\ \hline
%--------------------------------------------------------------------
$\frac{A''B''}{B''C''}=\phantom{\rule{42.67912pt}{0ex}}$
&
$\frac{A''B''}{A''C''}=\phantom{\rule{42.67912pt}{0ex}}$
&
$\frac{C''B''}{A''C''}=\phantom{\rule{42.67912pt}{0ex}}$
% make-rowspan-placeholders
\\ \hline
%--------------------------------------------------------------------
\end{tabular}
\end{center}
% \begin{center}{\small\bfseries Table 14.1}\end{center}
% \begin{caption}{\small\bfseries Table 14.1}\end{caption}
\end{table}
\par
What observations can you make about the ratios of the sides?\\
\\
These equal ratios are used to define the trigonometric ratios.\par 
\end{Investigation}


    

\section{Defining of the trigonometric ratios}
Consider a right-angled triangle.\par 

\setcounter{subfigure}{0}
\begin{figure}[H] % horizontal\label{m39408*id79667}
\begin{center}
\begin{pspicture}(-1,-0.6)(4,3)
%\psgrid[gridcolor=gray]
\pstTriangle(0,0){A}(2.4;90){B}(3.35;0){C}
\pstRightAngle{B}{A}{C}
\pstMarkAngle{B}{C}{A}{$\theta$}
\pcline[linestyle=none](A)(B)
\aput{:U}{opposite}
\pcline[linestyle=none](B)(C)
\aput{:U}{hypotenuse}
\pcline[linestyle=none](A)(C)
\bput{:U}{adjacent}
\end{pspicture}
\end{center}
\end{figure}       
\par 

\Note{In algebra, we often use the letter $x$ for our unknown variable (although we can use any other letter too, such as $a$, $b$, $k$, etc). In trigonometry, we often use the Greek symbol $\theta $ for an unknown angle (we also use $\alpha $ , $\beta $ , $\gamma $ etc).}


In the right-angled triangle, we refer to the lengths of the three sides according to how they are placed in relation to the angle $\theta $. The side opposite to the right angle is labeled the hypotenuse, the side opposite $\theta $ is labeled opposite, the side next to $\theta $ is labeled adjacent. Note that the choice of non-$90^{\circ}$ internal angle is arbitrary. You can choose either internal angle and then define the adjacent and opposite sides accordingly. However, the hypotenuse remains the same regardless of which internal angle you are referring to (because it is ALWAYS opposite the right angle and ALWAYS the longest side).\par 

We define the trigonometric ratios, sine, cosine and tangent of an angle, as follows:
\par 

\begin{equation*}
\begin{array}{ccc}\hfill sin~\theta & =& \frac{\mbox{opposite}}{\mbox{hypotenuse}}\hfill \\
\\
 \hfill cos~\theta & =& \frac{\mbox{adjacent}}{\mbox{hypotenuse}}\hfill \\
\\
 \hfill tan~\theta & =& \frac{\mbox{opposite}}{\mbox{adjacent}}\hfill 
\end{array}
\end{equation*}

These ratios, also known as trigonometric identities, relate the lengths of the sides of a right-angled triangle to its interior angles.\par 

These three ratios form the basis of trigonometry. \par

You may also hear people saying ``Soh Cah Toa''. This is a mnemonic technique for remembering the trigonometric ratios.\par 

% \textbf{m39408*id79953}\par
\begin{table}[H]
% \begin{table}[H]
% \\ '' '0'
\begin{center}
\label{m39408*id79953}
\noindent

\begin{tabular}{|l|}\hline

$\mathbf{s}in=\frac{\mbox{\textbf{o}pposite}}{\mbox{\textbf{h}ypotenuse}} $
% make-rowspan-placeholders

\\  \hline
%--------------------------------------------------------------------

$\mathbf{c}os=\frac{\mbox{\textbf{a}djacent}}{\mbox{\textbf{h}ypotenuse}} $
% make-rowspan-placeholders

\\ \hline
%--------------------------------------------------------------------

$\mathbf{t}an=\frac{\mbox{\textbf{o}pposite}}{\mbox{\textbf{a}djacent}} $
% make-rowspan-placeholders

\\   \hline
%--------------------------------------------------------------------
\end{tabular}
\end{center}
% \begin{center}{\small\bfseries Table 14.2}\end{center}
% \begin{caption}{\small\bfseries Table 14.2}\end{caption}
\end{table}
\par

\Tip{The definitions of opposite, adjacent and hypotenuse are only applicable when working with right-angled triangles! Always check to make sure your triangle has a right-angle before you use them, otherwise you will get the wrong answer.}

\begin{activity}{Calculator work}

It is very important that you are familiar with your calculator's trigonometric buttons and how they operate. \\
Use the following examples to make sure that you are using your calculator correctly:
\begin{enumerate}[noitemsep, label=\textbf{\arabic*}. ] 
 \item $cos~ 30^{\circ} = 0,866$
\item $sin~90^{\circ} = 1$
\item $tan~60^{\circ} = 1,732$
\item $4~sin~45^{\circ}=2,828$
\item $\frac{1}{3}~cos~60^{\circ}=1,5$
\end{enumerate}

\end{activity}

\begin{wex}{Using your calculator}

 {Use your calculator to calculate the following (correct to $2$ decimal places):
\begin{enumerate}[noitemsep, label=\textbf{\arabic*}. ] 
 \item $cos~48^{\circ}$
\item $2~sin~35^{\circ}$
\item $tan^{2}~81^{\circ}$
\item $3~sin^{2}~72^{\circ} \vspace{3pt}$
\item $\dfrac{1}{4}~cos~27^{\circ} \vspace{3pt}$
\item $\dfrac{5}{6}~tan~34^{\circ}$
\end{enumerate}

}
{

\westep{}
Press \fbox{COS} \fbox{48} \fbox{\LARGE =} $0,67$

\westep{}
Press \fbox{2} \fbox{SIN} \fbox{35} \fbox{\LARGE =} $1,15$

\westep{}
Press \fbox{(} \fbox{TAN} \fbox{81} \fbox{)} \fbox{x^{2}}  \fbox{\LARGE =} $39,83$
\\
OR\\
Press \fbox{TAN} \fbox{81} \fbox{\LARGE =} \fbox{ANS} \fbox{x^{2}} \fbox{\LARGE =} $39,83$

\westep{}
Press \fbox{3} \fbox{(} \fbox{SIN} \fbox{72} \fbox{)} \fbox{x^{2}} \fbox{\LARGE =} $2,71$
\\
OR\\
Press \fbox{SIN} \fbox{72} \fbox{\LARGE =} \fbox{ANS} \fbox{x^{2}} \fbox{\LARGE =} \fbox{$\times$} \fbox{3}

\westep{}
Press \fbox{(} \fbox{1} \fbox{$\div$} \fbox{4} \fbox{)} \fbox{COS} \fbox{27} \fbox{\LARGE =} $0,22$
\\
OR\\
Press \fbox{COS} \fbox{27} \fbox{\LARGE =} \fbox{ANS} \fbox{$\div$} \fbox{4} \fbox{\LARGE =} $0,22$

\westep{}
Press \fbox{(} \fbox{5} \fbox{$\div$} \fbox{6} \fbox{)} \fbox{TAN} \fbox{34} \fbox{\LARGE =} $0,56$
\\
OR\\
Press \fbox{TAN} \fbox{34} \fbox{\LARGE =} \fbox{ANS} \fbox{$\times$} \fbox{5} \fbox{$\div$} \fbox{6} \fbox{\LARGE =} $0,56$

}
\end{wex}
\Note{Most scientific calculators are quite similar but these steps might differ depending on the calculator you use. Make sure your calculator is in 'degrees' mode.}

\begin{wex}
{Calculator work}
{If $x=25^{\circ}$ and $y=65^{\circ}$, use your calculator to determine whether the following statement is true or false:
\begin{equation*}
sin^{2}~x + cos^{2}~(90^{\circ}-y) = 1
\end{equation*}
}
{
\westep{Calculate the left hand side of the equation}
Press \fbox{(} \fbox{SIN} \fbox{25} \fbox{)} \fbox{x^{2}}  \fbox{\LARGE =} $0,179$\\
Press \fbox{(} \fbox{COS} \fbox{(} \fbox{90} \fbox{\LARGE -} \fbox{65} \fbox{)} \fbox{)} \fbox{x^{2}}  \fbox{\LARGE =} $0,821$

\westep{Add both values together}
\begin{equation*}
 0,179+0,821 = 1
\end{equation*}
\westep{Write the final answer}
LHS = RHS therefore the statement is true.



}
\end{wex}

\begin{exercises}{}
{

In each of the following triangles, state whether $a$, $b$ and $c$ are the hypotenuse, opposite or adjacent sides of the triangle with respect to \theta. 
\setcounter{subfigure}{0}
\begin{center}
\scalebox{0.85} % Change this value to rescale the drawing.
{
% EACH FIGURE NEEDS TO BE NUMBERED 1 -6~
\begin{pspicture}(0,-3.7754679)(20.177345,3.9279697)
\rput (1,2){\textbf{1.}}
\rput (5.5,2){\textbf{2.}}
\rput (10.5,2){\textbf{3.}}
\rput (1.5,-1.5){\textbf{4.}}
\rput (6,-1.5){\textbf{5.}}
\rput (9.5,-1.5){\textbf{6.}}
\psdots[dotsize=0.027999999](2.9173439,-1.5720303)
\psline[linewidth=0.04cm](5.517344,2.5079696)(7.9173436,0.107969664)
\psline[linewidth=0.04cm](7.9173436,0.107969664)(9.0173435,1.2079697)
\psline[linewidth=0.04cm](5.517344,2.5079696)(9.0173435,1.2079697)
\psline[linewidth=0.04cm](7.717344,0.30796966)(7.9173436,0.5079697)
\psline[linewidth=0.04cm](7.9173436,0.5079697)(8.117344,0.30796966)
\psline[linewidth=0.04cm](11.0573435,2.2679696)(11.0573435,0.067969665)
\psline[linewidth=0.04cm](11.0573435,0.067969665)(14.757344,0.067969665)
\psline[linewidth=0.04cm](11.0573435,2.2679696)(14.757344,0.067969665)
\psline[linewidth=0.04cm](1.2373447,-3.3120303)(3.9373438,-0.71203035)
\psline[linewidth=0.04cm](3.9373438,-0.71203035)(3.9373438,-3.3120303)
\psline[linewidth=0.04cm](3.9373438,-3.3120303)(1.2373447,-3.3120303)
\psline[linewidth=0.04cm](7.477345,-1.2120303)(5.7773438,-3.3120303)
\psline[linewidth=0.04cm](7.477345,-1.2120303)(8.9773445,-2.4120302)
\psline[linewidth=0.04cm](8.9773445,-2.4120302)(5.7773438,-3.3120303)
\psline[linewidth=0.04cm](7.257343,-1.5120304)(7.537344,-1.7320304)
\psline[linewidth=0.04cm](7.517344,-1.7120303)(7.717344,-1.4320303)
\psline[linewidth=0.04cm](11.097343,0.38796967)(11.397344,0.38796967)
\psline[linewidth=0.04cm](11.397344,0.38796967)(11.397344,0.08796967)
\psline[linewidth=0.04cm](3.9173427,-3.0320303)(3.6173437,-3.0320303)
\psline[linewidth=0.04cm](3.6173437,-3.0320303)(3.6173437,-3.3320303)
\psline[linewidth=0.04cm](10.0173435,-1.7320304)(11.417343,-3.2320304)
\psline[linewidth=0.04cm](11.417343,-3.2320304)(13.0173435,-1.7320304)
\psline[linewidth=0.04cm](10.0173435,-1.7320304)(13.0173435,-1.7320304)
\psline[linewidth=0.04cm](11.217343,-3.0320303)(11.417343,-2.8320303)
\psline[linewidth=0.04cm](11.417343,-2.8320303)(11.617344,-3.0320303)
\usefont{T1}{ptm}{m}{n}
\rput(8.662344,1.1179696){$\theta$}
\usefont{T1}{ptm}{m}{n}
\rput(13.842344,0.29796967){$\theta$}
\usefont{T1}{ptm}{m}{n}
\rput(3.682343,-1.3020303){$\theta$}
\usefont{T1}{ptm}{m}{n}
\rput(6.362344,-2.9020302){$\theta$}
\usefont{T1}{ptm}{m}{n}
\rput(10.502343,-1.9420303){$\theta$}
\psline[linewidth=0.04cm](2.2973437,2.2879696)(0.09734377,0.18796967)
\psline[linewidth=0.04cm](0.09734377,0.18796967)(4.4973435,0.18796967)
\psline[linewidth=0.04cm](2.2973437,2.2879696)(4.4973435,0.18796967)
\psline[linewidth=0.04cm](2.0973437,2.0879698)(2.2973437,1.8879696)
\psline[linewidth=0.04cm](2.2973437,1.8879696)(2.4973438,2.0879698)
\usefont{T1}{ptm}{m}{n}
\rput(0.54234374,0.37796965){$\theta$}
\usefont{T1}{ptm}{m}{n}
\rput(1.1764063,1.5979697){$a$}
\rput (1, 2){1.}
\usefont{T1}{ptm}{m}{n}
\rput(2.1778126,-0.10203034){$b$}
\usefont{T1}{ptm}{m}{n}
\rput(3.759219,1.2979697){$c$}
\usefont{T1}{ptm}{m}{n}
\rput(6.596406,1.0179696){$a$}
\usefont{T1}{ptm}{m}{n}
\rput(7.479219,2.1179698){$c$}
\usefont{T1}{ptm}{m}{n}
\rput(8.797812,0.41796967){$b$}
\usefont{T1}{ptm}{m}{n}
\rput(13.076406,1.4979696){$a$}
\usefont{T1}{ptm}{m}{n}
\rput(10.777813,1.1979697){$b$}
\usefont{T1}{ptm}{m}{n}
\rput(12.659219,-0.20203033){$c$}
\usefont{T1}{ptm}{m}{n}
\rput(2.896407,-3.6220303){$a$}
\usefont{T1}{ptm}{m}{n}
\rput(2.2978134,-1.8220303){$b$}
\usefont{T1}{ptm}{m}{n}
\rput(4.2792187,-2.3220303){$c$}
\usefont{T1}{ptm}{m}{n}
\rput(6.596406,-1.8220303){$a$}
\usefont{T1}{ptm}{m}{n}
\rput(7.7978134,-3.1220303){$b$}
\usefont{T1}{ptm}{m}{n}
\rput(8.579218,-1.7220303){$c$}
\usefont{T1}{ptm}{m}{n}
\rput(10.496406,-2.6220303){$a$}
\usefont{T1}{ptm}{m}{n}
\rput(12.397812,-2.8220303){$b$}
\usefont{T1}{ptm}{m}{n}
\rput(11.579218,-1.4220303){$c$}
\rput{-66.69996}(-0.02258055,0.761629){\psarc[linewidth=0.04](0.5673438,0.39796966){0.35}{25.785578}{138.26501}}
\rput{106.83747}(12.386083,-6.856496){\psarc[linewidth=0.04](8.737344,1.1679693){0.34}{32.561234}{147.21713}}
\rput{93.56388}(15.110441,-13.562675){\psarc[linewidth=0.04](13.927344,0.31796986){0.39}{24.333387}{127.3505}}
\psarc[linewidth=0.04](14.697344,3.1279697){0.0}{0.0}{180.0}
\psarc[linewidth=0.04](14.837344,3.2279696){0.0}{0.0}{180.0}
\rput{153.95303}(6.6359906,-3.7989676){\psarc[linewidth=0.04](3.757346,-1.1320316){0.5}{50.00946}{137.84839}}
\psarc[linewidth=0.04](20.097343,3.9079697){0.0}{0.0}{180.0}
\psarc[linewidth=0.04](20.117344,3.8879697){0.0}{0.0}{180.0}
\psarc[linewidth=0.04](20.157345,3.8279696){0.0}{0.0}{180.0}
\rput{-46.143967}(4.180272,3.6097476){\psarc[linewidth=0.04](6.3273435,-3.1020286){0.57}{58.40874}{128.71255}}
\rput{-116.039185}(16.474606,6.6625834){\psarc[linewidth=0.04](10.317342,-1.8120308){0.52}{49.23777}{123.63383}}
\end{pspicture} 
}
\end{center}


Use your calculator to determine the value of the following (correct to $2$ decimal places):

\begin{enumerate}[noitemsep, label=\textbf{\arabic*}. ] 
\begin{multicols}{2} 
\setcounter{enumi}{6} 
\item $tan~65^{\circ}$
\item $sin~38^{\circ}$
\item $cos~74^{\circ}$
\item $sin~12^{\circ}$
\item $cos~26^{\circ}$
\item $tan~49^{\circ}$
\item $\frac{1}{4}~cos~20^{\circ}$
\item $3~tan~40^{\circ}$
\item $\frac{2}{3}~sin~90^{\circ}$
\end{multicols}
\end{enumerate}

If $x=39^{\circ}$ and $y=21^{\circ}$ use a calculator to determine whether the following statements are true or false:
\begin{enumerate}[noitemsep, label=\textbf{\arabic*}. ] 
\begin{multicols}{2} 
\setcounter{enumi}{15} 
\item $cos~x + 2~cos~x=3~cos~x$
\item $cos~2y = cos~y+cos~y$
\item $tan~x=\frac{sin~x}{cos~x}$
\item $cos~(x+y) = cos~x+cos~y$
\end{multicols}
\end{enumerate}


\item Complete each of the following (the first example has been done):
\begin{center}
% \begin{minipage}{0.25 \textwidth}
\setcounter{subfigure}{0}
\scalebox{1}{
\begin{pspicture}(0,-2.0990624)(4.21625,2.0990624)
\psline[linewidth=0.04cm](0.4775,-1.5590625)(3.7775,1.5409375)
\psline[linewidth=0.04cm](3.7775,1.5409375)(3.7775,-1.5590625)
\psline[linewidth=0.04cm](0.4775,-1.5590625)(3.7775,-1.5590625)
\psline[linewidth=0.04cm](3.4775,-1.5590625)(3.4775,-1.2590625)
\psline[linewidth=0.04cm](3.4775,-1.2590625)(3.7775,-1.2590625)
\rput(4.025,1.8759375){$A$}
\rput(3.874375,-1.9240625){ $B$}
\rput(0.10625,-1.8240625){$C$}
\end{pspicture} 
}
% \end{minipage}
\end{center}

\begin{enumerate}[noitemsep, label=\textbf{\arabic*}. ] 
\begin{multicols}{2}
\setcounter{enumi}{19}
\item $sin~\hat{A} = \frac{\mbox{opposite}}{\mbox{hypotenuse}}=\dfrac{CB}{AC} \vspace{3pt}$
\item $cos~\hat{A} = \vspace{3pt}$
\item $tan~\hat{A}= \vspace{3pt}$
\item $sin~\hat{C}= \vspace{3pt}$
\item $cos~\hat{C}= \vspace{3pt}$
\item $tan~\hat{C}= \vspace{3pt}$
\end{multicols}
\end{enumerate}
\\
Use the triangle below to complete the following:
\begin{center}
% \begin{minipage}{0.25\textwidth}
\scalebox{1} % Change this value to rescale the drawing.
{
\begin{pspicture}(0,-1.97)(5.3271875,1.97)
\psline[linewidth=0.04cm](0.361875,-1.95)(4.061875,-1.95)
\psline[linewidth=0.04cm](4.061875,-1.95)(4.061875,1.95)
\psline[linewidth=0.04cm](4.061875,1.95)(0.361875,-1.95)
\psline[linewidth=0.04cm](3.661875,-1.95)(3.661875,-1.55)
\psline[linewidth=0.04cm](3.661875,-1.55)(4.061875,-1.55)
\rput(1.7398437,0.26){$2$}
\rput(1.8,-2.2){$1$}
\rput(1.1554687,-1.64){$60^{\circ}$}
\rput(3.6554687,0.96){$30^{\circ}$}
\rput(4.405469,-0.44){$\sqrt{3}$}
\end{pspicture} 
}

% \end{minipage}
\end{center}
\\
\\
\begin{enumerate}[noitemsep, label=\textbf{\arabic*}. ] 
\begin{multicols}{2}
\setcounter{enumi}{25}
\item $sin~60^{\circ} = \vspace{3pt}$
\item $cos~60^{\circ} = \vspace{3pt}$
\item $tan~60^{\circ}= \vspace{3pt}$
\item $sin~30^{\circ}= \vspace{3pt}$
\item $cos~30^{\circ}= \vspace{3pt}$
\item $tan~30^{\circ}= \vspace{3pt}$
\end{multicols}
\end{enumerate}

\\
Use the triangle below to complete the following:
\begin{center}
% \begin{minipage}{0.25\textwidth}
\scalebox{1} % Change this value to rescale the drawing.
{
\begin{pspicture}(0,-2.24)(5.0271873,2.22)
\psline[linewidth=0.04cm](0.361875,-1.7)(4.061875,-1.7)
\psline[linewidth=0.04cm](4.061875,-1.7)(4.061875,2.2)
\psline[linewidth=0.04cm](4.061875,2.2)(0.361875,-1.7)
\psline[linewidth=0.04cm](3.661875,-1.7)(3.661875,-1.3)
\psline[linewidth=0.04cm](3.661875,-1.3)(4.061875,-1.3)
\rput(2.2314062,-2.09){$1$}
\rput(1.1554687,-1.39){$45^{\circ}$}
\rput(3.7554688,1.41){$45^{\circ}$}
\rput(1.6054688,0.51){$\sqrt{2}$}
\rput(4.331406,0.21){$1$}
\end{pspicture} 
}
% \end{minipage}
\end{center}

\begin{enumerate}[noitemsep, label=\textbf{\arabic*}. ] 

\setcounter{enumi}{31}
\item $sin~45^{\circ} = \vspace{3pt}$
\item $cos~45^{\circ} = \vspace{3pt}$
\item $tan~45^{\circ}= \vspace{3pt}$

\end{enumerate}
}
\end{exercises}

\section{Special angles}
For most angles $\theta $, we need a calculator to calculate the values of $sin~\theta $, $cos~\theta $ and $tan~\theta $. However, we saw in the previous exercise that we could work out these values for some special angles. Some of these angles are listed in the table below, along with the values of the trigonometric functions at these angles. \par
Remember that the lengths of the sides of a right angled triangle must obey the Theorem of Pythagoras: the square of the hypotenuse equals the sum of the squares of the two other sides.\par 
% \textbf{m39408*id80733}\par
\begin{table}[H]
% \begin{table}[H]
% \\ '' '0'
\begin{center}

\begin{tabular}{|l|l|l|l|l|l|}\hline
&
${0}^{\circ }$
&
${30}^{\circ }$
&
${45}^{\circ }$
&
${60}^{\circ }$
&
${90}^{\circ }$

% make-rowspan-placeholders\tabularnewline\cline{1-1}\cline{2-2}\cline{3-3}\cline{4-4}\cline{5-5}\cline{6-6}\cline{7-7}
\\ \hline
%--------------------------------------------------------------------
$cos~\theta $
&
$1$ &
$\frac{\sqrt{3}}{2}$
&
$\frac{1}{\sqrt{2}}$
&
$\frac{1}{2}$
&
$0 $

\\ \hline
%--------------------------------------------------------------------
$sin~\theta $
&
$0$ &
$\frac{1}{2}$
&
$\frac{1}{\sqrt{2}}$
&
$\frac{\sqrt{3}}{2}$
&
$1$ 
%  make-rowspan-placeholders
\\ \hline
%--------------------------------------------------------------------
$tan~\theta $
&
$0$ &
$\frac{1}{\sqrt{3}}$
&
$1$ &
$\sqrt{3}$
&


% make-rowspan-placeholders
\\ \hline
%--------------------------------------------------------------------
\end{tabular}
\end{center}
% \begin{center}{\small\bfseries Table 14.3}\end{center}
% \begin{caption}{\small\bfseries Table 14.3}\end{caption}
\end{table}
\par
These values are useful when asked to solve a problem involving trigonometric functions without using a calculator.\par 
\begin{exercises}{}
{
Calculate the value of the following without using a calculator (use the values in the table above):
\begin{enumerate}[noitemsep, label=\textbf{\arabic*}. ] 
\item $sin~45^{\circ} \times cos~45^{\circ}$
\item $cos~60^{\circ} + tan~45^{\circ}$
\item $sin~60^{\circ} - cos~60^{\circ}$
\end{enumerate}
\vspace {10pt}
Use the table to  show that:
\begin{enumerate}[noitemsep, label=\textbf{\arabic*}. ] 
\setcounter{enumi}{3}
\item $\dfrac{sin~60^{\circ}}{cos~60^{\circ}} = tan~60^{\circ} \vspace{3pt}$
\item $sin^{2}~45^{\circ}+ cos^{2}~45^{\circ} -1 \vspace{3pt}$
\item $cos~30^{\circ} =\sqrt{1- sin^{2}~30^{\circ}$
\end{enumerate}
\vspace {10pt}
Answer the following questions:
\begin{enumerate}[noitemsep, label=\textbf{\arabic*}. ] 
\setcounter{enumi}{6}
\item Explain why $sin~\alpha \leq 1$ for all values of $\alpha$.
\item Explain why $cos~\beta$ has a maximum value of $1$.
\item What can be said about the maximum value of $tan~\gamma$ ?
}
\end{exercises}


\section{Reciprocal functions}
Each of the three trigonometric functions has a reciprocal. The reciprocals are cosecant, secant and cotangent. These reciprocals are given below:

\begin{equation*}
\begin{array}{ccc}cosec~\theta & =& \dfrac{1}{sin~\theta } \vspace{3pt}\\
 sec~\theta & =& \dfrac{1}{cos~\theta } \vspace{3pt}\\
 cot~\theta & =& \dfrac{1}{tan~\theta }
\end{array}
\end{equation*}
We can also define these reciprocals for any right angled triangle:

\begin{equation*}
\begin{array}{ccc}\hfill cosec~\theta & =& \frac{\mbox{hypotenuse}}{\mbox{opposite}}\hfill \vspace{3pt}\\
 \hfill sec~\theta & =& \frac{\mbox{hypotenuse}}{\mbox{adjacent}}\hfill \vspace{3pt}\\
 \hfill cot~\theta & =& \frac{\mbox{adjacent}}{\mbox{opposite}}\hfill 
\end{array}
\end{equation*}

\section{Solving trigonometric equations}

\begin{wex}{Finding Lengths}{Find the length of $x$ in the following triangle: \\
\begin{center}
\scalebox{1} 
{
\begin{pspicture}(0,-1.02)(4.655469,2.02)
\psline[linewidth=0.04](4.0,-1.0)(1.0,-1.0)(4.0,2.0)(4.0,-1.0)
\rput(2.0275,0.77){$100$}
\psarc[linewidth=0.04](1.0,-1.0){1.0}{0.0}{45.0}
\rput(1.6450001,-0.75){$50^{\circ}$}
\rput(4.355,0.41){$x$}
\psline[linewidth=0.04cm](3.6,-0.98)(3.6,-0.58)
\psline[linewidth=0.04cm](3.6,-0.58)(4.0,-0.58)
\end{pspicture} 
}
\end{center}
}
{
\westep{Identify the opposite and adjacent sides and the hypotenuse}
\begin{equation*}
\begin{array}{ccl}
 
\hfill sin~\theta &=& \dfrac{\mbox{opposite}}{\mbox{hypotenuse}}  \hfill \vspace{5pt}\\
\hfill sin~ 50^\circ &=& \dfrac{x}{100}  \hfill \\
\end{array}
\end{equation*}



\westep{Rearrange the question to solve for $x$}
\begin{equation*}
 x=100 \times sin~50^{\circ}
\end{equation*}

\westep{Use your calculator to find the answer}
\begin{equation*}
x = 76.6
\end{equation*}
}
\end{wex}




The following videos provide a summary of what you have learnt so far.
\setcounter{subfigure}{0}
\begin{figure}[H] % horizontal\label{m39408*trigonometry-1}
\textnormal{Khan Academy video on trigonometry - 1}\vspace{.1in} \nopagebreak
\label{m39408*yt-media}\label{m39408*yt-video}
\raisebox{-5 pt}{ \includegraphics[width=0.5cm]{col11306.imgs/summary_www.png}} { (Video:  MG10102 )}
\vspace{2pt}
\vspace{.1in}
\end{figure}       \par \label{m39408*eip-33}
\setcounter{subfigure}{0}
\begin{figure}[H] % horizontal\label{m39408*trigonometry-2}
\textnormal{Khan Academy video on trigonometry - 2}\vspace{.1in} \nopagebreak
\label{m39408*yt-media2}\label{m39408*yt-video2}
\raisebox{-5 pt}{ \includegraphics[width=0.5cm]{col11306.imgs/summary_www.png}} { (Video:  MG10103 )}
\vspace{2pt}
\vspace{.1in}
\end{figure}       


\begin{exercises}{}
{

\item In each triangle find the length of the side marked with a letter. Give answers correct to $2$ decimal places.
\begin{center}
\scalebox{0.85} % Change this value to rescale the drawing.
{
\begin{pspicture}(0,-4.11)(11.1198435,5)
\psline[linewidth=0.04](4.974369,2.612122)(2.6350498,3.9603012)(0.6377475,0.49464345)(4.974369,2.612122)
\psline[linewidth=0.04cm](2.8949742,3.8105037)(2.7451768,3.550579)
\psline[linewidth=0.04cm](2.7451768,3.550579)(2.4852521,3.7003772)
\psline[linewidth=0.04](4.557088,-3.2752385)(4.589271,-0.5754303)(0.5895552,-0.52775127)(4.557088,-3.2752385)
\psline[linewidth=0.04cm](4.5856953,-0.8754088)(4.2857165,-0.87183315)
\psline[linewidth=0.04cm](4.2857165,-0.87183315)(4.2892923,-0.5718545)
\psline[linewidth=0.04](7.0587263,3.3569472)(6.922255,0.6603981)(10.917143,0.45821914)(7.0587263,3.3569472)
\psline[linewidth=0.04cm](6.9374194,0.9600146)(7.2370353,0.94485116)
\psline[linewidth=0.04cm](7.2370353,0.94485116)(7.2218714,0.64523476)
\psline[linewidth=0.04](7.9354,-3.9493167)(10.598035,-3.9603012)(10.613694,-0.16448934)(7.9354,-3.9493167)
\psline[linewidth=0.04cm](10.340361,-3.959238)(10.341382,-3.7116852)
\psline[linewidth=0.04cm](10.341382,-3.7116852)(10.599056,-3.7127483)
% \usefont{T1}{ptm}{m}{n}
\rput(0.12453125,3.7196066){\textbf{1.}}
% \usefont{T1}{ptm}{m}{n}
\rput(6.133125,3.7396066){\textbf{2.}}
% \usefont{T1}{ptm}{m}{n}
\rput(0.11421875,-0.26039344){\textbf{3.}}
% \usefont{T1}{ptm}{m}{n}
\rput(6.1343746,-0.28039345){\textbf{4.}}
\usefont{T1}{ptm}{m}{n}
\rput(4.197969,2.6346066){$37^\circ$}
\usefont{T1}{ptm}{m}{n}
\rput(3.0853126,1.2){$62$}
\usefont{T1}{ptm}{m}{n}
\rput(1.3824998,2.5846066){$a$}
\usefont{T1}{ptm}{m}{n}
\rput(9.804531,0.8146064){$23^\circ$}
\usefont{T1}{ptm}{m}{n}
\rput(8.595937,0.21460645){$21$}
\usefont{T1}{ptm}{m}{n}
\rput(6.663907,2.0646067){$b$}
\usefont{T1}{ptm}{m}{n}
\rput(4.2134376,-2.6453934){$ 55^\circ$}
\usefont{T1}{ptm}{m}{n}
\rput(2.166719,-2.1653934){$1$}
\usefont{T1}{ptm}{m}{n}
\rput(4.8759375,-1.8353934){$c$}
\usefont{T1}{ptm}{m}{n}
\rput(8.953907,-1.9653934){$33$}
\usefont{T1}{ptm}{m}{n}
\rput(10.22,-1.1453935){ $49^\circ$}
\usefont{T1}{ptm}{m}{n}
\rput(10.90422,-2.0153935){$d$}
\end{pspicture}    
}
\end{center}




\begin{center}
 \scalebox{0.85} % Change this value to rescale the drawing.
{
\begin{pspicture}(0,-3.8416457)(11.207031,3.8216705)
\psline[linewidth=0.04](2.3197074,3.8016455)(1.0997375,1.3929795)(4.668132,-0.4143837)(2.3197074,3.8016455)
\psline[linewidth=0.04cm](1.2352896,1.6606089)(1.5029193,1.5250567)
\psline[linewidth=0.04cm](1.5029193,1.5250567)(1.3673671,1.2574271)
% \usefont{T1}{ptm}{m}{n}
\rput(0.11234375,3.5211668){\textbf{5.}}
% \usefont{T1}{ptm}{m}{n}
\rput(6.084844,3.5211668){\textbf{6.}}
% \usefont{T1}{ptm}{m}{n}
\rput(0.14375,-0.39883313){\textbf{7.}}
% \usefont{T1}{ptm}{m}{n}
\rput(6.1403127,-0.498833){\textbf{8.}}
\psline[linewidth=0.04](1.0870312,-0.6888331)(1.0870312,-3.3888335)(5.0870314,-3.3888335)(1.0870312,-0.6888331)
\psline[linewidth=0.04cm](1.0870312,-3.0888333)(1.3870313,-3.0888333)
\psline[linewidth=0.04cm](1.3870313,-3.0888333)(1.3870313,-3.3888335)
\psline[linewidth=0.04](10.705838,-3.6336784)(10.738021,-0.93386954)(6.738305,-0.88619083)(10.705838,-3.6336784)
\psline[linewidth=0.04cm](10.734446,-1.2338486)(10.434467,-1.2302725)
\psline[linewidth=0.04cm](10.434467,-1.2302725)(10.438043,-0.9302939)
\psline[linewidth=0.04](11.200404,0.87534267)(9.325028,2.8177538)(6.4473815,0.039416883)(11.200404,0.87534267)
\psline[linewidth=0.04cm](9.533403,2.6019304)(9.31758,2.393555)
\psline[linewidth=0.04cm](9.31758,2.393555)(9.109204,2.6093786)
\usefont{T1}{ptm}{m}{n}
\rput(3.5887504,1.9261669){$e$}
\usefont{T1}{ptm}{m}{n}
\rput(3.8853126,0.3161668){$17^\circ$}
\usefont{T1}{ptm}{m}{n}
\rput(1.3671876,2.6761668){ $12$}
\usefont{T1}{ptm}{m}{n}
\rput(7.533282,0.51616687){$22^\circ$}
\usefont{T1}{ptm}{m}{n}
\rput(9.139219,0.2861668){$f$}
\usefont{T1}{ptm}{m}{n}
\rput(7.758125,1.7361668){ $31$}
\usefont{T1}{ptm}{m}{n}
\rput(3.028594,-1.683833){$32$}
\usefont{T1}{ptm}{m}{n}
\rput(4.1132817,-3.123833){$23^\circ$}
\usefont{T1}{ptm}{m}{n}
\rput(2.949844,-3.6138332){$g$}
\usefont{T1}{ptm}{m}{n}
\rput(10.928438,-2.2538335){$h$}
\usefont{T1}{ptm}{m}{n}
\rput(7.7667193,-1.2438331){ $30^\circ$}
\usefont{T1}{ptm}{m}{n}
\rput(8.245001,-2.3038335){$20$}
\end{pspicture} 
}
   
\end{center}  
\vspace{10pt}
\item Write down two ratios for each of the following in terms of the sides: $AB$; $BC$; $BD$; $AD$; $DC$ and $AC$\\
\begin{center}
\scalebox{1} % Change this value to rescale the drawing.
{
\begin{pspicture}(0,-1.2515883)(4.02125,1.2515885)
\psline[linewidth=0.04,fillstyle=solid](0.019002499,-1.2315885)(0.009700612,1.2315885)(4.00125,-1.1902716)(0.039492033,-1.2120903)(0.039492033,-1.2120903)(0.0,-1.1911051)
\psline[linewidth=0.04](0.009700612,-0.9884115)(0.20970061,-0.9884115)(0.20970061,-1.2284116)
\psline[linewidth=0.04,fillstyle=solid](1.3697007,0.4315885)(0.049700614,-1.1684115)(0.049700614,-1.1484115)
\psline[linewidth=0.04](1.2297006,0.2715885)(1.0697006,0.3515885)(1.1897006,0.5115885)
\usefont{T1}{ptm}{m}{n}
\rput(0,-1.5){$A$}
\rput(0, 1.5){$B$}
\rput(1.4, 0.7){$C$}
\rput(4, -1.5){$D$}
\end{pspicture} 
}
\end{center}
    \begin{enumerate}[noitemsep, label=\textbf{\arabic*}. ] 
    \item $sin~\hat{B}$
    \item $cos~\hat{B}$
    \item $tan~\hat{B}$
    \end{enumerate}
\vspace{10pt}
\item In $\triangle MNP$, $\hat{N}=90^{\circ}$, $MP=20$ and $\hat{P}=40^{\circ}$. Calculate $NP$ and $MN$ correct to $2$ decimal places.
\end{enumerate}

\par 
\label{m39408*eip-403}
\par \raisebox{-5 pt}{\includegraphics[width=0.5cm]{col11306.imgs/summary_www.png}} Find the answers with the shortcodes:
\par \begin{tabular}[h]{cccccc}
(1.) lc1  & \end{tabular}
}
\end{exercises}

\section{Finding an angle}
 
\begin{wex}{Finding Angles}
{
Find the value of $\theta$ in the following triangle: \\
\begin{center}
\scalebox{1}  
{ 
\begin{pspicture}(0,-1.3465625)(4.9834375,2.02) 
\psline[linewidth=0.04](4.0,-1.0)(1.0,-1.0)(4.0,2.0)(4.0,-1.0)(4.0,-1.0) 
\usefont{T1}{ptm}{m}{n} 
\rput(2.7546875,-1.19){$100$} 
\usefont{T1}{ptm}{m}{n} 
\rput(4.549844,0.61){$50$} 
\usefont{T1}{ptm}{m}{n} 
\rput(1.6235938,-0.69){$\theta$} 
\psarc[linewidth=0.04](1.0,-1.0){1.0}{0.0}{44.06081} 
\end{pspicture} 
}
\end{center} 
}
{

\westep{Identify the opposite and adjacent sides and the hypotenuse}
In this case you have the opposite side and the hypotenuse to the angle $\theta$. \\

\begin{equation*}
\begin{array}{ccl}
 
\hfill tan~\theta &=& \dfrac{\mbox{opposite}}{\mbox{adjacent}}  \hfill \vspace{5pt}\\
\hfill tann~ \theta = \dfrac{50}{100} \hfill \\
\end{array}
\end{equation*}

\westep{Use your calculator to solve for $\theta$ }
To solve for $\theta$, you will need to use the inverse function on your calculator: \vspace{10pt}
\\
Press \fbox{2ndF} \fbox{TAN} \fbox{(} \fbox{50} \fbox{\div} \fbox{100} \fbox{)} \fbox{\LARGE =} $26,6$
\westep{Write the final answer}
\begin{equation*}
\theta = 26,6^{\circ}
\end{equation*}

}
\end{wex}

\begin{exercises}{}
{
Determine the angle (correct to $1$ decimal place):
   \begin{enumerate}[noitemsep, label=\textbf{\arabic*}. ] 
\begin{multicols}{2}
 \item $tan~\theta = 1,7$
\item $sin~\theta = 0,8$
\item $cos~\alpha = 0,32$
\item $tan~\theta = 5\frac{3}{4}$
\item $sin~\theta = \frac{2}{3}$
\item $cos~\gamma = 1,2$
\item $4~cos~\theta = 3$
\item $cos~4\theta = 0,3$
\item $sin~\beta + 2= 2,65$
\item $sin~\theta = 0,8$
\item $3 ~tan~\beta = 1$
\item $sin~3\alpha = 1,2$
\item $tan~(\frac{\theta}{3}) = sin~48^{\circ}$
\item $\frac{1}{2}~cos~2\beta = 0,3$
\item $2~sin~3\theta +1= 2,6$
\end{multicols}
\end{enumerate}

Determine $\alpha$ in the following right-angled triangles:
\begin{center}
\scalebox{1} % Change this value to rescale the drawing.
{
\begin{pspicture}(0,-4.032389)(7.3778124,4.17849)

\rput(0, 3.5){\textbf{16.}}
\rput(3.5, 3.5){\textbf{17.}}
\rput(0, 0.5){\textbf{18.}}
\rput(3.5, 0.5){\textbf{19.}}
\rput(0, -2.0){\textbf{20.}}
\rput(3.5, -2.0){\textbf{21.}}
\psline[linewidth=0.04,fillstyle=solid](4.128174,1.9359901)(4.129576,3.691884)(6.1603127,1.9559901)(4.142015,1.9498184)(4.142015,1.9498184)(4.115519,1.9649143)
\psline[linewidth=0.04,fillstyle=solid](0.5403125,3.7206802)(2.3103564,3.7223134)(0.5605087,1.1162983)(0.5542749,3.7072453)(0.5542749,3.7072453)(0.5694489,3.7330344)
\psline[linewidth=0.04](0.8203125,3.71599)(0.8203125,3.41599)(0.5403125,3.41599)
\psline[linewidth=0.04](4.132775,2.2306225)(4.429114,2.2301245)(4.4286456,1.9555349)
\psline[linewidth=0.04,fillstyle=solid](0.58817375,-1.2040099)(0.5895763,0.5518841)(2.6203125,-1.1840099)(0.60201496,-1.1901817)(0.60201496,-1.1901817)(0.57551914,-1.1750857)
\psline[linewidth=0.04](0.59277505,-0.90937746)(0.889114,-0.9098756)(0.88864577,-1.184465)
\psline[linewidth=0.04,fillstyle=solid](0.57534015,-1.9803641)(2.331222,-1.9737015)(0.6046739,-4.0123897)(0.5892318,-1.9941416)(0.5892318,-1.9941416)(0.6042059,-1.9675767)
\psline[linewidth=0.04](0.8699906,-1.983612)(0.8708536,-2.2799501)(0.5962649,-2.2807431)
\psline[linewidth=0.04,fillstyle=solid](6.6596694,0.2178756)(6.6803126,-0.6840099)(3.9803126,0.21599011)(6.645639,0.20423959)(6.645639,0.20423959)(6.6719236,0.18877967)
\psline[linewidth=0.04](6.6710052,-0.076665334)(6.3803124,-0.06400988)(6.3789563,0.20247632)
\psline[linewidth=0.04,fillstyle=solid](5.604427,-1.7804683)(6.7567835,-3.1053228)(4.0853443,-3.1304228)(5.603074,-1.7999867)(5.603074,-1.7999867)(5.632968,-1.7939644)
\psline[linewidth=0.04](5.794496,-2.005642)(5.570731,-2.199926)(5.3907113,-1.9925796)
\rput{-34.695152}(-0.9369147,0.7926677){\psarc[linewidth=0.024](0.8003125,1.8959901){0.3}{39.289406}{180.0}}
\usefont{T1}{ptm}{m}{n}
\rput(1.4359375,3.9229398){$4$}
\usefont{T1}{ptm}{m}{n}
\rput(0.29078126,2.4629397){$9$}
\usefont{T1}{ptm}{m}{n}
\rput(5.470625,3.02294){$13$}
\usefont{T1}{ptm}{m}{n}
\rput(5.034219,1.5829399){$7,5$}
\usefont{T1}{ptm}{m}{n}
\rput(1.75375,-0.057060193){$2,2$}
\usefont{T1}{ptm}{m}{n}
\rput(0.2096875,-0.4770602){$1,7$}
\usefont{T1}{ptm}{m}{n}
\rput(5.716875,0.4829398){$9,1$}
\usefont{T1}{ptm}{m}{n}
\rput(7.0603123,-0.3170602){$4,5$}
\usefont{T1}{ptm}{m}{n}
\rput(0.26703125,-2.8770602){$12$}
\usefont{T1}{ptm}{m}{n}
\rput(1.9092188,-3.1370602){$15$}
\usefont{T1}{ptm}{m}{n}
\rput(4.5784373,-2.3170602){$1$}
\usefont{T1}{ptm}{m}{n}
\rput(5.54375,-3.47706){$\sqrt{2}$}
\usefont{T1}{ptm}{m}{n}
\rput(6.2634373,-2.93706){$\alpha$}
\usefont{T1}{ptm}{m}{n}
\rput(0.7834375,-3.4970603){$\alpha$}
\usefont{T1}{ptm}{m}{n}
\rput(1.9234375,-0.9970602){$\alpha$}
\usefont{T1}{ptm}{m}{n}
\rput(5.1634374,-0.017060194){$\alpha$}
\usefont{T1}{ptm}{m}{n}
\rput(4.3034377,3.1429398){$\alpha$}
\usefont{T1}{ptm}{m}{n}
\rput(0.7634375,1.8029398){$\alpha$}
\rput{-162.53017}(7.4652467,7.66313){\psarc[linewidth=0.024](4.3213253,3.258065){0.34310362}{39.289406}{156.30495}}
\rput{-89.91282}(5.095528,5.0594177){\psarc[linewidth=0.024](5.081325,-0.02193504){0.34310362}{53.61859}{134.10115}}
\rput{-319.68652}(-0.17349519,-1.5915118){\psarc[linewidth=0.024](2.0811038,-1.0320795){0.42839816}{67.16096}{156.30495}}
\rput{-30.447884}(1.9160283,-0.06363264){\psarc[linewidth=0.024](0.8411037,-3.5520794){0.42839816}{67.16096}{156.30495}}
\rput{-319.68652}(-0.41087502,-4.8648257){\psarc[linewidth=0.024](6.4211035,-2.9920795){0.42839816}{67.16096}{156.30495}}
\end{pspicture} 

}
\end{center}
\end{exercises}


\section{Defining ratios in the Cartesian plane}

We have defined the trigonometric functions using right-angled triangles. We can extend these definitions to any angle, noting that the definitions do not rely on the lengths of the sides of the triangle, but only on the size of the angle. So if we plot any point on the Cartesian plane and then draw a line from the origin to that point, we can work out the angle of that line. Inthe figure below points $P$ and $Q$ have been plotted. A line from the origin ($O$) to each point is drawn. The dotted lines show how we can construct right angle triangles for each point. Now we can find the angles $A$ and $B$.We can also extend the definitions of the reciprocals in the same way:
\label{m39411*id89342}\nopagebreak\noindent{}

\setcounter{subfigure}{0}
\begin{center}
\scalebox{1} % Change this value to rescale the drawing.
{
\begin{pspicture}(0,-4.0784373)(9.84,4.1184373)
\rput(4.0,-0.0784375){\psaxes[linewidth=0.04,arrowsize=0.05291667cm 2.0,arrowlength=1.4,arrowinset=0.4,ticksize=0.004cm]{<->}(0,0)(-4,-4)(4,4)}
\psline[linewidth=0.04cm,dotsize=0.07055555cm 2.0]{-*}(4.04,-0.0584375)(6.94,1.8615625)
\psline[linewidth=0.04cm,dotsize=0.07055555cm 2.0]{-*}(4.0,-0.0784375)(1.02,1.8815625)
\psline[linewidth=0.04cm,linestyle=dashed,dash=0.16cm 0.16cm](6.96,1.8615625)(6.96,-0.0984375)
\psline[linewidth=0.04cm,linestyle=dashed,dash=0.16cm 0.16cm](1.02,1.8015625)(1.02,-0.0584375)
\psline[linewidth=0.04,fillstyle=solid](9.82,-1.0984375)(9.82,-1.0984375)(9.82,-1.0984375)(9.82,-1.0984375)
\psline[linewidth=0.04,fillstyle=solid](1.02,0.2015625)(1.28,0.2015625)(1.28,-0.0384375)(1.28,-0.0584375)
\psline[linewidth=0.04,fillstyle=solid](6.7,-0.0584375)(6.7,0.1815625)(6.96,0.1815625)(6.96,0.2015625)
\rput{-279.77182}(2.8218253,-2.9484913){\psarc[linewidth=0.04](3.160762,0.20043099){0.48562816}{40.75301}{136.25127}}
\rput{-56.56522}(1.9601717,3.1761396){\psarc[linewidth=0.04](3.9315972,-0.23347138){0.83129716}{66.93879}{193.12923}}
\rput{-72.17863}(3.2203913,4.738851){\psarc[linewidth=0.04](4.860762,0.160431){0.48562816}{44.75812}{136.25127}}
\usefont{T1}{ptm}{m}{n}
\rput(5.0403123,0.1915625){$A$}
\usefont{T1}{ptm}{m}{n}
\rput(4.2,-0.3){$O$}
\usefont{T1}{ptm}{m}{n}
\rput(4.273125,0.7915625){$B$}
\usefont{T1}{ptm}{m}{n}
\rput(2.9954689,0.2115625){$x$}
\usefont{T1}{ptm}{m}{n}
\rput(7.2570314,2.2315626){$P(2;3)$}
\usefont{T1}{ptm}{m}{n}
\rput(0.9845312,2.1915624){$Q(-2;3)$}
\usefont{T1}{ptm}{m}{n}
\rput(8.235469,0.1315625){$x$}
\usefont{T1}{ptm}{m}{n}
\rput(4.275781,3.9915626){$y$}
\end{pspicture} 
} 
\end{center}
From the coordinates of $P(2;3)$, we know the length of the side opposite $\hat{A}$ is $3$ and the length of the adjacent side is $2$. Using $tan~\hat{A}=\frac{\mbox{\footnotesize opposite}}{\mbox{\footnotesize adjacent}} = \frac{3}{2}$ we calculate that $\hat{A}=56,3^{\circ}$.\par

We can also use the Theorem of Pythagoras to calculate the hypotenuse of the triangle and then calculate $\hat{A}$ using $sin~\hat{A} = \frac{\mbox{\footnotesize opposite}}{\mbox{\footnotesize hypotenuse}}$ or $cos~\hat{A} = \frac{\mbox{\footnotesize adjacent}}{\mbox{\footnotesize hypotenuse}}$. \par
Consider point $Q(-2;3)$. We define $\hat{B}$ as the angle formed between the line $OQ$ and the positive $x$-axis. This is called the standard position of an angle. Let $\hat{x}$ be the angle formed between the line $OQ$ and the negative $x$-axis such that $\hat{B} + \hat{x} = 180^{\circ}$.
\par

From the coordinates of $Q(-2;3)$, we know the length of the side opposite $\hat{x}$ is $3$ and the length of the adjacent side is $2$. Using $tan~\hat{x}=\frac{\mbox{\footnotesize opposite}}{\mbox{\footnotesize adjacent}} = \frac{3}{2}$ we calculate that $\hat{x}=56,3^{\circ}$.
\\Therefore $\hat{B}=180^{\circ} - \hat{x} = 123,7^{\circ}$.\par

Similarly, an alternative method is to calculate the hypotenuse using the Theorem of Pythagoras and calculate $\hat{x}$ using $sin~\hat{x} = \frac{\mbox{\footnotesize opposite}}{\mbox{\footnotesize hypotenuse}}$ or $cos~\hat{x} = \frac{\mbox{\footnotesize adjacent}}{\mbox{\footnotesize hypotenuse}}$. \par
If we were to draw a circle centered on the origin, then the length from the origin to point $P$ is the radius of the circle, which we denote $r$. We can rewrite all the trigonometric functions in terms of $x$, $y$ and $r$.


% You should find the angle $A$ is $63,{43}^{\circ }$. For angle B, you first work out x ($33,{69}^{\circ }$) and then B is ${180}^{\circ }-33,{69}^{\circ }=146,{31}^{\circ }$. But what if we wanted to do this without working out these angles and figuring out whether to add or subtract 180 or 90? Can we use the trig functions to do this? Consider point P in Figure~14.15. To find the angle you would have used one of the trig functions, e.g. $\mathrm{tan}\phantom{\rule{1pt}{0ex}}\theta $. You should also have noted that the side adjacent to the angle was just the x-co-ordinate and that the side opposite the angle was just the y-co-ordinate. But what about the hypotenuse? Well, you can find that using Pythagoras since you have two sides of a right angled triangle. If we were to draw a circle centered on the origin, then the length from the origin to point P is the radius of the circle, which we denote r. Now we can rewrite all our trig functions in terms of x, y and r. But how does this help us to find angle B? Well, we know that from point Q to the origin is r, and we have the co-ordinates of Q. So we simply use the newly defined trig functions to find angle B! (Try it for yourself and confirm that you get the same answer as before.) One final point to note is that when we go anti-clockwise around the Cartesian plane the angles are positive and when we go clockwise around the Cartesian plane, the angles are negative. \par \label{m39411*eip-634}
% \par

The general definitions for the trigonometric functions are:

\begin{equation*}
\begin{array}{cccccc}\hfill sin~\theta & =& \frac{y}{r}\hfill & cosec~\theta & =& \frac{r}{y}\hfill \\
 \hfill cos~\theta & =& \frac{x}{r}\hfill & sec~\theta & =& \frac{r}{x}\hfill \\
 \hfill tan~\theta & =& \frac{y}{x}\hfill & cot~\theta & =& \frac{x}{y}\hfill \end{array}
\end{equation*}
\Note{In trigonometry angles are always measured from the positive $x$-axis in an anti-clockwise direction.}

The Cartesian plane is divided into $4$ quadrants in an anti-clockwise direction as shown in the diagram below. The letters $C$, $A$, $S$ and $T$ indicate which of the trigonometric functions are positive in that quadrant: \\
\\
Quadrant I: All ratios are positive\\
Quadrant II: $y$ values are positive therefore $sin$ and $cosec$ are positive\\
Quadrant III: Both $x$ and $y$ values are negative therefore $tan$ and $cot$ are positive \\
Quadrant IV: $x$ values are positive therefore $cos$ and $sec$ are positive.\par


\Note{$r$ is a length and therefore is always positive.}
\begin{center}
\scalebox{1} % Change this value to rescale the drawing.
{
\begin{pspicture}(0,-4.1584377)(8.42125,4.1984377)
\rput(4.0,-0.1584375){\psaxes[linewidth=0.04,arrowsize=0.05291667cm 2.0,arrowlength=1.4,arrowinset=0.4,labels=none,ticks=none,ticksize=0.10583333cm]{<->}(0,0)(-4,-4)(4,4)}
% \usefont{T1}{ptm}{m}{n}
\rput(6.0853124,2.2715626){Quadrant I}
% \usefont{T1}{ptm}{m}{n}
\rput(1.7453125,2.3115625){Quadrant II}
% \usefont{T1}{ptm}{m}{n}
\rput(1.8053125,-2.3084376){Quadrant III}
% \usefont{T1}{ptm}{m}{n}
\rput(6.3515625,-2.2684374){Quadrant IV}
% \usefont{T1}{ptm}{m}{n}
\rput(4.7403126,0.8315625){A }
% \usefont{T1}{ptm}{m}{n}
\rput(4.7723436,0.4315625){all}
% \usefont{T1}{ptm}{m}{n}
\rput(3.2923439,0.8515625){S}
% \usefont{T1}{ptm}{m}{n}
\rput(3.2390625,0.4715625){$sin$}
% \usefont{T1}{ptm}{m}{n}
\rput(3.2109375,-0.8484375){T}
% \usefont{T1}{ptm}{m}{n}
\rput(3.1639063,-1.1684375){$tan$}
% \usefont{T1}{ptm}{m}{n}
\rput(4.7857814,-0.8484375){C}
% \usefont{T1}{ptm}{m}{n}
\rput(4.815,-1.1484375){$cos$}
\usefont{T1}{ptm}{m}{n}
\rput(8.275469,-0.1084375){$x$}
\usefont{T1}{ptm}{m}{n}
\rput(4.1557813,4.0715623){$y$}
\usefont{T1}{ptm}{m}{n}
\rput(7.810625,0.0715625){$0^{\circ}$}
\usefont{T1}{ptm}{m}{n}
\rput(7.60875,-0.4484375){$360^{\circ}$}
\usefont{T1}{ptm}{m}{n}
\rput(4.3410935,3.5515625){$90^{\circ}$}
\usefont{T1}{ptm}{m}{n}
\rput(0.32265624,0.0715625){$180^{\circ}$}
\usefont{T1}{ptm}{m}{n}
\rput(3.529375,-3.9084375){$270^{\circ}$}

\usefont{T1}{ptm}{m}{n}
\rput(4.170625,-0.4084375){$0$}
\end{pspicture} 
}
\end{center}
\\
 This diagram is known as the CAST diagram.


\begin{wex}{Ratios in the Cartesian plane}
{$P(-3;4)$ is a point in the Cartesian plane. $X\hat{O}P=\theta$. Without using a calculator, determine the value of:
  \begin{enumerate}[noitemsep, label=\textbf{\arabic*}. ] 
   \item $cos~\theta$
\item $3~tan~\theta$
\item $\frac{1}{2}~cosec~\theta$
  \end{enumerate}

}

{
\westep{Sketch point $P$ in the Cartesian plane and label $\theta$} 
\begin{center}
\scalebox{1} % Change this value to rescale the drawing.
{
\begin{pspicture}(0,-3.1584375)(6.44125,3.1984375)
\rput(3.0,-0.1584375){\psaxes[linewidth=0.04,arrowsize=0.05291667cm 2.0,arrowlength=1.4,arrowinset=0.4,labels=none,ticks=none,ticksize=0.10583333cm]{<->}(0,0)(-3,-3)(3,3)}
\usefont{T1}{ptm}{m}{n}
\rput(6.295469,0.0115625){$x$}
\usefont{T1}{ptm}{m}{n}
\rput(3.1757812,3.0715625){$y$}
\psline[linewidth=0.04cm,dotsize=0.07055555cm 2.0]{-*}(3.02,-0.1784375)(1.6,1.8615625)
\usefont{T1}{ptm}{m}{n}
\rput(1.3670312,2.2715626){$P(-3;4)$}
\rput{-45.048485}(0.95434314,2.1533978){\psarc[linewidth=0.04](3.073445,-0.073917985){0.6961048}{36.728504}{180.0}}
\usefont{T1}{ptm}{m}{n}
\rput(3.150625,-0.4484375){$0$}
\usefont{T1}{ptm}{m}{n}
\rput(3.2709374,0.1915625){$\theta$}
\end{pspicture} 
}
\end{center}

\westep{Use Pythagoras to calculate $r$}
\begin{equation*}
 \begin{array}{ccl}
    \hfill r^{2} &= & x^{2} + y^{2} \hfill \\
\hfill  &=& (-3)^{2} + (4)^{2} \hfill \\
\hfill  &=& 25 \hfill \\
\hfill r &=& \pm 5\hfill \\
\hfill \therefore r &=& 5 \hfill \\ 
\end{array}
\end{equation*}
\westep{Substitute values for $x$, $y$ and $r$ into the required ratios}
\begin{enumerate}[noitemsep, label=\textbf{\arabic*}. ] 
   \item $cos~\theta = \dfrac{x}{r} = -\dfrac{3}{5} \vspace{5pt}$
\item $3~tan~\theta = 3\Big(\dfrac{y}{x}\Big) = 3\Big(\dfrac{4}{-3}\Big) = -4 \vspace{5pt}$
\item $\dfrac{1}{2}~cosec~\theta = \dfrac{1}{2}\Big(\dfrac{r}{y}\Big) = \dfrac{1}{2}\Big(\dfrac{5}{4}\Big) = \dfrac{5}{8} \vspace{5pt}$
  \end{enumerate}
}
\end{wex}


\begin{wex}{Ratios in the Cartesian plane}{
$X\hat{O}K = \theta$ is an angle in the third quadrant and $K$ is the point $(-5;y)$. $OK$ is $13$ units. Determine without using a calculator:
  \begin{enumerate}[noitemsep, label=\textbf{\arabic*}. ] 
   \item The value of $y$
\item Prove that $tan^{2}~\theta + 1 = sec^{2}~\theta$
  \end{enumerate}
}

{
\westep{Sketch point $K$ in the Cartesian plane and label $\theta$} 
\begin{center}
\scalebox{1} % Change this value to rescale the drawing.
{
\begin{pspicture}(0,-3.1584375)(6.44125,3.1984375)
\rput(3.0,-0.1584375){\psaxes[linewidth=0.04,arrowsize=0.05291667cm 2.0,arrowlength=1.4,arrowinset=0.4,labels=none,ticks=none,ticksize=0.10583333cm]{<->}(0,0)(-3,-3)(3,3)}
\usefont{T1}{ptm}{m}{n}
\rput(6.295469,0.0115625){$x$}
\usefont{T1}{ptm}{m}{n}
\rput(3.1757812,3.0715625){$y$}
\psline[linewidth=0.04cm,dotsize=0.07055555cm 2.0]{-*}(2.963227,-0.18549357)(1.6,-2.1384375)
\usefont{T1}{ptm}{m}{n}
\rput(1.2170312,-2.5084374){$K(-5;y)$}
\rput{-45.048485}(1.0897402,2.0403726){\psarc[linewidth=0.04](3.0048735,-0.2936738){0.7345531}{56.389397}{272.69913}}
\usefont{T1}{ptm}{m}{n}
\rput(3.150625,-0.4484375){$0$}
\usefont{T1}{ptm}{m}{n}
\rput(2.4109375,0.5115625){$\theta$}
\usefont{T1}{ptm}{m}{n}
\rput(1.9303125,-1.1484375){$13$}
\end{pspicture} 
}
\end{center}
\westep{Use Pythagoras to calculate $y$}
\begin{equation*}
 \begin{array}{ccl}
    \hfill r^{2} &= & x^{2} + y^{2} \hfill \\
\hfill y^{2} &=& r^{2} - x^{2}\hfill \\
\hfill  &=& (13)^{2} - (-5)^{2} \hfill \\
\hfill  &=& 169-25 \hfill \\
\hfill  &=& 144 \hfill \\
\hfill  &=& \pm 12\hfill 

\end{array}
\end{equation*}
Given that $\theta$ lies in the third quadrant, $y$ must be negative.\\
\begin{equation*}
 \therefore y = -12
\end{equation*}

\westep{Substitute values for $x$, $y$ and $r$ and simplify}
\begin{equation*}
\begin{array}{ccl}
 \hfill tan^{2}~\theta + 1 &=& sec^{2}~\theta \hfill \\
\hfill \Big(\dfrac{x}{y}\Big)^{2} + 1 &=& \Big(\dfrac{r}{x}\Big)^{2} \hfill \vspace{5pt} \\
\hfill \Big(\dfrac{-12}{-5}\Big)^{2} + 1 &=& \Big(\dfrac{13}{-5}\Big)^{2} \hfill \vspace{5pt}\\
\hfill \Big (\dfrac{144}{25}\Big) + 1 &=& \Big(\dfrac{169}{25}\Big)^{2} \hfill \vspace{5pt}\\
\hfill \dfrac{144 + 25}{25} &=& \dfrac{169}{25} \hfill \vspace{5pt}\\
\hfill \dfrac{169}{25} &=& \dfrac{169}{25} \hfill \vspace{5pt}\\
\hfill \therefore \mbox{LHS} &=& \mbox{RHS} \hfill

\end{array}
\end{equation*}

}
\end{wex}

\Tip{Whenever you have to solve trigonometric problems without a calculator, always make a sketch.}

\begin{exercises}{}
{
$B$ is a point in the Cartesian plane. Determine without using a calculator:
\begin{center}
\scalebox{1} % Change this value to rescale the drawing.
{
\begin{pspicture}(0,-3.1584375)(6.44125,3.1984375)
\rput(3.0,-0.1584375){\psaxes[linewidth=0.04,arrowsize=0.05291667cm 2.0,arrowlength=1.4,arrowinset=0.4,labels=none,ticks=none,ticksize=0.10583333cm]{<->}(0,0)(-3,-3)(3,3)}
\usefont{T1}{ppl}{m}{n}
\rput(6.295469,0.0115625){$x$}
\usefont{T1}{ppl}{m}{n}
\rput(3.1757812,3.0715625){$y$}
\rput(3.4, -0.4){$O$}
\psline[linewidth=0.04cm,fillcolor=black,dotsize=0.07055555cm 2.0]{-*}(3.003227,-0.16549358)(3.78,-1.8384376)
\usefont{T1}{ppl}{m}{n}
\rput(4.577031,-1.9084375){$B(1;-3)$}
\rput{-45.048485}(1.0590175,2.0862904){\psarc[linewidth=0.04](3.0448735,-0.2336738){0.7345531}{50.714928}{340.5804}}
\usefont{T1}{ppl}{m}{n}
\rput(2.393125,0.5115625){$\beta$}
\end{pspicture} 
}
\end{center}
\begin{enumerate}[noitemsep, label=\textbf{\arabic*}. ] 
 \item $OB$
\item $cos~\beta$
\item $cosec~\beta$
\item $tan~\beta$
\end{enumerate}

\vspace{10pt}
If $sin~\theta= 0,4$ and $\theta$ is an obtuse angle, determine:
\begin{center}
\scalebox{1} % Change this value to rescale the drawing.
{
\begin{pspicture}(0,-3.1584375)(6.44125,3.1984375)
\rput(3.0,-0.1584375){\psaxes[linewidth=0.04,arrowsize=0.05291667cm 2.0,arrowlength=1.4,arrowinset=0.4,labels=none,ticks=none,ticksize=0.10583333cm]{<->}(0,0)(-3,-3)(3,3)}
\usefont{T1}{ppl}{m}{n}
\rput(6.295469,0.0115625){$x$}
\usefont{T1}{ppl}{m}{n}
\rput(3.1757812,3.0715625){$y$}
\psline[linewidth=0.04cm,fillcolor=black](3.0,-0.1784375)(0.8,0.7415625)
\usefont{T1}{ppl}{m}{n}
\rput(0.7292187,-0.4884375){$(x;2)$}
\rput{-45.048485}(1.0440966,2.0498228){\psarc[linewidth=0.04](2.9934452,-0.23391798){0.6961048}{50.894093}{199.87982}}
\usefont{T1}{ppl}{m}{n}
\rput(3.150625,-0.4484375){$0$}
\usefont{T1}{ppl}{m}{n}
\rput(3.6509376,0.5915625){$\theta$}
\psline[linewidth=0.04cm,linestyle=dashed,dash=0.16cm 0.16cm](0.8,0.7215625)(0.8,-0.1784375)
\usefont{T1}{ptm}{m}{n}
\rput(0.5234375,0.2315625){$2$}
\end{pspicture} 
}
\end{center}
\begin{enumerate}[noitemsep, label=\textbf{\arabic*}. ] 
 \item $cos~\theta$
\item $\sqrt{21}~tan~\theta$

\end{enumerate}
}
\end{exercises}

\section{Two-dimensional problems}
Trigonometry was developed in ancient civilisations to solve practical problems such as building construction and navigating by the stars. In this section we will show how trigonometry can be used to solve some other practical problems. We use the trig functions to solve problems in two dimensions that involve right angled triangles. 

\begin{wex}{Flying a kite}
{Mandla flies a kite on a $17~$m string at an inclination of $63^{\circ}$.
\begin{enumerate}[noitemsep, label=\textbf{\arabic*}. ] 
 \item What is the height ($h$) of the kite above Mandla's head?
\item If Mandla's friend Sipho stands directly below the kite, calculate the distance ($d$) between the two friends. 
\end{enumerate}
}
{
\westep{Make a sketch and identify the opposite and adjacent sides and the hypotenuse}
\begin{center}
\scalebox{1} % Change this value to rescale the drawing.
{
\begin{pspicture}(0,-1.9425)(4.3575,1.9125)
\psline[linewidth=0.025999999](3.98,1.3625)(3.98,1.4425)(3.96,-1.4975)(0.3,-1.4975)(3.98,1.4225)
\usefont{T1}{ptm}{m}{n}
\rput(4.202344,-0.1475){$h$}
\usefont{T1}{ptm}{m}{n}
\rput(2.3079689,-1.7875){$d$}
\usefont{T1}{ptm}{m}{n}
\rput(1.0176562,-1.2675){$63^{\circ}$}
\pscircle[linewidth=0.02,dimen=outer](4.18,-1.1911944){0.1}
\psline[linewidth=0.02cm](4.18,-1.2911946)(4.18,-1.6775)
\psline[linewidth=0.02cm](4.04,-1.8375)(4.18,-1.6775)
\psline[linewidth=0.02cm](4.18,-1.6720197)(4.3,-1.8375)
\psline[linewidth=0.02cm](4.06,-1.4599631)(4.3,-1.4599631)
\usefont{T1}{ptm}{m}{n}
\rput(2.249375,0.3925){$17$}
\psline[linewidth=0.02](3.9,1.7625)(4.06,1.9025)(4.16,1.7358333)(3.9866667,1.4425)(3.9173334,1.7521297)(4.16,1.7358333)
\psline[linewidth=0.02cm](3.98,1.4625)(4.06,1.8825)
\pscircle[linewidth=0.02,dimen=outer](0.14,-1.2311945){0.1}
\psline[linewidth=0.02cm](0.14,-1.3311945)(0.14,-1.7175)
\psline[linewidth=0.02cm](0.0,-1.8775)(0.14,-1.7175)
\psline[linewidth=0.02cm](0.14,-1.7120197)(0.26,-1.8775)
\psline[linewidth=0.02cm](0.02,-1.499963)(0.3,-1.4975)
\end{pspicture} 
} 
\end{center}

\westep{Use given information to determine appropriate ratio}
\begin{enumerate}[noitemsep, label=\textbf{\arabic*}. ] 
\item
\begin{equation*}
 \begin{array}{ccl}
\hfill sin~63^{\circ} &= & \frac{\mbox{\footnotesize opposite}}{\mbox{\footnotesize hypotenuse}} \hfill \vspace{5pt}\\
\hfill sin~63^{\circ} &=& \dfrac{h}{17} \hfill \\
\hfill \therefore h &= & 17~sin ~63^{\circ} \hfill \\
\hfill &=& 15,15~\mbox{m} \hfill \\
   \end{array}
\end{equation*}

\item
\begin{equation*}
 \begin{array}{ccl}
\hfill cos~63^{\circ} &= & \frac{\mbox{\footnotesize adjacent}}{\mbox{\footnotesize hypotenuse}} \hfill \vspace{5pt}\\
\hfill cos~63^{\circ} &=& \dfrac{d}{17} \hfill \\
\hfill \therefore d &= & 17~cos ~63^{\circ} \hfill \\
\hfill  &=& 7,72~\mbox{m} \hfill \\
\end{array}
\end{equation*}
\end{enumerate}
Note that the third side of the triangle can also be calculated using the Theorem of Pythagoras: $d^{2} = 17^{2} - h^{2}$.

\westep{Write final answer}
\begin{enumerate}[noitemsep, label=\textbf{\arabic*}. ] 
\item The kite is $15,15$ m above Mandla's head.
\item Mandla and Sipho are $7,72$ m apart.
\end{enumerate}
}
 
\end{wex}

\begin{wex}
{Calculating angles}
{
ABCD be a trapezium with $AB=4\mbox{cm}$, $CD=6\mbox{cm}$, $BC=5\mbox{cm}$ and $AD=5\mbox{cm}$. Point $E$ on diagonal $AC$ divides the diagonal such that $AE=3\mbox{cm}$. Find $A\hat{B}C$.
}
{
\westep{Draw trapezium and label all given lengths on diagram. Indicate that $\hat{E}$ is a right-angle.}
\begin{center}
\scalebox{1} % Change this value to rescale the drawing.
{
\begin{pspicture}(0,-2.8215597)(7.3365626,2.6459403)
\psline[linewidth=0.04](7.0446873,-2.3765597)(0.2846875,-2.3565598)(1.9846874,2.1834402)(5.5246873,2.1834402)(7.0246873,-2.3565598)
\psline[linewidth=0.04cm](2.0246875,2.1634402)(7.0246873,-2.3765597)
\psline[linewidth=0.04cm,linestyle=dashed,dash=0.16cm 0.16cm](5.5046873,2.1634402)(3.9646876,0.44344023)
\psline[linewidth=0.04](4.1846876,0.6834402)(4.4246874,0.48344022)(4.1646876,0.20344023)
\rput{124.910225}(10.204001,-1.1108128){\psarc[linewidth=0.04](5.3916802,2.1056097){0.48547706}{48.07423}{180.0}}
% \usefont{T1}{ptm}{m}{n}
\rput(5.7178125,2.4534402){$B$}
% \usefont{T1}{ptm}{m}{n}
\rput(7.23,-2.2465599){$C$}
% \usefont{T1}{ptm}{m}{n}
\rput(1.885,2.4534402){$A$}
% \usefont{T1}{ptm}{m}{n}
\rput(0.075,-2.2865598){$D$}
% \usefont{T1}{ptm}{m}{n}
\rput(3.7692187,0.15344024){$E$}
% \usefont{T1}{ptm}{m}{n}
\rput(3.7959375,2.4134402){$4$ cm}
% \usefont{T1}{ptm}{m}{n}
\rput(0.61078125,0.27344024){$5$ cm}
% \usefont{T1}{ptm}{m}{n}
\rput(6.790781,0.19344023){$5$ cm}
% \usefont{T1}{ptm}{m}{n}
\rput{-43.673504}(0.04067405,2.2905717){\rput(2.8509376,1.0734402){$3$ cm}}
% \usefont{T1}{ptm}{m}{n}
\rput(3.4921875,-2.6665597){$6$ cm}
\end{pspicture} 
}
\end{center}
     

\westep{Use $\triangle ABE$ and $\triangle BEC$ to determine the two angles at $\hat{B}$} 

\westep{Find the first angle $A\hat{B}E$}
Hypotenuse and opposite side are given for both triangles, therefore use $sin$ function:
In $\triangle ABE$, \\

\begin{equation*}
\begin{array}{ccl}\hfill sin~(A\hat{B}E)& =& \frac{\mbox{\footnotesize opposite}}{\mbox{\footnotesize hypotenuse}}\hfill \vspace{5pt}\\
 \hfill & =& \dfrac{3}{4}\hfill \\
 \therefore A\hat{B}E& =& {48,59}^{\circ }\hfill 
\end{array}
\end{equation*}

\westep{Use Theorem of Pythagoras to determine $EC$}
In $\triangle ABE$, \\
\begin{equation*}
\begin{array}{ccl}
 \hfill BE^{2} &=& AB^{2} - AE^{2} \hfill \\
\hfill &=& 4^{2} - 3^{2} \hfill \\
\hfill &=& 5 \hfill \\
\end{array}
\end{equation*}

In $\triangle BEC$, \\

\begin{equation*}
\begin{array}{ccl}
 \hfill EC^{2} &=& BC^{2} - BE^{2} \hfill \\
\hfill &=& 5^{2} - 5 \hfill \\
\hfill &=& 20 \hfill \\
\therefore EC &=& 4,47 \hfill
\end{array}
\end{equation*}

\westep{Find the second angle $C\hat{B}E$}
In $\triangle BEC$ 
\begin{equation*}
\begin{array}{ccl}\hfill sin~(C\hat{B}E)& =& \frac{\mbox{\footnotesize opposite}}{\mbox{\footnotesize hypotenuse}}\hfill \vspace{5pt}\\
 \hfill & =& \dfrac{4,47}{5}\hfill \\
 \therefore C\hat{B}E& =& {61,64}^{\circ }\hfill 
\end{array}
\end{equation*}

\westep{Calculate the sum of the angles}
\begin{equation*}
A\hat{B}C = 48,59^{\circ} + 61,64^{\circ} = 110,23^{\circ}
\end{equation*}
}
\end{wex}



  
Another application is using trigonometry to find the height of a building. We could use a tape measure lowered from the roof, but this is impractical (and dangerous) for tall buildings. It is much more sensible to use trigonometry to find the height of the building.\par 

\begin{wex}{Finding the height of a building}
{
% \setcounter{subfigure}{0}
% \begin{figure}[htbp]
\begin{center}
\begin{pspicture}(0,-0.5)(7,4)
\psframe[fillstyle=crosshatch, hatchangle=180](5,0)(7,4)
\psline[](0,0)(5,0)
\psline[linestyle=dashed](0,0)(5,4)
\pswedge[]{0.6}{0}{38.66}
\rput(2.5,-0.3){$100$ m}
\rput(1.1,0.3){$38.7^\circ$}
\rput (-0.3, 0){$Q$}
\rput(5,-0.3){$B$}
\rput(5,4.3){$T$}
\rput(4.7,2){$h$}
\end{pspicture}
\end{center}
\\

% \end{figure}
% \caption{Determining the height of a building using trigonometry.}
% \label{trig:height}

The given diagram shows a building of unknown height $h$. If we $(Q)$ walk $100$ m away from the building $(B)$ and measure the angle from the ground to the top of the building $(T)$, the angle is found to be $38,{7}^{\circ }$. This is called the angle of elevation. We have a right-angled triangle and know the length of one side and an angle. We can therefore calculate the height of the building.}
{
\westep{Identify the opposite and adjacent sides and the hypotenuse}

\westep{}
In $\triangle QTB$, \\
\begin{equation*}
\begin{array}{ccl}\hfill tan~38,7^{\circ }& =& \frac{\mbox{\footnotesize opposite}}{\mbox{\footnotesize adjacent}}\hfill \\
 & =& \dfrac{h}{100}\hfill \\
  \end{array}
\end{equation*}
\westep{Rearrange and solve for $h$}

\begin{equation*}
\begin{array}{ccl}

\hfill h& =& 100\times}tan~38,7^{\circ }\hfill \\
& =& 80\hfill
  \end{array}
\end{equation*}

\westep{Write final answer}
The height of the building in $80$ m.
}
\end{wex}


\begin{wex}{Angles of elevation and depression}
{A block of flats is $200$ m away from a cellphone tower. Someone stands at $B$. They measure the angle from $B$ up to the top of the tower $E$ to be $62^{\circ}$ (the angle of elevation). They then measure the angle from $B$ down to the bottom of the tower at $C$ to be $34^{\circ}$ (the angle of depression).
\\What is the height of the cellphone tower correct to the nearest metre?\\

\\
\begin{center}
\scalebox{0.9} % Change this value to rescale the drawing.
{
\begin{pspicture}(0,-3.8445313)(10.7125,3.8445313)
\definecolor{color194}{rgb}{0.2,0.2,0.2}
\definecolor{color353b}{rgb}{0.6,0.6,0.6}
\definecolor{color649b}{rgb}{0.8,0.8,1.0}
\psframe[linewidth=0.04,linecolor=color194,dimen=outer,fillstyle=solid](2.3,0.83953124)(0.18,-3.2204688)
\psframe[linewidth=0.04,linecolor=color194,dimen=outer,fillstyle=solid,fillcolor=color353b](0.62,-0.30046874)(0.4,-0.7004688)
\psframe[linewidth=0.04,linecolor=color194,dimen=outer,fillstyle=solid,fillcolor=color353b](1.32,-0.32046875)(1.1,-0.72046876)
\psframe[linewidth=0.04,linecolor=color194,dimen=outer,fillstyle=solid,fillcolor=color353b](0.62,-1.0404687)(0.4,-1.4404688)
\psframe[linewidth=0.04,linecolor=color194,dimen=outer,fillstyle=solid,fillcolor=color353b](0.62,0.47953126)(0.4,0.07953125)
\psframe[linewidth=0.04,linecolor=color194,dimen=outer,fillstyle=solid,fillcolor=color353b](1.32,0.47953126)(1.1,0.07953125)
\psframe[linewidth=0.04,linecolor=color194,dimen=outer,fillstyle=solid,fillcolor=color353b](2.02,0.47953126)(1.8,0.07953125)
\psframe[linewidth=0.04,linecolor=color194,dimen=outer,fillstyle=solid,fillcolor=color353b](2.02,-0.32046875)(1.8,-0.72046876)
\psframe[linewidth=0.04,linecolor=color194,dimen=outer,fillstyle=solid,fillcolor=color353b](1.32,-1.0404687)(1.1,-1.4404688)
\psframe[linewidth=0.04,linecolor=color194,dimen=outer,fillstyle=solid,fillcolor=color353b](2.02,-1.0404687)(1.8,-1.4404688)
\psframe[linewidth=0.04,linecolor=color194,dimen=outer,fillstyle=solid,fillcolor=color353b](0.62,-1.7404687)(0.4,-2.1404688)
\psframe[linewidth=0.04,linecolor=color194,dimen=outer,fillstyle=solid,fillcolor=color353b](1.32,-1.7404687)(1.1,-2.1404688)
\psframe[linewidth=0.04,linecolor=color194,dimen=outer,fillstyle=solid,fillcolor=color353b](2.02,-1.7404687)(1.8,-2.1404688)
\psframe[linewidth=0.04,linecolor=color194,dimen=outer,fillstyle=solid,fillcolor=color649b](0.5,-2.6204689)(0.2,-3.2204688)
\pscircle[linewidth=0.04,linecolor=color194,dimen=outer,fillstyle=gradient,gradlines=2000,gradmidpoint=1.0](0.35,-2.8704689){0.05}
\psframe[linewidth=0.04,linecolor=color194,dimen=outer,fillstyle=solid,fillcolor=color353b](10.2,2.2795312)(10.0,-3.2204688)
\psline[linewidth=0.04cm,linecolor=color194](10.1,3.2795312)(10.1,2.2795312)
\psline[linewidth=0.04cm,linecolor=color194](4.8,0.77953124)(4.8,0.77953124)
\psline[linewidth=0.025999999cm,linecolor=color194](2.24,0.81953126)(9.98,-3.1804688)
\psline[linewidth=0.024cm,linecolor=color194](2.28,0.83953124)(10.12,3.2795312)
\psdots[dotsize=0.12,linecolor=color194](10.1,3.2795312)
\psline[linewidth=0.024cm,linecolor=color194,linestyle=dashed,dash=0.16cm 0.16cm](2.26,0.81953126)(10.02,0.81953126)
\rput(5.8015623,-3.7004688){$200$ m}
\usefont{T1}{ptm}{m}{n}
\rput(2.5395312,-2.9804688){$ A$}
\usefont{T1}{ptm}{m}{n}
\rput(2.28125,1.1795312){$B$}
\usefont{T1}{ptm}{m}{n}
\rput(10.6,-3.1804688){$C$}
\usefont{T1}{ptm}{m}{n}
\rput(10.553281,0.81953126){$D$}
\usefont{T1}{ptm}{m}{n}
\rput(10.396563,3.6795313){$E$}
\usefont{T1}{ptm}{m}{n}
\rput(3.8975,1.0795312){$62^\circ$}
\usefont{T1}{ptm}{m}{n}
\rput(3.6775,0.43953124){ $34^\circ$}
\psarc[linewidth=0.024,linecolor=color194,arrowsize=0.05291667cm 2.0,arrowlength=1.4,arrowinset=0.4]{->}(4.07,1.0295312){0.49}{336.25052}{58.24052}
\psarc[linewidth=0.024,linecolor=color194,arrowsize=0.05291667cm 2.0,arrowlength=1.4,arrowinset=0.4]{<-}(3.56,0.7995312){0.94}{294.44397}{0.0}
\psframe[linewidth=0.04,linecolor=color194,dimen=outer,fillstyle=solid,fillcolor=color649b](10.38,-3.2204688)(0.0,-3.4004688)
% \usefont{T1}{ptm}{m}{n}

\end{pspicture} 
}
\end{center}
}
{
\westep{To determine height $EC$, first calculate $ED$ and $DC$}
$\triangle BDE$ and $\triangle BDC$ are both right-angled triangles. In each of the triangles, the length $BD$ is known. Therefore we can calculate the sides of the triangles.

\westep{Calculate $CD$}
The length $AC$ is given. $CABD$ is a rectangle so $BD = AC = 200\mbox{m}$.\\
In $\triangle CBD$, 
\begin{eqnarray*}
tan ~C\hat{B}D &=& \frac{CD}{BD}\\
\therrefore CD&=&BD\times tan~ C\hat{B}D \\
&=& 200\times tan~ 34^{\circ} \\
&=& 135\mbox{ m}
\end{eqnarray*}

\westep{Calculate $DE$}
In $\triangle DBE$,
\begin{eqnarray*}
tan~ D\hat{B}E &=& \frac{DE}{BD}\\
\therefore DE&=&BD\times tan ~D\hat{B}E \\
&=& 200\times tan~ 62^\circ \\
&=&376\mbox{ m}
\end{eqnarray*}

\westep{Add the two heights to get the final answer} 
The height of the tower $CE=CD+DE=135 \mbox{ m}+376\mbox{ m}=511\mbox{ m}$.
}
\end{wex}


\begin{wex}{Building plan}
{Mr Nkosi has a garage at his house and he decides to add a corrugated iron roof to the side of the garage. The garage is $4$ m high, and his sheet for the roof is $5$ m long. If the angle of the roof is $5^\circ$, how high must he build the wall $BD$? Give the answer correct to $1$ decimal place.
\\
\begin{center}
\scalebox{1} % Change this value to rescale the drawing.
{
\begin{pspicture}(0,-1.783125)(9.368906,1.783125)
\definecolor{color194}{rgb}{0.2,0.2,0.2}
\definecolor{color247b}{rgb}{0.8,0.8,0.8}
\psframe[linewidth=0.002,linecolor=white,linestyle=dotted,dotsep=0.16cm,dimen=outer,fillstyle=solid,fillcolor=color247b](8.471094,0.67968744)(8.311094,-1.6403126)
\psframe[linewidth=0.04,linecolor=white,dimen=outer,fillstyle=solid,fillcolor=color247b](4.3910937,0.25968745)(3.8510938,-1.6203126)
\psframe[linewidth=0.04,linecolor=white,dimen=outer,fillstyle=solid,fillcolor=color247b](1.2910937,0.25968745)(0.75109375,-1.6603125)
\psframe[linewidth=0.002,linecolor=white,linestyle=dotted,dotsep=0.16cm,dimen=outer,fillstyle=solid,fillcolor=color247b](4.3910937,1.3196875)(0.79109377,0.19968745)
% \usefont{T1}{ptm}{m}{n}
\rput(2.5167189,-0.20031255){Garage}
% \usefont{T1}{ptm}{m}{n}
\rput(7.78125,1.2396874){Roof}
% \usefont{T1}{ptm}{m}{n}
\rput(8.949843,-0.50031257){Wall}
\psline[linewidth=0.04cm](0.7710937,1.2996874)(0.7710937,-1.6003126)
\psline[linewidth=0.04cm](0.75109375,1.2996874)(4.351094,1.2996874)
\psline[linewidth=0.04cm](4.3710938,1.2996874)(4.3710938,-1.6003126)
\psline[linewidth=0.04cm](0.7710937,-1.6003126)(1.2710937,-1.6003126)
\psline[linewidth=0.04cm](1.2710937,-1.6003126)(1.2710937,0.19968745)
\psline[linewidth=0.04cm](1.2710937,0.19968745)(3.8710938,0.19968745)
\psline[linewidth=0.04cm](3.8710938,0.19968745)(3.8710938,-1.6003126)
\psline[linewidth=0.04cm](3.8710938,-1.6003126)(4.3710938,-1.6003126)
\psline[linewidth=0.024cm,linecolor=color194](4.3710938,1.2996874)(8.44,0.663125)
\psline[linewidth=0.027999999cm,linecolor=color194,linestyle=dashed,dash=0.16cm 0.16cm](8.4,0.663125)(4.2510934,0.6996874)
\psline[linewidth=0.04cm,linecolor=color194](8.291094,0.67968744)(8.291094,-1.6203126)
\psline[linewidth=0.04cm,linecolor=color194](8.451094,0.67968744)(8.451094,-1.6203126)
\psline[linewidth=0.018cm,linecolor=color194](8.46,-1.616875)(0.9710938,-1.6003126)
% \usefont{T1}{ptm}{m}{n}
\rput(0.3,0.07968745){$4$ m}
% \usefont{T1}{ptm}{m}{n}
\rput(6.3395314,1.2596875){$5$ m}
% \usefont{T1}{ptm}{m}{n}
\rput(8.906875,0.6996874){$ B$}
% \usefont{T1}{ptm}{m}{n}
\rput(4.6251564,0.41968745){$A$}
% \usefont{T1}{ptm}{m}{n}
\rput(6.073125,0.8796874){$5^\circ$}
% \usefont{T1}{ptm}{m}{n}
\rput(4.5270314,1.5796875){$C$}
\psdots[dotsize=0.12,linecolor=color194](4.3710938,0.6996874)
\psdots[dotsize=0.12,linecolor=color194](4.3710938,1.2996874)
% \usefont{T1}{ptm}{m}{n}
\rput(8.858906,-1.5803125){$ D$}
\psdots[dotsize=0.1378129,linecolor=color194](8.271093,0.67968744)
\end{pspicture} 
}
\end{center}

}{
\westep{Identify opposite and adjacent sides and hypotenuse}
$\triangle ABC$ is right-angled. The hypotenuse and an angle is known therefore we can calculate $AC$. The height of the wall $BD$ is then the height of the garage minus $AC$.
\begin{eqnarray*}
 sin~A\hat{B}C &=& \frac{AC}{BC} \\
\therefore AC &=& BC \times sin~A\hat{B}C\\
&=& 5~sin~5^{\circ}\\
&=& 0,4\mbox{ m}\\
\\
\therefore BD&=& 4\mbox{ m}-0,4\mbox{ m}\\
&=& 3,6\mbox{ m}
\end{eqnarray*}


\westep{Write the final answer}  
Mr Nkosi must build his wall to be $3,6$ m high.
}
\end{wex}

\begin{exercises}{}
{
\begin{enumerate}[noitemsep, label=\textbf{\arabic*}. ] 

\item A boy flying a kite is standing $30~$m from a point directly under the kite. If the kite's string is $50~$m long, find the angle of elevation of the kite.
\item What is the angle of elevation of the sun when a tree $7,15$ m tall casts a shadow $10,1$ m long?
\item Susan is $1,4$ m tall. From a distance of $300$ m, she looks up at the top of a lighthouse. The angle of elevation is $5^{\circ}$. Determine the height of the lighthouse to the nearest metre.
\item A ladder of length $25$ m is resting against a wall, the ladder makes an angle $37^{\circ}$ to the wall. Find the distance between the wall and the base of the ladder. 

\end{enumerate}

\par \raisebox{-5 pt}{\includegraphics[width=0.5cm]{col11306.imgs/summary_www.png}} Find the answers with the shortcodes:
\par \begin{tabular}[h]{cccccc}
(1.) lcY  &  (2.) lcr  & \end{tabular}
    
}
\end{exercises} 


\summary
\begin{itemize}[noitemsep]
\item We can define three trigonometric ratios for right-angled triangles: sine ($sin$), cosine ($cos$) and tangent ($tan$).
\item Each of these ratios have a reciprocal: cosecant ($cosec$), secant ($sec$) and cotangent ($cot$).
\item We can use the principles of solving equations and the trigonometric ratios to help us solve simple trigonometric equations.
\item We can solve problems in two dimensions that involve right angled triangles.
\item For some special angles, we can easily find the values of sin, cos and tan.
\item We can extend the definitions of the trigonometric functions to any angle.
\item Trigonometry is used to help us solve problems in 2-dimensions, such as finding the height of a building.
\end{itemize}


\begin{eocexercises}{}

\begin{enumerate}[noitemsep, label=\textbf{\arabic*}. ] 
\item Without using a calculator determine the value of 
\begin{equation*}
sin~60^{\circ}~cos~30^{\circ}-cos~60^{\circ}sin~30^{\circ} + tan~45^{\circ}
\end{equation*}
\item If $3~tan~\alpha = -5$ and $0^{\circ} < \alpha < 270^{\circ}$, use a sketch to determine
    \begin{enumerate}[noitemsep, label=\textbf{\alph*}. ] 
    \item $cos~\alpha$
    \item $tan^{2}~\alpha - sec^{2}~\alpha$
    \end{enumerate}
\item Solve for $\theta$ if $\theta$ is a postive, acute angle
    \begin{enumerate}[noitemsep, label=\textbf{\alph*}. ] 
    \item $2~sin~\theta = 1,34$
    \item $1 - tan~\theta = -1$
    \item $cos~2\theta = sin~40^{\circ}$ \vspace{3pt}
    \item $\dfrac{sin~\theta}{cos~\theta}= 1$
    \end{enumerate}


\item Calculate the unknown lengths
\begin{center}
\scalebox{1}  
{ 
\begin{pspicture}(0,-2.0390613)(8.035,2.0253136) 
\psline[linewidth=0.04cm](0.02,0.02093862)(3.06,0.02093862) 
\psline[linewidth=0.04cm](0.02,0.02093862)(0.02,-2.0190613) 
\psline[linewidth=0.04cm](0.04,-1.9990613)(3.04,0.02093862) 
\psframe[linewidth=0.04,dimen=outer](0.26,0.04093862)(0.0,-0.21906137) 
\psline[linewidth=0.04cm](0.9944844,1.4531701)(3.04,0.04093862) 
\psline[linewidth=0.04cm](0.96,1.4209386)(0.024381146,0.02542873) 
\rput{-33.90198}(-0.5512336,0.79249614){\psframe[linewidth=0.04,dimen=outer](1.1544285,1.4305158)(0.8944285,1.1705158)} 
\psline[linewidth=0.04cm](2.2384837,1.992771)(3.04,0.04093862) 
\psline[linewidth=0.04cm](2.238,1.9929386)(1.0103352,1.4765087) 
\rput{-66.64198}(-0.3803531,3.117777){\psframe[linewidth=0.04,dimen=outer](2.3111107,1.9781737)(2.0511107,1.7181737)} 
% \usefont{T1}{ptm}{m}{n} 
\rput(2.2601562,-0.21406138){$30^{\circ}$} 
% \usefont{T1}{ptm}{m}{n} 
\rput(2.3034375,0.22593862){$25^{\circ}$} 
% \usefont{T1}{ptm}{m}{n} 
\rput(2.4634376,0.77){$20^{\circ}$} 
% \usefont{T1}{ptm}{m}{n} 
\rput(1.8576562,-1.2540613){$16$ cm} 
% \usefont{T1}{ptm}{m}{it} 
\rput(1.3398438,0.20593862){$a$} 
% \usefont{T1}{ptm}{m}{it} 
\rput(1.9,1.1){$b$} 
% \usefont{T1}{ptm}{m}{it} 
\rput(1.58,1.9059386){$c$} 
\psline[linewidth=0.04cm](6.689009,-1.6241995)(6.0199666,-1.4501117) 
\rput{-90.46057}(4.6635084,7.686732){\psframe[linewidth=0.04,dimen=outer](6.2543488,1.6402806)(6.0343485,1.4202806)} 
\rput{-90.46057}(4.4806204,7.508202){\psframe[linewidth=0.04,dimen=outer](6.0743546,1.6417276)(5.8543544,1.4217275)} 
\rput{-104.9971}(9.7950115,4.4890895){\psframe[linewidth=0.04,dimen=outer](6.7298956,-1.4036404)(6.509896,-1.6236403)} 
\psline[linewidth=0.04cm](7.375193,1.6303898)(4.2549725,1.6154709) 
\psline[linewidth=0.04cm](6.055236,1.6410005)(6.030317,-1.4588993) 
\psline[linewidth=0.04cm](6.0304775,-1.4389)(4.2748113,1.5953108) 
\psline[linewidth=0.04cm](7.375193,1.6303898)(6.030317,-1.4588993) 
\psline[linewidth=0.04cm](7.375193,1.6303898)(6.689009,-1.6241995) 
% \usefont{T1}{ptm}{m}{it} 
\rput(4.880625,0.005938619){$d$} 
% \usefont{T1}{ptm}{m}{it} 
\rput(6.7414064,1.8259386){$e$} 
% \usefont{T1}{ptm}{m}{n} 
\rput(5.2226562,1.8259386){$5$ m} 
% \usefont{T1}{ptm}{m}{it} 
\rput(6.2709374,-1.7940614){$f$} 
% \usefont{T1}{ptm}{m}{it} 
\rput(7.2148438,-0.07406138){$g$} 
% \usefont{T1}{ptm}{m}{n} 
\rput(4.756875,1.3859386){ $50^{\circ}$} 
% \usefont{T1}{ptm}{m}{n} 
\rput(6.9460936,1.4059386){$60^{\circ}$} 
% \usefont{T1}{ptm}{m}{n} 
\rput(6.45,-1.2740613){$80^{\circ}$} 
\end{pspicture} 
}
\end{center}

\item{In the triangle $PQR$, $PR=20$~cm, $QR=22$~cm and $P\hat{R}Q = 30^{\circ}$. The perpendicular line from $P$ to $QR$ intersects $QR$ at $X$. Calculate 
\begin{enumerate} 
\item the length $XR$, 
\item the length $PX$, and 
\item the angle $Q\hat{P}X$ 
\end{enumerate}} 
\item A ladder of length $15$ m is resting against a wall, the base of the ladder is $5$ m from the wall. Find the angle between the wall and the ladder. 
\item In the following triangle find the angle $A\hat{B}C$
\begin{center}
\begin{pspicture}(0,-2.4701562)(5.49875,2.4701562) 
\pspolygon[linewidth=0.04](0.1665625,-1.7301563)(3.3665626,1.9698437)(5.1665626,-1.7301563)(4.1665626,-1.7301563) 
\psline[linewidth=0.04cm](3.3665626,1.9698437)(3.3665626,-1.7301563) 
\rput(3.3871875,2.2798438){$A$} 
\rput(5.3459377,-2.0201561){$B$} 
\rput(3.371875,-2.0201561){$C$} 
\rput(0.07546875,-2.0201561){$D$} 
\rput(3.6,0){$9$} 
\rput(2.7525,-2.3201563){$17$} 
\psline[linewidth=0.04cm,arrowsize=0.05291667cm 2.0,arrowlength=1.4,arrowinset=0.4]{->}(3.0665624,-2.3301563)(5.2665625,-2.3301563) 
\psline[linewidth=0.04cm,arrowsize=0.05291667cm 2.0,arrowlength=1.4,arrowinset=0.4]{->}(2.4665625,-2.3301563)(0.0665625,-2.3301563) 
\psline[linewidth=0.04cm](3.3665626,-1.5301563)(3.5665624,-1.5301563) 
\psline[linewidth=0.04cm](3.5665624,-1.5301563)(3.5665624,-1.7301563) 
\rput(0.8,-1.48){$41^{\circ}$} 
\end{pspicture} 
\end{center}
\item In the following triangle find the length of side $CD$
\begin{center}
\begin{pspicture}(0,-2.2234375)(6.091875,2.2234375) 
\pspolygon[linewidth=0.04](0.1665625,-1.776875)(5.1665626,-1.776875)(5.1665626,1.823125) 
\psline[linewidth=0.04cm](3.4665625,-1.776875)(5.1665626,1.823125) 
\rput(5.2871876,2.033125){$A$} 
\rput(5.3459377,-2.066875){$B$} 
\rput(3.471875,-2.066875){$C$} 
\rput(0.07546875,-2.066875){$D$} 
\rput(5.4,0){$9$} 
\rput(4.490156,0.95){$15^{\circ}$} 
\rput(3.960156,-1.5){$35^{\circ}$} 
\psline[linewidth=0.04cm](4.9665626,-1.576875)(5.1665626,-1.576875) 
\psline[linewidth=0.04cm](4.9665626,-1.576875)(4.9665626,-1.776875) 
\end{pspicture}
\end{center} 
\item $A(5;0)$ and $B(11;4)$. Find the angle between the line through $A$ and $B$ and the $x$-axis. 
\item $C(0;-13)$ and $D(-12;14)$. Find the angle between the line through $C$ and $D$ and the $y$-axis. 


\item A right-angled triangle has hypotenuse $13$ mm. Find the length of the other two sides if one of the angles of the triangle is $50^{\circ}$.
\item One of the angles of a rhombus with perimeter $20$ cm is $30^{\circ}$. 
\begin{enumerate} 
\item Find the sides of the rhombus. 
\item Find the length of both diagonals. 
\end{enumerate} 
\item Captain Jack was sailing towards a cliff with a height of $10$ m. 
\begin{enumerate} 
\item The distance from the boat to the top of the cliff is $30$ m, calculate the angle of elevation from the boat to the top of the cliff (to the nearest integer).
\item If the boat sails $7$ m closer to the cliff, what is the new angle of elevation from the boat to the top of the cliff? 
\end{enumerate} 
\item Given the points: $E(5;0)$; $F(6;2)$ and $G(8;-2)$. Find the angle $F\hat{E}G$. 
\item  A triangle with angles $40^{\circ}$; $40^{\circ}$ and $100^{\circ}$ has a perimeter of $20$ cm. Find the length of each side of the triangle. 

\end{enumerate}
\end{eocexercises}




















