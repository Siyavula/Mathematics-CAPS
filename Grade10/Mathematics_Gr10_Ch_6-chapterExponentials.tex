         \chapter{Exponents}
    \setcounter{figure}{1}
    \setcounter{subfigure}{1}
    \label{m38359}
    \section{ Introduction}
            \nopagebreak
            \label{m38359*cid2} $ \hspace{-5pt}\begin{array}{cccccccccccc}   \end{array} $ \hspace{2 pt}\raisebox{-5 pt}{\includegraphics[width=0.5cm]{col11306.imgs/summary_www.png}} {(section shortcode: MG10042 )} \par 
      \label{m38359*id62184}In this chapter, you will learn about the short cuts to writing $2\ensuremath{\times}2\ensuremath{\times}2\ensuremath{\times}2$ . This is known as writing a number in \textsl{exponential notation}.\par 
    \section{ Definition}
            \nopagebreak
            \label{m38359*cid3} $ \hspace{-5pt}\begin{array}{cccccccccccc}   \includegraphics[width=0.75cm]{col11306.imgs/summary_video.png} &   \end{array} $ \hspace{2 pt}\raisebox{-5 pt}{} {(section shortcode: MG10043 )} \par 
      \label{m38359*id62562}Exponential notation is a short way of writing the same number multiplied by
itself many times. For example, instead of $5\ensuremath{\times}5\ensuremath{\times}5$, we write \begin{math}{5}^{3}\end{math} to show that the number 5 is multiplied by itself 3 times and we say ``5 to the power of 3''. Likewise \begin{math}{5}^{2}\end{math} is \begin{math}5\ensuremath{\times}5\end{math} and \begin{math}{3}^{5}\end{math} is \begin{math}3\ensuremath{\times}3\ensuremath{\times}3\ensuremath{\times}3\ensuremath{\times}3\end{math}. We will now have a closer look at writing numbers using exponential notation.\par 
\label{m38359*fhsst!!!underscore!!!id74}\begin{definition}
	  \begin{tabular*}{15 cm}{m{15 mm}m{}}
	\hspace*{-50pt}  \includegraphics[width=0.5in]{col11306.imgs/psflag2.png}   & \Definition{   \label{id2492834}\textbf{ Exponential Notation }} { \label{m38359*meaningfhsst!!!underscore!!!id74}
      \label{m38359*id62672}Exponential notation means a number written like\par 
      \label{m38359*id62677}\nopagebreak\noindent{}
        \settowidth{\mymathboxwidth}{\begin{equation}
    {a}^{n}\tag{5.1}
      \end{equation}
    }
    \typeout{Columnwidth = \the\columnwidth}\typeout{math as usual width = \the\mymathboxwidth}
    \ifthenelse{\lengthtest{\mymathboxwidth < \columnwidth}}{% if the math fits, do it again, for real
    \begin{equation}
    {a}^{n}\tag{5.1}
      \end{equation}
    }{% else, if it doesn't fit
    \setlength{\mymathboxwidth}{\columnwidth}
      \addtolength{\mymathboxwidth}{-48pt}
    \par\vspace{12pt}\noindent\begin{minipage}{\columnwidth}
    \parbox[t]{\mymathboxwidth}{\large$
    {a}^{n}$}\hfill
    \parbox[t]{48pt}{\raggedleft 
    (5.1)}
    \end{minipage}\vspace{12pt}\par
    }% end of conditional for this bit of math
    \typeout{math as usual width = \the\mymathboxwidth}
      \label{m38359*id62692}where $n$ is an integer and \begin{math}a\end{math} can be any real number. \begin{math}a\end{math} is called the \textsl{base} and \begin{math}n\end{math} is called the \textsl{exponent} or \textsl{index}. \par 
       } 
      \end{tabular*}
      \end{definition}
      \label{m38359*id62750}The ${n}^{\mathrm{th}}$ power of \begin{math}a\end{math} is defined as:\par 
      \label{m38359*uid1}\nopagebreak\noindent{}\settowidth{\mymathboxwidth}{\begin{equation}
    {a}^{n}=a\ensuremath{\times}a\ensuremath{\times}\cdots \ensuremath{\times}a\phantom{\rule{2.em}{0ex}}\left(\mathrm{n\; times}\right)\tag{5.2}
      \end{equation}
    }
    \typeout{Columnwidth = \the\columnwidth}\typeout{math as usual width = \the\mymathboxwidth}
    \ifthenelse{\lengthtest{\mymathboxwidth < \columnwidth}}{% if the math fits, do it again, for real
    \begin{equation}
    {a}^{n}=a\ensuremath{\times}a\ensuremath{\times}\cdots \ensuremath{\times}a\phantom{\rule{2.em}{0ex}}\left(\mathrm{n\; times}\right)\tag{5.2}
      \end{equation}
    }{% else, if it doesn't fit
    \setlength{\mymathboxwidth}{\columnwidth}
      \addtolength{\mymathboxwidth}{-48pt}
    \par\vspace{12pt}\noindent\begin{minipage}{\columnwidth}
    \parbox[t]{\mymathboxwidth}{\large$
    {a}^{n}=a\ensuremath{\times}a\ensuremath{\times}\cdots \ensuremath{\times}a\phantom{\rule{2.em}{0ex}}\left(\mathrm{n\; times}\right)$}\hfill
    \parbox[t]{48pt}{\raggedleft 
    (5.2)}
    \end{minipage}\vspace{12pt}\par
    }% end of conditional for this bit of math
    \typeout{math as usual width = \the\mymathboxwidth}
      \label{m38359*id62819}with $a$ multiplied by itself \begin{math}n\end{math} times.\par 
      \label{m38359*id62840}We can also define what it means if we have a negative exponent $-n$. Then,\par 
      \label{m38359*uid2}\nopagebreak\noindent{}\settowidth{\mymathboxwidth}{\begin{equation}
    {a}^{-n}=\frac{1}{a\ensuremath{\times}a\ensuremath{\times}\cdots \ensuremath{\times}a\phantom{\rule{2.em}{0ex}}\left(\mathrm{n\; times}\right)}\tag{5.3}
      \end{equation}
    }
    \typeout{Columnwidth = \the\columnwidth}\typeout{math as usual width = \the\mymathboxwidth}
    \ifthenelse{\lengthtest{\mymathboxwidth < \columnwidth}}{% if the math fits, do it again, for real
    \begin{equation}
    {a}^{-n}=\frac{1}{a\ensuremath{\times}a\ensuremath{\times}\cdots \ensuremath{\times}a\phantom{\rule{2.em}{0ex}}\left(\mathrm{n\; times}\right)}\tag{5.3}
      \end{equation}
    }{% else, if it doesn't fit
    \setlength{\mymathboxwidth}{\columnwidth}
      \addtolength{\mymathboxwidth}{-48pt}
    \par\vspace{12pt}\noindent\begin{minipage}{\columnwidth}
    \parbox[t]{\mymathboxwidth}{\large$
    {a}^{-n}=\frac{1}{a\ensuremath{\times}a\ensuremath{\times}\cdots \ensuremath{\times}a\phantom{\rule{2.em}{0ex}}\left(\mathrm{n\; times}\right)}$}\hfill
    \parbox[t]{48pt}{\raggedleft 
    (5.3)}
    \end{minipage}\vspace{12pt}\par
    }% end of conditional for this bit of math
    \typeout{math as usual width = \the\mymathboxwidth}
\label{m38359*notfhsst!!!underscore!!!id145}
\begin{tabular}{cc}
	   \hspace*{-50pt}\raisebox{-8 mm}{ \includegraphics[width=0.5in]{col11306.imgs/pstip2.png}  }& 
	\begin{minipage}{0.85\textwidth}
	\begin{note}
      {tip: }Exponentials
	\end{note}
	\end{minipage}
	\end{tabular}
	\par
      \label{m38359*id62918}If $n$ is an even integer, then \begin{math}{a}^{n}\end{math} will always be positive for any non-zero real number \begin{math}a\end{math}. For example, although \begin{math}-2\end{math} is negative, \begin{math}{\left(-2\right)}^{2}=-2\ensuremath{\times}-2=4\end{math} is positive and so is \begin{math}{\left(-2\right)}^{-2}=\frac{1}{-2\ensuremath{\times}-2}=\frac{1}{4}\end{math}.\par \label{m38359*eip-180}
    \setcounter{subfigure}{0}
	\begin{figure}[H] % horizontal\label{m38359*Exponents-1}
    \textnormal{Khan Academy video on Exponents - 1}\vspace{.1in} \nopagebreak
  \label{m38359*yt-media1}\label{m38359*yt-video1}
            \raisebox{-5 pt}{ \includegraphics[width=0.5cm]{col11306.imgs/summary_www.png}} { (Video:  MG10044 )}
      \vspace{2pt}
    \vspace{.1in}
 \end{figure}       \par \label{m38359*eip-375}
    \setcounter{subfigure}{0}
	\begin{figure}[H] % horizontal\label{m38359*Exponents-2}
    \textnormal{Khan Academy video on Exponents-2}\vspace{.1in} \nopagebreak
  \label{m38359*yt-media2}\label{m38359*yt-video2}
            \raisebox{-5 pt}{ \includegraphics[width=0.5cm]{col11306.imgs/summary_www.png}} { (Video:  MG10045 )}
      \vspace{2pt}
    \vspace{.1in}
 \end{figure}       \par 
    \section{ Laws of Exponents}
            \nopagebreak
            \label{m38359*cid4} $ \hspace{-5pt}\begin{array}{cccccccccccc}   \includegraphics[width=0.75cm]{col11306.imgs/summary_fullmarks.png} &   \includegraphics[width=0.75cm]{col11306.imgs/summary_video.png} &   \end{array} $ \hspace{2 pt}\raisebox{-5 pt}{} {(section shortcode: MG10046 )} \par 
      \label{m38359*id63061}There are several laws we can use to make working with exponential numbers easier. Some of these laws might have been seen in earlier grades, but we will list all the laws here for easy reference and explain each law in detail, so that you can understand them and not only remember them.\par 
      \label{m38359*uid3}\nopagebreak\noindent{}
        \settowidth{\mymathboxwidth}{\begin{equation}
    \begin{array}{ccc}\hfill {a}^{0}& =& 1\hfill \\ \hfill {a}^{m}\ensuremath{\times}{a}^{n}& =& {a}^{m+n}\hfill \\ \hfill {a}^{-n}& =& \frac{1}{{a}^{n}}\hfill \\ \hfill {a}^{m}÷{a}^{n}& =& {a}^{m-n}\hfill \\ \hfill {\left(ab\right)}^{n}& =& {a}^{n}{b}^{n}\hfill \\ \hfill {\left({a}^{m}\right)}^{n}& =& {a}^{mn}\hfill \end{array}\tag{5.4}
      \end{equation}
    }
    \typeout{Columnwidth = \the\columnwidth}\typeout{math as usual width = \the\mymathboxwidth}
    \ifthenelse{\lengthtest{\mymathboxwidth < \columnwidth}}{% if the math fits, do it again, for real
    \begin{equation}
    \begin{array}{ccc}\hfill {a}^{0}& =& 1\hfill \\ \hfill {a}^{m}\ensuremath{\times}{a}^{n}& =& {a}^{m+n}\hfill \\ \hfill {a}^{-n}& =& \frac{1}{{a}^{n}}\hfill \\ \hfill {a}^{m}÷{a}^{n}& =& {a}^{m-n}\hfill \\ \hfill {\left(ab\right)}^{n}& =& {a}^{n}{b}^{n}\hfill \\ \hfill {\left({a}^{m}\right)}^{n}& =& {a}^{mn}\hfill \end{array}\tag{5.4}
      \end{equation}
    }{% else, if it doesn't fit
    \setlength{\mymathboxwidth}{\columnwidth}
      \addtolength{\mymathboxwidth}{-48pt}
    \par\vspace{12pt}\noindent\begin{minipage}{\columnwidth}
    \parbox[t]{\mymathboxwidth}{\large$
    {a}^{0}=1{a}^{m}\ensuremath{\times}{a}^{n}={a}^{m+n}{a}^{-n}=\frac{1}{{a}^{n}}{a}^{m}÷{a}^{n}={a}^{m-n}{\left(ab\right)}^{n}={a}^{n}{b}^{n}{\left({a}^{m}\right)}^{n}={a}^{mn}$}\hfill
    \parbox[t]{48pt}{\raggedleft 
    (5.4)}
    \end{minipage}\vspace{12pt}\par
    }% end of conditional for this bit of math
    \typeout{math as usual width = \the\mymathboxwidth}
      \label{m38359*uid4}
            \subsection{ Exponential Law 1: ${a}^{0}=1$}
            \nopagebreak
        \label{m38359*id63512}Our definition of exponential notation shows that\par 
        \label{m38359*uid5}\nopagebreak\noindent{}
          \settowidth{\mymathboxwidth}{\begin{equation}
    \begin{array}{ccc}\hfill {a}^{0}& =& 1,\left(a\ne 0\right)\hfill \end{array}\tag{5.5}
      \end{equation}
    }
    \typeout{Columnwidth = \the\columnwidth}\typeout{math as usual width = \the\mymathboxwidth}
    \ifthenelse{\lengthtest{\mymathboxwidth < \columnwidth}}{% if the math fits, do it again, for real
    \begin{equation}
    \begin{array}{ccc}\hfill {a}^{0}& =& 1,\left(a\ne 0\right)\hfill \end{array}\tag{5.5}
      \end{equation}
    }{% else, if it doesn't fit
    \setlength{\mymathboxwidth}{\columnwidth}
      \addtolength{\mymathboxwidth}{-48pt}
    \par\vspace{12pt}\noindent\begin{minipage}{\columnwidth}
    \parbox[t]{\mymathboxwidth}{\large$
    {a}^{0}=1,\left(a\ne 0\right)$}\hfill
    \parbox[t]{48pt}{\raggedleft 
    (5.5)}
    \end{minipage}\vspace{12pt}\par
    }% end of conditional for this bit of math
    \typeout{math as usual width = \the\mymathboxwidth}
        \label{m38359*eip-662}To convince yourself of why this is true, use the fourth exponential law above (division of exponents) and consider what happens when $m=n$.\par \label{m38359*id63571}For example, \begin{math}{x}^{0}=1\end{math} and \begin{math}{\left(1\phantom{\rule{0.277778em}{0ex}}000\phantom{\rule{0.277778em}{0ex}}000\right)}^{0}=1\end{math}.\par 
\label{m38359*secfhsst!!!underscore!!!id339}
            \subsubsection{  Application using Exponential Law 1: ${a}^{0}=1,\left(a\ne 0\right)$ }
            \nopagebreak
        \label{m38359*id63666}\begin{enumerate}[noitemsep, label=\textbf{\arabic*}. ] 
            \label{m38359*uid6}\item 
            ${16}^{0}$
      \label{m38359*uid7}\item 
        $16{a}^{0}$
      \label{m38359*uid8}\item 
        ${\left(16+a\right)}^{0}$
      \label{m38359*uid9}\item 
        ${\left(-16\right)}^{0}$
      \label{m38359*uid10}\item 
        $-{16}^{0}$ 
\newline
\newline
          \end{enumerate}
      \label{m38359*uid11}
\par \raisebox{-5 pt}{\includegraphics[width=0.5cm]{col11306.imgs/summary_www.png}} Find the answers with the shortcodes:
 \par \begin{tabular}[h]{cccccc}
 (1.) lOG  & \end{tabular}
            \subsection{ Exponential Law 2: ${a}^{m}\ensuremath{\times}{a}^{n}={a}^{m+n}$}
            \nopagebreak
        \label{m38359*eip-427}
    \setcounter{subfigure}{0}
	\begin{figure}[H] % horizontal\label{m38359*ExponentsRule1}
    \textnormal{Khan Academy video on Exponents - 3}\vspace{.1in} \nopagebreak
  \label{m38359*yt-media3}\label{m38359*yt-video3}
            \raisebox{-5 pt}{ \includegraphics[width=0.5cm]{col11306.imgs/summary_www.png}} { (Video:  MG10047 )}
      \vspace{2pt}
    \vspace{.1in}
 \end{figure}       \par \label{m38359*id63888}Our definition of exponential notation shows that\par 
        \label{m38359*uid12}\nopagebreak\noindent{}\settowidth{\mymathboxwidth}{\begin{equation}
    \begin{array}{cccc}\hfill {a}^{m}\ensuremath{\times}{a}^{n}& =& 1\ensuremath{\times}a\ensuremath{\times}...\ensuremath{\times}a\hfill & \left(\mathrm{m\; times}\right)\hfill \\ \hfill & & \phantom{\rule{-0.166667em}{0ex}}\phantom{\rule{-0.166667em}{0ex}}\phantom{\rule{-0.166667em}{0ex}}\phantom{\rule{-0.166667em}{0ex}}\ensuremath{\times}1\ensuremath{\times}a\ensuremath{\times}...\ensuremath{\times}a\hfill & \left(\mathrm{n\; times}\right)\hfill \\ \hfill & =& 1\ensuremath{\times}a\ensuremath{\times}...\ensuremath{\times}a\hfill & \left(\mathrm{m}+\mathrm{n\; times}\right)\hfill \\ \hfill & =& {a}^{m+n}\hfill & \end{array}\tag{5.6}
      \end{equation}
    }
    \typeout{Columnwidth = \the\columnwidth}\typeout{math as usual width = \the\mymathboxwidth}
    \ifthenelse{\lengthtest{\mymathboxwidth < \columnwidth}}{% if the math fits, do it again, for real
    \begin{equation}
    \begin{array}{cccc}\hfill {a}^{m}\ensuremath{\times}{a}^{n}& =& 1\ensuremath{\times}a\ensuremath{\times}...\ensuremath{\times}a\hfill & \left(\mathrm{m\; times}\right)\hfill \\ \hfill & & \phantom{\rule{-0.166667em}{0ex}}\phantom{\rule{-0.166667em}{0ex}}\phantom{\rule{-0.166667em}{0ex}}\phantom{\rule{-0.166667em}{0ex}}\ensuremath{\times}1\ensuremath{\times}a\ensuremath{\times}...\ensuremath{\times}a\hfill & \left(\mathrm{n\; times}\right)\hfill \\ \hfill & =& 1\ensuremath{\times}a\ensuremath{\times}...\ensuremath{\times}a\hfill & \left(\mathrm{m}+\mathrm{n\; times}\right)\hfill \\ \hfill & =& {a}^{m+n}\hfill & \end{array}\tag{5.6}
      \end{equation}
    }{% else, if it doesn't fit
    \setlength{\mymathboxwidth}{\columnwidth}
      \addtolength{\mymathboxwidth}{-48pt}
    \par\vspace{12pt}\noindent\begin{minipage}{\columnwidth}
    \parbox[t]{\mymathboxwidth}{\large$
    {a}^{m}\ensuremath{\times}{a}^{n}=1\ensuremath{\times}a\ensuremath{\times}...\ensuremath{\times}a\left(\mathrm{m\; times}\right)\phantom{\rule{-0.166667em}{0ex}}\phantom{\rule{-0.166667em}{0ex}}\phantom{\rule{-0.166667em}{0ex}}\phantom{\rule{-0.166667em}{0ex}}\ensuremath{\times}1\ensuremath{\times}a\ensuremath{\times}...\ensuremath{\times}a\left(\mathrm{n\; times}\right)=1\ensuremath{\times}a\ensuremath{\times}...\ensuremath{\times}a\left(\mathrm{m}+\mathrm{n\; times}\right)={a}^{m+n}$}\hfill
    \parbox[t]{48pt}{\raggedleft 
    (5.6)}
    \end{minipage}\vspace{12pt}\par
    }% end of conditional for this bit of math
    \typeout{math as usual width = \the\mymathboxwidth}
        \label{m38359*id64082}For example,\par 
        \label{m38359*id64085}\nopagebreak\noindent{}
          \settowidth{\mymathboxwidth}{\begin{equation}
    \begin{array}{ccc}\hfill {2}^{7}\ensuremath{\times}{2}^{3}& =& \left(2\ensuremath{\times}2\ensuremath{\times}2\ensuremath{\times}2\ensuremath{\times}2\ensuremath{\times}2\ensuremath{\times}2\right)\ensuremath{\times}\left(2\ensuremath{\times}2\ensuremath{\times}2\right)\hfill \\ & =& {2}^{7+3}\hfill \\ & =& {2}^{10}\hfill \end{array}\tag{5.7}
      \end{equation}
    }
    \typeout{Columnwidth = \the\columnwidth}\typeout{math as usual width = \the\mymathboxwidth}
    \ifthenelse{\lengthtest{\mymathboxwidth < \columnwidth}}{% if the math fits, do it again, for real
    \begin{equation}
    \begin{array}{ccc}\hfill {2}^{7}\ensuremath{\times}{2}^{3}& =& \left(2\ensuremath{\times}2\ensuremath{\times}2\ensuremath{\times}2\ensuremath{\times}2\ensuremath{\times}2\ensuremath{\times}2\right)\ensuremath{\times}\left(2\ensuremath{\times}2\ensuremath{\times}2\right)\hfill \\ & =& {2}^{7+3}\hfill \\ & =& {2}^{10}\hfill \end{array}\tag{5.7}
      \end{equation}
    }{% else, if it doesn't fit
    \setlength{\mymathboxwidth}{\columnwidth}
      \addtolength{\mymathboxwidth}{-48pt}
    \par\vspace{12pt}\noindent\begin{minipage}{\columnwidth}
    \parbox[t]{\mymathboxwidth}{\large$
    {2}^{7}\ensuremath{\times}{2}^{3}=\left(2\ensuremath{\times}2\ensuremath{\times}2\ensuremath{\times}2\ensuremath{\times}2\ensuremath{\times}2\ensuremath{\times}2\right)\ensuremath{\times}\left(2\ensuremath{\times}2\ensuremath{\times}2\right)={2}^{7+3}={2}^{10}$}\hfill
    \parbox[t]{48pt}{\raggedleft 
    (5.7)}
    \end{minipage}\vspace{12pt}\par
    }% end of conditional for this bit of math
    \typeout{math as usual width = \the\mymathboxwidth}
\label{m38359*notfhsst!!!underscore!!!id613}
\begin{tabular}{cc}
	\hspace*{-50pt}\raisebox{-8 mm}{\hspace{-0.2in}\includegraphics[width=0.75in]{col11306.imgs/psfact2.png} } & 
	\begin{minipage}{0.85\textwidth}
	\begin{note}
      {note: } This simple law is the reason why exponentials were originally invented. In the days before calculators, all multiplication had to be done by hand with a pencil and a pad of paper. Multiplication takes a very long time to do and is very tedious. Adding numbers however, is very easy and quick to do. If you look at what this law is saying you will realise that it means that adding the exponents of two exponential numbers (of the same base) is the same as multiplying the two numbers together. This meant that for certain numbers, there was no need to actually multiply the numbers together in order to find out what their multiple was. This saved mathematicians a lot of time, which they could use to do something more productive.
	\end{note}
	\end{minipage}
	\end{tabular}
	\par
\label{m38359*secfhsst!!!underscore!!!id614}
            \subsubsection{  Application using Exponential Law 2: ${a}^{m}\ensuremath{\times}{a}^{n}={a}^{m+n}$ }
            \nopagebreak
        \label{m38359*id64269}\begin{enumerate}[noitemsep, label=\textbf{\arabic*}. ] 
            \label{m38359*uid13}\item 
            ${x}^{2}\ensuremath{\cdot}{x}^{5}$
      \label{m38359*uid14}\item 
        ${2}^{3}\ensuremath{\cdot}{2}^{4}$
        [Take note that the base (2) stays the same.]
      \label{m38359*uid15}\item 
        $3\ensuremath{\times}{3}^{2a}\ensuremath{\times}{3}^{2}$
\newline
\newline
          \end{enumerate}
      \label{m38359*uid16}
\par \raisebox{-5 pt}{\includegraphics[width=0.5cm]{col11306.imgs/summary_www.png}} Find the answers with the shortcodes:
 \par \begin{tabular}[h]{cccccc}
 (1.) lO7  & \end{tabular}
            \subsection{ Exponential Law 3: ${a}^{-n}=\frac{1}{{a}^{n}},a\ne 0$}
            \nopagebreak
        \label{m38359*id64482}Our definition of exponential notation for a negative exponent shows that\par 
        \label{m38359*uid17}\nopagebreak\noindent{}\settowidth{\mymathboxwidth}{\begin{equation}
    \begin{array}{cccc}\hfill {a}^{-n}& =& 1÷a÷...÷a\hfill & \left(\mathrm{n\; times}\right)\hfill \\ \hfill & =& \frac{1}{1\ensuremath{\times}a\ensuremath{\times}\cdots \ensuremath{\times}a}\hfill & \left(\mathrm{n\; times}\right)\hfill \\ \hfill & =& \frac{1}{{a}^{n}}\hfill & \end{array}\tag{5.8}
      \end{equation}
    }
    \typeout{Columnwidth = \the\columnwidth}\typeout{math as usual width = \the\mymathboxwidth}
    \ifthenelse{\lengthtest{\mymathboxwidth < \columnwidth}}{% if the math fits, do it again, for real
    \begin{equation}
    \begin{array}{cccc}\hfill {a}^{-n}& =& 1÷a÷...÷a\hfill & \left(\mathrm{n\; times}\right)\hfill \\ \hfill & =& \frac{1}{1\ensuremath{\times}a\ensuremath{\times}\cdots \ensuremath{\times}a}\hfill & \left(\mathrm{n\; times}\right)\hfill \\ \hfill & =& \frac{1}{{a}^{n}}\hfill & \end{array}\tag{5.8}
      \end{equation}
    }{% else, if it doesn't fit
    \setlength{\mymathboxwidth}{\columnwidth}
      \addtolength{\mymathboxwidth}{-48pt}
    \par\vspace{12pt}\noindent\begin{minipage}{\columnwidth}
    \parbox[t]{\mymathboxwidth}{\large$
    {a}^{-n}=1÷a÷...÷a\left(\mathrm{n\; times}\right)=\frac{1}{1\ensuremath{\times}a\ensuremath{\times}\cdots \ensuremath{\times}a}\left(\mathrm{n\; times}\right)=\frac{1}{{a}^{n}}$}\hfill
    \parbox[t]{48pt}{\raggedleft 
    (5.8)}
    \end{minipage}\vspace{12pt}\par
    }% end of conditional for this bit of math
    \typeout{math as usual width = \the\mymathboxwidth}
        \label{m38359*id64624}This means that a minus sign in the exponent is just another way of showing that the whole exponential number is to be divided instead of multiplied.\par 
        \label{m38359*id64630}For example,\par 
        \label{m38359*id64634}\nopagebreak\noindent{}
          \settowidth{\mymathboxwidth}{\begin{equation}
    \begin{array}{ccc}\hfill {2}^{-7}& =& \frac{1}{2\ensuremath{\times}2\ensuremath{\times}2\ensuremath{\times}2\ensuremath{\times}2\ensuremath{\times}2\ensuremath{\times}2}\hfill \\ & =& \frac{1}{{2}^{7}}\hfill \end{array}\tag{5.9}
      \end{equation}
    }
    \typeout{Columnwidth = \the\columnwidth}\typeout{math as usual width = \the\mymathboxwidth}
    \ifthenelse{\lengthtest{\mymathboxwidth < \columnwidth}}{% if the math fits, do it again, for real
    \begin{equation}
    \begin{array}{ccc}\hfill {2}^{-7}& =& \frac{1}{2\ensuremath{\times}2\ensuremath{\times}2\ensuremath{\times}2\ensuremath{\times}2\ensuremath{\times}2\ensuremath{\times}2}\hfill \\ & =& \frac{1}{{2}^{7}}\hfill \end{array}\tag{5.9}
      \end{equation}
    }{% else, if it doesn't fit
    \setlength{\mymathboxwidth}{\columnwidth}
      \addtolength{\mymathboxwidth}{-48pt}
    \par\vspace{12pt}\noindent\begin{minipage}{\columnwidth}
    \parbox[t]{\mymathboxwidth}{\large$
    {2}^{-7}=\frac{1}{2\ensuremath{\times}2\ensuremath{\times}2\ensuremath{\times}2\ensuremath{\times}2\ensuremath{\times}2\ensuremath{\times}2}=\frac{1}{{2}^{7}}$}\hfill
    \parbox[t]{48pt}{\raggedleft 
    (5.9)}
    \end{minipage}\vspace{12pt}\par
    }% end of conditional for this bit of math
    \typeout{math as usual width = \the\mymathboxwidth}
\label{m38359*eip-294}This law is useful in helping us simplify fractions where there are exponents in both the denominator and the numerator. For example:
\label{m38359*id7902}\nopagebreak\noindent{}
\settowidth{\mymathboxwidth}{\begin{equation}
    \begin{array}{ccc}\frac{{a}^{-3}}{{a}^{4}}& =& \frac{1}{{a}^{3}{a}^{4}}\\ & =& \frac{1}{{a}^{7}}\end{array}\tag{5.10}
      \end{equation}
    }
    \typeout{Columnwidth = \the\columnwidth}\typeout{math as usual width = \the\mymathboxwidth}
    \ifthenelse{\lengthtest{\mymathboxwidth < \columnwidth}}{% if the math fits, do it again, for real
    \begin{equation}
    \begin{array}{ccc}\frac{{a}^{-3}}{{a}^{4}}& =& \frac{1}{{a}^{3}{a}^{4}}\\ & =& \frac{1}{{a}^{7}}\end{array}\tag{5.10}
      \end{equation}
    }{% else, if it doesn't fit
    \setlength{\mymathboxwidth}{\columnwidth}
      \addtolength{\mymathboxwidth}{-48pt}
    \par\vspace{12pt}\noindent\begin{minipage}{\columnwidth}
    \parbox[t]{\mymathboxwidth}{\large$
    \frac{{a}^{-3}}{{a}^{4}}=\frac{1}{{a}^{3}{a}^{4}}=\frac{1}{{a}^{7}}$}\hfill
    \parbox[t]{48pt}{\raggedleft 
    (5.10)}
    \end{minipage}\vspace{12pt}\par
    }% end of conditional for this bit of math
    \typeout{math as usual width = \the\mymathboxwidth}
\par \label{m38359*secfhsst!!!underscore!!!id835}
            \subsubsection{  Application using Exponential Law 3: ${a}^{-n}=\frac{1}{{a}^{n}},a\ne 0$ }
            \nopagebreak
        \label{m38359*id64771}\begin{enumerate}[noitemsep, label=\textbf{\arabic*}. ] 
            \label{m38359*uid18}\item 
            ${2}^{-2}=\frac{1}{{2}^{2}}$
      \label{m38359*uid19}\item 
        $\frac{{2}^{-2}}{{3}^{2}}$
      \label{m38359*uid20}\item 
        ${\left(\frac{2}{3}\right)}^{-3}$
      \label{m38359*uid21}\item 
        $\frac{m}{{n}^{-4}}$
      \label{m38359*uid22}\item 
        $\frac{{a}^{-3}\ensuremath{\cdot}{x}^{4}}{{a}^{5}\ensuremath{\cdot}{x}^{-2}}$
\newline
\newline
          \end{enumerate}
      \label{m38359*uid23}
\par \raisebox{-5 pt}{\includegraphics[width=0.5cm]{col11306.imgs/summary_www.png}} Find the answers with the shortcodes:
 \par \begin{tabular}[h]{cccccc}
 (1.) lcx  & \end{tabular}
            \subsection{ Exponential Law 4: ${a}^{m}÷{a}^{n}={a}^{m-n}$}
            \nopagebreak
        \label{m38359*id65186}We already realised with law 3 that a minus sign is another way of saying that the exponential number is to be divided instead of multiplied. Law 4 is just a more general way of saying the same thing. We get this law by multiplying law 3 by ${a}^{m}$ on both sides and using law 2.\par 
        \label{m38359*uid24}\nopagebreak\noindent{}
          \settowidth{\mymathboxwidth}{\begin{equation}
    \begin{array}{ccc}\hfill \frac{{a}^{m}}{{a}^{n}}& =& {a}^{m}{a}^{-n}\hfill \\ \hfill & =& {a}^{m-n}\hfill \end{array}\tag{5.11}
      \end{equation}
    }
    \typeout{Columnwidth = \the\columnwidth}\typeout{math as usual width = \the\mymathboxwidth}
    \ifthenelse{\lengthtest{\mymathboxwidth < \columnwidth}}{% if the math fits, do it again, for real
    \begin{equation}
    \begin{array}{ccc}\hfill \frac{{a}^{m}}{{a}^{n}}& =& {a}^{m}{a}^{-n}\hfill \\ \hfill & =& {a}^{m-n}\hfill \end{array}\tag{5.11}
      \end{equation}
    }{% else, if it doesn't fit
    \setlength{\mymathboxwidth}{\columnwidth}
      \addtolength{\mymathboxwidth}{-48pt}
    \par\vspace{12pt}\noindent\begin{minipage}{\columnwidth}
    \parbox[t]{\mymathboxwidth}{\large$
    \frac{{a}^{m}}{{a}^{n}}={a}^{m}{a}^{-n}={a}^{m-n}$}\hfill
    \parbox[t]{48pt}{\raggedleft 
    (5.11)}
    \end{minipage}\vspace{12pt}\par
    }% end of conditional for this bit of math
    \typeout{math as usual width = \the\mymathboxwidth}
        \label{m38359*id65293}For example,\par 
        \label{m38359*id65296}\nopagebreak\noindent{}
          \settowidth{\mymathboxwidth}{\begin{equation}
    \begin{array}{ccc}\hfill {2}^{7}÷{2}^{3}& =& \frac{2\ensuremath{\times}2\ensuremath{\times}2\ensuremath{\times}2\ensuremath{\times}2\ensuremath{\times}2\ensuremath{\times}2}{2\ensuremath{\times}2\ensuremath{\times}2}\hfill \\ & =& 2\ensuremath{\times}2\ensuremath{\times}2\ensuremath{\times}2\hfill \\ & =& {2}^{4}\hfill \\ & =& {2}^{7-3}\hfill \end{array}\tag{5.12}
      \end{equation}
    }
    \typeout{Columnwidth = \the\columnwidth}\typeout{math as usual width = \the\mymathboxwidth}
    \ifthenelse{\lengthtest{\mymathboxwidth < \columnwidth}}{% if the math fits, do it again, for real
    \begin{equation}
    \begin{array}{ccc}\hfill {2}^{7}÷{2}^{3}& =& \frac{2\ensuremath{\times}2\ensuremath{\times}2\ensuremath{\times}2\ensuremath{\times}2\ensuremath{\times}2\ensuremath{\times}2}{2\ensuremath{\times}2\ensuremath{\times}2}\hfill \\ & =& 2\ensuremath{\times}2\ensuremath{\times}2\ensuremath{\times}2\hfill \\ & =& {2}^{4}\hfill \\ & =& {2}^{7-3}\hfill \end{array}\tag{5.12}
      \end{equation}
    }{% else, if it doesn't fit
    \setlength{\mymathboxwidth}{\columnwidth}
      \addtolength{\mymathboxwidth}{-48pt}
    \par\vspace{12pt}\noindent\begin{minipage}{\columnwidth}
    \parbox[t]{\mymathboxwidth}{\large$
    {2}^{7}÷{2}^{3}=\frac{2\ensuremath{\times}2\ensuremath{\times}2\ensuremath{\times}2\ensuremath{\times}2\ensuremath{\times}2\ensuremath{\times}2}{2\ensuremath{\times}2\ensuremath{\times}2}=2\ensuremath{\times}2\ensuremath{\times}2\ensuremath{\times}2={2}^{4}={2}^{7-3}$}\hfill
    \parbox[t]{48pt}{\raggedleft 
    (5.12)}
    \end{minipage}\vspace{12pt}\par
    }% end of conditional for this bit of math
    \typeout{math as usual width = \the\mymathboxwidth}
\label{m38359*eip-693}
    \setcounter{subfigure}{0}
	\begin{figure}[H] % horizontal\label{m38359*exponents-5}
    \textnormal{Khan academy video on exponents - 4}\vspace{.1in} \nopagebreak
  \label{m38359*yt-media6}\label{m38359*yt-video6}
            \raisebox{-5 pt}{ \includegraphics[width=0.5cm]{col11306.imgs/summary_www.png}} { (Video:  MG10048 )}
      \vspace{2pt}
    \vspace{.1in}
 \end{figure}       \par \label{m38359*secfhsst!!!underscore!!!id1192}
            \subsubsection{ Application using Exponential Law 4: ${a}^{m}÷{a}^{n}={a}^{m-n}$}
            \nopagebreak
        \label{m38359*id65493}\begin{enumerate}[noitemsep, label=\textbf{\arabic*}. ] 
            \label{m38359*uid25}\item 
            $\frac{{a}^{6}}{{a}^{2}}={a}^{6-2}$
      \label{m38359*uid26}\item 
        $\frac{{3}^{2}}{{3}^{6}}$
      \label{m38359*uid27}\item 
        $\frac{32{a}^{2}}{4{a}^{8}}$
      \label{m38359*uid28}\item 
        $\frac{{a}^{3x}}{{a}^{4}}$
\newline
\newline
          \end{enumerate}
      \label{m38359*uid29}
\par \raisebox{-5 pt}{\includegraphics[width=0.5cm]{col11306.imgs/summary_www.png}} Find the answers with the shortcodes:
 \par \begin{tabular}[h]{cccccc}
 (1.) lOA  & \end{tabular}
            \subsection{ Exponential Law 5: ${\left(ab\right)}^{n}={a}^{n}{b}^{n}$}
            \nopagebreak
        \label{m38359*id65819}The order in which two real numbers are multiplied together does not matter. Therefore,\par 
        \label{m38359*uid30}\nopagebreak\noindent{}\settowidth{\mymathboxwidth}{\begin{equation}
    \begin{array}{cccc}\hfill {\left(ab\right)}^{n}& =& a\ensuremath{\times}b\ensuremath{\times}a\ensuremath{\times}b\ensuremath{\times}...\ensuremath{\times}a\ensuremath{\times}b\hfill & \left(\mathrm{n\; times}\right)\hfill \\ \hfill & =& a\ensuremath{\times}a\ensuremath{\times}...\ensuremath{\times}a\hfill & \left(\mathrm{n\; times}\right)\hfill \\ \hfill & & \phantom{\rule{-0.166667em}{0ex}}\phantom{\rule{-0.166667em}{0ex}}\phantom{\rule{-0.166667em}{0ex}}\phantom{\rule{-0.166667em}{0ex}}\ensuremath{\times}b\ensuremath{\times}b\ensuremath{\times}...\ensuremath{\times}b\hfill & \left(\mathrm{n\; times}\right)\hfill \\ \hfill & =& {a}^{n}{b}^{n}\hfill & \end{array}\tag{5.13}
      \end{equation}
    }
    \typeout{Columnwidth = \the\columnwidth}\typeout{math as usual width = \the\mymathboxwidth}
    \ifthenelse{\lengthtest{\mymathboxwidth < \columnwidth}}{% if the math fits, do it again, for real
    \begin{equation}
    \begin{array}{cccc}\hfill {\left(ab\right)}^{n}& =& a\ensuremath{\times}b\ensuremath{\times}a\ensuremath{\times}b\ensuremath{\times}...\ensuremath{\times}a\ensuremath{\times}b\hfill & \left(\mathrm{n\; times}\right)\hfill \\ \hfill & =& a\ensuremath{\times}a\ensuremath{\times}...\ensuremath{\times}a\hfill & \left(\mathrm{n\; times}\right)\hfill \\ \hfill & & \phantom{\rule{-0.166667em}{0ex}}\phantom{\rule{-0.166667em}{0ex}}\phantom{\rule{-0.166667em}{0ex}}\phantom{\rule{-0.166667em}{0ex}}\ensuremath{\times}b\ensuremath{\times}b\ensuremath{\times}...\ensuremath{\times}b\hfill & \left(\mathrm{n\; times}\right)\hfill \\ \hfill & =& {a}^{n}{b}^{n}\hfill & \end{array}\tag{5.13}
      \end{equation}
    }{% else, if it doesn't fit
    \setlength{\mymathboxwidth}{\columnwidth}
      \addtolength{\mymathboxwidth}{-48pt}
    \par\vspace{12pt}\noindent\begin{minipage}{\columnwidth}
    \parbox[t]{\mymathboxwidth}{\large$
    {\left(ab\right)}^{n}=a\ensuremath{\times}b\ensuremath{\times}a\ensuremath{\times}b\ensuremath{\times}...\ensuremath{\times}a\ensuremath{\times}b\left(\mathrm{n\; times}\right)=a\ensuremath{\times}a\ensuremath{\times}...\ensuremath{\times}a\left(\mathrm{n\; times}\right)\phantom{\rule{-0.166667em}{0ex}}\phantom{\rule{-0.166667em}{0ex}}\phantom{\rule{-0.166667em}{0ex}}\phantom{\rule{-0.166667em}{0ex}}\ensuremath{\times}b\ensuremath{\times}b\ensuremath{\times}...\ensuremath{\times}b\left(\mathrm{n\; times}\right)={a}^{n}{b}^{n}$}\hfill
    \parbox[t]{48pt}{\raggedleft 
    (5.13)}
    \end{minipage}\vspace{12pt}\par
    }% end of conditional for this bit of math
    \typeout{math as usual width = \the\mymathboxwidth}
        \label{m38359*id66030}For example,\par 
        \label{m38359*id66034}\nopagebreak\noindent{}
          \settowidth{\mymathboxwidth}{\begin{equation}
    \begin{array}{ccc}\hfill {\left(2\ensuremath{\cdot}3\right)}^{4}& =& \left(2\ensuremath{\cdot}3\right)\ensuremath{\times}\left(2\ensuremath{\cdot}3\right)\ensuremath{\times}\left(2\ensuremath{\cdot}3\right)\ensuremath{\times}\left(2\ensuremath{\cdot}3\right)\hfill \\ & =& \left(2\ensuremath{\times}2\ensuremath{\times}2\ensuremath{\times}2\right)\ensuremath{\times}\left(3\ensuremath{\times}3\ensuremath{\times}3\ensuremath{\times}3\right)\hfill \\ & =& \left({2}^{4}\right)\ensuremath{\times}\left({3}^{4}\right)\hfill \\ & =& {2}^{4}{3}^{4}\hfill \end{array}\tag{5.14}
      \end{equation}
    }
    \typeout{Columnwidth = \the\columnwidth}\typeout{math as usual width = \the\mymathboxwidth}
    \ifthenelse{\lengthtest{\mymathboxwidth < \columnwidth}}{% if the math fits, do it again, for real
    \begin{equation}
    \begin{array}{ccc}\hfill {\left(2\ensuremath{\cdot}3\right)}^{4}& =& \left(2\ensuremath{\cdot}3\right)\ensuremath{\times}\left(2\ensuremath{\cdot}3\right)\ensuremath{\times}\left(2\ensuremath{\cdot}3\right)\ensuremath{\times}\left(2\ensuremath{\cdot}3\right)\hfill \\ & =& \left(2\ensuremath{\times}2\ensuremath{\times}2\ensuremath{\times}2\right)\ensuremath{\times}\left(3\ensuremath{\times}3\ensuremath{\times}3\ensuremath{\times}3\right)\hfill \\ & =& \left({2}^{4}\right)\ensuremath{\times}\left({3}^{4}\right)\hfill \\ & =& {2}^{4}{3}^{4}\hfill \end{array}\tag{5.14}
      \end{equation}
    }{% else, if it doesn't fit
    \setlength{\mymathboxwidth}{\columnwidth}
      \addtolength{\mymathboxwidth}{-48pt}
    \par\vspace{12pt}\noindent\begin{minipage}{\columnwidth}
    \parbox[t]{\mymathboxwidth}{\large$
    {\left(2\ensuremath{\cdot}3\right)}^{4}=\left(2\ensuremath{\cdot}3\right)\ensuremath{\times}\left(2\ensuremath{\cdot}3\right)\ensuremath{\times}\left(2\ensuremath{\cdot}3\right)\ensuremath{\times}\left(2\ensuremath{\cdot}3\right)=\left(2\ensuremath{\times}2\ensuremath{\times}2\ensuremath{\times}2\right)\ensuremath{\times}\left(3\ensuremath{\times}3\ensuremath{\times}3\ensuremath{\times}3\right)=\left({2}^{4}\right)\ensuremath{\times}\left({3}^{4}\right)={2}^{4}{3}^{4}$}\hfill
    \parbox[t]{48pt}{\raggedleft 
    (5.14)}
    \end{minipage}\vspace{12pt}\par
    }% end of conditional for this bit of math
    \typeout{math as usual width = \the\mymathboxwidth}
\label{m38359*secfhsst!!!underscore!!!id1581}
            \subsubsection{ Application using Exponential Law 5: ${\left(ab\right)}^{n}={a}^{n}{b}^{n}$ }
            \nopagebreak
        \label{m38359*id66288}\begin{enumerate}[noitemsep, label=\textbf{\arabic*}. ] 
            \label{m38359*uid31}\item 
            ${\left(2xy\right)}^{3}={2}^{3}{x}^{3}{y}^{3}$
      \label{m38359*uid32}\item 
        ${\left(\frac{7a}{b}\right)}^{2}$
      \label{m38359*uid33}\item 
        ${\left(5a\right)}^{3}$
\newline
\newline
          \end{enumerate}
      \label{m38359*uid34}
\par \raisebox{-5 pt}{\includegraphics[width=0.5cm]{col11306.imgs/summary_www.png}} Find the answers with the shortcodes:
 \par \begin{tabular}[h]{cccccc}
 (1.) lOs  & \end{tabular}
            \subsection{ Exponential Law 6: ${\left({a}^{m}\right)}^{n}={a}^{mn}$}
            \nopagebreak
        \label{m38359*id66531}We can find the exponential of an exponential of a number. An exponential of a number is just a real number. So, even though the sentence sounds complicated, it is just saying that you can find the exponential of a number and then take the exponential of that number. You just take the exponential twice, using the answer of the first exponential as the argument for the second one.\par 
        \label{m38359*uid35}\nopagebreak\noindent{}\settowidth{\mymathboxwidth}{\begin{equation}
    \begin{array}{cccc}\hfill {\left({a}^{m}\right)}^{n}& =& {a}^{m}\ensuremath{\times}{a}^{m}\ensuremath{\times}...\ensuremath{\times}{a}^{m}\hfill & \left(\mathrm{n\; times}\right)\hfill \\ \hfill & =& a\ensuremath{\times}a\ensuremath{\times}...\ensuremath{\times}a\hfill & \left(\mathrm{m}\ensuremath{\times}\mathrm{n\; times}\right)\hfill \\ \hfill & =& {a}^{mn}\hfill & \end{array}\tag{5.15}
      \end{equation}
    }
    \typeout{Columnwidth = \the\columnwidth}\typeout{math as usual width = \the\mymathboxwidth}
    \ifthenelse{\lengthtest{\mymathboxwidth < \columnwidth}}{% if the math fits, do it again, for real
    \begin{equation}
    \begin{array}{cccc}\hfill {\left({a}^{m}\right)}^{n}& =& {a}^{m}\ensuremath{\times}{a}^{m}\ensuremath{\times}...\ensuremath{\times}{a}^{m}\hfill & \left(\mathrm{n\; times}\right)\hfill \\ \hfill & =& a\ensuremath{\times}a\ensuremath{\times}...\ensuremath{\times}a\hfill & \left(\mathrm{m}\ensuremath{\times}\mathrm{n\; times}\right)\hfill \\ \hfill & =& {a}^{mn}\hfill & \end{array}\tag{5.15}
      \end{equation}
    }{% else, if it doesn't fit
    \setlength{\mymathboxwidth}{\columnwidth}
      \addtolength{\mymathboxwidth}{-48pt}
    \par\vspace{12pt}\noindent\begin{minipage}{\columnwidth}
    \parbox[t]{\mymathboxwidth}{\large$
    {\left({a}^{m}\right)}^{n}={a}^{m}\ensuremath{\times}{a}^{m}\ensuremath{\times}...\ensuremath{\times}{a}^{m}\left(\mathrm{n\; times}\right)=a\ensuremath{\times}a\ensuremath{\times}...\ensuremath{\times}a\left(\mathrm{m}\ensuremath{\times}\mathrm{n\; times}\right)={a}^{mn}$}\hfill
    \parbox[t]{48pt}{\raggedleft 
    (5.15)}
    \end{minipage}\vspace{12pt}\par
    }% end of conditional for this bit of math
    \typeout{math as usual width = \the\mymathboxwidth}
        \label{m38359*id66694}For example,\par 
        \label{m38359*id66697}\nopagebreak\noindent{}
          \settowidth{\mymathboxwidth}{\begin{equation}
    \begin{array}{ccc}\hfill {\left({2}^{2}\right)}^{3}& =& \left({2}^{2}\right)\ensuremath{\times}\left({2}^{2}\right)\ensuremath{\times}\left({2}^{2}\right)\hfill \\ & =& \left(2\ensuremath{\times}2\right)\ensuremath{\times}\left(2\ensuremath{\times}2\right)\ensuremath{\times}\left(2\ensuremath{\times}2\right)\hfill \\ & =& \left({2}^{6}\right)\hfill \\ & =& {2}^{\left(2\ensuremath{\times}3\right)}\hfill \end{array}\tag{5.16}
      \end{equation}
    }
    \typeout{Columnwidth = \the\columnwidth}\typeout{math as usual width = \the\mymathboxwidth}
    \ifthenelse{\lengthtest{\mymathboxwidth < \columnwidth}}{% if the math fits, do it again, for real
    \begin{equation}
    \begin{array}{ccc}\hfill {\left({2}^{2}\right)}^{3}& =& \left({2}^{2}\right)\ensuremath{\times}\left({2}^{2}\right)\ensuremath{\times}\left({2}^{2}\right)\hfill \\ & =& \left(2\ensuremath{\times}2\right)\ensuremath{\times}\left(2\ensuremath{\times}2\right)\ensuremath{\times}\left(2\ensuremath{\times}2\right)\hfill \\ & =& \left({2}^{6}\right)\hfill \\ & =& {2}^{\left(2\ensuremath{\times}3\right)}\hfill \end{array}\tag{5.16}
      \end{equation}
    }{% else, if it doesn't fit
    \setlength{\mymathboxwidth}{\columnwidth}
      \addtolength{\mymathboxwidth}{-48pt}
    \par\vspace{12pt}\noindent\begin{minipage}{\columnwidth}
    \parbox[t]{\mymathboxwidth}{\large$
    {\left({2}^{2}\right)}^{3}=\left({2}^{2}\right)\ensuremath{\times}\left({2}^{2}\right)\ensuremath{\times}\left({2}^{2}\right)=\left(2\ensuremath{\times}2\right)\ensuremath{\times}\left(2\ensuremath{\times}2\right)\ensuremath{\times}\left(2\ensuremath{\times}2\right)=\left({2}^{6}\right)={2}^{\left(2\ensuremath{\times}3\right)}$}\hfill
    \parbox[t]{48pt}{\raggedleft 
    (5.16)}
    \end{minipage}\vspace{12pt}\par
    }% end of conditional for this bit of math
    \typeout{math as usual width = \the\mymathboxwidth}
\label{m38359*secfhsst!!!underscore!!!id1894}
            \subsubsection{ Application using Exponential Law 6: ${\left({a}^{m}\right)}^{n}={a}^{mn}$ }
            \nopagebreak
        \label{m38359*id66924}\begin{enumerate}[noitemsep, label=\textbf{\arabic*}. ] 
            \label{m38359*uid36}\item 
            ${\left({x}^{3}\right)}^{4}$
      \label{m38359*uid37}\item 
        ${\left[{\left({a}^{4}\right)}^{3}\right]}^{2}$
      \label{m38359*uid38}\item 
        ${\left({3}^{n+3}\right)}^{2}$
\newline
\newline
          \end{enumerate}
\label{m38359*eip-323}\par
            \label{m38359*secfhsst!!!underscore!!!id41}\vspace{.5cm} 
      \noindent
      \hspace*{-30pt}\includegraphics[width=0.5in]{col11306.imgs/pspencil2.png}   \raisebox{25mm}{   
      \begin{mdframed}[linewidth=4, leftmargin=40, rightmargin=40]  
      \begin{exercise}
    \noindent\textbf{Exercise 5.1:  Simplifying indices }
    \label{m38359*probfhsst!!!underscore!!!id42}
    \label{m38359*id117610}Simplify: $\frac{{5}^{2x-1}\ensuremath{\cdot}{9}^{x-2}}{{15}^{2x-3}}$ \par 
    \vspace{5pt}
    \label{m38359*solfhsst!!!underscore!!!id45}\noindent\textbf{Solution to Exercise } \label{m38359*listfhsst!!!underscore!!!id45}\begin{enumerate}[noitemsep, label=\textbf{Step} \textbf{\arabic*}. ] 
            \leftskip=20pt\rightskip=\leftskip\item  
    \label{m38359*id118023}\nopagebreak\noindent{}
      \settowidth{\mymathboxwidth}{\begin{equation}
    \begin{array}{ccc}& =& \frac{{5}^{2x-1}\ensuremath{\cdot}{\left({3}^{2}\right)}^{x-2}}{{\left(5.3\right)}^{2x-3}}\hfill \\ & =& \frac{{5}^{2x-1}\ensuremath{\cdot}{3}^{2x-4}}{{5}^{2x-3}\ensuremath{\cdot}{3}^{2x-3}}\hfill \end{array}\tag{5.17}
      \end{equation}
    }
    \typeout{Columnwidth = \the\columnwidth}\typeout{math as usual width = \the\mymathboxwidth}
    \ifthenelse{\lengthtest{\mymathboxwidth < \columnwidth}}{% if the math fits, do it again, for real
    \begin{equation}
    \begin{array}{ccc}& =& \frac{{5}^{2x-1}\ensuremath{\cdot}{\left({3}^{2}\right)}^{x-2}}{{\left(5.3\right)}^{2x-3}}\hfill \\ & =& \frac{{5}^{2x-1}\ensuremath{\cdot}{3}^{2x-4}}{{5}^{2x-3}\ensuremath{\cdot}{3}^{2x-3}}\hfill \end{array}\tag{5.17}
      \end{equation}
    }{% else, if it doesn't fit
    \setlength{\mymathboxwidth}{\columnwidth}
      \addtolength{\mymathboxwidth}{-48pt}
    \par\vspace{12pt}\noindent\begin{minipage}{\columnwidth}
    \parbox[t]{\mymathboxwidth}{\large$
    =\frac{{5}^{2x-1}\ensuremath{\cdot}{\left({3}^{2}\right)}^{x-2}}{{\left(5.3\right)}^{2x-3}}=\frac{{5}^{2x-1}\ensuremath{\cdot}{3}^{2x-4}}{{5}^{2x-3}\ensuremath{\cdot}{3}^{2x-3}}$}\hfill
    \parbox[t]{48pt}{\raggedleft 
    (5.17)}
    \end{minipage}\vspace{12pt}\par
    }% end of conditional for this bit of math
    \typeout{math as usual width = \the\mymathboxwidth}
    \item  
    \label{m38359*id118205}\nopagebreak\noindent{}
      \settowidth{\mymathboxwidth}{\begin{equation}
    \begin{array}{ccc}& =& {5}^{2x-1-2x+3}\ensuremath{\cdot}{3}^{2x-4-2x+3}\hfill \\ & =& {5}^{2}\ensuremath{\cdot}{3}^{-1}\hfill \end{array}\tag{5.18}
      \end{equation}
    }
    \typeout{Columnwidth = \the\columnwidth}\typeout{math as usual width = \the\mymathboxwidth}
    \ifthenelse{\lengthtest{\mymathboxwidth < \columnwidth}}{% if the math fits, do it again, for real
    \begin{equation}
    \begin{array}{ccc}& =& {5}^{2x-1-2x+3}\ensuremath{\cdot}{3}^{2x-4-2x+3}\hfill \\ & =& {5}^{2}\ensuremath{\cdot}{3}^{-1}\hfill \end{array}\tag{5.18}
      \end{equation}
    }{% else, if it doesn't fit
    \setlength{\mymathboxwidth}{\columnwidth}
      \addtolength{\mymathboxwidth}{-48pt}
    \par\vspace{12pt}\noindent\begin{minipage}{\columnwidth}
    \parbox[t]{\mymathboxwidth}{\large$
    ={5}^{2x-1-2x+3}\ensuremath{\cdot}{3}^{2x-4-2x+3}={5}^{2}\ensuremath{\cdot}{3}^{-1}$}\hfill
    \parbox[t]{48pt}{\raggedleft 
    (5.18)}
    \end{minipage}\vspace{12pt}\par
    }% end of conditional for this bit of math
    \typeout{math as usual width = \the\mymathboxwidth}
    \item  
    \label{m38359*id118312}\nopagebreak\noindent{}
      \settowidth{\mymathboxwidth}{\begin{equation}
    =\frac{25}{3}\tag{5.19}
      \end{equation}
    }
    \typeout{Columnwidth = \the\columnwidth}\typeout{math as usual width = \the\mymathboxwidth}
    \ifthenelse{\lengthtest{\mymathboxwidth < \columnwidth}}{% if the math fits, do it again, for real
    \begin{equation}
    =\frac{25}{3}\tag{5.19}
      \end{equation}
    }{% else, if it doesn't fit
    \setlength{\mymathboxwidth}{\columnwidth}
      \addtolength{\mymathboxwidth}{-48pt}
    \par\vspace{12pt}\noindent\begin{minipage}{\columnwidth}
    \parbox[t]{\mymathboxwidth}{\large$
    =\frac{25}{3}$}\hfill
    \parbox[t]{48pt}{\raggedleft 
    (5.19)}
    \end{minipage}\vspace{12pt}\par
    }% end of conditional for this bit of math
    \typeout{math as usual width = \the\mymathboxwidth}
    \end{enumerate}
    \end{exercise}
    \end{mdframed}
    }
    \noindent
\par 
\label{m38359*secfhsst!!!underscore!!!id2193}
\par \raisebox{-5 pt}{\includegraphics[width=0.5cm]{col11306.imgs/summary_www.png}} Find the answers with the shortcodes:
 \par \begin{tabular}[h]{cccccc}
 (1.) lO6  & \end{tabular}
\section{Rational Exponents}
\section{Exponential Equations}
            \subsubsection{  Exercise : Exponential Numbers }
            \nopagebreak
        \label{m38359*id67549}Match the answers to the questions, by filling in the correct answer into the \textbf{Answer} column.
Possible answers are: $\frac{3}{2}$, 1, \begin{math}-1\end{math}, \begin{math}-\frac{1}{3}\end{math}, 8. Answers may be repeated.\par 
    % \textbf{m38359*id67604}\par
    % how many colspecs?  2
          % name: cnx:colspec
            % colnum: 1
            % colwidth: 10*
            % latex-name: columna
            % colname: 
            % align/tgroup-align/default: //left
            % -------------------------
            % name: cnx:colspec
            % colnum: 2
            % colwidth: 10*
            % latex-name: columnb
            % colname: 
            % align/tgroup-align/default: //left
            % -------------------------
    \setlength\mytablespace{4\tabcolsep}
    \addtolength\mytablespace{3\arrayrulewidth}
    \setlength\mytablewidth{\linewidth}
    \setlength\mytableroom{\mytablewidth}
    \addtolength\mytableroom{-\mytablespace}
    \setlength\myfixedwidth{0pt}
    \setlength\mystarwidth{\mytableroom}
        \addtolength\mystarwidth{-\myfixedwidth}
        \divide\mystarwidth 20
      % ----- Begin capturing width of table in LR mode woof
      \settowidth{\mytableboxwidth}{\begin{tabular}[t]{|l|l|}\hline
    % count in rowspan-info-nodeset: 2
    % align/colidx: left,1
    % rowcount: '0' | start: 'false' | colidx: '1'
        % Formatting a regular cell and recurring on the next sibling
                  \textbf{Question}
                 &
      % align/colidx: left,2
    % rowcount: '0' | start: 'false' | colidx: '2'
        % Formatting a regular cell and recurring on the next sibling
                  \textbf{Answer}
                % make-rowspan-placeholders
    % rowspan info: col1 '0' | 'false' | '' || col2 '0' | 'false' | ''
     \tabularnewline\cline{1-1}\cline{2-2}
      %--------------------------------------------------------------------
    % align/colidx: left,1
    % rowcount: '0' | start: 'false' | colidx: '1'
        % Formatting a regular cell and recurring on the next sibling
                  ${2}^{3}$
                 &
      % align/colidx: left,2
    % rowcount: '0' | start: 'false' | colidx: '2'
        % Formatting a regular cell and recurring on the next sibling
        % make-rowspan-placeholders
    % rowspan info: col1 '0' | 'false' | '' || col2 '0' | 'false' | ''
     \tabularnewline\cline{1-1}\cline{2-2}
      %--------------------------------------------------------------------
    % align/colidx: left,1
    % rowcount: '0' | start: 'false' | colidx: '1'
        % Formatting a regular cell and recurring on the next sibling
                  ${7}^{3-3}$
                 &
      % align/colidx: left,2
    % rowcount: '0' | start: 'false' | colidx: '2'
        % Formatting a regular cell and recurring on the next sibling
        % make-rowspan-placeholders
    % rowspan info: col1 '0' | 'false' | '' || col2 '0' | 'false' | ''
     \tabularnewline\cline{1-1}\cline{2-2}
      %--------------------------------------------------------------------
    % align/colidx: left,1
    % rowcount: '0' | start: 'false' | colidx: '1'
        % Formatting a regular cell and recurring on the next sibling
                  ${\left(\frac{2}{3}\right)}^{-1}$
                 &
      % align/colidx: left,2
    % rowcount: '0' | start: 'false' | colidx: '2'
        % Formatting a regular cell and recurring on the next sibling
        % make-rowspan-placeholders
    % rowspan info: col1 '0' | 'false' | '' || col2 '0' | 'false' | ''
     \tabularnewline\cline{1-1}\cline{2-2}
      %--------------------------------------------------------------------
    % align/colidx: left,1
    % rowcount: '0' | start: 'false' | colidx: '1'
        % Formatting a regular cell and recurring on the next sibling
                  ${8}^{7-6}$
                 &
      % align/colidx: left,2
    % rowcount: '0' | start: 'false' | colidx: '2'
        % Formatting a regular cell and recurring on the next sibling
        % make-rowspan-placeholders
    % rowspan info: col1 '0' | 'false' | '' || col2 '0' | 'false' | ''
     \tabularnewline\cline{1-1}\cline{2-2}
      %--------------------------------------------------------------------
    % align/colidx: left,1
    % rowcount: '0' | start: 'false' | colidx: '1'
        % Formatting a regular cell and recurring on the next sibling
                  ${\left(-3\right)}^{-1}$
                 &
      % align/colidx: left,2
    % rowcount: '0' | start: 'false' | colidx: '2'
        % Formatting a regular cell and recurring on the next sibling
        % make-rowspan-placeholders
    % rowspan info: col1 '0' | 'false' | '' || col2 '0' | 'false' | ''
     \tabularnewline\cline{1-1}\cline{2-2}
      %--------------------------------------------------------------------
    % align/colidx: left,1
    % rowcount: '0' | start: 'false' | colidx: '1'
        % Formatting a regular cell and recurring on the next sibling
                  ${\left(-1\right)}^{23}$
                 &
      % align/colidx: left,2
    % rowcount: '0' | start: 'false' | colidx: '2'
        % Formatting a regular cell and recurring on the next sibling
        % make-rowspan-placeholders
    % rowspan info: col1 '0' | 'false' | '' || col2 '0' | 'false' | ''
     \tabularnewline\cline{1-1}\cline{2-2}
      %--------------------------------------------------------------------
    \end{tabular}} % end mytableboxwidth set      
      % ----- End capturing width of table in LR mode
        % ----- LR or paragraph mode: must test
        % ----- Begin capturing height of table
        \settoheight{\mytableboxheight}{\begin{tabular}[t]{|l|l|}\hline
    % count in rowspan-info-nodeset: 2
    % align/colidx: left,1
    % rowcount: '0' | start: 'false' | colidx: '1'
        % Formatting a regular cell and recurring on the next sibling
                  \textbf{Question}
                 &
      % align/colidx: left,2
    % rowcount: '0' | start: 'false' | colidx: '2'
        % Formatting a regular cell and recurring on the next sibling
                  \textbf{Answer}
                % make-rowspan-placeholders
    % rowspan info: col1 '0' | 'false' | '' || col2 '0' | 'false' | ''
     \tabularnewline\cline{1-1}\cline{2-2}
      %--------------------------------------------------------------------
    % align/colidx: left,1
    % rowcount: '0' | start: 'false' | colidx: '1'
        % Formatting a regular cell and recurring on the next sibling
                  ${2}^{3}$
                 &
      % align/colidx: left,2
    % rowcount: '0' | start: 'false' | colidx: '2'
        % Formatting a regular cell and recurring on the next sibling
        % make-rowspan-placeholders
    % rowspan info: col1 '0' | 'false' | '' || col2 '0' | 'false' | ''
     \tabularnewline\cline{1-1}\cline{2-2}
      %--------------------------------------------------------------------
    % align/colidx: left,1
    % rowcount: '0' | start: 'false' | colidx: '1'
        % Formatting a regular cell and recurring on the next sibling
                  ${7}^{3-3}$
                 &
      % align/colidx: left,2
    % rowcount: '0' | start: 'false' | colidx: '2'
        % Formatting a regular cell and recurring on the next sibling
        % make-rowspan-placeholders
    % rowspan info: col1 '0' | 'false' | '' || col2 '0' | 'false' | ''
     \tabularnewline\cline{1-1}\cline{2-2}
      %--------------------------------------------------------------------
    % align/colidx: left,1
    % rowcount: '0' | start: 'false' | colidx: '1'
        % Formatting a regular cell and recurring on the next sibling
                  ${\left(\frac{2}{3}\right)}^{-1}$
                 &
      % align/colidx: left,2
    % rowcount: '0' | start: 'false' | colidx: '2'
        % Formatting a regular cell and recurring on the next sibling
        % make-rowspan-placeholders
    % rowspan info: col1 '0' | 'false' | '' || col2 '0' | 'false' | ''
     \tabularnewline\cline{1-1}\cline{2-2}
      %--------------------------------------------------------------------
    % align/colidx: left,1
    % rowcount: '0' | start: 'false' | colidx: '1'
        % Formatting a regular cell and recurring on the next sibling
                  ${8}^{7-6}$
                 &
      % align/colidx: left,2
    % rowcount: '0' | start: 'false' | colidx: '2'
        % Formatting a regular cell and recurring on the next sibling
        % make-rowspan-placeholders
    % rowspan info: col1 '0' | 'false' | '' || col2 '0' | 'false' | ''
     \tabularnewline\cline{1-1}\cline{2-2}
      %--------------------------------------------------------------------
    % align/colidx: left,1
    % rowcount: '0' | start: 'false' | colidx: '1'
        % Formatting a regular cell and recurring on the next sibling
                  ${\left(-3\right)}^{-1}$
                 &
      % align/colidx: left,2
    % rowcount: '0' | start: 'false' | colidx: '2'
        % Formatting a regular cell and recurring on the next sibling
        % make-rowspan-placeholders
    % rowspan info: col1 '0' | 'false' | '' || col2 '0' | 'false' | ''
     \tabularnewline\cline{1-1}\cline{2-2}
      %--------------------------------------------------------------------
    % align/colidx: left,1
    % rowcount: '0' | start: 'false' | colidx: '1'
        % Formatting a regular cell and recurring on the next sibling
                  ${\left(-1\right)}^{23}$
                 &
      % align/colidx: left,2
    % rowcount: '0' | start: 'false' | colidx: '2'
        % Formatting a regular cell and recurring on the next sibling
        % make-rowspan-placeholders
    % rowspan info: col1 '0' | 'false' | '' || col2 '0' | 'false' | ''
     \tabularnewline\cline{1-1}\cline{2-2}
      %--------------------------------------------------------------------
    \end{tabular}} % end mytableboxheight set
        \settodepth{\mytableboxdepth}{\begin{tabular}[t]{|l|l|}\hline
    % count in rowspan-info-nodeset: 2
    % align/colidx: left,1
    % rowcount: '0' | start: 'false' | colidx: '1'
        % Formatting a regular cell and recurring on the next sibling
                  \textbf{Question}
                 &
      % align/colidx: left,2
    % rowcount: '0' | start: 'false' | colidx: '2'
        % Formatting a regular cell and recurring on the next sibling
                  \textbf{Answer}
                % make-rowspan-placeholders
    % rowspan info: col1 '0' | 'false' | '' || col2 '0' | 'false' | ''
     \tabularnewline\cline{1-1}\cline{2-2}
      %--------------------------------------------------------------------
    % align/colidx: left,1
    % rowcount: '0' | start: 'false' | colidx: '1'
        % Formatting a regular cell and recurring on the next sibling
                  ${2}^{3}$
                 &
      % align/colidx: left,2
    % rowcount: '0' | start: 'false' | colidx: '2'
        % Formatting a regular cell and recurring on the next sibling
        % make-rowspan-placeholders
    % rowspan info: col1 '0' | 'false' | '' || col2 '0' | 'false' | ''
     \tabularnewline\cline{1-1}\cline{2-2}
      %--------------------------------------------------------------------
    % align/colidx: left,1
    % rowcount: '0' | start: 'false' | colidx: '1'
        % Formatting a regular cell and recurring on the next sibling
                  ${7}^{3-3}$
                 &
      % align/colidx: left,2
    % rowcount: '0' | start: 'false' | colidx: '2'
        % Formatting a regular cell and recurring on the next sibling
        % make-rowspan-placeholders
    % rowspan info: col1 '0' | 'false' | '' || col2 '0' | 'false' | ''
     \tabularnewline\cline{1-1}\cline{2-2}
      %--------------------------------------------------------------------
    % align/colidx: left,1
    % rowcount: '0' | start: 'false' | colidx: '1'
        % Formatting a regular cell and recurring on the next sibling
                  ${\left(\frac{2}{3}\right)}^{-1}$
                 &
      % align/colidx: left,2
    % rowcount: '0' | start: 'false' | colidx: '2'
        % Formatting a regular cell and recurring on the next sibling
        % make-rowspan-placeholders
    % rowspan info: col1 '0' | 'false' | '' || col2 '0' | 'false' | ''
     \tabularnewline\cline{1-1}\cline{2-2}
      %--------------------------------------------------------------------
    % align/colidx: left,1
    % rowcount: '0' | start: 'false' | colidx: '1'
        % Formatting a regular cell and recurring on the next sibling
                  ${8}^{7-6}$
                 &
      % align/colidx: left,2
    % rowcount: '0' | start: 'false' | colidx: '2'
        % Formatting a regular cell and recurring on the next sibling
        % make-rowspan-placeholders
    % rowspan info: col1 '0' | 'false' | '' || col2 '0' | 'false' | ''
     \tabularnewline\cline{1-1}\cline{2-2}
      %--------------------------------------------------------------------
    % align/colidx: left,1
    % rowcount: '0' | start: 'false' | colidx: '1'
        % Formatting a regular cell and recurring on the next sibling
                  ${\left(-3\right)}^{-1}$
                 &
      % align/colidx: left,2
    % rowcount: '0' | start: 'false' | colidx: '2'
        % Formatting a regular cell and recurring on the next sibling
        % make-rowspan-placeholders
    % rowspan info: col1 '0' | 'false' | '' || col2 '0' | 'false' | ''
     \tabularnewline\cline{1-1}\cline{2-2}
      %--------------------------------------------------------------------
    % align/colidx: left,1
    % rowcount: '0' | start: 'false' | colidx: '1'
        % Formatting a regular cell and recurring on the next sibling
                  ${\left(-1\right)}^{23}$
                 &
      % align/colidx: left,2
    % rowcount: '0' | start: 'false' | colidx: '2'
        % Formatting a regular cell and recurring on the next sibling
        % make-rowspan-placeholders
    % rowspan info: col1 '0' | 'false' | '' || col2 '0' | 'false' | ''
     \tabularnewline\cline{1-1}\cline{2-2}
      %--------------------------------------------------------------------
    \end{tabular}} % end mytableboxdepth set
        \addtolength{\mytableboxheight}{\mytableboxdepth}
        % ----- End capturing height of table        
        \ifthenelse{\mytableboxwidth<\textwidth}{% the table fits in LR mode
          \addtolength{\mytableboxwidth}{-\mytablespace}
          \typeout{textheight: \the\textheight}
          \typeout{mytableboxheight: \the\mytableboxheight}
          \typeout{textwidth: \the\textwidth}
          \typeout{mytableboxwidth: \the\mytableboxwidth}
          \ifthenelse{\mytableboxheight<\textheight}{%
    % \begin{table}[H]
    % \\ '' '0'
        \begin{center}
      \label{m38359*id67604}
    \noindent
    \begin{tabular}[t]{|l|l|}\hline
    % count in rowspan-info-nodeset: 2
    % align/colidx: left,1
    % rowcount: '0' | start: 'false' | colidx: '1'
        % Formatting a regular cell and recurring on the next sibling
                  \textbf{Question}
                 &
      % align/colidx: left,2
    % rowcount: '0' | start: 'false' | colidx: '2'
        % Formatting a regular cell and recurring on the next sibling
                  \textbf{Answer}
                % make-rowspan-placeholders
    % rowspan info: col1 '0' | 'false' | '' || col2 '0' | 'false' | ''
     \tabularnewline\cline{1-1}\cline{2-2}
      %--------------------------------------------------------------------
    % align/colidx: left,1
    % rowcount: '0' | start: 'false' | colidx: '1'
        % Formatting a regular cell and recurring on the next sibling
                  ${2}^{3}$
                 &
      % align/colidx: left,2
    % rowcount: '0' | start: 'false' | colidx: '2'
        % Formatting a regular cell and recurring on the next sibling
        % make-rowspan-placeholders
    % rowspan info: col1 '0' | 'false' | '' || col2 '0' | 'false' | ''
     \tabularnewline\cline{1-1}\cline{2-2}
      %--------------------------------------------------------------------
    % align/colidx: left,1
    % rowcount: '0' | start: 'false' | colidx: '1'
        % Formatting a regular cell and recurring on the next sibling
                  ${7}^{3-3}$
                 &
      % align/colidx: left,2
    % rowcount: '0' | start: 'false' | colidx: '2'
        % Formatting a regular cell and recurring on the next sibling
        % make-rowspan-placeholders
    % rowspan info: col1 '0' | 'false' | '' || col2 '0' | 'false' | ''
     \tabularnewline\cline{1-1}\cline{2-2}
      %--------------------------------------------------------------------
    % align/colidx: left,1
    % rowcount: '0' | start: 'false' | colidx: '1'
        % Formatting a regular cell and recurring on the next sibling
                  ${\left(\frac{2}{3}\right)}^{-1}$
                 &
      % align/colidx: left,2
    % rowcount: '0' | start: 'false' | colidx: '2'
        % Formatting a regular cell and recurring on the next sibling
        % make-rowspan-placeholders
    % rowspan info: col1 '0' | 'false' | '' || col2 '0' | 'false' | ''
     \tabularnewline\cline{1-1}\cline{2-2}
      %--------------------------------------------------------------------
    % align/colidx: left,1
    % rowcount: '0' | start: 'false' | colidx: '1'
        % Formatting a regular cell and recurring on the next sibling
                  ${8}^{7-6}$
                 &
      % align/colidx: left,2
    % rowcount: '0' | start: 'false' | colidx: '2'
        % Formatting a regular cell and recurring on the next sibling
        % make-rowspan-placeholders
    % rowspan info: col1 '0' | 'false' | '' || col2 '0' | 'false' | ''
     \tabularnewline\cline{1-1}\cline{2-2}
      %--------------------------------------------------------------------
    % align/colidx: left,1
    % rowcount: '0' | start: 'false' | colidx: '1'
        % Formatting a regular cell and recurring on the next sibling
                  ${\left(-3\right)}^{-1}$
                 &
      % align/colidx: left,2
    % rowcount: '0' | start: 'false' | colidx: '2'
        % Formatting a regular cell and recurring on the next sibling
        % make-rowspan-placeholders
    % rowspan info: col1 '0' | 'false' | '' || col2 '0' | 'false' | ''
     \tabularnewline\cline{1-1}\cline{2-2}
      %--------------------------------------------------------------------
    % align/colidx: left,1
    % rowcount: '0' | start: 'false' | colidx: '1'
        % Formatting a regular cell and recurring on the next sibling
                  ${\left(-1\right)}^{23}$
                 &
      % align/colidx: left,2
    % rowcount: '0' | start: 'false' | colidx: '2'
        % Formatting a regular cell and recurring on the next sibling
        % make-rowspan-placeholders
    % rowspan info: col1 '0' | 'false' | '' || col2 '0' | 'false' | ''
     \tabularnewline\cline{1-1}\cline{2-2}
      %--------------------------------------------------------------------
    \end{tabular}
      \end{center}
    \begin{center}{\small\bfseries Table 5.1}\end{center}
    %\end{table}
          }{ % else
    % \begin{table}[H]
    % \\ '' '0'
        \begin{center}
      \label{m38359*id67604}
    \noindent
    \tabletail{%
        \hline
        \multicolumn{2}{|p{\mytableboxwidth}|}{\raggedleft \small \sl continued on next page}\\
        \hline
      }
      \tablelasttail{}
      \begin{xtabular}[t]{|l|l|}\hline
    % count in rowspan-info-nodeset: 2
    % align/colidx: left,1
    % rowcount: '0' | start: 'false' | colidx: '1'
        % Formatting a regular cell and recurring on the next sibling
                  \textbf{Question}
                 &
      % align/colidx: left,2
    % rowcount: '0' | start: 'false' | colidx: '2'
        % Formatting a regular cell and recurring on the next sibling
                  \textbf{Answer}
                % make-rowspan-placeholders
    % rowspan info: col1 '0' | 'false' | '' || col2 '0' | 'false' | ''
     \tabularnewline\cline{1-1}\cline{2-2}
      %--------------------------------------------------------------------
    % align/colidx: left,1
    % rowcount: '0' | start: 'false' | colidx: '1'
        % Formatting a regular cell and recurring on the next sibling
                  ${2}^{3}$
                 &
      % align/colidx: left,2
    % rowcount: '0' | start: 'false' | colidx: '2'
        % Formatting a regular cell and recurring on the next sibling
        % make-rowspan-placeholders
    % rowspan info: col1 '0' | 'false' | '' || col2 '0' | 'false' | ''
     \tabularnewline\cline{1-1}\cline{2-2}
      %--------------------------------------------------------------------
    % align/colidx: left,1
    % rowcount: '0' | start: 'false' | colidx: '1'
        % Formatting a regular cell and recurring on the next sibling
                  ${7}^{3-3}$
                 &
      % align/colidx: left,2
    % rowcount: '0' | start: 'false' | colidx: '2'
        % Formatting a regular cell and recurring on the next sibling
        % make-rowspan-placeholders
    % rowspan info: col1 '0' | 'false' | '' || col2 '0' | 'false' | ''
     \tabularnewline\cline{1-1}\cline{2-2}
      %--------------------------------------------------------------------
    % align/colidx: left,1
    % rowcount: '0' | start: 'false' | colidx: '1'
        % Formatting a regular cell and recurring on the next sibling
                  ${\left(\frac{2}{3}\right)}^{-1}$
                 &
      % align/colidx: left,2
    % rowcount: '0' | start: 'false' | colidx: '2'
        % Formatting a regular cell and recurring on the next sibling
        % make-rowspan-placeholders
    % rowspan info: col1 '0' | 'false' | '' || col2 '0' | 'false' | ''
     \tabularnewline\cline{1-1}\cline{2-2}
      %--------------------------------------------------------------------
    % align/colidx: left,1
    % rowcount: '0' | start: 'false' | colidx: '1'
        % Formatting a regular cell and recurring on the next sibling
                  ${8}^{7-6}$
                 &
      % align/colidx: left,2
    % rowcount: '0' | start: 'false' | colidx: '2'
        % Formatting a regular cell and recurring on the next sibling
        % make-rowspan-placeholders
    % rowspan info: col1 '0' | 'false' | '' || col2 '0' | 'false' | ''
     \tabularnewline\cline{1-1}\cline{2-2}
      %--------------------------------------------------------------------
    % align/colidx: left,1
    % rowcount: '0' | start: 'false' | colidx: '1'
        % Formatting a regular cell and recurring on the next sibling
                  ${\left(-3\right)}^{-1}$
                 &
      % align/colidx: left,2
    % rowcount: '0' | start: 'false' | colidx: '2'
        % Formatting a regular cell and recurring on the next sibling
        % make-rowspan-placeholders
    % rowspan info: col1 '0' | 'false' | '' || col2 '0' | 'false' | ''
     \tabularnewline\cline{1-1}\cline{2-2}
      %--------------------------------------------------------------------
    % align/colidx: left,1
    % rowcount: '0' | start: 'false' | colidx: '1'
        % Formatting a regular cell and recurring on the next sibling
                  ${\left(-1\right)}^{23}$
                 &
      % align/colidx: left,2
    % rowcount: '0' | start: 'false' | colidx: '2'
        % Formatting a regular cell and recurring on the next sibling
        % make-rowspan-placeholders
    % rowspan info: col1 '0' | 'false' | '' || col2 '0' | 'false' | ''
     \tabularnewline\cline{1-1}\cline{2-2}
      %--------------------------------------------------------------------
    \end{xtabular}
      \end{center}
    \begin{center}{\small\bfseries Table 5.1}\end{center}
    %\end{table}
          } % 
        }{% else
        % typeset the table in paragraph mode
        % ----- Begin capturing height of table
        \settoheight{\mytableboxheight}{\begin{tabular*}{\mytablewidth}[t]{|p{10\mystarwidth}|p{10\mystarwidth}|}\hline
    % count in rowspan-info-nodeset: 2
    % align/colidx: left,1
    % rowcount: '0' | start: 'false' | colidx: '1'
        % Formatting a regular cell and recurring on the next sibling
                  \textbf{Question}
                 &
      % align/colidx: left,2
    % rowcount: '0' | start: 'false' | colidx: '2'
        % Formatting a regular cell and recurring on the next sibling
                  \textbf{Answer}
                % make-rowspan-placeholders
    % rowspan info: col1 '0' | 'false' | '' || col2 '0' | 'false' | ''
     \tabularnewline\cline{1-1}\cline{2-2}
      %--------------------------------------------------------------------
    % align/colidx: left,1
    % rowcount: '0' | start: 'false' | colidx: '1'
        % Formatting a regular cell and recurring on the next sibling
                  ${2}^{3}$
                 &
      % align/colidx: left,2
    % rowcount: '0' | start: 'false' | colidx: '2'
        % Formatting a regular cell and recurring on the next sibling
        % make-rowspan-placeholders
    % rowspan info: col1 '0' | 'false' | '' || col2 '0' | 'false' | ''
     \tabularnewline\cline{1-1}\cline{2-2}
      %--------------------------------------------------------------------
    % align/colidx: left,1
    % rowcount: '0' | start: 'false' | colidx: '1'
        % Formatting a regular cell and recurring on the next sibling
                  ${7}^{3-3}$
                 &
      % align/colidx: left,2
    % rowcount: '0' | start: 'false' | colidx: '2'
        % Formatting a regular cell and recurring on the next sibling
        % make-rowspan-placeholders
    % rowspan info: col1 '0' | 'false' | '' || col2 '0' | 'false' | ''
     \tabularnewline\cline{1-1}\cline{2-2}
      %--------------------------------------------------------------------
    % align/colidx: left,1
    % rowcount: '0' | start: 'false' | colidx: '1'
        % Formatting a regular cell and recurring on the next sibling
                  ${\left(\frac{2}{3}\right)}^{-1}$
                 &
      % align/colidx: left,2
    % rowcount: '0' | start: 'false' | colidx: '2'
        % Formatting a regular cell and recurring on the next sibling
        % make-rowspan-placeholders
    % rowspan info: col1 '0' | 'false' | '' || col2 '0' | 'false' | ''
     \tabularnewline\cline{1-1}\cline{2-2}
      %--------------------------------------------------------------------
    % align/colidx: left,1
    % rowcount: '0' | start: 'false' | colidx: '1'
        % Formatting a regular cell and recurring on the next sibling
                  ${8}^{7-6}$
                 &
      % align/colidx: left,2
    % rowcount: '0' | start: 'false' | colidx: '2'
        % Formatting a regular cell and recurring on the next sibling
        % make-rowspan-placeholders
    % rowspan info: col1 '0' | 'false' | '' || col2 '0' | 'false' | ''
     \tabularnewline\cline{1-1}\cline{2-2}
      %--------------------------------------------------------------------
    % align/colidx: left,1
    % rowcount: '0' | start: 'false' | colidx: '1'
        % Formatting a regular cell and recurring on the next sibling
                  ${\left(-3\right)}^{-1}$
                 &
      % align/colidx: left,2
    % rowcount: '0' | start: 'false' | colidx: '2'
        % Formatting a regular cell and recurring on the next sibling
        % make-rowspan-placeholders
    % rowspan info: col1 '0' | 'false' | '' || col2 '0' | 'false' | ''
     \tabularnewline\cline{1-1}\cline{2-2}
      %--------------------------------------------------------------------
    % align/colidx: left,1
    % rowcount: '0' | start: 'false' | colidx: '1'
        % Formatting a regular cell and recurring on the next sibling
                  ${\left(-1\right)}^{23}$
                 &
      % align/colidx: left,2
    % rowcount: '0' | start: 'false' | colidx: '2'
        % Formatting a regular cell and recurring on the next sibling
        % make-rowspan-placeholders
    % rowspan info: col1 '0' | 'false' | '' || col2 '0' | 'false' | ''
     \tabularnewline\cline{1-1}\cline{2-2}
      %--------------------------------------------------------------------
    \end{tabular*}} % end mytableboxheight set
        \settodepth{\mytableboxdepth}{\begin{tabular*}{\mytablewidth}[t]{|p{10\mystarwidth}|p{10\mystarwidth}|}\hline
    % count in rowspan-info-nodeset: 2
    % align/colidx: left,1
    % rowcount: '0' | start: 'false' | colidx: '1'
        % Formatting a regular cell and recurring on the next sibling
                  \textbf{Question}
                 &
      % align/colidx: left,2
    % rowcount: '0' | start: 'false' | colidx: '2'
        % Formatting a regular cell and recurring on the next sibling
                  \textbf{Answer}
                % make-rowspan-placeholders
    % rowspan info: col1 '0' | 'false' | '' || col2 '0' | 'false' | ''
     \tabularnewline\cline{1-1}\cline{2-2}
      %--------------------------------------------------------------------
    % align/colidx: left,1
    % rowcount: '0' | start: 'false' | colidx: '1'
        % Formatting a regular cell and recurring on the next sibling
                  ${2}^{3}$
                 &
      % align/colidx: left,2
    % rowcount: '0' | start: 'false' | colidx: '2'
        % Formatting a regular cell and recurring on the next sibling
        % make-rowspan-placeholders
    % rowspan info: col1 '0' | 'false' | '' || col2 '0' | 'false' | ''
     \tabularnewline\cline{1-1}\cline{2-2}
      %--------------------------------------------------------------------
    % align/colidx: left,1
    % rowcount: '0' | start: 'false' | colidx: '1'
        % Formatting a regular cell and recurring on the next sibling
                  ${7}^{3-3}$
                 &
      % align/colidx: left,2
    % rowcount: '0' | start: 'false' | colidx: '2'
        % Formatting a regular cell and recurring on the next sibling
        % make-rowspan-placeholders
    % rowspan info: col1 '0' | 'false' | '' || col2 '0' | 'false' | ''
     \tabularnewline\cline{1-1}\cline{2-2}
      %--------------------------------------------------------------------
    % align/colidx: left,1
    % rowcount: '0' | start: 'false' | colidx: '1'
        % Formatting a regular cell and recurring on the next sibling
                  ${\left(\frac{2}{3}\right)}^{-1}$
                 &
      % align/colidx: left,2
    % rowcount: '0' | start: 'false' | colidx: '2'
        % Formatting a regular cell and recurring on the next sibling
        % make-rowspan-placeholders
    % rowspan info: col1 '0' | 'false' | '' || col2 '0' | 'false' | ''
     \tabularnewline\cline{1-1}\cline{2-2}
      %--------------------------------------------------------------------
    % align/colidx: left,1
    % rowcount: '0' | start: 'false' | colidx: '1'
        % Formatting a regular cell and recurring on the next sibling
                  ${8}^{7-6}$
                 &
      % align/colidx: left,2
    % rowcount: '0' | start: 'false' | colidx: '2'
        % Formatting a regular cell and recurring on the next sibling
        % make-rowspan-placeholders
    % rowspan info: col1 '0' | 'false' | '' || col2 '0' | 'false' | ''
     \tabularnewline\cline{1-1}\cline{2-2}
      %--------------------------------------------------------------------
    % align/colidx: left,1
    % rowcount: '0' | start: 'false' | colidx: '1'
        % Formatting a regular cell and recurring on the next sibling
                  ${\left(-3\right)}^{-1}$
                 &
      % align/colidx: left,2
    % rowcount: '0' | start: 'false' | colidx: '2'
        % Formatting a regular cell and recurring on the next sibling
        % make-rowspan-placeholders
    % rowspan info: col1 '0' | 'false' | '' || col2 '0' | 'false' | ''
     \tabularnewline\cline{1-1}\cline{2-2}
      %--------------------------------------------------------------------
    % align/colidx: left,1
    % rowcount: '0' | start: 'false' | colidx: '1'
        % Formatting a regular cell and recurring on the next sibling
                  ${\left(-1\right)}^{23}$
                 &
      % align/colidx: left,2
    % rowcount: '0' | start: 'false' | colidx: '2'
        % Formatting a regular cell and recurring on the next sibling
        % make-rowspan-placeholders
    % rowspan info: col1 '0' | 'false' | '' || col2 '0' | 'false' | ''
     \tabularnewline\cline{1-1}\cline{2-2}
      %--------------------------------------------------------------------
    \end{tabular*}} % end mytableboxdepth set
        \addtolength{\mytableboxheight}{\mytableboxdepth}
        % ----- End capturing height of table
        \typeout{textheight: \the\textheight}
        \typeout{mytableboxheight: \the\mytableboxheight}
        \typeout{table too wide, outputting in para mode}
    % \begin{table}[H]
    % \\ '' '0'
        \begin{center}
      \label{m38359*id67604}
    \noindent
    \tabletail{%
        \hline
        \multicolumn{2}{|p{\mytableroom}|}{\raggedleft \small \sl continued on next page}\\
        \hline
      }
      \tablelasttail{}
      \begin{xtabular*}{\mytablewidth}[t]{|p{10\mystarwidth}|p{10\mystarwidth}|}\hline
    % count in rowspan-info-nodeset: 2
    % align/colidx: left,1
    % rowcount: '0' | start: 'false' | colidx: '1'
        % Formatting a regular cell and recurring on the next sibling
                  \textbf{Question}
                 &
      % align/colidx: left,2
    % rowcount: '0' | start: 'false' | colidx: '2'
        % Formatting a regular cell and recurring on the next sibling
                  \textbf{Answer}
                % make-rowspan-placeholders
    % rowspan info: col1 '0' | 'false' | '' || col2 '0' | 'false' | ''
     \tabularnewline\cline{1-1}\cline{2-2}
      %--------------------------------------------------------------------
    % align/colidx: left,1
    % rowcount: '0' | start: 'false' | colidx: '1'
        % Formatting a regular cell and recurring on the next sibling
                  ${2}^{3}$
                 &
      % align/colidx: left,2
    % rowcount: '0' | start: 'false' | colidx: '2'
        % Formatting a regular cell and recurring on the next sibling
        % make-rowspan-placeholders
    % rowspan info: col1 '0' | 'false' | '' || col2 '0' | 'false' | ''
     \tabularnewline\cline{1-1}\cline{2-2}
      %--------------------------------------------------------------------
    % align/colidx: left,1
    % rowcount: '0' | start: 'false' | colidx: '1'
        % Formatting a regular cell and recurring on the next sibling
                  ${7}^{3-3}$
                 &
      % align/colidx: left,2
    % rowcount: '0' | start: 'false' | colidx: '2'
        % Formatting a regular cell and recurring on the next sibling
        % make-rowspan-placeholders
    % rowspan info: col1 '0' | 'false' | '' || col2 '0' | 'false' | ''
     \tabularnewline\cline{1-1}\cline{2-2}
      %--------------------------------------------------------------------
    % align/colidx: left,1
    % rowcount: '0' | start: 'false' | colidx: '1'
        % Formatting a regular cell and recurring on the next sibling
                  ${\left(\frac{2}{3}\right)}^{-1}$
                 &
      % align/colidx: left,2
    % rowcount: '0' | start: 'false' | colidx: '2'
        % Formatting a regular cell and recurring on the next sibling
        % make-rowspan-placeholders
    % rowspan info: col1 '0' | 'false' | '' || col2 '0' | 'false' | ''
     \tabularnewline\cline{1-1}\cline{2-2}
      %--------------------------------------------------------------------
    % align/colidx: left,1
    % rowcount: '0' | start: 'false' | colidx: '1'
        % Formatting a regular cell and recurring on the next sibling
                  ${8}^{7-6}$
                 &
      % align/colidx: left,2
    % rowcount: '0' | start: 'false' | colidx: '2'
        % Formatting a regular cell and recurring on the next sibling
        % make-rowspan-placeholders
    % rowspan info: col1 '0' | 'false' | '' || col2 '0' | 'false' | ''
     \tabularnewline\cline{1-1}\cline{2-2}
      %--------------------------------------------------------------------
    % align/colidx: left,1
    % rowcount: '0' | start: 'false' | colidx: '1'
        % Formatting a regular cell and recurring on the next sibling
                  ${\left(-3\right)}^{-1}$
                 &
      % align/colidx: left,2
    % rowcount: '0' | start: 'false' | colidx: '2'
        % Formatting a regular cell and recurring on the next sibling
        % make-rowspan-placeholders
    % rowspan info: col1 '0' | 'false' | '' || col2 '0' | 'false' | ''
     \tabularnewline\cline{1-1}\cline{2-2}
      %--------------------------------------------------------------------
    % align/colidx: left,1
    % rowcount: '0' | start: 'false' | colidx: '1'
        % Formatting a regular cell and recurring on the next sibling
                  ${\left(-1\right)}^{23}$
                 &
      % align/colidx: left,2
    % rowcount: '0' | start: 'false' | colidx: '2'
        % Formatting a regular cell and recurring on the next sibling
        % make-rowspan-placeholders
    % rowspan info: col1 '0' | 'false' | '' || col2 '0' | 'false' | ''
     \tabularnewline\cline{1-1}\cline{2-2}
      %--------------------------------------------------------------------
    \end{xtabular*}
      \end{center}
    \begin{center}{\small\bfseries Table 5.1}\end{center}
    %\end{table}
        }% ending lr/para test clause
    \par
      \label{m38359*eip-117}We will use all these laws in Equations and Inequalities\footnote{\raggedright{}"Equations and Inequalities - Grade 10 [CAPS]" <http://http://cnx.org/content/m38372/latest/>} to help us solve exponential equations.\par \label{m38359*eip-160}The following video gives an example on using some of the concepts covered in this chapter.
    \setcounter{subfigure}{0}
	\begin{figure}[H] % horizontal\label{m38359*ExponentsLaw3}
    \textnormal{Khan Academy video on Exponents - 5}\vspace{.1in} \nopagebreak
  \label{m38359*yt-media5}\label{m38359*yt-video5}
            \raisebox{-5 pt}{ \includegraphics[width=0.5cm]{col11306.imgs/summary_www.png}} { (Video:  MG10049 )}
      \vspace{2pt}
    \vspace{.1in}
 \end{figure}       \par 
    \section{ Summary}
            \nopagebreak
            \label{m38359*eip-908} $ \hspace{-5pt}\begin{array}{cccccccccccc}   \end{array} $ \hspace{2 pt}\raisebox{-5 pt}{\includegraphics[width=0.5cm]{col11306.imgs/summary_www.png}} {(section shortcode: MG10050 )} \par \label{m38359*eip-948}\begin{itemize}[noitemsep, label=\textbullet{}]
            \item Exponential notation means a number written like ${a}^{n}$ where \begin{math}n\end{math} is an integer and \begin{math}a\end{math} can be any real number.\item \begin{math}a\end{math} is called the \textsl{base} and \begin{math}n\end{math} is called the \textsl{exponent} or \textsl{index}.\item The \begin{math}{n}^{\mathrm{th}}\end{math} power of \begin{math}a\end{math} is defined as: \begin{math}{a}^{n}=a\ensuremath{\times}a\ensuremath{\times}\cdots \ensuremath{\times}a\phantom{\rule{2.em}{0ex}}\left(\mathrm{n\; times}\right)\end{math}\item  There are six laws of exponents: \label{m38359*uid098233}\begin{itemize}[noitemsep]
            \item Exponential Law 1: ${a}^{0}=1$\item Exponential Law 2: \begin{math}{a}^{m}\ensuremath{\times}{a}^{n}={a}^{m+n}\end{math}\item Exponential Law 3: \begin{math}{a}^{-n}=\frac{1}{{a}^{n}},a\ne 0\end{math}\item Exponential Law 4: \begin{math}{a}^{m}÷{a}^{n}={a}^{m-n}\end{math}\item Exponential Law 5: \begin{math}{\left(ab\right)}^{n}={a}^{n}{b}^{n}\end{math}\item Exponential Law 6: \begin{math}{\left({a}^{m}\right)}^{n}={a}^{mn}\end{math}\end{itemize}
        \end{itemize}
        \section{ End of Chapter Exercises}
            \nopagebreak
            \label{m38359*cid5} $ \hspace{-5pt}\begin{array}{cccccccccccc}   \end{array} $ \hspace{2 pt}\raisebox{-5 pt}{\includegraphics[width=0.5cm]{col11306.imgs/summary_www.png}} {(section shortcode: MG10051 )} \par 
      \label{m38359*id67892}\begin{enumerate}[noitemsep, label=\textbf{\arabic*}. ] 
            \label{m38359*uid39}\item Simplify as far as possible:
\label{m38359*id67908}\begin{enumerate}[noitemsep, label=\textbf{\alph*}. ] 
            \label{m38359*uid56}\item 
  ${302}^{0}$
   \label{m38359*uid57}\item 
   ${1}^{0}$
    \label{m38359*uid58}\item  
   ${\left(xyz\right)}^{0}$
    \label{m38359*uid59}\item  
   ${\left[{\left(3{x}^{4}{y}^{7}{z}^{12}\right)}^{5}{\left(-5{x}^{9}{y}^{3}{z}^{4}\right)}^{2}\right]}^{0}$
    \label{m38359*uid60}\item  
    ${\left(2x\right)}^{3}$
    \label{m38359*uid61}\item 
    ${\left(-2x\right)}^{3}$
    \label{m38359*uid62}\item 
    ${\left(2x\right)}^{4}$
    \label{m38359*uid63}\item 
    ${\left(-2x\right)}^{4}$
    \end{enumerate}
    \label{m38359*uid40}\item Simplify without using a calculator. Leave your answers with positive exponents.
\label{m38359*id68198}\begin{enumerate}[noitemsep, label=\textbf{\alph*}. ] 
            \label{m38359*uid41}\item $\frac{3{x}^{-3}}{{\left(3x\right)}^{2}}$\newline
    \label{m38359*uid42}\item $5{x}^{0}+{8}^{-2}-{\left(\frac{1}{2}\right)}^{-2}\ensuremath{\cdot}{1}^{x}$\label{m38359*uid43}\item \begin{math}\frac{{5}^{b-3}}{{5}^{b+1}}\end{math}
\end{enumerate}
                \label{m38359*uid44}\item Simplify, showing all steps:
\label{m38359*id68378}\begin{enumerate}[noitemsep, label=\textbf{\alph*}. ] 
            \label{m38359*uid45}\item $\frac{{2}^{a-2}.{3}^{a+3}}{{6}^{a}}$\newline
    \label{m38359*uid46}\item $\frac{{a}^{2m+n+p}}{{a}^{m+n+p}\ensuremath{\cdot}{a}^{m}}$\newline
    \label{m38359*uid47}\item $\frac{{3}^{n}\ensuremath{\cdot}{9}^{n-3}}{{27}^{n-1}}$\newline
    \label{m38359*uid48}\item ${\left(\frac{2{x}^{2a}}{{y}^{-b}}\right)}^{3}$\newline
    \newline
    \label{m38359*uid49}\item $\frac{{2}^{3x-1}\ensuremath{\cdot}{8}^{x+1}}{{4}^{2x-2}}$\newline
    \label{m38359*uid50}\item $\frac{{6}^{2x}\ensuremath{\cdot}{11}^{2x}}{{22}^{2x-1}\ensuremath{\cdot}{3}^{2x}}$
\end{enumerate}
                \label{m38359*uid51}\item Simplify, without using a calculator:
\label{m38359*id68759}\begin{enumerate}[noitemsep, label=\textbf{\alph*}. ] 
            \label{m38359*uid52}\item $\frac{{\left(-3\right)}^{-3}\ensuremath{\cdot}{\left(-3\right)}^{2}}{{\left(-3\right)}^{-4}}$\newline
    \label{m38359*uid53}\item ${\left({3}^{-1}+{2}^{-1}\right)}^{-1}$\newline
    \label{m38359*uid54}\item $\frac{{9}^{n-1}\ensuremath{\cdot}{27}^{3-2n}}{{81}^{2-n}}$\newline
    \label{m38359*uid55}\item $\frac{{2}^{3n+2}\ensuremath{\cdot}{8}^{n-3}}{{4}^{3n-2}}$
\end{enumerate}
                \end{enumerate}
  \label{m38359**end}
\par \raisebox{-5 pt}{\includegraphics[width=0.5cm]{col11306.imgs/summary_www.png}} Find the answers with the shortcodes:
 \par \begin{tabular}[h]{cccccc}
 (1.) lOJ  &  (2.) lOu  &  (3.) lOS  &  (4.) lOh  & \end{tabular}
