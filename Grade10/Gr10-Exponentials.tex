\chapter{Exponents}
\setcounter{figure}{1}
\setcounter{subfigure}{1}


$ \hspace{-5pt}\begin{array}{cccccccccccc}   \end{array} $ \hspace{2 pt}\raisebox{-5 pt}{\includegraphics[width=0.5cm]{col11306.imgs/summary_www.png}} {(section shortcode: MG10042 )} \par 



\label{m38359*cid3} $ \hspace{-5pt}\begin{array}{cccccccccccc}   \includegraphics[width=0.75cm]{col11306.imgs/summary_video.png} &   \end{array} $ \hspace{2 pt}\raisebox{-5 pt}{} {(section shortcode: MG10043 )} \par 
\label{m38359*id62562}Exponential notation is a short way of writing the same number multiplied by
itself many times.  We will now have a closer look at writing numbers using exponential notation. Exponents can also be called indices.

% \begin{Large}
\begin{center}
$ _{\mbox{base}~\leftarrow} $\begin{Large} $ ~a^{n~\rightarrow~$ \end{Large}$\mbox{exponent / index}} $
\end{center}
% \end{Large}

For any real number $a$ and natural number $n$, we can write $a$ multiplied by itself $n$ times as $a^n$.

 
\Definition{Exponential Notation}{
\begin{flushleft}
For $a \in \mathbb{R}, n \in \mathbb{N}$
\end{flushleft}
$$a^n = a \times a \times a \times \ldots \times a ~~ (n ~ \textrm{times}) $$
} 

Examples:
\begin{enumerate}[noitemsep, label=\textbf{\arabic*.}]
\item $3 \times 3 = 3^2$
\item $5 \times 5 \times 5 \times 5 = 5^4 $
\item $p \times p \times p = p^3$
\end{enumerate}

We now expand the definition to include 0 and the negative integers.

If exponent $n$ is 0, then we have 

$$ a^0 = 1 \hspace{1cm} a \ne 0 $$

\Tip{The restriction applies because $0^0$ is undefined }

We also define what it means if we have a negative exponent $a^{-n}$.
\begin{eqnarray*}
    a^{-n} &=& \frac{1}{a^n} \\
           &=& \frac{1}{a \times a \times a \times \ldots \times a ~~ (n ~ \textrm{times})} 
\end{eqnarray*}


\Tip{Here $a$ cannot equal 0 because $\frac{1}{0}$ is undefined}      
      
We always write the answer as numbers with a positive exponent.


If exponent $n$ is an even integer, then ${a}^{n}$ will always be positive for any non-zero real number $a$. 

For example, although $-3$ is negative, but $(-3)^2=-3 \times -3 = 9$ which is positive and $(-3)^{-2} = \frac{1}{-3 \times -3} = \frac{1}{9} $ is also positive.

If the exponent $n$ is an odd integer, then for any non-zero real number $a$ 

\begin{eqnarray*}
a^n ~~ \mbox{is positive if} ~~ a > 0 \\
a^n ~~ \mbox{is negative if} ~~ a < 0
\end{eqnarray*}

For example, $(-2)^3 = -2 \times -2 \times -2 = -8$ and $(2)^{-5} = \dfrac{1}{2 \times 2 \times 2 \times 2 \times 2} = \dfrac{1}{32}$.

\setcounter{subfigure}{0}
\begin{figure}[H] % horizontal\label{m38359*Exponents-1}
\textnormal{Khan Academy video on Exponents - 1}\vspace{.1in} 
\label{m38359*yt-media1}\label{m38359*yt-video1}
\raisebox{-5 pt}{ \includegraphics[width=0.5cm]{col11306.imgs/summary_www.png}} { (Video:  MG10044 )}
% \vspace{2pt}
% \vspace{.1in}
\end{figure}       


\setcounter{subfigure}{0}
\begin{figure}[H] % horizontal\label{m38359*Exponents-2}
\textnormal{Khan Academy video on Exponents-2}\vspace{.1in} 
\label{m38359*yt-media2}\label{m38359*yt-video2}
\raisebox{-5 pt}{ \includegraphics[width=0.5cm]{col11306.imgs/summary_www.png}} { (Video:  MG10045 )}
% \vspace{2pt}
% \vspace{.1in}
\end{figure}       




\section { Laws of Exponents}
\nopagebreak
\label{m38359*cid4} $ \hspace{-5pt}\begin{array}{cccccccccccc}   \includegraphics[width=0.75cm]{col11306.imgs/summary_fullmarks.png} &   \includegraphics[width=0.75cm]{col11306.imgs/summary_video.png} &   \end{array} $ \hspace{2 pt}\raisebox{-5 pt}{} {(section shortcode: MG10046 )} \par 
\label{m38359*id63061}There are several laws we can use to make working with exponential numbers easier. 
Some of these laws might have been done in earlier grades, but we will list all the laws here for easy reference and explain each one in detail. 


\begin{equation*}
    \begin{array}{ccl} 
	\hfill {a}^{m}\ensuremath{\times}{a}^{n}& =& {a}^{m+n}\hfill \\ 
	\hfill \frac{{a}^{m}}{{a}^{n}}& =& {a}^{m-n}\hfill \\ 
	\hfill {\left(ab\right)}^{n}& =& {a}^{n}{b}^{n}\hfill \\ 
	\hfill {\left({a}^{m}\right)}^{n}& =& {a}^{mn}\hfill 
    \end{array}
\end{equation}
where $a \in \mathbb{R}$ and $m,n \in \mathbb{Z}$

% 
%       \label{m38359*uid4}
%             \  \subsubsubsection { Exponential Law 1: ${a}^{0}=1$}
%             \nopagebreak
%         \label{m38359*id63512}Our definition of exponential notation shows that\par 
%         \label{m38359*uid5}\nopagebreak\noindent{}
%           
%     \begin{equation}
%     \begin{array}{ccl}\hfill {a}^{0}& =& 1,\left(a\ne 0\right)\hfill \end{array}
%       \end{equation}
%         \label{m38359*eip-662}To convince yourself of why this is true, use the fourth exponential law above (division of exponents) and consider what happens when $m=n$.\par \label{m38359*id63571}For example, ${x}^{0}=1$ and ${\left(1\phantom{\rule{0.277778em}{0ex}}000\phantom{\rule{0.277778em}{0ex}}000\right)}^{0}=1$.\par 
% \label{m38359*secfhsst!!!underscore!!!id339}
%               \subsubsubsection { Exercise: Exponential Law 1: ${a}^{0}=1,\left(a\ne 0\right)$ }
%             \nopagebreak
%         \label{m38359*id63666}\begin{enumerate}[noitemsep, label=\textbf{\arabic*}. ] 
%             \label{m38359*uid6}\item 
%             ${16}^{0}$
%       \label{m38359*uid7}\item 
%         $16{a}^{0}$
%       \label{m38359*uid8}\item 
%         ${\left(16+a\right)}^{0}$
%       \label{m38359*uid9}\item 
%         ${\left(-16\right)}^{0}$
%       \label{m38359*uid10}\item 
%         $-{16}^{0}$ 
% \newline
% \newline
%           \end{enumerate}
%       \label{m38359*uid11}
% \par \raisebox{-5 pt}{\includegraphics[width=0.5cm]{col11306.imgs/summary_www.png}} Find the answers with the shortcodes:
%  \par \begin{tabular}[h]{cccccc}
%  (1.) lOG  & \end{tabular}
%             \subsubsubsection{ Exponential Law 2: 

\begin{center}
  \begin{array}[lc]
      \textrm{\textbf{Exponential Law:}} & ${a}^{m}\ensuremath{\times}{a}^{n}={a}^{m+n}$ 
  \end{array}
\end{center}


\setcounter{subfigure}{0}
Our definition of exponential notation shows that

\begin{equation*}
      \begin{array}{cccc}\hfill {a}^{m}\ensuremath{\times}{a}^{n}& =& a\ensuremath{\times}a\ensuremath{\times} \ldots \ensuremath{\times}a\hfill & \left(\mathrm{m\; times}\right)\hfill \\ 
	      \hfill & & \phantom{\rule{-0.166667em}{0ex}}\phantom{\rule{-0.166667em}{0ex}}\phantom{\rule{-0.166667em}{0ex}}\phantom{\rule{-0.166667em}{0ex}}\ensuremath{\times}a\ensuremath{\times}a\ensuremath{\times}\ldots\ensuremath{\times}a\hfill & \left(\mathrm{n\; times}\right)\hfill \\ 
	      \hfill & =& a\ensuremath{\times}a\ensuremath{\times}...\ensuremath{\times}a\hfill & \left(\mathrm{m}+\mathrm{n\; times}\right)\hfill \\ 
	      \hfill & =& {a}^{m+n}\hfill & 
	      \end{array}
\end{equation}
\label{m38359*id64082}For example,\par 


\begin{equation*}
\begin{array}{ccl}\hfill {2}^{4}\ensuremath{\times}{2}^{3}& =& \left(2\ensuremath{\times}2\ensuremath{\times}2\ensuremath{\times}2\right)\ensuremath{\times}\left(2\ensuremath{\times}2\ensuremath{\times}2\right)\hfill \\
	      & =& {2}^{4+3}\hfill \\
	      & =& {2}^{7}\hfill \end{array}
\end{equation}

\Note{This simple law is the reason why exponentials were originally invented. In the days before calculators, all multiplication had to be done by hand with a pencil and paper. Multiplication takes a very long time to do and is very tedious. Adding numbers however, is very easy and quick to do. If you look at what this law is saying you will realise that it means that adding the exponents of two exponential numbers (of the same base) is the same as multiplying the two numbers together. This meant that for certain numbers, there was no need to actually multiply the numbers together in order to find out what their multiple was. This saved mathematicians a lot of time.}

\begin{figure}[H] % horizontal\label{m38359*ExponentsRule1}
\textnormal{Khan Academy video on Exponents - 3}\vspace{.1in} \nopagebreak
\label{m38359*yt-media3}\label{m38359*yt-video3}
\raisebox{-5 pt}{ \includegraphics[width=0.5cm]{col11306.imgs/summary_www.png}} { (Video:  MG10047 )}
% \vspace{2pt}
% \vspace{.1in}
\end{figure}
    
% \par
% \subsubsubsection{  Application using Exponential Law 2: ${a}^{m}\ensuremath{\times}{a}^{n}={a}^{m+n}$ }
%             \nopagebreak
%         \label{m38359*id64269}\begin{enumerate}[noitemsep, label=\textbf{\arabic*}. ] 
%             \label{m38359*uid13}\item 
%             ${x}^{2}\ensuremath{\cdot}{x}^{5}$
%       \label{m38359*uid14}\item 
%         ${2}^{3}\ensuremath{\cdot}{2}^{4}$
%         [Take note that the base (2) stays the same.]
%       \label{m38359*uid15}\item 
%         $3\ensuremath{\times}{3}^{2a}\ensuremath{\times}{3}^{2}$
% \newline
% \newline
%           \end{enumerate}
%       \label{m38359*uid16}
% \par \raisebox{-5 pt}{\includegraphics[width=0.5cm]{col11306.imgs/summary_www.png}} Find the answers with the shortcodes:
%  \par \begin{tabular}[h]{cccccc}
%  (1.) lO7  & \end{tabular}
%             \subsubsubsection{ Exponential Law 3: ${a}^{-n}=\frac{1}{{a}^{n}},a\ne 0$}
%             \nopagebreak
%         \label{m38359*id64482}Our definition of exponential notation for a negative exponent shows that\par 
%         \label{m38359*uid17}\nopagebreak\noindent{}
%     \begin{equation}
%     \begin{array}{cccc}\hfill {a}^{-n}& =& 1÷a÷...÷a\hfill & \left(\mathrm{n\; times}\right)\hfill \\ \hfill & =& \frac{1}{1\ensuremath{\times}a\ensuremath{\times}\cdots \ensuremath{\times}a}\hfill & \left(\mathrm{n\; times}\right)\hfill \\ \hfill & =& \frac{1}{{a}^{n}}\hfill & \end{array}\tag{5.8}
%       \end{equation}
%         \label{m38359*id64624}This means that a minus sign in the exponent is just another way of showing that the whole exponential number is to be divided instead of multiplied.\par 
%         \label{m38359*id64630}For example,\par 
%         \label{m38359*id64634}\nopagebreak\noindent{}
%           
%     \begin{equation}
%     \begin{array}{ccl}\hfill {2}^{-7}& =& \frac{1}{2\ensuremath{\times}2\ensuremath{\times}2\ensuremath{\times}2\ensuremath{\times}2\ensuremath{\times}2\ensuremath{\times}2}\hfill \\ & =& \frac{1}{{2}^{7}}\hfill \end{array}\tag{5.9}
%       \end{equation}
% \label{m38359*eip-294}This law is useful in helping us simplify fractions where there are exponents in both the denominator and the numerator. For example:
% \label{m38359*id7902}\nopagebreak\noindent{}
% 
%     \begin{equation}
%     \begin{array}{ccl}\frac{{a}^{-3}}{{a}^{4}}& =& \frac{1}{{a}^{3}{a}^{4}}\\ & =& \frac{1}{{a}^{7}}\end{array}\tag{5.10}
%       \end{equation}
% \par \label{m38359*secfhsst!!!underscore!!!id835}
%             \subsubsubsection{  Application using Exponential Law 3: ${a}^{-n}=\frac{1}{{a}^{n}},a\ne 0$ }
%             \nopagebreak
%         \label{m38359*id64771}\begin{enumerate}[noitemsep, label=\textbf{\arabic*}. ] 
%             \label{m38359*uid18}\item 
%             ${2}^{-2}=\frac{1}{{2}^{2}}$
%       \label{m38359*uid19}\item 
%         $\frac{{2}^{-2}}{{3}^{2}}$
%       \label{m38359*uid20}\item 
%         ${\left(\frac{2}{3}\right)}^{-3}$
%       \label{m38359*uid21}\item 
%         $\frac{m}{{n}^{-4}}$
%       \label{m38359*uid22}\item 
%         $\frac{{a}^{-3}\ensuremath{\cdot}{x}^{4}}{{a}^{5}\ensuremath{\cdot}{x}^{-2}}$
% \newline
% \newline
%           \end{enumerate}
%       \label{m38359*uid23}
% \par \raisebox{-5 pt}{\includegraphics[width=0.5cm]{col11306.imgs/summary_www.png}} Find the answers with the shortcodes:
%  \par \begin{tabular}[h]{cccccc}
%  (1.) lcx  & \end{tabular}

\begin{center}
\begin{array}[lc]
\textrm{\textbf{Exponential Law:}} & $$ \frac{ {a}^{m} }{ {a}^{n} }={a}^{m-n}$$
\end{array}
\end{center}

\begin{eqnarray*}
\frac{a^m}{a^n} &=& \frac{a \times a \times a \ldots ~ (m~\textrm{times})} {a \times a \times a \ldots ~ (n~\textrm{times})} \\
&= & a \times a \times a \ldots ~ (m-n~\textrm{times}) \\
&= & a^{m-n}
\end{eqnarray*}

          
\label{m38359*id65293}For example,

\begin{equation*}
    \begin{array}{ccl}\hfill \dfrac{{2}^{7}}{{2}^{3}}& =& \dfrac{2\ensuremath{\times}2\ensuremath{\times}2\ensuremath{\times}2\ensuremath{\times}2\ensuremath{\times}2\ensuremath{\times}2}{2\ensuremath{\times}2\ensuremath{\times}2}\hfill \\
						     & =& 2\ensuremath{\times}2\ensuremath{\times}2\ensuremath{\times}2\hfill \\
						     & =& {2}^{4}\hfill \hfill 
    \end{array}
\end{equation}

\setcounter{subfigure}{0}
\begin{figure}[H] % horizontal\label{m38359*exponents-5}
\textnormal{Khan academy video on exponents - 4}\vspace{.1in} \nopagebreak
\label{m38359*yt-media6}\label{m38359*yt-video6}
\raisebox{-5 pt}{ \includegraphics[width=0.5cm]{col11306.imgs/summary_www.png}} { (Video:  MG10048 )}
% \vspace{2pt}
% \vspace{.1in}
\end{figure}       
% \par \label{m38359*secfhsst!!!underscore!!!id1192}
%             \subsubsubsection{ Application using Exponential Law 4: ${a}^{m}÷{a}^{n}={a}^{m-n}$}
%             \nopagebreak
%         \label{m38359*id65493}\begin{enumerate}[noitemsep, label=\textbf{\arabic*}. ] 
%             \label{m38359*uid25}\item 
%             $\frac{{a}^{6}}{{a}^{2}}={a}^{6-2}$
%       \label{m38359*uid26}\item 
%         $\frac{{3}^{2}}{{3}^{6}}$
%       \label{m38359*uid27}\item 
%         $\frac{32{a}^{2}}{4{a}^{8}}$
%       \label{m38359*uid28}\item 
%         $\frac{{a}^{3x}}{{a}^{4}}$
% \newline8
% \newline
%           \end{enumerate}
%       \label{m38359*uid29}
% \par \raisebox{-5 pt}{\includegraphics[width=0.5cm]{col11306.imgs/summary_www.png}} Find the answers with the shortcodes:
%  \par \begin{tabular}[h]{cccccc}
%  (1.) lOA  & \end{tabular}


\begin{center}
  \begin{array}[lc]
      \textrm{\textbf{Exponential Law:}} & $$ {\left(ab\right)}^{n}={a}^{n}{b}^{n}$$
  \end{array}
\end{center}

%             \subsubsubsection{ Exponential Law 5: ${\left(ab\right)}^{n}={a}^{n}{b}^{n}$}

The order in which two numbers are multiplied together does not matter. We know that $a \times b = b \times a$ 

Therefore,
\begin{equation*}
    \begin{array}{cclc}\hfill {\left(ab\right)}^{n}& =& a\ensuremath{\times}b\ensuremath{\times}a\ensuremath{\times}b\ensuremath{\times}\ldots\ensuremath{\times}a\ensuremath{\times}b\hfill & \left(\mathrm{n\; times}\right)\hfill \\
	\hfill & =& a\ensuremath{\times}a\ensuremath{\times}\ldots\ensuremath{\times}a\hfill & \left(\mathrm{n\; times}\right)\hfill \\
	\hfill & & \phantom{\rule{-0.166667em}{0ex}}\phantom{\rule{-0.166667em}{0ex}}\phantom{\rule{-0.166667em}{0ex}}\phantom{\rule{-0.166667em}{0ex}}\ensuremath{\times}b\ensuremath{\times}b\ensuremath{\times}\ldots\ensuremath{\times}b\hfill & \left(\mathrm{n\; times}\right)\hfill \\
	\hfill & =& {a}^{n}{b}^{n}\hfill & 
\end{array}
\end{equation}
\label{m38359*id66030}For example,

\begin{equation*}
    \begin{array}{ccl}\hfill {\left(2\ensuremath{\cdot}3\right)}^{4}& =& \left(2\ensuremath{\cdot}3\right)\ensuremath{\times}\left(2\ensuremath{\cdot}3\right)\ensuremath{\times}\left(2\ensuremath{\cdot}3\right)\ensuremath{\times}\left(2\ensuremath{\cdot}3\right)\hfill \\
	    & =& \left(2\ensuremath{\times}2\ensuremath{\times}2\ensuremath{\times}2\right)\ensuremath{\times}\left(3\ensuremath{\times}3\ensuremath{\times}3\ensuremath{\times}3\right)\hfill \\
	    & =& \left({2}^{4}\right)\ensuremath{\times}\left({3}^{4}\right)\hfill \\ & =& {2}^{4}\cdot{3}^{4}\hfill 
    \end{array}
\end{equation}

%             \subsubsubsection{ Application using Exponential Law 5: ${\left(ab\right)}^{n}={a}^{n}{b}^{n}$ }
%             \nopagebreak
%         \label{m38359*id66288}\begin{enumerate}[noitemsep, label=\textbf{\arabic*}. ] 
%             \label{m38359*uid31}\item 
%             ${\left(2xy\right)}^{3}={2}^{3}{x}^{3}{y}^{3}$
%       \label{m38359*uid32}\item 
%         ${\left(\frac{7a}{b}\right)}^{2}$
%       \label{m38359*uid33}\item 
%         ${\left(5a\right)}^{3}$
% \newline
% \newline
%           \end{enumerate}
%       \label{m38359*uid34}
% \par \raisebox{-5 pt}{\includegraphics[width=0.5cm]{col11306.imgs/summary_www.png}} Find the answers with the shortcodes:
%  \par \begin{tabular}[h]{cccccc}
%  (1.) lOs  & \end{tabular}
            

\begin{center}
\begin{array}[lc]
  \textrm{\textbf{Exponential Law:}} & $$ {\left({a}^{m}\right)}^{n}={a}^{mn} $$
\end{array}
\end{center}
% \subsubsubsection{ Exponential Law 6: ${\left({a}^{m}\right)}^{n}={a}^{mn}$}
%             \nopagebreak
\label{m38359*id66531}We can find the exponential of an exponential of a number. Even though this sentence sounds complicated, it is just saying that you can find the exponential of a number and then take the exponential of that number. \par


\begin{equation*}
    \begin{array}{cclc}\hfill {\left({a}^{m}\right)}^{n}& =& {a}^{m}\ensuremath{\times}{a}^{m}\ensuremath{\times}\ldots\ensuremath{\times}{a}^{m}\hfill & \left(\mathrm{n\; times}\right)\hfill \\
	\hfill & =& a\ensuremath{\times}a\ensuremath{\times}\ldots\ensuremath{\times}a\hfill & \left(\mathrm{m}\ensuremath{\times}\mathrm{n\; times}\right)\hfill \\
	\hfill & =& {a}^{mn}\hfill & 
    \end{array}
\end{equation}
\label{m38359*id66694}For example,

\begin{equation*}
    \begin{array}{ccl}\hfill {\left({5}^{2}\right)}^{3}& =& \left({5}^{2}\right)\ensuremath{\times}\left({5}^{2}\right)\ensuremath{\times}\left({5}^{2}\right)\hfill \\ 
	      & =& \left(5\ensuremath{\times}5\right)\ensuremath{\times}\left(5\ensuremath{\times}5\right)\ensuremath{\times}\left(5\ensuremath{\times}5\right)\hfill \\
	      & =& \left({5}^{6}\right)\hfill
    \end{array}
\end{equation}

\Note{

\begin{equation*}
    5^2 \times 5^4 & = & 5^{2+4} & = & 5^6  \\
\end{equation*}

\begin{equation*}
    (5^2)^4        & = & 5^{2\times4} & = & 5^8 
\end{equation*}

}

      
% \label{m38359*secfhsst!!!underscore!!!id1894}
%             \subsubsubsection{ Application using Exponential Law 6: ${\left({a}^{m}\right)}^{n}={a}^{mn}$ }
%             \nopagebreak
%         \label{m38359*id66924}\begin{enumerate}[noitemsep, label=\textbf{\arabic*}. ] 
%             \label{m38359*uid36}\item 
%             ${\left({x}^{3}\right)}^{4}$
%       \label{m38359*uid37}\item 
%         ${\left[{\left({a}^{4}\right)}^{3}\right]}^{2}$
%       \label{m38359*uid38}\item 
%         ${\left({3}^{n+3}\right)}^{2}$
% \newline
% \newline
%           \end{enumerate}
% \label{m38359*eip-323}\par
%             \label{m38359*secfhsst!!!underscore!!!id41}\vspace{.5cm} 
\begin{wex}
{
Simplifying bases
}
{
Simplify: $$\frac{2^{2n} \cdot 4^n \cdot 2 }{ 16^n} $$
}
{

\westep{Make all bases the same}

$$ \frac{2^{2n} \cdot (2^2)^n \cdot 2 }{ (2^4)^n} $$


\westep{Simplify the exponents}

\begin{eqnarray*}
 & & \frac{ 2^{2n} \cdot 2^{2n} \cdot 2^1 }{ 2^{4n} } \\
& = & \frac{ 2^{2n + 2n +1}}{2^{4n}} \\
& = & \frac{2^{4n+1}}{2^{4n}} \\
& = & 2^{4n+1-(4n)} \\
& = & 2
\end{eqnarray*}
 
}
\end{wex}


     
\begin{wex}
{%title
Simplifying exponents
}
{
Simplify: $\dfrac{{5}^{2x-1}\ensuremath{\cdot}{9}^{x-2}}{{15}^{2x-3}}$
}
{
%     \begin{enumerate}[noitemsep, label=\textbf{Step} \textbf{\arabic*}. ] 
%             \leftskip=20pt\rightskip=\leftskip\item  
%     \label{m38359*id118023}\nopagebreak\noindent{}
    \westep{Change all bases to prime numbers}
      
\begin{equation*}
\begin{array}{ccl}& =& \dfrac{{5}^{2x-1}\ensuremath{\cdot}{\left({3}^{2}\right)}^{x-2}}{{\left(5.3\right)}^{2x-3}}\hfill \\
		  & =& \dfrac{{5}^{2x-1}\ensuremath{\cdot}{3}^{2x-4}}{{5}^{2x-3}\ensuremath{\cdot}{3}^{2x-3}}\hfill 
\end{array}
  \end{equation}

  
\westep{Subtract exponents (same base)}

\begin{equation*}
\begin{array}{ccl}& =& {5}^{(2x-1)-(2x-3)}\ensuremath{\cdot}{3}^{(2x-4)-(2x-3)}\hfill \\ 
& =& 5^{2x-1-2x+3} \cdot 3^{2x-4 - 2x+3} \\
& =& {5}^{2}\ensuremath{\cdot}{3}^{-1}\hfill \end{array}
\end{equation}


\westep{Write answer as a fraction}  
\begin{eqnarray*}
=\frac{25}{3} \\ = 8\frac{1}{3}
\end{eqnarray*}

}
\end{wex}



% \par 
% \label{m38359*secfhsst!!!underscore!!!id2193}
% \par \raisebox{-5 pt}{\includegraphics[width=0.5cm]{col11306.imgs/summary_www.png}} Find the answers with the shortcodes:
%  \par \begin{tabular}[h]{cccccc}
%  (1.) lO6  & \end{tabular}
\section{Rational Exponents}


\section{Exponential Equations}
\begin{exercises}{Exponential Numbers }
{
\label{m38359*id67549}Match the answers to the questions, by filling in the correct answer into the \textbf{Answer} column.
Possible answers are: $\frac{3}{2}$, $1$, $-1$, $-\frac{1}{3}$, $8$. Answers may be repeated.\par 
% \textbf{m38359*id67604}\par
  \begin{table}[H]
% \begin{table}[H]
% \\ '' '0'
\begin{center}
\label{m38359*id67604}
\noindent
\tabletail{%
\hline
\multicolumn{2}{|p{\mytableboxwidth}|}{\raggedleft \small \sl continued on next page}\\
\hline
}
\tablelasttail{}
\begin{xtabular}[t]{|l|l|}\hline
	  \textbf{Question}
	  &
	  \textbf{Answer}
	% make-rowspan-placeholders
\tabularnewline\cline{1-1}\cline{2-2}
%--------------------------------------------------------------------
	  ${2}^{3}$
	  &
% make-rowspan-placeholders
\tabularnewline\cline{1-1}\cline{2-2}
%--------------------------------------------------------------------
	  ${7}^{3-3}$
	  &
% make-rowspan-placeholders
\tabularnewline\cline{1-1}\cline{2-2}
%--------------------------------------------------------------------
	  ${\left(\frac{2}{3}\right)}^{-1}$
	  &
% make-rowspan-placeholders
\tabularnewline\cline{1-1}\cline{2-2}
%--------------------------------------------------------------------
	  ${8}^{7-6}$
	  &
% make-rowspan-placeholders
\tabularnewline\cline{1-1}\cline{2-2}
%--------------------------------------------------------------------
	  ${\left(-3\right)}^{-1}$
	  &
% make-rowspan-placeholders
\tabularnewline\cline{1-1}\cline{2-2}
%--------------------------------------------------------------------
	  ${\left(-1\right)}^{23}$
	  &
% make-rowspan-placeholders
\tabularnewline\cline{1-1}\cline{2-2}
%--------------------------------------------------------------------
\end{xtabular}
\end{center}
% \begin{center}{\small\bfseries Table 5.1}\end{center}
% \begin{caption}{\small\bfseries Table 5.1}\end{caption}
\end{table}
}




We will use all these laws in Equations and Inequalities\footnote{\raggedright{}"Equations and Inequalities - Grade 10 [CAPS]" <http://http://cnx.org/content/m38372/latest/>} to help us solve exponential equations.\par \label{m38359*eip-160}The following video gives an example on using some of the concepts covered in this chapter.
\setcounter{subfigure}{0}
\begin{figure}[H] % horizontal\label{m38359*ExponentsLaw3}
\textnormal{Khan Academy video on Exponents - 5}\vspace{.1in} \nopagebreak
\label{m38359*yt-media5}\label{m38359*yt-video5}
\raisebox{-5 pt}{ \includegraphics[width=0.5cm]{col11306.imgs/summary_www.png}} { (Video:  MG10049 )}
% \vspace{2pt}
% \vspace{.1in}
\end{figure}    


\section{ Summary}

\label{m38359*eip-908} $ \hspace{-5pt}\begin{array}{cccccccccccc}   \end{array} $ \hspace{2 pt}\raisebox{-5 pt}{\includegraphics[width=0.5cm]{col11306.imgs/summary_www.png}} {(section shortcode: MG10050 )} 
\begin{itemize}[noitemsep, label=\textbullet{}]
    \item Exponential notation means a number written like ${a}^{n}$ where $n$ is an integer and $a$ can be any real number.
    \item $a$ is called the \textsl{base} and $n$ is called the \textsl{exponent} or \textsl{index}.
    \item The ${n}^{\mathrm{th}}$ power of $a$ is defined as: ${a}^{n}=a\ensuremath{\times}a\ensuremath{\times}\cdots \ensuremath{\times}a\phantom{\rule{2.em}{0ex}}\left(\mathrm{n\; times}\right)$
    \item  There are six laws of exponents: 
	\begin{itemize}[noitemsep]
	    \item Exponential Law 1: ${a}^{0}=1$
	    \item Exponential Law 2: ${a}^{m}\ensuremath{\times}{a}^{n}={a}^{m+n}$
	    \item Exponential Law 3: ${a}^{-n}=\frac{1}{{a}^{n}},a\ne 0$
	    \item Exponential Law 4: ${a}^{m}÷{a}^{n}={a}^{m-n}$
	    \item Exponential Law 5: ${\left(ab\right)}^{n}={a}^{n}{b}^{n}$
	    \item Exponential Law 6: ${\left({a}^{m}\right)}^{n}={a}^{mn}$
	\end{itemize}
    \end{itemize}


\begin{eocexercises}{End of chapter exercises}
 $ \hspace{-5pt}\begin{array}{cccccccccccc}   \end{array} $ 
\hspace{2 pt}\raisebox{-5 pt}{\includegraphics[width=0.5cm]{col11306.imgs/summary_www.png}} {(section shortcode: MG10051 )} \par 
\label{m38359*id67892}

\begin{enumerate}[noitemsep, label=\textbf{\arabic*}. ] 
    \item Simplify as far as possible:
	\begin{enumerate}[noitemsep, label=\textbf{\alph*}. ] 
	    \item ${302}^{0}$
	    \item ${1}^{0}$
	    \item ${\left(xyz\right)}^{0}$
	    \item ${\left[{\left(3{x}^{4}{y}^{7}{z}^{12}\right)}^{5}{\left(-5{x}^{9}{y}^{3}{z}^{4}\right)}^{2}\right]}^{0}$
	    \item ${\left(2x\right)}^{3}$
	    \item ${\left(-2x\right)}^{3}$
	    \item ${\left(2x\right)}^{4}$
	    \item ${\left(-2x\right)}^{4}$
	\end{enumerate}
    \item Simplify without using a calculator. Leave your answers with positive exponents.
	\begin{enumerate}[noitemsep, label=\textbf{\alph*}. ] 
	    \item $\dfrac{3{x}^{-3}}{{\left(3x\right)}^{2}}$
	    \item $5{x}^{0}+{8}^{-2}-{\left(\frac{1}{2}\right)}^{-2}\ensuremath{\cdot}{1}^{x}$
	    \item $\dfrac{{5}^{b-3}}{{5}^{b+1}}$
	\end{enumerate}
    \item Simplify, showing all steps:
	\begin{enumerate}[noitemsep, label=\textbf{\alph*}. ] 
	    \item $\dfrac{{2}^{a-2}.{3}^{a+3}}{{6}^{a}}$
	    \item $\dfrac{{a}^{2m+n+p}}{{a}^{m+n+p}\ensuremath{\cdot}{a}^{m}}$
	    \item $\dfrac{{3}^{n}\ensuremath{\cdot}{9}^{n-3}}{{27}^{n-1}}$
	    \item ${\left(\dfrac{2{x}^{2a}}{{y}^{-b}}\right)}^{3}$
	    \item $\dfrac{{2}^{3x-1}\ensuremath{\cdot}{8}^{x+1}}{{4}^{2x-2}}$
	    \item $\dfrac{{6}^{2x}\ensuremath{\cdot}{11}^{2x}}{{22}^{2x-1}\ensuremath{\cdot}{3}^{2x}}$
	\end{enumerate}
    \item Simplify, without using a calculator:
	\begin{enumerate}[noitemsep, label=\textbf{\alph*}. ] 
	    \item $\dfrac{{\left(-3\right)}^{-3}\ensuremath{\cdot}{\left(-3\right)}^{2}}{{\left(-3\right)}^{-4}}$
	    \item ${\left({3}^{-1}+{2}^{-1}\right)}^{-1}$
	    \item $\dfrac{{9}^{n-1}\ensuremath{\cdot}{27}^{3-2n}}{{81}^{2-n}}$
	    \item $\dfrac{{2}^{3n+2}\ensuremath{\cdot}{8}^{n-3}}{{4}^{3n-2}}$
	\end{enumerate}
\end{enumerate}

\raisebox{-5 pt}{\includegraphics[width=0.5cm]{col11306.imgs/summary_www.png}} Find the answers with the shortcodes:
\begin{tabular}[h]{cccccc}
(1.) lOJ  &  (2.) lOu  &  (3.) lOS  &  (4.) lOh  & 
\end{tabular}
