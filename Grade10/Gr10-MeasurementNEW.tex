\chapter{  Measurements  }  
This chapter examines the surface areas and volumes of three dimensional objects otherwise
known as solids. In order to work with these, you need to know how to calculate the surface
area and perimeter of the two dimensional shapes below.

\section{ Areas of Polygons }

\begin{table*}[h]
\newcolumntype{C}{>{\centering\arraybackslash} m{1.5in} }
\newcolumntype{D}{>{\centering\arraybackslash} m{2in} }
\begin{tabular}{|C|D|C|}
\hline
\textbf{Name} & \textbf{Shape} & \textbf{Formulae} \\ \hline
Square &
        \begin{center}
\scalebox{0.9}{
        \begin{pspicture}(-2,-0.5)(5,2)
            \pspolygon(0,0)(0,1)(1,1)(1,0)
            \pspolygon(0,0)(0,0.15)(0.15,0.15)(0.15,0)
            \rput(-0.25,0.5){$h$}
            \rput(0.5,-0.25){$b$}
        \end{pspicture}}
        \end{center}
&

$b \times h$ \\ \hline

Rectangle &
    \begin{center}
\scalebox{0.9}{
        \begin{pspicture}(-2,-0.5)(5,2)
            \pspolygon(0,0)(0,1)(2.5,1)(2.5,0)
            \pspolygon(0,0)(0,0.15)(0.15,0.15)(0.15,0)
            \rput(2.75,0.5){$h$}
            \rput(1.25,-0.25){$b$}
        \end{pspicture}}
    \end{center}
& $ b \times h $ \\ \hline

Triangle &
\begin{center} 
\scalebox{0.9}{
    \begin{pspicture}(0,-0.5)(5,2)
        \pspolygon(0,0)(2,1)(3,0)
        \psline[linewidth=0.04cm,linestyle=dashed,dash=0.16cm 0.16cm](2,1)(2,0)
        \pspolygon(2,0)(2,0.15)(2.15,0.15)(2.15,0)
        \rput(1.8,0.5){$h$}
        \rput(1.5,-0.25){$b$}
    \end{pspicture}}
\end{center}

& $ \dfrac{1}{2} b \times h $ \\ \hline

Trapezium &
\begin{center}
\scalebox{0.9}{
    \begin{pspicture}(-2,-0.5)(5,2)
        \pspolygon(-1.5,0)(0,1)(2,1)(3,0)
        \psline[linewidth=0.04cm,linestyle=dashed,dash=0.16cm 0.16cm](2,1)(2,0)
        \pspolygon(2,0)(2,0.15)(2.15,0.15)(2.15,0)
        \rput(1.8,0.5){$h$}
        \rput(0.75,-0.25){$s_1$}
        \rput(1,1.25){$s_2$}
    \end{pspicture}}
\end{center}
& $ \dfrac{1}{2} (s_1 + s_2) \times h $ \\ \hline 
Parallelogram &

\begin{center}
\scalebox{0.9}{
    \begin{pspicture}(-2,-0.5)(5,2)
        \pspolygon(-1,0)(0,1)(3,1)(2,0)
        \psline[linewidth=0.04cm,linestyle=dashed,dash=0.16cm 0.16cm](0.5,1)(0.5,0)
        \pspolygon(0.5,0)(0.5,0.15)(0.65,0.15)(0.65,0)
        \rput(0.3,0.5){$h$}
        \rput(1,-0.25){$b$}
    \end{pspicture}}
\end{center}

& $ b \times h $ \\ \hline

Circle &

\begin{center}
\scalebox{0.9}{
    \begin{pspicture}(-2,-0.5)(5,2)
        \pscircle[dimen=outer](0.5,0.5){0.7}
        \psline[linestyle=dashed,dash=0.1cm 0.1cm](0.5,0.5)(1.2,0.5)
        \psdots[dotsize=0.12](0.5,0.5)
        \rput(0.85,0.75){$r$}
    \end{pspicture}}
\end{center} & 

$ \pi r^2 $ \\ \hline


\end{tabular}


%\caption{Remember that surface area is measured in square units, for example $cm^2$, $m^2$ or $mm^2$.}
\end{table*}
            

\begin{figure}[H]
    \textnormal{Khan Academy video on area and perimeter}
    \vspace{.1in}
    \nopagebreak
    \raisebox{-5 pt}{ 
    \includegraphics[width=0.5cm]{col11306.imgs/summary_www.png}} { (Video:  MG10096 )}
    \vspace{2pt}
    \vspace{.1in}
\end{figure}   
\par


\begin{figure}[H]
    \textnormal{Khan Academy video on area of a circle}
    \vspace{.1in}
    \nopagebreak
    \raisebox{-5 pt}{ \includegraphics[width=0.5cm]{col11306.imgs/summary_www.png}} { (Video:  MG10097 )}
    \vspace{2pt}
    \vspace{.1in}    
\end{figure}   

\begin{wex}{Finding the area of a parallelogram}{
    Find the area of the following figure:\\

    \begin{center}
	\begin{pspicture}(-2,-0.5)(5,2)
	    \pspolygon(-1,0)(0.5,2)(4,2)(2.5,0)
	    \psline[linewidth=1pt,linestyle=dashed,dash=0.16cm 0.16cm](0.5,2)(0.5,0)
	    \pspolygon(0.5,0)(0.5,0.3)(0.8,0.3)(0.8,0)
	    \rput(-1,-0.2){$A$}
	    \rput(0.5,2.2){$B$}
	    \rput(4,2.2){$C$}
	    \rput(2.5,-0.2){$D$}
	    \rput(0.5,-0.2){$E$}
	    \rput(-0.5,1){$5$}
	    \rput(-0.25,-0.2){$3$}
	    \rput(1.5,-0.2){$4$}
	\end{pspicture}
    \end{center}
    }{

    \westep{We first need to find the height, BE, of the parallelogram. We can use Pythagoras to do this}
    \begin{eqnarray*}
	BE^2 &=& AB^2 - AE^2\\
	&=& 5^2 - 3^2\\
	 &=& 16\\
	\therefore BE &=& 4
    \end{eqnarray*}
    
    \westep{We apply the formula for the area of a parallelogram to find the area}
    \begin{eqnarray*}
	\text{Area} &=& h \times b\\
		    &=& 4 \times 7\\
		    &=& 28
    \end{eqnarray*}
    }
\end{wex}


\begin{exercises}{Polygons}


Find the areas of each of the given figures -- remember area is measured in square units (cm${}^{2}$, m${}^{2}$, mm${}^{2}$).
\begin{center}
\scalebox{0.9}
{
\begin{pspicture}(0,-4.758281)(14.921875,4.758281)
\psline[linewidth=0.04cm](0.381875,1.7882812)(4.381875,1.7882812) 
\psline[linewidth=0.04cm](4.241875,1.7682812)(4.241875,1.7682812) 
\psline[linewidth=0.04cm,linestyle=dashed,dash=0.16cm 0.16cm](2.361875,3.8082812)(2.361875,2.8282812) 
\psline[linewidth=0.04cm](0.381875,1.7882812)(2.341875,3.7882812) 
\psline[linewidth=0.04cm](2.361875,3.7882812)(4.341875,1.8282813) 
% \usefont{T1}{ptm}{m}{n} 
\rput(2.4517188,2.5782812){$5$ cm} 
\psline[linewidth=0.04cm,linestyle=dashed,dash=0.16cm 0.16cm](2.361875,2.3482811)(2.361875,1.7682812) 
% \usefont{T1}{ptm}{m}{n} 
\rput(2.4365625,1.4182812){$10$ cm} 
% \usefont{T1}{ptm}{m}{n} 
\rput(0.16203125,3.6182814){\textbf{1.}} 
\psframe[linewidth=0.04,dimen=outer](9.141875,3.7782812)(5.141875,1.7782812) 
% \usefont{T1}{ptm}{m}{n} 
\rput(4.72375,3.5982811){\textbf{2.}} 
\psframe[linewidth=0.04,dimen=outer](5.471875,3.7782812)(5.151875,3.4582813) 
\psline[linewidth=0.04cm](7.121875,3.9282813)(7.321875,3.7682812) 
\psline[linewidth=0.04cm](7.1272163,3.6020272)(7.3165336,3.7745354) 
\psline[linewidth=0.04cm](7.121875,1.9482813)(7.321875,1.7882812) 
\psline[linewidth=0.04cm](7.1272163,1.6220272)(7.3165336,1.7945354)
\psline[linewidth=0.04cm](4.995617,2.7704508)(5.161965,2.965203)
\psline[linewidth=0.04cm](5.321874,2.7652996)(5.1555424,2.9600654) 
\psline[linewidth=0.04cm](4.995617,2.570451)(5.161965,2.765203) 
\psline[linewidth=0.04cm](5.321874,2.5652997)(5.1555424,2.7600653)
\psline[linewidth=0.04cm](8.955617,2.7704508)(9.121964,2.965203) 
\psline[linewidth=0.04cm](9.281874,2.7652996)(9.115542,2.9600654)
\psline[linewidth=0.04cm](8.955617,2.590451)(9.121964,2.785203) 
\psline[linewidth=0.04cm](9.281874,2.5852997)(9.115542,2.7800653)
%  \usefont{T1}{ptm}{m}{n} 
\rput(9.711719,2.7182813){$5$ cm}
%  \usefont{T1}{ptm}{m}{n} 
\rput(7.1965623,1.4182812){$10$ cm}
\psline[linewidth=0.04cm,linestyle=dashed,dash=0.16cm 0.16cm](10.901875,2.7482812)(14.901875,2.7482812) \pscircle[linewidth=0.04,dimen=outer](12.911875,2.7582812){2.0} 
\psdots[dotsize=0.16](12.901875,2.7482812)
%  \usefont{T1}{ptm}{m}{n} 
\rput(12.876562,2.4182813){$10$ cm} 
% \usefont{T1}{ptm}{m}{n}
\rput(10.631406,3.5982811){\textbf{3.}}
%  \usefont{T1}{ptm}{m}{n}
\rput(0.19140625,0.19828124){\textbf{4.}} 
% \usefont{T1}{ptm}{m}{n}
\rput(5.5117188,-0.8017188){$5$ cm}
%  \usefont{T1}{ptm}{m}{n}
\rput(2.5340624,0.29828125){$7$ cm}
\psline[linewidth=0.04cm](2.1600447,-1.3315424)(2.358717,-1.4931878) 
\psline[linewidth=0.04cm](2.162693,-1.6578295)(2.3534274,-1.4868897) 
\psline[linewidth=0.04cm](1.9601017,-1.3363228)(2.1587741,-1.4979682) 
\psline[linewidth=0.04cm](1.9627501,-1.6626099)(2.1534846,-1.4916701) 
\psline[linewidth=0.04cm](0.181875,-1.4917188)(4.181875,-1.4917188) 
\psline[linewidth=0.04cm](1.401875,0.04828125)(5.401875,0.04828125) 
\psline[linewidth=0.04cm](3.3800445,0.2084576)(3.578717,0.046812218)
\psline[linewidth=0.04cm](3.382693,-0.11782951)(3.5734274,0.053110242) 
\psline[linewidth=0.04cm](3.1801016,0.20367722)(3.3787742,0.042031832) 
\psline[linewidth=0.04cm](3.18275,-0.1226099)(3.3734844,0.048329853) 
\psline[linewidth=0.04cm](4.191091,-1.5022907)(5.394185,0.045147188) 
\psline[linewidth=0.04cm](4.621163,-0.6958544)(4.8714867,-0.6416498)
\psline[linewidth=0.04cm](4.877642,-0.89756864)(4.863264,-0.64184755) 
\psline[linewidth=0.04cm](0.21109095,-1.4822907)(1.4141852,0.065147184) 
\psline[linewidth=0.04cm](0.641163,-0.67585444)(0.89148647,-0.6216498) 
\psline[linewidth=0.04cm](0.8976424,-0.87756866)(0.8832641,-0.62184757)
\psline[linewidth=0.04cm,linestyle=dashed,dash=0.16cm 0.16cm](4.161875,-1.4917188)(4.161875,0.04828125) \psframe[linewidth=0.04,dimen=outer](4.471875,0.05828125)(4.151875,-0.26171875)
%  \usefont{T1}{ptm}{m}{n} 
\rput(4.7753124,0.27828124){$3$ cm} 
\psline[linewidth=0.04cm](6.021875,-1.4317187)(10.021875,-1.4317187)
\psline[linewidth=0.04cm](9.821875,-1.4517188)(9.821875,-1.4517188) 
% \usefont{T1}{ptm}{m}{n} 
\rput(7.5165625,-1.7217188){$12$ cm} 
\psline[linewidth=0.04cm](8.837452,0.20483598)(10.026298,-1.4282734) 
\psline[linewidth=0.04cm](8.821875,0.20828125)(6.021875,-1.4317187)
\psline[linewidth=0.04cm,linestyle=dashed,dash=0.16cm 0.16cm](8.821875,0.18828125)(8.821875,-1.4117187) \psframe[linewidth=0.04,dimen=outer](8.851875,-1.1217188)(8.531875,-1.4417187) 
% \usefont{T1}{ptm}{m}{n} 
\rput(9.217969,-1.6817187){$8$ cm}
%  \usefont{T1}{ptm}{m}{n} 
\rput(9.936563,-0.44171876){$10$ cm}
%  \usefont{T1}{ptm}{m}{n} 
\rput(6.2414064,0.23828125){\textbf{5.}} 
\psline[linewidth=0.04cm](11.561875,-1.3517188)(13.581875,-1.3517188) 
\psline[linewidth=0.04cm](11.564406,-1.3656596)(12.099343,0.58222204) 
\psline[linewidth=0.04cm](12.107254,0.5571398)(13.576495,-1.3405774) 
\psline[linewidth=0.04cm](11.621875,-0.49171874)(11.941875,-0.59171873) 
\psline[linewidth=0.04cm](11.681875,-0.35171875)(12.001875,-0.45171875) 
\psline[linewidth=0.04cm](12.161875,-1.1917187)(12.161875,-1.5317187) 
\psline[linewidth=0.04cm](12.161875,-1.4317187)(12.161875,-1.4317187) 
\psline[linewidth=0.04cm](12.321875,-1.1917187)(12.321875,-1.5317187) 
% \usefont{T1}{ptm}{m}{n} 
\rput(11.094063,0.23828125){\textbf{6.}} 
% \usefont{T1}{ptm}{m}{n} 
\rput(12.7517185,-1.6417187){$5$ cm}
%  \usefont{T1}{ptm}{m}{n} 
\rput(13.181562,-0.18171875){$6$ cm}
\pstriangle[linewidth=0.04,dimen=outer](2.031875,-4.491719)(3.4,2.4)
%\pstriangle[linewidth=0.04,dimen=outer](3.181875,-2.6717188)(0.0,0.0) 
%\pstriangle[linewidth=0.04,dimen=outer](3.161875,-2.7517188)(0.0,0.0) 
%\pstriangle[linewidth=0.04,dimen=outer](3.001875,-2.5317187)(0.0,0.0) 
% \usefont{T1}{ptm}{m}{n} 
\rput(0.23140626,-2.2617188){\textbf{7.}} 
\psline[linewidth=0.04cm](1.901875,-4.331719)(1.901875,-4.6517186) 
\psline[linewidth=0.04cm](1.101875,-3.1117187)(1.341875,-3.3317187)
\psline[linewidth=0.04cm](2.6844237,-3.3744361)(2.9193263,-3.1490014)
%  \usefont{T1}{ptm}{m}{n} 
\rput(3.6565626,-3.1417189){$10$ cm}
\pspolygon[linewidth=0.04](5.201875,-4.371719)(9.201875,-4.371719)(8.181875,-2.6717188)(6.181875,-2.6717188)(6.001875,-2.9717188) \psline[linewidth=0.04cm](6.161875,-2.6717188)(6.161875,-2.6717188) 
\psline[linewidth=0.04cm](6.161875,-2.6717188)(6.161875,-2.6717188)
\psline[linewidth=0.04cm,linestyle=dashed,dash=0.16cm 0.16cm](6.161875,-2.6717188)(6.161875,-4.351719) 
\psline[linewidth=0.04cm](7.106746,-2.5312762)(7.309965,-2.6871674)
\psline[linewidth=0.04cm](7.118735,-2.8573537)(7.3044972,-2.6810234)
\psline[linewidth=0.04cm](7.106746,-4.211276)(7.309965,-4.3671675) 
\psline[linewidth=0.04cm](7.118735,-4.5373535)(7.3044972,-4.3610234) 
\psframe[linewidth=0.04,dimen=outer](6.471875,-4.061719)(6.151875,-4.3817186)
%  \usefont{T1}{ptm}{m}{n}
\rput(5.3565626,-2.9817188){$15$ cm} 
% \usefont{T1}{ptm}{m}{n} 
\rput(5.5398436,-4.601719){$9$ cm} 
% \usefont{T1}{ptm}{m}{n} 
\rput(7.6565623,-2.4017189){$16$ cm} 
% \usefont{T1}{ptm}{m}{n}
\rput(7.9285936,-4.601719){$21$ cm} 
% \usefont{T1}{ptm}{m}{n} 
\rput(5.08375,-2.2617188){\textbf{8.}}
\end{pspicture}
}

\end{center}

}


\raisebox{-5 pt}{\includegraphics[width=0.5cm]{col11306.imgs/summary_www.png}}
Find the answers with the short codes:\\
\begin{tabularx}{\textwidth}{ XX }
(1)	&	(2)
\end{tabularx}
\end{exercises}


\section{Right prisms and cylinders }

A right prism is a polygon that has been stretched into a tube so that the height of the tube is
perpendicular to the base. The base and top surface is therefore the same shape and size. We
call it right prisms because the angle between the base and side form a right-angle.
\par
A square based prism has a square as its base, a rectangular based prism has a rectangle as its
base, and triangular based prism has a triangle as its base. A cylinder is another type of right
prism which has as circle as its base.
\par 

\begin{figure}[H]
    \begin{center}
	%% Rectangular Prism
	\begin{pspicture}(-2,-3.5)(3,1.5)
	    \psset{yunit=0.9,xunit=0.9}
	    \psset{Alpha=60,Beta=30}
	    \pstThreeDBox[hiddenLine](-1,1,2)(0,0,1)(2,0,0)(0,3.5,0)
	    \psset{fillcolor=cyan,fillstyle=solid,opacity=0.5,linestyle=none,dotstyle=none}
	    \pstThreeDSquare(-1,1,2)(2,0,0)(0,3.5,0)
	    
	\end{pspicture}
\hspace{10pt}
	%% Square Prism
	\begin{pspicture}(-2,-3.5)(3,1.5)
	    \psset{yunit=0.9,xunit=0.9}
	    \psset{Alpha=60,Beta=30}
	    {\psset{fillcolor=cyan,fillstyle=solid,opacity=0.5,linestyle=none,dotstyle=}
	    \pstThreeDSquare(-1,1,2)(2.5,0,0)(0,2.5,0)}
	    \pstThreeDBox[hiddenLine](-1,1,2)(0,0,2.5)(2.5,0,0)(0,2.5,0)
	\end{pspicture}
\hspace{10pt}
	%% Triangular Prism
	\begin{pspicture}(-2,-3.5)(3,1.5)
	    \psset{yunit=0.9,xunit=0.9}
	    \psset{Alpha=60,Beta=30}
\psSolid[object=prisme,action=draw,axe=0 0 1,base=-0.5 -0.5 0.5 -0.5 0 0.5,h=1.0]
{\psset{fillcolor=cyan,fillstyle=solid,opacity=0.5,linestyle=solid,dotstyle=}
	    \psSolid[object=face,base=-0.5 -0.5 0.5 -0.5 0 0.5](0,0,0)}
% 	    \psset{viewpoint=8 12 8}

	\end{pspicture}
\hspace{10pt}
	%% Cylindrical Prism
	\begin{pspicture}(-2,-3.5)(3,1.5)
	    \psset{yunit=0.9,xunit=0.9}
	    \psset{Alpha=60,Beta=30}
	    \psellipse[fillcolor=white,fillstyle=solid](0,-3)(1.0,0.5)
	    \psframe[linestyle=none,fillcolor=white,fillstyle=solid](-1,-3)(1,0)
	    \psellipse[fillcolor=cyan,opacity=0.5,fillstyle=solid,linestyle=dashed](0,-3)(1.0,0.5)
	    \psellipse[fillstyle=none](0,-0)(1.0,0.5)
	    \psline(-1.0,-3)(-1.0,-0)
	    \psline(1.0,-3)(1.0,-0)
	\end{pspicture}

	\vspace{0.75cm}
	\caption{Examples of a right prism, a right square prism, a right triangular prism and a cylinder.}
    \end{center}
\end{figure}   

\subsection*{Surface Area of prisms and cylinders}

 The term surface area refers to the total area of the exposed or outside surfaces of a prism.
This is easier to understand if you imagine the prism as a cardboard box that you can unfold.
A solid that is unfolded like this, is called a net.
\par 

 When a triangular prism is unfolded into a net, it is easier to see that it has two faces that are
triangles and three faces that are rectangles. Therefore, in order to calculate the surface area
of a prism you simply have to calculate the area of each face and add it up.
  \par
In the case of a cylinder the top and bottom faces are circles, while the curved surface flattens
into a rectangle with a length that is equal to the circumference of the circular base.

\Note{Remember that surface area is measured in square units, for example cm $^2$, m$^2$ or mm$^2$.}
\par
Below are examples of right prisms that have been unfolded to their nets:

\begin{figure}[H]
 \caption{Rectangular based prism}
   \begin{center}
	%% Rectangular Prism
    \scalebox{0.8}{
        \begin{pspicture}(-2,-3.5)(3,1.5)
	    \psset{yunit=0.9,xunit=0.9}
	    \psset{Alpha=60,Beta=30}
	    {\psset{fillcolor=cyan,fillstyle=solid,opacity=0.5,linestyle=none,dotstyle=}
	    \pstThreeDSquare(-1,1,2)(2,0,0)(0,3.5,0)}
	    \pstThreeDBox[hiddenLine](-1,1,2)(0,0,1)(2,0,0)(0,3.5,0)
	\end{pspicture}}

	%% Rectangular Prism Unfolded
    \scalebox{0.8}{
        \begin{pspicture}(-2,-3.5)(3,1.5)
	    \psset{yunit=0.9,xunit=0.9}
	    \psset{Alpha=60,Beta=30}
	    {\psset{fillcolor=cyan,fillstyle=solid,opacity=0.5,linestyle=solid,dotstyle=}
	    \pstThreeDSquare(-1,1,2)(2,0,0)(0,3.5,0)}
	    \pstThreeDSquare[fillcolor=white,fillstyle=solid,opacity=0.5](-2,1,2)(1,0,0)(0,3.5,0)
	    \pstThreeDSquare[fillcolor=white,fillstyle=solid,opacity=0.5](1,1,2)(1,0,0)(0,3.5,0)
	    \pstThreeDSquare[fillcolor=white,fillstyle=solid,opacity=0.5](2,1,2)(2,0,0)(0,3.5,0)
	    \pstThreeDSquare[fillcolor=white,fillstyle=solid,opacity=0.5](-1,4.5,2)(2,0,0)(0,1,0)
	    \pstThreeDSquare[fillcolor=white,fillstyle=solid,opacity=0.5](-1,0,2)(2,0,0)(0,1,0)
	\end{pspicture}}

\vspace{10pt}

	%% Rectangular Prism Unfolded Birdseye
\scalebox{0.8}{ 
	\begin{pspicture}(-2,-3.5)(3,1.5)
	    \psset{yunit=0.9,xunit=0.9}
	    \psset{Alpha=90,Beta=90}
	    {\psset{fillcolor=cyan,fillstyle=solid,opacity=0.5,linestyle=solid,dotstyle=}
	    \pstThreeDSquare(-1,1,2)(2,0,0)(0,3.5,0)}
	    \pstThreeDSquare[fillcolor=white,fillstyle=solid,opacity=0.5](-2,1,2)(1,0,0)(0,3.5,0)
	    \pstThreeDSquare[fillcolor=white,fillstyle=solid,opacity=0.5](1,1,2)(1,0,0)(0,3.5,0)
	    \pstThreeDSquare[fillcolor=white,fillstyle=solid,opacity=0.5](2,1,2)(2,0,0)(0,3.5,0)
	    \pstThreeDSquare[fillcolor=white,fillstyle=solid,opacity=0.5](-1,4.5,2)(2,0,0)(0,1,0)
	    \pstThreeDSquare[fillcolor=white,fillstyle=solid,opacity=0.5](-1,0,2)(2,0,0)(0,1,0)
	\end{pspicture}}
\newline
    \end{center}
\end{figure}   

A rectangular based prism as made up of six rectangles. To find the total surface area, one
would need to find the sum of the area of each face.


\begin{figure}[H]
\caption{Square based prism}
    \begin{center}

	%% Square Prism
    \scalebox{0.8}{ 
        \begin{pspicture}(-2,-3.5)(3,1.5)
            \psset{yunit=0.9,xunit=0.9}
	    \psset{Alpha=60,Beta=30}
	    {\psset{fillcolor=cyan,fillstyle=solid,opacity=0.5,linestyle=none,dotstyle=}
	    \pstThreeDSquare(-1,1,2)(2.5,0,0)(0,2.5,0)}
	    \pstThreeDBox[hiddenLine](-1,1,2)(0,0,2.5)(2.5,0,0)(0,2.5,0)
	\end{pspicture}}

	%% Square Prism Unfolded
    \scalebox{0.8}{
	\begin{pspicture}(-2,-3.5)(3,1.5)
	    \psset{yunit=0.9,xunit=0.9}
	    \psset{Alpha=60,Beta=30}
	    {\psset{fillcolor=cyan,fillstyle=solid,opacity=0.5,linestyle=solid,dotstyle=}
	    \pstThreeDSquare(-1,1,2)(2.5,0,0)(0,2.5,0)}
	    \pstThreeDSquare[fillcolor=white,fillstyle=solid,opacity=0.5](-3.5,1,2)(2.5,0,0)(0,2.5,0)
	    \pstThreeDSquare[fillcolor=white,fillstyle=solid,opacity=0.5](-1,3.5,2)(2.5,0,0)(0,2.5,0)
	    \pstThreeDSquare[fillcolor=white,fillstyle=solid,opacity=0.5](1.5,1,2)(2.5,0,0)(0,2.5,0)
	    \pstThreeDSquare[fillcolor=white,fillstyle=solid,opacity=0.5](-1,-1.5,2)(2.5,0,0)(0,2.5,0)
	    \pstThreeDSquare[fillcolor=white,fillstyle=solid,opacity=0.5](4,1,2)(2.5,0,0)(0,2.5,0)
	\end{pspicture}}

	%% Square Prism Unfolded Birdseye
    \scalebox{0.8}{
	\begin{pspicture}(-2,-3.5)(3,1.5)
	    \psset{yunit=0.9,xunit=0.9}
	    \psset{Alpha=0,Beta=90}
	    {\psset{fillcolor=cyan,fillstyle=solid,opacity=0.5,linestyle=solid,dotstyle=}
	    \pstThreeDSquare(-1,1,2)(2.5,0,0)(0,2.5,0)}
	    \pstThreeDSquare[fillcolor=white,fillstyle=solid,opacity=0.5](-3.5,1,2)(2.5,0,0)(0,2.5,0)
	    \pstThreeDSquare[fillcolor=white,fillstyle=solid,opacity=0.5](-1,3.5,2)(2.5,0,0)(0,2.5,0)
	    \pstThreeDSquare[fillcolor=white,fillstyle=solid,opacity=0.5](1.5,1,2)(2.5,0,0)(0,2.5,0)
	    \pstThreeDSquare[fillcolor=white,fillstyle=solid,opacity=0.5](-1,-1.5,2)(2.5,0,0)(0,2.5,0)
	    \pstThreeDSquare[fillcolor=white,fillstyle=solid,opacity=0.5](4,1,2)(2.5,0,0)(0,2.5,0)
	\end{pspicture}}

    \end{center}
\end{figure}   

A square based prism (or a cube) can be flattened to a net with six identical squares. To find
the total surface area of the solid, find the sum of the area of the faces.


\begin{figure}[H]
    \caption{Triangular based prism}
    \begin{center}
	%% Triangular Prism
    \scalebox{0.8}{
	\begin{pspicture}(-1.5,-1.5)(1,1)
	    \psset{yunit=0.9,xunit=0.9}
% 	    \psset{Alpha=60,Beta=30}
% 	    \psset{viewpoint=0 0 200}
	    \psSolid[object=face,fillcolor=cyan,opacity=0.5,base=-0.5 -0.5 0.5 -0.5 0 0.5](0,0,0)
	    \psSolid[object=prisme,action=draw,axe=0 0 1,base=-0.5 -0.5 0.5 -0.5 0 0.5,h=0.4]
	\end{pspicture}}

	%% Triangular Prism Unfolded
    \scalebox{0.8}{
	\begin{pspicture}(-2,-3.5)(3,1.5)
	    \psset{yunit=0.9,xunit=0.9}
% 	    \psset{Alpha=60,Beta=30}
% 	    \psset{viewpoint=8 12 8}
	    \psSolid[object=face,fillcolor=cyan,opacity=0.5,base=-0.5 -0.5 0.5 -0.5 0 0.5](0,0,0)
	    \psSolid[object=face,fillcolor=cyan,incolor=white,base=-0.5 -0.5 0.5 -0.5 0.5 -0.9 -0.5 -0.9](0,0,0)
	    \psSolid[object=face,fillcolor=cyan,incolor=white,base=-0.5 -0.9 0.5 -0.9 0 -1.9](0,0,0)
	    \psSolid[object=face,fillcolor=cyan,incolor=white,base=-0.5 -0.5 -0.8464102 -0.3 -0.3464102 0.7 0.0 0.5](0,0,0)
	    \psSolid[object=face,fillcolor=cyan,incolor=white,base=0.0 0.5 0.3464102 0.7 0.8464102 -0.3 0.5 -0.5](0,0,0)
	\end{pspicture}}

	%% Triangular Prism Unfolded Birdseye
    \scalebox{0.8}{
	\begin{pspicture}(-2,-3.5)(3,1.5)
	    \psset{yunit=0.9,xunit=0.9}
% 	    \psset{Alpha=60,Beta=30}
	    \psset{viewpoint=0 0 20,RotZ=0}
	    \psSolid[object=face,fillcolor=cyan,opacity=0.5,base=-0.5 -0.5 0.5 -0.5 0 0.5](0,0,0)
	    \psSolid[object=face,fillcolor=cyan,incolor=white,base=-0.5 -0.5 0.5 -0.5 0.5 -0.9 -0.5 -0.9](0,0,0)
	    \psSolid[object=face,fillcolor=cyan,incolor=white,base=-0.5 -0.9 0.5 -0.9 0 -1.9](0,0,0)
	    \psSolid[object=face,fillcolor=cyan,incolor=white,base=-0.5 -0.5 -0.8464102 -0.3 -0.3464102 0.7 0.0 0.5](0,0,0)
	    \psSolid[object=face,fillcolor=cyan,incolor=white,base=0.0 0.5 0.3464102 0.7 0.8464102 -0.3 0.5 -0.5](0,0,0)
	\end{pspicture}}


    \end{center}
\end{figure} 

The triangular based prism has two triangles and one rectangle which is made up of three
smaller rectangles. The length of the rectangle is equal to the perimeter of the triangle.


\begin{figure}[H]
\caption{Cylinder}
\begin{center}

	%% Cylindrical Prism
	\begin{pspicture}(-2,-3.5)(3,1.5)
	    \psset{yunit=0.9,xunit=0.9}
	    \psellipse[fillcolor=white,fillstyle=solid](0,-3)(1.0,0.5)
	    \psframe[linestyle=none,fillcolor=white,fillstyle=solid](-1,-3)(1,0)
	    \psellipse[fillcolor=cyan,opacity=0.5,fillstyle=solid,linestyle=dashed](0,-3)(1.0,0.5)
	    \psellipse[fillstyle=none](0,-0)(1.0,0.5)
	    \psline(-1.0,-3)(-1.0,-0)
	    \psline(1.0,-3)(1.0,-0)
	\end{pspicture}
\hspace{20pt}
	%% Cylindrical Prism Unfolded
	\begin{pspicture}(-2,-3.5)(3,1.5)
	    \psset{yunit=0.9,xunit=0.9}
	    \psellipse[fillcolor=cyan,opacity=0.5,fillstyle=solid,linestyle=solid](4,-4)(1.0,1.0)
	    \psellipse[fillstyle=none](0,1)(1.0,1)
	    \psframe[linestyle=solid,fillcolor=white,fillstyle=solid](-1,-3)(5,0)
	\end{pspicture}

    \end{center}
\end{figure}   

The net of a cylinder is made up of two identical circles and a rectangle with a length equal to
the circumference of the circles.


\begin{wex}
{Finding the surface area of a rectangular prism

}
{%Problem
Find the total surface area of the prism below
\begin{center}
\scalebox{0.9}{
    \begin{pspicture}(-2,-3.5)(3,1.5)
        \psset{yunit=0.9,xunit=0.9}
        \psset{Alpha=30,Beta=15}
        \pstThreeDBox(-1,1,2)(0,0,2)(10,0,0)(0,5,0)
        \pstThreeDPut(8,6,3){$2$ cm}
        \pstThreeDPut(10,3.5,2){$5$ cm}
        \pstThreeDPut(5,7,2){$10$ cm}
   \end{pspicture}}
\end{center}

}
{%Solution

\westep{Draw a rough sketch of the net of the solid, filling in the relevant values}

\begin{center}
	%% Rectangular Prism Unfolded Birdseye
\scalebox{1}{ 
	\begin{pspicture}(-2,-3.5)(3,1.5)
	    \psset{yunit=0.9,xunit=0.9}
	    \psset{Alpha=180,Beta=90}
	    {\psset{fillcolor=white,fillstyle=solid,opacity=0.5,linestyle=solid,dotstyle=}
	    \pstThreeDSquare(-1,1,2)(2,0,0)(0,3.5,0)}
	    \pstThreeDSquare[fillcolor=white,fillstyle=solid,opacity=0.5](-2,1,2)(1,0,0)(0,3.5,0)
	    \pstThreeDSquare[fillcolor=white,fillstyle=solid,opacity=0.5](1,1,2)(1,0,0)(0,3.5,0)
	    \pstThreeDSquare[fillcolor=white,fillstyle=solid,opacity=0.5](2,1,2)(2,0,0)(0,3.5,0)
	    \pstThreeDSquare[fillcolor=white,fillstyle=solid,opacity=0.5](-1,4.5,2)(2,0,0)(0,1,0)
	    \pstThreeDSquare[fillcolor=white,fillstyle=solid,opacity=0.5](-1,0,2)(2,0,0)(0,1,0)
            \rput(0,6.){$5$}
            \rput(1.5, 5.){$2$}
            \rput(4.5, 3.){$10$}
	\end{pspicture}}
\end{center}

\westep{Find the area of the different shapes}
\begin{align*} 
\mbox{large rectangle} &= \mbox{perimeter of small rectangle} \times \mbox{length} \\
                        &= (5 +2 + 5 + 2) \times 10 \\
                        &= 14 \times 10 \\
                        &= 140~\mbox{cm}^2  \\ \\
2 \times \mbox{small rectangle} &= 2(5+2) \\
                                &= 2(10) \\
                                &= 20~\mbox{cm}^2
\end{align*}



\westep{Find the sum of the the area of the faces}

$\mbox{large rectangle} + 2 \times \mbox{small rectangle} = 140 + 20 = 180~$cm$^2$

}
\end{wex}


\begin{wex}
{Finding the surface area of a triangular prism

}
{
Find the total surface area of the prism below:\\
\begin{center}
\scalebox{1}{
	%% Triangular Prism
\begin{pspicture}(0,-1.59875)(4.7,1.61875)
\pstriangle[linewidth=1pt,dimen=outer](1.09,-1.59875)(2.18,1.72)
\psline[linewidth=1pt](1.08,0.1012499)(3.38,1.59875)
\psline[linewidth=1pt,linestyle=dotted,dotsep=0.10583334cm](0.06,-1.55875)(2.76,0.15875)
\psline[linewidth=1pt](3.38,1.57875)(4.64,0.2012498)
\psline[linewidth=1pt](2.14,-1.57875)(4.66,0.20125)
\psline[linewidth=1pt,linestyle=dashed,dash=0.16cm 0.16cm](1.08,0.06125)(1.06,-1.57875)
\psline[linewidth=1pt,linestyle=dotted,dotsep=0.10583334cm](3.38,1.54125)(2.74,0.17875)
\psline[linewidth=1pt,linestyle=dotted,dotsep=0.10583334cm](4.68,0.19875)(2.78,0.17875)
\psline[linewidth=0.04](1.089162,-1.3)(1.309162,-1.3)(1.309162,-1.5)(1.309162,-1.6)
\rput(1.07, -1.9){$8$ cm}
\rput(4,-1){$12$ cm}
\rput(1.5,-1.1){$3$ cm}
\end{pspicture} 
}
\end{center}


}
{%Solution

\westep{Draw a rough sketch of the net of the solid, filling in the relevant values}

% LaTeX Draw
% \usepackage{pst-plot} % For axes
\begin{center}
\scalebox{1} % Change this value to rescale the drawing.
{
\begin{pspicture}(0,-3.538172)(7.7584376,3.5381718)
\psframe[linewidth=0.02,dimen=outer](6.3,2.2243125)(0.0,-2.2243125)
\pstriangle[linewidth=0.02,dimen=outer](3.15,2.2072406)(2.87,1.3345875)
\rput{-180.0}(6.3,-5.7490687){\pstriangle[linewidth=0.02,dimen=outer](3.15,-3.541828)(2.87,1.3345875)}
% \usefont{T1}{ptm}{m}{n}
\rput(3.541875,2.6831717){$3$ cm}
\psline[linewidth=0.02cm,linestyle=dashed,dash=0.16cm 0.16cm](3.14,3.5181718)(3.14,2.2181718)
% \usefont{T1}{ptm}{m}{n}
\rput(7,0.083171844){$12$ cm}
% \usefont{T1}{ptm}{m}{n}
\rput(3.3982813,1.7031718){$8$ cm}
\psline[linewidth=0.02cm,linestyle=dashed,dash=0.16cm 0.16cm](1.74,2.2181718)(1.76,-2.2218282)
\psline[linewidth=0.02cm,linestyle=dashed,dash=0.16cm 0.16cm](4.52,2.2381718)(4.54,-2.2018282)
\end{pspicture} 
}
\end{center}
\westep{Find the area of the different shapes}

large rectangle = perimeter of triangle $\times$ length

To find the perimeter of the triangle, we have to first find the length of the sides using Pythagoras:

\par
\begin{center}
\scalebox{1} % Change this value to rescale the drawing.
{
\begin{pspicture}(0,-0.97432816)(2.8562837,0.9306719)
\pstriangle[linewidth=0.02,dimen=outer](1.435,-0.40025938)(2.87,1.3345875)
\rput(1.826875,0.075671844){$3$ cm}
\psline[linewidth=0.02cm,linestyle=dashed,dash=0.16cm 0.16cm](1.425,0.91067183)(1.425,-0.38932815)
\rput(0.71140623,-0.6){$4$ cm}
\rput(2.0314062,-0.6){$4$ cm}
\end{pspicture} 
}
\end{center}


\begin{align*}
x^2 &= 3^2 + 4^2 \\
x &= \sqrt{25} \\
x &= 5
\end{align*}

$\therefore \mbox{perimeter of triangle} = 5 + 5 + 8 $

large rectange = perimeter of triangel + length

\begin{align*}
\phantom{stuffs} &= (5+5+8) \times 12 \\
                 &= 18 \times 12 \\
                 &= 216 \mbox{ cm}^2 \\
2 \times \mbox{triangle} &= 2(\dfrac{1}{2} \times 8 \times 3) \\
                    &= 2(12) \\
                    &= 24\mbox{ cm}^2\\
\end{align*}


\westep{Find the sum of the areas of the faces}

large rectangle + $2\times$ triangle $ = 216 + 24 = 240\mbox{ cm}^2$

}
\end{wex}



\begin{wex}
{Finding the area of a cylindrical prism
}
{% Question
Find the total surface are of the prism below to one decimal place.\\
\begin{center}
        \begin{pspicture}(-2,-3.5)(3,1.5)
	    \psset{yunit=0.9,xunit=0.9}
	    \psellipse(0,-3)(1.0,0.5)
	    \psframe[linestyle=none,](-1,-3)(1,0)
	    \psellipse[linestyle=dashed](0,-3)(1.0,0.5)
	    \psellipse[](0,-0)(1.0,0.5)
	    \psline(-1.0,-3)(-1.0,-0)
	    \psline(1.0,-3)(1.0,-0)
            \psline(0,0)(1.0,0)
            \rput(1.5,-1.5){$30$ cm}
            \rput(0.3,0.2){$10$ cm}
	\end{pspicture}
\end{center}



}
{% solution

\westep{Draw a rough sketch of the net of the solid, filling in the relevant values:}
\begin{center}
	%% Cylindrical Prism Unfolded
	\begin{pspicture}(-2,-3.5)(3,1.5)
	    \psset{yunit=0.9,xunit=0.9}
	    \psellipse[linestyle=solid](4,-4)(1.0,1.0)
	    \psellipse(0,1)(1.0,1)
	    \psframe[linestyle=solid](-1,-3)(5,0)
            \rput(0.2,1.2){$10$ cm}
            \rput(5.7,-1.5){$30$ cm}
\psline(0,1)(1,1)
	\end{pspicture}
\end{center}



\westep{Find the area of the different shapes}

\begin{align*}
\mbox{large rectangle} &= \mbox{circumference of circle} \times \mbox{length} \\
                        &= 2\pi(10) \times 30 \\
                        &= 1884,96 \mbox{ cm}^2 \\
       2\times \mbox{circle} &= 2(\pi(10^2)) \\
                             &= 628,32\mbox{ cm}^2
\end{align*}

\westep{Find the sum of the area of the faces}
large rectangle + $2\times$ circle = $1884,96 + 628,32$ = $2513,3\mbox{ cm}^2$


}
\end{wex}






\begin{exercises}{ }

Calculate the surface area in each of the following:

\setcounter{subfigure}{0}
\begin{figure}[H]
\begin{center}
\begin{pspicture}(0,-4.355625)(11.259218,4.355625)
\psline[linewidth=0.04cm](0.35640624,1.4571873)(1.3564062,2.4371874)
\psline[linewidth=0.04cm](3.3564062,1.4771873)(4.336406,2.4371874)
\psline[linewidth=0.04cm](0.37640625,1.4571873)(3.3564062,1.4771873)
\psline[linewidth=0.04cm](1.3564062,2.4571874)(4.336406,2.4571874)
\psline[linewidth=0.04cm](4.356406,3.4971871)(4.336406,2.3971872)
\psline[linewidth=0.04cm](1.3564062,3.4771872)(1.3764062,2.4571874)
\psline[linewidth=0.04cm](1.3764062,3.4571872)(4.356406,3.4771872)
\psline[linewidth=0.04cm](3.3564062,2.5371873)(3.3564062,1.4771873)
\psline[linewidth=0.04cm](0.35640624,2.4971874)(0.35640624,1.4371873)
\psline[linewidth=0.04cm](0.35640624,2.5171874)(3.3364062,2.5171874)
\psline[linewidth=0.04cm](0.37640625,2.5171874)(1.3564062,3.4571872)
\psline[linewidth=0.04cm](3.3564062,2.5171876)(4.336406,3.4571872)
\psellipse[linewidth=0.04,dimen=outer](8.986406,1.8371873)(0.99,0.38)
\psellipse[linewidth=0.04,dimen=outer](8.986406,3.1771872)(0.99,0.38)
\psline[linewidth=0.04cm](8.016406,3.1371875)(8.016406,1.8771874)
\psline[linewidth=0.04cm](9.956407,3.1771872)(9.956407,1.8571873)
\psline[linewidth=0.04cm,linestyle=dashed,dash=0.16cm 0.16cm](9.016406,1.8171873)(9.936406,1.8371873)
% \usefont{T1}{ptm}{m}{n}
\rput(9.005782,1.9671873){$5$ cm}
% \usefont{T1}{ptm}{m}{n}
\rput(10.454218,2.5871873){$10$ cm}
\pstriangle[linewidth=0.04,dimen=outer](2.1664062,-3.9028125)(2.18,1.72)
\psline[linewidth=0.04cm](2.1564062,-2.2028127)(4.516406,-0.7828127)
\psline[linewidth=0.04cm](1.1364063,-3.8628125)(4.3164062,-1.7828125)
\psline[linewidth=0.04cm](4.516406,-0.7628127)(5.7364063,-2.1028128)
\psline[linewidth=0.04cm](3.2164063,-3.8828125)(5.7364063,-2.1028125)
\psline[linewidth=0.04cm,linestyle=dashed,dash=0.16cm 0.16cm](2.1564062,-2.2428124)(2.1364062,-3.8828125)
% \usefont{T1}{ptm}{m}{n}
\rput(1.4389063,-1.2678125){\textbf{3.}}
% \usefont{T1}{ptm}{m}{n}
\rput(4.8967185,-3.1528125){$20$ cm}
% \usefont{T1}{ptm}{m}{n}
\rput(2.5,-3.4){$5$ cm}
% \usefont{T1}{ptm}{m}{n}
\rput(2.2596877,-4.1528125){$10$ cm}
% \usefont{T1}{ptm}{m}{n}
\rput(0.96203125,4.1521873){\textbf{1.}}
% \usefont{T1}{ptm}{m}{n}
\rput(7.763125,4.0121875){\textbf{2.}}
% \usefont{T1}{ptm}{m}{n}
\rput(1.7796875,1.2271875){$6$ cm}
% \usefont{T1}{ptm}{m}{n}
\rput(4.3628125,1.8271875){$7$ cm}
% \usefont{T1}{ptm}{m}{n}
\rput(4.8039064,2.9271874){$10$ cm}
\psline[linewidth=0.04cm](4.516406,-0.7628125)(4.3164062,-1.8028125)
\psline[linewidth=0.04cm](5.7364063,-2.1028125)(4.2764063,-1.8028125)
\end{pspicture}     
\end{center}
\end{figure}   

If a litre of paint covers an area of $2$ m$^{2}$, how much paint does a painter need to cover:
\begin{enumerate}[noitemsep, label=\textbf{\arabic*}. ] 
\setcounter{enumi}{3}
\item A rectangular swimming pool with dimensions $4$ m $\times~3$ m $\times~2,5$ m, inside walls and floor only.
\item The inside walls and floor
of a circular reservoir with diameter $4$ m and height $2,5$ m 
\end{enumerate}

\setcounter{subfigure}{0}


\begin{figure}[H] % horizontal\label{m39357*id62926}
\begin{center}
\begin{pspicture}(0,-1.6)(3.5434375,1.6) 
\psellipse[linewidth=0.04,dimen=outer](1.29,-1.09)(1.27,0.51) 
\psellipse[linewidth=0.04,dimen=outer](1.27,1.09)(1.27,0.51) 
\psline[linewidth=0.04cm](0.04,-1.1)(0.02,1.14) 
\psline[linewidth=0.04cm](2.54,1.12)(2.56,-1.1) 
\psline[linewidth=0.04cm,linestyle=dashed,dash=0.16cm 0.16cm](0.06, -1.1)(2.52,-1.1) 
\rput(1.2925,-0.95){$4$ m} 
\rput(3.0567188,0.25){$2,5$ m} 
\end{pspicture} 
\end{center}

\end{figure}   

\addtocounter{footnote}{-0}

        
}
\end{exercises}        
\subsection*{Volume of prisms and cylinders}

The volume of a right prism is very easy to calculate. Once you have found the area of the
base of the solid, multiply it by the height. Remember that the volume is measured in cubic
units, for example cm$^3$, m$^3$ or mm$^3$.


\begin{table*}[h]
\newcolumntype{C}{>{\centering\arraybackslash} m{0.75in} }
\newcolumntype{D}{>{\centering\arraybackslash} m{2.25in} }
\begin{tabular}{|C|D|D|}
\hline
\textbf{Rectangular prism}
&
\begin{center}
\begin{pspicture}(0,-1.2412498)(4.584375,1.2212498)
\psline[linewidth=0.04cm](0.0,-0.83875)(1.0,0.14125004)
\psline[linewidth=0.04cm](3.0,-0.81875)(3.98,0.14125004)
\psline[linewidth=0.04cm](0.02,-0.83875)(3.0,-0.81875)
\psline[linewidth=0.04cm](1.0,0.16124995)(3.98,0.16124995)
\psline[linewidth=0.04cm](4.0,1.2012498)(3.98,0.10124995)
\psline[linewidth=0.04cm](1.0,1.1812499)(1.02,0.16125014)
\psline[linewidth=0.04cm](1.02,1.1612499)(4.0,1.1812499)
\psline[linewidth=0.04cm](3.0,0.24124995)(3.0,-0.81875)
\psline[linewidth=0.04cm](0.0,0.20125005)(0.0,-0.85875)
\psline[linewidth=0.04cm](0.0,0.22125006)(2.98,0.22125006)
\psline[linewidth=0.04cm](0.02,0.22125006)(1.0,1.1612499)
\psline[linewidth=0.04cm](3.0,0.22125015)(3.98,1.1612499)
% \usefont{T1}{ptm}{m}{n}
\rput(1.2896875,-1.0687499){$l$}
% \usefont{T1}{ptm}{m}{n}
\rput(3.9228127,-0.46874985){$b$}
% \usefont{T1}{ptm}{m}{n}
\rput(4.273906,0.63125014){$h$}
\end{pspicture}
\end{center} 
&
$
\begin{aligned}
\mbox{Volume} &= \mbox{area of base} \times \mbox{height} \\
                &= \mbox{area of rectangle} \times \mbox{height} \\
                &= l \times b \times h \\
\end{aligned}$   \\ \hline


\textbf{Triangular prism} &

\scalebox{1} % Change this value to rescale the drawing.
{
\begin{pspicture}(0,-1.8337499)(4.9209375,1.8137499)
\pstriangle[linewidth=0.04,dimen=outer](1.3309375,-1.3462498)(2.18,1.72)
\psline[linewidth=0.04cm](1.3209375,0.35375)(3.6809375,1.77375)
\psline[linewidth=0.04cm](3.6809375,1.7937499)(4.9009376,0.4537499)
\psline[linewidth=0.04cm](2.3809376,-1.32625)(4.9009376,0.4537501)
\psline[linewidth=0.04cm,linestyle=dashed,dash=0.16cm 0.16cm](1.3209375,0.3137501)(1.3009375,-1.32625)
\psframe[linewidth=0.04,dimen=outer](1.4809375,-1.1462499)(1.2809376,-1.3462499)
\psline[linewidth=0.04cm](0.6209375,-0.3462499)(1.0009375,-0.5462499)
\psline[linewidth=0.04cm](0.5609375,-0.4462499)(0.9409375,-0.6462499)
\psline[linewidth=0.04cm](2.0409374,-0.3462499)(1.6609375,-0.5462499)
\psline[linewidth=0.04cm](2.1009376,-0.4462499)(1.7209375,-0.6462499)
% \usefont{T1}{ptm}{m}{n}
\rput(4.1923437,-0.5212499){$H$}
% \usefont{T1}{ptm}{m}{n}
\rput(1.2723438,-1.6612499){$b$}
% \usefont{T1}{ptm}{m}{n}
\rput(0.21234375,-0.2412499){$s$}
% \usefont{T1}{ptm}{m}{n}
\rput(1.4723438,-0.7612499){$h$}
\end{pspicture} 
}

&

$\begin{aligned}
\mbox{Volume} &= \mbox{area of base} \times \mbox{height} \\
                &= \mbox{area of triangle} \times \mbox{height} \\
                &= \dfrac{1}{2}b\times h \times H \\
\end{aligned}$  \\ \hline

\textbf{Cylinder} &

\scalebox{1} % Change this value to rescale the drawing.
{
\begin{pspicture}(0,-1.05)(2.761875,1.05)
\psellipse[linewidth=0.04,dimen=outer](0.99,-0.66999996)(0.99,0.38)
\psellipse[linewidth=0.04,dimen=outer](0.99,0.66999996)(0.99,0.38)
\psline[linewidth=0.04cm](0.02,0.63000023)(0.02,-0.6299999)
\psline[linewidth=0.04cm](1.96,0.66999996)(1.96,-0.65)
\psline[linewidth=0.04cm,linestyle=dashed,dash=0.16cm 0.16cm](1.02,-0.68999994)(1.94,-0.66999996)
\usefont{T1}{ptm}{m}{n}
\rput(2.4514062,0.09500025){$h$}
\usefont{T1}{ptm}{m}{n}
\rput(1.1014062,-0.50499976){$r$}
\end{pspicture} 
}

&

$\begin{aligned}
\mbox{Volume} &= \mbox{area of base} \times \mbox{height} \\
                &= \mbox{area of circle} \times \mbox{height} \\
                &= \pi r^2 \times h \\
\end{aligned}$  \\ \hline



\end{tabular}
\end{table*}






\begin{wex}
{% Title
Finding the volume of a cube
}
{% question
Find the volume of the following solid:
\begin{center}
\scalebox{1} % Change this value to rescale the drawing.
{
\begin{pspicture}(0,-1.0258813)(5.2842736,1.3541187)
\psdiamond[linewidth=0.04,dimen=outer,gangle=-49.7](1.6365356,0.0)(1.27,1.0687643)
\psdiamond[linewidth=0.04,dimen=outer,gangle=50.0](3.2365355,0.0)(1.27,1.0687643)
\psline[linewidth=0.04](0.8223987,0.96411866)(2.5223987,1.2441187)(4.0023985,1.0003091)
\psline[linewidth=0.04cm](0.6423987,0.14411865)(1.0223987,0.14411865)
\psline[linewidth=0.04cm](2.2423987,-0.115881346)(2.6223986,-0.115881346)
\psline[linewidth=0.04cm](3.8423986,0.044118654)(4.2223988,0.044118654)
\psline[linewidth=0.04cm](1.7123986,0.95411867)(1.7123986,0.5741187)
\psline[linewidth=0.04cm](1.5723987,-0.60588133)(1.5723987,-0.9858813)
\psline[linewidth=0.04cm](3.2723987,-0.6258814)(3.2723987,-1.0058813)
\psline[linewidth=0.04cm](3.1723988,1.0141187)(3.1723988,0.6341187)
\psline[linewidth=0.04cm](1.8723987,1.3341186)(1.8723987,0.95411867)
\psline[linewidth=0.04cm](3.0923986,1.3341186)(3.0923986,0.95411867)
% \usefont{T1}{ptm}{m}{n}
\rput(4.753805,0.109118655){$3$ cm}
\end{pspicture} 
}
\end{center}
}
{%solution

\westep{Find the area of the base}
\begin{align*}
\mbox{area of square} &= 3 \times 3 \\
                    &= 9~\mbox{ cm}^2
\end{align*}

\westep{Multiply this by the height of the solid in order to find the volume}
\begin{align*}
\mbox{volume} &= \mbox{area of base} \times \mbox{height}\\
                    &= 9 \times 3 \\
                    &= 27\mbox{ cm}^3 
\end{align*}

}
\end{wex}




\begin{wex}
{Finding the volume of a triangular prism
}

{%problem
Find the volume of the following solid:\\

\begin{center}
\scalebox{1} % Change this value to rescale the drawing.
{
\begin{pspicture}(0,-1.8337499)(4.6800003,1.8137499)
\pstriangle[linewidth=0.04,dimen=outer](1.09,-1.3462498)(2.18,1.72)
\psline[linewidth=0.04cm](1.08,0.35375)(3.44,1.77375)
\psline[linewidth=0.04cm](3.44,1.7937499)(4.6600003,0.4537499)
\psline[linewidth=0.04cm](2.14,-1.32625)(4.6600003,0.4537501)
\psline[linewidth=0.04cm,linestyle=dashed,dash=0.16cm 0.16cm](1.08,0.3137501)(1.06,-1.32625)
\psframe[linewidth=0.04,dimen=outer](1.24,-1.1462499)(1.0400001,-1.3462499)
\psline[linewidth=0.04cm](0.38,-0.3462499)(0.76,-0.5462499)
\psline[linewidth=0.04cm](0.32,-0.4462499)(0.7,-0.6462499)
\psline[linewidth=0.04cm](1.8,-0.3462499)(1.42,-0.5462499)
\psline[linewidth=0.04cm](1.8600001,-0.4462499)(1.48,-0.6462499)
% \usefont{T1}{ptm}{m}{n}
\rput(3.9828124,-0.5212499){$20$ cm}
% \usefont{T1}{ptm}{m}{n}
\rput(1.2328125,-1.6612499){$8$ cm}
% \usefont{T1}{ptm}{m}{n}
\rput(1.5,-0.8){$10$ cm}
\end{pspicture} 
}

\end{center}
}
{%solution
\westep{Find the area of the base}
\begin{align*}
\mbox{area of triangle} &= \dfrac{1}{2} \times 8 \times 10\\
                        &= 40\mbox{ cm}^2
\end{align*}


\westep{Multiply this by the height of the solid to find the volume}
\begin{align*}
\mbox{volume} &= \mbox{area of base} \times \mbox{height}\\
                        &= 40\mbox{ cm} \times 20 \\
                        &= 800\mbox{ cm}^3
\end{align*}


}
\end{wex}

\begin{wex}{Finding the volume of a cylindrical prism}
 {Find the volume of the following solid (correct to one decimal place):\\
\begin{center}
        \begin{pspicture}(-2,-3.5)(3,1.5)
	    \psset{yunit=0.9,xunit=0.9}
	    \psellipse(0,-3)(1.0,0.5)
	    \psframe[linestyle=none,](-1,-3)(1,0)
	    \psellipse[linestyle=dashed](0,-3)(1.0,0.5)
	    \psellipse[](0,-0)(1.0,0.5)
	    \psline(-1.0,-3)(-1.0,-0)
	    \psline(1.0,-3)(1.0,-0)
            \psline(0,0)(1.0,0)
            \rput(1.5,-1.5){$15$ cm}
            \rput(0.3,0.2){$4$ cm}
	\end{pspicture}
\end{center}
}
{
\westep{Find the area of the base}


\begin{align*}
\mbox{area of circle} &= \pi~r^2\\
&= \pi(4)^{2} \\
&= 50,265\mbox{ cm}^{2}\\

\end{align*}

\westep{Multiply this by the height of the solid to find the volume}
\begin{align*}
\mbox{volume} &= \mbox{area of base } \times \mbox{ height}\\
&= 50,265\times 15\\
&= 754,0\mbox{ cm}^{3}\\

\end{align*}
}
\end{wex}

\begin{exercises}{}
 
\end{exercises}

