\chapter{ Measurements  }  
This chapter examines the surface areas and volumes of three dimensional objects otherwise
known as solids. In order to work with these, you need to know how to calculate the surface
area and perimeter of the two dimensional shapes below.

\section{Areas of Polygons }

\begin{table*}[h]
\newcolumntype{C}{>{\centering\arraybackslash} m{1.5in} }
\newcolumntype{D}{>{\centering\arraybackslash} m{2in} }
\begin{tabular}{|C|D|C|}
\hline
\textbf{Name} & \textbf{Shape} & \textbf{Formulae} \\ \hline
Square &
        \begin{center}
\scalebox{0.9}{
        \begin{pspicture}(-2,-0.5)(5,2)
            \pspolygon(0,0)(0,1)(1,1)(1,0)
            \pspolygon(0,0)(0,0.15)(0.15,0.15)(0.15,0)
            \rput(-0.25,0.5){$h$}
            \rput(0.5,-0.25){$b$}
        \end{pspicture}}
        \end{center}
&

$b \times h$ \\ \hline

Rectangle &
    \begin{center}
\scalebox{0.9}{
        \begin{pspicture}(-2,-0.5)(5,2)
            \pspolygon(0,0)(0,1)(2.5,1)(2.5,0)
            \pspolygon(0,0)(0,0.15)(0.15,0.15)(0.15,0)
            \rput(2.75,0.5){$h$}
            \rput(1.25,-0.25){$b$}
        \end{pspicture}}
    \end{center}
& $ b \times h $ \\ \hline

Triangle &
\begin{center} 
\scalebox{0.9}{
    \begin{pspicture}(0,-0.5)(5,2)
        \pspolygon(0,0)(2,1)(3,0)
        \psline[linewidth=0.02cm,linestyle=dashed,dash=0.16cm 0.16cm](2,1)(2,0)
        \pspolygon(2,0)(2,0.15)(2.15,0.15)(2.15,0)
        \rput(1.8,0.5){$h$}
        \rput(1.5,-0.25){$b$}
    \end{pspicture}}
\end{center}

& $ \dfrac{1}{2} b \times h $ \\ \hline

Trapezium &
\begin{center}
\scalebox{0.9}{
    \begin{pspicture}(-2,-0.5)(5,2)
        \pspolygon(-1.5,0)(0,1)(2,1)(3,0)
        \psline[linewidth=0.02cm,linestyle=dashed,dash=0.16cm 0.16cm](2,1)(2,0)
        \pspolygon(2,0)(2,0.15)(2.15,0.15)(2.15,0)
        \rput(1.8,0.5){$h$}
        \rput(0.75,-0.25){$s_1$}
        \rput(1,1.25){$s_2$}
    \end{pspicture}}
\end{center}
& $ \dfrac{1}{2} (s_1 + s_2) \times h $ \\ \hline 
Parallelogram &

\begin{center}
\scalebox{0.9}{
    \begin{pspicture}(-2,-0.5)(5,2)
        \pspolygon(-1,0)(0,1)(3,1)(2,0)
        \psline[linewidth=0.02cm,linestyle=dashed,dash=0.16cm 0.16cm](0.5,1)(0.5,0)
        \pspolygon(0.5,0)(0.5,0.15)(0.65,0.15)(0.65,0)
        \rput(0.3,0.5){$h$}
        \rput(1,-0.25){$b$}
    \end{pspicture}}
\end{center}

& $ b \times h $ \\ \hline

Circle &

\begin{center}
\scalebox{0.9}{
    \begin{pspicture}(-2,-0.5)(5,2)
        \pscircle[dimen=outer](0.5,0.5){0.7}
        \psline[linestyle=dashed,dash=0.1cm 0.1cm](0.5,0.5)(1.2,0.5)
        \psdots[dotsize=0.08](0.5,0.5)
        \rput(0.85,0.75){$r$}
    \end{pspicture}}
\end{center} & 

$ \pi r^2 $ \\ \hline


\end{tabular}


%\caption{Remember that surface area is measured in square units, for example $cm^2$, $m^2$ or $mm^2$.}
\end{table*}
            

\begin{figure}[H]
    \textnormal{Khan Academy video on area and perimeter}
    \vspace{.1in}
    \nopagebreak
    \raisebox{-5 pt}{ 
    \includegraphics[width=0.5cm]{col11306.imgs/summary_www.png}} { (Video:  MG10096 )}
    \vspace{2pt}
    \vspace{.1in}
\end{figure}   
\par


\begin{figure}[H]
    \textnormal{Khan Academy video on area of a circle}
    \vspace{.1in}
    \nopagebreak
    \raisebox{-5 pt}{ \includegraphics[width=0.5cm]{col11306.imgs/summary_www.png}} { (Video:  MG10097 )}
    \vspace{2pt}
    \vspace{.1in}    
\end{figure}   

\begin{wex}{Finding the area of a parallelogram}{
    Find the area of the following figure:\\

    \begin{center}
	\begin{pspicture}(-2,-0.5)(5,2)
	    \pspolygon(-1,0)(0.5,2)(4,2)(2.5,0)
	    \psline[linewidth=0.4pt,linestyle=dashed,dash=0.16cm 0.16cm](0.5,2)(0.5,0)
	    \pspolygon(0.5,0)(0.5,0.3)(0.8,0.3)(0.8,0)
	    \rput(-1,-0.2){$A$}
	    \rput(0.5,2.2){$B$}
	    \rput(4,2.2){$C$}
	    \rput(2.5,-0.2){$D$}
	    \rput(0.5,-0.2){$E$}
	    \rput(-0.5,1){$5$}
	    \rput(-0.25,-0.2){$3$}
	    \rput(1.5,-0.2){$4$}
	\end{pspicture}
    \end{center}
    }{

    \westep{We first need to find the height,$ BE$, of the parallelogram. We can use the Theorem of Pythagoras to do this}
    \begin{eqnarray*}
	BE^2 &=& AB^2 - AE^2\\
	&=& 5^2 - 3^2\\
	 &=& 16\\
	\therefore BE &=& 4
    \end{eqnarray*}
    
    \westep{We apply the formula for the area of a parallelogram to find the area}
    \begin{eqnarray*}
	\text{Area} &=& h \times b\\
		    &=& 4 \times 7\\
		    &=& 28
    \end{eqnarray*}
    }
\end{wex}


\begin{exercises}{Polygons}


Find the areas of each of the given figures
\begin{center}
\scalebox{0.9}
{
\begin{pspicture}(0,-4.758281)(14.921875,4.758281)
\psline[linewidth=0.04cm](0.381875,1.7882812)(4.381875,1.7882812) 
\psline[linewidth=0.04cm](4.241875,1.7682812)(4.241875,1.7682812) 
\psline[linewidth=0.04cm,linestyle=dashed,dash=0.16cm 0.16cm](2.361875,3.75)(2.361875,2.8282812) 
\psline[linewidth=0.04cm](0.381875,1.7882812)(2.341875,3.7882812) 
\psline[linewidth=0.04cm](2.361875,3.7882812)(4.341875,1.8282813) 
% \usefont{T1}{ptm}{m}{n} 
\rput(2.4517188,2.5782812){$5$ cm} 
\psline[linewidth=0.04cm,linestyle=dashed,dash=0.16cm 0.16cm](2.361875,2.3482811)(2.361875,1.8) 
% \usefont{T1}{ptm}{m}{n} 
\rput(2.4365625,1.4182812){$10$ cm} 
% \usefont{T1}{ptm}{m}{n} 
\rput(0.16203125,3.6182814){\textbf{1.}} 
\psframe[linewidth=0.04,dimen=outer](9.141875,3.7782812)(5.141875,1.7782812) 
% \usefont{T1}{ptm}{m}{n} 
\rput(4.72375,3.5982811){\textbf{2.}} 
\psframe[linewidth=0.04,dimen=outer](5.471875,3.7782812)(5.151875,3.4582813) 
\psline[linewidth=0.04cm](7.121875,3.9282813)(7.321875,3.7682812) 
\psline[linewidth=0.04cm](7.1272163,3.6020272)(7.3165336,3.7745354) 
\psline[linewidth=0.04cm](7.121875,1.9482813)(7.321875,1.7882812) 
\psline[linewidth=0.04cm](7.1272163,1.6220272)(7.3165336,1.7945354)
\psline[linewidth=0.04cm](4.995617,2.7704508)(5.161965,2.965203)
\psline[linewidth=0.04cm](5.321874,2.7652996)(5.1555424,2.9600654) 
\psline[linewidth=0.04cm](4.995617,2.570451)(5.161965,2.765203) 
\psline[linewidth=0.04cm](5.321874,2.5652997)(5.1555424,2.7600653)
\psline[linewidth=0.04cm](8.955617,2.7704508)(9.121964,2.965203) 
\psline[linewidth=0.04cm](9.281874,2.7652996)(9.115542,2.9600654)
\psline[linewidth=0.04cm](8.955617,2.590451)(9.121964,2.785203) 
\psline[linewidth=0.04cm](9.281874,2.5852997)(9.115542,2.7800653)
%  \usefont{T1}{ptm}{m}{n} 
\rput(9.711719,2.7182813){$5$ cm}
%  \usefont{T1}{ptm}{m}{n} 
\rput(7.1965623,1.4182812){$10$ cm}
\psline[linewidth=0.04cm,linestyle=dashed,dash=0.16cm 0.16cm](10.901875,2.7482812)(14.901875,2.7482812) \pscircle[linewidth=0.04,dimen=outer](12.911875,2.7582812){2.0} 
\psdots[dotsize=0.16](12.901875,2.7482812)
%  \usefont{T1}{ptm}{m}{n} 
\rput(12.876562,2.4182813){$10$ cm} 
% \usefont{T1}{ptm}{m}{n}
\rput(10.631406,3.5982811){\textbf{3.}}
%  \usefont{T1}{ptm}{m}{n}
\rput(0.19140625,0.19828124){\textbf{4.}} 
% \usefont{T1}{ptm}{m}{n}
\rput(5.5117188,-0.8017188){$5$ cm}
%  \usefont{T1}{ptm}{m}{n}
\rput(2.5340624,0.29828125){$7$ cm}
\psline[linewidth=0.04cm](2.1600447,-1.3315424)(2.358717,-1.4931878) 
\psline[linewidth=0.04cm](2.162693,-1.6578295)(2.3534274,-1.4868897) 
\psline[linewidth=0.04cm](1.9601017,-1.3363228)(2.1587741,-1.4979682) 
\psline[linewidth=0.04cm](1.9627501,-1.6626099)(2.1534846,-1.4916701) 
\psline[linewidth=0.04cm](0.181875,-1.4917188)(4.181875,-1.4917188) 
\psline[linewidth=0.04cm](1.401875,0.04828125)(5.401875,0.04828125) 
\psline[linewidth=0.04cm](3.3800445,0.2084576)(3.578717,0.046812218)
\psline[linewidth=0.04cm](3.382693,-0.11782951)(3.5734274,0.053110242) 
\psline[linewidth=0.04cm](3.1801016,0.20367722)(3.3787742,0.042031832) 
\psline[linewidth=0.04cm](3.18275,-0.1226099)(3.3734844,0.048329853) 
\psline[linewidth=0.04cm](4.191091,-1.5022907)(5.394185,0.045147188) 
\psline[linewidth=0.04cm](4.621163,-0.6958544)(4.8714867,-0.6416498)
\psline[linewidth=0.04cm](4.877642,-0.89756864)(4.863264,-0.64184755) 
\psline[linewidth=0.04cm](0.21109095,-1.4822907)(1.4141852,0.065147184) 
\psline[linewidth=0.04cm](0.641163,-0.67585444)(0.89148647,-0.6216498) 
\psline[linewidth=0.04cm](0.8976424,-0.87756866)(0.8832641,-0.62184757)
\psline[linewidth=0.04cm,linestyle=dashed,dash=0.16cm 0.16cm](4.161875,-1.4917188)(4.161875,0.04828125) \psframe[linewidth=0.04,dimen=outer](4.471875,0.05828125)(4.151875,-0.26171875)
%  \usefont{T1}{ptm}{m}{n} 
\rput(4.7753124,0.27828124){$3$ cm} 
\psline[linewidth=0.04cm](6.021875,-1.4317187)(10.021875,-1.4317187)
\psline[linewidth=0.04cm](9.821875,-1.4517188)(9.821875,-1.4517188) 
% \usefont{T1}{ptm}{m}{n} 
\rput(7.5165625,-1.7217188){$12$ cm} 
\psline[linewidth=0.04cm](8.837452,0.20483598)(10.026298,-1.4282734) 
\psline[linewidth=0.04cm](8.821875,0.20828125)(6.021875,-1.4317187)
\psline[linewidth=0.04cm,linestyle=dashed,dash=0.16cm 0.16cm](8.821875,0.18828125)(8.821875,-1.4117187) \psframe[linewidth=0.04,dimen=outer](8.851875,-1.1217188)(8.531875,-1.4417187) 
% \usefont{T1}{ptm}{m}{n} 
\rput(9.217969,-1.6817187){$8$ cm}
%  \usefont{T1}{ptm}{m}{n} 
\rput(9.936563,-0.44171876){$10$ cm}
%  \usefont{T1}{ptm}{m}{n} 
\rput(6.2414064,0.23828125){\textbf{5.}} 
\psline[linewidth=0.04cm](11.561875,-1.3517188)(13.581875,-1.3517188) 
\psline[linewidth=0.04cm](11.564406,-1.3656596)(12.099343,0.58222204) 
\psline[linewidth=0.04cm](12.107254,0.5571398)(13.576495,-1.3405774) 
\psline[linewidth=0.04cm](11.621875,-0.49171874)(11.941875,-0.59171873) 
\psline[linewidth=0.04cm](11.681875,-0.35171875)(12.001875,-0.45171875) 
\psline[linewidth=0.04cm](12.161875,-1.1917187)(12.161875,-1.5317187) 
\psline[linewidth=0.04cm](12.161875,-1.4317187)(12.161875,-1.4317187) 
\psline[linewidth=0.04cm](12.321875,-1.1917187)(12.321875,-1.5317187) 
% \usefont{T1}{ptm}{m}{n} 
\rput(11.094063,0.23828125){\textbf{6.}} 
% \usefont{T1}{ptm}{m}{n} 
\rput(12.7517185,-1.6417187){$5$ cm}
%  \usefont{T1}{ptm}{m}{n} 
\rput(13.181562,-0.18171875){$6$ cm}
\pstriangle[linewidth=0.04,dimen=outer](2.031875,-4.491719)(3.4,2.4)
%\pstriangle[linewidth=0.04,dimen=outer](3.181875,-2.6717188)(0.0,0.0) 
%\pstriangle[linewidth=0.04,dimen=outer](3.161875,-2.7517188)(0.0,0.0) 
%\pstriangle[linewidth=0.04,dimen=outer](3.001875,-2.5317187)(0.0,0.0) 
% \usefont{T1}{ptm}{m}{n} 
\rput(0.23140626,-2.2617188){\textbf{7.}} 
\psline[linewidth=0.04cm](1.901875,-4.331719)(1.901875,-4.6517186) 
\psline[linewidth=0.04cm](1.101875,-3.1117187)(1.341875,-3.3317187)
\psline[linewidth=0.04cm](2.6844237,-3.3744361)(2.9193263,-3.1490014)
%  \usefont{T1}{ptm}{m}{n} 
\rput(3.6565626,-3.1417189){$10$ cm}
\pspolygon[linewidth=0.04](5.201875,-4.371719)(9.201875,-4.371719)(8.181875,-2.6717188)(6.181875,-2.6717188)(6.001875,-2.9717188) \psline[linewidth=0.04cm](6.161875,-2.6717188)(6.161875,-2.6717188) 
\psline[linewidth=0.04cm](6.161875,-2.6717188)(6.161875,-2.6717188)
\psline[linewidth=0.04cm,linestyle=dashed,dash=0.16cm 0.16cm](6.161875,-2.6717188)(6.161875,-4.351719) 
\psline[linewidth=0.04cm](7.106746,-2.5312762)(7.309965,-2.6871674)
\psline[linewidth=0.04cm](7.118735,-2.8573537)(7.3044972,-2.6810234)
\psline[linewidth=0.04cm](7.106746,-4.211276)(7.309965,-4.3671675) 
\psline[linewidth=0.04cm](7.118735,-4.5373535)(7.3044972,-4.3610234) 
\psframe[linewidth=0.04,dimen=outer](6.471875,-4.061719)(6.151875,-4.3817186)
%  \usefont{T1}{ptm}{m}{n}
\rput(5.3565626,-2.9817188){$15$ cm} 
% \usefont{T1}{ptm}{m}{n} 
\rput(5.5398436,-4.601719){$9$ cm} 
% \usefont{T1}{ptm}{m}{n} 
\rput(7.6565623,-2.4017189){$16$ cm} 
% \usefont{T1}{ptm}{m}{n}
\rput(7.9285936,-4.601719){$21$ cm} 
% \usefont{T1}{ptm}{m}{n} 
\rput(5.08375,-2.2617188){\textbf{8.}}
\end{pspicture}
}

\end{center}

}


\raisebox{-5 pt}{\includegraphics[width=0.5cm]{col11306.imgs/summary_www.png}}
Find the answers with the short codes:\\
\begin{tabularx}{\textwidth}{ XX }
(1)	&	(2)
\end{tabularx}
\end{exercises}


\section{Right prisms and cylinders }

A right prism is a polygon that has been stretched into a tube so that the height of the tube is
perpendicular to the base. The base and top surface is therefore the same shape and size. We
call it right prisms because the angle between the base and side form a right-angle.
\par
A square based prism has a square as its base, a rectangular based prism has a rectangle as its
base, and triangular based prism has a triangle as its base. A cylinder is another type of right
prism which has as circle as its base.
\par 

\begin{figure}[H]
    \begin{center}
	%% Rectangular Prism
	\begin{pspicture}(-2,-3.5)(3,1.5)
	    \psset{yunit=0.9,xunit=0.9}
	    \psset{Alpha=60,Beta=30}
	    \pstThreeDBox[hiddenLine](-1,1,2)(0,0,1)(2,0,0)(0,3.5,0)
	    \psset{fillcolor=lightgray,fillstyle=solid,opacity=0.5,linestyle=none,dotstyle=none}
	    \pstThreeDSquare(-1,1,2)(2,0,0)(0,3.5,0)
	    
	\end{pspicture}
\hspace{10pt}
	%% Square Prism
	\begin{pspicture}(-2,-3.5)(3,1.5)
	    \psset{yunit=0.9,xunit=0.9}
	    \psset{Alpha=60,Beta=30}
	    {\psset{fillcolor=lightgray,fillstyle=solid,opacity=0.5,linestyle=none,dotstyle=}
	    \pstThreeDSquare(-1,1,2)(2.5,0,0)(0,2.5,0)}
	    \pstThreeDBox[hiddenLine](-1,1,2)(0,0,2.5)(2.5,0,0)(0,2.5,0)
	\end{pspicture}
\hspace{10pt}
	%% Triangular Prism
	\begin{pspicture}(-2,-3.5)(3,1.5)
	    \psset{yunit=0.9,xunit=0.9}
	    \psset{Alpha=60,Beta=30}
\psSolid[object=prisme,action=draw,axe=0 0 1,base=-0.5 -0.5 0.5 -0.5 0 0.5,h=1.0]
{\psset{fillcolor=lightgray,fillstyle=solid,opacity=0.5,linestyle=solid,dotstyle=}
	    \psSolid[object=face,base=-0.5 -0.5 0.5 -0.5 0 0.5](0,0,0)}
% 	    \psset{viewpoint=8 12 8}

	\end{pspicture}
\hspace{10pt}
	%% Cylindrical Prism
	\begin{pspicture}(-2,-3.5)(3,1.5)
	    \psset{yunit=0.9,xunit=0.9}
	    \psset{Alpha=60,Beta=30}
	    \psellipse[fillcolor=white,fillstyle=solid](0,-3)(1.0,0.5)
	    \psframe[linestyle=none,fillcolor=white,fillstyle=solid](-1,-3)(1,0)
	    \psellipse[fillcolor=lightgray,opacity=0.5,fillstyle=solid,linestyle=dashed](0,-3)(1.0,0.5)
	    \psellipse[fillstyle=none](0,-0)(1.0,0.5)
	    \psline(-1.0,-3)(-1.0,-0)
	    \psline(1.0,-3)(1.0,-0)
	\end{pspicture}

	\vspace{0.75cm}
	\caption{Examples of a right prism, a right square prism, a right triangular prism and a cylinder.}
    \end{center}
\end{figure}   

\subsection*{Surface Area of prisms and cylinders}

 The term surface area refers to the total area of the exposed or outside surfaces of a prism.
This is easier to understand if you imagine the prism as a cardboard box that you can unfold.
A solid that is unfolded like this, is called a net.
\par 

 When a triangular prism is unfolded into a net, it is easier to see that it has two faces that are
triangles and three faces that are rectangles. Therefore, in order to calculate the surface area
of a prism you simply have to calculate the area of each face and add it up.
  \par
In the case of a cylinder the top and bottom faces are circles, while the curved surface flattens
into a rectangle with a length that is equal to the circumference of the circular base.

\Note{Remember that surface area is measured in square units, for example cm $^2$, m$^2$ or mm$^2$.}
\par
Below are examples of right prisms that have been unfolded to their nets:

\begin{figure}[H]
 \caption{Rectangular based prism}
   \begin{center}
	%% Rectangular Prism
    \scalebox{0.8}{
        \begin{pspicture}(-2,-3.5)(3,1.5)
	    \psset{yunit=0.9,xunit=0.9}
	    \psset{Alpha=60,Beta=30}
	    {\psset{fillcolor=lightgray,fillstyle=solid,opacity=0.5,linestyle=solid,dotstyle=}
	    \pstThreeDSquare(-1,1,2)(2,0,0)(0,3.5,0)}
	    \pstThreeDBox[hiddenLine](-1,1,2)(0,0,1)(2,0,0)(0,3.5,0)
	\end{pspicture}}

	%% Rectangular Prism Unfolded
    \scalebox{0.8}{
        \begin{pspicture}(-2,-3.5)(3,1.5)
	    \psset{yunit=0.9,xunit=0.9}
	    \psset{Alpha=60,Beta=30}
	    {\psset{fillcolor=lightgray,fillstyle=solid,opacity=0.5,linestyle=solid,dotstyle=}
	    \pstThreeDSquare(-1,1,2)(2,0,0)(0,3.5,0)}
	    \pstThreeDSquare[fillcolor=white,fillstyle=solid,opacity=0.5](-2,1,2)(1,0,0)(0,3.5,0)
	    \pstThreeDSquare[fillcolor=white,fillstyle=solid,opacity=0.5](1,1,2)(1,0,0)(0,3.5,0)
	    \pstThreeDSquare[fillcolor=white,fillstyle=solid,opacity=0.5](2,1,2)(2,0,0)(0,3.5,0)
	    \pstThreeDSquare[fillcolor=white,fillstyle=solid,opacity=0.5](-1,4.5,2)(2,0,0)(0,1,0)
	    \pstThreeDSquare[fillcolor=white,fillstyle=solid,opacity=0.5](-1,0,2)(2,0,0)(0,1,0)
	\end{pspicture}}

\vspace{10pt}

	%% Rectangular Prism Unfolded Birdseye
\scalebox{0.8}{ 
	\begin{pspicture}(-2,-3.5)(3,1.5)
	    \psset{yunit=0.9,xunit=0.9}
	    \psset{Alpha=90,Beta=90}
	    {\psset{fillcolor=lightgray,fillstyle=solid,opacity=0.5,linestyle=solid,dotstyle=}
	    \pstThreeDSquare(-1,1,2)(2,0,0)(0,3.5,0)}
	    \pstThreeDSquare[fillcolor=white,fillstyle=solid,opacity=0.5](-2,1,2)(1,0,0)(0,3.5,0)
	    \pstThreeDSquare[fillcolor=white,fillstyle=solid,opacity=0.5](1,1,2)(1,0,0)(0,3.5,0)
	    \pstThreeDSquare[fillcolor=white,fillstyle=solid,opacity=0.5](2,1,2)(2,0,0)(0,3.5,0)
	    \pstThreeDSquare[fillcolor=white,fillstyle=solid,opacity=0.5](-1,4.5,2)(2,0,0)(0,1,0)
	    \pstThreeDSquare[fillcolor=white,fillstyle=solid,opacity=0.5](-1,0,2)(2,0,0)(0,1,0)
	\end{pspicture}}
\newline
    \end{center}
\end{figure}   

A rectangular based prism as made up of six rectangles. To find the total surface area, one
would need to find the sum of the area of each face.


\begin{figure}[H]
\caption{Square based prism}
    \begin{center}

	%% Square Prism
    \scalebox{0.8}{ 
        \begin{pspicture}(-2,-3.5)(3,1.5)
            \psset{yunit=0.9,xunit=0.9}
	    \psset{Alpha=60,Beta=30}
	    {\psset{fillcolor=lightgray,fillstyle=solid,opacity=0.5,linestyle=none,dotstyle=}
	    \pstThreeDSquare(-1,1,2)(2.5,0,0)(0,2.5,0)}
	    \pstThreeDBox[hiddenLine](-1,1,2)(0,0,2.5)(2.5,0,0)(0,2.5,0)
	\end{pspicture}}

	%% Square Prism Unfolded
    \scalebox{0.8}{
	\begin{pspicture}(-2,-3.5)(3,1.5)
	    \psset{yunit=0.9,xunit=0.9}
	    \psset{Alpha=60,Beta=30}
	    {\psset{fillcolor=lightgray,fillstyle=solid,opacity=0.5,linestyle=solid,dotstyle=}
	    \pstThreeDSquare(-1,1,2)(2.5,0,0)(0,2.5,0)}
	    \pstThreeDSquare[fillcolor=white,fillstyle=solid,opacity=0.5](-3.5,1,2)(2.5,0,0)(0,2.5,0)
	    \pstThreeDSquare[fillcolor=white,fillstyle=solid,opacity=0.5](-1,3.5,2)(2.5,0,0)(0,2.5,0)
	    \pstThreeDSquare[fillcolor=white,fillstyle=solid,opacity=0.5](1.5,1,2)(2.5,0,0)(0,2.5,0)
	    \pstThreeDSquare[fillcolor=white,fillstyle=solid,opacity=0.5](-1,-1.5,2)(2.5,0,0)(0,2.5,0)
	    \pstThreeDSquare[fillcolor=white,fillstyle=solid,opacity=0.5](4,1,2)(2.5,0,0)(0,2.5,0)
	\end{pspicture}}

	%% Square Prism Unfolded Birdseye
    \scalebox{0.8}{
	\begin{pspicture}(-2,-3.5)(3,1.5)
	    \psset{yunit=0.9,xunit=0.9}
	    \psset{Alpha=0,Beta=90}
	    {\psset{fillcolor=lightgray,fillstyle=solid,opacity=0.5,linestyle=solid,dotstyle=}
	    \pstThreeDSquare(-1,1,2)(2.5,0,0)(0,2.5,0)}
	    \pstThreeDSquare[fillcolor=white,fillstyle=solid,opacity=0.5](-3.5,1,2)(2.5,0,0)(0,2.5,0)
	    \pstThreeDSquare[fillcolor=white,fillstyle=solid,opacity=0.5](-1,3.5,2)(2.5,0,0)(0,2.5,0)
	    \pstThreeDSquare[fillcolor=white,fillstyle=solid,opacity=0.5](1.5,1,2)(2.5,0,0)(0,2.5,0)
	    \pstThreeDSquare[fillcolor=white,fillstyle=solid,opacity=0.5](-1,-1.5,2)(2.5,0,0)(0,2.5,0)
	    \pstThreeDSquare[fillcolor=white,fillstyle=solid,opacity=0.5](4,1,2)(2.5,0,0)(0,2.5,0)
	\end{pspicture}}

    \end{center}
\end{figure}   

A square based prism (or a cube) can be flattened to a net with six identical squares. To find
the total surface area of the solid, find the sum of the area of the faces.


\begin{figure}[H]
    \caption{Triangular based prism}
    \begin{center}
	%% Triangular Prism
    \scalebox{0.8}{
	\begin{pspicture}(-1.5,-1.5)(1,1)
	    \psset{yunit=0.9,xunit=0.9}
% 	    \psset{Alpha=60,Beta=30}
% 	    \psset{viewpoint=0 0 200}
	    \psSolid[object=face,fillcolor=lightgray,opacity=0.5,base=-0.5 -0.5 0.5 -0.5 0 0.5](0,0,0)
	    \psSolid[object=prisme,action=draw,axe=0 0 1,base=-0.5 -0.5 0.5 -0.5 0 0.5,h=0.4]
	\end{pspicture}}

	%% Triangular Prism Unfolded
    \scalebox{0.8}{
	\begin{pspicture}(-2,-3.5)(3,1.5)
	    \psset{yunit=0.9,xunit=0.9}
% 	    \psset{Alpha=60,Beta=30}
% 	    \psset{viewpoint=8 12 8}
	    \psSolid[object=face,fillcolor=lightgray,opacity=0.5,base=-0.5 -0.5 0.5 -0.5 0 0.5](0,0,0)
	    \psSolid[object=face,fillcolor=lightgray,incolor=white,base=-0.5 -0.5 0.5 -0.5 0.5 -0.9 -0.5 -0.9](0,0,0)
	    \psSolid[object=face,fillcolor=lightgray,incolor=white,base=-0.5 -0.9 0.5 -0.9 0 -1.9](0,0,0)
	    \psSolid[object=face,fillcolor=lightgray,incolor=white,base=-0.5 -0.5 -0.8464102 -0.3 -0.3464102 0.7 0.0 0.5](0,0,0)
	    \psSolid[object=face,fillcolor=lightgray,incolor=white,base=0.0 0.5 0.3464102 0.7 0.8464102 -0.3 0.5 -0.5](0,0,0)
	\end{pspicture}}

	%% Triangular Prism Unfolded Birdseye
    \scalebox{0.8}{
	\begin{pspicture}(-2,-3.5)(3,1.5)
	    \psset{yunit=0.9,xunit=0.9}
% 	    \psset{Alpha=60,Beta=30}
	    \psset{viewpoint=0 0 20,RotZ=0}
	    \psSolid[object=face,fillcolor=lightgray,opacity=0.5,base=-0.5 -0.5 0.5 -0.5 0 0.5](0,0,0)
	    \psSolid[object=face,fillcolor=lightgray,incolor=white,base=-0.5 -0.5 0.5 -0.5 0.5 -0.9 -0.5 -0.9](0,0,0)
	    \psSolid[object=face,fillcolor=lightgray,incolor=white,base=-0.5 -0.9 0.5 -0.9 0 -1.9](0,0,0)
	    \psSolid[object=face,fillcolor=lightgray,incolor=white,base=-0.5 -0.5 -0.8464102 -0.3 -0.3464102 0.7 0.0 0.5](0,0,0)
	    \psSolid[object=face,fillcolor=lightgray,incolor=white,base=0.0 0.5 0.3464102 0.7 0.8464102 -0.3 0.5 -0.5](0,0,0)
	\end{pspicture}}


    \end{center}
\end{figure} 

The triangular based prism has two triangles and one rectangle which is made up of three
smaller rectangles. The length of the rectangle is equal to the perimeter of the triangle.


\begin{figure}[H]
\caption{Cylinder}
\begin{center}

	%% Cylindrical Prism
	\begin{pspicture}(-2,-3.5)(3,1.5)
	    \psset{yunit=0.9,xunit=0.9}
	    \psellipse[fillcolor=white,fillstyle=solid](0,-3)(1.0,0.5)
	    \psframe[linestyle=none,fillcolor=white,fillstyle=solid](-1,-3)(1,0)
	    \psellipse[fillcolor=lightgray,opacity=0.5,fillstyle=solid,linestyle=dashed](0,-3)(1.0,0.5)
	    \psellipse[fillstyle=none](0,-0)(1.0,0.5)
	    \psline(-1.0,-3)(-1.0,-0)
	    \psline(1.0,-3)(1.0,-0)
	\end{pspicture}
\hspace{20pt}
	%% Cylindrical Prism Unfolded
	\begin{pspicture}(-2,-3.5)(3,1.5)
	    \psset{yunit=0.9,xunit=0.9}
	    \psellipse[fillcolor=lightgray,opacity=0.5,fillstyle=solid,linestyle=solid](4,-4)(1.0,1.0)
	    \psellipse[fillstyle=none](0,1)(1.0,1)
	    \psframe[linestyle=solid,fillcolor=white,fillstyle=solid](-1,-3)(5,0)
	\end{pspicture}

    \end{center}
\end{figure}   

The net of a cylinder is made up of two identical circles and a rectangle with a length equal to
the circumference of the circles.


\begin{wex}
{Finding the surface area of a rectangular prism

}
{%Problem
Find the total surface area of the prism below
\begin{center}
\scalebox{0.9}{
    \begin{pspicture}(-2,-3.5)(3,1.5)
        \psset{yunit=0.9,xunit=0.9}
        \psset{Alpha=30,Beta=15}
        \pstThreeDBox(-1,1,2)(0,0,2)(10,0,0)(0,5,0)
        \pstThreeDPut(8,6,3){$2$ cm}
        \pstThreeDPut(10,3.5,2){$5$ cm}
        \pstThreeDPut(5,7,2){$10$ cm}
   \end{pspicture}}
\end{center}

}
{%Solution

\westep{Draw a rough sketch of the net of the solid, filling in the relevant values}

\begin{center}
	%% Rectangular Prism Unfolded Birdseye
\scalebox{1}{ 
	\begin{pspicture}(-2,-3.5)(3,1.5)
	    \psset{yunit=0.9,xunit=0.9}
	    \psset{Alpha=180,Beta=90}
	    {\psset{fillcolor=white,fillstyle=solid,opacity=0,linestyle=solid,dotstyle=}
	    \pstThreeDSquare(-1,1,2)(2,0,0)(0,3.5,0)}
	    \pstThreeDSquare[fillcolor=white,fillstyle=solid,opacity=0](-2,1,2)(1,0,0)(0,3.5,0)
	    \pstThreeDSquare[fillcolor=white,fillstyle=solid,opacity=0](1,1,2)(1,0,0)(0,3.5,0)
	    \pstThreeDSquare[fillcolor=white,fillstyle=solid,opacity=0](2,1,2)(2,0,0)(0,3.5,0)
	    \pstThreeDSquare[fillcolor=white,fillstyle=solid,opacity=0](-1,4.5,2)(2,0,0)(0,1,0)
	    \pstThreeDSquare[fillcolor=white,fillstyle=solid,opacity=0](-1,0,2)(2,0,0)(0,1,0)
            \rput(0,6.){$5$ cm}
            \rput(1.5, 5.){$2$ cm}
            \rput(4.5, 3.){$10$ cm}
	\end{pspicture}}
\end{center}

\westep{Find the area of the different shapes}
\begin{align*} 
\mbox{large rectangle} &= \mbox{perimeter of small rectangle} \times \mbox{length} \\
                        &= (5 +2 + 5 + 2) \times 10 \\
                        &= 14 \times 10 \\
                        &= 140~\mbox{cm}^2  \\ \\
2 \times \mbox{small rectangle} &= 2(5+2) \\
                                &= 2(10) \\
                                &= 20~\mbox{cm}^2
\end{align*}



\westep{Find the sum of the the area of the faces}

$\mbox{large rectangle} + 2 \times \mbox{small rectangle} = 140 + 20 = 180~$cm$^2$

}
\end{wex}


\begin{wex}
{Finding the surface area of a triangular prism

}
{
Find the total surface area of the prism below:\\
\begin{center}
\scalebox{1}{
	%% Triangular Prism
\begin{pspicture}(0,-1.59875)(4.7,1.61875)
\pstriangle[linewidth=0.04,dimen=outer](1.09,-1.59875)(2.18,1.72)
\psline[linewidth=1pt](1.08,0.1012499)(3.38,1.59875)
\psline[linewidth=1pt,linestyle=dotted,dotsep=0.10583334cm](0.06,-1.55875)(2.76,0.15875)
\psline[linewidth=1pt](3.38,1.57875)(4.64,0.2012498)
\psline[linewidth=1pt](2.14,-1.57875)(4.66,0.20125)
\psline[linewidth=1pt,linestyle=dashed,dash=0.16cm 0.16cm](1.08,0.06125)(1.06,-1.57875)
\psline[linewidth=1pt,linestyle=dotted,dotsep=0.10583334cm](3.38,1.54125)(2.74,0.17875)
\psline[linewidth=1pt,linestyle=dotted,dotsep=0.10583334cm](4.68,0.19875)(2.78,0.17875)
\psline[linewidth=0.03](1.089162,-1.3)(1.309162,-1.3)(1.309162,-1.5)(1.309162,-1.6)
\rput(1.07, -1.9){$8$ cm}
\rput(4,-1){$12$ cm}
\rput(1.5,-1.1){$3$ cm}
\end{pspicture} 
}
\end{center}


}
{%Solution

\westep{Draw a rough sketch of the net of the solid, filling in the relevant values}

% LaTeX Draw
% \usepackage{pst-plot} % For axes
\begin{center}
\scalebox{1} % Change this value to rescale the drawing.
{
\begin{pspicture}(0,-3.538172)(7.7584376,3.5381718)
\psframe[linewidth=0.02,dimen=outer](6.3,2.2243125)(0.0,-2.2243125)
\pstriangle[linewidth=0.02,dimen=outer](3.15,2.2072406)(2.87,1.3345875)
\rput{-180.0}(6.3,-5.7490687){\pstriangle[linewidth=0.02,dimen=outer](3.15,-3.541828)(2.87,1.3345875)}
% \usefont{T1}{ptm}{m}{n}
\rput(3.541875,2.6831717){$3$ cm}
\psline[linewidth=0.02cm,linestyle=dashed,dash=0.16cm 0.16cm](3.14,3.5181718)(3.14,2.2181718)
% \usefont{T1}{ptm}{m}{n}
\rput(7,0.083171844){$12$ cm}
% \usefont{T1}{ptm}{m}{n}
\rput(3.3982813,1.7031718){$8$ cm}
\psline[linewidth=0.02cm,linestyle=dashed,dash=0.16cm 0.16cm](1.74,2.2181718)(1.76,-2.2218282)
\psline[linewidth=0.02cm,linestyle=dashed,dash=0.16cm 0.16cm](4.52,2.2381718)(4.54,-2.2018282)
\end{pspicture} 
}
\end{center}
\westep{Find the area of the different shapes}

large rectangle = perimeter of triangle $\times$ length

To find the perimeter of the triangle, we have to first find the length of the sides using Pythagoras:

\par
\begin{center}
\scalebox{1} % Change this value to rescale the drawing.
{
\begin{pspicture}(0,-0.97432816)(2.8562837,0.9306719)
\pstriangle[linewidth=0.02,dimen=outer](1.435,-0.40025938)(2.87,1.3345875)
\rput(1.826875,0.075671844){$3$ cm}
\psline[linewidth=0.02cm,linestyle=dashed,dash=0.16cm 0.16cm](1.425,0.91067183)(1.425,-0.38932815)
\rput(0.71140623,-0.6){$4$ cm}
\rput(2.0314062,-0.6){$4$ cm}
\end{pspicture} 
}
\end{center}


\begin{align*}
x^2 &= 3^2 + 4^2 \\
x &= \sqrt{25} \\
x &= 5
\end{align*}

$\therefore \mbox{perimeter of triangle} = 5 + 5 + 8 $

large rectange = perimeter of triangel + length

\begin{align*}
\phantom{stuffs} &= (5+5+8) \times 12 \\
                 &= 18 \times 12 \\
                 &= 216 \mbox{ cm}^2 \\
2 \times \mbox{triangle} &= 2(\dfrac{1}{2} \times 8 \times 3) \\
                    &= 2(12) \\
                    &= 24\mbox{ cm}^2\\
\end{align*}


\westep{Find the sum of the areas of the faces}

large rectangle + $2\times$ triangle $ = 216 + 24 = 240\mbox{ cm}^2$

}
\end{wex}



\begin{wex}
{Finding the area of a cylindrical prism
}
{% Question
Find the total surface are of the prism below to one decimal place.\\
\begin{center}
        \begin{pspicture}(-2,-3.5)(3,1.5)
	    \psset{yunit=0.9,xunit=0.9}
	    \psellipse(0,-3)(1.0,0.5)
	    \psframe[linestyle=none,](-1,-3)(1,0)
	    \psellipse[linestyle=dashed](0,-3)(1.0,0.5)
	    \psellipse[](0,-0)(1.0,0.5)
	    \psline(-1.0,-3)(-1.0,-0)
	    \psline(1.0,-3)(1.0,-0)
            \psline(0,0)(1.0,0)
            \rput(1.5,-1.5){$30$ cm}
            \rput(0.3,0.2){$10$ cm}
	\end{pspicture}
\end{center}



}
{% solution

\westep{Draw a rough sketch of the net of the solid, filling in the relevant values:}
\begin{center}
	%% Cylindrical Prism Unfolded
	\begin{pspicture}(-2,-3.5)(3,1.5)
	    \psset{yunit=0.9,xunit=0.9}
	    \psellipse[linestyle=solid](4,-4)(1.0,1.0)
	    \psellipse(0,1)(1.0,1)
	    \psframe[linestyle=solid](-1,-3)(5,0)
            \rput(0.2,1.2){$10$ cm}
            \rput(5.7,-1.5){$30$ cm}
\psline(0,1)(1,1)
	\end{pspicture}
\end{center}



\westep{Find the area of the different shapes}

\begin{align*}
\mbox{large rectangle} &= \mbox{circumference of circle} \times \mbox{length} \\
                        &= 2\pi(10) \times 30 \\
                        &= 1884,96 \mbox{ cm}^2 \\
       2\times \mbox{circle} &= 2(\pi(10^2)) \\
                             &= 628,32\mbox{ cm}^2
\end{align*}

\westep{Find the sum of the area of the faces}
large rectangle + $2\times$ circle = $1884,96 + 628,32$ = $2513,3\mbox{ cm}^2$


}
\end{wex}






\begin{exercises}{ }

Calculate the surface area in each of the following:

\setcounter{subfigure}{0}
\begin{figure}[H]
\begin{center}
\begin{pspicture}(0,-4.355625)(11.259218,4.355625)
\psline[linewidth=0.04cm](0.35640624,1.4571873)(1.3564062,2.4371874)
\psline[linewidth=0.04cm](3.3564062,1.4771873)(4.336406,2.4371874)
\psline[linewidth=0.04cm](0.37640625,1.4571873)(3.3564062,1.4771873)
\psline[linewidth=0.04cm](1.3564062,2.4571874)(4.336406,2.4571874)
\psline[linewidth=0.04cm](4.356406,3.4971871)(4.336406,2.3971872)
\psline[linewidth=0.04cm](1.3564062,3.4771872)(1.3764062,2.4571874)
\psline[linewidth=0.04cm](1.3764062,3.4571872)(4.356406,3.4771872)
\psline[linewidth=0.04cm](3.3564062,2.5371873)(3.3564062,1.4771873)
\psline[linewidth=0.04cm](0.35640624,2.4971874)(0.35640624,1.4371873)
\psline[linewidth=0.04cm](0.35640624,2.5171874)(3.3364062,2.5171874)
\psline[linewidth=0.04cm](0.37640625,2.5171874)(1.3564062,3.4571872)
\psline[linewidth=0.04cm](3.3564062,2.5171876)(4.336406,3.4571872)
\psellipse[linewidth=0.04,dimen=outer](8.986406,1.8371873)(0.99,0.38)
\psellipse[linewidth=0.04,dimen=outer](8.986406,3.1771872)(0.99,0.38)
\psline[linewidth=0.04cm](8.016406,3.1371875)(8.016406,1.8771874)
\psline[linewidth=0.04cm](9.956407,3.1771872)(9.956407,1.8571873)
\psline[linewidth=0.04cm,linestyle=dashed,dash=0.16cm 0.16cm](9.016406,1.8171873)(9.936406,1.8371873)
% \usefont{T1}{ptm}{m}{n}
\rput(9.005782,1.9671873){$5$ cm}
% \usefont{T1}{ptm}{m}{n}
\rput(10.454218,2.5871873){$10$ cm}
\pstriangle[linewidth=0.04,dimen=outer](2.1664062,-3.9028125)(2.18,1.72)
\psline[linewidth=0.04cm](2.1564062,-2.2028127)(4.516406,-0.7828127)
\psline[linewidth=0.04cm](1.1364063,-3.8628125)(4.3164062,-1.7828125)
\psline[linewidth=0.04cm](4.516406,-0.7628127)(5.7364063,-2.1028128)
\psline[linewidth=0.04cm](3.2164063,-3.8828125)(5.7364063,-2.1028125)
\psline[linewidth=0.04cm,linestyle=dashed,dash=0.16cm 0.16cm](2.1564062,-2.2428124)(2.1364062,-3.8828125)
% \usefont{T1}{ptm}{m}{n}
\rput(1.4389063,-1.2678125){\textbf{3.}}
% \usefont{T1}{ptm}{m}{n}
\rput(4.8967185,-3.1528125){$20$ cm}
% \usefont{T1}{ptm}{m}{n}
\rput(2.5,-3.4){$5$ cm}
% \usefont{T1}{ptm}{m}{n}
\rput(2.2596877,-4.1528125){$10$ cm}
% \usefont{T1}{ptm}{m}{n}
\rput(0.96203125,4.1521873){\textbf{1.}}
% \usefont{T1}{ptm}{m}{n}
\rput(7.763125,4.0121875){\textbf{2.}}
% \usefont{T1}{ptm}{m}{n}
\rput(1.7796875,1.2271875){$6$ cm}
% \usefont{T1}{ptm}{m}{n}
\rput(4.3628125,1.8271875){$7$ cm}
% \usefont{T1}{ptm}{m}{n}
\rput(4.8039064,2.9271874){$10$ cm}
\psline[linewidth=0.04cm](4.516406,-0.7628125)(4.3164062,-1.8028125)
\psline[linewidth=0.04cm](5.7364063,-2.1028125)(4.2764063,-1.8028125)
\end{pspicture}     
\end{center}
\end{figure}   

If a litre of paint covers an area of $2$ m$^{2}$, how much paint does a painter need to cover:
\begin{enumerate}[noitemsep, label=\textbf{\arabic*}. ] 
\setcounter{enumi}{3}
\item A rectangular swimming pool with dimensions $4$ m $\times~3$ m $\times~2,5$ m, inside walls and floor only.
\item The inside walls and floor
of a circular reservoir with diameter $4$ m and height $2,5$ m 
\end{enumerate}

\setcounter{subfigure}{0}


\begin{figure}[H] % horizontal\label{m39357*id62926}
\begin{center}
\begin{pspicture}(0,-1.6)(3.5434375,1.6) 
\psellipse[linewidth=0.04,dimen=outer](1.29,-1.09)(1.27,0.51) 
\psellipse[linewidth=0.04,dimen=outer](1.27,1.09)(1.27,0.51) 
\psline[linewidth=0.04cm](0.04,-1.1)(0.02,1.14) 
\psline[linewidth=0.04cm](2.54,1.12)(2.56,-1.1) 
\psline[linewidth=0.04cm,linestyle=dashed,dash=0.16cm 0.16cm](0.06, -1.1)(2.52,-1.1) 
\rput(1.2925,-0.95){$4$ m} 
\rput(3.0567188,0.25){$2,5$ m} 
\end{pspicture} 
\end{center}

\end{figure}   

\addtocounter{footnote}{-0}

        
}
\end{exercises}        
\subsection*{Volume of prisms and cylinders}

The volume of a right prism is very easy to calculate. Once you have found the area of the
base of the solid, multiply it by the height. Remember that the volume is measured in cubic
units, for example cm$^3$, m$^3$ or mm$^3$.


\begin{table*}[h]
\newcolumntype{C}{>{\centering\arraybackslash} m{0.75in} }
\newcolumntype{D}{>{\centering\arraybackslash} m{2.25in} }
\begin{tabular}{|C|D|D|}
\hline
\textbf{Rectangular prism}
&
\begin{center}
\begin{pspicture}(0,-1.2412498)(4.584375,1.2212498)
\psline[linewidth=0.04cm](0.0,-0.83875)(1.0,0.14125004)
\psline[linewidth=0.04cm](3.0,-0.81875)(3.98,0.14125004)
\psline[linewidth=0.04cm](0.02,-0.83875)(3.0,-0.81875)
\psline[linewidth=0.04cm](1.0,0.16124995)(3.98,0.16124995)
\psline[linewidth=0.04cm](4.0,1.2012498)(3.98,0.10124995)
\psline[linewidth=0.04cm](1.0,1.1812499)(1.02,0.16125014)
\psline[linewidth=0.04cm](1.02,1.1612499)(4.0,1.1812499)
\psline[linewidth=0.04cm](3.0,0.24124995)(3.0,-0.81875)
\psline[linewidth=0.04cm](0.0,0.20125005)(0.0,-0.85875)
\psline[linewidth=0.04cm](0.0,0.22125006)(2.98,0.22125006)
\psline[linewidth=0.04cm](0.02,0.22125006)(1.0,1.1612499)
\psline[linewidth=0.04cm](3.0,0.22125015)(3.98,1.1612499)
% \usefont{T1}{ptm}{m}{n}
\rput(1.2896875,-1.0687499){$l$}
% \usefont{T1}{ptm}{m}{n}
\rput(3.9228127,-0.46874985){$b$}
% \usefont{T1}{ptm}{m}{n}
\rput(4.273906,0.63125014){$h$}
\end{pspicture}
\end{center} 
&
$
\begin{aligned}
\mbox{Volume} &= \mbox{area of base} \times \mbox{height} \\
                &= \mbox{area of rectangle} \times \mbox{height} \\
                &= l \times b \times h \\
\end{aligned}$   \\ \hline


\textbf{Triangular prism} &

\scalebox{1} % Change this value to rescale the drawing.
{
\begin{pspicture}(0,-1.8337499)(4.9209375,1.8137499)
\pstriangle[linewidth=0.04,dimen=outer](1.3309375,-1.3462498)(2.18,1.72)
\psline[linewidth=0.04cm](1.3209375,0.35375)(3.6809375,1.77375)
\psline[linewidth=0.04cm](3.6809375,1.7937499)(4.9009376,0.4537499)
\psline[linewidth=0.04cm](2.3809376,-1.32625)(4.9009376,0.4537501)
\psline[linewidth=0.04cm,linestyle=dashed,dash=0.16cm 0.16cm](1.3209375,0.3137501)(1.3009375,-1.32625)
\psframe[linewidth=0.04,dimen=outer](1.4809375,-1.1462499)(1.2809376,-1.3462499)
\psline[linewidth=0.04cm](0.6209375,-0.3462499)(1.0009375,-0.5462499)
\psline[linewidth=0.04cm](0.5609375,-0.4462499)(0.9409375,-0.6462499)
\psline[linewidth=0.04cm](2.0409374,-0.3462499)(1.6609375,-0.5462499)
\psline[linewidth=0.04cm](2.1009376,-0.4462499)(1.7209375,-0.6462499)
% \usefont{T1}{ptm}{m}{n}
\rput(4.1923437,-0.5212499){$H$}
% \usefont{T1}{ptm}{m}{n}
\rput(1.2723438,-1.6612499){$b$}
% \usefont{T1}{ptm}{m}{n}
\rput(0.21234375,-0.2412499){$s$}
% \usefont{T1}{ptm}{m}{n}
\rput(1.4723438,-0.7612499){$h$}
\end{pspicture} 
}

&

$\begin{aligned}
\mbox{Volume} &= \mbox{area of base} \times \mbox{height} \\
                &= \mbox{area of triangle} \times \mbox{height} \\
                &= \dfrac{1}{2}b\times h \times H \\
\end{aligned}$  \\ \hline

\textbf{Cylinder} &

\scalebox{1} % Change this value to rescale the drawing.
{
\begin{pspicture}(0,-1.05)(2.761875,1.05)
\psellipse[linewidth=0.04,dimen=outer](0.99,-0.66999996)(0.99,0.38)
\psellipse[linewidth=0.04,dimen=outer](0.99,0.66999996)(0.99,0.38)
\psline[linewidth=0.04cm](0.02,0.63000023)(0.02,-0.6299999)
\psline[linewidth=0.04cm](1.96,0.66999996)(1.96,-0.65)
\psline[linewidth=0.04cm,linestyle=dashed,dash=0.16cm 0.16cm](1.02,-0.68999994)(1.94,-0.66999996)
\usefont{T1}{ptm}{m}{n}
\rput(2.4514062,0.09500025){$h$}
\usefont{T1}{ptm}{m}{n}
\rput(1.1014062,-0.50499976){$r$}
\end{pspicture} 
}

&

$\begin{aligned}
\mbox{Volume} &= \mbox{area of base} \times \mbox{height} \\
                &= \mbox{area of circle} \times \mbox{height} \\
                &= \pi r^2 \times h \\
\end{aligned}$  \\ \hline



\end{tabular}
\end{table*}






\begin{wex}
{% Title
Finding the volume of a cube
}
{% question
Find the volume of the following solid:
\begin{center}
\scalebox{1} % Change this value to rescale the drawing.
{
\begin{pspicture}(0,-1.0258813)(5.2842736,1.3541187)
\psdiamond[linewidth=0.04,dimen=outer,gangle=-49.7](1.6365356,0.0)(1.27,1.0687643)
\psdiamond[linewidth=0.04,dimen=outer,gangle=50.0](3.2365355,0.0)(1.27,1.0687643)
\psline[linewidth=0.04](0.8223987,0.96411866)(2.5223987,1.2441187)(4.0023985,1.0003091)
\psline[linewidth=0.04cm](0.6423987,0.14411865)(1.0223987,0.14411865)
\psline[linewidth=0.04cm](2.2423987,-0.115881346)(2.6223986,-0.115881346)
\psline[linewidth=0.04cm](3.8423986,0.044118654)(4.2223988,0.044118654)
\psline[linewidth=0.04cm](1.7123986,0.95411867)(1.7123986,0.5741187)
\psline[linewidth=0.04cm](1.5723987,-0.60588133)(1.5723987,-0.9858813)
\psline[linewidth=0.04cm](3.2723987,-0.6258814)(3.2723987,-1.0058813)
\psline[linewidth=0.04cm](3.1723988,1.0141187)(3.1723988,0.6341187)
\psline[linewidth=0.04cm](1.8723987,1.3341186)(1.8723987,0.95411867)
\psline[linewidth=0.04cm](3.0923986,1.3341186)(3.0923986,0.95411867)
% \usefont{T1}{ptm}{m}{n}
\rput(4.753805,0.109118655){$3$ cm}
\end{pspicture} 
}
\end{center}
}
{%solution

\westep{Find the area of the base}
\begin{align*}
\mbox{area of square} &= 3 \times 3 \\
                    &= 9~\mbox{ cm}^2
\end{align*}

\westep{Multiply this by the height of the solid in order to find the volume}
\begin{align*}
\mbox{volume} &= \mbox{area of base} \times \mbox{height}\\
                    &= 9 \times 3 \\
                    &= 27\mbox{ cm}^3 
\end{align*}

}
\end{wex}




\begin{wex}
{Finding the volume of a triangular prism
}

{%problem
Find the volume of the following solid:\\

\begin{center}
\scalebox{1} % Change this value to rescale the drawing.
{
\begin{pspicture}(0,-1.8489062)(4.7628126,1.8289062)
\pstriangle[linewidth=0.04,dimen=outer](1.09,-1.3310935)(2.18,1.72)
\psline[linewidth=0.04cm](1.08,0.36890626)(3.44,1.7889062)
\psline[linewidth=0.04cm](3.44,1.8089062)(4.6600003,0.46890616)
\psline[linewidth=0.04cm](2.14,-1.3110938)(4.6600003,0.46890634)
\psline[linewidth=0.04cm,linestyle=dashed,dash=0.16cm 0.16cm](1.08,0.32890636)(1.06,-1.3110938)
\psframe[linewidth=0.04,dimen=outer](1.3077112,-1.0710938)(1.0400001,-1.3310937)
\psline[linewidth=0.04cm](0.38,-0.33109364)(0.76,-0.53109366)
\psline[linewidth=0.04cm](0.32,-0.43109366)(0.7,-0.6310936)
\psline[linewidth=0.04cm](1.8,-0.33109364)(1.42,-0.53109366)
\psline[linewidth=0.04cm](1.8600001,-0.43109366)(1.48,-0.6310936)
\usefont{T1}{ptm}{m}{n}
\rput(3.9578123,-0.5060936){$20$ cm}
\usefont{T1}{ptm}{m}{n}
\rput(1.2078125,-1.6460936){$8$ cm}
\usefont{T1}{ptm}{m}{n}
\rput(1.475,-0.78484374){$10$ cm}
\end{pspicture} 
}

\end{center}
}
{%solution
\westep{Find the area of the base}
\begin{align*}
\mbox{area of triangle} &= \dfrac{1}{2} \times 8 \times 10\\
                        &= 40\mbox{ cm}^2
\end{align*}


\westep{Multiply this by the height of the solid to find the volume}
\begin{align*}
\mbox{volume} &= \mbox{area of base} \times \mbox{height}\\
                        &= 40\mbox{ cm} \times 20 \\
                        &= 800\mbox{ cm}^3
\end{align*}


}
\end{wex}

\begin{wex}{Finding the volume of a cylindrical prism}
 {Find the volume of the following solid (correct to one decimal place):\\
\begin{center}
        \begin{pspicture}(-2,-3.5)(3,1.5)
	    \psset{yunit=0.9,xunit=0.9}
	    \psellipse(0,-3)(1.0,0.5)
	    \psframe[linestyle=none,](-1,-3)(1,0)
	    \psellipse[linestyle=dashed](0,-3)(1.0,0.5)
	    \psellipse[](0,-0)(1.0,0.5)
	    \psline(-1.0,-3)(-1.0,-0)
	    \psline(1.0,-3)(1.0,-0)
            \psline(0,0)(1.0,0)
            \rput(1.5,-1.5){$15$ cm}
            \rput(0.3,0.2){$4$ cm}
	\end{pspicture}
\end{center}
}
{
\westep{Find the area of the base}


\begin{align*}
\mbox{area of circle} &= \pi~r^2\\
&= \pi(4)^{2} \\
&= 50,265\mbox{ cm}^{2}\\

\end{align*}

\westep{Multiply this by the height of the solid to find the volume}
\begin{align*}
\mbox{volume} &= \mbox{area of base } \times \mbox{ height}\\
&= 50,265\times 15\\
&= 754,0\mbox{ cm}^{3}\\

\end{align*}
}
\end{wex}

\begin{exercises}{}
Calculate the following volumes (correct to one decimal place):
\begin{center}
 \begin{pspicture}(0,-4.03375)(11.525,4.01375)
\psline[linewidth=0.04cm](0.385,2.65125)(0.385,1.77125)
\psline[linewidth=0.04cm](0.365,2.65125)(2.825,2.65125)
\psline[linewidth=0.04cm](2.845,1.75125)(3.585,2.21125)
\psline[linewidth=0.04cm](0.385,1.77125)(1.125,2.23125)
\psline[linewidth=0.04cm](0.425,2.65125)(1.165,3.11125)
\psline[linewidth=0.04cm](2.825,2.63125)(3.565,3.09125)
% \usefont{T1}{ptm}{m}{n}
\rput(1.7434374,1.42125){$6$ cm}
\psline[linewidth=0.04cm](1.105,2.23125)(3.565,2.23125)
\psline[linewidth=0.04cm](0.365,1.77125)(2.825,1.77125)
\psline[linewidth=0.04cm](1.145,3.11125)(3.605,3.11125)
\psline[linewidth=0.04cm](2.845,2.63125)(2.845,1.75125)
\psline[linewidth=0.04cm](1.145,3.09125)(1.145,2.21125)
\psline[linewidth=0.04cm](3.585,3.11125)(3.585,2.23125)
% \usefont{T1}{ptm}{m}{n}
\rput(4.2189064,2.64125){$10$ cm}
% \usefont{T1}{ptm}{m}{n}
\rput(3.6479688,1.76375){$7$ cm}
% \usefont{T1}{ptm}{m}{n}
\rput(8.083593,1.04125){$5$ cm}
% \usefont{T1}{ptm}{m}{n}
\rput(8.329375,0.46125){$10$ cm}
\psline[linewidth=0.04cm](7.925,2.03375)(10.565,3.99375)
\psline[linewidth=0.04cm](7.085,0.8137501)(7.925,2.05375)
\psline[linewidth=0.04cm](7.065,0.8137501)(8.865,0.8137501)
\psline[linewidth=0.04cm](8.885,0.7737501)(7.905,2.07375)
\psline[linewidth=0.04cm](9.705,2.7137501)(10.545,3.95375)
\psline[linewidth=0.04cm](9.685,2.7137501)(11.485,2.7137501)
\psline[linewidth=0.04cm](11.505,2.6937501)(10.525,3.99375)
\psline[linewidth=0.04cm](7.125,0.8137501)(9.765,2.77375)
\psline[linewidth=0.04cm](8.845,0.7937501)(11.485,2.75375)
\psline[linewidth=0.04cm,linestyle=dashed,dash=0.16cm 0.16cm](7.925,2.01375)(7.945,0.7937501)
% \usefont{T1}{ptm}{m}{n}
\rput(10.873125,1.4837501){$20$ cm }
\psline[linewidth=0.04cm](1.485,-1.20875)(1.485,-3.44875)
\psline[linewidth=0.04cm](3.345,-1.20875)(3.345,-3.44875)
\psellipse[linewidth=0.04,dimen=outer](2.415,-1.22875)(0.95,0.26)
\psellipse[linewidth=0.04,dimen=outer](2.415,-3.48875)(0.95,0.26)
\psdots[dotsize=0.12](2.445,-3.48875)
\psline[linewidth=0.03cm](2.445,-3.48675)(3.325,-3.48675)
% \usefont{T1}{ptm}{m}{n}
\rput(3.148281,-3.87875){$5$ cm}
% \usefont{T1}{ptm}{m}{n}
\rput(3.9589062,-2.43625){$10$ cm}
% \usefont{T1}{ptm}{m}{n}
\rput(0.16140625,3.76875){\textbf{1.}}
% \usefont{T1}{ptm}{m}{n}
\rput(7.1796875,3.74875){\textbf{2.}}
% \usefont{T1}{ptm}{m}{n}
\rput(0.14171875,-0.85125){\textbf{3.}}
\end{pspicture} 
\end{center}
\end{exercises}

\subsection*{Right pyramids, right cones and spheres}
A pyramid is a geometric solid that has a polygon base and the base is joined to a point,
called the apex. The triangular based pyramid and square based pyramid take their names
from the shape of their base. Cones are similar to pyramids except that their bases are circles
instead of polygons. Spheres are solids that are perfectly round and look the same from any
direction.

\begin{figure}[ht]
\begin{center}
% \usepackage[usenames,dvipsnames]{pstricks}
% \usepackage{epsfig}
% \usepackage{pst-grad} % For gradients
% \usepackage{pst-plot} % For axes
\scalebox{1.2} % Change this value to rescale the drawing.
{
\begin{pspicture}(0,-0.9969444)(6.534111,0.99705553)
\pspolygon[linewidth=0.028222222,fillstyle=solid](2.621611,-0.26305556)(3.099611,0.46944445)(4.068611,-0.47205555)
\pspolygon[linewidth=0.028222222,fillstyle=solid](2.6316354,-0.66294444)(3.06,0.41705555)(2.54,-0.26294443)
\pspolygon[linewidth=0.028222222,fillstyle=solid](0.03411105,-0.32955554)(0.869611,0.48944443)(1.361111,-0.24505556)
\pspolygon[linewidth=0.028222222,fillstyle=solid](1.705111,-0.6149472)(0.869611,0.48944443)(0.37811106,-0.70294446)
\pspolygon[linewidth=0.028222222,fillstyle=solid](0.38,-0.70294446)(0.84,0.45705557)(0.0,-0.34294444)
\pspolygon[linewidth=0.028222222,fillstyle=solid](4.1,-0.49978656)(3.1,0.47705555)(2.62,-0.6829444)
\psline[linewidth=0.0139999995,arrowsize=0.05291667cm 2.0,arrowlength=1.4,arrowinset=0.4]{<->}(0.9,0.037055556)(0.9,-0.42294446)(1.1,-0.5629445)
\psline[linewidth=0.014111111cm,arrowsize=0.05291667cm 2.0,arrowlength=1.4,arrowinset=0.4]{->}(0.9,-0.42294446)(0.58,-0.48294446)
\psline[linewidth=0.022cm,linestyle=dashed,dash=0.1cm 0.1cm](0.0,-0.34294444)(1.08,-0.26294443)
\psline[linewidth=0.022cm,linestyle=dashed,dash=0.1cm 0.1cm](2.54,-0.26294443)(4.08,-0.48294446)
\psline[linewidth=0.022,linestyle=dashed,dash=0.1cm 0.1cm](1.72,-0.6229445)(1.08,-0.26294443)(0.86,0.45705557)
\psline[linewidth=0.0139999995,arrowsize=0.05291667cm 2.0,arrowlength=1.4,arrowinset=0.4]{<->}(3.1,0.07705556)(3.1,-0.38294443)(3.34,-0.50294447)
\psline[linewidth=0.014111111cm,arrowsize=0.05291667cm 2.0,arrowlength=1.4,arrowinset=0.4]{->}(3.12,-0.38294443)(2.82,-0.48294446)
\psline[linewidth=0.028222222](5.12,-0.48294446)(5.7990556,0.9829444)(6.52,-0.5170556)
\psbezier[linewidth=0.027999999](5.1394444,-0.44294444)(5.04,-0.66294444)(5.34,-0.94294447)(5.7,-0.96294445)(6.06,-0.9829444)(6.52,-0.82294446)(6.52,-0.50294447)
\psbezier[linewidth=0.022,linestyle=dashed,dash=0.1cm 0.1cm](6.52,-0.52294445)(6.4,-0.34294444)(6.18,-0.22227074)(5.89007,-0.21260759)(5.60014,-0.20294444)(5.36,-0.22294444)(5.12,-0.50294447)
\pscircle[linewidth=0.027999999,dimen=outer](8.435889,-0.052944463){0.83}
\psellipse[linewidth=0.027999999,linestyle=dashed,dash=0.10cm 0.10cm,dimen=outer](8.435889,-0.11294446)(0.83,0.19)
\psline[linewidth=0.02cm,linestyle=dashed,dash=0.10cm 0.10cm](8.4458885,-0.12294446)(9.225889,-0.10294446)
\psdots[dotsize=0.068](8.425889,-0.12294446)
\end{pspicture} 

}
\caption{Examples of a square pyramid, a triangular pyramid, a cone and a sphere.}
\label{fig:mg:sav:pyramids}
\end{center}
\end{figure}

\subsection*{Surface area of pyramids, cones and spheres}



\begin{table}[H]
\newcolumntype{C}{>{\centering\arraybackslash} m{0.6in} }
\newcolumntype{D}{>{\centering\arraybackslash} m{2in} }
\newcolumntype{E}{>{\centering\arraybackslash} m{6cm} }
\begin{tabular}{|C|D|E|}
\hline
\textbf{Square based pyramid}
&
\begin{center}
\scalebox{0.8} % Change this value to rescale the drawing.
{
\begin{pspicture}(0,-1.7533308)(4.7674937,1.7533307)
\pspolygon[linewidth=0.028222222,fillstyle=solid](0.095092274,-0.6499687)(2.4242375,1.7392195)(3.794405,-0.4034652)
\pspolygon[linewidth=0.028222222,fillstyle=solid](4.753383,-1.4825138)(2.4242375,1.7392195)(1.0540702,-1.7392195)
\pspolygon[linewidth=0.028222222,fillstyle=solid](1.0593361,-1.7392195)(2.3858888,1.7068307)(0.0,-0.68902683)
\psline[linewidth=0.022cm,linestyle=dashed,dash=0.1cm 0.1cm](0.0,-0.68902683)(3.0107446,-0.45565066)
\psline[linewidth=0.027999999,linestyle=dotted,dotsep=0.1cm](2.465889,1.6668307)(2.3658888,-1.0331693)(3.3458889,-1.5931693)(2.445889,1.6268307)(2.445889,1.6268307)(2.485889,1.5868307)
\psline[linewidth=0.024cm,linestyle=dashed,dash=0.1cm 0.1cm](3.025889,-0.4731693)(4.7258887,-1.4731693)
\psline[linewidth=0.02](2.3658888,-0.8331693)(2.545889,-0.9131693)(2.545889,-1.1331693)(2.3658888,-1.0531694)
\psline[linewidth=0.02cm](2.3658888,-0.8531693)(2.3658888,-1.0531694)
\end{pspicture} 
}
\end{center} 
&
$\begin{aligned}
\mbox{\small Surface area} &=  \mbox{\small sum of all the areas of all the faces}\\
&= \mbox{\small area of base } + \textstyle {\frac{1}{2}} \times \\
&~~~~~~\mbox{\small perimeter of base } \times \\
&~~~~~~\mbox{\small slant height of pyramid}
 \end{aligned}$
\\ \hline


\textbf{Triangular based pyramid} &
\begin{center}
\scalebox{0.8} % Change this value to rescale the drawing.
{
\begin{pspicture}(0,-1.9570557)(4.1857533,1.9570556)
\pspolygon[linewidth=0.028222222,fillstyle=solid](0.21823838,-0.536357)(1.4964722,1.917448)(4.0877037,-1.236487)
\pspolygon[linewidth=0.028222222,fillstyle=solid](0.20588888,-1.9170556)(1.4658889,1.8429444)(0.0,-0.5359847)
\pspolygon[linewidth=0.028222222,fillstyle=solid](4.171642,-1.3293833)(1.4975125,1.9429444)(0.21393035,-1.9429444)
\psline[linewidth=0.022cm,linestyle=dashed,dash=0.1cm 0.1cm](0.0,-0.5359847)(4.1181593,-1.2729638)
\psline[linewidth=0.024,linestyle=dotted,dotsep=0.1cm](1.4858888,1.8829445)(1.4458889,-1.2770555)(4.145889,-1.3170556)
\psline[linewidth=0.02](1.4458889,-1.0570556)(1.6658889,-1.0570556)(1.6645871,-1.2770555)(1.4458889,-1.2770555)(1.4458889,-1.0570556)
\end{pspicture} 
}
\end{center}
&

$\begin{aligned}
\mbox{\small Surface area} &=  \mbox{\small sum of all the areas of all the faces}\\
&= \mbox{\small area of base } + \textstyle {\frac{1}{2}} \times \\
&~~~~~~\mbox{\small perimeter of base } \times \\
&~~~~~~\mbox{\small slant height of pyramid}
 \end{aligned}$
 \\ \hline

\textbf{Right cone} &
\begin{center}
 \scalebox{0.7} % Change this value to rescale the drawing.
{
\begin{pspicture}(0,-3.1099448)(4.5657654,3.1100557)
\psline[linewidth=0.028222222](0.24603535,-1.5211127)(2.3344314,3.0959444)(4.551654,-1.6285512)
\psbezier[linewidth=0.027999999](0.3058355,-1.3951261)(0.0,-2.088052)(0.9226326,-2.969958)(2.0297916,-3.0329514)(3.1369507,-3.0959444)(4.551654,-2.5919983)(4.551654,-1.584106)
\psbezier[linewidth=0.022,linestyle=dashed,dash=0.1cm 0.1cm](4.551654,-1.6470991)(4.182601,-1.0801597)(3.5060039,-0.70007807)(2.6143408,-0.6696424)(1.7226781,-0.6392067)(0.9841414,-0.7022)(0.24603535,-1.584106)
\psline[linewidth=0.04,linestyle=dotted,dotsep=0.1cm](2.3444264,3.0043032)(2.261968,-1.7245709)(4.488356,-1.7523878)(4.488356,-1.6133032)(4.46087,-1.6133032)
\psframe[linewidth=0.04,dimen=outer](2.6258888,-1.3299444)(2.225889,-1.7299443)
% \usefont{T1}{ptm}{m}{n}
\rput(3.8704202,1.0400556){$h$}
% \usefont{T1}{ptm}{m}{n}
\rput(3.56042,-1.4399444){$r$}
% \usefont{T1}{ptm}{m}{n}
\rput(1.8404201,0.24005564){$H$}
\end{pspicture} 
}
\end{center}



&

$\begin{aligned}
\mbox{\small Surface area} &=  \mbox{\small area of base }+\mbox{\small area of walls}\\
&= \pi r^{2} +(\frac{1}{2} \times 2\pi rh)\\
&=\pi r^{2} +\pi rh\\
&=\pi r(r+h)\\
 \end{aligned}$\\ \hline

\textbf{Sphere} &
\begin{center}
\scalebox{0.8} % Change this value to rescale the drawing.
{
\begin{pspicture}(0,-1.9)(3.8,1.9)
\pscircle[linewidth=0.027999999,dimen=outer](1.9,0.0){1.9}
\psellipse[linewidth=0.027999999,linestyle=dashed,dash=0.16cm 0.16cm,dimen=outer](1.9,-0.13)(1.84,0.23)
\psline[linewidth=0.04,linestyle=dotted,dotsep=0.1cm](1.96,-0.14)(3.74,-0.14)
\psdots[dotsize=0.09](1.92,-0.16)
\usefont{T1}{ppl}{m}{n}
\rput(2.6345313,0.2){$r$}
\end{pspicture} 
}

\end{center}


&
$\begin{aligned}
\mbox{\small Surface area} &=  4\pi r^{2}
 \end{aligned}$\\ \hline


\end{tabular}
\end{table}

\begin{wex}{Finding the surface area of a triangular pyramid}
 {Find the total surface area of the following solid (correct to one decimal place):\\
\begin{center}
\scalebox{1} % Change this value to rescale the drawing.
{
\begin{pspicture}(0,-2.1164062)(4.2998643,2.0964062)
\pspolygon[linewidth=0.028222222](0.32,-1.7835938)(1.58,1.9764062)(0.11411111,-0.40252286)
\pspolygon[linewidth=0.028222222](4.2857533,-1.1959213)(1.6116236,2.0764062)(0.32804146,-1.8094827)
\psline[linewidth=0.022cm,linestyle=dashed,dash=0.16cm 0.16cm](0.11411111,-0.40252286)(4.2322707,-1.1395019)
\psline[linewidth=0.04cm,linestyle=dotted,dotsep=0.16cm](1.6,2.0764062)(2.26,-1.5035938)
\psline[linewidth=0.024](2.22,-1.2635938)(2.44,-1.2235936)(2.48,-1.4635936)(2.5,-1.4635936)
\psline[linewidth=0.04cm](1.86,-0.5835937)(1.8,-0.8435938)
\psline[linewidth=0.04cm](0.0,-1.0835936)(0.34,-1.0235938)
\psline[linewidth=0.04cm](1.78,-1.4435936)(1.96,-1.7035936)
\usefont{T1}{ppl}{m}{n}
\rput(2.46,-1.9135938){$6$ cm}
\usefont{T1}{ppl}{m}{n}
\rput(2.375,0.3864063){$10$ cm}
\end{pspicture} 
}
\end{center}
}

{
\westep{Find the area of the base of the shape}
\begin{align*}
 \mbox{area of triangle} = \frac{1}{2} bh
\end{align*}
We don't have the height yet but it can be calculated using the Theorem of Pythagoras:
\\
\begin{center}
\scalebox{0.8} % Change this value to rescale the drawing.
{
\begin{pspicture}(0,-2.1564062)(4.64,2.097329)
\pstriangle[linewidth=0.04,dimen=outer](2.24,-1.6435938)(4.48,3.76)
\psline[linewidth=0.04cm,linestyle=dotted,dotsep=0.16cm](2.28,1.9164063)(2.16,-1.6035937)
\psframe[linewidth=0.04,dimen=outer](2.5,-1.2635938)(2.14,-1.6235938)
% \usefont{T1}{ptm}{m}{n}
\rput(3.165,-1.9535937){$3$ cm}
% \usefont{T1}{ptm}{m}{n}
\rput(3.945,0.62640625){$6$ cm}
% \usefont{T1}{ptm}{m}{n}
\rput(2.5845313,0.04640625){$h$}
\end{pspicture} 
}
\end{center}

\begin{align*}
 6^2 &= 3^2+h^2\\
\therefore h&=\sqrt{6^2-3^2}\\
&=3\sqrt{3}\\
\therefore \mbox{ area of triangle} &= \frac{1}{2} \times 6 \times 3\sqrt{3}\\
&=9\sqrt{3}\mbox{ cm}^2
\end{align*}
\westep{Find the area of the sides}
\begin{align*}
 \mbox{area of sides} &= \frac{1}{2} \times \mbox{perimeter of base}\times\mbox{slant height of pyramid}\\
&= \frac{1}{2} \times (6+6+6)\times10\\
&=90\mbox{ cm}^2
\end{align*}
\westep{Find the sume of the areas}
\begin{align*}
 \mbox{total surface area} &= 9\sqrt{3} + 90\\
&=105,9\mbox{ cm}^2
\end{align*}
}
\end{wex}

\begin{wex}{Finding the surface area of a cone}
 {Find the total surface area of the following solid (correct to one decimal place):
\begin{center}
 \scalebox{0.8} % Change this value to rescale the drawing.
{
\begin{pspicture}(0,-3.1099448)(4.5657654,3.1100557)
\psline[linewidth=0.028222222](0.24603535,-1.5211127)(2.3344314,3.0959444)(4.551654,-1.6285512)
\psbezier[linewidth=0.027999999](0.3058355,-1.3951261)(0.0,-2.088052)(0.9226326,-2.969958)(2.0297916,-3.0329514)(3.1369507,-3.0959444)(4.551654,-2.5919983)(4.551654,-1.584106)
\psbezier[linewidth=0.022,linestyle=dashed,dash=0.1cm 0.1cm](4.551654,-1.6470991)(4.182601,-1.0801597)(3.5060039,-0.70007807)(2.6143408,-0.6696424)(1.7226781,-0.6392067)(0.9841414,-0.7022)(0.24603535,-1.584106)
\psline[linewidth=0.04,linestyle=dotted,dotsep=0.1cm](2.3444264,3.0043032)(2.261968,-1.7245709)(4.488356,-1.7523878)(4.488356,-1.6133032)(4.46087,-1.6133032)
\psframe[linewidth=0.04,dimen=outer](2.6258888,-1.3299444)(2.225889,-1.7299443)
% \usefont{T1}{ptm}{m}{n}
\rput(3.8704202,1.0400556){$h$}
% \usefont{T1}{ptm}{m}{n}
\rput(3.56042,-1.4399444){$4$ cm}
% \usefont{T1}{ptm}{m}{n}
\rput(1.8404201,0.24005564){$14$ cm}
\end{pspicture} 
}
\end{center}
}
{
\westep{Find the area of the base of the shape}
\begin{align*}
 \mbox{area of circle} &= \pi r^2\\
&= \pi4^2\\
&=16\pi
\end{align*}

\westep{Find the area of the sides}
\begin{align*}
 \mbox{area of sides} &= \pi rh
\end{align*}
We still need to find the slant height $h$. We can do this using the Theorem of Pythagoras:\\
\begin{center}
\scalebox{0.8} % Change this value to rescale the drawing.
{
\begin{pspicture}(0,-2.1564062)(4.64,2.097329)
\pstriangle[linewidth=0.04,dimen=outer](2.24,-1.6435938)(4.48,3.76)
\psline[linewidth=0.04cm,linestyle=dotted,dotsep=0.16cm](2.28,1.9164063)(2.16,-1.6035937)
\psframe[linewidth=0.04,dimen=outer](2.5,-1.2635938)(2.14,-1.6235938)
% \usefont{T1}{ptm}{m}{n}
\rput(3.165,-1.9535937){$4$ cm}
% \usefont{T1}{ptm}{m}{n}
\rput(3.7,0.62640625){$H$}
% \usefont{T1}{ptm}{m}{n}
\rput(2.7,0.04640625){$14$ cm}
\end{pspicture}
 
}
\end{center}
\begin{align*}
 h^2 &= 4^2 + 14^2\\
\therefore h &= \sqrt{4^2 + 14^2}\\
&= 2\sqrt{53}\\
\therefore \mbox{ area of sides } &= \pi(4)(2\sqrt{53})\\
&= 8\sqrt{53}~\pi
\end{align*}
\westep{Find the sum of the areas}
\begin{align*}
\mbox{total surface area } &= 16\pi + 8\sqrt{53}~\pi\\
&=233,2\mbox{ cm}^2
\end{align*}
}
\end{wex}

\begin{wex}{Finding the surface area of a sphere}
 {Find the total surface area of the following solid (correct to one decimal place):
\begin{center}
\scalebox{0.9} % Change this value to rescale the drawing.
{
\begin{pspicture}(0,-1.9)(3.8,1.9)
\pscircle[linewidth=0.027999999,dimen=outer](1.9,0.0){1.9}
\psellipse[linewidth=0.027999999,linestyle=dashed,dash=0.16cm 0.16cm,dimen=outer](1.9,-0.13)(1.84,0.23)
\psline[linewidth=0.04,linestyle=dotted,dotsep=0.15cm](1.96,-0.14)(3.74,-0.14)
\psdots[dotsize=0.09](1.92,-0.16)
\usefont{T1}{ppl}{m}{n}
\rput(2.6345313,0.2){$5$ cm}
\end{pspicture} 
}

\end{center}
}
{
\westep{Use the formula to find the surface area}
\begin{align*}
 \mbox{surface area of sphere} &= 4 \pi r^2\\
&= 4\pi5^2\\
&=100\pi\\
&=314,2\mbox{ cm}^2
\end{align*}
}

\end{wex}


\begin{wex}
{Surface Area}
{If a cone has a height of $h$ and a base of radius $r$, show that the surface area is $\pi r^2 + \pi r \sqrt{r^2+h^2}$.}
{
\westep{Draw a picture}
\begin{center}

\begin{pspicture}(0,-2.07)(6.5141115,2.0841112)
\psellipse[linewidth=0.028222222,dimen=outer](1.5000001,-1.0699999)(1.5000001,1.0)
\psellipse[linewidth=0.028222222,linestyle=dotted,dotsep=0.10583334cm,dimen=outer](1.5,-1.07)(1.5,1.0)
\psline[linewidth=0.028222222](0.02,-0.93)(1.52,2.07)(2.985889,-0.9158889)
\psline[linewidth=0.028222222cm,arrowsize=0.05291667cm 2.0,arrowlength=1.4,arrowinset=0.4]{<->}(0.0,-1.07)(1.5,-1.07)
\usefont{T1}{ptm}{m}{n}
\rput(0.71453124,-1.35){$r$}
\psline[linewidth=0.028222222cm,arrowsize=0.05291667cm 2.0,arrowlength=1.4,arrowinset=0.4]{<->}(1.5,-1.07)(1.5058889,1.7241111)
\usefont{T1}{ptm}{m}{n}
\rput(1.7345313,0.37){$h$}
\psline[linewidth=0.028222222](1.1,-1.07)(1.1,-0.67)(1.5,-0.67)
\pspolygon[linewidth=0.028222222](3.5,-1.07)(5.0,1.93)(6.5,-1.07)
\psline[linewidth=0.028222222cm,arrowsize=0.05291667cm 2.0,arrowlength=1.4,arrowinset=0.4]{<->}(3.5,-1.27)(5.0,-1.27)
\usefont{T1}{ptm}{m}{n}
\rput(4.2945313,-1.57){$r$}
\psline[linewidth=0.028222222cm,arrowsize=0.05291667cm 2.0,arrowlength=1.4,arrowinset=0.4]{<->}(5.0,-1.07)(5.0,1.93)
\usefont{T1}{ptm}{m}{n}
\rput(5.134531,0.25){$h$}
\usefont{T1}{ptm}{m}{n}
\rput(3.9745312,0.63){$a$}
\psline[linewidth=0.028222222](4.6,-1.07)(4.6,-0.67)(5.0,-0.67)
\end{pspicture} 
\end{center}


\westep{Identify the faces that make up the cone}
The cone has two faces: the base and the walls. The base is a circle of radius $r$ and the walls can be opened out to a sector of a circle. \\

 \scalebox{1} % Change this value to rescale the drawing. 
{ \begin{pspicture}(0,-1.2046875)(7.8553123,1.1646875) 
\psline[linewidth=0.04cm](1.24,1.1046875)(0.02,-0.3753125) 
\psline[linewidth=0.04cm](1.22,1.1046875)(2.44,-0.3753125) 
\psarc[linewidth=0.04](1.24,1.0046875){1.84}{228.81407}{312.27368} 
\psline[linewidth=0.04cm](1.24,1.0846875)(0.18,-0.4953125) 
\psline[linewidth=0.04cm](1.24,1.0846875)(0.34,-0.5953125) 
\psline[linewidth=0.04cm](1.22,1.1046875)(0.5,-0.6753125) 
\psline[linewidth=0.04cm](1.22,1.1046875)(0.7,-0.7353125) 
\psline[linewidth=0.04cm](1.22,1.0846875)(0.92,-0.8153125) 
\psline[linewidth=0.04cm](3.98,-0.6753125)(7.6,-0.6753125) 
\psline[linewidth=0.04cm](3.98,-0.6753125)(4.16,1.1246876) 
\psline[linewidth=0.04cm](4.16,1.1246876)(4.24,-0.6753125) 
\psline[linewidth=0.04cm](4.24,-0.6553125)(4.42,1.1446875) 
\psline[linewidth=0.04cm](4.42,1.1446875)(4.5,-0.6553125) 
\psline[linewidth=0.04cm](4.5,-0.6953125)(4.68,1.1046875) 
\psline[linewidth=0.04cm](4.68,1.1046875)(4.76,-0.6953125) 
\psline[linewidth=0.04cm](4.76,-0.6753125)(4.94,1.1246876) 
\psline[linewidth=0.04cm](4.94,1.1246876)(5.02,-0.6753125) 
\psline[linewidth=0.04cm](5.02,-0.6753125)(5.2,1.1246876) 
\psline[linewidth=0.04cm](5.2,1.1246876)(5.28,-0.6753125) 
\psline[linewidth=0.04cm](5.28,-0.6753125)(5.46,1.1246876) 
\psline[linewidth=0.04cm](5.46,1.1246876)(5.54,-0.6753125) 
\psline[linewidth=0.04cm](3.96,-0.6753125)(5.58,-0.6753125) 
\psline[linewidth=0.04cm](5.54,-0.6753125)(5.72,1.1246876) 
\psline[linewidth=0.04cm](5.72,1.1246876)(5.8,-0.6753125) 
\psline[linewidth=0.04cm](5.8,-0.6753125)(5.98,1.1246876) 
\psline[linewidth=0.04cm](5.98,1.1246876)(6.06,-0.6753125) 
\psline[linewidth=0.04cm](6.06,-0.6953125)(6.24,1.1046875) 
\psline[linewidth=0.04cm](6.24,1.1046875)(6.32,-0.6953125) 
\psline[linewidth=0.04cm](6.32,-0.6553125)(6.5,1.1446875) 
\psline[linewidth=0.04cm](6.5,1.1446875)(6.58,-0.6553125) 
\psline[linewidth=0.04cm](6.58,-0.6753125)(6.76,1.1246876) 
\psline[linewidth=0.04cm](6.76,1.1246876)(6.84,-0.6753125) 
\psline[linewidth=0.04cm](6.84,-0.6753125)(7.02,1.1246876) 
\psline[linewidth=0.04cm](7.02,1.1246876)(7.1,-0.6753125) 
\psline[linewidth=0.04cm](7.1,-0.6753125)(7.28,1.1246876) 
\psline[linewidth=0.04cm](7.28,1.1246876)(7.36,-0.6753125) 
\psline[linewidth=0.04cm](7.36,-0.6553125)(7.54,1.1446875) 
\psline[linewidth=0.04cm](7.54,1.1446875)(7.62,-0.6553125) 
\psline[linewidth=0.03cm,linestyle=dashed,dash=0.16cm 0.16cm](1.22,1.0846875)(1.16,-0.8153125) 
\psline[linewidth=0.03cm,linestyle=dashed,dash=0.16cm 0.16cm](1.22,1.1246876)(1.42,-0.8553125) 
\psline[linewidth=0.03cm,linestyle=dashed,dash=0.16cm 0.16cm](1.22,1.0846875)(1.66,-0.7553125) 
\psline[linewidth=0.03cm,linestyle=dashed,dash=0.16cm 0.16cm](1.22,1.0846875)(1.94,-0.6553125) 
\psline[linewidth=0.03cm,linestyle=dashed,dash=0.16cm 0.16cm](1.2,1.0846875)(2.22,-0.5353125) 
\psline[linewidth=0.11cm,arrowsize=0.05291667cm 2.0,arrowlength=1.4,arrowinset=0.4]{->}(2.46,0.4246875)(3.04,0.4246875) 
\rput(3.6859374,0.4146875){$a$} 
\psline[linewidth=0.03cm,linestyle=dashed,dash=0.16cm 0.16cm,arrowsize=0.05291667cm 2.0,arrowlength=1.4,arrowinset=0.4]{->}(3.68,0.2646875)(3.68,-0.6753125) 
\psline[linewidth=0.03cm,linestyle=dashed,dash=0.16cm 0.16cm,arrowsize=0.05291667cm 2.0,arrowlength=1.4,arrowinset=0.4]{->}(3.68,0.5446875)(3.68,1.1246876) 
\rput(5.6826563,-0.9853125){$2\pi r$ = circumference} \end{pspicture} } 
\\ This curved surface can be cut into many thin triangles with height close to $a$ ($a$ is called the \emph{slant height}). The area of these triangles will add up to $\frac{1}{2}\times$base$\times$height(of a small triangle) which is $\frac{1}{2}\times2\pi r \times a = \pi r a $ 

\westep{Calculate $a$}
    $a$ can be calculated by using the Theorem of Pythagoras. Therefore:
\begin{equation*}
a = \sqrt{r^{2} + h^{2}}
\end{equation*}
\westep{Calculate the area of the circular base}
\begin{equation*}
A_{b} = \pi r^{2}
\end{equation*}
\westep{Calculate the area of the curved walls}
\begin{eqnarray*}
A_{w} &=& \pi r a \\
&=& \pi r \sqrt{r^{2}+h^{2}}
\end{eqnarray*}
\westep{Calculate surface area $A$}
\begin{eqnarray*}
 A &=& A_{b} + A_{w} \\
  &=& \pi r^{2} + \pi r \sqrt{r^{2}+h^{2}}
\end{eqnarray*}
}
\end{wex}

\subsection*{Volume of pyramids, cones and spheres}

Volume is measured in cubic units, for example cm$^2$, m$^2$ or mm$^2$.

\begin{table}[H]
\newcolumntype{C}{>{\centering\arraybackslash} m{0.6in} }
\newcolumntype{D}{>{\centering\arraybackslash} m{2in} }
\newcolumntype{E}{>{\centering\arraybackslash} m{6cm} }
\begin{tabular}{|C|D|E|}
\hline
\textbf{Square based pyramid}
&
\begin{center}
\scalebox{0.8} % Change this value to rescale the drawing.
{
\begin{pspicture}(0,-1.7533308)(4.7674937,1.7533307)
\pspolygon[linewidth=0.028222222,fillstyle=solid](0.095092274,-0.6499687)(2.4242375,1.7392195)(3.794405,-0.4034652)
\pspolygon[linewidth=0.028222222,fillstyle=solid](4.753383,-1.4825138)(2.4242375,1.7392195)(1.0540702,-1.7392195)
\pspolygon[linewidth=0.028222222,fillstyle=solid](1.0593361,-1.7392195)(2.3858888,1.7068307)(0.0,-0.68902683)
\psline[linewidth=0.022cm,linestyle=dashed,dash=0.1cm 0.1cm](0.0,-0.68902683)(3.0107446,-0.45565066)
\psline[linewidth=0.027999999,linestyle=dotted,dotsep=0.1cm](2.465889,1.6668307)(2.3658888,-1.0331693)(3.3458889,-1.5931693)(2.445889,1.6268307)(2.445889,1.6268307)(2.485889,1.5868307)
\psline[linewidth=0.024cm,linestyle=dashed,dash=0.1cm 0.1cm](3.025889,-0.4731693)(4.7258887,-1.4731693)
\psline[linewidth=0.02](2.3658888,-0.8331693)(2.545889,-0.9131693)(2.545889,-1.1331693)(2.3658888,-1.0531694)
\psline[linewidth=0.02cm](2.3658888,-0.8531693)(2.3658888,-1.0531694)
\end{pspicture} 
}
\end{center} 
&
$\begin{aligned}
\mbox{ Volume} &=  \frac{1}{3} \mbox{ area of base}\\
&~~~~~~\times \mbox{height of pyramid }
 \end{aligned}$
\\ \hline


\textbf{Triangular based pyramid} &
\begin{center}
\scalebox{0.8} % Change this value to rescale the drawing.
{
\begin{pspicture}(0,-1.9570557)(4.1857533,1.9570556)
\pspolygon[linewidth=0.028222222,fillstyle=solid](0.21823838,-0.536357)(1.4964722,1.917448)(4.0877037,-1.236487)
\pspolygon[linewidth=0.028222222,fillstyle=solid](0.20588888,-1.9170556)(1.4658889,1.8429444)(0.0,-0.5359847)
\pspolygon[linewidth=0.028222222,fillstyle=solid](4.171642,-1.3293833)(1.4975125,1.9429444)(0.21393035,-1.9429444)
\psline[linewidth=0.022cm,linestyle=dashed,dash=0.1cm 0.1cm](0.0,-0.5359847)(4.1181593,-1.2729638)
\psline[linewidth=0.024,linestyle=dotted,dotsep=0.1cm](1.4858888,1.8829445)(1.4458889,-1.2770555)(4.145889,-1.3170556)
\psline[linewidth=0.02](1.4458889,-1.0570556)(1.6658889,-1.0570556)(1.6645871,-1.2770555)(1.4458889,-1.2770555)(1.4458889,-1.0570556)
\end{pspicture} 
}
\end{center}
&

$\begin{aligned}
\mbox{Volume} &= \frac{1}{3} \mbox{ area of base}\\
&~~~~~~\times \mbox{ height of pyramid }
 \end{aligned}$
 \\ \hline

\textbf{Right cone} &
\begin{center}
 \scalebox{0.7} % Change this value to rescale the drawing.
{
\begin{pspicture}(0,-3.1099448)(4.5657654,3.1100557)
\psline[linewidth=0.028222222](0.24603535,-1.5211127)(2.3344314,3.0959444)(4.551654,-1.6285512)
\psbezier[linewidth=0.027999999](0.3058355,-1.3951261)(0.0,-2.088052)(0.9226326,-2.969958)(2.0297916,-3.0329514)(3.1369507,-3.0959444)(4.551654,-2.5919983)(4.551654,-1.584106)
\psbezier[linewidth=0.022,linestyle=dashed,dash=0.1cm 0.1cm](4.551654,-1.6470991)(4.182601,-1.0801597)(3.5060039,-0.70007807)(2.6143408,-0.6696424)(1.7226781,-0.6392067)(0.9841414,-0.7022)(0.24603535,-1.584106)
\psline[linewidth=0.04,linestyle=dotted,dotsep=0.1cm](2.3444264,3.0043032)(2.261968,-1.7245709)(4.488356,-1.7523878)(4.488356,-1.6133032)(4.46087,-1.6133032)
\psframe[linewidth=0.04,dimen=outer](2.6258888,-1.3299444)(2.225889,-1.7299443)
% \usefont{T1}{ptm}{m}{n}
\rput(3.8704202,1.0400556){$h$}
% \usefont{T1}{ptm}{m}{n}
\rput(3.56042,-1.4399444){$r$}
% \usefont{T1}{ptm}{m}{n}
\rput(1.8404201,0.24005564){$H$}
\end{pspicture} 
}
\end{center}



&

$\begin{aligned}
\mbox{Volume} &=  \frac{1}{3}\mbox{ area of base}\\
&~~~~~~\times \mbox{ height of cone }\\
&= \frac{1}{3} \times \pi r^2 \times h
 \end{aligned}$\\ \hline

\textbf{Sphere} &
\begin{center}
\scalebox{0.8} % Change this value to rescale the drawing.
{
\begin{pspicture}(0,-1.9)(3.8,1.9)
\pscircle[linewidth=0.027999999,dimen=outer](1.9,0.0){1.9}
\psellipse[linewidth=0.027999999,linestyle=dashed,dash=0.16cm 0.16cm,dimen=outer](1.9,-0.13)(1.84,0.23)
\psline[linewidth=0.04,linestyle=dotted,dotsep=0.1cm](1.96,-0.14)(3.74,-0.14)
\psdots[dotsize=0.09](1.92,-0.16)
\usefont{T1}{ppl}{m}{n}
\rput(2.6345313,0.2){$r$}
\end{pspicture} 
}

\end{center}


&
$\begin{aligned}
\mbox{Volume} &=  \frac{4}{3}\pi r^{3}
 \end{aligned}$\\ \hline


\end{tabular}
\end{table}
\pagebreak
\begin{wex}{Volume of a Pyramid}
{
What is the volume of a square pyramid, $3$ cm high with a side length of $2$ cm?
}
{
\westep{Determine the correct formula}
The volume of a pyramid is 
$$V=\frac{1}{3}A\cdot h,$$ \\
where $A$ is the area of the base and $h$ is the height of the pyramid. For a square base this means
$$V = \frac{1}{3}a\cdot a \cdot h$$ \\
where $a$ is the length of the side of the square base.

\begin{center}
\scalebox{0.8} % Change this value to rescale the drawing.
{

\begin{pspicture}(0,-2.89)(5.58,2.89)
% \usefont{T1}{ptm}{m}{n}
\rput{0.6029805}(-0.022694819,-0.047077317){\rput(4.4319425,-2.1803145){$2~$cm}}
% \usefont{T1}{ptm}{m}{n}
\rput{-1.0300905}(0.040004794,0.011890239){\rput(1,-2.2194574){$2~$cm}}
% \usefont{T1}{ptm}{m}{n}
\rput(2.0929687,1.255){\small $3~$cm}
\psline[linewidth=0.04cm](5.54,-0.97)(2.74,2.85)
\psline[linewidth=0.04cm](2.9,-2.85)(2.74,2.87)
\psline[linewidth=0.04cm](2.74,2.87)(0.02,-0.87)
\psline[linewidth=0.04cm](0.04,-0.87)(2.78,0.89)
\psline[linewidth=0.04cm](2.76,0.89)(5.56,-0.99)
\psline[linewidth=0.04cm](0.0,-0.87)(2.9,-2.87)
\psline[linewidth=0.04cm](2.9,-2.87)(5.56,-0.99)
\psdots[dotsize=0.12](2.62,-0.83)
\psline[linewidth=0.04cm,linestyle=dashed,dash=0.17638889cm 0.10583334cm](2.64,-0.81)(2.72,2.75)
\psline[linewidth=0.04cm,linestyle=dashed,dash=0.17638889cm 0.10583334cm](2.62,-0.85)(1.44,0.03)
\psline[linewidth=0.04cm](2.4,-0.69)(2.4,-0.33)
\psline[linewidth=0.04cm](2.38,-0.33)(2.66,-0.53)
\end{pspicture} 
}
\end{center}

\westep{Substitute the given values}
\begin{eqnarray*}
&=&\frac{1}{3}\cdot 2 \cdot 2 \cdot 3\\
&=&\frac{1}{3} \cdot 12\\
&=&4\textnormal{ cm}^3
\end{eqnarray*}
}
\end{wex}


\begin{wex}{Finding the volume of a triangular pyramid}
 {Find the volume of the following solid (correct to one decimal place):\\
\begin{center}
\scalebox{1} % Change this value to rescale the drawing.
{
\begin{pspicture}(0,-2.1164062)(4.2998643,2.0964062)
\pspolygon[linewidth=0.028222222](0.32,-1.7835938)(1.58,1.9764062)(0.11411111,-0.40252286)
\pspolygon[linewidth=0.028222222](4.2857533,-1.1959213)(1.6116236,2.0764062)(0.32804146,-1.8094827)
\psline[linewidth=0.022cm,linestyle=dashed,dash=0.1cm 0.1cm](0.11411111,-0.40252286)(4.2322707,-1.1395019)
\psline[linewidth=0.04cm,linestyle=dotted,dotsep=0.1cm](1.6,2.0764062)(2.26,-1.5035938)
\psline[linewidth=0.024](2.22,-1.2635938)(2.44,-1.2235936)(2.48,-1.4635936)(2.5,-1.4635936)
\psline[linewidth=0.04cm](1.86,-0.5835937)(1.8,-0.8435938)
\psline[linewidth=0.04cm](0.0,-1.0835936)(0.34,-1.0235938)
\psline[linewidth=0.04cm](1.78,-1.4435936)(1.96,-1.7035936)
% \usefont{T1}{ptm}{m}{n}
\rput(2.46,-1.9135938){$8$ cm}
% \usefont{T1}{ptm}{m}{n}
\rput(2.375,0.3864063){$12$ cm}
\end{pspicture} 
}
\end{center}
}

{
\westep{Find the area of the base of the shape}
\begin{align*}
 \mbox{area of triangle} = \frac{1}{2} bh
\end{align*}
We don't have the height yet but it can be calculated using the Theorem of Pythagoras:
\\
\begin{center}
\scalebox{0.8} % Change this value to rescale the drawing.
{
\begin{pspicture}(0,-2.1564062)(4.64,2.097329)
\pstriangle[linewidth=0.04,dimen=outer](2.24,-1.6435938)(4.48,3.76)
\psline[linewidth=0.04cm,linestyle=dotted,dotsep=0.16cm](2.28,1.9164063)(2.16,-1.6035937)
\psframe[linewidth=0.04,dimen=outer](2.5,-1.2635938)(2.14,-1.6235938)
% \usefont{T1}{ptm}{m}{n}
\rput(3.165,-1.9535937){$4$ cm}
% \usefont{T1}{ptm}{m}{n}
\rput(3.945,0.62640625){$8$ cm}
% \usefont{T1}{ptm}{m}{n}
\rput(2.5845313,0.04640625){$h$}
\end{pspicture} 
}
\end{center}

\begin{align*}
 8^2 &= 4^2+h^2\\
\therefore h&=\sqrt{8^2-3^2}\\
&=4\sqrt{3}\\
\therefore \mbox{ area of triangle} &= \frac{1}{2} \times 8 \times 3\sqrt{3}\\
&=16\sqrt{3}\mbox{ cm}^2
\end{align*}

\westep{Find the height $H$ of the triangle to calculate the volume}
We don't have the height yet but it can be calculated by bisecting the pyramid abd using the Theorem of Pythagoras:
\\
\begin{center}
\scalebox{0.8} % Change this value to rescale the drawing.
{
\begin{pspicture}(0,-2.1564062)(4.64,2.097329)
\pstriangle[linewidth=0.04,dimen=outer](2.24,-1.6435938)(4.48,3.76)
\psline[linewidth=0.04cm,linestyle=dotted,dotsep=0.16cm](2.28,1.9164063)(2.16,-1.6035937)
\psframe[linewidth=0.04,dimen=outer](2.5,-1.2635938)(2.14,-1.6235938)
% \usefont{T1}{ptm}{m}{n}
\rput(3.165,-1.9535937){$4$ cm}
% \usefont{T1}{ptm}{m}{n}
\rput(3.945,0.62640625){$12$ cm}
% \usefont{T1}{ptm}{m}{n}
\rput(2.5845313,0.04640625){$H$}
\end{pspicture} 
}
\end{center}

\begin{align*}
 12^2 &= 4^2+h^2\\
\therefore H&=\sqrt{12^2-4^2}\\
&=8\sqrt{2}\\
\therefore \mbox{ volume} &= \frac{1}{3} \times 16\sqrt{3} \times 8\sqrt{2}\\
&=104,5\mbox{ cm}^3
\end{align*}
}
\end{wex}

\begin{wex}{Finding the volume of a cone}
 {Find the volume of the following solid (correct to one decimal place):
\begin{center}
 \scalebox{0.8} % Change this value to rescale the drawing.
{
\begin{pspicture}(0,-3.1099448)(4.5657654,3.1100557)
\psline[linewidth=0.028222222](0.24603535,-1.5211127)(2.3344314,3.0959444)(4.551654,-1.6285512)
\psbezier[linewidth=0.027999999](0.3058355,-1.3951261)(0.0,-2.088052)(0.9226326,-2.969958)(2.0297916,-3.0329514)(3.1369507,-3.0959444)(4.551654,-2.5919983)(4.551654,-1.584106)
\psbezier[linewidth=0.022,linestyle=dashed,dash=0.1cm 0.1cm](4.551654,-1.6470991)(4.182601,-1.0801597)(3.5060039,-0.70007807)(2.6143408,-0.6696424)(1.7226781,-0.6392067)(0.9841414,-0.7022)(0.24603535,-1.584106)
\psline[linewidth=0.04,linestyle=dotted,dotsep=0.1cm](2.3444264,3.0043032)(2.261968,-1.7245709)(4.488356,-1.7523878)(4.488356,-1.6133032)(4.46087,-1.6133032)
\psframe[linewidth=0.04,dimen=outer](2.6258888,-1.3299444)(2.225889,-1.7299443)
% \usefont{T1}{ptm}{m}{n}
\rput(3.8704202,1.0400556){$h$}
% \usefont{T1}{ptm}{m}{n}
\rput(3.56042,-1.4399444){$4$ cm}
% \usefont{T1}{ptm}{m}{n}
\rput(1.8404201,0.24005564){$11$ cm}
\end{pspicture} 
}
\end{center}
}
{
\westep{Find the area of the base of the shape}
\begin{align*}
 \mbox{area of circle} &= \pi r^2\\
&= \pi3^2\\
&=9\pi
\end{align*}

\westep{Multiply the area of the base by $\frac{1}{3} \times $ height of pyramid to get the volume}
\begin{align*}
 \mbox{volume} &= \frac{1}{3} \times 9\pi 11\\
&=103,7 \mbox{ units}^3
\end{align*}
}
\end{wex}

\begin{wex}{Finding the volume of a sphere}
 {Find the total volume of the following solid (correct to one decimal place):
\begin{center}
\scalebox{0.9} % Change this value to rescale the drawing.
{
\begin{pspicture}(0,-1.9)(3.8,1.9)
\pscircle[linewidth=0.027999999,dimen=outer](1.9,0.0){1.9}
\psellipse[linewidth=0.027999999,linestyle=dashed,dash=0.16cm 0.16cm,dimen=outer](1.9,-0.13)(1.84,0.23)
\psline[linewidth=0.04,linestyle=dotted,dotsep=0.1cm](1.96,-0.14)(3.74,-0.14)
\psdots[dotsize=0.09](1.92,-0.16)
\usefont{T1}{ppl}{m}{n}
\rput(2.6345313,0.2){$4$ cm}
\end{pspicture} 
}

\end{center}
}
{
\westep{Use the formula to find the volume}
\begin{align*}
 \mbox{volume} &= \frac{4}{3} \pi r^3\\
&= \frac{4}{3}\pi(4)^2\\
&=268,1\mbox{ cm}^2
\end{align*}
}

\end{wex}
\pagebreak %Forced pagebreak to get wex to stop overflowing
\begin{wex}{Finding volume}
 {A triangular pyramid is placed on top of a triangular prism. 
The prism has an equilateral triangle of side length $20$ cm as a base, and has a height of $42$ cm. The pyramid has a height of $12$ cm.

\begin{enumerate}
\item Find the total volume of the object.
\item Find the surface area of each face of the pyramid.
\item Find the total surface area of the object.
\end{enumerate}

\begin{center}

\scalebox{1} % Change this value to rescale the drawing.
{
\begin{pspicture}(0,-2.0664062)(5.57,2.0464063)
\definecolor{color338b}{rgb}{0.6,0.6,0.6}
\definecolor{color376b}{rgb}{0.8,0.8,0.8}
\pspolygon[linewidth=0.04,fillstyle=solid,fillcolor=color338b](3.45,0.6053162)(3.87,1.7664063)(3.87,-0.6525037)(3.45,-1.8135937)
\pspolygon[linewidth=0.04,fillstyle=solid,fillcolor=color338b](3.03,2.0264063)(3.87,1.8321205)(3.45,0.6064063)
\pspolygon[linewidth=0.04,fillstyle=solid,fillcolor=color376b](3.03,2.0264063)(2.23,1.3264062)(3.43,0.62640625)
\pspolygon[linewidth=0.04,fillstyle=solid,fillcolor=color376b](2.25,1.3264062)(3.45,0.62640625)(3.45,-1.8735938)(2.25,-1.1735938)
\psline[linewidth=0.04cm,linestyle=dashed,dash=0.16cm 0.16cm](2.25,1.3664062)(3.83,1.8064063)
\psline[linewidth=0.04cm,linestyle=dashed,dash=0.16cm 0.16cm](2.25,-1.1735938)(3.83,-0.61359376)
\psline[linewidth=0.04cm](3.03,-0.75359374)(3.13,-0.9735938)
\psline[linewidth=0.04cm](2.93,-1.4135938)(2.77,-1.6135937)
\psline[linewidth=0.04cm](3.53,-1.1335938)(3.77,-1.1935937)
% \usefont{T1}{ptm}{m}{n}
\rput(2.265,-1.8635937){$20$ cm}
% \usefont{T1}{ptm}{m}{n}
\rput(4.765,0.5964062){$20$ cm}
\psline[linewidth=0.03,linestyle=dotted,dotsep=0.1cm](3.03,1.9664062)(3.05,1.2664063)(2.79,1.0464063)(2.79,1.0464063)(2.79,1.0464063)(2.75,1.0464063)
\psline[linewidth=0.04cm](3.05,1.6064062)(1.51,1.6064062)
% \usefont{T1}{ptm}{m}{n}
\rput(0.705,1.6164062){$12$ cm}
\psline[linewidth=0.02](3.09,1.4064063)(2.97,1.3464062)(2.97,1.2064062)
\end{pspicture} 
}
\end{center}
}
{
\westep{Calcuate the total volume by finding the individual volumes and adding them together}

\scalebox{0.8} % Change this value to rescale the drawing.
{
\begin{pspicture}(0,-1.9570554)(4.1857533,1.9570554)
\pspolygon[linewidth=0.028222222](0.20588888,-1.9170556)(1.4658889,1.8429444)(0.0,-0.5359847)
\pspolygon[linewidth=0.028222222](4.171642,-1.3293831)(1.4975125,1.9429444)(0.21393035,-1.9429444)
\psline[linewidth=0.022cm,linestyle=dashed,dash=0.1cm 0.1cm](0.0,-0.5359847)(4.1181593,-1.2729638)
\psline[linewidth=0.024,linestyle=dotted,dotsep=0.1cm](1.4858888,1.8829445)(1.4458889,-1.2770555)(4.145889,-1.3170556)
\psline[linewidth=0.02](1.4458889,-1.0570556)(1.6658889,-1.0629445)(1.6645871,-1.2770555)(1.4458889,-1.2770555)(1.4458889,-1.0570556)
% \usefont{T1}{ptm}{m}{n}
\rput(1.9,0.44346178){$12 $ cm}
% \usefont{T1}{ptm}{m}{n}
\rput(2.2608888,-1.8565382){$20 $cm}
\end{pspicture} 
}
\hspace{10pt}
\scalebox{0.8} % Change this value to rescale the drawing.
{
\begin{pspicture}(0,-2.1564062)(4.64,2.097329)
\pstriangle[linewidth=0.04,dimen=outer](2.24,-1.6435938)(4.48,3.76)
\psline[linewidth=0.04cm,linestyle=dotted,dotsep=0.16cm](2.28,1.9164063)(2.16,-1.6035937)
\psframe[linewidth=0.04,dimen=outer](2.5,-1.2635938)(2.14,-1.6235938)
% \usefont{T1}{ptm}{m}{n}
\rput(3.165,-1.9535937){$10$ cm}
% \usefont{T1}{ptm}{m}{n}
\rput(3.945,0.62640625){$20$ cm}
% \usefont{T1}{ptm}{m}{n}
\rput(2.5845313,0.04640625){$h$}
\end{pspicture} 
}

\\To find the height of the base triangle:
\begin{align*}
 20^2 &= 10^2 + h^2\\
\therfore h&= \sqrt{20^2-10^2}\\
&=10 \sqrt{3}
\end{align*}
\begin{center}
\scalebox{0.8} % Change this value to rescale the drawing.
{
\begin{pspicture}(0,-2.1564062)(4.64,2.097329)
\pstriangle[linewidth=0.04,dimen=outer](2.24,-1.6435938)(4.48,3.76)
\psline[linewidth=0.04cm,linestyle=dotted,dotsep=0.16cm](2.28,1.9164063)(2.16,-1.6035937)
\psframe[linewidth=0.04,dimen=outer](2.5,-1.2635938)(2.14,-1.6235938)
% \usefont{T1}{ptm}{m}{n}
\rput(3.165,-1.9535937){$4$ cm}
% \usefont{T1}{ptm}{m}{n}
\rput(3.945,0.62640625){$12$ cm}
% \usefont{T1}{ptm}{m}{n}
% \rput(2.5845313,0.04640625){$H$}
\end{pspicture} 
}
\end{center}


To find the area of the base:
\begin{align*}
\mbox{area of triangle } &= \frac{1}{2} \times 20 \times 10 \sqrt{3}\\
&=100 \sqrt{3}
\end{align*}
To find the volume of the pyramid:\\
We already have the area of the base triangle above, therefore:\\

\begin{align*}
\mbox{volume } &= 100 \sqrt{3} \times 42\\
&=7274,661
\end{align*}
Total volume:
\begin{align*}
\mbox{total volume } &= 7274,661 + 692,82\\
&=7967,4 \mbox{ cm}^3
\end{align*}
\westep{Find the surface area of each exposed face of the solid}
\begin{center}
\scalebox{1} % Change this value to rescale the drawing.
{
\begin{pspicture}(0,-4.873536)(7.0,3.3485236)
\psframe[linewidth=0.04,dimen=outer](7.0,1.3682163)(0.0,-2.8717837)
\psline[linewidth=0.04cm](2.3,1.3082162)(2.3,-2.8517838)
\psline[linewidth=0.04cm](4.68,1.3282163)(4.7,-2.8317838)
\pstriangle[linewidth=0.04,dimen=outer](3.53,1.3282162)(2.38,2.04)
\rput{59.82581}(3.4844987,-0.4001394){\pstriangle[linewidth=0.04,dimen=outer](2.09,1.8082162)(2.38,2.04)}
\rput{59.386894}(4.588479,-2.390166){\pstriangle[linewidth=0.04,dimen=outer](4.39,1.8082162)(2.38,2.04)}
\rput{60.432247}(-1.2791417,-4.507204){\pstriangle[linewidth=0.04,dimen=outer](3.23,-4.3717837)(2.38,2.04)}
\usefont{T1}{ptm}{m}{n}
\rput(2.2545311,2.6982162){$A$}
\usefont{T1}{ptm}{m}{n}
\rput(3.4545312,1.9182162){$B$}
\usefont{T1}{ptm}{m}{n}
\rput(4.6445312,2.6382163){$C$}
\usefont{T1}{ptm}{m}{n}
\rput(1.0945313,-0.56178373){$D$}
\usefont{T1}{ptm}{m}{n}
\rput(3.4445312,-0.54178375){$E$}
\usefont{T1}{ptm}{m}{n}
\rput(5.8545313,-0.54178375){$F$}
\usefont{T1}{ptm}{m}{n}
\rput(3.5745313,-3.4017837){$G$}
\end{pspicture} 
}
\end{center}
\\
Above is a rough sketch of the net of the solid. Faces $A$, $B$, $C$ are congruent as they are the
sides of the triangular based pyramid. The length of the rectangle (made up of the $3$ smaller
rectangles $D$, $E$, $F$) is equal to the perimeter of the base triangle. The base triangle is exactly
the same area as the base triangle of the pyramid which we calculated earlier.\\

To find the area of $A$, $B$, and $C$:\\
bisect the pyramid to find the slant height of each face:
% \begin{figure}
\begin{center}
\scalebox{0.8} % Change this value to rescale the drawing.
{
\begin{pspicture}(0,-2.1564062)(4.64,2.097329)
\pstriangle[linewidth=0.04,dimen=outer](2.24,-1.6435938)(4.48,3.76)
\psline[linewidth=0.04cm,linestyle=dotted,dotsep=0.16cm](2.28,1.9164063)(2.16,-1.6035937)
\psframe[linewidth=0.04,dimen=outer](2.5,-1.2635938)(2.14,-1.6235938)
% \usefont{T1}{ptm}{m}{n}
\rput(3.165,-1.9535937){$10$ cm}
% \usefont{T1}{ptm}{m}{n}
\rput(3.945,0.62640625){$h$}
% \usefont{T1}{ptm}{m}{n}
\rput(2.5845313,0.04640625){$12$ cm}
\end{pspicture} 
}
\end{center}
% \caption{}
% \end{figure}
A bisected view of the pyramid to find the slant height of each face
\begin{align*}
h^2 &= 12^2+10^2\\
\therefore h &= \sqrt{12^2+10^2}\\
&=2\sqrt{61} %CHECK THIS?!
\end{align*}
% \begin{figure}
\begin{center}
\scalebox{0.8} % Change this value to rescale the drawing.
{
\begin{pspicture}(0,-2.1564062)(4.64,2.097329)
\pstriangle[linewidth=0.04,dimen=outer](2.24,-1.6435938)(4.48,3.76)
\psline[linewidth=0.04cm,linestyle=dotted,dotsep=0.16cm](2.28,1.9164063)(2.16,-1.6035937)
\psframe[linewidth=0.04,dimen=outer](2.5,-1.2635938)(2.14,-1.6235938)
% \usefont{T1}{ptm}{m}{n}
% \rput(3.165,-1.9535937){$10$ cm}
% \usefont{T1}{ptm}{m}{n}
% \rput(3.945,0.62640625){$h$}
% \usefont{T1}{ptm}{m}{n}
\rput(2.7,0.04640625){$2\sqrt{61}$}
\end{pspicture} 
}
\end{center}
% \caption{A rough sketch of face $B$}
% \end{figure}
A rough sketch of face $B$
\begin{align*}
\mbox{area of }B &= \frac{1}{2} \times 20 \times 2\sqrt{61}\\
&=20\sqrt{61}\\
\therefore A=B=C&=156\mbox{ cm}^2
\end{align*}

Finding the area of $D$, $E$, and $F$:
\begin{align*}
\mbox{area of rectangle} &= \mbox{perimeter of base triangle } \times \mbox{ height of prism}\\
&= (20+20+20) \times 42\\
\therefore D+E+F&=2520\mbox{ cm}^2
\end{align*}

State the area of $G$:
\begin{align*}
\mbox{area of }G &= 100\sqrt{3}\\
&=173,21\mbox{ cm}^2
\end{align*}

\westep{Find the total surface area by adding together all the areas}
\begin{align*}
\mbox{total surface area } &= 100\sqrt{3} + 2520 + 3(20\sqrt{61})\\
&=3161,8\mbox{ cm}^2
\end{align*}
}
\end{wex}

\begin{exercises}{}
 {
Find the total surface area of the following solids (round off to one decimal place if needed):
\begin{center}
\scalebox{0.7} % Change this value to rescale the drawing.
{
\begin{pspicture}(0,-6.555141)(11.58,6.555141)
\psline[linewidth=0.028222222](0.7205463,1.9239725)(2.8089423,6.5410295)(5.026165,1.816534)
\psline[linewidth=0.04,linestyle=dotted,dotsep=0.1cm](2.8189373,6.4493885)(2.736479,1.7205143)(4.962867,1.6926974)(4.962867,1.831782)(4.935381,1.831782)
% \usefont{T1}{ptm}{m}{n}
\rput(4.7499313,4.485141){$13$ cm}
\psbezier[linewidth=0.027999999](0.78034645,2.0499592)(0.47451097,1.3570333)(1.3971436,0.47512725)(2.5043025,0.41213384)(3.6114616,0.34914044)(5.026165,0.85308695)(5.026165,1.8609792)
\psbezier[linewidth=0.022,linestyle=dashed,dash=0.1cm 0.1cm](5.026165,1.7979861)(4.657112,2.3649256)(3.9805148,2.7450073)(3.0888517,2.7754428)(2.1971886,2.8058784)(1.4586524,2.7428854)(0.7205463,1.8609792)
\psframe[linewidth=0.04,dimen=outer](3.1003997,2.115141)(2.7003999,1.7151409)
% \usefont{T1}{ptm}{m}{n}
\rput(4.009931,2.0051408){$5$ cm}
\pspolygon[linewidth=0.028222222](7.28,1.7148592)(8.54,5.474859)(7.074111,3.09593)
\pspolygon[linewidth=0.028222222](11.245753,2.3025317)(8.571624,5.574859)(7.2880416,1.6889703)
\psline[linewidth=0.022cm,linestyle=dashed,dash=0.1cm 0.1cm](7.074111,3.09593)(11.19227,2.358951)
\psline[linewidth=0.04cm,linestyle=dotted,dotsep=0.1cm](8.56,5.574859)(9.22,1.9948592)
\psline[linewidth=0.024](9.18,2.2348592)(9.4,2.2748594)(9.44,2.0348594)(9.46,2.0348594)
\psline[linewidth=0.04cm](8.82,2.9148593)(8.76,2.6548593)
\psline[linewidth=0.04cm](6.96,2.4148595)(7.3,2.4748592)
\psline[linewidth=0.04cm](8.74,2.0548594)(8.92,1.7948594)
% \usefont{T1}{ptm}{m}{n}
\rput(9.395,1.5848593){$6$ cm}
% \usefont{T1}{ptm}{m}{n}
\rput(9.31,3.8848593){$10$ cm}
% \usefont{T1}{ptm}{m}{n}
\rput(1.0020312,5.3748593){\LARGE \textbf{1.}}
% \usefont{T1}{ptm}{m}{n}
\rput(6.9220314,5.3748593){\LARGE\textbf{2.}}
\pscircle[linewidth=0.027999999,dimen=outer](9.68,-3.4851408){1.9}
\psellipse[linewidth=0.027999999,linestyle=dashed,dash=0.16cm 0.16cm,dimen=outer](9.68,-3.6151407)(1.84,0.23)
\psline[linewidth=0.027999999cm,linestyle=dotted,dotsep=0.1cm](9.74,-3.6251407)(11.52,-3.6251407)
\psdots[dotsize=0.09](9.7,-3.6451406)
% \usefont{T1}{ppl}{m}{n}
\rput(10.499532,-3.1651409){$10$ cm}
% \usefont{T1}{ptm}{m}{n}
\rput{0.6029805}(-0.05992648,-0.0624806){\rput(5.8769474,-5.725771){$6$ cm}}
% \usefont{T1}{ptm}{m}{n}
\rput{-1.0300905}(0.10617459,0.030957164){\rput(1.7449429,-5.8903265){$5$ cm}}
% \usefont{T1}{ptm}{m}{n}
\rput(5.04,-2.8101408){$12$ cm}
\psline[linewidth=0.04cm](6.4,-4.635141)(3.6,-0.8151407)
\psline[linewidth=0.04cm](3.76,-6.5151405)(3.6,-0.79514074)
\psline[linewidth=0.04cm](3.6,-0.79514074)(0.88,-4.5351405)
\psline[linewidth=0.04cm](0.9,-4.5351405)(3.64,-2.7751408)
\psline[linewidth=0.04cm](3.62,-2.7751408)(6.42,-4.655141)
\psline[linewidth=0.04cm](0.86,-4.5351405)(3.76,-6.5351405)
\psline[linewidth=0.04cm](3.76,-6.5351405)(6.42,-4.655141)
\psline[linewidth=0.04cm,linestyle=dashed,dash=0.17638889cm 0.10583334cm](5.26,-5.4551406)(3.58,-0.91514075)
\psline[linewidth=0.04cm](5.16,-5.155141)(5.4,-4.9751406)
\psline[linewidth=0.04cm](5.38,-4.9751406)(5.5,-5.2951407)
% \usefont{T1}{ptm}{m}{n}
\rput(1.3220313,-1.3851408){\LARGE\textbf{3.}}
% \usefont{T1}{ptm}{m}{n}
\rput(7.5220313,-1.4251407){\LARGE\textbf{4.}}
\end{pspicture} 
}

\end{center}
\\

Find the total surface area of the following solids (round off to one decimal place if needed):
\begin{center}
\scalebox{0.7} % Change this value to rescale the drawing.
{
\begin{pspicture}(0,-6.555141)(11.58,6.555141)
\psline[linewidth=0.028222222](0.7205463,1.9239725)(2.8089423,6.5410295)(5.026165,1.816534)
\psline[linewidth=0.04,linestyle=dotted,dotsep=0.1cm](2.8189373,6.4493885)(2.736479,1.7205143)(4.962867,1.6926974)(4.962867,1.831782)(4.935381,1.831782)
% \usefont{T1}{ptm}{m}{n}
\rput(4.7499313,4.485141){$13$ cm}
\psbezier[linewidth=0.027999999](0.78034645,2.0499592)(0.47451097,1.3570333)(1.3971436,0.47512725)(2.5043025,0.41213384)(3.6114616,0.34914044)(5.026165,0.85308695)(5.026165,1.8609792)
\psbezier[linewidth=0.022,linestyle=dashed,dash=0.1cm 0.1cm](5.026165,1.7979861)(4.657112,2.3649256)(3.9805148,2.7450073)(3.0888517,2.7754428)(2.1971886,2.8058784)(1.4586524,2.7428854)(0.7205463,1.8609792)
\psframe[linewidth=0.04,dimen=outer](3.1003997,2.115141)(2.7003999,1.7151409)
% \usefont{T1}{ptm}{m}{n}
\rput(4.009931,2.0051408){$5$ cm}
\pspolygon[linewidth=0.028222222](7.28,1.7148592)(8.54,5.474859)(7.074111,3.09593)
\pspolygon[linewidth=0.028222222](11.245753,2.3025317)(8.571624,5.574859)(7.2880416,1.6889703)
\psline[linewidth=0.022cm,linestyle=dashed,dash=0.1cm 0.1cm](7.074111,3.09593)(11.19227,2.358951)
\psline[linewidth=0.04cm,linestyle=dotted,dotsep=0.1cm](8.56,5.574859)(9.22,1.9948592)
\psline[linewidth=0.024](9.18,2.2348592)(9.4,2.2748594)(9.44,2.0348594)(9.46,2.0348594)
\psline[linewidth=0.04cm](8.82,2.9148593)(8.76,2.6548593)
\psline[linewidth=0.04cm](6.96,2.4148595)(7.3,2.4748592)
\psline[linewidth=0.04cm](8.74,2.0548594)(8.92,1.7948594)
% \usefont{T1}{ptm}{m}{n}
\rput(9.395,1.5848593){$6$ cm}
% \usefont{T1}{ptm}{m}{n}
\rput(9.31,3.8848593){$10$ cm}
% \usefont{T1}{ptm}{m}{n}
\rput(1.0020312,5.3748593){\LARGE\textbf{5.}}
% \usefont{T1}{ptm}{m}{n}
\rput(6.9220314,5.3748593){\LARGE\textbf{6.}}
\pscircle[linewidth=0.027999999,dimen=outer](9.68,-3.4851408){1.9}
\psellipse[linewidth=0.027999999,linestyle=dashed,dash=0.16cm 0.16cm,dimen=outer](9.68,-3.6151407)(1.84,0.23)
\psline[linewidth=0.027999999cm,linestyle=dotted,dotsep=0.1cm](9.74,-3.6251407)(11.52,-3.6251407)
\psdots[dotsize=0.09](9.7,-3.6451406)
% \usefont{T1}{ppl}{m}{n}
\rput(10.499532,-3.1651409){$10$ cm}
% \usefont{T1}{ptm}{m}{n}
\rput{0.6029805}(-0.05992648,-0.0624806){\rput(5.8769474,-5.725771){$6$ cm}}
% \usefont{T1}{ptm}{m}{n}
\rput{-1.0300905}(0.10617459,0.030957164){\rput(1.7449429,-5.8903265){$5$ cm}}
% \usefont{T1}{ptm}{m}{n}
\rput(5.04,-2.8101408){$12$ cm}
\psline[linewidth=0.04cm](6.4,-4.635141)(3.6,-0.8151407)
\psline[linewidth=0.04cm](3.76,-6.5151405)(3.6,-0.79514074)
\psline[linewidth=0.04cm](3.6,-0.79514074)(0.88,-4.5351405)
\psline[linewidth=0.04cm](0.9,-4.5351405)(3.64,-2.7751408)
\psline[linewidth=0.04cm](3.62,-2.7751408)(6.42,-4.655141)
\psline[linewidth=0.04cm](0.86,-4.5351405)(3.76,-6.5351405)
\psline[linewidth=0.04cm](3.76,-6.5351405)(6.42,-4.655141)
\psline[linewidth=0.04cm,linestyle=dashed,dash=0.17638889cm 0.10583334cm](5.26,-5.4551406)(3.58,-0.91514075)
\psline[linewidth=0.04cm](5.16,-5.155141)(5.4,-4.9751406)
\psline[linewidth=0.04cm](5.38,-4.9751406)(5.5,-5.2951407)
% \usefont{T1}{ptm}{m}{n}
\rput(1.3220313,-1.3851408){\LARGE\textbf{7.}}
% \usefont{T1}{ptm}{m}{n}
\rput(7.5220313,-1.4251407){\LARGE\textbf{8.}}
\end{pspicture} 
}

\end{center}
\\
The solid below is made up of a cube and a square based pyramid. Find it's volume and surface area (round off to one decimal place if needed):
\begin{center}
 \scalebox{1} % Change this value to rescale the drawing.
{
\begin{pspicture}(0,-1.8370537)(6.719142,2.2091963)
\psdiamond[linewidth=0.04,dimen=outer,gangle=130.79651](1.6439155,-0.50671875)(1.2616725,1.0826066)
\psdiamond[linewidth=0.04,dimen=outer,gangle=50.0](3.2356775,-0.49835977)(1.27,1.0687643)
\psline[linewidth=0.027999999,linestyle=dashed,dash=0.17638889cm 0.10583334cm](0.85914207,0.46575883)(2.497392,0.74575895)(3.979142,0.49575883)
\psline[linewidth=0.027999999,linestyle=dashed,dash=0.17638889cm 0.10583334cm](0.8815407,-1.1342412)(2.499142,-0.78424114)(4.019142,-1.1642412)
\psline[linewidth=0.04](0.839142,0.43575883)(2.439142,1.9957589)(2.439142,0.23575884)(2.439142,0.17575884)(2.459142,0.19575883)(2.459142,0.21575883)
\psline[linewidth=0.04cm](4.039142,0.45575884)(2.439142,2.0157588)
\psline[linewidth=0.02](4.619142,2.075759)(4.979142,2.075759)(4.999142,-1.1042411)(4.659142,-1.1042411)
% \usefont{T1}{ptm}{m}{n}
\rput(5.914142,0.5057588){$11$ cm}
\usefont{T1}{ptm}{m}{n}
\rput(1.524142,-1.6342412){$5$ cm}
% \usefont{T1}{ptm}{m}{n}
\rput(1.1611733,2.0057588){\textbf{9.}}
\end{pspicture} 
}
\end{center}

}
\end{exercises}
\section{The effect of multiplying a dimension with the factor $k$}
When one or all of the dimensions of a prism or cylinder is multiplied by a constant, the
surface area and volume will change. The new surface area and volume can be calculated by
using the same formulas used in the preceding section.\par
It is possible to see a relationship between the change in dimensions and the resulting change
in surface area and volume. Knowing these relationships can make it easier and quicker to
calculate the new volumes or surface areas of a prism or cylinder when the dimensions are
scaled up or down.\par
An example is an engineer designing a highway bridge. He is interested in the road surface of
the bridge and how many car lanes it can take. He is limited by the amount of weight which
the foundations can support. He might not expect that when he doubles all the dimensions of
the bridge, the overall volume and concrete weight of the bridge will increases $8$ times
instead of $2$ times.\par
Consider a rectangular prism of dimensions $l$, $b$ and $h$. Below we multiply one, two and then
three of its dimensions by a constant factor of $2$ and record the new volume and surface area.\par
\begin{center}
\begin{table}[H]
 \begin{tabular}{|m{5cm}|c|c|}
\hline
Dimensions & 
Volume & 
Surface \\ \hline
Original dimensions 
\begin{center}
\scalebox{0.5} % Change this value to rescale the drawing.
{
\begin{pspicture}(0,-1.1364063)(4.0990624,1.1164062)
\psline[linewidth=0.04cm](0.02,0.8564063)(0.02,-0.02359375)
\psline[linewidth=0.04cm](0.0,0.83640623)(2.32,0.27640626)
\psline[linewidth=0.04cm](2.2794049,-0.5718663)(3.1088746,-0.30507085)
\psline[linewidth=0.027999999cm,linestyle=dashed,dash=0.17638889cm 0.10583334cm](0.01770038,0.004156353)(0.84,0.31640625)
\psline[linewidth=0.04cm](0.009931098,0.8281822)(0.83940095,1.0949776)
\psline[linewidth=0.04cm](2.293427,0.2667113)(3.1228967,0.5335067)
\psline[linewidth=0.027999999cm,linestyle=dashed,dash=0.17638889cm 0.10583334cm](0.84,0.29640624)(3.0943224,-0.3008174)
\psline[linewidth=0.04cm](0.0,-0.00359375)(2.2848525,-0.5676128)
\psline[linewidth=0.04cm](0.8,1.0964062)(3.126553,0.5432085)
\psline[linewidth=0.04cm](2.3,0.29640624)(2.2994049,-0.5718663)
\psline[linewidth=0.027999999cm,linestyle=dashed,dash=0.17638889cm 0.10583334cm](0.84,1.0764062)(0.84,0.31640625)
\psline[linewidth=0.04cm](3.1,0.5364063)(3.1,-0.32359374)
\psline[linewidth=0.02cm,arrowsize=0.05291667cm 2.0,arrowlength=1.4,arrowinset=0.4]{<->}(0.04,-0.26359376)(1.94,-0.76359373)
\psline[linewidth=0.02cm,arrowsize=0.05291667cm 2.0,arrowlength=1.4,arrowinset=0.4]{<->}(2.52,-0.7235938)(3.08,-0.5235937)
\psline[linewidth=0.02cm,arrowsize=0.05291667cm 2.0,arrowlength=1.4,arrowinset=0.4]{<->}(3.32,0.47640625)(3.32,-0.26359376)
% \usefont{T1}{ptm}{m}{n}
\rput(0.8045313,-0.73359376){$l$}
% \usefont{T1}{ptm}{m}{n}
\rput(2.9245312,-0.93359375){$b$}
% \usefont{T1}{ptm}{m}{n}
\rput(3.7245312,0.14640625){$h$}
\end{pspicture} 
}
\end{center}
&
 \begin{aligned}
  $V &= l \times b \times h \\
 &= lbh$
\end{aligned} & 
\begin{aligned} 
 $A&=2[(l\times h) + (l \times b)+ (b \times h)]\\
&= 2(lh + lb + bh)$
\end{aligned} \\ \hline

Multiply one dimension by $2$ 
\begin{center}
\scalebox{0.5} % Change this value to rescale the drawing.
{
\begin{pspicture}(0,-1.5764062)(4.1190624,1.5364063)
\psline[linewidth=0.04cm](0.04,0.41640624)(0.04,-0.46359375)
\psline[linewidth=0.04cm](2.2994049,-1.0118662)(3.1288748,-0.7450709)
\psline[linewidth=0.04cm](0.02,-0.44359374)(2.3048525,-1.0076128)
\psline[linewidth=0.04cm](2.32,-0.14359374)(2.3194048,-1.0118662)
\psline[linewidth=0.04cm](3.12,0.09640625)(3.12,-0.76359373)
\psline[linewidth=0.02cm,arrowsize=0.05291667cm 2.0,arrowlength=1.4,arrowinset=0.4]{<->}(0.06,-0.70359373)(1.96,-1.2035937)
\psline[linewidth=0.02cm,arrowsize=0.05291667cm 2.0,arrowlength=1.4,arrowinset=0.4]{<->}(2.54,-1.1635938)(3.1,-0.9635937)
\psline[linewidth=0.02cm,arrowsize=0.05291667cm 2.0,arrowlength=1.4,arrowinset=0.4]{<->}(3.32,0.89640623)(3.34,-0.70359373)
\usefont{T1}{ptm}{m}{n}
\rput(0.82453126,-1.1735938){$l$}
\usefont{T1}{ptm}{m}{n}
\rput(2.9445312,-1.3735938){$b$}
\usefont{T1}{ptm}{m}{n}
\rput(3.7445312,0.20640625){$h$}
\psline[linewidth=0.04cm](0.04,1.2764063)(0.04,0.39640626)
\psline[linewidth=0.04cm](0.02,1.2564063)(2.34,0.69640625)
\psline[linewidth=0.04cm](2.2994049,-0.15186627)(3.1288748,0.11492915)
\psline[linewidth=0.04cm](0.029931096,1.2481822)(0.8594009,1.5149776)
\psline[linewidth=0.04cm](2.313427,0.6867113)(3.1428967,0.9535067)
\psline[linewidth=0.04cm](0.02,0.41640624)(2.3048525,-0.14761281)
\psline[linewidth=0.04cm](2.32,0.7164062)(2.3194048,-0.15186627)
\psline[linewidth=0.04cm](3.12,0.95640624)(3.12,0.09640625)
\psline[linewidth=0.04cm](0.82,1.5164063)(3.1048524,0.9523872)
\end{pspicture} 
}
\end{center}
& 

\begin{aligned}
  $V &= l \times b \times 2h \\
 &= 2(lbh)\\
&=2V$
\end{aligned} & 
\begin{aligned} 
 $A_1&=2[(l\times 2h) + (l \times b)+ (b \times 2h)]\\
&= 2(2lh + lb + 2bh)$
\end{aligned} \\ \hline

Multiply two dimensions by $2$ 
\begin{center}
\scalebox{0.5} % Change this value to rescale the drawing.
{
\begin{pspicture}(0,-1.8564062)(6.3790627,1.8164062)
\psline[linewidth=0.04cm](2.3,0.13640624)(2.3,-0.74359375)
\psline[linewidth=0.04cm](4.559405,-1.2918663)(5.3888745,-1.0250709)
\psline[linewidth=0.04cm](2.28,-0.7235938)(4.5648527,-1.2876128)
\psline[linewidth=0.04cm](4.58,-0.42359376)(4.579405,-1.2918663)
\psline[linewidth=0.04cm](5.38,-0.18359375)(5.38,-1.0435938)
\psline[linewidth=0.02cm,arrowsize=0.05291667cm 2.0,arrowlength=1.4,arrowinset=0.4]{<->}(0.22,-0.46359375)(4.22,-1.4835937)
\psline[linewidth=0.02cm,arrowsize=0.05291667cm 2.0,arrowlength=1.4,arrowinset=0.4]{<->}(4.8,-1.4435937)(5.36,-1.2435937)
\psline[linewidth=0.02cm,arrowsize=0.05291667cm 2.0,arrowlength=1.4,arrowinset=0.4]{<->}(5.58,0.61640626)(5.6,-0.98359376)
\usefont{T1}{ptm}{m}{n}
\rput(1.8445313,-1.3335937){$l$}
\usefont{T1}{ptm}{m}{n}
\rput(5.204531,-1.6535938){$b$}
\usefont{T1}{ptm}{m}{n}
\rput(6.0045314,-0.07359375){$h$}
\psline[linewidth=0.04cm](2.3,0.99640626)(2.3,0.11640625)
\psline[linewidth=0.04cm](2.28,0.9764063)(4.6,0.41640624)
\psline[linewidth=0.04cm](4.559405,-0.43186626)(5.3888745,-0.16507086)
\psline[linewidth=0.04cm](2.289931,0.9681822)(3.119401,1.2349776)
\psline[linewidth=0.04cm](4.5734267,0.4067113)(5.402897,0.67350674)
\psline[linewidth=0.04cm](2.28,0.13640624)(4.5648527,-0.4276128)
\psline[linewidth=0.04cm](4.58,0.43640625)(4.579405,-0.43186626)
\psline[linewidth=0.04cm](5.38,0.67640626)(5.38,-0.18359375)
\psline[linewidth=0.04cm](3.08,1.2364062)(5.3648524,0.6723872)
\psline[linewidth=0.04cm](0.04,0.69640625)(0.04,-0.18359375)
\psline[linewidth=0.04cm](0.02,-0.16359375)(2.3048525,-0.7276128)
\psline[linewidth=0.04cm](0.04,1.5564063)(0.04,0.67640626)
\psline[linewidth=0.04cm](0.02,1.5364063)(2.34,0.9764063)
\psline[linewidth=0.04cm](0.029931096,1.5281823)(0.8594009,1.7949777)
\psline[linewidth=0.04cm](0.02,0.69640625)(2.3048525,0.13238719)
\psline[linewidth=0.04cm](0.82,1.7964063)(3.1048524,1.2323872)
\end{pspicture} 
}
\end{center}
& 

\begin{aligned}
  $V &= 2l \times b \times 2h \\
 &= 2.2(lbh)\\
&=4V$
\end{aligned} & 
\begin{aligned} 
 $A_2&=2[(2l\times 2h) + (2l \times b)+ (b \times 2h)]\\
&= 2\times 2(2lh + lb + bh)$
\end{aligned} \\ \hline

Multiply three dimensions by $2$ 
\begin{center}
\scalebox{0.5} % Change this value to rescale the drawing.
{
\begin{pspicture}(0,-1.8964063)(7.2390623,1.8764062)
\psline[linewidth=0.04cm](5.339405,-1.2318662)(6.1688747,-0.96507084)
\psline[linewidth=0.04cm](5.36,-0.36359376)(5.359405,-1.2318662)
\psline[linewidth=0.04cm](6.16,-0.12359375)(6.16,-0.98359376)
\psline[linewidth=0.04cm](5.339405,-0.37186626)(6.1688747,-0.10507085)
\psline[linewidth=0.04cm](3.069931,1.0281823)(3.899401,1.2949777)
\psline[linewidth=0.04cm](5.353427,0.4667113)(6.1828966,0.73350674)
\psline[linewidth=0.04cm](5.36,0.49640626)(5.359405,-0.37186626)
\psline[linewidth=0.04cm](6.16,0.73640627)(6.16,-0.12359375)
\psline[linewidth=0.04cm](3.86,1.2964063)(6.1448526,0.7323872)
\psline[linewidth=0.04cm](0.8099311,1.5881822)(1.639401,1.8549776)
\psline[linewidth=0.04cm](1.6,1.8564062)(3.8848524,1.2923872)
\psline[linewidth=0.04cm](2.28,-0.06359375)(2.28,-0.94359374)
\psline[linewidth=0.04cm](4.539405,-1.4918662)(5.3688745,-1.2250708)
\psline[linewidth=0.04cm](2.26,-0.92359376)(4.5448527,-1.4876128)
\psline[linewidth=0.04cm](4.56,-0.62359375)(4.559405,-1.4918662)
\psline[linewidth=0.04cm](5.36,-0.38359374)(5.36,-1.2435937)
\psline[linewidth=0.02cm,arrowsize=0.05291667cm 2.0,arrowlength=1.4,arrowinset=0.4]{<->}(0.2,-0.66359377)(4.2,-1.6835938)
\psline[linewidth=0.02cm,arrowsize=0.05291667cm 2.0,arrowlength=1.4,arrowinset=0.4]{<->}(4.8,-1.6435938)(6.18,-1.1635938)
\usefont{T1}{ptm}{m}{n}
\rput(1.8245312,-1.5335938){$l$}
\usefont{T1}{ptm}{m}{n}
\rput(5.664531,-1.6935937){$b$}
\psline[linewidth=0.04cm](2.28,0.79640627)(2.28,-0.08359375)
\psline[linewidth=0.04cm](2.26,0.7764062)(4.58,0.21640626)
\psline[linewidth=0.04cm](4.539405,-0.6318663)(5.3688745,-0.36507085)
\psline[linewidth=0.04cm](2.269931,0.7681822)(3.099401,1.0349777)
\psline[linewidth=0.04cm](4.5534267,0.2067113)(5.382897,0.47350672)
\psline[linewidth=0.04cm](2.26,-0.06359375)(4.5448527,-0.6276128)
\psline[linewidth=0.04cm](4.56,0.23640625)(4.559405,-0.6318663)
\psline[linewidth=0.04cm](5.36,0.47640625)(5.36,-0.38359374)
\psline[linewidth=0.04cm](3.06,1.0364063)(5.3448524,0.4723872)
\psline[linewidth=0.04cm](0.02,0.49640626)(0.02,-0.38359374)
\psline[linewidth=0.04cm](0.0,-0.36359376)(2.2848525,-0.92761284)
\psline[linewidth=0.04cm](0.02,1.3564062)(0.02,0.47640625)
\psline[linewidth=0.04cm](0.0,1.3364062)(2.32,0.7764062)
\psline[linewidth=0.04cm](0.009931098,1.3281822)(0.83940095,1.5949776)
\psline[linewidth=0.04cm](0.0,0.49640626)(2.2848525,-0.06761281)
\psline[linewidth=0.04cm](0.8,1.5964062)(3.0848525,1.0323871)
\psline[linewidth=0.02cm,arrowsize=0.05291667cm 2.0,arrowlength=1.4,arrowinset=0.4]{<->}(6.4,0.57640624)(6.38,-0.88359374)
\usefont{T1}{ptm}{m}{n}
\rput(6.864531,-0.21359375){$h$}
\end{pspicture} 
}
\end{center}
& 


\begin{aligned}
  $V &= 2l \times 2b \times 2h \\
 &= 8(lbh)\\
&=2^3V$
\end{aligned} & 
\begin{aligned} 
 $A_3&=2[(2l\times 2h) + (2l \times 2b)+ (2b \times 2h)]\\
&= 2(4lh + 4lb + 42bh)\\
&=2.2 \times 2(lh + lb + bh)\\
&=2^2A$
\end{aligned} \\ \hline
 \end{tabular}

\end{table}
\end{center}

In this case we examined the effect of multiplying with a constant factor of $2$. We can express
it more in terms of any common factor $k$:
\begin{wex}{}
 {
Consider a square prism with a height of 4cm and base lengths of $3$ cm.
\begin{center}
\scalebox{1} % Change this value to rescale the drawing.
{
\begin{pspicture}(0,-1.293208)(5.519142,1.2436045)
\psdiamond[linewidth=0.04,dimen=outer,gangle=130.79651](1.6439155,-0.0028731462)(1.2616725,1.0826066)
\psdiamond[linewidth=0.04,dimen=outer,gangle=50.0](3.2356775,0.0054858387)(1.27,1.0687643)
\psline[linewidth=0.027999999](0.85914207,0.94960445)(2.5183952,1.2296045)(4.019142,0.9796044)
\psline[linewidth=0.027999999,linestyle=dashed,dash=0.16cm 0.16cm](0.8815407,-0.63039553)(2.419142,-0.32039556)(4.019142,-0.66039556)
\usefont{T1}{ppl}{m}{n}
\rput(4.824142,0.20960444){$4$ cm}
\usefont{T1}{ppl}{m}{n}
\rput(1.424142,-1.0903956){$3$ cm}
\usefont{T1}{ppl}{m}{n}
\rput(3.624142,-1.0303955){$3$ cm}
\psline[linewidth=0.04cm,linestyle=dashed,dash=0.16cm 0.16cm](2.499142,1.2196045)(2.5191422,0.6796044)
\end{pspicture} 
}
\end{center}

\begin{enumerate}[noitemsep, label=\textbf{\arabic*}. ] 
 \item calculate the original surface area and volume
\item calculate the new surface area and volume if the base length is multiplied by a constant factor of $3$
\item express the new surface area and volume as a factor of the original surface area and volume
\end{enumerate}
}
{
\westep{Calculate the original volume and surface area}
\begin{align*}
V&=l\times b\timesh\\
&=3\times3\times4\\
&= 36\mbox{ cm}^3
\end{align*}
\begin{align*}
A &= 2[(l \times h) + (l \times b) + (b \times h)]\\
&= 2[(3 \times 4) + (3 \times 3) + (3 \times 4)]\\
&= 66 \mbox{ cm}^2
\end{align*}
\westep{Calculate the new volume and surface area}
\begin{align*}
V_n&=l\timesb\timesh\\
&=3.3\times3.3\times4.3\\
&= 324\mbox{ cm}^3
\end{align*}
\begin{align*}
A_n &= 2[(l \times h) + (l \times b) + (b \times h)]\\
&= 2[(3.3 \times 4) + (3.3 \times 3) + (3.3 \times 4)]\\
&= 306 \mbox{ cm}^2
\end{align*}
\westep{express the new surface area and volume as a factor of the original surface area and volume}
\begin{align*}
V&=36\mbox{ cm}^3\\
V_n&= 324\mbox{ cm}^3\\
\therefore V_n&=9V=3^{2}V
\end{align*}
\begin{align*}
A &= 66 \mbox{ cm}^2\\
A_n&= 306\mbox{ cm}^2\\
\therefore A_n&= 4A
\end{align*}
}
\end{wex}

\begin{wex}{}
 {
Prove that if the height of a rectangular prism with dimensions $l$,$b$ and $h$ is multiplied by a
constant value of $k$, the volume will also increase by a factor $k$.
\begin{center}
\scalebox{1} % Change this value to rescale the drawing.
{
\begin{pspicture}(0,-1.1364063)(4.0990624,1.1164062)
\psline[linewidth=0.04cm](0.02,0.8564063)(0.02,-0.02359375)
\psline[linewidth=0.04cm](0.0,0.83640623)(2.32,0.27640626)
\psline[linewidth=0.04cm](2.2794049,-0.5718663)(3.1088746,-0.30507085)
\psline[linewidth=0.027999999cm,linestyle=dashed,dash=0.16cm 0.16cm](0.01770038,0.004156353)(0.84,0.31640625)
\psline[linewidth=0.04cm](0.009931098,0.8281822)(0.83940095,1.0949776)
\psline[linewidth=0.04cm](2.293427,0.2667113)(3.1228967,0.5335067)
\psline[linewidth=0.027999999cm,linestyle=dashed,dash=0.16cm 0.16cm](0.84,0.29640624)(3.0943224,-0.3008174)
\psline[linewidth=0.04cm](0.0,-0.00359375)(2.2848525,-0.5676128)
\psline[linewidth=0.04cm](0.8,1.0964062)(3.126553,0.5432085)
\psline[linewidth=0.04cm](2.3,0.29640624)(2.2994049,-0.5718663)
\psline[linewidth=0.027999999cm,linestyle=dashed,dash=0.16cm 0.16cm](0.84,1.0764062)(0.84,0.31640625)
\psline[linewidth=0.04cm](3.1,0.5364063)(3.1,-0.32359374)
\psline[linewidth=0.02cm,arrowsize=0.05291667cm 2.0,arrowlength=1.4,arrowinset=0.4]{<->}(0.04,-0.26359376)(1.94,-0.76359373)
\psline[linewidth=0.02cm,arrowsize=0.05291667cm 2.0,arrowlength=1.4,arrowinset=0.4]{<->}(2.52,-0.7235938)(3.08,-0.5235937)
\psline[linewidth=0.02cm,arrowsize=0.05291667cm 2.0,arrowlength=1.4,arrowinset=0.4]{<->}(3.32,0.47640625)(3.32,-0.26359376)
\usefont{T1}{ppl}{m}{n}
\rput(0.8045313,-0.73359376){$l$}
\usefont{T1}{ppl}{m}{n}
\rput(2.9245312,-0.93359375){$b$}
\usefont{T1}{ppl}{m}{n}
\rput(3.7245312,0.14640625){$h$}
\end{pspicture} 
}
\end{center}
}
{
\westep{}
We are given the original dimensions $l$,$b$ and $h$\\
therefore the original volume is $V = lbh$\\
\\
The new dimensions are $l$,$b$, and $kh$\\
therefore the new volume is 
\begin{align*}
V_n &= lb(kh)\\
& = k(lbh)\\
&= kV
\end{align*}

It can be seen the new volume $V_n$ is $k$ times the original volume $V$ or increased with a factor
$k$.
}
\end{wex}

\begin{wex}{}
{Consider a cylinder with a radius of $r$ and a height of $h$. Calculate the new volume and surface area (expressed in terms of $r$ and $h$)
if the radius is multiplied by a constant factor of $k$.
\begin{center}
\begin{pspicture}(0,-1.6)(3.5434375,1.6) 
\psellipse[linewidth=0.04,dimen=outer](1.29,-1.09)(1.27,0.51) 
\psellipse[linewidth=0.04,dimen=outer](1.27,1.09)(1.27,0.51) 
\psline[linewidth=0.04cm](0.04,-1.1)(0.02,1.14) 
\psline[linewidth=0.04cm](2.54,1.12)(2.56,-1.1) 
\psline[linewidth=0.04cm,linestyle=dashed,dash=0.16cm 0.16cm](1.3, -1.1)(2.52,-1.1) 
\rput(1.7,-0.95){$r$} 
\rput(2.9,0.25){$h$} 
\end{pspicture} 
\end{center}

}

{
\westep{Calculate the original volume and area of the cylinder in terms of $r$ and $h$}
\begin{align*}
 V&= \pi r^2 \times h\\
\end{align*}
\begin{align*}
A&= \pi r^2 + 2\pi rh
\end{align*}
\westep{Calculate the new volume and area of the cylinder in terms of $r$ and $h$}
\begin{align*}
 V_n&= \pi (kr)^{2} \times h\\
&= \pi k^{2}r^{2} \times h\\
&=k^{2} \times \pi r^{2} h\\
&= k^{2}V
\end{align*}
\begin{align*}
A&= \pi (kr)^{2} + 2\pi (kr)h\\
&= \pi k^{2}r^{2} = 2\pi krh 
\end{align*}
}
\end{wex}

\begin{exercises}{}
 {
\begin{enumerate}[noitemsep, label=\textbf{\arabic*}. ] 
 \item If the height of a prism is doubled, by how much does its volume increase?
\item Describe the change in the volume of a rectangular prism if:
\begin{enumerate}[noitemsep, label=\textbf{\alph*}. ] 
\item length and breadth increase by a constant factor of $3$
\item length, breadth and height are multiplied by a constant factor of $2$
\end{enumerate}
\item Given a prism with a volume of $493$ cm$^{2}$ and a surface area of $6007$ cm$^{2}$, 
find the new surface area and volume for a prism if all dimensions are increased by a constant factor of $4$. 
\end{enumerate}

}
\end{exercises}

\begin{eocexercises}{}
Consider the solids below and answer the questions that follow (rounded to one decimal place, if necessary):
\begin{center}
\scalebox{1} % Change this value to rescale the drawing.
{
\begin{pspicture}(0,-3.9464064)(10.693593,3.9264061)
\psline[linewidth=0.04cm](0.75328124,-3.513594)(1.7532812,-2.533594)
\psline[linewidth=0.04cm](3.753281,-3.493594)(4.733281,-2.533594)
\psline[linewidth=0.04cm](0.7732813,-3.513594)(3.753281,-3.493594)
\psline[linewidth=0.04cm](1.7532812,-2.513594)(4.733281,-2.513594)
\psline[linewidth=0.04cm](4.753281,-1.4735942)(4.733281,-2.573594)
\psline[linewidth=0.04cm](1.7532812,-1.493594)(1.7732812,-2.513594)
\psline[linewidth=0.04cm](1.7732812,-1.513594)(4.753281,-1.493594)
\psline[linewidth=0.04cm](3.753281,-2.433594)(3.753281,-3.493594)
\psline[linewidth=0.04cm](0.75328124,-2.473594)(0.75328124,-3.533594)
\psline[linewidth=0.04cm](0.75328124,-2.4535937)(3.7332811,-2.4535937)
\psline[linewidth=0.04cm](0.7732813,-2.4535937)(1.7532812,-1.513594)
\psline[linewidth=0.04cm](3.753281,-2.4535937)(4.733281,-1.513594)
% \usefont{T1}{ptm}{m}{n}
% \rput(1.3409375,-0.818594){\textbf{1.}}%number for box
% \usefont{T1}{ptm}{m}{n}
\rput(2.1515625,-3.7435937){$5$ cm}
% \usefont{T1}{ptm}{m}{n}
\rput(4.7346873,-3.1435938){$4$ cm}
% \usefont{T1}{ptm}{m}{n}
\rput(5.2,-2.043594){$2$ cm}
\psellipse[linewidth=0.04,dimen=outer](2.203281,1.1064061)(0.99,0.38)
\psellipse[linewidth=0.04,dimen=outer](2.203281,2.446406)(0.99,0.38)
\psline[linewidth=0.04cm](1.233281,2.4064062)(1.233281,1.1464062)
\psline[linewidth=0.04cm](3.173282,2.446406)(3.173282,1.1264061)
\psline[linewidth=0.04cm,linestyle=dashed,dash=0.16cm 0.16cm](2.233281,1.086406)(3.153281,1.1064061)
% \usefont{T1}{ptm}{m}{n}
\rput(2.197657,1.2364061){$4$ cm}
% \usefont{T1}{ptm}{m}{n}
\rput(3.646093,1.8564061){$10$ cm}
% \usefont{T1}{ptm}{m}{n}
% \rput(0.96203125,3.2814062){\textbf{3.}}%number for cylinder
\pstriangle[linewidth=0.04,dimen=outer](7.083281,0.76640624)(2.18,1.72)
\psline[linewidth=0.04cm](7.0732813,2.466406)(9.433281,3.886406)
\psline[linewidth=0.04cm](6.0532813,0.80640626)(9.233281,2.8864062)
\psline[linewidth=0.04cm](9.433281,3.9064062)(10.653281,2.566406)
\psline[linewidth=0.04cm](8.133282,0.7864063)(10.653281,2.5664062)
\psline[linewidth=0.04cm,linestyle=dashed,dash=0.16cm 0.16cm](7.0732813,2.4264064)(7.0532813,0.7864063)
% \usefont{T1}{ptm}{m}{n}
% \rput(6.3378124,3.4014063){\textbf{2.}}%number for prism
% \usefont{T1}{ptm}{m}{n}
\rput(9.888594,1.4764062){$20$ cm}
% \usefont{T1}{ptm}{m}{n}
\rput(7.591875,1.2892188){$3$ cm}
% \usefont{T1}{ptm}{m}{n}
\rput(7.0415626,0.51640624){$8$ cm}
\psline[linewidth=0.04cm](9.433281,3.9064062)(9.233281,2.8664062)
\psline[linewidth=0.04cm](10.653281,2.5664062)(9.193281,2.8664062)
% \usefont{T1}{ptm}{m}{n}
\rput{58.291424}(4.633092,-4.489355){\rput(6.311875,1.9092188){$5$ cm}}
\end{pspicture} 
}
\end{center}
 \begin{enumerate}[noitemsep, label=\textbf{\arabic*}. ] 
  \item Calculate the surface area of each solid
\item Calculate volume of each solid
\item If the dimensions of the solid are increased by a factor of $3$, calculate the new surface area of each solid.
\item If the dimensions of the solid are increased by a factor of $3$, calculate the new volume of each solid.
 \end{enumerate}
Consider the figures below:
\begin{center}
\scalebox{0.7} % Change this value to rescale the drawing.
{
\begin{pspicture}(0,-5.4207597)(13.96,5.741175)
\definecolor{color3715b}{rgb}{0.996078431372549,0.996078431372549,0.996078431372549}
\psline[linewidth=0.028222222](1.5205463,1.1100069)(3.6089423,5.727064)(5.826165,1.0025685)
\psline[linewidth=0.04,linestyle=dotted,dotsep=0.16cm](3.6189375,5.6354227)(3.536479,0.9065488)(5.762867,0.8787319)(5.762867,1.0178165)(5.735381,1.0178165)
% \usefont{T1}{ppl}{m}{n}
\rput(4.409931,3.5911753){\LARGE $10$ cm}
\psbezier[linewidth=0.027999999](1.5803465,1.2359936)(1.274511,0.5430677)(2.1971436,-0.3388383)(3.3043027,-0.40183172)(4.4114614,-0.46482512)(5.826165,0.039121386)(5.826165,1.0470136)
\psbezier[linewidth=0.022,linestyle=dashed,dash=0.16cm 0.16cm](5.826165,0.9840206)(5.457112,1.55096)(4.7805147,1.9310416)(3.888852,1.9614773)(2.9971886,1.991913)(2.2586524,1.9289197)(1.5205463,1.0470136)
\psframe[linewidth=0.04,dimen=outer](3.9003997,1.3011752)(3.5004,0.9011754)
% \usefont{T1}{ppl}{m}{n}
\rput(4.809931,1.1911753){\LARGE $3$ cm}
% \usefont{T1}{ppl}{m}{n}
% \rput(1.5420313,5.5008936){\textbf{1.}}
% \usefont{T1}{ppl}{m}{n}
% \rput(8.442031,5.5008936){\textbf{2.}}
% \usefont{T1}{ppl}{m}{n}
% \rput(0.96203125,-2.3191063){\textbf{3.}}
% \usefont{T1}{ptm}{m}{n}
\rput{0.6029805}(0.004622919,-0.1373882){\rput(13.026947,0.3702635){\LARGE $15$ cm}}
% \usefont{T1}{ptm}{m}{n}
\rput{-1.0300905}(-0.006498412,0.16914946){\rput(9.374943,0.4457077){\LARGE $15$ cm}}
% \usefont{T1}{ptm}{m}{n}
\rput{90.771416}(14.303988,-7.492464){\rput(10.82,3.3058937){\LARGE  $12$ cm}}
\psline[linewidth=0.04cm](13.92,1.7008936)(11.12,5.5208936)
\psline[linewidth=0.04cm](11.28,-0.1791063)(11.12,5.5408936)
\psline[linewidth=0.04cm](11.12,5.5408936)(8.4,1.8008937)
\psline[linewidth=0.04cm](8.42,1.8008937)(11.16,3.5608938)
\psline[linewidth=0.04cm](11.14,3.5608938)(13.94,1.6808937)
\psline[linewidth=0.04cm](8.38,1.8008937)(11.28,-0.1991063)
\psline[linewidth=0.04cm](11.28,-0.1991063)(13.94,1.6808937)
\psdots[dotsize=0.12](11.0,1.8408937)
\psline[linewidth=0.04cm,linestyle=dashed,dash=0.17638889cm 0.10583334cm](11.02,1.8608937)(11.1,5.4208937)
\psline[linewidth=0.04cm,linestyle=dashed,dash=0.17638889cm 0.10583334cm](11.0,1.8208936)(9.82,2.7008936)
\psline[linewidth=0.04cm](10.78,1.9808937)(10.78,2.3408937)
\psline[linewidth=0.04cm](10.76,2.3408937)(11.04,2.1408937)
\rput{14.5046}(-0.70283186,-0.89219034){\pswedge[linewidth=0.04](3.1540546,-3.2075646){2.0675611}{178.8065}{0.0}}
\rput{13.78588}(-0.6714317,-0.8455882){\psellipse[linewidth=0.04,dimen=outer,fillstyle=solid,fillcolor=color3715b](3.1616797,-3.1998694)(2.0922363,0.47717163)}
\psdots[dotsize=0.12](3.1473527,-3.2047362)
\psline[linewidth=0.04cm,linestyle=dashed,dash=0.16cm 0.16cm](3.1332812,-3.2387989)(5.147353,-2.7047362)
% \usefont{T1}{ptm}{m}{n}
\rput{14.285164}(-0.57707244,-1.0637248){\rput(3.9257896,-2.8347287){\LARGE $4$ cm}}
\end{pspicture} 
}
\end{center}
\begin{enumerate}[noitemsep, label=\textbf{\arabic*}. ] 
\setcounter{enumi}{4}
 \item Calculate the surface area of each of the solids
\item Calculate the volume of each of the solids

\end{enumerate}
\begin{enumerate}[noitemsep, label=\textbf{\arabic*}. ] 
\setcounter{enumi}{6}
\item Calculate the volume and surface area of the solid below (correct to 1 decimal place):
\end{enumerate}
\begin{center}
\scalebox{1} % Change this value to rescale the drawing.
{
\begin{pspicture}(0,-1.4992187)(4.081249,1.5192188)
\psellipse[linewidth=0.04,dimen=outer](2.2743747,-1.1192187)(0.99,0.38)
\psellipse[linewidth=0.04,dimen=outer](2.2743747,0.22078115)(0.99,0.38)
\psline[linewidth=0.04cm](1.3043747,0.18078145)(1.3043747,-1.0792186)
\psline[linewidth=0.04cm](3.2443757,0.22078115)(3.2443757,-1.0992187)
\psline[linewidth=0.04cm,linestyle=dashed,dash=0.16cm 0.16cm](2.3043747,-1.1392188)(3.2243748,-1.1192187)
\usefont{T1}{ppl}{m}{n}
\rput(2.518282,-0.96921873){$40$}
\usefont{T1}{ppl}{m}{n}
\rput(3.596718,-0.44921875){$50$}
\psline[linewidth=0.04](1.31,0.27921876)(2.33,1.4992187)(3.23,0.29921874)(3.23,0.27921876)(3.23,0.27921876)
\psline[linewidth=0.027999999](0.93,1.4592187)(0.67,1.4592187)(0.67,0.29921874)(0.93,0.29921874)
\usefont{T1}{ppl}{m}{n}
\rput(0.38453126,0.88921875){$30$}
\end{pspicture} 
}

\end{center}

}
\end{eocexercises}

