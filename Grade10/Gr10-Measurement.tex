
\chapter{Measurements  }  
This chapter examines the surface areas of two dimensional objects and
volumes of three dimensional objects, otherwise known as solids. In
order to work with these objects, you need to know how to calculate
the surface area and perimeter of the two dimensional shapes below.
\chapterstartvideo{VMcmk}
\section{Area of a polygon }

\Definition{Area}{Area is the two dimensional space inside the
  boundary of a flat object. It is measured in square units.}

\begin{table}[H]
  \newcolumntype{C}{>{\centering\arraybackslash} m{1.5in} }
  \newcolumntype{D}{>{\centering\arraybackslash} m{2in} }
  \begin{tabular}{|C|D|C|}
    \hline
    \textbf{Name} & \textbf{Shape} & \textbf{Formula} \\ \hline
    Square &
    \begin{center}
      \scalebox{0.9}{
        \begin{pspicture}(-2,-0.5)(5,2)
          \pspolygon(0,0)(0,1)(1,1)(1,0)
          \pspolygon(0,0)(0,0.15)(0.15,0.15)(0.15,0)
          \rput(-0.25,0.5){$s$}
          \rput(0.5,-0.25){$s$}
        \end{pspicture}}
    \end{center}
    & $\mbox{Area} = s^2$ \\ \hline

    Rectangle &
    \begin{center}
      \scalebox{0.9}{
        \begin{pspicture}(-2,-0.5)(5,2)
          \pspolygon(0,0)(0,1)(2.5,1)(2.5,0)
          \pspolygon(0,0)(0,0.15)(0.15,0.15)(0.15,0)
          \rput(2.75,0.5){$h$}
          \rput(1.25,-0.25){$b$}
        \end{pspicture}}
    \end{center}
    & $\mbox{Area} =  b \times h $ \\ \hline

    Triangle &
    \begin{center} 
      \scalebox{0.9}{
        \begin{pspicture}(0,-0.5)(5,2)
          \pspolygon(0,0)(2,1)(3,0)
          \psline[linewidth=0.02cm,linestyle=dashed,dash=0.16cm 0.16cm](2,1)(2,0)
          \pspolygon(2,0)(2,0.15)(2.15,0.15)(2.15,0)
          \rput(1.8,0.5){$h$}
          \rput(1.5,-0.25){$b$}
        \end{pspicture}}
    \end{center}
    & $\mbox{Area} = \dfrac{1}{2} b \times h$ \\ \hline
  \end{tabular}
  % NOTE: table manually split in half for formatting purposes (no time to learn about/make long tables/supertabular)
\end{table}

\begin{table}[H]
  \newcolumntype{C}{>{\centering\arraybackslash} m{1.5in} }
  \newcolumntype{D}{>{\centering\arraybackslash} m{2in} }
  \begin{tabular}{|C|D|C|}
    \hline
    \textbf{Name} & \textbf{Shape} & \textbf{Formula} \\ \hline
    % \caption{Remember that surface area is measured in square units, for example, $cm^2$, $m^2$ or $mm^2$.}

    Trapezium &
    \begin{center}
      \scalebox{0.9}{
        \begin{pspicture}(-2,-0.5)(5,2)
          \pspolygon(-1.5,0)(0,1)(2,1)(3,0)
          \psline[linewidth=0.02cm,linestyle=dashed,dash=0.16cm 0.16cm](2,1)(2,0)
          \pspolygon(2,0)(2,0.15)(2.15,0.15)(2.15,0)
          \rput(1.8,0.5){$h$}
          \rput(0.75,-0.25){$b$}
          \rput(0.75,1.25){$a$}
        \end{pspicture}}
    \end{center}
    & $\mbox{Area} = \dfrac{1}{2} (a + b) \times h $ \\ \hline 

    Parallelogram &
    \begin{center}
      \scalebox{0.9}{
        \begin{pspicture}(-2,-0.5)(5,2)
          \pspolygon(-1,0)(0,1)(3,1)(2,0)
          \psline[linewidth=0.02cm,linestyle=dashed,dash=0.16cm 0.16cm](0.5,1)(0.5,0)
          \pspolygon(0.5,0)(0.5,0.15)(0.65,0.15)(0.65,0)
          \rput(0.3,0.5){$h$}
          \rput(0.7,-0.25){$b$}
        \end{pspicture}}
    \end{center}
    & $\mbox{Area} =  b \times h $ \\ \hline

    Circle &
    \begin{center}
      \scalebox{0.9}{
        \begin{pspicture}(-2,-0.5)(5,2)
          \pscircle[dimen=outer](0.5,0.5){0.7}
          \psline[linestyle=dashed,dash=0.1cm 0.1cm](0.5,0.5)(1.2,0.5)
          \psdots[dotsize=0.08](0.5,0.5)
          \rput(0.85,0.75){$r$}
        \end{pspicture}}
    \end{center} &
%    \begin{aligned}$
%      &\mbox{Area} = \pi r^2 \\
%      &(\mbox{Circumference} = 2\pi r)$
%    \end{aligned}
    $$\mbox{Area} = \pi r^2$$\vspace{-24pt}
    $$(\mbox{Circumference} = 2\pi r)$$
    \\ \hline
  \end{tabular}
  % \caption{Remember that surface area is measured in square units, for example, $cm^2$, $m^2$ or $mm^2$.}
\end{table}

\Note{Remember that area is measured in square units, for example, cm$^2$, m$^2$ or mm$^2$.}     

\mindsetvid{Area and perimeter}{VMdxu}%Khan Academy
\par
\mindsetvid{Area of a circle}{VMdya}%Khan academy

\begin{wex}{Finding the area of a polygon}
{Find the area of the following parallelogram:\\
  \begin{center}
\scalebox{1} % Change this value to rescale the drawing.
{
\begin{pspicture}(0,-1.403125)(5.6790624,1.403125)
\pspolygon[linewidth=0.028222222](0.31,-1.0003124)(1.81,0.9996875)(5.31,0.9996875)(3.81,-1.0003124)
\psline[linewidth=0.014111111cm,linestyle=dashed,dash=0.16cm 0.16cm](1.81,0.9996875)(1.81,-1.0003124)
\pspolygon[linewidth=0.028222222](1.81,-1.0003124)(1.81,-0.7003125)(2.11,-0.7003125)(2.11,-1.0003124)
% \usefont{T1}{ppl}{m}{n}
\rput(0.28453124,-1.2003125){$A$}
% \usefont{T1}{ppl}{m}{n}
\rput(1.7845312,1.1996875){$B$}
% \usefont{T1}{ppl}{m}{n}
\rput(5.284531,1.1996875){$C$}
% \usefont{T1}{ppl}{m}{n}
\rput(3.7845314,-1.2003125){$D$}
% \usefont{T1}{ppl}{m}{n}
\rput(1.7845312,-1.2003125){$E$}
% \usefont{T1}{ppl}{m}{n}
\rput(0.55,0){$5$ mm}
% \usefont{T1}{ppl}{m}{n}
\rput(1.0345312,-1.2003125){$3$ mm}
% \usefont{T1}{ppl}{m}{n}
\rput(2.7845314,-1.2003125){$4$ mm}
\psline[linewidth=0.04](4.287036,-0.07316559)(4.5628867,0.011133511)(4.550078,-0.28859293)
\psline[linewidth=0.04](1.0870358,0.3068344)(1.3628865,0.39113352)(1.3500777,0.09140708)
\psline[linewidth=0.04](4.427036,0.08683441)(4.7028866,0.1711335)(4.690078,-0.12859292)
\psline[linewidth=0.04](1.2270358,0.4868344)(1.5028865,0.5711335)(1.4900777,0.27140707)
\psline[linewidth=0.04](3.1252131,1.1690861)(3.3666022,1.0111897)(3.1281846,0.8290991)
\psline[linewidth=0.04](2.245213,-0.83091384)(2.4866023,-0.98881024)(2.2481847,-1.1709008)
\end{pspicture} 
}
\end{center}
}{
  \westep{Find the height $ BE$ of the parallelogram using the Theorem of Pythagoras}
\begin{align*}
  AB^2 &= BE^2 + AE^2\\
  \therefore BE^2 &= AB^2 - AE^2\\
  &= 5^2 - 3^2\\
  &= 16\\
  \therefore BE &= 4\mbox{ mm}
\end{align*}
    
    \westep{Find the area using the formula for a parallelogram}
\begin{align*}
  \text{Area} &= b \times h\\
  &= AD \times BE \\
  &= 7 \times 4\\
  &= 28 \mbox{ mm}^2
\end{align*}
}
\end{wex}


\begin{exercises}{}
{Find the areas of each of the polygons below:
\begin{center}
\scalebox{0.9}{
\begin{pspicture}(0,-4.758281)(14.921875,4.758281)
\psline[linewidth=0.04cm](0.381875,1.7882812)(4.381875,1.7882812) 
\psline[linewidth=0.04cm](4.241875,1.7682812)(4.241875,1.7682812) 
\psline[linewidth=0.04cm,linestyle=dashed,dash=0.16cm 0.16cm](2.361875,3.75)(2.361875,2.8282812) 
\psline[linewidth=0.04cm](0.381875,1.7882812)(2.341875,3.7882812) 
\psline[linewidth=0.04cm](2.361875,3.7882812)(4.341875,1.8282813) 
% \usefont{T1}{ptm}{m}{n} 
\rput(2.4517188,2.5782812){$5$ cm} 
\psline[linewidth=0.04cm,linestyle=dashed,dash=0.16cm 0.16cm](2.361875,2.3482811)(2.361875,1.8) 
% \usefont{T1}{ptm}{m}{n} 
\rput(2.4365625,1.4182812){$10$ cm} 
% \usefont{T1}{ptm}{m}{n} 
\rput(0.16203125,3.6182814){\textbf{1.}} 
\psframe[linewidth=0.04,dimen=outer](9.141875,3.7782812)(5.141875,1.7782812) 
% \usefont{T1}{ptm}{m}{n} 
\rput(4.72375,3.5982811){\textbf{2.}} 
\psframe[linewidth=0.04,dimen=outer](5.471875,3.7782812)(5.151875,3.4582813) 
\psline[linewidth=0.04cm](7.121875,3.9282813)(7.321875,3.7682812) 
\psline[linewidth=0.04cm](7.1272163,3.6020272)(7.3165336,3.7745354) 
\psline[linewidth=0.04cm](7.121875,1.9482813)(7.321875,1.7882812) 
\psline[linewidth=0.04cm](7.1272163,1.6220272)(7.3165336,1.7945354)
\psline[linewidth=0.04cm](4.995617,2.7704508)(5.161965,2.965203)
\psline[linewidth=0.04cm](5.321874,2.7652996)(5.1555424,2.9600654) 
\psline[linewidth=0.04cm](4.995617,2.570451)(5.161965,2.765203) 
\psline[linewidth=0.04cm](5.321874,2.5652997)(5.1555424,2.7600653)
\psline[linewidth=0.04cm](8.955617,2.7704508)(9.121964,2.965203) 
\psline[linewidth=0.04cm](9.281874,2.7652996)(9.115542,2.9600654)
\psline[linewidth=0.04cm](8.955617,2.590451)(9.121964,2.785203) 
\psline[linewidth=0.04cm](9.281874,2.5852997)(9.115542,2.7800653)
%  \usefont{T1}{ptm}{m}{n} 
\rput(9.711719,2.7182813){$5$ cm}
%  \usefont{T1}{ptm}{m}{n} 
\rput(7.1965623,1.4182812){$10$ cm}
\psline[linewidth=0.04cm,linestyle=dashed,dash=0.16cm 0.16cm](10.901875,2.7482812)(14.901875,2.7482812) \pscircle[linewidth=0.04,dimen=outer](12.911875,2.7582812){2.0} 
\psdots[dotsize=0.16](12.901875,2.7482812)
%  \usefont{T1}{ptm}{m}{n} 
\rput(12.876562,2.4182813){$10$ cm} 
% \usefont{T1}{ptm}{m}{n}
\rput(10.631406,3.5982811){\textbf{3.}}
%  \usefont{T1}{ptm}{m}{n}
\rput(0.19140625,0.19828124){\textbf{4.}} 
% \usefont{T1}{ptm}{m}{n}
\rput(5.5117188,-0.8017188){$5$ cm}
%  \usefont{T1}{ptm}{m}{n}
\rput(2.5340624,0.29828125){$7$ cm}
\psline[linewidth=0.04cm](2.1600447,-1.3315424)(2.358717,-1.4931878) 
\psline[linewidth=0.04cm](2.162693,-1.6578295)(2.3534274,-1.4868897) 
\psline[linewidth=0.04cm](1.9601017,-1.3363228)(2.1587741,-1.4979682) 
\psline[linewidth=0.04cm](1.9627501,-1.6626099)(2.1534846,-1.4916701) 
\psline[linewidth=0.04cm](0.181875,-1.4917188)(4.181875,-1.4917188) 
\psline[linewidth=0.04cm](1.401875,0.04828125)(5.401875,0.04828125) 
\psline[linewidth=0.04cm](3.3800445,0.2084576)(3.578717,0.046812218)
\psline[linewidth=0.04cm](3.382693,-0.11782951)(3.5734274,0.053110242) 
\psline[linewidth=0.04cm](3.1801016,0.20367722)(3.3787742,0.042031832) 
\psline[linewidth=0.04cm](3.18275,-0.1226099)(3.3734844,0.048329853) 
\psline[linewidth=0.04cm](4.191091,-1.5022907)(5.394185,0.045147188) 
\psline[linewidth=0.04cm](4.621163,-0.6958544)(4.8714867,-0.6416498)
\psline[linewidth=0.04cm](4.877642,-0.89756864)(4.863264,-0.64184755) 
\psline[linewidth=0.04cm](0.21109095,-1.4822907)(1.4141852,0.065147184) 
\psline[linewidth=0.04cm](0.641163,-0.67585444)(0.89148647,-0.6216498) 
\psline[linewidth=0.04cm](0.8976424,-0.87756866)(0.8832641,-0.62184757)
\psline[linewidth=0.04cm,linestyle=dashed,dash=0.16cm 0.16cm](4.161875,-1.4917188)(4.161875,0.04828125) \psframe[linewidth=0.04,dimen=outer](4.471875,0.05828125)(4.151875,-0.26171875)
%  \usefont{T1}{ptm}{m}{n} 
\rput(4.7753124,0.27828124){$3$ cm} 
\psline[linewidth=0.04cm](6.021875,-1.4317187)(10.021875,-1.4317187)
\psline[linewidth=0.04cm](9.821875,-1.4517188)(9.821875,-1.4517188) 
% \usefont{T1}{ptm}{m}{n} 
\rput(7.5165625,-1.7217188){$12$ cm} 
\psline[linewidth=0.04cm](8.837452,0.20483598)(10.026298,-1.4282734) 
\psline[linewidth=0.04cm](8.821875,0.20828125)(6.021875,-1.4317187)
\psline[linewidth=0.04cm,linestyle=dashed,dash=0.16cm 0.16cm](8.821875,0.18828125)(8.821875,-1.4117187) \psframe[linewidth=0.04,dimen=outer](8.851875,-1.1217188)(8.531875,-1.4417187) 
% \usefont{T1}{ptm}{m}{n} 
\rput(9.217969,-1.6817187){$8$ cm}
%  \usefont{T1}{ptm}{m}{n} 
\rput(9.936563,-0.44171876){$10$ cm}
%  \usefont{T1}{ptm}{m}{n} 
\rput(6.2414064,0.23828125){\textbf{5.}} 
\psline[linewidth=0.04cm](11.561875,-1.3517188)(13.581875,-1.3517188) 
\psline[linewidth=0.04cm](11.564406,-1.3656596)(12.099343,0.58222204) 
\psline[linewidth=0.04cm](12.107254,0.5571398)(13.576495,-1.3405774) 
\psline[linewidth=0.04cm](11.621875,-0.49171874)(11.941875,-0.59171873) 
\psline[linewidth=0.04cm](11.681875,-0.35171875)(12.001875,-0.45171875) 
\psline[linewidth=0.04cm](12.161875,-1.1917187)(12.161875,-1.5317187) 
\psline[linewidth=0.04cm](12.161875,-1.4317187)(12.161875,-1.4317187) 
\psline[linewidth=0.04cm](12.321875,-1.1917187)(12.321875,-1.5317187) 
% \usefont{T1}{ptm}{m}{n} 
\rput(11.094063,0.23828125){\textbf{6.}} 
% \usefont{T1}{ptm}{m}{n} 
\rput(12.7517185,-1.6417187){$5$ cm}
%  \usefont{T1}{ptm}{m}{n} 
\rput(13.181562,-0.18171875){$6$ cm}
\pstriangle[linewidth=0.04,dimen=outer](2.031875,-4.491719)(3.4,2.4)
%\pstriangle[linewidth=0.04,dimen=outer](3.181875,-2.6717188)(0.0,0.0) 
%\pstriangle[linewidth=0.04,dimen=outer](3.161875,-2.7517188)(0.0,0.0) 
%\pstriangle[linewidth=0.04,dimen=outer](3.001875,-2.5317187)(0.0,0.0) 
% \usefont{T1}{ptm}{m}{n} 
\rput(0.23140626,-2.2617188){\textbf{7.}} 
\psline[linewidth=0.04cm](1.901875,-4.331719)(1.901875,-4.6517186) 
\psline[linewidth=0.04cm](1.101875,-3.1117187)(1.341875,-3.3317187)
\psline[linewidth=0.04cm](2.6844237,-3.3744361)(2.9193263,-3.1490014)
%  \usefont{T1}{ptm}{m}{n} 
\rput(3.6565626,-3.1417189){$10$ cm}
\pspolygon[linewidth=0.04](5.201875,-4.371719)(9.201875,-4.371719)(8.181875,-2.6717188)(6.181875,-2.6717188)(6.001875,-2.9717188) \psline[linewidth=0.04cm](6.161875,-2.6717188)(6.161875,-2.6717188) 
\psline[linewidth=0.04cm](6.161875,-2.6717188)(6.161875,-2.6717188)
\psline[linewidth=0.04cm,linestyle=dashed,dash=0.16cm 0.16cm](6.161875,-2.6717188)(6.161875,-4.351719) 
\psline[linewidth=0.04cm](7.106746,-2.5312762)(7.309965,-2.6871674)
\psline[linewidth=0.04cm](7.118735,-2.8573537)(7.3044972,-2.6810234)
\psline[linewidth=0.04cm](7.106746,-4.211276)(7.309965,-4.3671675) 
\psline[linewidth=0.04cm](7.118735,-4.5373535)(7.3044972,-4.3610234) 
\psframe[linewidth=0.04,dimen=outer](6.471875,-4.061719)(6.151875,-4.3817186)
%  \usefont{T1}{ptm}{m}{n}
\rput(5.3565626,-2.9817188){$15$ cm} 
% \usefont{T1}{ptm}{m}{n} 
\rput(5.5398436,-4.601719){$9$ cm} 
% \usefont{T1}{ptm}{m}{n} 
\rput(7.6565623,-2.4017189){$16$ cm} 
% \usefont{T1}{ptm}{m}{n}
\rput(7.9285936,-4.601719){$21$ cm} 
% \usefont{T1}{ptm}{m}{n} 
\rput(5.08375,-2.2617188){\textbf{8.}}
\end{pspicture}
}
\end{center}
\practiceinfo
\begin{tabularx}{\textwidth}{ XX }
(1 - 8.) 00hi&
\end{tabularx}
}
\end{exercises}


\section{Right prisms and cylinders }

\Definition{Right prism}{A right prism is a geometric solid that has a
  polygon as its base and vertical sides perpendicular to the
  base. The base and top surface are the same shape and size. It is
  called a ``right'' prism because the angles between the base and
  sides are right angles.}

A triangular prism has a triangle as its base, a rectangular prism has
a rectangle as its base, and a cube is a rectangular prism with all
its sides of equal length.
A cylinder is another type of right prism which has a circle as its
base. Examples of right prisms are given below: a rectangular prism, a
cube, a triangular prism and a cylinder.
\mindsetvid{Different viewpoints}{VMcph}
\par 
\setcounter{subfigure}{0}
\begin{figure}[H]
    \begin{center}
	%% Rectangular Prism
	\begin{pspicture}(-2,-3.5)(3,1.5)
	    \psset{yunit=0.9,xunit=0.9}
	    \psset{Alpha=60,Beta=30}
	    \pstThreeDBox[hiddenLine](-1,1,2)(0,0,1)(2,0,0)(0,3.5,0)
	    \psset{fillcolor=lightgray,fillstyle=solid,opacity=0.5,linestyle=none,dotstyle=none}
	    \pstThreeDSquare(-1,1,2)(2,0,0)(0,3.5,0)
	    
	\end{pspicture}
\hspace{10pt}
	%% Square Prism
	\begin{pspicture}(-2,-3.5)(3,1.5)
	    \psset{yunit=0.9,xunit=0.9}
	    \psset{Alpha=60,Beta=30}
	    {\psset{fillcolor=lightgray,fillstyle=solid,opacity=0.5,linestyle=none,dotstyle=}
	    \pstThreeDSquare(-1,1,2)(2.5,0,0)(0,2.5,0)}
	    \pstThreeDBox[hiddenLine](-1,1,2)(0,0,2.5)(2.5,0,0)(0,2.5,0)
	\end{pspicture}
\hspace{10pt}
	%% Triangular Prism
	\begin{pspicture}(-2,-3.5)(3,1.5)
	    \psset{yunit=0.9,xunit=0.9}
	    \psset{Alpha=60,Beta=30}
\psSolid[object=prisme,action=draw,axe=0 0 1,base=-0.5 -0.5 0.5 -0.5 0 0.5,h=1.0]
{\psset{fillcolor=lightgray,fillstyle=solid,opacity=0.5,linestyle=solid,dotstyle=}
	    \psSolid[object=face,base=-0.5 -0.5 0.5 -0.5 0 0.5](0,0,0)}
% 	    \psset{viewpoint=8 12 8}

	\end{pspicture}
\hspace{10pt}
	%% Cylindrical Prism
	\begin{pspicture}(-2,-3.5)(3,1.5)
	    \psset{yunit=0.9,xunit=0.9}
	    \psset{Alpha=60,Beta=30}
	    \psellipse[fillcolor=white,fillstyle=solid](0,-3)(1.0,0.5)
	    \psframe[linestyle=none,fillcolor=white,fillstyle=solid](-1,-3)(1,0)
	    \psellipse[fillcolor=lightgray,opacity=0.5,fillstyle=solid,linestyle=dashed](0,-3)(1.0,0.5)
	    \psellipse[fillstyle=none](0,-0)(1.0,0.5)
	    \psline(-1.0,-3)(-1.0,-0)
	    \psline(1.0,-3)(1.0,-0)
	\end{pspicture}

	\vspace{0.75cm}
% 	\begin{caption*}{}\end{caption}
    \end{center}
\end{figure}   

\subsection{Surface area of prisms and cylinders}

\Definition{Surface area}{Surface area is the total area of the
  exposed or outer surfaces of a prism.}

This is easier to understand if we imagine the prism to be a cardboard
box that we can unfold.  A solid that is unfolded like this is called
a net. When a prism is unfolded into a net, we can clearly see each of
its faces. In order to calculate the surface area of the prism, we can
then simply calculate the area of each face, and add them all
together.

\mindsetvid{Triangular prisms}{VMcre}

For example, when a triangular prism is unfolded into a net, we can
see that it has two faces that are triangles and three faces that are
rectangles. To calculate the surface area of the prism, we find the
area of each triangle and each rectangle, and add them together.

In the case of a cylinder the top and bottom faces are circles and the
curved surface flattens into a rectangle with a length that is equal
to the circumference of the circular base. To calculate the surface
area we therefore find the area of the two circles and the rectangle
and add them together.

\Note{Remember that surface area is measured in square units, for example, cm$^2$, m$^2$ or mm$^2$.}

Below are examples of right prisms that have been unfolded into nets:

\begin{figure}[H]
 \begin{caption*}{\textbf{Rectangular prism}}\end{caption*}
   \begin{center}
	%% Rectangular Prism
    \scalebox{0.8}{
        \begin{pspicture}(-2,-3.5)(3,1.5)
	    \psset{yunit=0.9,xunit=0.9}
	    \psset{Alpha=60,Beta=30}
	    {\psset{fillcolor=lightgray,fillstyle=solid,opacity=0.5,linestyle=solid,dotstyle=}
	    \pstThreeDSquare(-1,1,2)(2,0,0)(0,3.5,0)}
	    \pstThreeDBox[hiddenLine](-1,1,2)(0,0,1)(2,0,0)(0,3.5,0)
	\end{pspicture}}

	%% Rectangular Prism Unfolded
    \scalebox{0.8}{
        \begin{pspicture}(-2,-3.5)(3,1.5)
	    \psset{yunit=0.9,xunit=0.9}
	    \psset{Alpha=60,Beta=30}
	    {\psset{fillcolor=lightgray,fillstyle=solid,opacity=0.5,linestyle=solid,dotstyle=}
	    \pstThreeDSquare(-1,1,2)(2,0,0)(0,3.5,0)}
	    \pstThreeDSquare[fillcolor=white,fillstyle=solid,opacity=0.5](-2,1,2)(1,0,0)(0,3.5,0)
	    \pstThreeDSquare[fillcolor=white,fillstyle=solid,opacity=0.5](1,1,2)(1,0,0)(0,3.5,0)
	    \pstThreeDSquare[fillcolor=white,fillstyle=solid,opacity=0.5](2,1,2)(2,0,0)(0,3.5,0)
	    \pstThreeDSquare[fillcolor=white,fillstyle=solid,opacity=0.5](-1,4.5,2)(2,0,0)(0,1,0)
	    \pstThreeDSquare[fillcolor=white,fillstyle=solid,opacity=0.5](-1,0,2)(2,0,0)(0,1,0)
	\end{pspicture}}

\vspace{10pt}

	%% Rectangular Prism Unfolded Birdseye
\scalebox{0.8}{ 
	\begin{pspicture}(-2,-3.5)(3,1.5)
	    \psset{yunit=0.9,xunit=0.9}
	    \psset{Alpha=90,Beta=90}
	    {\psset{fillcolor=lightgray,fillstyle=solid,opacity=0.5,linestyle=solid,dotstyle=}
	    \pstThreeDSquare(-1,1,2)(2,0,0)(0,3.5,0)}
	    \pstThreeDSquare[fillcolor=white,fillstyle=solid,opacity=0.5](-2,1,2)(1,0,0)(0,3.5,0)
	    \pstThreeDSquare[fillcolor=white,fillstyle=solid,opacity=0.5](1,1,2)(1,0,0)(0,3.5,0)
	    \pstThreeDSquare[fillcolor=white,fillstyle=solid,opacity=0.5](2,1,2)(2,0,0)(0,3.5,0)
	    \pstThreeDSquare[fillcolor=white,fillstyle=solid,opacity=0.5](-1,4.5,2)(2,0,0)(0,1,0)
	    \pstThreeDSquare[fillcolor=white,fillstyle=solid,opacity=0.5](-1,0,2)(2,0,0)(0,1,0)
	\end{pspicture}}
\newline
    \end{center}
\end{figure}   

A rectangular prism unfolded into a net is made up of six rectangles. 


\begin{figure}[H]
 \begin{caption*}{\textbf{Cube}}\end{caption*}
    \begin{center}

	%% Square Prism
    \scalebox{0.8}{ 
        \begin{pspicture}(-2,-3.5)(3,1.5)
            \psset{yunit=0.9,xunit=0.9}
	    \psset{Alpha=60,Beta=30}
	    {\psset{fillcolor=lightgray,fillstyle=solid,opacity=0.5,linestyle=none,dotstyle=}
	    \pstThreeDSquare(-1,1,2)(2.5,0,0)(0,2.5,0)}
	    \pstThreeDBox[hiddenLine](-1,1,2)(0,0,2.5)(2.5,0,0)(0,2.5,0)
	\end{pspicture}}

	%% Square Prism Unfolded
    \scalebox{0.8}{
	\begin{pspicture}(-2,-3.5)(3,1.5)
	    \psset{yunit=0.9,xunit=0.9}
	    \psset{Alpha=60,Beta=30}
	    {\psset{fillcolor=lightgray,fillstyle=solid,opacity=0.5,linestyle=solid,dotstyle=}
	    \pstThreeDSquare(-1,1,2)(2.5,0,0)(0,2.5,0)}
	    \pstThreeDSquare[fillcolor=white,fillstyle=solid,opacity=0.5](-3.5,1,2)(2.5,0,0)(0,2.5,0)
	    \pstThreeDSquare[fillcolor=white,fillstyle=solid,opacity=0.5](-1,3.5,2)(2.5,0,0)(0,2.5,0)
	    \pstThreeDSquare[fillcolor=white,fillstyle=solid,opacity=0.5](1.5,1,2)(2.5,0,0)(0,2.5,0)
	    \pstThreeDSquare[fillcolor=white,fillstyle=solid,opacity=0.5](-1,-1.5,2)(2.5,0,0)(0,2.5,0)
	    \pstThreeDSquare[fillcolor=white,fillstyle=solid,opacity=0.5](4,1,2)(2.5,0,0)(0,2.5,0)
	\end{pspicture}}

	%% Square Prism Unfolded Birdseye
    \scalebox{0.8}{
	\begin{pspicture}(-2,-3.5)(3,1.5)
	    \psset{yunit=0.9,xunit=0.9}
	    \psset{Alpha=0,Beta=90}
	    {\psset{fillcolor=lightgray,fillstyle=solid,opacity=0.5,linestyle=solid,dotstyle=}
	    \pstThreeDSquare(-1,1,2)(2.5,0,0)(0,2.5,0)}
	    \pstThreeDSquare[fillcolor=white,fillstyle=solid,opacity=0.5](-3.5,1,2)(2.5,0,0)(0,2.5,0)
	    \pstThreeDSquare[fillcolor=white,fillstyle=solid,opacity=0.5](-1,3.5,2)(2.5,0,0)(0,2.5,0)
	    \pstThreeDSquare[fillcolor=white,fillstyle=solid,opacity=0.5](1.5,1,2)(2.5,0,0)(0,2.5,0)
	    \pstThreeDSquare[fillcolor=white,fillstyle=solid,opacity=0.5](-1,-1.5,2)(2.5,0,0)(0,2.5,0)
	    \pstThreeDSquare[fillcolor=white,fillstyle=solid,opacity=0.5](4,1,2)(2.5,0,0)(0,2.5,0)
	\end{pspicture}}

    \end{center}
\end{figure}   

A cube unfolded into a net is made up of six identical squares.

\begin{figure}[H]
    \begin{caption*}{\textbf{Triangular prism}}\end{caption*}
    \begin{center}
	%% Triangular Prism
    \scalebox{0.8}{
	\begin{pspicture}(-1.5,-1.5)(1,1)
	    \psset{yunit=0.9,xunit=0.9}
% 	    \psset{Alpha=60,Beta=30}
% 	    \psset{viewpoint=0 0 200}
	    \psSolid[object=face,fillcolor=lightgray,opacity=0.5,base=-0.5 -0.5 0.5 -0.5 0 0.5](0,0,0)
	    \psSolid[object=prisme,action=draw,axe=0 0 1,base=-0.5 -0.5 0.5 -0.5 0 0.5,h=0.4]
	\end{pspicture}}

	%% Triangular Prism Unfolded
    \scalebox{0.8}{
	\begin{pspicture}(-2,-3.5)(3,1.5)
	    \psset{yunit=0.9,xunit=0.9}
% 	    \psset{Alpha=60,Beta=30}
% 	    \psset{viewpoint=8 12 8}
	    \psSolid[object=face,fillcolor=lightgray,opacity=0.5,base=-0.5 -0.5 0.5 -0.5 0 0.5](0,0,0)
	    \psSolid[object=face,fillcolor=lightgray,incolor=white,base=-0.5 -0.5 0.5 -0.5 0.5 -0.9 -0.5 -0.9](0,0,0)
	    \psSolid[object=face,fillcolor=lightgray,incolor=white,base=-0.5 -0.9 0.5 -0.9 0 -1.9](0,0,0)
	    \psSolid[object=face,fillcolor=lightgray,incolor=white,base=-0.5 -0.5 -0.8464102 -0.3 -0.3464102 0.7 0.0 0.5](0,0,0)
	    \psSolid[object=face,fillcolor=lightgray,incolor=white,base=0.0 0.5 0.3464102 0.7 0.8464102 -0.3 0.5 -0.5](0,0,0)
	\end{pspicture}}

	%% Triangular Prism Unfolded Birdseye
    \scalebox{0.8}{
	\begin{pspicture}(-2,-3.5)(3,1.5)
	    \psset{yunit=0.9,xunit=0.9}
% 	    \psset{Alpha=60,Beta=30}
	    \psset{viewpoint=0 0 20,RotZ=0}
	    \psSolid[object=face,fillcolor=lightgray,opacity=0.5,base=-0.5 -0.5 0.5 -0.5 0 0.5](0,0,0)
	    \psSolid[object=face,fillcolor=lightgray,incolor=white,base=-0.5 -0.5 0.5 -0.5 0.5 -0.9 -0.5 -0.9](0,0,0)
	    \psSolid[object=face,fillcolor=lightgray,incolor=white,base=-0.5 -0.9 0.5 -0.9 0 -1.9](0,0,0)
	    \psSolid[object=face,fillcolor=lightgray,incolor=white,base=-0.5 -0.5 -0.8464102 -0.3 -0.3464102 0.7 0.0 0.5](0,0,0)
	    \psSolid[object=face,fillcolor=lightgray,incolor=white,base=0.0 0.5 0.3464102 0.7 0.8464102 -0.3 0.5 -0.5](0,0,0)
	\end{pspicture}}


    \end{center}
\end{figure} 

A triangular prism unfolded into a net is made up of two triangles and
three rectangles. The sum of the lengths of the rectangles is equal to
the perimeter of the triangles.

\begin{figure}[H]
\begin{caption*}{\textbf{Cylinder}}\end{caption*}
\begin{center}

	%% Cylindrical Prism
	\begin{pspicture}(-2,-3.5)(3,1.5)
	    \psset{yunit=0.9,xunit=0.9}
	    \psellipse[fillcolor=white,fillstyle=solid](0,-3)(1.0,0.5)
	    \psframe[linestyle=none,fillcolor=white,fillstyle=solid](-1,-3)(1,0)
	    \psellipse[fillcolor=lightgray,opacity=0.5,fillstyle=solid,linestyle=dashed](0,-3)(1.0,0.5)
	    \psellipse[fillstyle=none](0,-0)(1.0,0.5)
	    \psline(-1.0,-3)(-1.0,-0)
	    \psline(1.0,-3)(1.0,-0)
	\end{pspicture}
\hspace{20pt}
	%% Cylindrical Prism Unfolded
	\begin{pspicture}(-2,-3.5)(3,1.5)
	    \psset{yunit=0.9,xunit=0.9}
	    \psellipse[fillcolor=lightgray,opacity=0.5,fillstyle=solid,linestyle=solid](4,-4)(1.0,1.0)
	    \psellipse[fillstyle=none](0,1)(1.0,1)
	    \psframe[linestyle=solid,fillcolor=white,fillstyle=solid](-1,-3)(5,0)
	\end{pspicture}

    \end{center}
\end{figure}   

A cylinder unfolded into a net is made up of two identical circles and a rectangle with a length equal to
the circumference of the circles.


\begin{wex}
{Finding the surface area of a rectangular prism}
{Find the surface area of the following rectangular prism:
\begin{center}
\scalebox{0.9}{
    \begin{pspicture}(-2,-3.5)(3,1.5)
        \psset{yunit=0.9,xunit=0.9}
        \psset{Alpha=30,Beta=15}
        \pstThreeDBox(-1,1,2)(0,0,2)(10,0,0)(0,5,0)
        \pstThreeDPut(8,6,3){$2$ cm}
        \pstThreeDPut(10,3.5,2){$5$ cm}
        \pstThreeDPut(5,7,2){$10$ cm}
   \end{pspicture}}
\end{center}
}
{
\westep{Sketch and label the net of the prism}

\begin{center}
	%% Rectangular Prism Unfolded Birdseye
\scalebox{1}{ 
	\begin{pspicture}(-2,-3.5)(3,1.5)
	    \psset{yunit=0.9,xunit=0.9}
	    \psset{Alpha=180,Beta=90}
	    {\psset{fillcolor=white,fillstyle=solid,opacity=0,linestyle=solid,dotstyle=}
	    \pstThreeDSquare(-1,1,2)(2,0,0)(0,3.5,0)}
	    \pstThreeDSquare[fillcolor=white,fillstyle=solid,opacity=0](-2,1,2)(1,0,0)(0,3.5,0)
	    \pstThreeDSquare[fillcolor=white,fillstyle=solid,opacity=0](1,1,2)(1,0,0)(0,3.5,0)
	    \pstThreeDSquare[fillcolor=white,fillstyle=solid,opacity=0](2,1,2)(2,0,0)(0,3.5,0)
	    \pstThreeDSquare[fillcolor=white,fillstyle=solid,opacity=0](-1,4.5,2)(2,0,0)(0,1,0)
	    \pstThreeDSquare[fillcolor=white,fillstyle=solid,opacity=0](-1,0,2)(2,0,0)(0,1,0)
            \rput(0,6.){$5$ cm}
            \rput(1.5, 5){$2$ cm}
 \rput(3,5){$5$ cm}
            \rput(-1.7, 5){$2$ cm}
            \rput(4.5, 3){$10$ cm}
	\end{pspicture}}
\end{center}

\westep{Find the areas of the different shapes in the net}
\begin{align*}
\mbox{large rectangle} &= \mbox{perimeter of small rectangle} \times \mbox{length} \\
                        &= (2+ 5 +2 + 5) \times 10 \\
                        &= 14 \times 10 \\
                        &= 140~\mbox{cm}^2  \\ \\
2 \times \mbox{small rectangle} &= 2(5 \times 2) \\
                                &= 2(10) \\
                                &= 20~\mbox{cm}^2
\end{align*}


\westep{Find the sum of the areas of the faces}

$\mbox{large rectangle} + 2 \times \mbox{small rectangle} = 140 + 20 = 160$ cm$^2$.
\westep{Write the final answer}
The surface area of the rectangular prism is 160~cm$^2$.
}
\end{wex}


\begin{wex}
{Finding the surface area of a triangular prism}
{Find the surface area of the following triangular prism:\\
\begin{center}
\scalebox{1}{
	%% Triangular Prism
\scalebox{1} % Change this value to rescale the drawing.
{
\begin{pspicture}(0,-1.8707813)(4.78,1.8484201)
\pstriangle[linewidth=0.04,dimen=outer](1.09,-1.3667188)(2.18,1.72)
\psline[linewidth=0.035277776cm](1.08,0.33328116)(3.38,1.8307812)
\psline[linewidth=0.035277776cm,linestyle=dotted,dotsep=0.10583334cm](0.06,-1.3267188)(2.76,0.39078125)
\psline[linewidth=0.035277776cm](3.38,1.8107812)(4.64,0.43328106)
\psline[linewidth=0.035277776cm](2.14,-1.3467188)(4.66,0.43328124)
\psline[linewidth=0.035277776cm,linestyle=dashed,dash=0.16cm 0.16cm](1.08,0.29328126)(1.06,-1.3467188)
\psline[linewidth=0.035277776cm,linestyle=dotted,dotsep=0.10583334cm](3.38,1.7732812)(2.74,0.41078126)
\psline[linewidth=0.035277776cm,linestyle=dotted,dotsep=0.10583334cm](4.68,0.43078125)(2.78,0.41078126)
\psline[linewidth=0.03](1.089162,-1.0679687)(1.309162,-1.0679687)(1.309162,-1.2679688)(1.309162,-1.3679688)
% \usefont{T1}{ptm}{m}{n}
\rput(1.045,-1.6679688){$8$ cm}
% \usefont{T1}{ptm}{m}{n}
\rput(3.975,-0.7679688){$12$ cm}
% \usefont{T1}{ptm}{m}{n}
\rput(1.475,-0.86796874){$3$ cm}
\psline[linewidth=0.04cm](1.4877112,-0.53157985)(1.5077113,-0.53157985)
\psline[linewidth=0.04cm](1.5077113,-0.53157985)(1.7077112,-0.37157986)
\psline[linewidth=0.04cm](0.68771124,-0.61157984)(0.44771126,-0.47157985)
\end{pspicture} 
}
}
\end{center}


}
{%Solution

\westep{Sketch and label the net of the prism}

% LaTeX Draw
% \usepackage{pst-plot} % For axes
\begin{center}
\scalebox{1} % Change this value to rescale the drawing.
{
\begin{pspicture}(0,-3.538172)(7.78,3.5381718)
\psframe[linewidth=0.02,dimen=outer](6.3,2.2243125)(0.0,-2.2243125)
\pstriangle[linewidth=0.02,dimen=outer](3.15,2.2072406)(2.87,1.3345875)
\rput{-180.0}(6.3,-5.7490687){\pstriangle[linewidth=0.02,dimen=outer](3.15,-3.541828)(2.87,1.3345875)}
% \usefont{T1}{ptm}{m}{n}
\rput(3.696875,2.5231717){$3$ cm}
\psline[linewidth=0.02cm,linestyle=dashed,dash=0.16cm 0.16cm](3.14,3.5181718)(3.14,2.2181718)
% \usefont{T1}{ptm}{m}{n}
\rput(6.975,0.0831719){$12$ cm}
% \usefont{T1}{ptm}{m}{n}
\rput(3.175,1.9000001){$8$ cm}
\psline[linewidth=0.02cm,linestyle=dashed,dash=0.16cm 0.16cm](1.74,2.2181718)(1.76,-2.2218282)
\psline[linewidth=0.02cm,linestyle=dashed,dash=0.16cm 0.16cm](4.52,2.2381718)(4.54,-2.2018282)
\psline[linewidth=0.04cm](3.56,2.8781717)(3.82,3.1381717)
\psline[linewidth=0.04cm](2.4066818,3.0833232)(2.6933181,2.8530202)
\end{pspicture} 
}
\end{center}
\westep{Find the area of the different shapes in the net}
To find the area of the rectangle, we need to calculate its length, which is equal to the perimeter of the triangles.\\
\\
To find the perimeter of the triangle, we have to first find the length of its sides using the Theorem of Pythagoras:

\begin{center}
\scalebox{1} % Change this value to rescale the drawing.
{
\begin{pspicture}(0,-0.8885703)(2.856284,0.844914)
\pstriangle[linewidth=0.02,dimen=outer](1.435,-0.4860172)(2.87,1.3345875)
% \usefont{T1}{ptm}{m}{n}
\rput(1.801875,-0.010085966){$3$ cm}
\psline[linewidth=0.02cm,linestyle=dashed,dash=0.16cm 0.16cm](1.425,0.82491404)(1.425,-0.47508597)
% \usefont{T1}{ptm}{m}{n}
\rput(1.475,-0.6857578){$8$ cm}
% \usefont{T1}{ptm}{m}{n}
\rput(0.42534634,0.26915622){$x$}
\psline[linewidth=0.04cm](1.8662839,0.26491404)(2.026284,0.42491403)
\psline[linewidth=0.04cm](0.96628386,0.26491404)(0.78628385,0.42491403)
\end{pspicture} 
}
\end{center}


\begin{align*}
  x^2 &= 3^2 + \left(\frac{8}{2}\right)^2 \\
  x^2 &= 3^2 + 4^2 \\
  &= 25 \\
  \therefore x &= 5\mbox{ cm}\\
  \therefore \mbox{perimeter of triangle} &= 5 + 5 + 8 \\
  &= 18\mbox{ cm}
\end{align*}

\begin{align*}
  \therefore \mbox{area of large rectangle}
  &= \mbox{perimeter of triangle} \times \mbox{length}\\
  &= 18 \times 12 \\
  &= 216 \mbox{ cm}^2
\end{align*}

\begin{align*}
  \mbox{area of triangle} &= \frac{1}{2}b \times h\\
  &= \frac{1}{2} \times 8 \times 3\\
  &= 12 \mbox{ cm}^2
\end{align*}

\westep{Find the sum of the areas of the faces}

\begin{align*}
  \mbox{surface area}
  &= \mbox{area large rectangle} + (2 \times \mbox{area of triangle}) \\
  &= 216 + 2(12) \\
  &= 240\mbox{ cm}^2
\end{align*}

\westep{Write the final answer}
The surface area of the triangular prism is $240$ cm$^2$.
}
\end{wex}


\begin{wex}
{Finding the surface area of a cylindrical prism}
{Find the surface area of the following cylinder (correct to $1$ decimal place):
\begin{center}
        \begin{pspicture}(-2,-3.5)(3,1.5)
	    \psset{yunit=0.9,xunit=0.9}
	    \psellipse(0,-3)(1.0,0.5)
	    \psframe[linestyle=none,](-1,-3)(1,0)
	    \psellipse[linestyle=dashed](0,-3)(1.0,0.5)
	    \psellipse[](0,-0)(1.0,0.5)
	    \psline(-1.0,-3)(-1.0,-0)
	    \psline(1.0,-3)(1.0,-0)
            \psline(0,0)(1.0,0)
            \rput(1.5,-1.5){$30$ cm}
            \rput(0.3,0.2){$10$ cm}
	\end{pspicture}
\end{center}



}
{% solution

\westep{Sketch and label the net of the prism}
\begin{center}
	%% Cylindrical Prism Unfolded
	\begin{pspicture}(-2,-3.5)(3,1.5)
	    \psset{yunit=0.9,xunit=0.9}
	    \psellipse[linestyle=solid](4,-4)(1.0,1.0)
	    \psellipse(0,1)(1.0,1)
	    \psframe[linestyle=solid](-1,-3)(5,0)
            \rput(0.2,1.2){$10$ cm}
            \rput(5.7,-1.5){$30$ cm}
\psline(0,1)(1,1)
	\end{pspicture}
\end{center}



\westep{Find the area of the different shapes in the net}

\begin{align*}
  \mbox{area of large rectangle}
  &= \mbox{circumference of circle} \times \mbox{length} \\
  &= 2\pi r \times l \\
  &= 2\pi(10) \times 30 \\
  &= 1884,96 \mbox{ cm}^2 \\[10pt]
  \mbox{area of circle}
  &= \pi r^2 \\
  &= \pi10^2 \\
  &= 314,16\mbox{ cm}^2
\end{align*}

\begin{align*}
  \mbox{surface area}
  &= \mbox{area large rectangle} + (2 \times \mbox{area of circle}) \\
  &= 1884,96 + 2(314,16) \\
  &= 2513,28\mbox{ cm}^2
\end{align*}

\westep{Write the final answer}
The surface area of the cylinder is $2513,28\mbox{ cm}^2$.
}
\end{wex}

\begin{exercises}{}{
\begin{enumerate}[noitemsep, label=\textbf{\arabic*}. ] 
\item Calculate the surface area of the following prisms:
  \setcounter{subfigure}{0}
  \begin{figure}[H]
    \begin{center}
      \scalebox{1}{% Change this value to rescale the drawing.
        \begin{pspicture}(0,-4.3556247)(11.315155,4.3556247)
          \psline[linewidth=0.04cm,linestyle=dashed,dash=0.17638889cm 0.10583334cm](0.90234375,0.8371875)(1.9023438,1.8171875)
          \psline[linewidth=0.04cm](2.6423438,0.8571875)(3.58,1.8756249)
          \psline[linewidth=0.04cm](0.9,0.8556249)(2.64,0.8556249)
          \psline[linewidth=0.04cm,linestyle=dashed,dash=0.17638889cm 0.10583334cm](1.9023438,1.8371875)(3.54,1.8356249)
          \psline[linewidth=0.04cm](1.9,3.795625)(1.9223437,1.8371875)
          \psline[linewidth=0.04cm](1.8823438,3.8171875)(3.58,3.815625)
          \psline[linewidth=0.04cm](2.6,2.875625)(2.6223435,0.8371875)
          \psline[linewidth=0.04cm](0.92,2.8956249)(0.92234373,0.8571875)
          \psline[linewidth=0.04cm](0.9423438,2.8571877)(2.6,2.855625)
          \psline[linewidth=0.04cm](0.92234373,2.8771877)(1.9023438,3.8171875)
          \psellipse[linewidth=0.04,dimen=outer](9.092343,1.8371875)(0.99,0.38)
          \psellipse[linewidth=0.04,dimen=outer](9.092343,3.1771872)(0.99,0.38)
          \psline[linewidth=0.04cm](8.122344,3.1371875)(8.122344,1.8771876)
          \psline[linewidth=0.04cm](10.062345,3.1771872)(10.062345,1.8571875)
          \psline[linewidth=0.04cm,linestyle=dotted,dotsep=0.10583334cm](9.122344,1.8171875)(10.042343,1.8371875)
          % \usefont{T1}{ptm}{m}{n}
          \rput(9.06172,1.9671875){$5$ cm}
          % \usefont{T1}{ptm}{m}{n}
          \rput(10.510155,2.5871873){$10$ cm}
          \pstriangle[linewidth=0.04,dimen=outer](2.2723436,-3.9028125)(2.18,1.72)
          \psline[linewidth=0.04cm](2.2623436,-2.2028127)(4.6223435,-0.7828125)
          \psline[linewidth=0.04cm,linestyle=dashed,dash=0.17638889cm 0.10583334cm](1.2423439,-3.8628125)(4.4223437,-1.7828124)
          \psline[linewidth=0.04cm](4.6223435,-0.7628125)(5.842344,-2.1028128)
          \psline[linewidth=0.04cm](3.3223438,-3.8828125)(5.842344,-2.1028123)
          \psline[linewidth=0.04cm,linestyle=dotted,dotsep=0.10583334cm](2.2623436,-2.2428124)(2.2423437,-3.8828125)
          % \usefont{T1}{ptm}{m}{n}
          \rput(1.5089062,-1.2678123){\textbf{(c)}}
          % \usefont{T1}{ptm}{m}{n}
          \rput(4.952656,-3.1528125){$20$ cm}
          % \usefont{T1}{ptm}{m}{n}
          \rput(2.5559375,-3.3999999){$5$ cm}
          % \usefont{T1}{ptm}{m}{n}
          \rput(2.3156252,-4.1528125){$10$ cm}
          % \usefont{T1}{ptm}{m}{n}
          \rput(1.0320313,4.1521873){\textbf{(a)}}
          % \usefont{T1}{ptm}{m}{n}
          \rput(7.833125,4.0121875){\textbf{(b)}}
          % \usefont{T1}{ptm}{m}{n}
          \rput(1.915625,0.6071877){$6$ cm}
          % \usefont{T1}{ptm}{m}{n}
          \rput(3.8587499,1.3071878){$7$ cm}
          % \usefont{T1}{ptm}{m}{n}
          \rput(4.319844,2.9471874){$10$ cm}
          \psline[linewidth=0.04cm,linestyle=dashed,dash=0.17638889cm 0.10583334cm](4.6223435,-0.7628123)(4.4223437,-1.8028125)
          \psline[linewidth=0.04cm,linestyle=dashed,dash=0.17638889cm 0.10583334cm](5.842344,-2.1028123)(4.382344,-1.8028125)
          \psline[linewidth=0.04cm](2.7279687,-3.104375)(2.8879688,-2.984375)
          \psline[linewidth=0.04cm](1.8279687,-3.1643748)(1.6279688,-3.004375)
          \psline[linewidth=0.04cm](2.6,2.855625)(3.5623438,3.8171875)
          \psline[linewidth=0.04cm](3.56,3.815625)(3.56,1.8356249)
        \end{pspicture} 
      }
    \end{center}
  \end{figure}   
\item If a litre of paint covers an area of $2$ m$^{2}$, how much paint does a painter need to cover:
  \begin{enumerate}[noitemsep, label=\textbf{(\alph*)} ]
  % \setcounter{enumi}{3}
  \item a rectangular swimming pool with dimensions $4$ m $\times~3$ m $\times~2,5$ m (the inside walls and floor only);
  \item the inside walls and floor of a circular reservoir with diameter $4$ m and height $2,5$ m.
  \end{enumerate}
  \setcounter{subfigure}{0}
  \begin{figure}[H] % horizontal\label{m39357*id62926}
    \begin{center}
      \scalebox{1}{% Change this value to rescale the drawing.
        \begin{pspicture}(0,-1.6)(3.8067188,1.6)
          \psellipse[linewidth=0.04,dimen=outer](1.29,-1.09)(1.27,0.51)
          \psellipse[linewidth=0.04,dimen=outer](1.27,1.09)(1.27,0.51)
          \psline[linewidth=0.04cm](0.04,-1.1)(0.02,1.14)
          \psline[linewidth=0.04cm](2.54,1.12)(2.56,-1.1)
          \psline[linewidth=0.04cm,linestyle=dotted,dotsep=0.10583334cm](0.06,-1.1)(2.52,-1.1)
          % \usefont{T1}{ptm}{m}{n}
          \rput(1.2675,-0.95){$4$ m}
          % \usefont{T1}{ptm}{m}{n}
          \rput(3.0317187,0.25){$2,5$ m}
        \end{pspicture} 
      }
    \end{center}
  \end{figure}   
  \addtocounter{footnote}{-0}
\end{enumerate}
\practiceinfo
\begin{tabularx}{\textwidth}{ XX }
(1.) 00hj&	(2.) 00hk& (3.) 00hm\end{tabularx}       
}
\end{exercises}        
\subsection{Volume of prisms and cylinders}
\Definition{Volume}{Volume is the three dimensional space occupied by an object, or the contents of an object. It is measured in cubic units.}
The volume of a right prism is simply calculated by multiplying the area of the
base of a solid by the height of the solid. 
\mindsetvid{Calculating volume}{VMcth}



\begin{table*}[h]
\newcolumntype{C}{>{\centering\arraybackslash} m{0.75in} }
\newcolumntype{D}{>{\centering\arraybackslash} m{2.1in} }
\begin{tabular}{|C|D|D|}
\hline
\textbf{Rectangular prism}
&
\begin{center}
\begin{pspicture}(0,-1.2412498)(4.584375,1.2212498)
\psline[linewidth=0.04cm](0.0,-0.83875)(1.0,0.14125004)
\psline[linewidth=0.04cm](3.0,-0.81875)(3.98,0.14125004)
\psline[linewidth=0.04cm](0.02,-0.83875)(3.0,-0.81875)
\psline[linewidth=0.04cm](1.0,0.16124995)(3.98,0.16124995)
\psline[linewidth=0.04cm](4.0,1.2012498)(3.98,0.10124995)
\psline[linewidth=0.04cm](1.0,1.1812499)(1.02,0.16125014)
\psline[linewidth=0.04cm](1.02,1.1612499)(4.0,1.1812499)
\psline[linewidth=0.04cm](3.0,0.24124995)(3.0,-0.81875)
\psline[linewidth=0.04cm](0.0,0.20125005)(0.0,-0.85875)
\psline[linewidth=0.04cm](0.0,0.22125006)(2.98,0.22125006)
\psline[linewidth=0.04cm](0.02,0.22125006)(1.0,1.1612499)
\psline[linewidth=0.04cm](3.0,0.22125015)(3.98,1.1612499)
% \usefont{T1}{ptm}{m}{n}
\rput(1.2896875,-1.0687499){$l$}
% \usefont{T1}{ptm}{m}{n}
\rput(3.9228127,-0.46874985){$b$}
% \usefont{T1}{ptm}{m}{n}
\rput(4.273906,0.63125014){$h$}
\end{pspicture}
\end{center} 
&
$
\begin{aligned}
\mbox{Volume} &= \mbox{area of base} \times \mbox{height} \\
              &= \mbox{area of rectangle} \times \mbox{height} \\
              &= l \times b \times h \\
\end{aligned}$   \\ \hline


\textbf{Triangular prism} &
\begin{center}
\scalebox{1} % Change this value to rescale the drawing.
{
\begin{pspicture}(0,-1.8337499)(4.9209375,1.8137499)
\pstriangle[linewidth=0.04,dimen=outer](1.3309375,-1.3462498)(2.18,1.72)
\psline[linewidth=0.04cm](1.3209375,0.35375)(3.6809375,1.77375)
\psline[linewidth=0.04cm](3.6809375,1.7937499)(4.9009376,0.4537499)
\psline[linewidth=0.04cm](2.3809376,-1.32625)(4.9009376,0.4537501)
\psline[linewidth=0.04cm,linestyle=dashed,dash=0.16cm 0.16cm](1.3209375,0.3137501)(1.3009375,-1.32625)
\psframe[linewidth=0.04,dimen=outer](1.4809375,-1.1462499)(1.2809376,-1.3462499)
\psline[linewidth=0.04cm](0.6209375,-0.3462499)(1.0009375,-0.5462499)
\psline[linewidth=0.04cm](0.5609375,-0.4462499)(0.9409375,-0.6462499)
\psline[linewidth=0.04cm](2.0409374,-0.3462499)(1.6609375,-0.5462499)
\psline[linewidth=0.04cm](2.1009376,-0.4462499)(1.7209375,-0.6462499)
% \usefont{T1}{ptm}{m}{n}
\rput(4.1923437,-0.5212499){$H$}
% \usefont{T1}{ptm}{m}{n}
\rput(1.2723438,-1.6612499){$b$}
% \usefont{T1}{ptm}{m}{n}
% \rput(0.21234375,-0.2412499){$s$}
% \usefont{T1}{ptm}{m}{n}
\rput(1.4723438,-0.7612499){$h$}
\end{pspicture} 
}
\end{center}
&

$\begin{aligned}
\mbox{Volume} &= \mbox{area of base} \times \mbox{height} \\
              &= \mbox{area of triangle} \times \mbox{height} \\
              &= \left(\frac{1}{2}b\times h\right) \times H \\
\end{aligned}$  \\ \hline

\textbf{Cylinder} &
\begin{center}
\scalebox{1} % Change this value to rescale the drawing.
{
\begin{pspicture}(0,-1.05)(2.761875,1.05)
\psellipse[linewidth=0.04,dimen=outer](0.99,-0.66999996)(0.99,0.38)
\psellipse[linewidth=0.04,dimen=outer](0.99,0.66999996)(0.99,0.38)
\psline[linewidth=0.04cm](0.02,0.63000023)(0.02,-0.6299999)
\psline[linewidth=0.04cm](1.96,0.66999996)(1.96,-0.65)
\psline[linewidth=0.04cm,linestyle=dashed,dash=0.16cm 0.16cm](1.02,-0.68999994)(1.94,-0.66999996)
\usefont{T1}{ptm}{m}{n}
\rput(2.4514062,0.09500025){$h$}
\usefont{T1}{ptm}{m}{n}
\rput(1.1014062,-0.50499976){$r$}
\end{pspicture} 
}
\end{center}
&

$\begin{aligned}
\mbox{Volume} &= \mbox{area of base} \times \mbox{height} \\
              &= \mbox{area of circle} \times \mbox{height} \\
              &= \pi r^2 \times h \\
\end{aligned}$  \\ \hline



\end{tabular}
\end{table*}

\Note{Remember that the volume is measured in cubic units, for example, cm$^3$, m$^3$ or mm$^3$.}


\begin{wex}
{Finding the volume of a cube}
{Find the volume of the following cube:
\begin{center}
\scalebox{1} % Change this value to rescale the drawing.
{
\begin{pspicture}(0,-1.0258813)(5.423805,1.3541186)
\psdiamond[linewidth=0.04,dimen=outer,gangle=-49.7](1.6365356,0.0)(1.27,1.0687643)
\psdiamond[linewidth=0.04,dimen=outer,gangle=50.0](3.2365355,0.0)(1.27,1.0687643)
\psline[linewidth=0.04](0.8223988,0.96411866)(2.5318084,1.2441187)(4.02,0.9541186)
\psline[linewidth=0.04cm](0.6423988,0.14411865)(1.0223988,0.14411865)
\psline[linewidth=0.04cm](2.2423987,-0.115881346)(2.6223986,-0.115881346)
\psline[linewidth=0.04cm](3.8423986,0.044118654)(4.2223988,0.044118654)
\psline[linewidth=0.04cm](1.7123986,0.95411867)(1.7123986,0.5741187)
\psline[linewidth=0.04cm](1.5723988,-0.60588133)(1.5723988,-0.9858813)
\psline[linewidth=0.04cm](3.2723987,-0.6258814)(3.2723987,-1.0058813)
\psline[linewidth=0.04cm](3.292399,1.0141187)(3.292399,0.6341187)
\psline[linewidth=0.04cm](1.8723989,1.3341186)(1.8723989,0.95411867)
\psline[linewidth=0.04cm](3.0723987,1.3341186)(3.0723987,0.95411867)
% \usefont{T1}{ptm}{m}{n}
\rput(4.728805,0.109118655){$3$ cm}
\end{pspicture} 
}
\end{center}
}
{%solution

\westep{Find the area of the base}
\begin{align*}
  \mbox{area of square} 
  &= s^2 \\
  &= 3^2 \\
  &= 9\mbox{ cm}^2
\end{align*}

\westep{Multiply the area of the base by the height of the solid to find the volume}
\begin{align*}
\mbox{volume} &= \mbox{area of base} \times \mbox{height}\\
                    &= 9 \times 3 \\
                    &= 27\mbox{ cm}^3 
\end{align*}
\westep{Write the final answer}
The volume of the cube is $27\mbox{ cm}^3 $. 
}
\end{wex}




\begin{wex}
{Finding the volume of a triangular prism}
{Find the volume of the triangular prism:\\
\begin{center}
\scalebox{1} % Change this value to rescale the drawing.
{
\begin{pspicture}(0,-1.8489062)(4.7628126,1.8289062)
\pstriangle[linewidth=0.04,dimen=outer](1.09,-1.3310935)(2.18,1.72)
\psline[linewidth=0.04cm](1.08,0.36890626)(3.44,1.7889062)
\psline[linewidth=0.04cm](3.44,1.8089062)(4.6600003,0.46890616)
\psline[linewidth=0.04cm](2.14,-1.3110938)(4.6600003,0.46890634)
\psline[linewidth=0.04cm,linestyle=dashed,dash=0.16cm 0.16cm](1.08,0.32890636)(1.06,-1.3110938)
\psframe[linewidth=0.04,dimen=outer](1.3077112,-1.0710938)(1.0400001,-1.3310937)
\psline[linewidth=0.04cm](0.38,-0.33109364)(0.76,-0.53109366)
\psline[linewidth=0.04cm](0.32,-0.43109366)(0.7,-0.6310936)
\psline[linewidth=0.04cm](1.8,-0.33109364)(1.42,-0.53109366)
\psline[linewidth=0.04cm](1.8600001,-0.43109366)(1.48,-0.6310936)
% \usefont{T1}{ptm}{m}{n}
\rput(3.9578123,-0.5060936){$20$ cm}
% \usefont{T1}{ptm}{m}{n}
\rput(1.2078125,-1.6460936){$8$ cm}
% \usefont{T1}{ptm}{m}{n}
\rput(1.475,-0.78484374){$10$ cm}
\end{pspicture} 
}

\end{center}
}
{%solution
\westep{Find the area of the base}
\begin{align*}
  \mbox{area of triangle} 
  &= \frac{1}{2}b \times h\\
  &= \left(\frac{1}{2} \times 8 \right) \times 10\\
  &= 40\mbox{ cm}^2
\end{align*}

\westep{Multiply the area of the base by the height of the solid to find the volume}
\begin{align*}
  \mbox{volume} &= \mbox{area of base} \times \mbox{height}\\
  &= \frac{1}{2}b \times h \times H \\
  &= 40 \times 20 \\
  &= 800\mbox{ cm}^3
\end{align*}

\westep{Write the final answer}
The volume of the triangular prism is $800\mbox{ cm}^3$.
}
\end{wex}

\begin{wex}{Finding the volume of a cylindrical prism}
{Find the volume of the following cylinder (correct to $1$ decimal place):
\begin{center}
        \begin{pspicture}(-2,-3.5)(3,1.5)
	    \psset{yunit=0.9,xunit=0.9}
	    \psellipse(0,-3)(1.0,0.5)
	    \psframe[linestyle=none,](-1,-3)(1,0)
	    \psellipse[linestyle=dashed](0,-3)(1.0,0.5)
	    \psellipse[](0,-0)(1.0,0.5)
	    \psline(-1.0,-3)(-1.0,-0)
	    \psline(1.0,-3)(1.0,-0)
            \psline(0,0)(1.0,0)
            \rput(1.5,-1.5){$15$ cm}
            \rput(0.3,0.2){$4$ cm}
	\end{pspicture}
\end{center}
}
{
\westep{Find the area of the base}
\begin{align*}
  \mbox{area of circle} &= \pi r^2 \\
  &= \pi(4)^{2} \\
  &= 16\pi\mbox{ cm}^{2}
\end{align*}

\westep{Multiply the area of the base by the height of the solid to find the volume}
\begin{align*}
  \mbox{volume} &= \mbox{area of base} \times \mbox{height} \\
  &= \pi r^{2} \times h\\
  &= 16\pi \times 15\\
  &= 754,0\mbox{ cm}^{3}\\
\end{align*}

\westep{Write the final answer}
The volume of the cylinder is $754,0\mbox{ cm}^{3}$.
}
\end{wex}

\begin{exercises}{}
{Calculate the volumes of the following prisms (correct to one decimal place):
\begin{center}
\scalebox{1} % Change this value to rescale the drawing.
{
\scalebox{1} % Change this value to rescale the drawing.
{
\begin{pspicture}(0,-4.0576563)(12.394375,4.0376563)
\psline[linewidth=0.04cm,linestyle=dashed,dash=0.17638889cm 0.10583334cm](1.2203125,0.64156264)(2.2203126,1.6215626)
\psline[linewidth=0.04cm](2.9603124,0.6615626)(3.8979688,1.6800001)
\psline[linewidth=0.04cm](1.2179687,0.66)(2.9579687,0.66)
\psline[linewidth=0.04cm,linestyle=dashed,dash=0.17638889cm 0.10583334cm](2.2203126,1.6415627)(3.8579688,1.6400001)
\psline[linewidth=0.04cm](2.2179687,3.6000001)(2.2403126,1.6415627)
\psline[linewidth=0.04cm](2.9179688,2.68)(2.9403124,0.64156264)
\psline[linewidth=0.04cm](1.2379688,2.7)(1.2403125,0.6615626)
\psline[linewidth=0.04cm](1.2603126,2.661563)(2.9179688,2.66)
\psline[linewidth=0.04cm](1.2403125,2.681563)(2.2203126,3.6215627)
% \usefont{T1}{ptm}{m}{n}
\rput(2.2335937,0.41156286){$6$ cm}
% \usefont{T1}{ptm}{m}{n}
\rput(4.1767187,1.111563){$7$ cm}
% \usefont{T1}{ptm}{m}{n}
\rput(4.6378126,2.7515626){$10$ cm}
\psline[linewidth=0.04cm](2.9179688,2.66)(3.8803124,3.6215627)
\psline[linewidth=0.04cm](3.8779688,3.6200001)(3.8779688,1.6400001)
% \usefont{T1}{ptm}{m}{n}
\rput(8.889843,1.0651562){$5$ cm}
% \usefont{T1}{ptm}{m}{n}
\rput(9.135625,0.4851563){$10$ cm}
\psline[linewidth=0.04cm](8.78125,2.0576563)(11.421249,4.0176563)
\psline[linewidth=0.04cm](7.94125,0.83765644)(8.78125,2.0776563)
\psline[linewidth=0.04cm](7.92125,0.83765644)(9.72125,0.83765644)
\psline[linewidth=0.04cm](9.74125,0.7976564)(8.76125,2.0976562)
\psline[linewidth=0.04cm,linestyle=dashed,dash=0.17638889cm 0.10583334cm](10.561251,2.7376564)(11.401251,3.9776564)
\psline[linewidth=0.04cm,linestyle=dashed,dash=0.17638889cm 0.10583334cm](10.541249,2.7376564)(12.341249,2.7376564)
\psline[linewidth=0.04cm](12.36125,2.7176564)(11.381249,4.0176563)
\psline[linewidth=0.04cm,linestyle=dashed,dash=0.17638889cm 0.10583334cm](7.98125,0.83765644)(10.62125,2.7976563)
\psline[linewidth=0.04cm](9.70125,0.8176564)(12.341249,2.7776563)
\psline[linewidth=0.04cm,linestyle=dashed,dash=0.16cm 0.16cm](8.78125,2.0376563)(8.8012495,0.8176564)
% \usefont{T1}{ptm}{m}{n}
\rput(11.579374,1.5076563){$20$ cm }
\psline[linewidth=0.04cm](2.34125,-1.1848438)(2.34125,-3.4248438)
\psline[linewidth=0.04cm](4.20125,-1.1848438)(4.20125,-3.4248438)
\psellipse[linewidth=0.04,dimen=outer](3.2712502,-1.2048438)(0.95,0.26)
\psellipse[linewidth=0.04,dimen=outer](3.2712502,-3.4648438)(0.95,0.26)
\psdots[dotsize=0.12](3.30125,-3.4648438)
\psline[linewidth=0.03cm,linestyle=dotted,dotsep=0.10583334cm](3.30125,-3.4628437)(4.18125,-3.4628437)
% \usefont{T1}{ptm}{m}{n}
\rput(3.954531,-3.8548439){$5$ cm}
% \usefont{T1}{ptm}{m}{n}
\rput(4.765156,-2.4123437){$10$ cm}
% \usefont{T1}{ptm}{m}{n}
\rput(0.9817188,3.7926562){\textbf{1.}}
% \usefont{T1}{ptm}{m}{n}
\rput(8.0,3.7726562){\textbf{2.}}
% \usefont{T1}{ptm}{m}{n}
\rput(0.96203125,-0.8273437){\textbf{3.}}
\psline[linewidth=0.04cm](2.2203126,3.6015627)(3.8779688,3.6000001)
\end{pspicture} 
}
}
\end{center}
\practiceinfo
\begin{tabularx}{\textwidth}{ XX }
(1.) 00hn&	(2.) 00hp& (3.) 00hq\end{tabularx}
}
\end{exercises}

\section{Right pyramids, right cones and spheres}
\Definition{Pyramid}{A pyramid is a geometric solid that has a polygon
  as its base and sides that converge at a point called the apex. In
  other words the sides are \textbf{not} perpendicular to the base.}

The triangular pyramid and square pyramid take their names from the
shape of their base. We call a pyramid a ``right pyramid'' if the line
between the apex and the centre of the base is perpendicular to the
base. Cones are similar to pyramids except that their bases are circles
instead of polygons. Spheres are solids that are perfectly round and look the same from any
direction.\par
Examples of a square pyramid, a triangular pyramid, a cone and a sphere:
\begin{figure}[ht]
\begin{center}
% \usepackage[usenames,dvipsnames]{pstricks}
% \usepackage{epsfig}
% \usepackage{pst-grad} % For gradients
% \usepackage{pst-plot} % For axes
\scalebox{1.4} % Change this value to rescale the drawing.
{
\begin{pspicture}(0,-0.9969445)(9.265889,0.99705553)
\pspolygon[linewidth=0.028222222,fillstyle=solid](2.621611,-0.26305547)(3.099611,0.46944445)(4.068611,-0.4720555)
\pspolygon[linewidth=0.028222222,fillstyle=solid](2.6316354,-0.66294444)(3.06,0.4170556)(2.54,-0.26294434)
\pspolygon[linewidth=0.028222222,fillstyle=solid](0.03411105,-0.32955545)(0.869611,0.48944443)(1.361111,-0.24505551)
\pspolygon[linewidth=0.028222222,fillstyle=solid](1.705111,-0.6149472)(0.869611,0.48944443)(0.37811106,-0.70294446)
\pspolygon[linewidth=0.028222222,fillstyle=solid](0.38,-0.70294446)(0.84,0.45705566)(0.0,-0.34294435)
\pspolygon[linewidth=0.028222222,fillstyle=solid](4.1,-0.49978656)(3.1,0.4770556)(2.62,-0.68294436)
\psline[linewidth=0.035,linestyle=dotted,dotsep=0.09cm](0.8858889,0.4570556)(0.8658889,-0.46294445)(1.125889,-0.64294446)
\psline[linewidth=0.01cm,linestyle=dashed,dash=0.1cm 0.1cm](0.0,-0.34294435)(1.08,-0.26294434)
\psline[linewidth=0.01cm,linestyle=dashed,dash=0.1cm 0.1cm](2.54,-0.26294434)(4.08,-0.48294446)
\psline[linewidth=0.01,linestyle=dashed,dash=0.1cm 0.1cm](1.72,-0.6229445)(1.08,-0.26294434)(0.86,0.45705566)
\psline[linewidth=0.028222222](5.12,-0.48294446)(5.7990556,0.9829444)(6.52,-0.5170556)
\psbezier[linewidth=0.027999999](5.1394444,-0.44294444)(5.04,-0.66294444)(5.34,-0.94294447)(5.7,-0.96294445)(6.06,-0.9829444)(6.52,-0.82294446)(6.52,-0.50294447)
\psbezier[linewidth=0.01,linestyle=dashed,dash=0.1cm 0.1cm](6.52,-0.52294445)(6.4,-0.34294435)(6.18,-0.2222707)(5.89007,-0.21260755)(5.60014,-0.2029444)(5.36,-0.2229444)(5.12,-0.50294447)
\pscircle[linewidth=0.027999999,dimen=outer](8.435889,-0.052944418){0.83}
\psellipse[linewidth=0.01,linestyle=dashed,dash=0.1cm 0.1cm,dimen=outer](8.435889,-0.11294442)(0.83,0.19)
\psline[linewidth=0.01cm,linestyle=dashed,dash=0.1cm 0.1cm](8.4458885,-0.122944415)(9.225889,-0.10294441)
\psdots[dotsize=0.068](8.425889,-0.122944415)
\psline[linewidth=0.02](0.8658889,-0.3629444)(0.8658889,-0.46294445)(0.9458889,-0.52294445)(0.9458889,-0.42294446)(0.8658889,-0.3629444)
\psline[linewidth=0.035,linestyle=dotted,dotsep=0.09cm](3.1258888,0.4770556)(3.1458888,-0.44294447)(2.6458888,-0.64294446)
\psline[linewidth=0.02](3.1458888,-0.3429444)(3.1458888,-0.44294447)(3.065889,-0.48294446)(3.065889,-0.3829444)(3.1458888,-0.3429444)
\psline[linewidth=0.036,linestyle=dotted,dotsep=0.16cm](5.7858887,0.9570555)(5.7858887,-0.52294445)(6.505889,-0.52294445)
\psframe[linewidth=0.02,dimen=outer](5.925889,-0.38294446)(5.7658887,-0.5429445)
\end{pspicture} 

}

% \begin{caption*}{Examples of a square pyramid, a triangular pyramid, a cone and a sphere.}\end{caption*}
% \label{fig:mg:sav:pyramids}
\end{center}
\end{figure}

\subsection{Surface area of pyramids, cones and spheres}

\begin{table}[H]
\newcolumntype{C}{>{\centering\arraybackslash} m{0.6in} }
\newcolumntype{D}{>{\centering\arraybackslash} m{2in} }
\newcolumntype{E}{>{\centering\arraybackslash} m{6cm} }
\begin{tabular}{|C|D|E|}
\hline
\textbf{Square pyramid}
&
\begin{center}

\scalebox{0.7} % Change this value to rescale the drawing.
{
\begin{pspicture}(0,-1.9993507)(4.767494,1.9734617)
\pspolygon[linewidth=0.028222222,fillstyle=solid](0.095092274,-0.42983752)(2.4242375,1.9593507)(3.794405,-0.18333401)
\pspolygon[linewidth=0.028222222,fillstyle=solid](4.753383,-1.2623826)(2.4242375,1.9593507)(1.0540702,-1.5190883)
\pspolygon[linewidth=0.028222222,fillstyle=solid](1.0593361,-1.5190883)(2.3858888,1.9269619)(0.0,-0.46889564)
\psline[linewidth=0.022cm,linestyle=dashed,dash=0.16cm 0.16cm](0.0,-0.46889564)(3.0107446,-0.23551947)
\psline[linewidth=0.027999999,linestyle=dotted,dotsep=0.16cm](2.465889,1.8869619)(2.3658888,-0.8130381)(3.305889,-1.3465382)(2.445889,1.8469619)(2.485889,1.8069619)
\psline[linewidth=0.024cm,linestyle=dashed,dash=0.16cm 0.16cm](3.025889,-0.2530381)(4.7258887,-1.2530382)
\psline[linewidth=0.02](2.3658888,-0.6130381)(2.545889,-0.6930381)(2.545889,-0.91303813)(2.3658888,-0.8330382)
\psline[linewidth=0.02cm](2.3658888,-0.6330381)(2.3658888,-0.8330382)
\psline[linewidth=0.04cm](0.3058889,-0.9465382)(0.70588887,-0.9465382)
\psline[linewidth=0.04cm](2.8458889,-1.5265381)(2.6258888,-1.2465382)
\psline[linewidth=0.04cm](3.6458888,-0.7065382)(4.025889,-0.7265382)
\psline[linewidth=0.04cm](1.9058889,-0.44653818)(1.6858889,-0.20653819)
\usefont{T1}{ppl}{m}{n}
\rput(3.1304202,0.24346182){$h$}
\usefont{T1}{ppl}{m}{n}
\rput(2.9704201,-1.7965382){$b$}
\psline[linewidth=0.02](3.245889,-1.1265382)(3.305889,-1.3665382)(3.545889,-1.3465382)(3.485889,-1.1065382)(3.245889,-1.1265382)
\usefont{T1}{ptm}{m}{n}
\rput(2.2,0.6834618){$H$}
\end{pspicture} 
}
\end{center} 
&
\begin{equation*}
  \begin{array}{r@{\;}c@{\;}l}
    \mbox{Surface area} &=& \mbox{area of base } +\\
    && \mbox{area of triangular sides} \\
    &=&b^{2} + 4\left(\frac{1}{2}bh\right)\\
    &=&b(b+2h)
  \end{array}
\end{equation*}
\\ \hline

\textbf{Triangular pyramid} &
\begin{center}
\scalebox{0.7} % Change this value to rescale the drawing.
{
\begin{pspicture}(0,-2.0793507)(4.2198644,2.0534618)
\pspolygon[linewidth=0.028222222,fillstyle=solid](0.2523495,-0.43995067)(1.5305833,2.0138543)(4.1218147,-1.1400807)
\pspolygon[linewidth=0.028222222,fillstyle=solid](0.24,-1.8206493)(1.5,1.9393507)(0.034111112,-0.43957838)
\pspolygon[linewidth=0.028222222,fillstyle=solid](4.2057533,-1.232977)(1.5316236,2.0393507)(0.24804147,-1.8465381)
\psline[linewidth=0.022cm,linestyle=dashed,dash=0.16cm 0.16cm](0.034111112,-0.43957838)(4.1522703,-1.1765574)
\psline[linewidth=0.024,linestyle=dotted,dotsep=0.16cm](1.5199999,1.9793508)(1.48,-1.1806492)(4.1800003,-1.2206492)
\psline[linewidth=0.02](1.48,-0.96064925)(1.7,-0.96064925)(1.7,-1.1865382)(1.48,-1.1806492)(1.48,-0.96064925)
\psline[linewidth=0.04cm](1.86,-0.84653825)(2.02,-0.6865382)
\psline[linewidth=0.04cm](1.72,-1.5265383)(1.76,-1.7465383)
\psline[linewidth=0.04cm](0.24,-1.0865382)(0.0,-1.1265383)
\psline[linewidth=0.04cm,linestyle=dotted,dotsep=0.16cm](1.54,2.0334618)(2.5,-1.5065383)
\psline[linewidth=0.02](2.44,-1.2465383)(2.496,-1.4865382)(2.72,-1.4465382)(2.66,-1.2065382)(2.44,-1.2465383)
\psline[linewidth=0.024cm,linestyle=dotted,dotsep=0.16cm](1.56,-1.1865382)(0.12,-1.1465383)
\usefont{T1}{ppl}{m}{n}
\rput(2.4145312,0.08346178){$h_s$}
\usefont{T1}{ppl}{m}{n}
\rput(2.1045313,-1.8765383){$b$}
\usefont{T1}{ppl}{m}{n}
\rput(1.3845313,-1.2965382){$h_b$}
\usefont{T1}{ppl}{m}{n}
\rput(1.3,0.3){$H$}
\end{pspicture} 
}
\end{center}
&
\begin{equation*}
  \begin{array}{r@{\;}c@{\;}l}
    \mbox{Surface area} &=& \mbox{area of base } +\\
    &&\mbox{area of triangular sides} \\
    &=&\left(\frac{1}{2}b \times h_b\right) + 3\left(\frac{1}{2}b \times h_s\right)\\
    &=&\frac{1}{2}b(h_b + 3h_s)
  \end{array}
\end{equation*}
\\ \hline

\textbf{Right cone} &
\begin{center}
 \scalebox{0.5} % Change this value to rescale the drawing.
{
\begin{pspicture}(0,-3.1099448)(4.5657654,3.1100557)
\psline[linewidth=0.028222222](0.24603535,-1.5211127)(2.3344314,3.0959444)(4.551654,-1.6285512)
\psbezier[linewidth=0.027999999](0.3058355,-1.3951261)(0.0,-2.088052)(0.9226326,-2.969958)(2.0297916,-3.0329514)(3.1369507,-3.0959444)(4.551654,-2.5919983)(4.551654,-1.584106)
\psbezier[linewidth=0.022,linestyle=dashed,dash=0.1cm 0.1cm](4.551654,-1.6470991)(4.182601,-1.0801597)(3.5060039,-0.70007807)(2.6143408,-0.6696424)(1.7226781,-0.6392067)(0.9841414,-0.7022)(0.24603535,-1.584106)
\psline[linewidth=0.04,linestyle=dotted,dotsep=0.1cm](2.3444264,3.0043032)(2.261968,-1.7245709)(4.488356,-1.7523878)(4.488356,-1.6133032)(4.46087,-1.6133032)
\psframe[linewidth=0.04,dimen=outer](2.6258888,-1.3299444)(2.225889,-1.7299443)
% \usefont{T1}{ptm}{m}{n}
\rput(3.8704202,1.0400556){\LARGE$h$}
% \usefont{T1}{ptm}{m}{n}
\rput(3.56042,-1.4399444){\LARGE$r$}
% \usefont{T1}{ptm}{m}{n}
\rput(1.8404201,0.24005564){\LARGE$H$}
\end{pspicture} 
}
\end{center}
&
\begin{equation*}
  \begin{array}{r@{\;}c@{\;}l}
    \mbox{Surface area} &=& \mbox{area of base } +\\
    &&\mbox{area of walls}\\
    &=& \pi r^{2} + \frac{1}{2} \times 2\pi rh \\
    &=& \pi r(r+h)\\
  \end{array}
\end{equation*}
\\ \hline

\textbf{Sphere} &
\begin{center}
\scalebox{0.7} % Change this value to rescale the drawing.
{
\begin{pspicture}(0,-1.9)(3.8,1.9)
\pscircle[linewidth=0.027999999,dimen=outer](1.9,0.0){1.9}
\psellipse[linewidth=0.027999999,linestyle=dashed,dash=0.16cm 0.16cm,dimen=outer](1.9,-0.13)(1.84,0.23)
\psline[linewidth=0.04,linestyle=dotted,dotsep=0.1cm](1.96,-0.14)(3.74,-0.14)
\psdots[dotsize=0.09](1.92,-0.16)
\usefont{T1}{ppl}{m}{n}
\rput(2.6345313,0.2){$r$}
\end{pspicture} 
}
\end{center}
&
$\mbox{Surface area} =  4\pi r^{2}$
\\ \hline
\end{tabular}
\end{table}

\begin{wex}{Finding the surface area of a triangular pyramid}
{Find the surface area of the following triangular pyramid (correct to one decimal place):\\
\begin{center}
\scalebox{1} % Change this value to rescale the drawing.
{
\begin{pspicture}(0,-2.1164062)(4.2998643,2.0964062)
\pspolygon[linewidth=0.028222222](0.32,-1.7835938)(1.58,1.9764062)(0.11411111,-0.40252286)
\pspolygon[linewidth=0.028222222](4.2857533,-1.1959213)(1.6116236,2.0764062)(0.32804146,-1.8094827)
\psline[linewidth=0.022cm,linestyle=dashed,dash=0.16cm 0.16cm](0.11411111,-0.40252286)(4.2322707,-1.1395019)
\psline[linewidth=0.04cm,linestyle=dotted,dotsep=0.16cm](1.6,2.0764062)(2.26,-1.5035938)
\psline[linewidth=0.024](2.22,-1.2635938)(2.44,-1.2235936)(2.48,-1.4635936)(2.5,-1.4635936)
\psline[linewidth=0.04cm](1.86,-0.5835937)(1.8,-0.8435938)
\psline[linewidth=0.04cm](0.0,-1.0835936)(0.34,-1.0235938)
\psline[linewidth=0.04cm](1.78,-1.4435936)(1.96,-1.7035936)
\usefont{T1}{ppl}{m}{n}
\rput(2.46,-1.9135938){$6$ cm}
\usefont{T1}{ppl}{m}{n}
\rput(2.375,0.3864063){$10$ cm}
\psline[linewidth=0.024cm](2.22,-1.2635938)(2.26,-1.5035938)
\end{pspicture} 
}
\end{center}
}
{
\westep{Find the area of the base}
\begin{equation*}
  \mbox{area of base triangle} = \frac{1}{2} bh_b\\
\end{equation*}
To find the height of the base triangle $(h_b)$ we use the Theorem of Pythagoras:
\\
\begin{center}
\scalebox{0.9} % Change this value to rescale the drawing.
{
\begin{pspicture}(0,-2.1564062)(4.64,2.097329)
\pstriangle[linewidth=0.04,dimen=outer](2.24,-1.6435938)(4.48,3.76)
\psline[linewidth=0.04cm,linestyle=dotted,dotsep=0.16cm](2.28,1.9164063)(2.16,-1.6035937)
\psframe[linewidth=0.04,dimen=outer](2.5,-1.2635938)(2.14,-1.6235938)
% \usefont{T1}{ptm}{m}{n}
\rput(3.165,-1.9535937){$3$ cm}
% \usefont{T1}{ptm}{m}{n}
\rput(3.945,0.62640625){$6$ cm}
% \usefont{T1}{ptm}{m}{n}
\rput(2.5845313,0.04640625){$h_b$}
\end{pspicture} 
}
\end{center}

\begin{align*}
  6^2 &= 3^2+h_b^2\\
  \therefore h_b&=\sqrt{6^2-3^2}\\
  &=3\sqrt{3}\\
  \therefore \mbox{ area of base triangle} &= \frac{1}{2} \times 6 \times 3\sqrt{3}\\
  &=9\sqrt{3}\mbox{ cm}^2
\end{align*}

\westep{Find the area of the sides}
\begin{align*}
  \mbox{area of sides} &= 3\Big(\frac{1}{2} \times b\times h_s\Big)\\
  &=3\Big(\frac{1}{2} \times 6 \times 10\Big)\\
  &=90\mbox{ cm}^2
\end{align*}

\westep{Find the sum of the areas}
\begin{align*}
  9\sqrt{3} + 90&=105,6\mbox{ cm}^2\\
\end{align*}
\westep{Write the final answer}
The surface area of the triangular pyramid is $105,6$ cm$^{2}$.
}
\end{wex}

\begin{wex}{Finding the surface area of a cone}
{Find the surface area of the following cone (correct to one decimal place):
\begin{center}
 \scalebox{0.8} % Change this value to rescale the drawing.
{
\begin{pspicture}(0,-3.1099448)(4.5657654,3.1100557)
\psline[linewidth=0.028222222](0.24603535,-1.5211127)(2.3344314,3.0959444)(4.551654,-1.6285512)
\psbezier[linewidth=0.027999999](0.3058355,-1.3951261)(0.0,-2.088052)(0.9226326,-2.969958)(2.0297916,-3.0329514)(3.1369507,-3.0959444)(4.551654,-2.5919983)(4.551654,-1.584106)
\psbezier[linewidth=0.022,linestyle=dashed,dash=0.1cm 0.1cm](4.551654,-1.6470991)(4.182601,-1.0801597)(3.5060039,-0.70007807)(2.6143408,-0.6696424)(1.7226781,-0.6392067)(0.9841414,-0.7022)(0.24603535,-1.584106)
\psline[linewidth=0.04,linestyle=dotted,dotsep=0.1cm](2.3444264,3.0043032)(2.261968,-1.7245709)(4.488356,-1.7523878)(4.488356,-1.6133032)(4.46087,-1.6133032)
\psframe[linewidth=0.04,dimen=outer](2.6258888,-1.3299444)(2.225889,-1.7299443)
% \usefont{T1}{ptm}{m}{n}
\rput(3.8704202,1.0400556){$h$}
% \usefont{T1}{ptm}{m}{n}
\rput(3.56042,-1.4399444){$4$ cm}
% \usefont{T1}{ptm}{m}{n}
\rput(1.8404201,0.24005564){$14$ cm}
\end{pspicture} 
}
\end{center}
}
{
\westep{Find the area of the base}
\begin{align*}
  \mbox{area of base circle} &= \pi r^2\\
  &= \pi \times 4^2\\
  &= 16\pi
\end{align*}

\westep{Find the area of the walls}
\begin{align*}
  \mbox{area of sides} &= \pi rh
\end{align*}
To find the slant height $h$ we use the Theorem of Pythagoras:\\
\begin{center}
\scalebox{0.8} % Change this value to rescale the drawing. walls
{
\begin{pspicture}(0,-2.1564062)(4.64,2.097329)
\pstriangle[linewidth=0.04,dimen=outer](2.24,-1.6435938)(4.48,3.76)
\psline[linewidth=0.04cm,linestyle=dotted,dotsep=0.16cm](2.28,1.9164063)(2.16,-1.6035937)
\psframe[linewidth=0.04,dimen=outer](2.5,-1.2635938)(2.14,-1.6235938)
% \usefont{T1}{ptm}{m}{n}
\rput(3.165,-1.9535937){$4$ cm}
% \usefont{T1}{ptm}{m}{n}
\rput(3.7,0.62640625){$h$}
% \usefont{T1}{ptm}{m}{n}
\rput(2.7,0.04640625){$14$ cm}
\end{pspicture}
}
\end{center}

\begin{align*}
  h^2 &= 4^2 + 14^2\\
  \therefore h &= \sqrt{4^2 + 14^2}\\
  &= 2\sqrt{53}\mbox{ cm}^2
\end{align*}

\begin{align*}
  \mbox{area of walls} &= \frac{1}{2}2\pi r h\\
  &=\pi(4)(2\sqrt{53})\\
  &= 8\pi\sqrt{53}\mbox{ cm}^2
\end{align*}

\westep{Find the sum of the areas}
\begin{align*}
  \mbox{total surface area} &= 16\pi + 8\pi\sqrt{53}\\
  &=233,2\mbox{ cm}^2
\end{align*}

\westep{Write the final answer}
The surface area of the cone is $233,2\mbox{ cm}^2$.
}
\end{wex}

\begin{wex}{Finding the surface area of a sphere}
{Find the surface area of the following sphere (correct to $1$ decimal place):
\begin{center}
\scalebox{0.9} % Change this value to rescale the drawing.
{
\begin{pspicture}(0,-1.9)(3.8,1.9)
\definecolor{color129b}{rgb}{0.9490196078431372,0.9490196078431372,0.9490196078431372}
\pscircle[linewidth=0.027999999,dimen=outer](1.9,0.0){1.9}
\psellipse[linewidth=0.027999999,linestyle=dashed,dash=0.16cm 0.16cm,dimen=outer](1.9,-0.13)(1.84,0.23)
\psline[linewidth=0.04cm,linestyle=dotted,dotsep=0.15cm](1.96,-0.14)(3.74,-0.14)
\psdots[dotsize=0.09](1.92,-0.16)
% \usefont{T1}{ptm}{m}{n}
\rput(2.7095313,0.12){\psframebox[linewidth=0.002,linecolor=color129b,fillstyle=solid,fillcolor=color129b]{$5$ cm}}%this framebox background colour is tweaked to match the wex colour - otherwise radius label sits too close between the dotted lines. framebox behind label blanks out some of eliptical dotted line to make it more legible
\end{pspicture} 
}
\end{center}
}
{
\begin{align*}
  \mbox{surface area of sphere} &= 4 \pi r^2\\
  &= 4\pi5^2\\
  &=100\pi\\
  &=314,2\mbox{ cm}^2
\end{align*}
}
\end{wex}


\begin{wex}
{Examining the surface area of a cone}
{If a cone has a height of $h$ and a base of radius $r$, show that the surface area is:\\ $\pi r^2 + \pi r \sqrt{r^2+h^2}$.}
{
\westep{Sketch and label the cone}
\begin{center}

\scalebox{1} % Change this value to rescale the drawing.
{
\begin{pspicture}(0,-2.07)(6.5141115,2.0841112)
\psellipse[linewidth=0.028222222,dimen=outer](1.5000001,-1.0699999)(1.5000001,1.0)
\psellipse[linewidth=0.028222222,linestyle=dotted,dotsep=0.10583334cm,dimen=outer](1.5,-1.07)(1.5,1.0)
\psline[linewidth=0.028222222](0.02,-0.93)(1.52,2.07)(2.985889,-0.9158889)
\psline[linewidth=0.028222222cm,arrowsize=0.05291667cm 2.0,arrowlength=1.4,arrowinset=0.4]{<->}(0.0,-1.07)(1.5,-1.07)
\usefont{T1}{ptm}{m}{n}
\rput(0.6890625,-1.35){$r$}
\psline[linewidth=0.028222222cm,arrowsize=0.05291667cm 2.0,arrowlength=1.4,arrowinset=0.4]{<->}(1.5,-1.07)(1.5258889,1.9641111)
\usefont{T1}{ptm}{m}{n}
\rput(1.7090626,0.37){$h$}
\psline[linewidth=0.028222222](1.1,-1.07)(1.1,-0.67)(1.5,-0.67)
\pspolygon[linewidth=0.028222222](3.5,-1.07)(5.0,1.93)(6.5,-1.07)
\psline[linewidth=0.028222222cm,arrowsize=0.05291667cm 2.0,arrowlength=1.4,arrowinset=0.4]{<->}(3.5,-1.27)(5.0,-1.27)
\usefont{T1}{ptm}{m}{n}
\rput(4.2690625,-1.57){$r$}
\psline[linewidth=0.028222222cm,arrowsize=0.05291667cm 2.0,arrowlength=1.4,arrowinset=0.4]{<->}(5.0,-1.07)(5.0,1.93)
\usefont{T1}{ptm}{m}{n}
\rput(5.2745314,0.25){$h$}
\usefont{T1}{ptm}{m}{n}
\rput(3.9490623,0.63){$a$}
\psline[linewidth=0.028222222](4.6,-1.07)(4.6,-0.67)(5.0,-0.67)
\end{pspicture} 
}
\end{center}


\westep{Identify the faces that make up the cone}
The cone has two faces: the base and the walls. The base is a circle of radius $r$ and the walls can be opened out to a sector of a circle:
\begin{center}
 \scalebox{1} % Change this value to rescale the drawing. 
{ \begin{pspicture}(0,-1.2046875)(7.8553123,1.1646875) 
\psline[linewidth=0.04cm](1.24,1.1046875)(0.02,-0.3753125) 
\psline[linewidth=0.04cm](1.22,1.1046875)(2.44,-0.3753125) 
\psarc[linewidth=0.04](1.24,1.0046875){1.84}{228.81407}{312.27368} 
\psline[linewidth=0.04cm](1.24,1.0846875)(0.18,-0.4953125) 
\psline[linewidth=0.04cm](1.24,1.0846875)(0.34,-0.5953125) 
\psline[linewidth=0.04cm](1.22,1.1046875)(0.5,-0.6753125) 
\psline[linewidth=0.04cm](1.22,1.1046875)(0.7,-0.7353125) 
\psline[linewidth=0.04cm](1.22,1.0846875)(0.92,-0.8153125) 
\psline[linewidth=0.04cm](3.98,-0.6753125)(7.6,-0.6753125) 
\psline[linewidth=0.04cm](3.98,-0.6753125)(4.16,1.1246876) 
\psline[linewidth=0.04cm](4.16,1.1246876)(4.24,-0.6753125) 
\psline[linewidth=0.04cm](4.24,-0.6553125)(4.42,1.1446875) 
\psline[linewidth=0.04cm](4.42,1.1446875)(4.5,-0.6553125) 
\psline[linewidth=0.04cm](4.5,-0.6953125)(4.68,1.1046875) 
\psline[linewidth=0.04cm](4.68,1.1046875)(4.76,-0.6953125) 
\psline[linewidth=0.04cm](4.76,-0.6753125)(4.94,1.1246876) 
\psline[linewidth=0.04cm](4.94,1.1246876)(5.02,-0.6753125) 
\psline[linewidth=0.04cm](5.02,-0.6753125)(5.2,1.1246876) 
\psline[linewidth=0.04cm](5.2,1.1246876)(5.28,-0.6753125) 
\psline[linewidth=0.04cm](5.28,-0.6753125)(5.46,1.1246876) 
\psline[linewidth=0.04cm](5.46,1.1246876)(5.54,-0.6753125) 
\psline[linewidth=0.04cm](3.96,-0.6753125)(5.58,-0.6753125) 
\psline[linewidth=0.04cm](5.54,-0.6753125)(5.72,1.1246876) 
\psline[linewidth=0.04cm](5.72,1.1246876)(5.8,-0.6753125) 
\psline[linewidth=0.04cm](5.8,-0.6753125)(5.98,1.1246876) 
\psline[linewidth=0.04cm](5.98,1.1246876)(6.06,-0.6753125) 
\psline[linewidth=0.04cm](6.06,-0.6953125)(6.24,1.1046875) 
\psline[linewidth=0.04cm](6.24,1.1046875)(6.32,-0.6953125) 
\psline[linewidth=0.04cm](6.32,-0.6553125)(6.5,1.1446875) 
\psline[linewidth=0.04cm](6.5,1.1446875)(6.58,-0.6553125) 
\psline[linewidth=0.04cm](6.58,-0.6753125)(6.76,1.1246876) 
\psline[linewidth=0.04cm](6.76,1.1246876)(6.84,-0.6753125) 
\psline[linewidth=0.04cm](6.84,-0.6753125)(7.02,1.1246876) 
\psline[linewidth=0.04cm](7.02,1.1246876)(7.1,-0.6753125) 
\psline[linewidth=0.04cm](7.1,-0.6753125)(7.28,1.1246876) 
\psline[linewidth=0.04cm](7.28,1.1246876)(7.36,-0.6753125) 
\psline[linewidth=0.04cm](7.36,-0.6553125)(7.54,1.1446875) 
\psline[linewidth=0.04cm](7.54,1.1446875)(7.62,-0.6553125) 
\psline[linewidth=0.03cm,linestyle=dashed,dash=0.16cm 0.16cm](1.22,1.0846875)(1.16,-0.8153125) 
\psline[linewidth=0.03cm,linestyle=dashed,dash=0.16cm 0.16cm](1.22,1.1246876)(1.42,-0.8553125) 
\psline[linewidth=0.03cm,linestyle=dashed,dash=0.16cm 0.16cm](1.22,1.0846875)(1.66,-0.7553125) 
\psline[linewidth=0.03cm,linestyle=dashed,dash=0.16cm 0.16cm](1.22,1.0846875)(1.94,-0.6553125) 
\psline[linewidth=0.03cm,linestyle=dashed,dash=0.16cm 0.16cm](1.2,1.0846875)(2.22,-0.5353125) 
\psline[linewidth=0.11cm,arrowsize=0.05291667cm 2.0,arrowlength=1.4,arrowinset=0.4]{->}(2.46,0.4246875)(3.04,0.4246875) 
\rput(3.6859374,0.4146875){$a$} 
\psline[linewidth=0.03cm,linestyle=dashed,dash=0.16cm 0.16cm,arrowsize=0.05291667cm 2.0,arrowlength=1.4,arrowinset=0.4]{->}(3.68,0.2646875)(3.68,-0.6753125) 
\psline[linewidth=0.03cm,linestyle=dashed,dash=0.16cm 0.16cm,arrowsize=0.05291667cm 2.0,arrowlength=1.4,arrowinset=0.4]{->}(3.68,0.5446875)(3.68,1.1246876) 
\rput(5.6826563,-0.9853125){$2\pi r$ = circumference} \end{pspicture} } 
\end{center}
\\ This curved surface can be cut into many thin triangles with height close to $a$ (where $a$ is the slant height). The area of these triangles or sectors can be summed as follows: 
\begin{align*}
  \mbox{Area of sector}
  &= \frac{1}{2}\times \mbox{base} \times \mbox{height (of a small triangle)}\\
  &=\frac{1}{2}\times2\pi r \times a \\
  &= \pi r a 
\end{align*}

\westep{Calculate $a$}
$a$ can be calculated using the Theorem of Pythagoras:
\begin{equation*}
  a = \sqrt{r^{2} + h^{2}}
\end{equation*}

\westep{Calculate the area of the circular base ($A_{b}$)}
\begin{equation*}
  A_{b} = \pi r^{2}
\end{equation*}

\westep{Calculate the area of the curved walls ($A_{w}$)}
\begin{align*}
  A_{w} &= \pi r a \\
  &= \pi r \sqrt{r^{2}+h^{2}}
\end{align*}

\westep{Find the sum of the areas $A$}
\begin{align*}
  A &= A_{b} + A_{w} \\
  &= \pi r^{2} + \pi r \sqrt{r^{2}+h^{2}}\\
  &= \pi r(r + \sqrt{r^{2}+h^{2}})
\end{align*}
}
\end{wex}

\subsection{Volume of pyramids, cones and spheres}

\begin{table}[H]
\newcolumntype{C}{>{\centering\arraybackslash} m{0.6in} }
\newcolumntype{D}{>{\centering\arraybackslash} m{2in} }
\newcolumntype{E}{>{\centering\arraybackslash} m{6cm} }
\begin{tabular}{|C|D|E|}
\hline
\textbf{Square pyramid}
&
\begin{center}
\scalebox{1} % Change this value to rescale the drawing.
{
\scalebox{1} % Change this value to rescale the drawing.
{
\begin{pspicture}(0,-1.9993507)(4.767494,1.9734619)
\pspolygon[linewidth=0.028222222,fillstyle=solid](0.095092274,-0.42983752)(2.4242375,1.9593507)(3.794405,-0.18333401)
\pspolygon[linewidth=0.028222222,fillstyle=solid](4.753383,-1.2623826)(2.4242375,1.9593507)(1.0540702,-1.5190883)
\pspolygon[linewidth=0.028222222,fillstyle=solid](1.0593361,-1.5190883)(2.3858888,1.9269619)(0.0,-0.46889564)
\psline[linewidth=0.022cm,linestyle=dashed,dash=0.16cm 0.16cm](0.0,-0.46889564)(3.0107446,-0.23551947)
\psline[linewidth=0.027999999,linestyle=dotted,dotsep=0.16cm](2.465889,1.8869619)(2.3658888,-0.8130381)(3.305889,-1.3465382)(2.445889,1.8469619)(2.485889,1.8069619)
\psline[linewidth=0.024cm,linestyle=dashed,dash=0.16cm 0.16cm](3.025889,-0.2530381)(4.7258887,-1.2530382)
\psline[linewidth=0.02](2.3658888,-0.6130381)(2.545889,-0.6930381)(2.545889,-0.91303813)(2.3658888,-0.8330382)
\psline[linewidth=0.02cm](2.3658888,-0.6330381)(2.3658888,-0.8330382)
\psline[linewidth=0.04cm](0.3058889,-0.9465382)(0.70588887,-0.9465382)
\psline[linewidth=0.04cm](2.8458889,-1.5265381)(2.6258888,-1.2465382)
\psline[linewidth=0.04cm](3.6458888,-0.7065382)(4.025889,-0.7265382)
\psline[linewidth=0.04cm](1.9058889,-0.44653818)(1.6858889,-0.20653819)
\usefont{T1}{ppl}{m}{n}
\rput(2.9449513,-1.7965382){$b$}
\usefont{T1}{ptm}{m}{n}
\rput(2.1745312,0.2834618){$H$}
\end{pspicture} 
}

}
\end{center} 
&
\begin{equation*}
  \begin{array}{r@{\;}c@{\;}l}
    \mbox{Volume} &=& \frac{1}{3} \times \mbox{area of base } \times\\
    && \mbox{height of pyramid} \\
    &=& \frac{1}{3}\times b^{2} \times H
  \end{array}
\end{equation*}
\\ \hline


\textbf{Triangular pyramid} &
\begin{center}
\scalebox{0.8} % Change this value to rescale the drawing.
{
\begin{pspicture}(0,-2.0793507)(4.2198644,2.0534618)
\pspolygon[linewidth=0.028222222,fillstyle=solid](0.2523495,-0.43995062)(1.5305833,2.0138543)(4.1218147,-1.1400807)
\pspolygon[linewidth=0.028222222,fillstyle=solid](0.24,-1.8206493)(1.5,1.9393507)(0.034111112,-0.43957832)
\pspolygon[linewidth=0.028222222,fillstyle=solid](4.2057533,-1.2329769)(1.5316236,2.0393507)(0.24804147,-1.8465381)
\psline[linewidth=0.022cm,linestyle=dashed,dash=0.16cm 0.16cm](0.034111112,-0.43957832)(4.1522703,-1.1765573)
\psline[linewidth=0.024,linestyle=dotted,dotsep=0.16cm](1.5199999,1.9793508)(1.48,-1.1806492)(4.1800003,-1.2206491)
\psline[linewidth=0.02](1.48,-0.9606492)(1.7,-0.9606492)(1.7,-1.1865381)(1.48,-1.1806492)(1.48,-0.9606492)
\psline[linewidth=0.04cm](1.86,-0.8465382)(2.02,-0.68653816)
\psline[linewidth=0.04cm](1.72,-1.5265383)(1.76,-1.7465383)
\psline[linewidth=0.04cm](0.24,-1.0865382)(0.0,-1.1265383)
\psline[linewidth=0.024cm,linestyle=dotted,dotsep=0.16cm](1.56,-1.1865381)(0.12,-1.1465383)
\usefont{T1}{ppl}{m}{n}
\rput(2.0790625,-1.8765383){$b$}
\usefont{T1}{ppl}{m}{n}
\rput(0.99906254,-1.3965381){$h$}
\usefont{T1}{ppl}{m}{n}
\rput(1.2545313,0.3434618){$H$}
\end{pspicture} 
}
\end{center}
&
\begin{equation*}
  \begin{array}{r@{\;}c@{\;}l}
    \mbox{Volume} &=& \frac{1}{3} \times \mbox{area of base } \times \\
    && \mbox{height of pyramid} \\
    &=& \frac{1}{3} \times \frac{1}{2}bh \times H
  \end{array}
\end{equation*}
 \\ \hline

\textbf{Right cone} &
\begin{center}
 \scalebox{0.7} % Change this value to rescale the drawing.
{
\begin{pspicture}(0,-3.1099448)(4.5657654,3.1100557)
\psline[linewidth=0.028222222](0.24603535,-1.5211127)(2.3344314,3.0959444)(4.551654,-1.6285512)
\psbezier[linewidth=0.027999999](0.3058355,-1.3951261)(0.0,-2.088052)(0.9226326,-2.969958)(2.0297916,-3.0329514)(3.1369507,-3.0959444)(4.551654,-2.5919983)(4.551654,-1.584106)
\psbezier[linewidth=0.022,linestyle=dashed,dash=0.1cm 0.1cm](4.551654,-1.6470991)(4.182601,-1.0801597)(3.5060039,-0.70007807)(2.6143408,-0.6696424)(1.7226781,-0.6392067)(0.9841414,-0.7022)(0.24603535,-1.584106)
\psline[linewidth=0.04,linestyle=dotted,dotsep=0.1cm](2.3444264,3.0043032)(2.261968,-1.7245709)(4.488356,-1.7523878)(4.488356,-1.6133032)(4.46087,-1.6133032)
\psframe[linewidth=0.04,dimen=outer](2.6258888,-1.3299444)(2.225889,-1.7299443)
% \usefont{T1}{ptm}{m}{n}
% \rput(3.8704202,1.0400556){$h$}
% \usefont{T1}{ptm}{m}{n}
\rput(3.56042,-1.4399444){$r$}
% \usefont{T1}{ptm}{m}{n}
\rput(1.8404201,0.24005564){$H$}
\end{pspicture} 
}
\end{center}
&
\begin{equation*}
  \begin{array}{r@{\;}c@{\;}l}
    \mbox{Volume} &=& \frac{1}{3} \times \mbox{area of base } \times \\
    && \mbox{height of cone} \\
    &=& \frac{1}{3} \times \pi r^2 \times H
  \end{array}
\end{equation*}
\\ \hline

\textbf{Sphere} &
\begin{center}
\scalebox{0.8} % Change this value to rescale the drawing.
{
\begin{pspicture}(0,-1.9)(3.8,1.9)
\pscircle[linewidth=0.027999999,dimen=outer](1.9,0.0){1.9}
\psellipse[linewidth=0.027999999,linestyle=dashed,dash=0.16cm 0.16cm,dimen=outer](1.9,-0.13)(1.84,0.23)
\psline[linewidth=0.04,linestyle=dotted,dotsep=0.1cm](1.96,-0.14)(3.74,-0.14)
\psdots[dotsize=0.09](1.92,-0.16)
\usefont{T1}{ppl}{m}{n}
\rput(2.6345313,0.2){$r$}
\end{pspicture} 
}
\end{center}
&
$\mbox{Volume} = \frac{4}{3}\pi r^{3}$
\\ \hline
\end{tabular}
\end{table}

\begin{wex}{Finding the volume of a square pyramid}
{Find the volume of a square pyramid with a height of $3$ cm and a side length of $2$ cm.}
{
\westep{Sketch and label the pyramid}
\begin{center}
\scalebox{0.8} % Change this value to rescale the drawing.
{
\begin{pspicture}(0,-2.89)(5.58,2.89)
% \usefont{T1}{ptm}{m}{n}
\rput{0.6029805}(-0.022694819,-0.047077317){\rput(4.4319425,-2.1803145){$2~$cm}}
% \usefont{T1}{ptm}{m}{n}
\rput{-1.0300905}(0.040004794,0.011890239){\rput(1,-2.2194574){$2~$cm}}
% \usefont{T1}{ptm}{m}{n}
\rput(2.0929687,1.255){\small $3~$cm}
\psline[linewidth=0.04cm](5.54,-0.97)(2.74,2.85)
\psline[linewidth=0.04cm](2.9,-2.85)(2.74,2.87)
\psline[linewidth=0.04cm](2.74,2.87)(0.02,-0.87)
\psline[linewidth=0.04cm](0.04,-0.87)(2.78,0.89)
\psline[linewidth=0.04cm](2.76,0.89)(5.56,-0.99)
\psline[linewidth=0.04cm](0.0,-0.87)(2.9,-2.87)
\psline[linewidth=0.04cm](2.9,-2.87)(5.56,-0.99)
\psdots[dotsize=0.12](2.62,-0.83)
\psline[linewidth=0.04cm,linestyle=dashed,dash=0.17638889cm 0.10583334cm](2.64,-0.81)(2.72,2.75)
\psline[linewidth=0.04cm,linestyle=dashed,dash=0.17638889cm 0.10583334cm](2.62,-0.85)(1.44,0.03)
\psline[linewidth=0.04cm](2.4,-0.69)(2.4,-0.33)
\psline[linewidth=0.04cm](2.38,-0.33)(2.66,-0.53)
\end{pspicture} 
}
\end{center}
\westep{Select the correct formula and substitute the given values}
\begin{equation*}
  \mbox{volume} = \frac{1}{3} \times b^{2} \times H
\end{equation*}
We are given $b=2$ and $H=3$, therefore
\begin{align*}
  \mbox{volume} &= \frac{1}{3} \times 2^{2} \times 3 \\
  &= \frac{1}{3} \times 12 \\
  &= 4\mbox{ cm}^3
\end{align*}
\westep{Write the final answer}
The volume of the square pyramid is $4\mbox{ cm}^3$.
}
\end{wex}


\begin{wex}{Finding the volume of a triangular pyramid}
{Find the volume of the following triangular pyramid (correct to $1$ decimal place):\\
\begin{center}
\scalebox{1} % Change this value to rescale the drawing.
{
\begin{pspicture}(0,-2.0893507)(4.2198644,2.0634618)
\pspolygon[linewidth=0.028222222](0.24,-1.8106492)(1.5,1.9493507)(0.034111112,-0.42957833)
\pspolygon[linewidth=0.028222222](4.2057533,-1.222977)(1.5316236,2.0493507)(0.24804147,-1.8365381)
\psline[linewidth=0.022cm,linestyle=dashed,dash=0.16cm 0.16cm](0.034111112,-0.42957833)(4.1522703,-1.1665573)
\psline[linewidth=0.024,linestyle=dotted,dotsep=0.16cm](1.5199999,1.9893508)(1.48,-1.1706492)(4.1800003,-1.2106491)
\psline[linewidth=0.02](1.48,-0.9506492)(1.7,-0.9506492)(1.7,-1.1765381)(1.48,-1.1706492)(1.48,-0.9506492)
\psline[linewidth=0.04cm](1.86,-0.8365382)(2.02,-0.67653817)
\psline[linewidth=0.04cm](2.1,-1.4565382)(2.14,-1.6765382)
\psline[linewidth=0.04cm](0.24,-1.0765382)(0.0,-1.1165383)
\psline[linewidth=0.04cm,linestyle=dotted,dotsep=0.16cm](1.54,2.0434618)(2.5,-1.4965383)
\psline[linewidth=0.02](2.44,-1.2365383)(2.496,-1.4765382)(2.72,-1.4365381)(2.66,-1.1965381)(2.44,-1.2365383)
\psline[linewidth=0.024cm,linestyle=dotted,dotsep=0.16cm](1.56,-1.1765381)(0.12,-1.1365383)
% \usefont{T1}{ppl}{m}{n}
\rput(2.6495311,0.09346183){$12$ cm}
% \usefont{T1}{ppl}{m}{n}
\rput(2.3795314,-1.8865383){$8$ cm}
% \usefont{T1}{ppl}{m}{n}
\rput(1.3590626,-1.2865381){$h_b$}
% \usefont{T1}{ppl}{m}{n}
\rput(1.2745312,0.31000006){$H$}
\end{pspicture} 
}
\end{center}
}
{
\westep{Sketch the base triangle and calculate its area}
\begin{center}
\scalebox{0.9} % Change this value to rescale the drawing.
{
\begin{pspicture}(0,-2.1564062)(4.64,2.097329)
\pstriangle[linewidth=0.04,dimen=outer](2.24,-1.6435938)(4.48,3.76)
\psline[linewidth=0.04cm,linestyle=dotted,dotsep=0.16cm](2.28,1.9164063)(2.16,-1.6035937)
\psframe[linewidth=0.04,dimen=outer](2.5,-1.2635938)(2.14,-1.6235938)
% \usefont{T1}{ptm}{m}{n}
\rput(3.165,-1.9535937){$4$ cm}
% \usefont{T1}{ptm}{m}{n}
\rput(3.945,0.62640625){$8$ cm}
% \usefont{T1}{ptm}{m}{n}
\rput(2.5845313,0.04640625){$h_b$}
\end{pspicture} 
}
\end{center}
The height of the base triangle ($h_b$) can be calculated using the Theorem of Pythagoras:
\begin{align*}
  8^2 &= 4^2+h_b^2\\
  \therefore h_b&=\sqrt{8^2-4^2}\\
  &=4\sqrt{3}\mbox{ cm}^2
\end{align*}

\begin{align*}
  \therefore \mbox{ area of base triangle} &= \frac{1}{2} b \times h_b \\
  &= \frac{1}{2} \times 8 \times 4\sqrt{3} \\
  &= 16\sqrt{3}\mbox{ cm}^2
\end{align*}


\westep{Sketch the side triangle and calculate pyramid height $H$}

\begin{center}
\scalebox{0.9} % Change this value to rescale the drawing.
{
\begin{pspicture}(0,-2.1564062)(4.64,2.097329)
\pstriangle[linewidth=0.04,dimen=outer](2.24,-1.6435938)(4.48,3.76)
\psline[linewidth=0.04cm,linestyle=dotted,dotsep=0.16cm](2.28,1.9164063)(2.16,-1.6035937)
\psframe[linewidth=0.04,dimen=outer](2.5,-1.2635938)(2.14,-1.6235938)
% \usefont{T1}{ptm}{m}{n}
\rput(3.165,-1.9535937){$4$ cm}
% \usefont{T1}{ptm}{m}{n}
\rput(3.945,0.62640625){$12$ cm}
% \usefont{T1}{ptm}{m}{n}
\rput(2.5845313,0.04640625){$H$}
\end{pspicture} 
}
\end{center}
\begin{align*}
  12^2 &= 4^2+H^2 \\
  \therefore H &=\sqrt{12^2-4^2} \\
  &= 8\sqrt{2}\mbox{ cm}^2
\end{align*}

\westep{Calculate the volume of the pyramid}
\begin{align*}
  \mbox{volume}
  &= \frac{1}{3} \times \frac{1}{2}bh_b \times H \\ 
  &= \frac{1}{3} \times 16\sqrt{3} \times 8\sqrt{2} \\
  &= 104,5\mbox{ cm}^3
\end{align*}

\westep{Write the final answer}
The volume of the triangular pyramid is $104,5\mbox{ cm}^3$.
}
\end{wex}

\begin{wex}{Finding the volume of a cone}
{Find the volume of the following cone (correct to $1$ decimal place):
\begin{center}
 \scalebox{0.8} % Change this value to rescale the drawing.
{
\begin{pspicture}(0,-3.1099448)(4.5657654,3.1100557)
\psline[linewidth=0.028222222](0.24603535,-1.5211127)(2.3344314,3.0959444)(4.551654,-1.6285512)
\psbezier[linewidth=0.027999999](0.3058355,-1.3951261)(0.0,-2.088052)(0.9226326,-2.969958)(2.0297916,-3.0329514)(3.1369507,-3.0959444)(4.551654,-2.5919983)(4.551654,-1.584106)
\psbezier[linewidth=0.022,linestyle=dashed,dash=0.1cm 0.1cm](4.551654,-1.6470991)(4.182601,-1.0801597)(3.5060039,-0.70007807)(2.6143408,-0.6696424)(1.7226781,-0.6392067)(0.9841414,-0.7022)(0.24603535,-1.584106)
\psline[linewidth=0.04,linestyle=dotted,dotsep=0.1cm](2.3444264,3.0043032)(2.261968,-1.7245709)(4.488356,-1.7523878)(4.488356,-1.6133032)(4.46087,-1.6133032)
\psframe[linewidth=0.04,dimen=outer](2.6258888,-1.3299444)(2.225889,-1.7299443)
% \usefont{T1}{ptm}{m}{n}
% \rput(3.8704202,1.0400556){$h$}
% \usefont{T1}{ptm}{m}{n}
\rput(3.56042,-1.4399444){$3$ cm}
% \usefont{T1}{ptm}{m}{n}
\rput(1.8404201,0.24005564){$11$ cm}
\end{pspicture} 
}
\end{center}
}
{
\westep{Find the area of the base}
\begin{align*}
  \mbox{area of circle} &= \pi r^2 \\
  &= \pi \times 3^2 \\
  &= 9\pi\mbox{ cm}^2
\end{align*}

\westep{Calculate the volume}
\begin{align*}
  \mbox{volume}
  &= \frac{1}{3} \times \pi r^{2} \times H \\
  &= \frac{1}{3} \times 9\pi \times 11 \\
  &= 103,7\mbox{ cm}^3
\end{align*}

\westep{Write the final answer}
The volume of the cone is $ 103,7\mbox{ cm}^3$.
}
\end{wex}

\begin{wex}{Finding the volume of a sphere}
{Find the volume of the following sphere (correct to $1$ decimal place):
\begin{center}
\scalebox{0.9} % Change this value to rescale the drawing.
{
\begin{pspicture}(0,-1.9)(3.8,1.9)
\definecolor{color129b}{rgb}{0.9490196078431372,0.9490196078431372,0.9490196078431372}
\pscircle[linewidth=0.027999999,dimen=outer](1.9,0.0){1.9}
\psellipse[linewidth=0.027999999,linestyle=dashed,dash=0.16cm 0.16cm,dimen=outer](1.9,-0.13)(1.84,0.23)
\psline[linewidth=0.04cm,linestyle=dotted,dotsep=0.15cm](1.96,-0.14)(3.74,-0.14)
\psdots[dotsize=0.09](1.92,-0.16)
% \usefont{T1}{ptm}{m}{n}
\rput(2.7095313,0.12){\psframebox[linewidth=0.002,linecolor=color129b,fillstyle=solid,fillcolor=color129b]{$4$ cm}}%this framebox background colour is tweaked to match the wex colour - otherwise radius label sits too close between the dotted lines. framebox behind label blanks out some of eliptical dotted line to make it more legible
\end{pspicture} 
}
\end{center}
}
{
\westep{Use the formula to find the volume}
\begin{align*}
  \mbox{volume} &= \frac{4}{3} \pi r^3 \\
  &= \frac{4}{3}\pi(4)^3 \\
  &= 268,1\mbox{ cm}^3
\end{align*}

\westep{Write the final answer}
The volume of the sphere is $268,1\mbox{ cm}^3$
}
\end{wex}

\begin{wex}{Finding the volume of a complex object}
{A triangular pyramid is placed on top of a triangular prism, as shown
  below. The base of the prism is an equilateral triangle of side
  length $20$ cm and the height of the prism is $42$ cm. The pyramid
  has a height of $12$ cm. Answer the following questions (correct to
  $1$ decimal place):
\begin{enumerate}[noitemsep, label=\textbf{\arabic*}. ] 
\item Calculate the total volume of the object.
\item Calculate the surface area of each exposed face of the pyramid.
\item Calculate the total surface area of the object.
\end{enumerate}
\begin{center}
\scalebox{1} % Change this value to rescale the drawing.
{
\begin{pspicture}(0,-2.0664062)(5.57,2.0464063)
\definecolor{color338b}{rgb}{0.6,0.6,0.6}
\definecolor{color376b}{rgb}{0.8,0.8,0.8}
\pspolygon[linewidth=0.04,fillstyle=solid,fillcolor=color338b](3.45,0.6053162)(3.87,1.7664063)(3.87,-0.6525037)(3.45,-1.8135937)
\pspolygon[linewidth=0.04,fillstyle=solid,fillcolor=color338b](3.03,2.0264063)(3.87,1.8321205)(3.45,0.6064063)
\pspolygon[linewidth=0.04,fillstyle=solid,fillcolor=color376b](3.03,2.0264063)(2.23,1.3264062)(3.43,0.62640625)
\pspolygon[linewidth=0.04,fillstyle=solid,fillcolor=color376b](2.25,1.3264062)(3.45,0.62640625)(3.45,-1.8735938)(2.25,-1.1735938)
\psline[linewidth=0.04cm,linestyle=dashed,dash=0.16cm 0.16cm](2.25,1.3664062)(3.83,1.8064063)
\psline[linewidth=0.04cm,linestyle=dashed,dash=0.16cm 0.16cm](2.25,-1.1735938)(3.83,-0.61359376)
\psline[linewidth=0.04cm](3.03,-0.75359374)(3.13,-0.9735938)
\psline[linewidth=0.04cm](2.93,-1.4135938)(2.77,-1.6135937)
\psline[linewidth=0.04cm](3.53,-1.1335938)(3.77,-1.1935937)
% \usefont{T1}{ptm}{m}{n}
\rput(2.265,-1.8635937){$20$ cm}
% \usefont{T1}{ptm}{m}{n}
\rput(4.765,0.5964062){$42$ cm}
\psline[linewidth=0.03,linestyle=dotted,dotsep=0.1cm](3.03,1.9664062)(3.05,1.2664063)(2.79,1.0464063)(2.79,1.0464063)(2.79,1.0464063)(2.75,1.0464063)
\psline[linewidth=0.04cm](3.05,1.6064062)(1.51,1.6064062)
% \usefont{T1}{ptm}{m}{n}
\rput(0.705,1.6164062){$12$ cm}
\psline[linewidth=0.02](3.09,1.4064063)(2.97,1.3464062)(2.97,1.2064062)
\end{pspicture} 
}
\end{center}
}
{
\westep{Calculate the volume of the object}
a) Calculate the volume of the prism\\
\\
i) Find the height of the base triangle
\begin{center}
 \scalebox{0.8}
{
\begin{pspicture}(0,-2.1564062)(4.64,2.097329)
\pstriangle[linewidth=0.04,dimen=outer](2.24,-1.6435938)(4.48,3.76)
\psline[linewidth=0.04cm,linestyle=dotted,dotsep=0.16cm](2.28,1.9164063)(2.16,-1.6035937)
\psframe[linewidth=0.04,dimen=outer](2.5,-1.2635938)(2.14,-1.6235938)
% \usefont{T1}{ptm}{m}{n}
\rput(3.165,-1.9535937){\LARGE $10$ cm}
% \usefont{T1}{ptm}{m}{n}
\rput(3.945,0.62640625){\LARGE $20$ cm}
% \usefont{T1}{ptm}{m}{n}
\rput(2.5845313,0.04640625){\LARGE $h_b$}
\end{pspicture} 
}
\end{center}

\begin{align*}
  20^2 &= 10^2 + h_b^2 \\
  \therefore h_b &= \sqrt{20^2-10^2} \\
  &= 10 \sqrt{3}\mbox{ cm}^3
\end{align*}

ii) Find the area of the base triangle
% \begin{center}
% 
% \scalebox{0.8} % Change this value to rescale the drawing.
% {
% \begin{pspicture}(0,-2.1564062)(4.64,2.097329)
% \pstriangle[linewidth=0.04,dimen=outer](2.24,-1.6435938)(4.48,3.76)
% \psline[linewidth=0.04cm,linestyle=dotted,dotsep=0.16cm](2.28,1.9164063)(2.16,-1.6035937)
% \psframe[linewidth=0.04,dimen=outer](2.5,-1.2635938)(2.14,-1.6235938)
% % \usefont{T1}{ptm}{m}{n}
% \rput(3.165,-1.9535937){\LARGE $4$ cm}
% % \usefont{T1}{ptm}{m}{n}
% \rput(3.945,0.62640625){\LARGE $12$ cm}
% % \usefont{T1}{ptm}{m}{n}
% % \rput(2.5845313,0.04640625){$H$}
% \end{pspicture} 
% }
% \end{center}

\begin{align*}
  \mbox{area of base triangle}
  &= \frac{1}{2} \times 20 \times 10 \sqrt{3} \\
  &= 100 \sqrt{3}
\end{align*}

iii) Find the volume of the prism
\begin{align*}
  \therefore \mbox{volume of prism}
  &= \mbox{area of base triangle} \times \mbox{height of prism} \\
  &= 100 \sqrt{3} \times 42 \\
  &= 4~200\,\sqrt{3}\mbox{ cm}^3
\end{align*}

b) Find the volume of the pyramid\\
\\
The area of the base triangle is equal to the area of the base of the pyramid.

\begin{align*}
  \therefore \mbox{volume of pyramid}
  &= \frac{1}{3} (\mbox{area of base}) \times H \\
  &= \frac{1}{3} \times 100 \sqrt{3} \times 12 \\
  &= 400\sqrt{3}\mbox{ cm}^3
\end{align*}
c) Calculate the total volume
\begin{align*}
  \mbox{total volume}
  &= 4~200\,\sqrt{3}+400\sqrt{3} \\
  &= 4~600\,\sqrt{3} \\
  &= 7~967,4\mbox{ cm}^3
\end{align*}

Therefore the total volume of the object is $7~967,4\mbox{ cm}^3$.

\westep{Calculate the surface area of each exposed face of the pyramid}
a) To calculate the surface area of the sides of the pyramid, we need to calculate its slant height ($h_s$)
\begin{center}
 \scalebox{0.9}
{
\begin{pspicture}(0,-2.1564062)(4.64,2.097329)
\pstriangle[linewidth=0.04,dimen=outer](2.24,-1.6435938)(4.48,3.76)
\psline[linewidth=0.04cm,linestyle=dotted,dotsep=0.16cm](2.28,1.9164063)(2.16,-1.6035937)
\psframe[linewidth=0.04,dimen=outer](2.5,-1.2635938)(2.14,-1.6235938)
% \usefont{T1}{ptm}{m}{n}
\rput(2.5,-1.9535937){ $h_b$ = $10\sqrt{3}$ cm}
% \usefont{T1}{ptm}{m}{n}
\rput(3.945,0.62640625){ $h_s$}
% \usefont{T1}{ptm}{m}{n}
\rput(2.35,0.04640625){$H = 12$ cm}
\end{pspicture} 
}
\end{center}
b) We can calculate $h_s$ by bisecting the pyramid and using the Theorem of Pythagoras

\begin{align*}
  h_s^2 &= \left(\frac{h_b}{2}\right)^{2} + H^{2} \\
  &= \left(\frac{10\sqrt{3}}{2}\right)^{2} + 12^{2} \\
  &= 219 \\
  \therefore h_s &= \sqrt{219}\mbox{ cm}
\end{align*}

c) We can now calculate the area of each face of the pyramid
\begin{align*}
  \mbox{area of one pyramid face}
  &= \frac{1}{2}b \times h_s \\
  &= \frac{1}{2} \times 20 \times \sqrt{219} \\
  &= 10\sqrt{219}\mbox{ cm}^2
\end{align*}
Because the base triangle is equilateral, each face has the same base, and therefore the same surface area. \\
Therefore the surface area for each face of the pyramid is $148$ cm$^{2}$.

\westep{Calculate the total surface area of the object}
We need to calculate the surface area of the sides of the prism. This
is made up of $3$ rectangles with base $b = 20$ cm and height $h_p =
42$ cm.
\\
\\
a) Calculate the  area of the sides of the prism
\begin{align*}
  \mbox{area of prism sides}
  &= 3(b \times h_p) \\
  &= 3(20 \times 42) \\
  &= 2~520\mbox{ cm}^2
\end{align*}


b) Calculate the total surface area of the object
\begin{equation*}
  \begin{array}{r@{\;}c@{\;}l}
    \mbox{total surface area} &=& \mbox{area of base of prism} + \mbox{area of sides of prism } + \\
                              & & \mbox{area of exposed faces of pyramid} \\
    &=& (100 \sqrt{3}) + 2~520 + 3(10\sqrt{219}) \\
    &=& 3~137,2\mbox{ cm}^3
  \end{array}
\end{equation*}
Therefore the total surface area (of the exposed faces) of the object is $3~137,2$ cm$^{2}$.
}
\end{wex}

\begin{exercises}{}{
\begin{enumerate}[itemsep=6pt, label=\textbf{\arabic*}. ] 
\item Find the total surface area of the following objects (correct to $1$ decimal place if necessary):
\begin{center}
\scalebox{0.8} % Change this value to rescale the drawing.
{
\begin{pspicture}(0,-6.555141)(11.58,6.555141)
\psline[linewidth=0.028222222](0.7205463,1.9239725)(2.8089423,6.5410295)(5.026165,1.816534)
\psline[linewidth=0.04,linestyle=dotted,dotsep=0.1cm](2.8189373,6.4493885)(2.736479,1.7205143)(4.962867,1.6926974)(4.962867,1.831782)(4.935381,1.831782)
% \usefont{T1}{ptm}{m}{n}
\rput(4.7499313,4.485141){$13$ cm}
\psbezier[linewidth=0.027999999](0.78034645,2.0499592)(0.47451097,1.3570333)(1.3971436,0.47512725)(2.5043025,0.41213384)(3.6114616,0.34914044)(5.026165,0.85308695)(5.026165,1.8609792)
\psbezier[linewidth=0.022,linestyle=dashed,dash=0.1cm 0.1cm](5.026165,1.7979861)(4.657112,2.3649256)(3.9805148,2.7450073)(3.0888517,2.7754428)(2.1971886,2.8058784)(1.4586524,2.7428854)(0.7205463,1.8609792)
\psframe[linewidth=0.04,dimen=outer](3.1003997,2.115141)(2.7003999,1.7151409)
% \usefont{T1}{ptm}{m}{n}
\rput(4.009931,2.0051408){$5$ cm}
\pspolygon[linewidth=0.028222222](7.28,1.7148592)(8.54,5.474859)(7.074111,3.09593)
\pspolygon[linewidth=0.028222222](11.245753,2.3025317)(8.571624,5.574859)(7.2880416,1.6889703)
\psline[linewidth=0.022cm,linestyle=dashed,dash=0.1cm 0.1cm](7.074111,3.09593)(11.19227,2.358951)
\psline[linewidth=0.04cm,linestyle=dotted,dotsep=0.1cm](8.56,5.574859)(9.22,1.9948592)
\psline[linewidth=0.024](9.18,2.2348592)(9.4,2.2748594)(9.44,2.0348594)(9.46,2.0348594)
\psline[linewidth=0.04cm](8.82,2.9148593)(8.76,2.6548593)
\psline[linewidth=0.04cm](6.96,2.4148595)(7.3,2.4748592)
\psline[linewidth=0.04cm](8.74,2.0548594)(8.92,1.7948594)
% \usefont{T1}{ptm}{m}{n}
\rput(9.395,1.5848593){$6$ cm}
% \usefont{T1}{ptm}{m}{n}
\rput(9.31,3.8848593){$10$ cm}
% \usefont{T1}{ptm}{m}{n}
\rput(1.0020312,5.3748593){\LARGE \textbf{(a)}}
% \usefont{T1}{ptm}{m}{n}
\rput(6.9220314,5.3748593){\LARGE\textbf{(b)}}
\pscircle[linewidth=0.027999999,dimen=outer](9.68,-3.4851408){1.9}
\psellipse[linewidth=0.027999999,linestyle=dashed,dash=0.16cm 0.16cm,dimen=outer](9.68,-3.6151407)(1.84,0.23)
\psline[linewidth=0.027999999cm,linestyle=dotted,dotsep=0.1cm](9.74,-3.6251407)(11.52,-3.6251407)
\psdots[dotsize=0.09](9.7,-3.6451406)
% \usefont{T1}{ppl}{m}{n}
\rput(10.499532,-3.1651409){$10$ cm}
% \usefont{T1}{ptm}{m}{n}
\rput{0.6029805}(-0.05992648,-0.0624806){\rput(5.8769474,-5.725771){$6$ cm}}
% \usefont{T1}{ptm}{m}{n}
\rput{-1.0300905}(0.10617459,0.030957164){\rput(1.7449429,-5.8903265){$6$ cm}}
% \usefont{T1}{ptm}{m}{n}
\rput(5.04,-2.8101408){$12$ cm}
\psline[linewidth=0.04cm](6.4,-4.635141)(3.6,-0.8151407)
\psline[linewidth=0.04cm](3.76,-6.5151405)(3.6,-0.79514074)
\psline[linewidth=0.04cm](3.6,-0.79514074)(0.88,-4.5351405)
\psline[linewidth=0.04cm](0.9,-4.5351405)(3.64,-2.7751408)
\psline[linewidth=0.04cm](3.62,-2.7751408)(6.42,-4.655141)
\psline[linewidth=0.04cm](0.86,-4.5351405)(3.76,-6.5351405)
\psline[linewidth=0.04cm](3.76,-6.5351405)(6.42,-4.655141)
\psline[linewidth=0.04cm,linestyle=dashed,dash=0.17638889cm 0.10583334cm](5.26,-5.4551406)(3.58,-0.91514075)
\psline[linewidth=0.04cm](5.16,-5.155141)(5.4,-4.9751406)
\psline[linewidth=0.04cm](5.38,-4.9751406)(5.5,-5.2951407)
% \usefont{T1}{ptm}{m}{n}
\rput(1.3220313,-1.3851408){\LARGE \textbf{(c)}}
% \usefont{T1}{ptm}{m}{n}
\rput(7.5220313,-1.4251407){\LARGE \textbf{(d)}}
\end{pspicture} 
}
\end{center}

\item 
Find the volume of the following objects (round off to $1$ decimal place if needed):
\begin{center}
\scalebox{0.8} % Change this value to rescale the drawing.
{
\begin{pspicture}(0,-6.555141)(11.58,6.555141)
\psline[linewidth=0.028222222](0.7205463,1.9239725)(2.8089423,6.5410295)(5.026165,1.816534)
\psline[linewidth=0.04,linestyle=dotted,dotsep=0.1cm](2.8189373,6.4493885)(2.736479,1.7205143)(4.962867,1.6926974)(4.962867,1.831782)(4.935381,1.831782)
% \usefont{T1}{ptm}{m}{n}
\rput(4.7499313,4.485141){$13$ cm}
\psbezier[linewidth=0.027999999](0.78034645,2.0499592)(0.47451097,1.3570333)(1.3971436,0.47512725)(2.5043025,0.41213384)(3.6114616,0.34914044)(5.026165,0.85308695)(5.026165,1.8609792)
\psbezier[linewidth=0.022,linestyle=dashed,dash=0.1cm 0.1cm](5.026165,1.7979861)(4.657112,2.3649256)(3.9805148,2.7450073)(3.0888517,2.7754428)(2.1971886,2.8058784)(1.4586524,2.7428854)(0.7205463,1.8609792)
\psframe[linewidth=0.04,dimen=outer](3.1003997,2.115141)(2.7003999,1.7151409)
% \usefont{T1}{ptm}{m}{n}
\rput(4.009931,2.0051408){$5$ cm}
\pspolygon[linewidth=0.028222222](7.28,1.7148592)(8.54,5.474859)(7.074111,3.09593)
\pspolygon[linewidth=0.028222222](11.245753,2.3025317)(8.571624,5.574859)(7.2880416,1.6889703)
\psline[linewidth=0.022cm,linestyle=dashed,dash=0.1cm 0.1cm](7.074111,3.09593)(11.19227,2.358951)
\psline[linewidth=0.04cm,linestyle=dotted,dotsep=0.1cm](8.56,5.574859)(9.22,1.9948592)
\psline[linewidth=0.024](9.18,2.2348592)(9.4,2.2748594)(9.44,2.0348594)(9.46,2.0348594)
\psline[linewidth=0.04cm](8.82,2.9148593)(8.76,2.6548593)
\psline[linewidth=0.04cm](6.96,2.4148595)(7.3,2.4748592)
\psline[linewidth=0.04cm](8.74,2.0548594)(8.92,1.7948594)
% \usefont{T1}{ptm}{m}{n}
\rput(9.395,1.5848593){$6$ cm}
% \usefont{T1}{ptm}{m}{n}
\rput(9.31,3.8848593){$10$ cm}
% \usefont{T1}{ptm}{m}{n}
\rput(1.0020312,5.3748593){\LARGE\textbf{(a)}}
% \usefont{T1}{ptm}{m}{n}
\rput(6.9220314,5.3748593){\LARGE\textbf{(b)}}
\pscircle[linewidth=0.027999999,dimen=outer](9.68,-3.4851408){1.9}
\psellipse[linewidth=0.027999999,linestyle=dashed,dash=0.16cm 0.16cm,dimen=outer](9.68,-3.6151407)(1.84,0.23)
\psline[linewidth=0.027999999cm,linestyle=dotted,dotsep=0.1cm](9.74,-3.6251407)(11.52,-3.6251407)
\psdots[dotsize=0.09](9.7,-3.6451406)
% \usefont{T1}{ppl}{m}{n}
\rput(10.499532,-3.1651409){$10$ cm}
% \usefont{T1}{ptm}{m}{n}
\rput{0.6029805}(-0.05992648,-0.0624806){\rput(5.8769474,-5.725771){$6$ cm}}
% \usefont{T1}{ptm}{m}{n}
\rput{-1.0300905}(0.10617459,0.030957164){\rput(1.7449429,-5.8903265){$6$ cm}}
% \usefont{T1}{ptm}{m}{n}
\rput(5.04,-2.8101408){$12$ cm}
\psline[linewidth=0.04cm](6.4,-4.635141)(3.6,-0.8151407)
\psline[linewidth=0.04cm](3.76,-6.5151405)(3.6,-0.79514074)
\psline[linewidth=0.04cm](3.6,-0.79514074)(0.88,-4.5351405)
\psline[linewidth=0.04cm](0.9,-4.5351405)(3.64,-2.7751408)
\psline[linewidth=0.04cm](3.62,-2.7751408)(6.42,-4.655141)
\psline[linewidth=0.04cm](0.86,-4.5351405)(3.76,-6.5351405)
\psline[linewidth=0.04cm](3.76,-6.5351405)(6.42,-4.655141)
\psline[linewidth=0.04cm,linestyle=dashed,dash=0.17638889cm 0.10583334cm](5.26,-5.4551406)(3.58,-0.91514075)
\psline[linewidth=0.04cm](5.16,-5.155141)(5.4,-4.9751406)
\psline[linewidth=0.04cm](5.38,-4.9751406)(5.5,-5.2951407)
% \usefont{T1}{ptm}{m}{n}
\rput(1.3220313,-1.3851408){\LARGE\textbf{(c)}}
% \usefont{T1}{ptm}{m}{n}
\rput(7.5220313,-1.4251407){\LARGE\textbf{(d)}}
\end{pspicture} 
}

\end{center}
\item
The solid below is made up of a cube and a square pyramid. Find it's volume and surface area (correct to $1$ decimal place):
\begin{center}
 \scalebox{1} % Change this value to rescale the drawing.
{
\begin{pspicture}(0,-1.8370537)(6.719142,2.2091963)
\psdiamond[linewidth=0.04,dimen=outer,gangle=130.79651](1.6439155,-0.50671875)(1.2616725,1.0826066)
\psdiamond[linewidth=0.04,dimen=outer,gangle=50.0](3.2356775,-0.49835977)(1.27,1.0687643)
\psline[linewidth=0.027999999,linestyle=dashed,dash=0.17638889cm 0.10583334cm](0.85914207,0.46575883)(2.497392,0.74575895)(3.979142,0.49575883)
\psline[linewidth=0.027999999,linestyle=dashed,dash=0.17638889cm 0.10583334cm](0.8815407,-1.1342412)(2.499142,-0.78424114)(4.019142,-1.1642412)
\psline[linewidth=0.04](0.839142,0.43575883)(2.439142,1.9957589)(2.439142,0.23575884)(2.439142,0.17575884)(2.459142,0.19575883)(2.459142,0.21575883)
\psline[linewidth=0.04cm](4.039142,0.45575884)(2.439142,2.0157588)
\psline[linewidth=0.02](4.619142,2.075759)(4.979142,2.075759)(4.999142,-1.1042411)(4.659142,-1.1042411)
% \usefont{T1}{ptm}{m}{n}
\rput(5.914142,0.5057588){$11$ cm}
\usefont{T1}{ptm}{m}{n}
\rput(1.524142,-1.6342412){$5$ cm}
% \usefont{T1}{ptm}{m}{n}
% \rput(1.1611733,2.0057588){\textbf{9.}}
\end{pspicture} 
}
\end{center}
\end{enumerate}
\practiceinfo

\begin{tabularx}{\textwidth}{ XX }
(1.) 00hr&	(2.) 00hs& (3.) 00ht\end{tabularx}
}
\end{exercises}
% slight mod to this long heading. Makes it fit in book and TOC.
\section{The effect of multiplying a dimension by a factor of $k$}
When one or more of the dimensions of a prism or cylinder is multiplied by a constant, the
surface area and volume will change. The new surface area and volume can be calculated by
using the formulae from the preceding section.\par
It is possible to see a relationship between the change in dimensions and the resulting change
in surface area and volume. These relationships make it simpler to
calculate the new volume or surface area of an object when its dimensions are
scaled up or down.\par
\mindsetvid{Increasing surface area}{VMctl}
Consider a rectangular prism of dimensions $l$, $b$ and $h$. Below we multiply one, two and
three of its dimensions by a constant factor of $5$ and calculate the new volume and surface area.\par
\begin{center}
\begin{table}[H]
 \begin{tabular}{|m{5cm}|c|c|}
\hline
Dimensions & 
Volume & 
Surface \\ \hline
Original dimensions 
\begin{center}
\scalebox{0.9} % Change this value to rescale the drawing.
{
\begin{pspicture}(0,-0.9664062)(2.8190625,0.94640625)
\psline[linewidth=0.02cm,arrowsize=0.05291667cm 2.0,arrowlength=1.4,arrowinset=0.4]{<->}(0.0,-0.33359376)(0.92,-0.5935938)
\psline[linewidth=0.02cm,arrowsize=0.05291667cm 2.0,arrowlength=1.4,arrowinset=0.4]{<->}(1.28,-0.49359375)(1.84,-0.29359376)
\psline[linewidth=0.02cm,arrowsize=0.05291667cm 2.0,arrowlength=1.4,arrowinset=0.4]{<->}(2.06,0.64640623)(2.06,-0.13359375)
\usefont{T1}{ppl}{m}{n}
\rput(0.26453125,-0.76359373){$l$}
\usefont{T1}{ppl}{m}{n}
\rput(1.6845312,-0.70359373){$b$}
\usefont{T1}{ppl}{m}{n}
\rput(2.4445312,0.27640626){$h$}
\psline[linewidth=0.02cm](0.06,0.6664063)(0.06,-0.13359375)
\psline[linewidth=0.02cm](0.06,0.6664063)(1.08,0.44640625)
\psline[linewidth=0.02cm](1.0394049,-0.40186626)(1.8688747,-0.13507086)
\psline[linewidth=0.02cm](0.0899311,0.6581822)(0.91940093,0.92497766)
\psline[linewidth=0.02cm](1.0534269,0.4367113)(1.8828968,0.7035067)
\psline[linewidth=0.02cm](0.04,-0.15359375)(1.0448525,-0.3976128)
\psline[linewidth=0.02cm](1.06,0.46640626)(1.0594049,-0.40186626)
\psline[linewidth=0.02cm](1.86,0.70640624)(1.86,-0.15359375)
\psline[linewidth=0.02cm](0.9,0.92640626)(1.8448526,0.7023872)
\end{pspicture} 

}
\end{center}
&
\begin{equation*}
  \begin{array}{r@{\;}l}
    V
    &= l \times b \times h \\
    &= lbh
  \end{array}
\end{equation*}
& 
\begin{equation*}
  \begin{array}{r@{\;}l}
  A
  &= 2[(l\times h) + (l \times b)+ (b \times h)] \\
  &= 2(lh + lb + bh)
  \end{array}
\end{equation*}
\\ \hline

Multiply one dimension by $5$ 
\begin{center}
\scalebox{1} % Change this value to rescale the drawing.
{
\begin{pspicture}(0,-1.4867684)(2.0090625,1.4667684)
\psline[linewidth=0.02cm](0.6727914,-0.97690886)(1.0821153,-0.84186465)
\psline[linewidth=0.02cm](0.19934765,-0.8512404)(0.6754797,-0.9747559)
\psline[linewidth=0.02cm](0.68295467,-0.53741413)(0.68266094,-0.97690886)
\psline[linewidth=0.02cm,arrowsize=0.05291667cm 2.0,arrowlength=1.4,arrowinset=0.4]{<->}(0.13,-0.96235126)(0.59,-1.0939559)
\psline[linewidth=0.02cm,arrowsize=0.05291667cm 2.0,arrowlength=1.4,arrowinset=0.4]{<->}(0.81,-1.0739558)(1.19,-0.93395585)
\psline[linewidth=0.02cm](0.68295467,-0.102106705)(0.68266094,-0.5416014)
\psline[linewidth=0.02cm](0.6779195,-0.5416014)(1.0821153,-0.40655723)
\psline[linewidth=0.02cm](0.19,-0.41395584)(0.6805741,-0.53944844)
\psline[linewidth=0.02cm](0.17934765,1.2744327)(0.63355875,1.4459476)
\psline[linewidth=0.02cm](0.61366063,1.4467683)(1.0908254,1.3180801)
\psline[linewidth=0.02cm](0.68295467,1.1633219)(0.68266094,0.744074)
\psline[linewidth=0.02cm](0.68295467,0.33320072)(0.68266094,-0.10629402)
\psline[linewidth=0.02cm](0.68295467,0.76850814)(0.68266094,0.3290134)
\psline[linewidth=0.02cm,arrowsize=0.05291667cm 2.0,arrowlength=1.4,arrowinset=0.4]{<->}(1.29,1.3260442)(1.29,-0.81395584)
\usefont{T1}{ppl}{m}{n}
\rput(0.21453124,-1.2839558){$l$}
\usefont{T1}{ppl}{m}{n}
\rput(1.1145313,-1.2439559){$b$}
\usefont{T1}{ppl}{m}{n}
\rput(1.6345313,0.13604417){$5h$}
\psline[linewidth=0.02cm](1.09,1.3260442)(1.09,-0.8539558)
\psline[linewidth=0.02cm](0.6927914,-0.12160144)(1.1021153,0.013442772)
\psline[linewidth=0.02cm](0.1996086,0.004067012)(0.6954797,-0.11944846)
\psline[linewidth=0.02cm](0.6927914,0.31839857)(1.1021153,0.45344278)
\psline[linewidth=0.02cm](0.1996086,0.444067)(0.6954797,0.32055154)
\psline[linewidth=0.02cm](0.6727914,0.75839853)(1.0821153,0.89344275)
\psline[linewidth=0.02cm](0.1796086,0.884067)(0.6754797,0.7605515)
\psline[linewidth=0.02cm](0.19,1.2860441)(0.19,-0.87395585)
\psline[linewidth=0.02cm](0.6927914,1.1583985)(1.1021153,1.2934427)
\psline[linewidth=0.02cm](0.1996086,1.284067)(0.6954797,1.1605515)
\end{pspicture} 
}
\end{center}
& 
\begin{equation*}
  \begin{array}{r@{\;}l}
  V_1
  &= l \times b \times 5h \\
  &= 5(lbh) \\
  &= 5V
  \end{array}
\end{equation*}
& 
\begin{equation*}
  \begin{array}{r@{\;}l}
  A_1
  &= 2[(l\times 5h) + (l \times b)+ (b \times 5h)] \\
  &= 2(5lh + lb + 5bh)
  \end{array}
\end{equation*}
\\ \hline

Multiply two dimensions by $5$ 
\begin{center}
\scalebox{1} % Change this value to rescale the drawing.
{
\begin{pspicture}(0,-1.696225)(4.1590624,1.716225)
\psline[linewidth=0.02cm](2.7097116,-1.1182724)(3.1489298,-0.98734313)
\psline[linewidth=0.02cm](2.720617,-0.6921695)(2.7203019,-1.1182724)
\psline[linewidth=0.02cm](2.720617,-0.27012625)(2.7203019,-0.6962292)
\psline[linewidth=0.02cm](2.7152143,-0.6962292)(3.1489298,-0.5652999)
\psline[linewidth=0.02cm](2.18023,1.0644687)(2.6676135,1.2307575)
\psline[linewidth=0.02cm](0.60357034,1.696225)(3.1797369,1.1067861)
\psline[linewidth=0.02cm](2.720617,0.95674354)(2.7203019,0.5502705)
\psline[linewidth=0.02cm](2.720617,0.15191695)(2.7203019,-0.274186)
\psline[linewidth=0.02cm](2.720617,0.5739602)(2.7203019,0.14785723)
\psline[linewidth=0.02cm](2.7311723,-0.28902698)(3.1703906,-0.15809768)
\psline[linewidth=0.02cm](2.7311723,0.13756584)(3.1703906,0.2684951)
\psline[linewidth=0.02cm](2.7097116,0.5641586)(3.1489298,0.6950879)
\psline[linewidth=0.02cm](2.1916602,1.0757264)(2.1916602,-1.0184565)
\psline[linewidth=0.02cm](2.7311723,0.9519703)(3.1703906,1.0828996)
\psline[linewidth=0.02cm](1.676604,1.1920699)(1.676604,-0.90211296)
\psline[linewidth=0.02cm](1.6866343,1.1808122)(2.1740181,1.347101)
\psline[linewidth=0.02cm](1.1830086,1.3084134)(1.1830086,-0.78576946)
\psline[linewidth=0.02cm](1.1715782,1.2971557)(1.658962,1.4634445)
\psline[linewidth=0.02cm](0.6464917,1.4247569)(0.6464917,-0.66942596)
\psline[linewidth=0.02cm](0.6350614,1.4134992)(1.1224451,1.579788)
\psline[linewidth=0.02cm](0.13143557,1.560491)(0.13143557,-0.5336919)
\psline[linewidth=0.02cm](0.12000524,1.5492333)(0.62503105,1.696225)
\psline[linewidth=0.02cm,arrowsize=0.05291667cm 2.0,arrowlength=1.4,arrowinset=0.4]{<->}(0.0,-0.7234125)(2.6,-1.3234125)
\usefont{T1}{ppl}{m}{n}
\rput(1.0845313,-1.3334125){$5l$}
\psline[linewidth=0.02cm,arrowsize=0.05291667cm 2.0,arrowlength=1.4,arrowinset=0.4]{<->}(2.8,-1.3034126)(3.18,-1.1634126)
\usefont{T1}{ppl}{m}{n}
\rput(3.1045313,-1.4934125){$b$}
\psline[linewidth=0.02cm,arrowsize=0.05291667cm 2.0,arrowlength=1.4,arrowinset=0.4]{<->}(3.38,1.0965874)(3.36,-0.9634125)
\usefont{T1}{ppl}{m}{n}
\rput(3.7845314,0.04658747){$5h$}
\psline[linewidth=0.02cm](3.1573904,1.1145076)(3.1573904,-0.9990659)
\psline[linewidth=0.02cm](0.13143557,1.5411004)(2.707602,0.9516615)
\psline[linewidth=0.02cm](0.13143557,1.1532887)(2.707602,0.56384987)
\psline[linewidth=0.02cm](0.13143557,0.72669595)(2.707602,0.13725708)
\psline[linewidth=0.02cm](0.13143557,0.30010313)(2.707602,-0.28933573)
\psline[linewidth=0.02cm](0.15289624,-0.10709909)(2.7290628,-0.696538)
\psline[linewidth=0.02cm](0.13143557,-0.5336919)(2.707602,-1.1231308)
\end{pspicture} 
}

\end{center}
& 
\begin{equation*}
  \begin{array}{r@{\;}l}
    V_2
    &= 5l \times b \times 5h \\
    &= 5.5(lbh) \\
    &= 5^2 \times V
  \end{array}
\end{equation*}
&
\begin{equation*}
  \begin{array}{r@{\;}l}
    A_2
    &= 2[(5l\times 5h) + (5l \times b)+ (b \times 5h)] \\
    &= 2\times 5(5lh + lb + bh)
  \end{array}
\end{equation*}
\\ \hline

Multiply all three dimensions by $5$ 
\begin{center}
\scalebox{0.8} % Change this value to rescale the drawing.
{
\begin{pspicture}(0,-1.8564062)(5.8090625,1.8764062)
\psline[linewidth=0.02cm](2.690617,-0.9723507)(2.690302,-1.3984536)
\psline[linewidth=0.02cm](2.690617,-0.55030745)(2.690302,-0.9764104)
\psline[linewidth=0.02cm](0.5735704,1.3960438)(3.149737,0.806605)
\psline[linewidth=0.02cm](2.690617,0.6765623)(2.690302,0.2700893)
\psline[linewidth=0.02cm](2.690617,-0.12826428)(2.690302,-0.5543672)
\psline[linewidth=0.02cm](2.690617,0.29377893)(2.690302,-0.132324)
\psline[linewidth=0.02cm](2.1616602,0.79554516)(2.1616602,-1.2986376)
\psline[linewidth=0.02cm](1.646604,0.91188866)(1.646604,-1.1822941)
\psline[linewidth=0.02cm](1.1530086,1.0282322)(1.1530086,-1.0659506)
\psline[linewidth=0.02cm](0.61649173,1.1445757)(0.61649173,-0.9496072)
\psline[linewidth=0.02cm](0.10143557,1.2803097)(0.10143557,-0.8138731)
\psline[linewidth=0.02cm](3.13,0.81640625)(3.1273904,-1.279247)
\psline[linewidth=0.02cm](0.10143557,1.2609191)(2.677602,0.6714803)
\psline[linewidth=0.02cm](0.10143557,0.8731075)(2.677602,0.28366867)
\psline[linewidth=0.02cm](0.10143557,0.4465147)(2.677602,-0.14292414)
\psline[linewidth=0.02cm](0.10143557,0.019921906)(2.677602,-0.56951696)
\psline[linewidth=0.02cm](0.12289625,-0.38728032)(2.6990628,-0.97671914)
\psline[linewidth=0.02cm](0.10143557,-0.8138731)(2.677602,-1.403312)
\psline[linewidth=0.02cm,arrowsize=0.05291667cm 2.0,arrowlength=1.4,arrowinset=0.4]{<->}(0.01,-1.0435938)(2.61,-1.6435938)
\usefont{T1}{ppl}{m}{n}
\rput(1.0945313,-1.6535938){\LARGE$5l$}
\psline[linewidth=0.02cm,arrowsize=0.05291667cm 2.0,arrowlength=1.4,arrowinset=0.4]{<->}(2.81,-1.6035937)(4.81,-1.0035938)
\usefont{T1}{ppl}{m}{n}
\rput(3.9545312,-1.5735937){\LARGE$ 5b$}
\psline[linewidth=0.02cm,arrowsize=0.05291667cm 2.0,arrowlength=1.4,arrowinset=0.4]{<->}(5.03,1.2764063)(5.01,-0.7835938)
\usefont{T1}{ppl}{m}{n}
\rput(5.434531,0.22640625){\LARGE$5h$}
\psline[linewidth=0.02cm](0.97357035,1.4960438)(3.5497367,0.90660495)
\psline[linewidth=0.02cm](3.53,0.9164063)(3.5473905,-1.159247)
\psline[linewidth=0.02cm](1.3935704,1.6160438)(3.9697368,1.026605)
\psline[linewidth=0.02cm](3.95,1.0364063)(3.9473903,-1.039247)
\psline[linewidth=0.02cm](2.67,-1.4035938)(4.7989297,-0.7675243)
\psline[linewidth=0.02cm](2.2335703,1.8560438)(4.8097367,1.266605)
\psline[linewidth=0.02cm](4.79,1.2764063)(4.78739,-0.7792471)
\psline[linewidth=0.02cm](1.8135704,1.7360438)(4.3897367,1.146605)
\psline[linewidth=0.02cm](4.3873906,1.1743263)(4.39,-0.88359374)
\psline[linewidth=0.02cm](2.67,0.67640626)(4.79,1.2564063)
\psline[linewidth=0.02cm](2.69,-0.9635937)(4.79,-0.36359376)
\psline[linewidth=0.02cm](2.67,-0.56359375)(4.79,0.05640625)
\psline[linewidth=0.02cm](2.69,-0.14359374)(4.77,0.45640624)
\psline[linewidth=0.02cm](2.69,0.27640626)(4.77,0.89640623)
\psline[linewidth=0.02cm](0.11,1.2764063)(2.23,1.8564062)
\psline[linewidth=0.02cm](0.61,1.1364063)(2.77,1.7364062)
\psline[linewidth=0.02cm](1.15,1.0164063)(3.31,1.6164062)
\psline[linewidth=0.02cm](1.67,0.89640623)(3.83,1.4964062)
\psline[linewidth=0.02cm](2.19,0.7764062)(4.37,1.3764062)
\end{pspicture} 
}
\end{center}
& 
\begin{equation*}
  \begin{array}{r@{\;}l}
    V_3
    &= 5l \times 5b \times 5h \\
    &= 5^3(lbh) \\
    &= 5^3V
  \end{array}
\end{equation*}
& 
\begin{equation*}
  \begin{array}{r@{\;}l}
    A_3
    &= 2[(5l\times 5h) + (5l \times 5b)+ (5b \times 5h)]\\
    &= 2(5^2 lh + 5^2 lb +  5^2 bh) \\
    &= 5^2 \times 2(lh + lb + bh) \\
    &= 5^2 A
  \end{array}
\end{equation*}
\\ \hline

Multiply all three dimensions by $k$ 
\begin{center}
\scalebox{0.8} % Change this value to rescale the drawing.
{
\begin{pspicture}(0,-1.8990625)(6.586406,1.8690625)
\psline[linewidth=0.02cm,linecolor=gray,linestyle=dashed,dash=0.1cm 0.1cm](1.1835704,1.3987)(3.7597368,0.8092612)
\psline[linewidth=0.02cm,linecolor=gray,linestyle=dashed,dash=0.1cm 0.1cm](1.7630085,1.0308884)(1.7630085,-1.0632944)
\psline[linewidth=0.02cm,linecolor=gray,linestyle=dashed,dash=0.1cm 0.1cm](1.2264917,1.1472318)(1.2264917,-0.946951)
\psline[linewidth=0.02cm,linecolor=gray,linestyle=dashed,dash=0.1cm 0.1cm](0.71143556,1.2829659)(0.71143556,-0.81121683)
\psline[linewidth=0.02cm,linecolor=gray,linestyle=dashed,dash=0.1cm 0.1cm](0.71143556,1.2635753)(3.287602,0.6741366)
\psline[linewidth=0.02cm,linecolor=gray,linestyle=dashed,dash=0.1cm 0.1cm](0.71143556,0.8757638)(3.287602,0.28632492)
\psline[linewidth=0.02cm,linecolor=gray,linestyle=dashed,dash=0.1cm 0.1cm](0.71143556,0.44917095)(3.287602,-0.1402679)
\psline[linewidth=0.02cm](2.28,-0.3409375)(3.287602,-0.56686074)
\psline[linewidth=0.02cm,linecolor=gray,linestyle=dashed,dash=0.1cm 0.1cm](0.73289627,0.015375933)(2.3,-0.3409375)
\psline[linewidth=0.02cm,linestyle=dotted,dotsep=0.16cm,arrowsize=0.05291667cm 2.0,arrowlength=1.4,arrowinset=0.4]{<->}(0.0,-0.84093744)(3.1,-1.5809375)
\usefont{T1}{ppl}{m}{n}
\rput(1.5356251,-1.5909375){\LARGE $kl$}
\psline[linewidth=0.02cm,linestyle=dotted,dotsep=0.16cm,arrowsize=0.05291667cm 2.0,arrowlength=1.4,arrowinset=0.4]{<->}(3.54,-1.5409374)(6.14,-0.74093753)
\usefont{T1}{ppl}{m}{n}
\rput(4.855625,-1.4509375){\LARGE $kb$}
\psline[linewidth=0.02cm,linestyle=dotted,dotsep=0.16cm,arrowsize=0.05291667cm 2.0,arrowlength=1.4,arrowinset=0.4]{<->}(5.62,1.6790625)(5.62,-0.6409375)
\usefont{T1}{ppl}{m}{n}
\rput(5.975625,0.6890625){\LARGE $kh$}
\psline[linewidth=0.02cm,linecolor=gray,linestyle=dashed,dash=0.1cm 0.1cm](1.5835704,1.4987)(4.1597366,0.90926117)
\psline[linewidth=0.02cm,linecolor=gray,linestyle=dashed,dash=0.1cm 0.1cm](2.0035703,1.6187001)(4.5797367,1.0292612)
\psline[linewidth=0.02cm,linecolor=gray,linestyle=dashed,dash=0.1cm 0.1cm](4.56,1.0390625)(4.55739,-1.0365908)
\psline[linewidth=0.02cm,linecolor=gray,linestyle=dashed,dash=0.1cm 0.1cm](2.8435705,1.8587002)(5.419737,1.2692612)
\psline[linewidth=0.02cm,linecolor=gray,linestyle=dashed,dash=0.1cm 0.1cm](5.4,1.2790625)(5.3973904,-0.7765908)
\psline[linewidth=0.02cm,linecolor=gray,linestyle=dashed,dash=0.1cm 0.1cm](2.4235704,1.7387002)(4.999737,1.1492612)
\psline[linewidth=0.02cm,linecolor=gray,linestyle=dashed,dash=0.1cm 0.1cm](4.9973903,1.1769825)(5.0,-0.8809375)
\psline[linewidth=0.02cm,linecolor=gray,linestyle=dashed,dash=0.1cm 0.1cm](3.28,0.6790625)(5.4,1.2590624)
\psline[linewidth=0.02cm,linecolor=gray,linestyle=dashed,dash=0.1cm 0.1cm](4.14,-0.72093743)(5.4,-0.3209375)
\psline[linewidth=0.02cm](3.3,-0.96093756)(4.14,-0.72093743)
\psline[linewidth=0.02cm,linecolor=gray,linestyle=dashed,dash=0.1cm 0.1cm](3.3,-0.1409375)(5.38,0.4590625)
\psline[linewidth=0.02cm,linecolor=gray,linestyle=dashed,dash=0.1cm 0.1cm](3.3,0.2790625)(5.38,0.89906245)
\psline[linewidth=0.02cm,linecolor=gray,linestyle=dashed,dash=0.1cm 0.1cm](0.72,1.2790625)(2.84,1.8590626)
\psline[linewidth=0.02cm,linecolor=gray,linestyle=dashed,dash=0.1cm 0.1cm](1.22,1.1390624)(3.38,1.7390625)
\psline[linewidth=0.02cm,linecolor=gray,linestyle=dashed,dash=0.1cm 0.1cm](1.76,1.0190624)(3.92,1.6190624)
\psline[linewidth=0.02cm,linecolor=gray,linestyle=dashed,dash=0.1cm 0.1cm](2.28,0.89906245)(4.44,1.4990624)
\psline[linewidth=0.02cm,linecolor=gray,linestyle=dashed,dash=0.1cm 0.1cm](2.8,0.77906257)(4.98,1.3790624)
\psline[linewidth=0.02cm,linecolor=gray,linestyle=dashed,dash=0.1cm 0.1cm](4.12,-0.3209375)(5.38,0.0790625)
\psline[linewidth=0.02cm](3.28,-0.5609375)(4.12,-0.3209375)
\psline[linewidth=0.02cm,linecolor=gray,linestyle=dashed,dash=0.1cm 0.1cm](4.12,-1.1609374)(5.38,-0.7609375)
\psline[linewidth=0.02cm](3.28,-1.4009374)(4.12,-1.1609374)
\psline[linewidth=0.02cm,linecolor=gray,linestyle=dashed,dash=0.1cm 0.1cm](3.28,0.65906256)(3.28,-0.5809375)
\psline[linewidth=0.02cm](3.28,-0.5809375)(3.28,-1.4009374)
\psline[linewidth=0.02cm,linecolor=gray,linestyle=dashed,dash=0.1cm 0.1cm](3.74,0.7990625)(3.74,-0.4409375)
\psline[linewidth=0.02cm](3.74,-0.4409375)(3.74,-1.2609375)
\psline[linewidth=0.02cm,linecolor=gray,linestyle=dashed,dash=0.1cm 0.1cm](2.78,0.75906247)(2.78,-0.48093754)
\psline[linewidth=0.02cm](2.78,-0.46093747)(2.78,-1.3009374)
\psline[linewidth=0.02cm,linecolor=gray,linestyle=dashed,dash=0.1cm 0.1cm](2.3,0.89906245)(2.3,-0.3409375)
\psline[linewidth=0.02cm](2.3,-0.36093748)(2.3,-1.1809374)
\psline[linewidth=0.02cm,linecolor=gray,linestyle=dashed,dash=0.1cm 0.1cm](0.69289625,-0.36462408)(2.26,-0.72093743)
\psline[linewidth=0.02cm,linecolor=gray,linestyle=dashed,dash=0.1cm 0.1cm](0.69289625,-0.804624)(2.26,-1.1609374)
\psline[linewidth=0.02cm](2.3,-0.74093753)(3.307602,-0.9668608)
\psline[linewidth=0.02cm](2.28,-1.1809374)(3.287602,-1.4068607)
\psline[linewidth=0.02cm,linecolor=gray,linestyle=dashed,dash=0.1cm 0.1cm](4.16,0.9190625)(4.16,-0.3209375)
\psline[linewidth=0.02cm](4.16,-0.3209375)(4.16,-1.1409374)
\psline[linewidth=0.02cm](2.74,-0.20093751)(3.747602,-0.42686072)
\psline[linewidth=0.02cm](3.16,-0.10093751)(4.167602,-0.32686073)
\psline[linewidth=0.02cm](2.32,-0.3409375)(3.16,-0.1009375)
\psline[linewidth=0.02cm](2.78,-0.44093752)(3.62,-0.2009375)
\end{pspicture} 
}
\end{center}
& 
\begin{equation*}
  \begin{array}{r@{\;}l}
    V_k
    &= kl \times kb \times kh \\
    &= k^3(lbh) \\
    &= k^3V
  \end{array}
\end{equation*}
& 
\begin{equation*}
  \begin{array}{r@{\;}l}
    A_k
    &= 2[(kl\times kh) + (kl \times kb)+ (kb \times kh)] \\
    &= k^2 \times 2(lh+lb+bh) \\
    &= k^2 A
  \end{array}
\end{equation*}
\\ \hline
\end{tabular}

\end{table}
\end{center}


\begin{wex}{Calculating the new dimensions of a rectangular prism}
{Consider a square prism with a height of $4$ cm and base lengths of $3$ cm.
\begin{center}
\scalebox{1} % Change this value to rescale the drawing.
{
\begin{pspicture}(0,-1.2891959)(5.494142,1.2016168)
\psdiamond[linewidth=0.04,dimen=outer,gangle=130.79651](1.6473784,-0.0028731404)(1.2669724,1.0826066)
\psdiamond[linewidth=0.04,dimen=outer,gangle=50.0](3.2356775,0.009498077)(1.27,1.0687643)
\psline[linewidth=0.027999999](0.8591421,0.9536167)(2.5,1.1876167)(4.019142,0.9808309)
\psline[linewidth=0.027999999,linestyle=dashed,dash=0.16cm 0.16cm](0.8815408,-0.6263833)(2.56,-0.31238326)(4.019142,-0.65638334)
\usefont{T1}{ptm}{m}{n}
\rput(4.799142,0.21361668){$4$ cm}
\usefont{T1}{ptm}{m}{n}
\rput(1.399142,-1.0863833){$3$ cm}
\usefont{T1}{ptm}{m}{n}
\rput(3.599142,-1.0263833){$3$ cm}
\psline[linewidth=0.027999999cm,linestyle=dashed,dash=0.16cm 0.16cm](2.52,1.1676167)(2.54,-0.33238328)
\end{pspicture} 
}
\end{center}

\begin{enumerate}[noitemsep, label=\textbf{\arabic*}. ] 
\item Calculate the surface area and volume.
\item Calculate the new surface area ($A_n$) and volume ($V_n$) if the
  base lengths are multiplied by a constant factor of $3$.
\item Express the new surface area and volume as a factor of the
  original surface area and volume.
\end{enumerate}
}
{
\westep{Calculate the original volume and surface area}
\begin{align*}
  V
  &= l \times b \times h \\
  &= 3 \times 3 \times 4 \\
  &= 36\mbox{ cm}^3 \\
  \\
  A
  &= 2[(l \times h) + (l \times b) + (b \times h)] \\
  &= 2[(3 \times 4) + (3 \times 3) + (3 \times 4)] \\
  &= 66\mbox{ cm}^2
\end{align*}

\westep{Calculate the new volume and surface area}
Two of the dimensions are multiplied by a factor of $3$.
\begin{align*}
  V_n
  &= 3l \times 3b \times h \\
  &= 3.3 \times 3.3 \times 4 \\
  &= 324\mbox{ cm}^3 \\
  \\
  A_n
  &= 2[(3l \times h) + (3l \times 3b) + (3b \times h)] \\
  &= 2[(3.3 \times 4) + (3.3 \times 3.3) + (3.3 \times 4)] \\
  &= 306\mbox{ cm}^2
\end{align*}

\westep{Express the new dimensions as a factor of the original dimensions}
\begin{align*}
  V &= 36 \\
  V_n &= 324 \\
  \frac{V_n}{V} &= \frac{324}{36} \\
  &= 9 \\
  \therefore V_n &= 9V \\
  &= 3^2 V \\
  \\
  A &= 66 \\
  A_n &= 306 \\
  \frac{A_n}{A} &= \frac{306}{66} \\
  \therefore A_n &= \frac{306}{66}A \\
  &= \frac{51}{11}A\\
\end{align*}
}
\end{wex}


\begin{wex}{Multiplying the dimensions of a rectangular prism by $k$}
{Prove that if the height of a rectangular prism with dimensions $l$,
  $b$ and $h$ is multiplied by a constant value of $k$, the volume
  will also increase by a factor $k$.
\begin{center}
\scalebox{1} % Change this value to rescale the drawing.
{
\begin{pspicture}(0,-1.1364063)(4.0990624,1.1164062)
\psline[linewidth=0.04cm](0.02,0.8564063)(0.02,-0.02359375)
\psline[linewidth=0.04cm](0.0,0.83640623)(2.32,0.27640626)
\psline[linewidth=0.04cm](2.2794049,-0.5718663)(3.1088746,-0.30507085)
\psline[linewidth=0.027999999cm,linestyle=dashed,dash=0.16cm 0.16cm](0.01770038,0.004156353)(0.84,0.31640625)
\psline[linewidth=0.04cm](0.009931098,0.8281822)(0.83940095,1.0949776)
\psline[linewidth=0.04cm](2.293427,0.2667113)(3.1228967,0.5335067)
\psline[linewidth=0.027999999cm,linestyle=dashed,dash=0.16cm 0.16cm](0.84,0.29640624)(3.0943224,-0.3008174)
\psline[linewidth=0.04cm](0.0,-0.00359375)(2.2848525,-0.5676128)
\psline[linewidth=0.04cm](0.8,1.0964062)(3.126553,0.5432085)
\psline[linewidth=0.04cm](2.3,0.29640624)(2.2994049,-0.5718663)
\psline[linewidth=0.027999999cm,linestyle=dashed,dash=0.16cm 0.16cm](0.84,1.0764062)(0.84,0.31640625)
\psline[linewidth=0.04cm](3.1,0.5364063)(3.1,-0.32359374)
\psline[linewidth=0.02cm,arrowsize=0.05291667cm 2.0,arrowlength=1.4,arrowinset=0.4]{<->}(0.04,-0.26359376)(1.94,-0.76359373)
\psline[linewidth=0.02cm,arrowsize=0.05291667cm 2.0,arrowlength=1.4,arrowinset=0.4]{<->}(2.52,-0.7235938)(3.08,-0.5235937)
\psline[linewidth=0.02cm,arrowsize=0.05291667cm 2.0,arrowlength=1.4,arrowinset=0.4]{<->}(3.32,0.47640625)(3.32,-0.26359376)
\usefont{T1}{ppl}{m}{n}
\rput(0.8045313,-0.73359376){$l$}
\usefont{T1}{ppl}{m}{n}
\rput(2.9245312,-0.93359375){$b$}
\usefont{T1}{ppl}{m}{n}
\rput(3.7245312,0.14640625){$h$}
\end{pspicture} 
}
\end{center}
}
{
\westep{Calculate the original volume}
We are given the original dimensions $l$, $b$ and $h$ and so the
original volume is $V = l \times b \times h$.

\westep{Calculate the new volume}
The new dimensions are $l$, $b$, and $kh$ and so the new volume is
\begin{align*}
  V_n
  &= l \times b \times (kh) \\
  &= k(lbh) \\
  &= kV
\end{align*}

\westep{Write the final answer}
If the height of a rectangular prism is multiplied by a constant $k$,
then the volume also increases by a factor of $k$.
}
\end{wex}

\begin{wex}{Multiplying the dimensions of a cylinder by $k$}
{Consider a cylinder with a radius of $r$ and a height of
  $h$. Calculate the new volume and surface area (expressed in terms
  of $r$ and $h$) if the radius is multiplied by a constant factor of
  $k$.
\begin{center}
\begin{pspicture}(0,-1.6)(3.5434375,1.6) 
\psellipse[linewidth=0.04,dimen=outer](1.29,-1.09)(1.27,0.51) 
\psellipse[linewidth=0.04,dimen=outer](1.27,1.09)(1.27,0.51) 
\psline[linewidth=0.04cm](0.04,-1.1)(0.02,1.14) 
\psline[linewidth=0.04cm](2.54,1.12)(2.56,-1.1) 
\psline[linewidth=0.04cm,linestyle=dashed,dash=0.16cm 0.16cm](1.3, -1.1)(2.52,-1.1) 
\rput(1.7,-0.95){$r$} 
\rput(2.9,0.25){$h$} 
\end{pspicture} 
\end{center}
}
{
\westep{Calculate the original volume and surface area}
\begin{align*}
  V &= \pi r^2 \times h\\
  A &= \pi r^2 + 2\pi rh
\end{align*}

\westep{Calculate the new volume and surface area}
The new dimensions are $kr$ and $h$.
\begin{align*}
  V_n
  &= \pi (kr)^{2} \times h \\
  &= \pi k^{2}r^{2} \times h \\
  &= k^{2} \times \pi r^{2} h \\
  &= k^{2}V \\
  \\
  A_n
  &= \pi (kr)^{2} + 2\pi (kr)h \\
  &= \pi k^{2}r^{2} +2\pi krh \\
  &= k^2(\pi r^2) + k(2\pi rh) 
\end{align*}
}
\end{wex}

\begin{exercises}{}{
\begin{enumerate}[noitemsep, label=\textbf{\arabic*}. ] 
 \item If the height of a prism is doubled, how much will its volume increase?
\item Describe the change in the volume of a rectangular prism if:
\begin{enumerate}[noitemsep, label=\textbf{(\alph*)} ] 
\item length and breadth increase by a constant factor of $3$;
\item length, breadth and height are multiplied by a constant factor of $2$.
\end{enumerate}
\item Given a prism with a volume of $493$ cm$^{3}$ and a surface area of $6007$ cm$^{2}$, 
find the new surface area and volume for a prism if all dimensions are increased by a constant factor of $4$. 
\end{enumerate}
\practiceinfo

\begin{tabularx}{\textwidth}{ XX }
(1.) 00hu&	(2.) 00hv& (3.) 00hw\end{tabularx}
}
\end{exercises}

\summary{VMdml}
\begin{itemize}
\item Area is the two dimensional space inside the boundary of a flat
  object. It is measured in square units.
\item Area formulae:
  \begin{itemize}
  \item square: $s^2$
  \item rectangle: $b \times h$
  \item triangle: $\frac{1}{2} b \times h$
  \item trapezium: $\frac{1}{2} (a+b) \times h$
  \item parallelogram: $b \times h$
  \item circle: $\pi r^2$
  \end{itemize}
\item Surface area is the total area of the exposed or outer surfaces of a prism.
\item A net is the unfolded ``plan'' of a solid.
\item Volume is the three dimensional space occupied by an object, or the contents of an
  object. It is measured in cubic units.
  \begin{itemize}
  \item Volume of a rectangular prism: $l \times b \times h$
  \item Volume of a triangular prism: $(\frac{1}{2} b \times h) \times H$
  \item Volume of a square prism or cube: $s^3$
  \item Volume of a cylinder: $\pi r^2 \times h$
  \end{itemize}
\item A pyramid is a geometric solid that has a polygon as its base
  and sides that converge at a point called the apex. The sides are
  not perpendicular to the base.
\item Surface area formulae:
  \begin{itemize}
  \item square pyramid: $b(b+2h)$
  \item triangular pyramid: $\frac{1}{2}b(h_b +3h_s)$
  \item right cone: $\pi r(r+h_s)$
  \item sphere: $4\pi r^2$
  \end{itemize}
\item Volume formulae:
  \begin{itemize}[noitemsep]
  \item square pyramid: $\frac{1}{3} \times b^2 \times H$
  \item triangular pyramid: $\frac{1}{3} \times \frac{1}{2}bh \times H$
  \item right cone: $\frac{1}{3} \times \pi r^2 \times H$
  \item sphere: $\frac{4}{3} \pi r^3$
  \end{itemize}
\item Multiplying one or more dimensions of a prism or cylinder by a
  constant $k$ affects the surface area and volume.
\end{itemize}

\begin{eocexercises}{}
\begin{enumerate}[itemsep=6pt, label=\textbf{\arabic*}. ] 
\item Consider the solids below and answer the questions that follow (correct to $1$ decimal place, if necessary):
  \begin{center}
    \scalebox{0.8}{% Change this value to rescale the drawing.
      \begin{pspicture}(0,-3.9464064)(10.693593,3.9264061)
        \psline[linewidth=0.04cm](0.75328124,-3.513594)(1.7532812,-2.533594)
        \psline[linewidth=0.04cm](3.753281,-3.493594)(4.733281,-2.533594)
        \psline[linewidth=0.04cm](0.7732813,-3.513594)(3.753281,-3.493594)
        \psline[linewidth=0.04cm](1.7532812,-2.513594)(4.733281,-2.513594)
        \psline[linewidth=0.04cm](4.753281,-1.4735942)(4.733281,-2.573594)
        \psline[linewidth=0.04cm](1.7532812,-1.493594)(1.7732812,-2.513594)
        \psline[linewidth=0.04cm](1.7732812,-1.513594)(4.753281,-1.493594)
        \psline[linewidth=0.04cm](3.753281,-2.433594)(3.753281,-3.493594)
        \psline[linewidth=0.04cm](0.75328124,-2.473594)(0.75328124,-3.533594)
        \psline[linewidth=0.04cm](0.75328124,-2.4535937)(3.7332811,-2.4535937)
        \psline[linewidth=0.04cm](0.7732813,-2.4535937)(1.7532812,-1.513594)
        \psline[linewidth=0.04cm](3.753281,-2.4535937)(4.733281,-1.513594)
        % \usefont{T1}{ptm}{m}{n}
        % \rput(1.3409375,-0.818594){\textbf{1.}}%number for box
        % \usefont{T1}{ptm}{m}{n}
        \rput(2.1515625,-3.7435937){$5$ cm}
        % \usefont{T1}{ptm}{m}{n}
        \rput(4.7346873,-3.1435938){$4$ cm}
        % \usefont{T1}{ptm}{m}{n}
        \rput(5.2,-2.043594){$2$ cm}
        \psellipse[linewidth=0.04,dimen=outer](2.203281,1.1064061)(0.99,0.38)
        \psellipse[linewidth=0.04,dimen=outer](2.203281,2.446406)(0.99,0.38)
        \psline[linewidth=0.04cm](1.233281,2.4064062)(1.233281,1.1464062)
        \psline[linewidth=0.04cm](3.173282,2.446406)(3.173282,1.1264061)
        \psline[linewidth=0.04cm,linestyle=dashed,dash=0.16cm 0.16cm](2.233281,1.086406)(3.153281,1.1064061)
        % \usefont{T1}{ptm}{m}{n}
        \rput(2.197657,1.2364061){$4$ cm}
        % \usefont{T1}{ptm}{m}{n}
        \rput(3.646093,1.8564061){$10$ cm}
        % \usefont{T1}{ptm}{m}{n}
        % \rput(0.96203125,3.2814062){\textbf{3.}}%number for cylinder
        \pstriangle[linewidth=0.04,dimen=outer](7.083281,0.76640624)(2.18,1.72)
        \psline[linewidth=0.04cm](7.0732813,2.466406)(9.433281,3.886406)
        \psline[linewidth=0.04cm](6.0532813,0.80640626)(9.233281,2.8864062)
        \psline[linewidth=0.04cm](9.433281,3.9064062)(10.653281,2.566406)
        \psline[linewidth=0.04cm](8.133282,0.7864063)(10.653281,2.5664062)
        \psline[linewidth=0.04cm,linestyle=dashed,dash=0.16cm 0.16cm](7.0732813,2.4264064)(7.0532813,0.7864063)
        % \usefont{T1}{ptm}{m}{n}
        % \rput(6.3378124,3.4014063){\textbf{2.}}%number for prism
        % \usefont{T1}{ptm}{m}{n}
        \rput(9.888594,1.4764062){$20$ cm}
        % \usefont{T1}{ptm}{m}{n}
        \rput(7.591875,1.2892188){$3$ cm}
        % \usefont{T1}{ptm}{m}{n}
        \rput(7.0415626,0.51640624){$8$ cm}
        \psline[linewidth=0.04cm](9.433281,3.9064062)(9.233281,2.8664062)
        \psline[linewidth=0.04cm](10.653281,2.5664062)(9.193281,2.8664062)
        % \usefont{T1}{ptm}{m}{n}
        \rput{58.291424}(4.633092,-4.489355){\rput(6.311875,1.9092188){$5$ cm}}
      \end{pspicture} 
    }
  \end{center}
  \begin{enumerate}[noitemsep, label=\textbf{(\alph*)} ]
  \item Calculate the surface area of each solid.
  \item Calculate volume of each solid.
  \item If each dimension of the solids is increased by a factor of $3$, calculate the new surface area of each solid.
  \item If each dimension of the solids is increased by a factor of $3$, calculate the new volume of each solid.
  \end{enumerate}
\item Consider the solids below:
  \begin{center}
    \scalebox{0.7}{% Change this value to rescale the drawing.
      \begin{pspicture}(0,-5.4207597)(13.96,5.741175)
        \definecolor{color3715b}{rgb}{0.996078431372549,0.996078431372549,0.996078431372549}
        \psline[linewidth=0.028222222](1.5205463,1.1100069)(3.6089423,5.727064)(5.826165,1.0025685)
        \psline[linewidth=0.04,linestyle=dotted,dotsep=0.16cm](3.6189375,5.6354227)(3.536479,0.9065488)(5.762867,0.8787319)(5.762867,1.0178165)(5.735381,1.0178165)
        % \usefont{T1}{ppl}{m}{n}
        \rput(4.409931,3.5911753){\LARGE $10$ cm}
        \psbezier[linewidth=0.027999999](1.5803465,1.2359936)(1.274511,0.5430677)(2.1971436,-0.3388383)(3.3043027,-0.40183172)(4.4114614,-0.46482512)(5.826165,0.039121386)(5.826165,1.0470136)
        \psbezier[linewidth=0.022,linestyle=dashed,dash=0.16cm 0.16cm](5.826165,0.9840206)(5.457112,1.55096)(4.7805147,1.9310416)(3.888852,1.9614773)(2.9971886,1.991913)(2.2586524,1.9289197)(1.5205463,1.0470136)
        \psframe[linewidth=0.04,dimen=outer](3.9003997,1.3011752)(3.5004,0.9011754)
        % \usefont{T1}{ppl}{m}{n}
        \rput(4.809931,1.1911753){\LARGE $3$ cm}
        % \usefont{T1}{ppl}{m}{n}
        % \rput(1.5420313,5.5008936){\textbf{1.}}
        % \usefont{T1}{ppl}{m}{n}
        % \rput(8.442031,5.5008936){\textbf{2.}}
        % \usefont{T1}{ppl}{m}{n}
        % \rput(0.96203125,-2.3191063){\textbf{3.}}
        % \usefont{T1}{ptm}{m}{n}
        \rput{0.6029805}(0.004622919,-0.1373882){\rput(13.026947,0.3702635){\LARGE $15$ cm}}
        % \usefont{T1}{ptm}{m}{n}
        \rput{-1.0300905}(-0.006498412,0.16914946){\rput(9.374943,0.4457077){\LARGE $15$ cm}}
        % \usefont{T1}{ptm}{m}{n}
        \rput{90.771416}(14.303988,-7.492464){\rput(10.82,3.3058937){\LARGE  $12$ cm}}
        \psline[linewidth=0.04cm](13.92,1.7008936)(11.12,5.5208936)
        \psline[linewidth=0.04cm](11.28,-0.1791063)(11.12,5.5408936)
        \psline[linewidth=0.04cm](11.12,5.5408936)(8.4,1.8008937)
        \psline[linewidth=0.04cm](8.42,1.8008937)(11.16,3.5608938)
        \psline[linewidth=0.04cm](11.14,3.5608938)(13.94,1.6808937)
        \psline[linewidth=0.04cm](8.38,1.8008937)(11.28,-0.1991063)
        \psline[linewidth=0.04cm](11.28,-0.1991063)(13.94,1.6808937)
        \psdots[dotsize=0.12](11.0,1.8408937)
        \psline[linewidth=0.04cm,linestyle=dashed,dash=0.17638889cm 0.10583334cm](11.02,1.8608937)(11.1,5.4208937)
        \psline[linewidth=0.04cm,linestyle=dashed,dash=0.17638889cm 0.10583334cm](11.0,1.8208936)(9.82,2.7008936)
        \psline[linewidth=0.04cm](10.78,1.9808937)(10.78,2.3408937)
        \psline[linewidth=0.04cm](10.76,2.3408937)(11.04,2.1408937)
        \rput{14.5046}(-0.70283186,-0.89219034){\pswedge[linewidth=0.04](3.1540546,-3.2075646){2.0675611}{178.8065}{0.0}}
        \rput{13.78588}(-0.6714317,-0.8455882){\psellipse[linewidth=0.04,dimen=outer,fillstyle=solid,fillcolor=color3715b](3.1616797,-3.1998694)(2.0922363,0.47717163)}
        \psdots[dotsize=0.12](3.1473527,-3.2047362)
        \psline[linewidth=0.04cm,linestyle=dashed,dash=0.16cm 0.16cm](3.1332812,-3.2387989)(5.147353,-2.7047362)
        % \usefont{T1}{ptm}{m}{n}
        \rput{14.285164}(-0.57707244,-1.0637248){\rput(3.9257896,-2.8347287){\LARGE $4$ cm}}
      \end{pspicture} 
    }
  \end{center}
  \begin{enumerate}[noitemsep, label=\textbf{(\alph*)} ]
  % \setcounter{enumi}{4}
  \item Calculate the surface area of each solid.
  \item Calculate the volume of each solid.
  \end{enumerate}
% \setcounter{enumi}{6}
\item Calculate the volume and surface area of the solid below (correct to $1$ decimal place):
  \begin{center}
    \scalebox{1}{% Change this value to rescale the drawing.
      \begin{pspicture}(0,-1.4992187)(4.081249,1.5192188)
        \psellipse[linewidth=0.02,dimen=outer](2.2743747,-1.1192187)(0.99,0.38)
        \psellipse[linewidth=0.02,dimen=outer](2.2743747,0.22078115)(0.99,0.38)
        \psline[linewidth=0.02cm](1.3043747,0.18078145)(1.3043747,-1.0792186)
        \psline[linewidth=0.02cm](3.2443757,0.22078115)(3.2443757,-1.0992187)
        \psline[linewidth=0.02cm,linestyle=dashed,dash=0.16cm 0.16cm](2.3043747,-1.1392188)(3.2243748,-1.1192187)
        % \usefont{T1}{ppl}{m}{n}
        \rput(2.518282,-0.96921873){$40$ cm}
        % \usefont{T1}{ppl}{m}{n}
        \rput(3.8,-0.44921875){$50$ cm}
        \psline[linewidth=0.02](1.31,0.27921876)(2.33,1.4992187)(3.23,0.29921874)(3.23,0.27921876)(3.23,0.27921876)
        \psline[linewidth=0.027999999](0.93,1.4592187)(0.67,1.4592187)(0.67,0.29921874)(0.93,0.29921874)
        % \usefont{T1}{ppl}{m}{n}
        \rput(0.1,0.88921875){$30$ cm}
      \end{pspicture} 
    }
  \end{center}
\end{enumerate}
\practiceinfo

\begin{tabularx}{\textwidth}{ XX }
(1.) 00hx&	(2.) 00hy& (3.) 00hz\end{tabularx}
\end{eocexercises}
