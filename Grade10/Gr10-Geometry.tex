\chapter{Euclidian Geometry}
\setcounter{figure}{1}
\setcounter{subfigure}{1}
\section{Geometry Revision}
\setcounter{figure}{1}
\setcounter{subfigure}{1}

Geometry (from the Greek ``geo`` = earth and ``metria`` = measure) arose as the field of knowledge
dealing with spatial relationships. Geometry can be split into Euclidean geometry and analytical geometry. 
Analtyical geometry deals with space and shape in an algebraic way using a coordinate system. 
Euclidean geometry organizes knowledge about space and shape using a system of logical deductions.\par 

\subsection*{Angles}
An angle is formed when two straight lines meet at a point known as a vertex. 
Angles are labelled with a $\hat{}$ called a caret on a letter, for example $\hat{B}$.
Angles can also be labelled according to the line segments that make up the
angle. An angle can be referred to as $C\hat{B}A$ or $A\hat{B}C$. 
The $\angle $ symbol is a short method of writing angle in
geometry and is used often used for statements such as ''sum of $\angle$s in a $\triangle$``.\par 
Angles are measured in degrees which is denoted by $^{\circ }$, a small circle
raised above the text similar to an exponent.\par 

\Note{
Angles can also be measured in radians. In high school you will only use
degrees, so make sure that your calculator is set to ''deg`` mode.}

\par 
\setcounter{subfigure}{0}
\begin{figure}[H] % horizontal\label{m39370*uid9}
\begin{center}
\rule[.1in]{\figurerulewidth}{.005in} \\
\label{m39370*uid9!!!underscore!!!media}
\label{m39370*uid9!!!underscore!!!printimage}\includegraphics{
col11306.imgs/m39370_MG10C13_002.png} % m39370;MG10C13\_002.png;;;6.0;8.5;
\vspace{2pt}
\vspace{\rubberspace}\par 
% \begin{cnxcaption}
% \small \textbf{Figure 12.2: }Angle labelled as $\hat{B}$, $\angle CBA$ or
% $\angle ABC$
% \end{cnxcaption}
\vspace{.1in}
\rule[.1in]{\figurerulewidth}{.005in} \\
\end{center}
\end{figure} 
%       
% \setcounter{subfigure}{0}
% \begin{figure}[H] % horizontal\label{m39370*uid10}
% \begin{center}
% \rule[.1in]{\figurerulewidth}{.005in} \\
% \label{m39370*uid10!!!underscore!!!media}\label{
% m39370*uid10!!!underscore!!!printimage}\includegraphics{
% col11306.imgs/m39370_MG10C13_003.png} % m39370;MG10C13\_003.png;;;6.0;8.5;
% \vspace{2pt}
% \vspace{\rubberspace}\par \begin{cnxcaption}
% \small \textbf{Figure 12.3: }Examples of angles. $\hat{A}=\hat{E}$, even though
% the lines making up the angles are of different lengths.
% \end{cnxcaption}
% \vspace{.1in}
% \rule[.1in]{\figurerulewidth}{.005in} \\
% \end{center}
% \end{figure}       
% \subsubsection{ Measuring angles}
% \nopagebreak
% The size of an angle does not depend on the length of the lines that are joined
% to make up the angle, but depends only on how both the lines are placed as can
% be seen in Figure~12.3. This means that the idea of length cannot be used to
% measure angles. An angle is a rotation around the vertex.\par 
% 
% \subsubsection{ Using a Protractor}
% \nopagebreak
% A protractor is a simple tool that is used to measure angles. A picture of a
% protractor is shown in Figure~12.4.\par 
% \setcounter{subfigure}{0}
% \begin{figure}[H] % horizontal\label{m39370*uid13}
% \begin{center}
% \rule[.1in]{\figurerulewidth}{.005in} \\
% \label{m39370*uid13!!!underscore!!!media}\label{
% m39370*uid13!!!underscore!!!printimage}\includegraphics{
% col11306.imgs/m39370_MG10C13_004.png} % m39370;MG10C13\_004.png;;;6.0;8.5;
% \vspace{2pt}
% \vspace{\rubberspace}\par \begin{cnxcaption}
% \small \textbf{Figure 12.4: }Diagram of a protractor.
% \end{cnxcaption}
% \vspace{.1in}
% \rule[.1in]{\figurerulewidth}{.005in} \\
% \end{center}
% \end{figure}       
% 
% \textbf{Method:}
% \par 
% Using a protractor\par 
% \begin{enumerate}[noitemsep, label=\textbf{\arabic*}. ] 
% \item Place the bottom line of the protractor along one line of the angle so
% that the other line of the angle points at the degree markings.
% \item Move the protractor along the line so that the centre point on the
% protractor is at the vertex of the two lines that make up the angle.
% \item Follow the second line until it meets the marking on the protractor and
% read off the angle. Make sure you start measuring at 0$^{\circ }$.
% \end{enumerate}
% 
% \subsubsection{  Measuring Angles : Use a protractor to measure the following
% angles:}
% 
% 
% \setcounter{subfigure}{0}
% \begin{figure}[H] % horizontal\label{m39370*id314484}
% \begin{center}
% \label{m39370*id314484!!!underscore!!!media}\label{
% m39370*id314484!!!underscore!!!printimage}\includegraphics{
% col11306.imgs/m39370_MG10C13_005.png} % m39370;MG10C13\_005.png;;;6.0;8.5;
% \vspace{2pt}
% \vspace{.1in}
% \end{center}
% \end{figure}       
% \par 
% 
% \subsubsection{ Special Angles}
% \nopagebreak
% What is the smallest angle that can be drawn? The figure below shows two lines
% ($CA$ and $AB$) making an angle at a common vertex $A$. If line $CA$ is rotated
% around the common vertex $A$, down towards line $AB$, then the smallest angle
% that can be drawn occurs when the two lines are pointing in the same direction.
% This gives an angle of 0$^{\circ }$. This is shown in Figure~12.6\par 
% 
% \setcounter{subfigure}{0}
% \begin{figure}[H] % horizontal\label{m39370*id314593}
% \begin{center}
% \label{m39370*id314593!!!underscore!!!media}\label{
% m39370*id314593!!!underscore!!!printimage}\includegraphics[width=.8\columnwidth]
% {col11306.imgs/m39370_MG10C13_006.png} % m39370;MG10C13\_006.png;;;6.0;8.5;
% \vspace{2pt}
% \vspace{.1in}
% \end{center}
% \end{figure}       
% \par 
% If line $CA$ is now swung upwards, any other angle can be obtained. If line $CA$
% and line $AB$ point in opposite directions (the third case in Figure~12.6) then
% this forms an angle of 180$^{\circ }$.\par 
% 
% \Tip{If three points $A$, $B$ and $C$ lie on a straight line, then the angle
% between them is 180$^{\circ }$. Conversely, if the angle between three points is
% 180$^{\circ }$, then the points lie on a straight line.}
% 
% \par
% An angle of 90$^{\circ }$ is called a right angle. A right angle is half the
% size of the angle made by a straight line (180$^{\circ }$). We say $CA$ is
% perpendicular to $AB$ or $CA\perp AB$. An angle twice the size of a straight
% line is 360$^{\circ }$. An angle measuring 360$^{\circ }$ looks identical to an
% angle of 0$^{\circ }$, except for the labelling. We call this a revolution.\par 
% \setcounter{subfigure}{0}
% \begin{figure}[H] % horizontal\label{m39370*uid18}
% \begin{center}
% \rule[.1in]{\figurerulewidth}{.005in} \\
% \label{m39370*uid18!!!underscore!!!media}\label{
% m39370*uid18!!!underscore!!!printimage}\includegraphics[width=.8\columnwidth]{
% col11306.imgs/m39370_MG10C13_007.png} % m39370;MG10C13\_007.png;;;6.0;8.5;
% \vspace{2pt}
% \vspace{\rubberspace}\par \begin{cnxcaption}
% \small \textbf{Figure 12.7: }An angle of 90$^{\circ }$ is known as a right
% angle.
% \end{cnxcaption}
% \vspace{.1in}
% \rule[.1in]{\figurerulewidth}{.005in} \\
% \end{center}
% \end{figure}       
% 
% \subsubsection{  Angles larger than 360$^{\circ }$ }
% \nopagebreak
% All angles larger than 360$^{\circ }$ also look like we have seen them before.
% If you are given an angle that is larger than 360$^{\circ }$, continue
% subtracting 360$^{\circ }$ from the angle, until you get an answer that is
% between 0$^{\circ }$and 360$^{\circ }$. Angles that measure more than
% 360$^{\circ }$ are largely for mathematical convenience. \par 
% 
% \Tip{
% \begin{itemize}[noitemsep]
% \item Acute angle: An angle $\ge {0}^{\circ }$ and $<{90}^{\circ }$.
% \item Right angle: An angle measuring ${90}^{\circ }$.
% \item Obtuse angle: An angle $>{90}^{\circ }$ and $<{180}^{\circ }$.
% \item Straight angle: An angle measuring 180$^{\circ }$.
% \item Reflex angle: An angle $>{180}^{\circ }$ and $<{360}^{\circ }$.
% \item Revolution: An angle measuring ${360}^{\circ }$.
% \end{itemize}
% These are simply labels for angles in particular ranges, shown in Figure~12.8.}
% \par
% \setcounter{subfigure}{0}
% \begin{figure}[H] % horizontal\label{m39370*uid25}
% \begin{center}
% \rule[.1in]{\figurerulewidth}{.005in} \\
% \label{m39370*uid25!!!underscore!!!media}\label{
% m39370*uid25!!!underscore!!!printimage}\includegraphics[width=.8\columnwidth]{
% col11306.imgs/m39370_MG10C13_008.png} % m39370;MG10C13\_008.png;;;6.0;8.5;
% \vspace{2pt}
% \vspace{\rubberspace}\par \begin{cnxcaption}
% \small \textbf{Figure 12.8: }Three types of angles defined according to their
% ranges.
% \end{cnxcaption}
% \vspace{.1in}
% \rule[.1in]{\figurerulewidth}{.005in} \\
% \end{center}
% \end{figure}       
% Once angles can be measured, they can then be compared. For example, all right
% angles are 90$^{\circ }$, therefore all right angles are equal and an obtuse
% angle will always be larger than an acute angle.\par The following video
% summarizes what you have learnt so far about angles.
% \setcounter{subfigure}{0}
% \begin{figure}[H] % horizontal\label{m39370*angles-1}
% \textnormal{Khan Academy video on angles - 1}\vspace{.1in} \nopagebreak
% \label{m39370*yt-media1}\label{m39370*yt-video1}
% \raisebox{-5 pt}{ \includegraphics[width=0.5cm]{col11306.imgs/summary_www.png}}
% { (Video:  MG10088 )}
% \vspace{2pt}
% \vspace{.1in}
% \end{figure}       
% Note that for high school trigonometry you will be using degrees, not radians as
% stated in the video. \par 

\subsection*{Properties and notation}
\nopagebreak
In the diagram below, the two straight lines $AB$ and $CD$ intersect at point X, forming the four
angles $\hat{A}$, $\hat{B}$, $\hat{C}$ and $\hat{D}$.\par 
% Need new diagram here 
\setcounter{subfigure}{0}
\begin{figure}[H] % horizontal\label{m39370*uid27}
\begin{center}
\rule[.1in]{\figurerulewidth}{.005in}
\label{m39370*uid27!!!underscore!!!media}\label{
m39370*uid27!!!underscore!!!printimage}\includegraphics[width=.5\columnwidth]{
col11306.imgs/m39370_MG10C13_009.png} % m39370;MG10C13\_009.png;;;6.0;8.5;
\vspace{2pt}
\vspace{\rubberspace}\par 
% \begin{cnxcaption}
% \small \textbf{Figure 12.10: }Two intersecting straight lines with vertical
% angles $\hat{{X}_{1}},\hat{{X}_{3}}$ and $\hat{{X}_{2}},\hat{{X}_{4}}$.
% \end{cnxcaption}
\vspace{.1in}
\rule[.1in]{\figurerulewidth}{.005in} \\
\end{center}
\end{figure}       
The table summarises the special angle pairs that result.\par 
% \textbf{m39370*id315548}\par
\begin{table}[H]
\begin{center}
\begin{tabular}{|l|l|l|} \hline
Term & Property & Example\\ \hline
Acute angle & $0^{\circ} \leq \mbox{angle} \leq 90^{\circ}$ & $\hat{A}$; $\hat{C}$ \\ \hline
Right angle & Angle $= 90^{\circ}$ &  \\ \hline
Obtuse angle & $90^{\circ} \leq \mbox{angle} \leq 180^{\circ}$ & $\hat{B}$; $\hat{D}$ \\ \hline
Straight angle & Angle $= 180^{\circ}$ & $\hat{A} + \hat{B}$\\
& & $\hat{B} + \hat{C}$ etc.  \\ \hline
Reflex angle & $180^{\circ} < \mbox{angle} < 180^{\circ}$ &  \\ \hline
Adjacent angles & Angles that share a common vertex and a common side. & $\hat{A}$ and $\hat{D}$ \\ 
& & $\hat{C}$ and $\hat{D}$ etc. \\ \hline
Vertically opposite angles & Angles opposite each other when two lines intersect.& $\hat{A}=\hat{C}$\\
 & They share a vertex and are equal. & $\hat{B}=\hat{D}$\\ \hline
Supplementary angles & Angles that add up to $180^{\circ}$ & $\hat{A}+\hat{B}=180^{\circ}$\\
& & $\hat{B}+\hat{C}=180^{\circ}$ etc. \\ \hline
Complentary angles & Angles that add up to $90^{\circ}$ & \\ \hline
Revolution & Angle$=360^{\circ}$ $\hat{A}+\hat{B}+\hat{C}+\hat{D}=180^{\circ}$& \\

\end{tabular}
\end{center}
\end{table}
\par
\par
\par

\Tip{The opposite angles formed by the intersection of two straight lines are
equal. Adjacent angles on a straight line are supplementary.
}
The following video summarises what you have learnt so far
\setcounter{subfigure}{0}
\begin{figure}[H] % horizontal\label{m39370*angles-2}
\textnormal{Khan Academy video on angles - 2}\vspace{.1in} \nopagebreak
\label{m39370*yt-media2}\label{m39370*yt-video2}
\raisebox{-5 pt}{ \includegraphics[width=0.5cm]{col11306.imgs/summary_www.png}}
{ (Video:  MG10089 )}
\vspace{2pt}
\vspace{.1in}
\end{figure}       \par 

\subsubsection{ Parallel Lines intersected by Transversal Lines}
Two lines intersect if they cross each other at a point. For example, at a
traffic intersection two or more streets intersect; the middle of the
intersection is the common point between the streets.\par 
Parallel lines are lines that never intersect. For example the tracks of a
railway line are parallel (for convenience, sometimes mathematicians say they
intersect at 'a point at infinity', i.e. an infinite distance away). We wouldn't
want the tracks to intersect after as that would be catastrophic for the
train!\par 

\setcounter{subfigure}{0}
\begin{figure}[H] % horizontal\label{m39370*id316228}
\begin{center}
\label{m39370*id316228!!!underscore!!!media}\label{
m39370*id316228!!!underscore!!!printimage}\includegraphics{
col11306.imgs/m39370_MG10C13_010.png} % m39370;MG10C13\_010.png;;;6.0;8.5;
\vspace{2pt}
\vspace{.1in}
\end{center}
\end{figure}       
\par 
All these lines are parallel to each other. Notice the pair of arrow symbols for
parallel.\par 

\Note{
A section of the Australian National Railways Trans-Australian line is perhaps
one of the longest pairs of man-made parallel lines.
The Australian National Railways Trans-Australian line over the Nullarbor Plain,
is 478~km (297 miles) dead straight, from Mile 496, between Nurina and Loongana,
Western Australia, to Mile 793, between Ooldea and Watson, South
Australia.(Source: www.guinnessworldrecords.com)} % end \textsl

A transversal of two or more lines is a line that intersects these lines. For
example in Figure~12.13, $AB$ and $CD$ are two parallel lines and $EF$ is a
transversal. We say $AB\parallel CD$. The properties of the angles formed by
these intersecting lines are summarised in the table below.\par 
\setcounter{subfigure}{0}
\begin{figure}[H] % horizontal\label{m39370*uid29}
\begin{center}
\rule[.1in]{\figurerulewidth}{.005in} \\
\label{m39370*uid29!!!underscore!!!media}\label{
m39370*uid29!!!underscore!!!printimage}\includegraphics[width=.8\columnwidth]{
col11306.imgs/m39370_MG10C13_011.png} % m39370;MG10C13\_011.png;;;6.0;8.5;
\vspace{2pt}
\vspace{\rubberspace}\par \begin{cnxcaption}
\small \textbf{Figure 12.13: }Parallel lines intersected by a transversal
\end{cnxcaption}
\vspace{.1in}
\rule[.1in]{\figurerulewidth}{.005in} \\
\end{center}
\end{figure}       
% \textbf{m39370*uid30}\par
\begin{table}[H]
% \begin{table}[H]
% \\ '' '0'
\begin{center}


	  \small \textbf{Figure 13.13: }Parallel lines intersected by a
transversal
	\end{cnxcaption}
      
    \vspace{.1in}
    \rule[.1in]{\figurerulewidth}{.005in} \\
        
    \end{center}

 \end{figure}   

    \addtocounter{footnote}{-0}
    
        
    % \textbf{m38380*uid30}\par
    
    % how many colspecs?  4
          % name: cnx:colspec
            % colnum: 1
            % colwidth: 3cm
            % latex-name: columna
            % colname: angle
            % align/tgroup-align/default: //left
            % -------------------------
            % name: cnx:colspec
            % colnum: 2
            % colwidth: 3cm
            % latex-name: columnb
            % colname: Def
            % align/tgroup-align/default: //left
            % -------------------------
            % name: cnx:colspec
            % colnum: 3
            % colwidth: 3cm
            % latex-name: columnc
            % colname: Eg
            % align/tgroup-align/default: //left
            % -------------------------
            % name: cnx:colspec
            % colnum: 4
            % colwidth: 20*
            % latex-name: columnd
            % colname: notes
            % align/tgroup-align/default: left//left
            % -------------------------
      
    
    \setlength\mytablespace{8\tabcolsep}
    \addtolength\mytablespace{5\arrayrulewidth}
    \setlength\mytablewidth{\linewidth}
        
    
    \setlength\mytableroom{\mytablewidth}
    \addtolength\mytableroom{-\mytablespace}
    
    \setlength\myfixedwidth{0pt}
        \addtolength\myfixedwidth{3cm}
    \addtolength\myfixedwidth{3cm}
    \addtolength\myfixedwidth{3cm}
\setlength\mystarwidth{\mytableroom}
        \addtolength\mystarwidth{-\myfixedwidth}
        \divide\mystarwidth 20
        
    
            % ----- Table with code
            
    % \begin{table}[H]
    % \\ '' '0'
    
        \begin{center}
      
      \label{m38380*uid30}
      
    \noindent
    \tabletail{%
        \hline
        \multicolumn{4}{|p{\mytableroom}|}{\raggedleft \small \sl continued on
next page}\\
        \hline
      }
      \tablelasttail{}
     
\begin{xtabular*}{\mytablewidth}[t]{|p{3cm}|p{3cm}|p{3cm}|p{20\mystarwidth}|}
\hline
    % count in rowspan-info-nodeset: 4
    % align/colidx: left,1
    
    % rowcount: '0' | start: 'false' | colidx: '1'
    
        % Formatting a regular cell and recurring on the next sibling
        
                  \textbf{Name of angle}
                 &
      % align/colidx: left,2
    
    % rowcount: '0' | start: 'false' | colidx: '2'
    
        % Formatting a regular cell and recurring on the next sibling
        
                  \textbf{Definition}
                 &
      % align/colidx: left,3
    
    % rowcount: '0' | start: 'false' | colidx: '3'
    
        % Formatting a regular cell and recurring on the next sibling
        
                  \textbf{Examples}
                 &
      % align/colidx: left,4
    
    % rowcount: '0' | start: 'false' | colidx: '4'
    
        % Formatting a regular cell and recurring on the next sibling
        
                  \textbf{Notes}
                % make-rowspan-placeholders
    % rowspan info: col1 '0' | 'false' | '' || col2 '0' | 'false' | '' || col3
'0' | 'false' | '' || col4 '0' | 'false' | ''
     \tabularnewline\cline{1-1}\cline{2-2}\cline{3-3}\cline{4-4}
      %--------------------------------------------------------------------
    % align/colidx: left,1
    
    % rowcount: '0' | start: 'false' | colidx: '1'
    
        % Formatting a regular cell and recurring on the next sibling
        interior angles &
      % align/colidx: left,2
    
    % rowcount: '0' | start: 'false' | colidx: '2'
    
        % Formatting a regular cell and recurring on the next sibling
        the angles that lie inside the parallel lines &
      % align/colidx: left,3
    
    % rowcount: '0' | start: 'false' | colidx: '3'
    
        % Formatting a regular cell and recurring on the next sibling
        in Figure~13.13 \begin{math}a\end{math}, \begin{math}b\end{math},
\begin{math}c\end{math} and \begin{math}d\end{math} are interior angles &
      % align/colidx: left,4
    
    % rowcount: '0' | start: 'false' | colidx: '4'
    
        % Formatting a regular cell and recurring on the next sibling
        the word \textsl{interior} means inside% make-rowspan-placeholders
    % rowspan info: col1 '0' | 'false' | '' || col2 '0' | 'false' | '' || col3
'0' | 'false' | '' || col4 '0' | 'false' | ''
     \tabularnewline\cline{1-1}\cline{2-2}\cline{3-3}\cline{4-4}
      %--------------------------------------------------------------------
    % align/colidx: left,1
    
    % rowcount: '0' | start: 'false' | colidx: '1'
    
        % Formatting a regular cell and recurring on the next sibling
        adjacent angles &
      % align/colidx: left,2
    
    % rowcount: '0' | start: 'false' | colidx: '2'
    
        % Formatting a regular cell and recurring on the next sibling
        the angles share a common vertex point and line &
      % align/colidx: left,3
    
    % rowcount: '0' | start: 'false' | colidx: '3'
    
        % Formatting a regular cell and recurring on the next sibling
        in Figure~13.13 (\begin{math}a\end{math}, \begin{math}h\end{math}) are
adjacent and so are (\begin{math}h\end{math}, \begin{math}g\end{math});
(\begin{math}g\end{math}, \begin{math}b\end{math}); (\begin{math}b\end{math},
\begin{math}a\end{math}) &
      % align/colidx: left,4
    
    % rowcount: '0' | start: 'false' | colidx: '4'
    
        % Formatting a regular cell and recurring on the next sibling
        % make-rowspan-placeholders
    % rowspan info: col1 '0' | 'false' | '' || col2 '0' | 'false' | '' || col3
'0' | 'false' | '' || col4 '0' | 'false' | ''
     \tabularnewline\cline{1-1}\cline{2-2}\cline{3-3}\cline{4-4}
      %--------------------------------------------------------------------
    % align/colidx: left,1
    
    % rowcount: '0' | start: 'false' | colidx: '1'
    
        % Formatting a regular cell and recurring on the next sibling
        exterior angles &
      % align/colidx: left,2
    
    % rowcount: '0' | start: 'false' | colidx: '2'
    
        % Formatting a regular cell and recurring on the next sibling
        the angles that lie outside the parallel lines &
      % align/colidx: left,3
    
    % rowcount: '0' | start: 'false' | colidx: '3'
    
        % Formatting a regular cell and recurring on the next sibling
        in Figure~13.13 \begin{math}e\end{math}, \begin{math}f\end{math},
\begin{math}g\end{math} and \begin{math}h\end{math} are exterior angles &
      % align/colidx: left,4
    
    % rowcount: '0' | start: 'false' | colidx: '4'
    
        % Formatting a regular cell and recurring on the next sibling
        the word \textsl{exterior} means outside% make-rowspan-placeholders
    % rowspan info: col1 '0' | 'false' | '' || col2 '0' | 'false' | '' || col3
'0' | 'false' | '' || col4 '0' | 'false' | ''
     \tabularnewline\cline{1-1}\cline{2-2}\cline{3-3}\cline{4-4}
      %--------------------------------------------------------------------
    % align/colidx: left,1
    
    % rowcount: '0' | start: 'false' | colidx: '1'
    
        % Formatting a regular cell and recurring on the next sibling
        alternate interior angles &
      % align/colidx: left,2
    
    % rowcount: '0' | start: 'false' | colidx: '2'
    
        % Formatting a regular cell and recurring on the next sibling
        the interior angles that lie on opposite sides of the transversal &
      % align/colidx: left,3
    
    % rowcount: '0' | start: 'false' | colidx: '3'
    
        % Formatting a regular cell and recurring on the next sibling
        in Figure~13.13 (\begin{math}a,c\end{math}) and
(\begin{math}b\end{math},\begin{math}d\end{math}) are pairs of alternate
interior angles, \begin{math}a=c\end{math}, \begin{math}b=d\end{math} &
      % align/colidx: left,4
    
    % rowcount: '0' | start: 'false' | colidx: '4'
    
        % Formatting a regular cell and recurring on the next sibling
        
                  
    \setcounter{subfigure}{0}

\label{m38380*id316794}
    \begin{center}
   
\label{m38380*id316794!!!underscore!!!media}\label{
m38380*id316794!!!underscore!!!printimage}\includegraphics[width=0.1\textwidth]{
col11306.imgs/m38380_MG10C13_012.png} % m38380;MG10C13\_012.png;;;6.0;8.5;
        
      \vspace{2pt}
    \vspace{.1in}
    
    \end{center}



    \addtocounter{footnote}{-0}
    
                % make-rowspan-placeholders
    % rowspan info: col1 '0' | 'false' | '' || col2 '0' | 'false' | '' || col3
'0' | 'false' | '' || col4 '0' | 'false' | ''
     \tabularnewline\cline{1-1}\cline{2-2}\cline{3-3}\cline{4-4}
      %--------------------------------------------------------------------
    % align/colidx: left,1
    
    % rowcount: '0' | start: 'false' | colidx: '1'
    
        % Formatting a regular cell and recurring on the next sibling
        co-interior angles on the same side &
      % align/colidx: left,2
    
    % rowcount: '0' | start: 'false' | colidx: '2'
    
        % Formatting a regular cell and recurring on the next sibling
        co-interior angles that lie on the same side of the transversal &
      % align/colidx: left,3
    
    % rowcount: '0' | start: 'false' | colidx: '3'
    
        % Formatting a regular cell and recurring on the next sibling
        in Figure~13.13 (\begin{math}a\end{math},\begin{math}d\end{math}) and
(\begin{math}b\end{math},\begin{math}c\end{math}) are interior angles on the
same side. \begin{math}a+d={180}^{\circ }\end{math},
\begin{math}b+c={180}^{\circ }\end{math} &
      % align/colidx: left,4
    
    % rowcount: '0' | start: 'false' | colidx: '4'
    
        % Formatting a regular cell and recurring on the next sibling
        
                  
    \setcounter{subfigure}{0}

\label{m38380*id316923}
    \begin{center}
   
\label{m38380*id316923!!!underscore!!!media}\label{
m38380*id316923!!!underscore!!!printimage}\includegraphics[width=0.1\textwidth]{
col11306.imgs/m38380_MG10C13_013.png} % m38380;MG10C13\_013.png;;;6.0;8.5;
        
      \vspace{2pt}
    \vspace{.1in}
    
    \end{center}



    \addtocounter{footnote}{-0}
    
                % make-rowspan-placeholders
    % rowspan info: col1 '0' | 'false' | '' || col2 '0' | 'false' | '' || col3
'0' | 'false' | '' || col4 '0' | 'false' | ''
     \tabularnewline\cline{1-1}\cline{2-2}\cline{3-3}\cline{4-4}
      %--------------------------------------------------------------------
    % align/colidx: left,1
    
    % rowcount: '0' | start: 'false' | colidx: '1'
    
        % Formatting a regular cell and recurring on the next sibling
        corresponding angles &
      % align/colidx: left,2
    
    % rowcount: '0' | start: 'false' | colidx: '2'
    
        % Formatting a regular cell and recurring on the next sibling
        the angles on the same side of the transversal and the same side of the
parallel lines &
      % align/colidx: left,3
    
    % rowcount: '0' | start: 'false' | colidx: '3'
    
        % Formatting a regular cell and recurring on the next sibling
        in Figure~13.13 \begin{math}\left(a,e\right)\end{math},
\begin{math}\left(b,f\right)\end{math}, \begin{math}\left(c,g\right)\end{math}
and \begin{math}\left(d,h\right)\end{math} are pairs of corresponding angles. 
\begin{math}a=e\end{math}, \begin{math}b=f\end{math}, \begin{math}c=g\end{math},
\begin{math}d=h\end{math} &
      % align/colidx: left,4
    
    % rowcount: '0' | start: 'false' | colidx: '4'
    
        % Formatting a regular cell and recurring on the next sibling
        
                  
    \setcounter{subfigure}{0}

\label{m38380*id317099}
    \begin{center}
   
\label{m38380*id317099!!!underscore!!!media}\label{
m38380*id317099!!!underscore!!!printimage}\includegraphics[width=0.1\textwidth]{
col11306.imgs/m38380_MG10C13_014.png} % m38380;MG10C13\_014.png;;;6.0;8.5;
        
      \vspace{2pt}
    \vspace{.1in}
    
    \end{center}



    \addtocounter{footnote}{-0}
    
                % make-rowspan-placeholders
    % rowspan info: col1 '0' | 'false' | '' || col2 '0' | 'false' | '' || col3
'0' | 'false' | '' || col4 '0' | 'false' | ''
     \tabularnewline\cline{1-1}\cline{2-2}\cline{3-3}\cline{4-4}
      %--------------------------------------------------------------------
    \end{xtabular*}
      \end{center}
    \begin{center}{\small\bfseries Table 13.3}\end{center}
    %\end{table}
    
    \addtocounter{footnote}{-0}
    
    \par
  
\label{m38380*eip-918}The following video summarises what you have learnt so far


    \setcounter{subfigure}{0}


	\begin{figure}[H] % horizontal\label{m38380*angles-3}
    
    
    \textnormal{Khan Academy video on angles - 3}\vspace{.1in} \nopagebreak
  \label{m38380*yt-media3}\label{m38380*yt-video3}
            \raisebox{-5 pt}{
\includegraphics[width=0.5cm]{col11306.imgs/summary_www.png}} { (Video:  P10116
)}
      
      \vspace{2pt}
    \vspace{.1in}
    
    

 \end{figure}   

    \addtocounter{footnote}{-0}
    \par \label{m38380*eip-933}
\begin{tabular}{cc}
	\hspace*{-50pt}\raisebox{-8
mm}{\hspace{-0.2in}\includegraphics[width=0.75in]{col11306.imgs/psfact2.png} }
& 

	\begin{minipage}{0.85\textwidth}
	\begin{note}
      {note: }\textbf{Euclid's Parallel line postulate.} If a straight line
falling across two other straight lines makes the two interior angles on the
same side less than two right angles (180\begin{math}{}^{\circ }\end{math}), the
two straight lines, if produced indefinitely, will meet on that side.
This postulate can be used to prove many identities about the angles formed when
two parallel lines are cut by a transversal. 
	\end{note}
	\end{minipage}
	\end{tabular}
	\par
      
\label{m38380*notfhsst!!!underscore!!!id534}
\begin{tabular}{cc}
	   \hspace*{-50pt}\raisebox{-8 mm}{
\includegraphics[width=0.5in]{col11306.imgs/pstip2.png}  }& 

	\begin{minipage}{0.85\textwidth}
	\begin{note}
      {tip: }
        \label{m38380*id317145}\begin{enumerate}[noitemsep,
label=\textbf{\arabic*}. ] 
            \label{m38380*uid31}\item If two parallel lines are intersected by a
transversal, the sum of the co-interior angles on the same side of the
transversal is 180\begin{math}{}^{\circ }\end{math}.
\label{m38380*uid32}\item If two parallel lines are intersected by a
transversal, the alternate interior angles are equal.
\label{m38380*uid33}\item If two parallel lines are intersected by a
transversal, the corresponding angles are equal.
\label{m38380*uid34}\item If two lines are intersected by a transversal such
that any pair of co-interior angles on the same side is supplementary, then the
two lines are parallel.
\label{m38380*uid35}\item If two lines are intersected by a transversal such
that a pair of alternate interior angles are equal, then the lines are parallel.
\label{m38380*uid36}\item If two lines are intersected by a transversal such
that a pair of alternate corresponding angles are equal, then the lines are
parallel.
\end{enumerate}
        

	\end{note}
	\end{minipage}
	\end{tabular}
	\par
      
\par \pagebreak
            \label{m38380*eip-499}\vspace{.5cm} 
      
      \noindent
      \hspace*{-30pt}\includegraphics[width=0.5in]{col11306.imgs/pspencil2.png} 
 \raisebox{25mm}{   
      \begin{mdframed}[linewidth=4, leftmargin=40, rightmargin=40]  
      \begin{exercise}
    \noindent\textbf{Exercise 13.1: Finding angles}\label{m38380*eip-769}
  \label{m38380*eip-384}
    Find all the unknown angles in the following figure:

    \setcounter{subfigure}{0}


	\begin{figure}[H] % horizontal\label{m38380*id63478}
    \begin{center}
   
\label{m38380*id63478!!!underscore!!!media}\label{
m38380*id63478!!!underscore!!!printimage}\includegraphics[width=0.5\textwidth]{
col11306.imgs/m38380_angle1.png} % m38380;angle1.png;;;6.0;8.5;
        
      \vspace{2pt}
    \vspace{.1in}
    
    \end{center}

 \end{figure}   

    \addtocounter{footnote}{-0}
    
  \par 
\vspace{5pt}

\label{m38380*eip-775}\noindent\textbf{Solution to Exercise }
  \label{m38380*eip-312}\begin{enumerate}[noitemsep, label=\textbf{Step}
\textbf{\arabic*}. ] 
            \leftskip=20pt\rightskip=\leftskip\item
\begin{math}\text{AB}\parallel \text{CD}\end{math}. So 
\begin{math}x={30}^{\ensuremath{{\,}^{\circ}}}\end{math} (alternate interior
angles)\item
\label{m38380*eid6734}\nopagebreak\noindent{}\settowidth{\mymathboxwidth}{\begin
{equation}
    \begin{array}{ccc}\hfill 160+y& =& 180\hfill \\ \hfill y& =&
{20}^{\ensuremath{{\,}^{\circ}}}\hfill \end{array}\tag{13.1}
      \end{equation}
    }
    \typeout{Columnwidth = \the\columnwidth}\typeout{math as usual width =
\the\mymathboxwidth}
    \ifthenelse{\lengthtest{\mymathboxwidth < \columnwidth}}{% if the math fits,
do it again, for real
    \begin{equation}
    \begin{array}{ccc}\hfill 160+y& =& 180\hfill \\ \hfill y& =&
{20}^{\ensuremath{{\,}^{\circ}}}\hfill \end{array}\tag{13.1}
      \end{equation}
    }{% else, if it doesn't fit
    \setlength{\mymathboxwidth}{\columnwidth}
      \addtolength{\mymathboxwidth}{-48pt}
    \par\vspace{12pt}\noindent\begin{minipage}{\columnwidth}
    \parbox[t]{\mymathboxwidth}{\large\begin{math}
    160+y=180y={20}^{\ensuremath{{\,}^{\circ}}}\end{math}}\hfill
    \parbox[t]{48pt}{\raggedleft 
    (13.1)}
    \end{minipage}\vspace{12pt}\par
    }% end of conditional for this bit of math
    \typeout{math as usual width = \the\mymathboxwidth}
     (co-interior angles on the same side)\end{enumerate}
        


    \end{exercise}
    \end{mdframed}
    }
    \noindent
  \par
            \label{m38380*eip-882}\vspace{.5cm} 
      
      \noindent
      \hspace*{-30pt}\includegraphics[width=0.5in]{col11306.imgs/pspencil2.png} 
 \raisebox{25mm}{   
      \begin{mdframed}[linewidth=4, leftmargin=40, rightmargin=40]  
      \begin{exercise}
    \noindent\textbf{Exercise 13.2: Parallel lines}\label{m38380*eip-529}
  \label{m38380*eip-438}
    Determine if there are any parallel lines in the following figure:

    \setcounter{subfigure}{0}


	\begin{figure}[H] % horizontal\label{m38380*id9876}
    \begin{center}
   
\label{m38380*id9876!!!underscore!!!media}\label{
m38380*id9876!!!underscore!!!printimage}\includegraphics[width=0.2\textwidth]{
col11306.imgs/m38380_angle2.png} % m38380;angle2.png;;;6.0;8.5;
        
      \vspace{2pt}
    \vspace{.1in}
    
    \end{center}

 \end{figure}   

    \addtocounter{footnote}{-0}
    
  \par 
\vspace{10pt}

\label{m38380*eip-668}\noindent\textbf{Solution to Exercise }
  \label{m38380*eip-82}\begin{enumerate}[noitemsep, label=\textbf{Step}
\textbf{\arabic*}. ] 
            \leftskip=20pt\rightskip=\leftskip\item Line EF cannot be parallel
to either AB or CD since it cuts both these lines. Lines AB and CD may be
parallel.\item We can show that two lines are parallel if we can find one of the
pairs of special angles. We know that 
\begin{math}{\hat{E}}_{2}={25}^{\ensuremath{{\,}^{\circ}}}\end{math}(opposite
angles). And then we note that 
\label{m38380*eid6634}\nopagebreak\noindent{}\settowidth{\mymathboxwidth}{\begin
{equation}
    \begin{array}{ccc}\hfill {\hat{E}}_{2}& =& {\hat{F}}_{4}\hfill \\ & =&
{25}^{\ensuremath{{\,}^{\circ}}}\hfill \end{array}\tag{13.2}
      \end{equation}
    }
    \typeout{Columnwidth = \the\columnwidth}\typeout{math as usual width =
\the\mymathboxwidth}
    \ifthenelse{\lengthtest{\mymathboxwidth < \columnwidth}}{% if the math fits,
do it again, for real
    \begin{equation}
    \begin{array}{ccc}\hfill {\hat{E}}_{2}& =& {\hat{F}}_{4}\hfill \\ & =&
{25}^{\ensuremath{{\,}^{\circ}}}\hfill \end{array}\tag{13.2}
      \end{equation}
    }{% else, if it doesn't fit
    \setlength{\mymathboxwidth}{\columnwidth}
      \addtolength{\mymathboxwidth}{-48pt}
    \par\vspace{12pt}\noindent\begin{minipage}{\columnwidth}
    \parbox[t]{\mymathboxwidth}{\large\begin{math}
   
{\hat{E}}_{2}={\hat{F}}_{4}={25}^{\ensuremath{{\,}^{\circ}}}\end{math}}\hfill
    \parbox[t]{48pt}{\raggedleft 
    (13.2)}
    \end{minipage}\vspace{12pt}\par
    }% end of conditional for this bit of math
    \typeout{math as usual width = \the\mymathboxwidth}
     So we have shown that 
\begin{math}\text{AB}\parallel \text{CD}\end{math}\hspace{1ex}(corresponding
angles)\end{enumerate}
        


    \end{exercise}
    \end{mdframed}
    }
    \noindent
\vspace{-1cm}
  \label{m38380*secfhsst!!!underscore!!!id550}
        \subsubsection{ Angles }
        \nopagebreak
        \label{m38380*eip-407}\begin{enumerate}[noitemsep,
label=\textbf{\arabic*}. ] 
            \item Use adjacent, corresponding, co-interior and alternate angles
to fill in all the angles labeled with letters in the diagram below:

    \setcounter{subfigure}{0}


	\begin{figure}[H] % horizontal\label{m38380*id317272}
    \begin{center}
   
\label{m38380*id317272!!!underscore!!!media}\label{
m38380*id317272!!!underscore!!!printimage}\includegraphics[width=0.35\textwidth]
{col11306.imgs/m38380_MG10C13_015.png} % m38380;MG10C13\_015.png;;;6.0;8.5;
        
      \vspace{2pt}
    \vspace{.1in}
    
    \end{center}

 \end{figure}   

    \addtocounter{footnote}{-0}
            \item Find all the unknown angles in the figure below:

    \setcounter{subfigure}{0}


	\begin{figure}[H] % horizontal\label{m38380*id317298}
    \begin{center}
   
\label{m38380*id317298!!!underscore!!!media}\label{
m38380*id317298!!!underscore!!!printimage}\includegraphics[width=0.4\textwidth]{
col11306.imgs/m38380_MG10C13_016.png} % m38380;MG10C13\_016.png;;;6.0;8.5;
        
      \vspace{2pt}
    \vspace{.1in}
    
    \end{center}

 \end{figure}   

    \addtocounter{footnote}{-0}
\pagebreak
            \item Find the value of \begin{math}x\end{math} in the figure below:

    \setcounter{subfigure}{0}


	\begin{figure}[H] % horizontal\label{m38380*id317330}
    \begin{center}
   
\label{m38380*id317330!!!underscore!!!media}\label{
m38380*id317330!!!underscore!!!printimage}\includegraphics[width=0.2\textwidth]{
col11306.imgs/m38380_MG10C13_017.png} % m38380;MG10C13\_017.png;;;6.0;8.5;
        
      \vspace{2pt}
    \vspace{.1in}
    
    \end{center}

 \end{figure}   

    \addtocounter{footnote}{-0}
            \item  Determine whether there are pairs of parallel lines in the
following figures.
\label{m38380*id79123}\begin{enumerate}[noitemsep, label=\textbf{\alph*}. ] 
            \item 

    \setcounter{subfigure}{0}


	\begin{figure}[H] % horizontal\label{m38380*id317353}
    \begin{center}
   
\label{m38380*id317353!!!underscore!!!media}\label{
m38380*id317353!!!underscore!!!printimage}\includegraphics[width=0.4\textwidth]{
col11306.imgs/m38380_MG10C13_018.png} % m38380;MG10C13\_018.png;;;6.0;8.5;
        
      \vspace{2pt}
    \vspace{.1in}
    
    \end{center}

 \end{figure}   

    \addtocounter{footnote}{-0}
    
\item 

    \setcounter{subfigure}{0}


	\begin{figure}[H] % horizontal\label{m38380*id317367}
    \begin{center}
   
\label{m38380*id317367!!!underscore!!!media}\label{
m38380*id317367!!!underscore!!!printimage}\includegraphics[width=0.4\textwidth]{
col11306.imgs/m38380_MG10C13_019.png} % m38380;MG10C13\_019.png;;;6.0;8.5;
        
      \vspace{2pt}
    \vspace{.1in}
    
    \end{center}

 \end{figure}   

    \addtocounter{footnote}{-0}
    \item 

    \setcounter{subfigure}{0}


	\begin{figure}[H] % horizontal\label{m38380*id317384}
    \begin{center}
   
\label{m38380*id317384!!!underscore!!!media}\label{
m38380*id317384!!!underscore!!!printimage}\includegraphics[width=0.4\textwidth]{
col11306.imgs/m38380_MG10C13_020.png} % m38380;MG10C13\_020.png;;;6.0;8.5;
        
      \vspace{2pt}
    \vspace{.1in}
    
    \end{center}

 \end{figure}   

    \addtocounter{footnote}{-0}
    \end{enumerate}
                \item If AB is parallel to CD and AB is parallel to EF, prove
that CD is parallel to EF:


    \setcounter{subfigure}{0}


	\begin{figure}[H] % horizontal\label{m38380*id317408}
    \begin{center}
   
\label{m38380*id317408!!!underscore!!!media}\label{
m38380*id317408!!!underscore!!!printimage}\includegraphics[width=0.35\textwidth]
{col11306.imgs/m38380_MG10C13_021.png} % m38380;MG10C13\_021.png;;;6.0;8.5;
        
      \vspace{2pt}
    \vspace{.1in}
    
    \end{center}

 \end{figure}   

    \addtocounter{footnote}{-0}
            \end{enumerate}
        \label{m38380*eip-115}The following video shows some problems with their
solutions


    \setcounter{subfigure}{0}


	\begin{figure}[H] % horizontal\label{m38380*angles-4}
    
    
    \textnormal{Khan Academy video on angles - 4}\vspace{.1in} \nopagebreak
  \label{m38380*yt-media4}\label{m38380*yt-video4}
            \raisebox{-5 pt}{
\includegraphics[width=0.5cm]{col11306.imgs/summary_www.png}} { (Video:  P10117
)}
      
      \vspace{2pt}
    \vspace{.1in}
    
    

 \end{figure}   

    \addtocounter{footnote}{-0}
    \par 
        

      
    
\par \raisebox{-5
pt}{\includegraphics[width=0.5cm]{col11306.imgs/summary_www.png}} Find the
answers with the shortcodes:
 \par \begin{tabular}[h]{cccccc}
 (1.) lxF  &  (2.) lxL  &  (3.) lxM  &  (4.) lxe  &  (5.) lxt  & \end{tabular}



    \section{ Polygons}
        \nopagebreak
        \label{m38380} $ \hspace{-5pt}\begin{array}{cccccccccccc}  
\includegraphics[width=0.75cm]{col11306.imgs/summary_fullmarks.png} &  
\end{array} $ \hspace{2 pt}\raisebox{-5 pt}{} {(section shortcode: P10118 )}
\par 
      
      \label{m38380*id317435}If you take some lines and join them such that the
end point of the first line meets the starting point of the last line, you will
get a \textsl{polygon}. Each line that makes up the polygon is known as a
\textsl{side}. A polygon has interior angles. These are the angles that are
inside the polygon. The number of sides of a polygon equals the number of
interior angles. If a polygon has equal length sides and equal interior angles,
then the polygon is called a \textsl{regular polygon}. Some examples of polygons
are shown in Figure~13.28.\par 
      
    \setcounter{subfigure}{0}


	\begin{figure}[H] % horizontal\label{m38380*uid37}
    \begin{center}
    \rule[.1in]{\figurerulewidth}{.005in} \\
       
\label{m38380*uid37!!!underscore!!!media}\label{
m38380*uid37!!!underscore!!!printimage}\includegraphics[width=0.3\textwidth]{
col11306.imgs/m38380_MG10C13_0221.png} % m38380;MG10C13\_0221.png;;;6.0;8.5;
        
      \vspace{2pt}
    \vspace{\rubberspace}\par \begin{cnxcaption}
	  \small \textbf{Figure 13.28: }Examples of polygons. They are all
regular, except for the one marked *
	\end{cnxcaption}
      
    \vspace{.1in}
    \rule[.1in]{\figurerulewidth}{.005in} \\
        
    \end{center}

 \end{figure}   

    \addtocounter{footnote}{-0}
\vspace{-1cm}    
      \label{m38380*uid38}
        \subsection{ Triangles}
        \nopagebreak
        \label{m38380*id317485}A triangle is a three-sided polygon. Triangles
are usually split into three categories: equilateral, isosceles, and scalene,
depending on how many of the sides are of equal length. A fourth category,
right-angled triangle (or simply 'right triangle') is used to refer to triangles
with one right angle. Note that all right-angled triangles are also either
isosceles (if the other two sides are equal) or scalene (it should be clear why
you cannot have an equilateral right triangle!). The properties of these
triangles are summarised in Table 13.4.\par 
        
    % \textbf{m38380*uid39}\par
    
    % how many colspecs?  3
          % name: cnx:colspec
            % colnum: 1
            % colwidth: 10*
            % latex-name: columna
            % colname: 
            % align/tgroup-align/default: //left
            % -------------------------
            % name: cnx:colspec
            % colnum: 2
            % colwidth: 10*
            % latex-name: columnb
            % colname: 
            % align/tgroup-align/default: //left
            % -------------------------
            % name: cnx:colspec
            % colnum: 3
            % colwidth: 10*
            % latex-name: columnc
            % colname: 
            % align/tgroup-align/default: //left
            % -------------------------
      
    
    \setlength\mytablespace{6\tabcolsep}
    \addtolength\mytablespace{4\arrayrulewidth}
    \setlength\mytablewidth{\linewidth}
        
    
    \setlength\mytableroom{\mytablewidth}
    \addtolength\mytableroom{-\mytablespace}
    
    \setlength\myfixedwidth{0pt}
    \setlength\mystarwidth{\mytableroom}
        \addtolength\mystarwidth{-\myfixedwidth}
        \divide\mystarwidth 30
        
    
            % ----- Table with code
            
    % \begin{table}[H]
    % \\ '' '0'
    
        \begin{center}
      
      \label{m38380*uid39}
      
    \noindent
    \tabletail{%
        \hline
        \multicolumn{3}{|p{\mytableroom}|}{\raggedleft \small \sl continued on
next page}\\
        \hline
      }
      \tablelasttail{}
     
\begin{xtabular*}{\mytablewidth}[t]{|p{10\mystarwidth}|p{10\mystarwidth}|p{
10\mystarwidth}|}\hline
    % count in rowspan-info-nodeset: 3
    % align/colidx: left,1
    
    % rowcount: '0' | start: 'false' | colidx: '1'
    
        % Formatting a regular cell and recurring on the next sibling
        Name &
      % align/colidx: left,2
    
    % rowcount: '0' | start: 'false' | colidx: '2'
    
        % Formatting a regular cell and recurring on the next sibling
        Diagram &
      % align/colidx: left,3
    
    % rowcount: '0' | start: 'false' | colidx: '3'
    
        % Formatting a regular cell and recurring on the next sibling
        Properties% make-rowspan-placeholders
    % rowspan info: col1 '0' | 'false' | '' || col2 '0' | 'false' | '' || col3
'0' | 'false' | ''
     \tabularnewline\cline{1-1}\cline{2-2}\cline{3-3}
      %--------------------------------------------------------------------
    % align/colidx: left,1
    
    % rowcount: '0' | start: 'false' | colidx: '1'
    
        % Formatting a regular cell and recurring on the next sibling
        equilateral &
      % align/colidx: left,2
    
    % rowcount: '0' | start: 'false' | colidx: '2'
    
        % Formatting a regular cell and recurring on the next sibling
        
                  
    \setcounter{subfigure}{0}

\label{m38380*id317558}
    \begin{center}
   
\label{m38380*id317558!!!underscore!!!media}\label{
m38380*id317558!!!underscore!!!printimage}\includegraphics[height=0.15\textwidth
]{col11306.imgs/m38380_MG10C13_023.png} % m38380;MG10C13\_023.png;;;6.0;8.5;
        
      \vspace{2pt}
    \vspace{.1in}
    
    \end{center}



    \addtocounter{footnote}{-0}
    
                 &
      % align/colidx: left,3
    
    % rowcount: '0' | start: 'false' | colidx: '3'
    
        % Formatting a regular cell and recurring on the next sibling
        All three sides are equal in length (denoted by the short lines drawn
through all the sides of equal length) and all three angles are equal.%
make-rowspan-placeholders
    % rowspan info: col1 '0' | 'false' | '' || col2 '0' | 'false' | '' || col3
'0' | 'false' | ''
     \tabularnewline\cline{1-1}\cline{2-2}\cline{3-3}
      %--------------------------------------------------------------------
    % align/colidx: left,1
    
    % rowcount: '0' | start: 'false' | colidx: '1'
    
        % Formatting a regular cell and recurring on the next sibling
        isosceles &
      % align/colidx: left,2
    
    % rowcount: '0' | start: 'false' | colidx: '2'
    
        % Formatting a regular cell and recurring on the next sibling
        
                  
    \setcounter{subfigure}{0}

\label{m38380*id317593}
    \begin{center}
   
\label{m38380*id317593!!!underscore!!!media}\label{
m38380*id317593!!!underscore!!!printimage}\includegraphics[height=0.15\textwidth
]{col11306.imgs/m38380_MG10C13_024.png} % m38380;MG10C13\_024.png;;;6.0;8.5;
        
      \vspace{2pt}
    \vspace{.1in}
    
    \end{center}



    \addtocounter{footnote}{-0}
    
                 &
      % align/colidx: left,3
    
    % rowcount: '0' | start: 'false' | colidx: '3'
    
        % Formatting a regular cell and recurring on the next sibling
        Two sides are equal in length. The angles opposite the equal sides are
equal.% make-rowspan-placeholders
    % rowspan info: col1 '0' | 'false' | '' || col2 '0' | 'false' | '' || col3
'0' | 'false' | ''
     \tabularnewline\cline{1-1}\cline{2-2}\cline{3-3}
      %--------------------------------------------------------------------
    % align/colidx: left,1
    
    % rowcount: '0' | start: 'false' | colidx: '1'
    
        % Formatting a regular cell and recurring on the next sibling
        right-angled &
      % align/colidx: left,2
    
    % rowcount: '0' | start: 'false' | colidx: '2'
    
        % Formatting a regular cell and recurring on the next sibling
        
                  
    \setcounter{subfigure}{0}

\label{m38380*id317628}
    \begin{center}
   
\label{m38380*id317628!!!underscore!!!media}\label{
m38380*id317628!!!underscore!!!printimage}\includegraphics[height=0.15\textwidth
]{col11306.imgs/m38380_MG10C13_025.png} % m38380;MG10C13\_025.png;;;6.0;8.5;
        
      \vspace{2pt}
    \vspace{.1in}
    
    \end{center}



    \addtocounter{footnote}{-0}
    
                 &
      % align/colidx: left,3
    
    % rowcount: '0' | start: 'false' | colidx: '3'
    
        % Formatting a regular cell and recurring on the next sibling
        This triangle has one right angle. The side opposite this angle is
called the \textsl{hypotenuse}.% make-rowspan-placeholders
    % rowspan info: col1 '0' | 'false' | '' || col2 '0' | 'false' | '' || col3
'0' | 'false' | ''
     \tabularnewline\cline{1-1}\cline{2-2}\cline{3-3}
      %--------------------------------------------------------------------
    % align/colidx: left,1
    
    % rowcount: '0' | start: 'false' | colidx: '1'
    
        % Formatting a regular cell and recurring on the next sibling
        scalene (non-syllabus) &
      % align/colidx: left,2
    
    % rowcount: '0' | start: 'false' | colidx: '2'
    
        % Formatting a regular cell and recurring on the next sibling
        
                  
    \setcounter{subfigure}{0}

\label{m38380*id317668}
    \begin{center}
   
\label{m38380*id317668!!!underscore!!!media}\label{
m38380*id317668!!!underscore!!!printimage}\includegraphics[width=0.15\textwidth]
{col11306.imgs/m38380_MG10C13_026.png} % m38380;MG10C13\_026.png;;;6.0;8.5;
        
      \vspace{2pt}
    \vspace{.1in}
    
    \end{center}



    \addtocounter{footnote}{-0}
    
                 &
      % align/colidx: left,3
    
    % rowcount: '0' | start: 'false' | colidx: '3'
    
        % Formatting a regular cell and recurring on the next sibling
        All sides and angles are different.% make-rowspan-placeholders
    % rowspan info: col1 '0' | 'false' | '' || col2 '0' | 'false' | '' || col3
'0' | 'false' | ''
     \tabularnewline\cline{1-1}\cline{2-2}\cline{3-3}
      %--------------------------------------------------------------------
    \end{xtabular*}
      \end{center}
    \begin{center}{\small\bfseries Table 13.4}: Types of Triangles\end{center}
    %\end{table}
    
    \addtocounter{footnote}{-0}
    
    \par
  
        \label{m38380*id317683}We use the notation \begin{math}▵ABC\end{math} to
refer to a triangle with corners labeled \begin{math}A\end{math},
\begin{math}B\end{math}, and \begin{math}C\end{math}.\par 
        \label{m38380*uid40}
        \subsubsection{ Properties of Triangles}
        \nopagebreak
        
          
\label{m38380*secfhsst!!!underscore!!!id655}
        \subsubsection{  Investigation : Sum of the angles in a triangle }
        \nopagebreak
        
          \label{m38380*id317720}\begin{enumerate}[noitemsep,
label=\textbf{\arabic*}. ] 
            \label{m38380*uid41}\item Draw on a piece of paper a triangle of any
size and shape
\label{m38380*uid42}\item Cut it out and label the angles
\begin{math}\hat{A}\end{math}, \begin{math}\hat{B}\end{math} and
\begin{math}\hat{C}\end{math} on both sides of the paper
\label{m38380*uid43}\item Draw dotted lines as shown and cut along these lines
to get three pieces of paper
\label{m38380*uid44}\item Place them along your ruler as shown to see that
\begin{math}\hat{A}+\hat{B}+\hat{C}={180}^{\circ }\end{math}\end{enumerate}
        
          \label{m38380*id317868}
    \setcounter{subfigure}{0}


	\begin{figure}[H] % horizontal\label{m38380*id317874}
    \begin{center}
   
\label{m38380*id317874!!!underscore!!!media}\label{
m38380*id317874!!!underscore!!!printimage}\includegraphics[width=0.4\textwidth]{
col11306.imgs/m38380_MG10C13_027.png} % m38380;MG10C13\_027.png;;;6.0;8.5;
        
      \vspace{2pt}
    \vspace{.1in}
    
    \end{center}

 \end{figure}   

    \addtocounter{footnote}{-0}
    
    \setcounter{subfigure}{0}


	\begin{figure}[H] % horizontal\label{m38380*id317886}
    \begin{center}
   
\label{m38380*id317886!!!underscore!!!media}\label{
m38380*id317886!!!underscore!!!printimage}\includegraphics[width=0.4\textwidth]{
col11306.imgs/m38380_MG10C13_028.png} % m38380;MG10C13\_028.png;;;6.0;8.5;
        
      \vspace{2pt}
    \vspace{.1in}
    
    \end{center}

 \end{figure}   

    \addtocounter{footnote}{-0}
    


 \par 

\label{m38380*notfhsst!!!underscore!!!id670}
\begin{tabular}{cc}
	   \hspace*{-50pt}\raisebox{-8 mm}{
\includegraphics[width=0.5in]{col11306.imgs/pstip2.png}  }& 

	\begin{minipage}{0.85\textwidth}
	\begin{note}
      {tip: }The sum of the angles in a triangle is 180\begin{math}{}^{\circ
}\end{math}.
	\end{note}
	\end{minipage}
	\end{tabular}
	\par
      
          
    \setcounter{subfigure}{0}


	\begin{figure}[H] % horizontal\label{m38380*uid45}
    \begin{center}
    \rule[.1in]{\figurerulewidth}{.005in} \\
       
\label{m38380*uid45!!!underscore!!!media}\label{
m38380*uid45!!!underscore!!!printimage}\includegraphics[width=0.2\textwidth]{
col11306.imgs/m38380_MG10C13_029.png} % m38380;MG10C13\_029.png;;;6.0;8.5;
        
      \vspace{2pt}
    \vspace{\rubberspace}\par \begin{cnxcaption}
	  \small \textbf{Figure 13.35: }In any triangle, \begin{math}\angle
A+\angle B+\angle C={180}^{\circ }\end{math}
	\end{cnxcaption}
      
    \vspace{.1in}
    \rule[.1in]{\figurerulewidth}{.005in} \\
        
    \end{center}

 \end{figure}   

    \addtocounter{footnote}{-0}
    
\label{m38380*notfhsst!!!underscore!!!id678}
\begin{tabular}{cc}
	   \hspace*{-50pt}\raisebox{-8 mm}{
\includegraphics[width=0.5in]{col11306.imgs/pstip2.png}  }& 

	\begin{minipage}{0.85\textwidth}
	\begin{note}
      {tip: }Any exterior angle of a triangle is equal to the sum of the two
opposite interior angles. An exterior angle is formed by extending any one of
the sides.
	\end{note}
	\end{minipage}
	\end{tabular}
	\par
      
          
    \setcounter{subfigure}{0}


	\begin{figure}[H] % horizontal\label{m38380*uid46}
    \begin{center}
    \rule[.1in]{\figurerulewidth}{.005in} \\
       
\label{m38380*uid46!!!underscore!!!media}\label{
m38380*uid46!!!underscore!!!printimage}\includegraphics[width=300px]{
col11306.imgs/m38380_MG10C13_030.png} % m38380;MG10C13\_030.png;;;6.0;8.5;
        
      \vspace{2pt}
    \vspace{\rubberspace}\par \begin{cnxcaption}
	  \small \textbf{Figure 13.36: }In any triangle, any exterior angle is
equal to the sum of the two opposite interior angles.
	\end{cnxcaption}
      
    \vspace{.1in}
    \rule[.1in]{\figurerulewidth}{.005in} \\
        
    \end{center}

 \end{figure}   

    \addtocounter{footnote}{-0}
    
        
        \label{m38380*uid47}
        \subsubsection{ Congruent Triangles}
        \nopagebreak
        
          
          \label{m38380*eip-370}Two triangles are called congruent if one of
them can be superimposed, that is moved on top of to exactly cover, the other.
In other words, if both triangles have all of the same angles and sides, then
they are called congruent. To decide whether two triangles are congruent, it is
not necessary to check every side and angle. The following list describes
various requirements that are sufficient to know when two triangles are
congruent.\par 
    % \textbf{m38380*id317997}\par
    
    % how many colspecs?  3
          % name: cnx:colspec
            % colnum: 1
            % colwidth: 10*
            % latex-name: columna
            % colname: 
            % align/tgroup-align/default: //left
            % -------------------------
            % name: cnx:colspec
            % colnum: 2
            % colwidth: 10*
            % latex-name: columnb
            % colname: 
            % align/tgroup-align/default: //left
            % -------------------------
            % name: cnx:colspec
            % colnum: 3
            % colwidth: 10*
            % latex-name: columnc
            % colname: 
            % align/tgroup-align/default: //left
            % -------------------------
      
    
    \setlength\mytablespace{6\tabcolsep}
    \addtolength\mytablespace{4\arrayrulewidth}
    \setlength\mytablewidth{\linewidth}
        
    
    \setlength\mytableroom{\mytablewidth}
    \addtolength\mytableroom{-\mytablespace}
    
    \setlength\myfixedwidth{0pt}
    \setlength\mystarwidth{\mytableroom}
        \addtolength\mystarwidth{-\myfixedwidth}
        \divide\mystarwidth 30
        
    
            % ----- Table with code
            
    % \begin{table}[H]
    % \\ '' '0'
    
        \begin{center}
      
      \label{m38380*id317997}
      
    \noindent
    \tabletail{%
        \hline
        \multicolumn{3}{|p{\mytableroom}|}{\raggedleft \small \sl continued on
next page}\\
        \hline
      }
      \tablelasttail{}
     
\begin{xtabular*}{\mytablewidth}[t]{|p{10\mystarwidth}|p{10\mystarwidth}|p{
10\mystarwidth}|}\hline
    % count in rowspan-info-nodeset: 3
    % align/colidx: left,1
    
    % rowcount: '0' | start: 'false' | colidx: '1'
    
        % Formatting a regular cell and recurring on the next sibling
        
                    \textbf{Label}
                   &
      % align/colidx: left,2
    
    % rowcount: '0' | start: 'false' | colidx: '2'
    
        % Formatting a regular cell and recurring on the next sibling
        
                    \textbf{Description}
                   &
      % align/colidx: left,3
    
    % rowcount: '0' | start: 'false' | colidx: '3'
    
        % Formatting a regular cell and recurring on the next sibling
        
                    \textbf{Diagram}
                  % make-rowspan-placeholders
    % rowspan info: col1 '0' | 'false' | '' || col2 '0' | 'false' | '' || col3
'0' | 'false' | ''
     \tabularnewline\cline{1-1}\cline{2-2}\cline{3-3}
      %--------------------------------------------------------------------
    % align/colidx: left,1
    
    % rowcount: '0' | start: 'false' | colidx: '1'
    
        % Formatting a regular cell and recurring on the next sibling
        RHS &
      % align/colidx: left,2
    
    % rowcount: '0' | start: 'false' | colidx: '2'
    
        % Formatting a regular cell and recurring on the next sibling
        If the hypotenuse and one side of a right-angled triangle are equal to
the hypotenuse and the respective side of another triangle, then the triangles
are congruent. &
      % align/colidx: left,3
    
    % rowcount: '0' | start: 'false' | colidx: '3'
    
        % Formatting a regular cell and recurring on the next sibling
        
                    
    \setcounter{subfigure}{0}

\label{m38380*id318071}
    \begin{center}
   
\label{m38380*id318071!!!underscore!!!media}\label{
m38380*id318071!!!underscore!!!printimage}\includegraphics[width=0.3\textwidth]{
col11306.imgs/m38380_MG10C13_031.png} % m38380;MG10C13\_031.png;;;6.0;8.5;
        
      \vspace{2pt}
    \vspace{.1in}
    
    \end{center}



    \addtocounter{footnote}{-0}
    
                  % make-rowspan-placeholders
    % rowspan info: col1 '0' | 'false' | '' || col2 '0' | 'false' | '' || col3
'0' | 'false' | ''
     \tabularnewline\cline{1-1}\cline{2-2}\cline{3-3}
      %--------------------------------------------------------------------
    % align/colidx: left,1
    
    % rowcount: '0' | start: 'false' | colidx: '1'
    
        % Formatting a regular cell and recurring on the next sibling
        SSS &
      % align/colidx: left,2
    
    % rowcount: '0' | start: 'false' | colidx: '2'
    
        % Formatting a regular cell and recurring on the next sibling
        If three sides of a triangle are equal in length to the same sides of
another triangle, then the two triangles are congruent &
      % align/colidx: left,3
    
    % rowcount: '0' | start: 'false' | colidx: '3'
    
        % Formatting a regular cell and recurring on the next sibling
        
                    
    \setcounter{subfigure}{0}

\label{m38380*id318107}
    \begin{center}
   
\label{m38380*id318107!!!underscore!!!media}\label{
m38380*id318107!!!underscore!!!printimage}\includegraphics[width=0.3\textwidth]{
col11306.imgs/m38380_MG10C13_032.png} % m38380;MG10C13\_032.png;;;6.0;8.5;
        
      \vspace{2pt}
    \vspace{.1in}
    
    \end{center}



    \addtocounter{footnote}{-0}
    
                  % make-rowspan-placeholders
    % rowspan info: col1 '0' | 'false' | '' || col2 '0' | 'false' | '' || col3
'0' | 'false' | ''
     \tabularnewline\cline{1-1}\cline{2-2}\cline{3-3}
      %--------------------------------------------------------------------
    % align/colidx: left,1
    
    % rowcount: '0' | start: 'false' | colidx: '1'
    
        % Formatting a regular cell and recurring on the next sibling
        SAS &
      % align/colidx: left,2
    
    % rowcount: '0' | start: 'false' | colidx: '2'
    
        % Formatting a regular cell and recurring on the next sibling
        If two sides and the included angle of one triangle are equal to the
same two sides and included angle of another triangle, then the two triangles
are congruent. &
      % align/colidx: left,3
    
    % rowcount: '0' | start: 'false' | colidx: '3'
    
        % Formatting a regular cell and recurring on the next sibling
        
                    
    \setcounter{subfigure}{0}

\label{m38380*id318143}
    \begin{center}
   
\label{m38380*id318143!!!underscore!!!media}\label{
m38380*id318143!!!underscore!!!printimage}\includegraphics[width=0.3\textwidth]{
col11306.imgs/m38380_MG10C13_033.png} % m38380;MG10C13\_033.png;;;6.0;8.5;
        
      \vspace{2pt}
    \vspace{.1in}
    
    \end{center}



    \addtocounter{footnote}{-0}
    
                  % make-rowspan-placeholders
    % rowspan info: col1 '0' | 'false' | '' || col2 '0' | 'false' | '' || col3
'0' | 'false' | ''
     \tabularnewline\cline{1-1}\cline{2-2}\cline{3-3}
      %--------------------------------------------------------------------
    % align/colidx: left,1
    
    % rowcount: '0' | start: 'false' | colidx: '1'
    
        % Formatting a regular cell and recurring on the next sibling
        AAS &
      % align/colidx: left,2
    
    % rowcount: '0' | start: 'false' | colidx: '2'
    
        % Formatting a regular cell and recurring on the next sibling
        If one side and two angles of one triangle are equal to the same one
side and two angles of another triangle, then the two triangles are congruent. &
      % align/colidx: left,3
    
    % rowcount: '0' | start: 'false' | colidx: '3'
    
        % Formatting a regular cell and recurring on the next sibling
        
                    
    \setcounter{subfigure}{0}

\label{m38380*id318178}
    \begin{center}
   
\label{m38380*id318178!!!underscore!!!media}\label{
m38380*id318178!!!underscore!!!printimage}\includegraphics[width=0.3\textwidth]{
col11306.imgs/m38380_MG10C13_034.png} % m38380;MG10C13\_034.png;;;6.0;8.5;
        
      \vspace{2pt}
    \vspace{.1in}
    
    \end{center}



    \addtocounter{footnote}{-0}
    
                  % make-rowspan-placeholders
    % rowspan info: col1 '0' | 'false' | '' || col2 '0' | 'false' | '' || col3
'0' | 'false' | ''
     \tabularnewline\cline{1-1}\cline{2-2}\cline{3-3}
      %--------------------------------------------------------------------
    \end{xtabular*}
      \end{center}
    \begin{center}{\small\bfseries Table 13.5}\end{center}
    %\end{table}
    
    \addtocounter{footnote}{-0}
    
    \par
  
 \pagebreak       
        \label{m38380*uid48}
        \subsubsection{ Similar Triangles}
        \nopagebreak
        
          
          \label{m38380*eip-665}Two triangles are called similar if it is
possible to proportionally shrink or stretch one of them to a triangle congruent
to the other. Congruent triangles are similar triangles, but similar triangles
are only congruent if they are the same size to begin with.\par 
    % \textbf{m38380*id318196}\par
    
    % how many colspecs?  2
          % name: cnx:colspec
            % colnum: 1
            % colwidth: 10*
            % latex-name: columna
            % colname: 
            % align/tgroup-align/default: //left
            % -------------------------
            % name: cnx:colspec
            % colnum: 2
            % colwidth: 10*
            % latex-name: columnb
            % colname: 
            % align/tgroup-align/default: //left
            % -------------------------
      
    
    \setlength\mytablespace{4\tabcolsep}
    \addtolength\mytablespace{3\arrayrulewidth}
    \setlength\mytablewidth{\linewidth}
        
    
    \setlength\mytableroom{\mytablewidth}
    \addtolength\mytableroom{-\mytablespace}
    
    \setlength\myfixedwidth{0pt}
    \setlength\mystarwidth{\mytableroom}
        \addtolength\mystarwidth{-\myfixedwidth}
        \divide\mystarwidth 20
        
    
            % ----- Table with code
            
    % \begin{table}[H]
    % \\ '' '0'
    
        \begin{center}
      
      \label{m38380*id318196}
      
    \noindent
    \tabletail{%
        \hline
        \multicolumn{2}{|p{\mytableroom}|}{\raggedleft \small \sl continued on
next page}\\
        \hline
      }
      \tablelasttail{}
     
\begin{xtabular*}{\mytablewidth}[t]{|p{10\mystarwidth}|p{10\mystarwidth}|}\hline
    % count in rowspan-info-nodeset: 2
    % align/colidx: left,1
    
    % rowcount: '0' | start: 'false' | colidx: '1'
    
        % Formatting a regular cell and recurring on the next sibling
        
                    \textbf{Description}
                   &
      % align/colidx: left,2
    
    % rowcount: '0' | start: 'false' | colidx: '2'
    
        % Formatting a regular cell and recurring on the next sibling
        
                    \textbf{Diagram}
                  % make-rowspan-placeholders
    % rowspan info: col1 '0' | 'false' | '' || col2 '0' | 'false' | ''
     \tabularnewline\cline{1-1}\cline{2-2}
      %--------------------------------------------------------------------
    % align/colidx: left,1
    
    % rowcount: '0' | start: 'false' | colidx: '1'
    
        % Formatting a regular cell and recurring on the next sibling
        If all three pairs of corresponding angles of two triangles are equal,
then the triangles are similar. &
      % align/colidx: left,2
    
    % rowcount: '0' | start: 'false' | colidx: '2'
    
        % Formatting a regular cell and recurring on the next sibling
        
                    
    \setcounter{subfigure}{0}

\label{m38380*id318251}
    \begin{center}
   
\label{m38380*id318251!!!underscore!!!media}\label{
m38380*id318251!!!underscore!!!printimage}\includegraphics[width=0.35\textwidth]
{col11306.imgs/m38380_MG10C13_035.png} % m38380;MG10C13\_035.png;;;6.0;8.5;
        
      \vspace{2pt}
    \vspace{.1in}
    
    \end{center}



    \addtocounter{footnote}{-0}
    
                  % make-rowspan-placeholders
    % rowspan info: col1 '0' | 'false' | '' || col2 '0' | 'false' | ''
     \tabularnewline\cline{1-1}\cline{2-2}
      %--------------------------------------------------------------------
    % align/colidx: left,1
    
    % rowcount: '0' | start: 'false' | colidx: '1'
    
        % Formatting a regular cell and recurring on the next sibling
        If all pairs of corresponding sides of two triangles are in proportion,
then the triangles are similar. &
      % align/colidx: left,2
    
    % rowcount: '0' | start: 'false' | colidx: '2'
    
        % Formatting a regular cell and recurring on the next sibling
        
                    
    \setcounter{subfigure}{0}

\label{m38380*id318279}
    \begin{center}
   
\label{m38380*id318279!!!underscore!!!media}\label{
m38380*id318279!!!underscore!!!printimage}\includegraphics[width=0.35\textwidth]
{col11306.imgs/m38380_MG10C13_036.png} % m38380;MG10C13\_036.png;;;6.0;8.5;
        
      \vspace{2pt}
    \vspace{.1in}
    
    \end{center}



    \addtocounter{footnote}{-0}
    
                    \begin{math}\frac{x}{p}=\frac{y}{q}=\frac{z}{r}\end{math}
                  % make-rowspan-placeholders
    % rowspan info: col1 '0' | 'false' | '' || col2 '0' | 'false' | ''
     \tabularnewline\cline{1-1}\cline{2-2}
      %--------------------------------------------------------------------
    \end{xtabular*}
      \end{center}
    \begin{center}{\small\bfseries Table 13.6}\end{center}
    %\end{table}
    
    \addtocounter{footnote}{-0}
    
    \par
  
        
        \label{m38380*uid49}
        \subsubsection{ The theorem of Pythagoras}
        \nopagebreak
        
          
          \label{m38380*id318328}
    \setcounter{subfigure}{0}


	\begin{figure}[H] % horizontal\label{m38380*id318334}
    \begin{center}
   
\label{m38380*id318334!!!underscore!!!media}\label{
m38380*id318334!!!underscore!!!printimage}\includegraphics[height=0.3\textwidth]
{col11306.imgs/m38380_MG10C13_037.png} % m38380;MG10C13\_037.png;;;6.0;8.5;
        
      \vspace{2pt}
    \vspace{.1in}
    
    \end{center}

 \end{figure}   

    \addtocounter{footnote}{-0}
    



If \begin{math}▵\end{math}ABC is right-angled (\begin{math}\hat{B}={90}^{\circ
}\end{math}) then
\begin{math}{b}^{2}={a}^{2}+{c}^{2}\end{math}\newline
    \textbf{Converse:}
If \begin{math}{b}^{2}={a}^{2}+{c}^{2}\end{math}, then
\begin{math}▵\end{math}ABC is right-angled (\begin{math}\hat{B}={90}^{\circ
}\end{math}).

\vspace{\rubberspace}\par 
\label{m38380*eip-693}\vspace{.5cm} 
      
      \noindent
      \hspace*{-30pt}\includegraphics[width=0.5in]{col11306.imgs/pspencil2.png} 
 \raisebox{25mm}{   
      \begin{mdframed}[linewidth=4, leftmargin=40, rightmargin=40]  
      \begin{exercise}
    \noindent\textbf{Exercise 13.3: Triangles}\label{m38380*eip-221}
  \label{m38380*eip-154}
   In the following figure, determine if the two triangles are congruent, then
use the result to help you find the unknown letters.

    \setcounter{subfigure}{0}


	\begin{figure}[H] % horizontal\label{m38380*id5433}
    \begin{center}
   
\label{m38380*id5433!!!underscore!!!media}\label{
m38380*id5433!!!underscore!!!printimage}\includegraphics[width=300px]{
col11306.imgs/m38380_triangle1.png} % m38380;triangle1.png;;;6.0;8.5;
        
      \vspace{2pt}
    \vspace{.1in}
    
    \end{center}

 \end{figure}   

    \addtocounter{footnote}{-0}
    
  \par 
\vspace{5pt}

\label{m38380*eip-477}\noindent\textbf{Solution to Exercise }
  \label{m38380*eip-156}\begin{enumerate}[noitemsep, label=\textbf{Step}
\textbf{\arabic*}. ] 
            \leftskip=20pt\rightskip=\leftskip\item
\label{m38380*id65234}\begin{math}D\hat{E}C=B\hat{A}C={55}^{\ensuremath{{\,}^{
\circ}}}\end{math}\hspace{1ex}(angles in a triangle add up to 
\begin{math}{180}^{\ensuremath{{\,}^{\circ}}}\end{math}).\par 
     
\label{m38380*id1166232344812}\begin{math}A\hat{B}C=C\hat{D}E={90}^{\ensuremath{
{\,}^{\circ}}}\end{math}\hspace{1ex} (given)\par 
     
\label{m38380*id1166233687055}\begin{math}\text{DE}=\text{AB}=3\end{math}\hspace
{1ex} (given)\par 
      \label{m38380*id1166230595268}\nopagebreak\noindent{}
        \settowidth{\mymathboxwidth}{\begin{equation}
    \therefore \Delta \text{ABC}\equiv \Delta \text{CDE}\tag{13.3}
      \end{equation}
    }
    \typeout{Columnwidth = \the\columnwidth}\typeout{math as usual width =
\the\mymathboxwidth}
    \ifthenelse{\lengthtest{\mymathboxwidth < \columnwidth}}{% if the math fits,
do it again, for real
    \begin{equation}
    \therefore \Delta \text{ABC}\equiv \Delta \text{CDE}\tag{13.3}
      \end{equation}
    }{% else, if it doesn't fit
    \setlength{\mymathboxwidth}{\columnwidth}
      \addtolength{\mymathboxwidth}{-48pt}
    \par\vspace{12pt}\noindent\begin{minipage}{\columnwidth}
    \parbox[t]{\mymathboxwidth}{\large\begin{math}
    \therefore \Delta \text{ABC}\equiv \Delta \text{CDE}\end{math}}\hfill
    \parbox[t]{48pt}{\raggedleft 
    (13.3)}
    \end{minipage}\vspace{12pt}\par
    }% end of conditional for this bit of math
    \typeout{math as usual width = \the\mymathboxwidth}
    
      \item \label{m38380*id6543}We use Pythagoras to find x: 
     
\label{m38380*id1166232298669}\nopagebreak\noindent{}\settowidth{\mymathboxwidth
}{\begin{equation}
    \begin{array}{ccc}\hfill {\text{CE}}^{2}& =&
{\text{DE}}^{2}+{\text{DC}}^{2}\hfill \\ \hfill {5}^{2}& =&
{3}^{2}+{x}^{2}\hfill \\ \hfill {x}^{2}& =& 16\hfill \\ \hfill x& =& 4\hfill
\end{array}\tag{13.4}
      \end{equation}
    }
    \typeout{Columnwidth = \the\columnwidth}\typeout{math as usual width =
\the\mymathboxwidth}
    \ifthenelse{\lengthtest{\mymathboxwidth < \columnwidth}}{% if the math fits,
do it again, for real
    \begin{equation}
    \begin{array}{ccc}\hfill {\text{CE}}^{2}& =&
{\text{DE}}^{2}+{\text{DC}}^{2}\hfill \\ \hfill {5}^{2}& =&
{3}^{2}+{x}^{2}\hfill \\ \hfill {x}^{2}& =& 16\hfill \\ \hfill x& =& 4\hfill
\end{array}\tag{13.4}
      \end{equation}
    }{% else, if it doesn't fit
    \setlength{\mymathboxwidth}{\columnwidth}
      \addtolength{\mymathboxwidth}{-48pt}
    \par\vspace{12pt}\noindent\begin{minipage}{\columnwidth}
    \parbox[t]{\mymathboxwidth}{\large\begin{math}
   
{\text{CE}}^{2}={\text{DE}}^{2}+{\text{DC}}^{2}{5}^{2}={3}^{2}+{x}^{2}{x}^{2}
=16x=4\end{math}}\hfill
    \parbox[t]{48pt}{\raggedleft 
    (13.4)}
    \end{minipage}\vspace{12pt}\par
    }% end of conditional for this bit of math
    \typeout{math as usual width = \the\mymathboxwidth}
    
      \par 
     
\label{m38380*id1166229837179}\begin{math}y={35}^{\ensuremath{{\,}^{\circ}}}\end
{math}\hspace{1ex}(angles in a triangle)\par 
     
\label{m38380*id1166228117285}\begin{math}z=5\end{math}\hspace{1ex}(congruent
triangles, 
\begin{math}\text{AC}=\text{CE}\end{math})\par \end{enumerate}
        


    \end{exercise}
    \end{mdframed}
    }
    \noindent
\pagebreak
  \label{m38380*secfhsst!!!underscore!!!id824}
        \subsubsection{  Triangles }
        \nopagebreak
        
          \label{m38380*id318528}\begin{enumerate}[noitemsep,
label=\textbf{\arabic*}. ] 
            \label{m38380*uid50}\item Calculate the unknown variables in each of
the following figures. All
lengths are in mm.

    \setcounter{subfigure}{0}


	\begin{figure}[H] % horizontal\label{m38380*id318548}
    \begin{center}
   
\label{m38380*id318548!!!underscore!!!media}\label{
m38380*id318548!!!underscore!!!printimage}\includegraphics[width=0.6\textwidth]{
col11306.imgs/m38380_MG10C13_038.png} % m38380;MG10C13\_038.png;;;6.0;8.5;
        
  %    \vspace{2pt}
   % \vspace{.1in}
    
    \end{center}

 \end{figure}   

    \addtocounter{footnote}{-0}
            \label{m38380*uid51}\item State whether or not the following pairs
of triangles are congruent or not.
Give reasons for your answers. If there is not enough information to make a
descision, say why.

    \setcounter{subfigure}{0}


	\begin{figure}[H] % horizontal\label{m38380*id318571}
    \begin{center}
   
\label{m38380*id318571!!!underscore!!!media}\label{
m38380*id318571!!!underscore!!!printimage}\includegraphics[width=0.5\textwidth]{
col11306.imgs/m38380_MG10C13_039.png} % m38380;MG10C13\_039.png;;;6.0;8.5;
        
   %   \vspace{2pt}
   % \vspace{.1in}
    
    \end{center}

 \end{figure}   

    \addtocounter{footnote}{-0}
            \end{enumerate}
        
          

        
  
      
      \label{m38380*eip-75}
\par \raisebox{-5
pt}{\includegraphics[width=0.5cm]{col11306.imgs/summary_www.png}} Find the
answers with the shortcodes:
 \par \begin{tabular}[h]{cccccc}
 (1.) lxz  &  (2.) lxu  & \end{tabular}



        \subsubsection{ Quadrilaterals}
        \nopagebreak
        \label{m38380*eip-366}
A quadrilateral is a four sided figure. There are some special quadrilaterals
(trapezium, parallelogram, kite, rhombus, square, rectangle) which you will
learn about in Geometry (Chapter~14). 
\par \label{m38380*uid91}
        \subsubsection{ Other polygons}
        \nopagebreak
        
        
        \label{m38380*id319439}There are many other polygons, some of which are
given in the table below.\par 
        
    % \textbf{m38380*uid92}\par
    
    % how many colspecs?  2
          % name: cnx:colspec
            % colnum: 1
            % colwidth: 10*
            % latex-name: columna
            % colname: 
            % align/tgroup-align/default: //left
            % -------------------------
            % name: cnx:colspec
            % colnum: 2
            % colwidth: 10*
            % latex-name: columnb
            % colname: 
            % align/tgroup-align/default: //left
            % -------------------------
      
    
    \setlength\mytablespace{4\tabcolsep}
    \addtolength\mytablespace{3\arrayrulewidth}
    \setlength\mytablewidth{\linewidth}
        
    
    \setlength\mytableroom{\mytablewidth}
    \addtolength\mytableroom{-\mytablespace}
    
    \setlength\myfixedwidth{0pt}
    \setlength\mystarwidth{\mytableroom}
        \addtolength\mystarwidth{-\myfixedwidth}
        \divide\mystarwidth 20
        
    
      % ----- Begin capturing width of table in LR mode woof
      \settowidth{\mytableboxwidth}{\begin{tabular}[t]{|l|l|}\hline
    % count in rowspan-info-nodeset: 2
    % align/colidx: left,1
    
    % rowcount: '0' | start: 'false' | colidx: '1'
    
        % Formatting a regular cell and recurring on the next sibling
        Sides &
      % align/colidx: left,2
    
    % rowcount: '0' | start: 'false' | colidx: '2'
    
        % Formatting a regular cell and recurring on the next sibling
        Name% make-rowspan-placeholders
    % rowspan info: col1 '0' | 'false' | '' || col2 '0' | 'false' | ''
     \tabularnewline\cline{1-1}\cline{2-2}
      %--------------------------------------------------------------------
    % align/colidx: left,1
    
    % rowcount: '0' | start: 'false' | colidx: '1'
    
        % Formatting a regular cell and recurring on the next sibling
        5 &
      % align/colidx: left,2
    
    % rowcount: '0' | start: 'false' | colidx: '2'
    
        % Formatting a regular cell and recurring on the next sibling
        pentagon% make-rowspan-placeholders
    % rowspan info: col1 '0' | 'false' | '' || col2 '0' | 'false' | ''
     \tabularnewline\cline{1-1}\cline{2-2}
      %--------------------------------------------------------------------
    % align/colidx: left,1
    
    % rowcount: '0' | start: 'false' | colidx: '1'
    
        % Formatting a regular cell and recurring on the next sibling
        6 &
      % align/colidx: left,2
    
    % rowcount: '0' | start: 'false' | colidx: '2'
    
        % Formatting a regular cell and recurring on the next sibling
        hexagon% make-rowspan-placeholders
    % rowspan info: col1 '0' | 'false' | '' || col2 '0' | 'false' | ''
     \tabularnewline\cline{1-1}\cline{2-2}
      %--------------------------------------------------------------------
    % align/colidx: left,1
    
    % rowcount: '0' | start: 'false' | colidx: '1'
    
        % Formatting a regular cell and recurring on the next sibling
        7 &
      % align/colidx: left,2
    
    % rowcount: '0' | start: 'false' | colidx: '2'
    
        % Formatting a regular cell and recurring on the next sibling
        heptagon% make-rowspan-placeholders
    % rowspan info: col1 '0' | 'false' | '' || col2 '0' | 'false' | ''
     \tabularnewline\cline{1-1}\cline{2-2}
      %--------------------------------------------------------------------
    % align/colidx: left,1
    
    % rowcount: '0' | start: 'false' | colidx: '1'
    
        % Formatting a regular cell and recurring on the next sibling
        8 &
      % align/colidx: left,2
    
    % rowcount: '0' | start: 'false' | colidx: '2'
    
        % Formatting a regular cell and recurring on the next sibling
        octagon% make-rowspan-placeholders
    % rowspan info: col1 '0' | 'false' | '' || col2 '0' | 'false' | ''
     \tabularnewline\cline{1-1}\cline{2-2}
      %--------------------------------------------------------------------
    % align/colidx: left,1
    
    % rowcount: '0' | start: 'false' | colidx: '1'
    
        % Formatting a regular cell and recurring on the next sibling
        10 &
      % align/colidx: left,2
    
    % rowcount: '0' | start: 'false' | colidx: '2'
    
        % Formatting a regular cell and recurring on the next sibling
        decagon% make-rowspan-placeholders
    % rowspan info: col1 '0' | 'false' | '' || col2 '0' | 'false' | ''
     \tabularnewline\cline{1-1}\cline{2-2}
      %--------------------------------------------------------------------
    % align/colidx: left,1
    
    % rowcount: '0' | start: 'false' | colidx: '1'
    
        % Formatting a regular cell and recurring on the next sibling
        15 &
      % align/colidx: left,2
    
    % rowcount: '0' | start: 'false' | colidx: '2'
    
        % Formatting a regular cell and recurring on the next sibling
        pentadecagon% make-rowspan-placeholders
    % rowspan info: col1 '0' | 'false' | '' || col2 '0' | 'false' | ''
     \tabularnewline\cline{1-1}\cline{2-2}
      %--------------------------------------------------------------------
    \end{tabular}} % end mytableboxwidth set
      \addtocounter{footnote}{-0}
      
      % ----- End capturing width of table in LR mode
    
        % ----- LR or paragraph mode: must test
        % ----- Begin capturing height of table
        \settoheight{\mytableboxheight}{\begin{tabular}[t]{|l|l|}\hline
    % count in rowspan-info-nodeset: 2
    % align/colidx: left,1
    
    % rowcount: '0' | start: 'false' | colidx: '1'
    
        % Formatting a regular cell and recurring on the next sibling
        Sides &
      % align/colidx: left,2
    
    % rowcount: '0' | start: 'false' | colidx: '2'
    
        % Formatting a regular cell and recurring on the next sibling
        Name% make-rowspan-placeholders
    % rowspan info: col1 '0' | 'false' | '' || col2 '0' | 'false' | ''
     \tabularnewline\cline{1-1}\cline{2-2}
      %--------------------------------------------------------------------
    % align/colidx: left,1
    
    % rowcount: '0' | start: 'false' | colidx: '1'
    
        % Formatting a regular cell and recurring on the next sibling
        5 &
      % align/colidx: left,2
    
    % rowcount: '0' | start: 'false' | colidx: '2'
    
        % Formatting a regular cell and recurring on the next sibling
        pentagon% make-rowspan-placeholders
    % rowspan info: col1 '0' | 'false' | '' || col2 '0' | 'false' | ''
     \tabularnewline\cline{1-1}\cline{2-2}
      %--------------------------------------------------------------------
    % align/colidx: left,1
    
    % rowcount: '0' | start: 'false' | colidx: '1'
    
        % Formatting a regular cell and recurring on the next sibling
        6 &
      % align/colidx: left,2
    
    % rowcount: '0' | start: 'false' | colidx: '2'
    
        % Formatting a regular cell and recurring on the next sibling
        hexagon% make-rowspan-placeholders
    % rowspan info: col1 '0' | 'false' | '' || col2 '0' | 'false' | ''
     \tabularnewline\cline{1-1}\cline{2-2}
      %--------------------------------------------------------------------
    % align/colidx: left,1
    
    % rowcount: '0' | start: 'false' | colidx: '1'
    
        % Formatting a regular cell and recurring on the next sibling
        7 &
      % align/colidx: left,2
    
    % rowcount: '0' | start: 'false' | colidx: '2'
    
        % Formatting a regular cell and recurring on the next sibling
        heptagon% make-rowspan-placeholders
    % rowspan info: col1 '0' | 'false' | '' || col2 '0' | 'false' | ''
     \tabularnewline\cline{1-1}\cline{2-2}
      %--------------------------------------------------------------------
    % align/colidx: left,1
    
    % rowcount: '0' | start: 'false' | colidx: '1'
    
        % Formatting a regular cell and recurring on the next sibling
        8 &
      % align/colidx: left,2
    
    % rowcount: '0' | start: 'false' | colidx: '2'
    
        % Formatting a regular cell and recurring on the next sibling
        octagon% make-rowspan-placeholders
    % rowspan info: col1 '0' | 'false' | '' || col2 '0' | 'false' | ''
     \tabularnewline\cline{1-1}\cline{2-2}
      %--------------------------------------------------------------------
    % align/colidx: left,1
    
    % rowcount: '0' | start: 'false' | colidx: '1'
    
        % Formatting a regular cell and recurring on the next sibling
        10 &
      % align/colidx: left,2
    
    % rowcount: '0' | start: 'false' | colidx: '2'
    
        % Formatting a regular cell and recurring on the next sibling
        decagon% make-rowspan-placeholders
    % rowspan info: col1 '0' | 'false' | '' || col2 '0' | 'false' | ''
     \tabularnewline\cline{1-1}\cline{2-2}
      %--------------------------------------------------------------------
    % align/colidx: left,1
    
    % rowcount: '0' | start: 'false' | colidx: '1'
    
        % Formatting a regular cell and recurring on the next sibling
        15 &
      % align/colidx: left,2
    
    % rowcount: '0' | start: 'false' | colidx: '2'
    
        % Formatting a regular cell and recurring on the next sibling
        pentadecagon% make-rowspan-placeholders
    % rowspan info: col1 '0' | 'false' | '' || col2 '0' | 'false' | ''
     \tabularnewline\cline{1-1}\cline{2-2}
      %--------------------------------------------------------------------
    \end{tabular}} % end mytableboxheight set
        \settodepth{\mytableboxdepth}{\begin{tabular}[t]{|l|l|}\hline
    % count in rowspan-info-nodeset: 2
    % align/colidx: left,1
    
    % rowcount: '0' | start: 'false' | colidx: '1'
    
        % Formatting a regular cell and recurring on the next sibling
        Sides &
      % align/colidx: left,2
    
    % rowcount: '0' | start: 'false' | colidx: '2'
    
        % Formatting a regular cell and recurring on the next sibling
        Name% make-rowspan-placeholders
    % rowspan info: col1 '0' | 'false' | '' || col2 '0' | 'false' | ''
     \tabularnewline\cline{1-1}\cline{2-2}
      %--------------------------------------------------------------------
    % align/colidx: left,1
    
    % rowcount: '0' | start: 'false' | colidx: '1'
    
        % Formatting a regular cell and recurring on the next sibling
        5 &
      % align/colidx: left,2
    
    % rowcount: '0' | start: 'false' | colidx: '2'
    
        % Formatting a regular cell and recurring on the next sibling
        pentagon% make-rowspan-placeholders
    % rowspan info: col1 '0' | 'false' | '' || col2 '0' | 'false' | ''
     \tabularnewline\cline{1-1}\cline{2-2}
      %--------------------------------------------------------------------
    % align/colidx: left,1
    
    % rowcount: '0' | start: 'false' | colidx: '1'
    
        % Formatting a regular cell and recurring on the next sibling
        6 &
      % align/colidx: left,2
    
    % rowcount: '0' | start: 'false' | colidx: '2'
    
        % Formatting a regular cell and recurring on the next sibling
        hexagon% make-rowspan-placeholders
    % rowspan info: col1 '0' | 'false' | '' || col2 '0' | 'false' | ''
     \tabularnewline\cline{1-1}\cline{2-2}
      %--------------------------------------------------------------------
    % align/colidx: left,1
    
    % rowcount: '0' | start: 'false' | colidx: '1'
    
        % Formatting a regular cell and recurring on the next sibling
        7 &
      % align/colidx: left,2
    
    % rowcount: '0' | start: 'false' | colidx: '2'
    
        % Formatting a regular cell and recurring on the next sibling
        heptagon% make-rowspan-placeholders
    % rowspan info: col1 '0' | 'false' | '' || col2 '0' | 'false' | ''
     \tabularnewline\cline{1-1}\cline{2-2}
      %--------------------------------------------------------------------
    % align/colidx: left,1
    
    % rowcount: '0' | start: 'false' | colidx: '1'
    
        % Formatting a regular cell and recurring on the next sibling
        8 &
      % align/colidx: left,2
    
    % rowcount: '0' | start: 'false' | colidx: '2'
    
        % Formatting a regular cell and recurring on the next sibling
        octagon% make-rowspan-placeholders
    % rowspan info: col1 '0' | 'false' | '' || col2 '0' | 'false' | ''
     \tabularnewline\cline{1-1}\cline{2-2}
      %--------------------------------------------------------------------
    % align/colidx: left,1
    
    % rowcount: '0' | start: 'false' | colidx: '1'
    
        % Formatting a regular cell and recurring on the next sibling
        10 &
      % align/colidx: left,2
    
    % rowcount: '0' | start: 'false' | colidx: '2'
    
        % Formatting a regular cell and recurring on the next sibling
        decagon% make-rowspan-placeholders
    % rowspan info: col1 '0' | 'false' | '' || col2 '0' | 'false' | ''
     \tabularnewline\cline{1-1}\cline{2-2}
      %--------------------------------------------------------------------
    % align/colidx: left,1
    
    % rowcount: '0' | start: 'false' | colidx: '1'
    
        % Formatting a regular cell and recurring on the next sibling
        15 &
      % align/colidx: left,2
    
    % rowcount: '0' | start: 'false' | colidx: '2'
    
        % Formatting a regular cell and recurring on the next sibling
        pentadecagon% make-rowspan-placeholders
    % rowspan info: col1 '0' | 'false' | '' || col2 '0' | 'false' | ''
     \tabularnewline\cline{1-1}\cline{2-2}
      %--------------------------------------------------------------------
    \end{tabular}} % end mytableboxdepth set
        \addtolength{\mytableboxheight}{\mytableboxdepth}
        % ----- End capturing height of table
        \addtocounter{footnote}{-0}
        
        \ifthenelse{\mytableboxwidth<\textwidth}{% the table fits in LR mode
          \addtolength{\mytableboxwidth}{-\mytablespace}
          \typeout{textheight: \the\textheight}
          \typeout{mytableboxheight: \the\mytableboxheight}
          \typeout{textwidth: \the\textwidth}
          \typeout{mytableboxwidth: \the\mytableboxwidth}
          \ifthenelse{\mytableboxheight<\textheight}{%
        
    % \begin{table}[H]
    % \\ '' '0'
    
        \begin{center}
      
      \label{m38380*uid92}
      
    \noindent
    \begin{tabular}[t]{|l|l|}\hline
    % count in rowspan-info-nodeset: 2
    % align/colidx: left,1
    
    % rowcount: '0' | start: 'false' | colidx: '1'
    
        % Formatting a regular cell and recurring on the next sibling
        Sides &
      % align/colidx: left,2
    
    % rowcount: '0' | start: 'false' | colidx: '2'
    
        % Formatting a regular cell and recurring on the next sibling
        Name% make-rowspan-placeholders
    % rowspan info: col1 '0' | 'false' | '' || col2 '0' | 'false' | ''
     \tabularnewline\cline{1-1}\cline{2-2}
      %--------------------------------------------------------------------
    % align/colidx: left,1
    
    % rowcount: '0' | start: 'false' | colidx: '1'
    
        % Formatting a regular cell and recurring on the next sibling
        5 &
      % align/colidx: left,2
    
    % rowcount: '0' | start: 'false' | colidx: '2'
    
        % Formatting a regular cell and recurring on the next sibling
        pentagon% make-rowspan-placeholders
    % rowspan info: col1 '0' | 'false' | '' || col2 '0' | 'false' | ''
     \tabularnewline\cline{1-1}\cline{2-2}
      %--------------------------------------------------------------------
    % align/colidx: left,1
    
    % rowcount: '0' | start: 'false' | colidx: '1'
    
        % Formatting a regular cell and recurring on the next sibling
        6 &
      % align/colidx: left,2
    
    % rowcount: '0' | start: 'false' | colidx: '2'
    
        % Formatting a regular cell and recurring on the next sibling
        hexagon% make-rowspan-placeholders
    % rowspan info: col1 '0' | 'false' | '' || col2 '0' | 'false' | ''
     \tabularnewline\cline{1-1}\cline{2-2}
      %--------------------------------------------------------------------
    % align/colidx: left,1
    
    % rowcount: '0' | start: 'false' | colidx: '1'
    
        % Formatting a regular cell and recurring on the next sibling
        7 &
      % align/colidx: left,2
    
    % rowcount: '0' | start: 'false' | colidx: '2'
    
        % Formatting a regular cell and recurring on the next sibling
        heptagon% make-rowspan-placeholders
    % rowspan info: col1 '0' | 'false' | '' || col2 '0' | 'false' | ''
     \tabularnewline\cline{1-1}\cline{2-2}
      %--------------------------------------------------------------------
    % align/colidx: left,1
    
    % rowcount: '0' | start: 'false' | colidx: '1'
    
        % Formatting a regular cell and recurring on the next sibling
        8 &
      % align/colidx: left,2
    
    % rowcount: '0' | start: 'false' | colidx: '2'
    
        % Formatting a regular cell and recurring on the next sibling
        octagon% make-rowspan-placeholders
    % rowspan info: col1 '0' | 'false' | '' || col2 '0' | 'false' | ''
     \tabularnewline\cline{1-1}\cline{2-2}
      %--------------------------------------------------------------------
    % align/colidx: left,1
    
    % rowcount: '0' | start: 'false' | colidx: '1'
    
        % Formatting a regular cell and recurring on the next sibling
        10 &
      % align/colidx: left,2
    
    % rowcount: '0' | start: 'false' | colidx: '2'
    
        % Formatting a regular cell and recurring on the next sibling
        decagon% make-rowspan-placeholders
    % rowspan info: col1 '0' | 'false' | '' || col2 '0' | 'false' | ''
     \tabularnewline\cline{1-1}\cline{2-2}
      %--------------------------------------------------------------------
    % align/colidx: left,1
    
    % rowcount: '0' | start: 'false' | colidx: '1'
    
        % Formatting a regular cell and recurring on the next sibling
        15 &
      % align/colidx: left,2
    
    % rowcount: '0' | start: 'false' | colidx: '2'
    
        % Formatting a regular cell and recurring on the next sibling
        pentadecagon% make-rowspan-placeholders
    % rowspan info: col1 '0' | 'false' | '' || col2 '0' | 'false' | ''
     \tabularnewline\cline{1-1}\cline{2-2}
      %--------------------------------------------------------------------
    \end{tabular}
      \end{center}
    \begin{center}{\small\bfseries Table 13.7}: Table of some polygons and their
number of sides.\end{center}
    %\end{table}
    
    \addtocounter{footnote}{-0}
    
          }{ % else
        
    % \begin{table}[H]
    % \\ '' '0'
    
        \begin{center}
      
      \label{m38380*uid92}
      
    \noindent
    \tabletail{%
        \hline
        \multicolumn{2}{|p{\mytableboxwidth}|}{\raggedleft \small \sl continued
on next page}\\
        \hline
      }
      \tablelasttail{}
      \begin{xtabular}[t]{|l|l|}\hline
    % count in rowspan-info-nodeset: 2
    % align/colidx: left,1
    
    % rowcount: '0' | start: 'false' | colidx: '1'
    
        % Formatting a regular cell and recurring on the next sibling
        Sides &
      % align/colidx: left,2
    
    % rowcount: '0' | start: 'false' | colidx: '2'
    
        % Formatting a regular cell and recurring on the next sibling
        Name% make-rowspan-placeholders
    % rowspan info: col1 '0' | 'false' | '' || col2 '0' | 'false' | ''
     \tabularnewline\cline{1-1}\cline{2-2}
      %--------------------------------------------------------------------
    % align/colidx: left,1
    
    % rowcount: '0' | start: 'false' | colidx: '1'
    
        % Formatting a regular cell and recurring on the next sibling
        5 &
      % align/colidx: left,2
    
    % rowcount: '0' | start: 'false' | colidx: '2'
    
        % Formatting a regular cell and recurring on the next sibling
        pentagon% make-rowspan-placeholders
    % rowspan info: col1 '0' | 'false' | '' || col2 '0' | 'false' | ''
     \tabularnewline\cline{1-1}\cline{2-2}
      %--------------------------------------------------------------------
    % align/colidx: left,1
    
    % rowcount: '0' | start: 'false' | colidx: '1'
    
        % Formatting a regular cell and recurring on the next sibling
        6 &
      % align/colidx: left,2
    
    % rowcount: '0' | start: 'false' | colidx: '2'
    
        % Formatting a regular cell and recurring on the next sibling
        hexagon% make-rowspan-placeholders
    % rowspan info: col1 '0' | 'false' | '' || col2 '0' | 'false' | ''
     \tabularnewline\cline{1-1}\cline{2-2}
      %--------------------------------------------------------------------
    % align/colidx: left,1
    
    % rowcount: '0' | start: 'false' | colidx: '1'
    
        % Formatting a regular cell and recurring on the next sibling
        7 &
      % align/colidx: left,2
    
    % rowcount: '0' | start: 'false' | colidx: '2'
    
        % Formatting a regular cell and recurring on the next sibling
        heptagon% make-rowspan-placeholders
    % rowspan info: col1 '0' | 'false' | '' || col2 '0' | 'false' | ''
     \tabularnewline\cline{1-1}\cline{2-2}
      %--------------------------------------------------------------------
    % align/colidx: left,1
    
    % rowcount: '0' | start: 'false' | colidx: '1'
    
        % Formatting a regular cell and recurring on the next sibling
        8 &
      % align/colidx: left,2
    
    % rowcount: '0' | start: 'false' | colidx: '2'
    
        % Formatting a regular cell and recurring on the next sibling
        octagon% make-rowspan-placeholders
    % rowspan info: col1 '0' | 'false' | '' || col2 '0' | 'false' | ''
     \tabularnewline\cline{1-1}\cline{2-2}
      %--------------------------------------------------------------------
    % align/colidx: left,1
    
    % rowcount: '0' | start: 'false' | colidx: '1'
    
        % Formatting a regular cell and recurring on the next sibling
        10 &
      % align/colidx: left,2
    
    % rowcount: '0' | start: 'false' | colidx: '2'
    
        % Formatting a regular cell and recurring on the next sibling
        decagon% make-rowspan-placeholders
    % rowspan info: col1 '0' | 'false' | '' || col2 '0' | 'false' | ''
     \tabularnewline\cline{1-1}\cline{2-2}
      %--------------------------------------------------------------------
    % align/colidx: left,1
    
    % rowcount: '0' | start: 'false' | colidx: '1'
    
        % Formatting a regular cell and recurring on the next sibling
        15 &
      % align/colidx: left,2
    
    % rowcount: '0' | start: 'false' | colidx: '2'
    
        % Formatting a regular cell and recurring on the next sibling
        pentadecagon% make-rowspan-placeholders
    % rowspan info: col1 '0' | 'false' | '' || col2 '0' | 'false' | ''
     \tabularnewline\cline{1-1}\cline{2-2}
      %--------------------------------------------------------------------
    \end{xtabular}
      \end{center}
    \begin{center}{\small\bfseries Table 13.7}: Table of some polygons and their
number of sides.\end{center}
    %\end{table}
    
    \addtocounter{footnote}{-0}
    
          } % 
        }{% else
        % typeset the table in paragraph mode
        % ----- Begin capturing height of table
       
\settoheight{\mytableboxheight}{\begin{tabular*}{\mytablewidth}[t]{|p{
10\mystarwidth}|p{10\mystarwidth}|}\hline
    % count in rowspan-info-nodeset: 2
    % align/colidx: left,1
    
    % rowcount: '0' | start: 'false' | colidx: '1'
    
        % Formatting a regular cell and recurring on the next sibling
        Sides &
      % align/colidx: left,2
    
    % rowcount: '0' | start: 'false' | colidx: '2'
    
        % Formatting a regular cell and recurring on the next sibling
        Name% make-rowspan-placeholders
    % rowspan info: col1 '0' | 'false' | '' || col2 '0' | 'false' | ''
     \tabularnewline\cline{1-1}\cline{2-2}
      %--------------------------------------------------------------------
    % align/colidx: left,1
    
    % rowcount: '0' | start: 'false' | colidx: '1'
    
        % Formatting a regular cell and recurring on the next sibling
        5 &
      % align/colidx: left,2
    
    % rowcount: '0' | start: 'false' | colidx: '2'
    
        % Formatting a regular cell and recurring on the next sibling
        pentagon% make-rowspan-placeholders
    % rowspan info: col1 '0' | 'false' | '' || col2 '0' | 'false' | ''
     \tabularnewline\cline{1-1}\cline{2-2}
      %--------------------------------------------------------------------
    % align/colidx: left,1
    
    % rowcount: '0' | start: 'false' | colidx: '1'
    
        % Formatting a regular cell and recurring on the next sibling
        6 &
      % align/colidx: left,2
    
    % rowcount: '0' | start: 'false' | colidx: '2'
    
        % Formatting a regular cell and recurring on the next sibling
        hexagon% make-rowspan-placeholders
    % rowspan info: col1 '0' | 'false' | '' || col2 '0' | 'false' | ''
     \tabularnewline\cline{1-1}\cline{2-2}
      %--------------------------------------------------------------------
    % align/colidx: left,1
    
    % rowcount: '0' | start: 'false' | colidx: '1'
    
        % Formatting a regular cell and recurring on the next sibling
        7 &
      % align/colidx: left,2
    
    % rowcount: '0' | start: 'false' | colidx: '2'
    
        % Formatting a regular cell and recurring on the next sibling
        heptagon% make-rowspan-placeholders
    % rowspan info: col1 '0' | 'false' | '' || col2 '0' | 'false' | ''
     \tabularnewline\cline{1-1}\cline{2-2}
      %--------------------------------------------------------------------
    % align/colidx: left,1
    
    % rowcount: '0' | start: 'false' | colidx: '1'
    
        % Formatting a regular cell and recurring on the next sibling
        8 &
      % align/colidx: left,2
    
    % rowcount: '0' | start: 'false' | colidx: '2'
    
        % Formatting a regular cell and recurring on the next sibling
        octagon% make-rowspan-placeholders
    % rowspan info: col1 '0' | 'false' | '' || col2 '0' | 'false' | ''
     \tabularnewline\cline{1-1}\cline{2-2}
      %--------------------------------------------------------------------
    % align/colidx: left,1
    
    % rowcount: '0' | start: 'false' | colidx: '1'
    
        % Formatting a regular cell and recurring on the next sibling
        10 &
      % align/colidx: left,2
    
    % rowcount: '0' | start: 'false' | colidx: '2'
    
        % Formatting a regular cell and recurring on the next sibling
        decagon% make-rowspan-placeholders
    % rowspan info: col1 '0' | 'false' | '' || col2 '0' | 'false' | ''
     \tabularnewline\cline{1-1}\cline{2-2}
      %--------------------------------------------------------------------
    % align/colidx: left,1
    
    % rowcount: '0' | start: 'false' | colidx: '1'
    
        % Formatting a regular cell and recurring on the next sibling
        15 &
      % align/colidx: left,2
    
    % rowcount: '0' | start: 'false' | colidx: '2'
    
        % Formatting a regular cell and recurring on the next sibling
        pentadecagon% make-rowspan-placeholders
    % rowspan info: col1 '0' | 'false' | '' || col2 '0' | 'false' | ''
     \tabularnewline\cline{1-1}\cline{2-2}
      %--------------------------------------------------------------------
    \end{tabular*}} % end mytableboxheight set
       
\settodepth{\mytableboxdepth}{\begin{tabular*}{\mytablewidth}[t]{|p{
10\mystarwidth}|p{10\mystarwidth}|}\hline
    % count in rowspan-info-nodeset: 2
    % align/colidx: left,1
    
    % rowcount: '0' | start: 'false' | colidx: '1'
    
        % Formatting a regular cell and recurring on the next sibling
        Sides &
      % align/colidx: left,2
    
    % rowcount: '0' | start: 'false' | colidx: '2'
    
        % Formatting a regular cell and recurring on the next sibling
        Name% make-rowspan-placeholders
    % rowspan info: col1 '0' | 'false' | '' || col2 '0' | 'false' | ''
     \tabularnewline\cline{1-1}\cline{2-2}
      %--------------------------------------------------------------------
    % align/colidx: left,1
    
    % rowcount: '0' | start: 'false' | colidx: '1'
    
        % Formatting a regular cell and recurring on the next sibling
        5 &
      % align/colidx: left,2
    
    % rowcount: '0' | start: 'false' | colidx: '2'
    
        % Formatting a regular cell and recurring on the next sibling
        pentagon% make-rowspan-placeholders
    % rowspan info: col1 '0' | 'false' | '' || col2 '0' | 'false' | ''
     \tabularnewline\cline{1-1}\cline{2-2}
      %--------------------------------------------------------------------
    % align/colidx: left,1
    
    % rowcount: '0' | start: 'false' | colidx: '1'
    
        % Formatting a regular cell and recurring on the next sibling
        6 &
      % align/colidx: left,2
    
    % rowcount: '0' | start: 'false' | colidx: '2'
    
        % Formatting a regular cell and recurring on the next sibling
        hexagon% make-rowspan-placeholders
    % rowspan info: col1 '0' | 'false' | '' || col2 '0' | 'false' | ''
     \tabularnewline\cline{1-1}\cline{2-2}
      %--------------------------------------------------------------------
    % align/colidx: left,1
    
    % rowcount: '0' | start: 'false' | colidx: '1'
    
        % Formatting a regular cell and recurring on the next sibling
        7 &
      % align/colidx: left,2
    
    % rowcount: '0' | start: 'false' | colidx: '2'
    
        % Formatting a regular cell and recurring on the next sibling
        heptagon% make-rowspan-placeholders
    % rowspan info: col1 '0' | 'false' | '' || col2 '0' | 'false' | ''
     \tabularnewline\cline{1-1}\cline{2-2}
      %--------------------------------------------------------------------
    % align/colidx: left,1
    
    % rowcount: '0' | start: 'false' | colidx: '1'
    
        % Formatting a regular cell and recurring on the next sibling
        8 &
      % align/colidx: left,2
    
    % rowcount: '0' | start: 'false' | colidx: '2'
    
        % Formatting a regular cell and recurring on the next sibling
        octagon% make-rowspan-placeholders
    % rowspan info: col1 '0' | 'false' | '' || col2 '0' | 'false' | ''
     \tabularnewline\cline{1-1}\cline{2-2}
      %--------------------------------------------------------------------
    % align/colidx: left,1
    
    % rowcount: '0' | start: 'false' | colidx: '1'
    
        % Formatting a regular cell and recurring on the next sibling
        10 &
      % align/colidx: left,2
    
    % rowcount: '0' | start: 'false' | colidx: '2'
    
        % Formatting a regular cell and recurring on the next sibling
        decagon% make-rowspan-placeholders
    % rowspan info: col1 '0' | 'false' | '' || col2 '0' | 'false' | ''
     \tabularnewline\cline{1-1}\cline{2-2}
      %--------------------------------------------------------------------
    % align/colidx: left,1
    
    % rowcount: '0' | start: 'false' | colidx: '1'
    
        % Formatting a regular cell and recurring on the next sibling
        15 &
      % align/colidx: left,2
    
    % rowcount: '0' | start: 'false' | colidx: '2'
    
        % Formatting a regular cell and recurring on the next sibling
        pentadecagon% make-rowspan-placeholders
    % rowspan info: col1 '0' | 'false' | '' || col2 '0' | 'false' | ''
     \tabularnewline\cline{1-1}\cline{2-2}
      %--------------------------------------------------------------------
    \end{tabular*}} % end mytableboxdepth set
        \addtolength{\mytableboxheight}{\mytableboxdepth}
        % ----- End capturing height of table
        \typeout{textheight: \the\textheight}
        \typeout{mytableboxheight: \the\mytableboxheight}
        \typeout{table too wide, outputting in para mode}
        
    % \begin{table}[H]
    % \\ '' '0'
    
        \begin{center}
      
      \label{m38380*uid92}
      
    \noindent
    \tabletail{%
        \hline
        \multicolumn{2}{|p{\mytableroom}|}{\raggedleft \small \sl continued on
next page}\\
        \hline
      }
      \tablelasttail{}
     
\begin{xtabular*}{\mytablewidth}[t]{|p{10\mystarwidth}|p{10\mystarwidth}|}\hline
    % count in rowspan-info-nodeset: 2
    % align/colidx: left,1
    
    % rowcount: '0' | start: 'false' | colidx: '1'
    
        % Formatting a regular cell and recurring on the next sibling
        Sides &
      % align/colidx: left,2
    
    % rowcount: '0' | start: 'false' | colidx: '2'
    
        % Formatting a regular cell and recurring on the next sibling
        Name% make-rowspan-placeholders
    % rowspan info: col1 '0' | 'false' | '' || col2 '0' | 'false' | ''
     \tabularnewline\cline{1-1}\cline{2-2}
      %--------------------------------------------------------------------
    % align/colidx: left,1
    
    % rowcount: '0' | start: 'false' | colidx: '1'
    
        % Formatting a regular cell and recurring on the next sibling
        5 &
      % align/colidx: left,2
    
    % rowcount: '0' | start: 'false' | colidx: '2'
    
        % Formatting a regular cell and recurring on the next sibling
        pentagon% make-rowspan-placeholders
    % rowspan info: col1 '0' | 'false' | '' || col2 '0' | 'false' | ''
     \tabularnewline\cline{1-1}\cline{2-2}
      %--------------------------------------------------------------------
    % align/colidx: left,1
    
    % rowcount: '0' | start: 'false' | colidx: '1'
    
        % Formatting a regular cell and recurring on the next sibling
        6 &
      % align/colidx: left,2
    
    % rowcount: '0' | start: 'false' | colidx: '2'
    
        % Formatting a regular cell and recurring on the next sibling
        hexagon% make-rowspan-placeholders
    % rowspan info: col1 '0' | 'false' | '' || col2 '0' | 'false' | ''
     \tabularnewline\cline{1-1}\cline{2-2}
      %--------------------------------------------------------------------
    % align/colidx: left,1
    
    % rowcount: '0' | start: 'false' | colidx: '1'
    
        % Formatting a regular cell and recurring on the next sibling
        7 &
      % align/colidx: left,2
    
    % rowcount: '0' | start: 'false' | colidx: '2'
    
        % Formatting a regular cell and recurring on the next sibling
        heptagon% make-rowspan-placeholders
    % rowspan info: col1 '0' | 'false' | '' || col2 '0' | 'false' | ''
     \tabularnewline\cline{1-1}\cline{2-2}
      %--------------------------------------------------------------------
    % align/colidx: left,1
    
    % rowcount: '0' | start: 'false' | colidx: '1'
    
        % Formatting a regular cell and recurring on the next sibling
        8 &
      % align/colidx: left,2
    
    % rowcount: '0' | start: 'false' | colidx: '2'
    
        % Formatting a regular cell and recurring on the next sibling
        octagon% make-rowspan-placeholders
    % rowspan info: col1 '0' | 'false' | '' || col2 '0' | 'false' | ''
     \tabularnewline\cline{1-1}\cline{2-2}
      %--------------------------------------------------------------------
    % align/colidx: left,1
    
    % rowcount: '0' | start: 'false' | colidx: '1'
    
        % Formatting a regular cell and recurring on the next sibling
        10 &
      % align/colidx: left,2
    
    % rowcount: '0' | start: 'false' | colidx: '2'
    
        % Formatting a regular cell and recurring on the next sibling
        decagon% make-rowspan-placeholders
    % rowspan info: col1 '0' | 'false' | '' || col2 '0' | 'false' | ''
     \tabularnewline\cline{1-1}\cline{2-2}
      %--------------------------------------------------------------------
    % align/colidx: left,1
    
    % rowcount: '0' | start: 'false' | colidx: '1'
    
        % Formatting a regular cell and recurring on the next sibling
        15 &
      % align/colidx: left,2
    
    % rowcount: '0' | start: 'false' | colidx: '2'
    
        % Formatting a regular cell and recurring on the next sibling
        pentadecagon% make-rowspan-placeholders
    % rowspan info: col1 '0' | 'false' | '' || col2 '0' | 'false' | ''
     \tabularnewline\cline{1-1}\cline{2-2}
      %--------------------------------------------------------------------
    \end{xtabular*}
      \end{center}
    \begin{center}{\small\bfseries Table 13.7}: Table of some polygons and their
number of sides.\end{center}
    %\end{table}
    
    \addtocounter{footnote}{-0}
    
        }% ending lr/para test clause
      
    \par
  
        
    \setcounter{subfigure}{0}


	\begin{figure}[H] % horizontal\label{m38380*uid93}
    \begin{center}
    \rule[.1in]{\figurerulewidth}{.005in} \\
       
\label{m38380*uid93!!!underscore!!!media}\label{
m38380*uid93!!!underscore!!!printimage}\includegraphics[width=0.4\textwidth]{
col11306.imgs/m38380_MG10C13_046.png} % m38380;MG10C13\_046.png;;;6.0;8.5;
        
      \vspace{2pt}
    \vspace{\rubberspace}\par \begin{cnxcaption}
	  \small \textbf{Figure 13.47: }Examples of other polygons.
	\end{cnxcaption}
      
   % \vspace{.1in}
    \rule[.1in]{\figurerulewidth}{.005in} \\
        
    \end{center}

 \end{figure}   

    \addtocounter{footnote}{-0}
 \vspace{-1cm}   
      \label{m38380*eip-210}
        \subsubsection{ Angles of Regular Polygons}
        \nopagebreak
        \label{m38380*id319627}Polygons need not have all sides the same. When
they do, they are called regular polygons. You can calculate the size of the
interior angle of a regular polygon by using:\par 
          \label{m38380*uid96}\nopagebreak\noindent{}
            \settowidth{\mymathboxwidth}{\begin{equation}
    \hat{A}=\frac{n-2}{n}\ensuremath{\times}{180}^{\circ }\tag{13.5}
      \end{equation}
    }
    \typeout{Columnwidth = \the\columnwidth}\typeout{math as usual width =
\the\mymathboxwidth}
    \ifthenelse{\lengthtest{\mymathboxwidth < \columnwidth}}{% if the math fits,
do it again, for real
    \begin{equation}
    \hat{A}=\frac{n-2}{n}\ensuremath{\times}{180}^{\circ }\tag{13.5}
      \end{equation}
    }{% else, if it doesn't fit
    \setlength{\mymathboxwidth}{\columnwidth}
      \addtolength{\mymathboxwidth}{-48pt}
    \par\vspace{12pt}\noindent\begin{minipage}{\columnwidth}
    \parbox[t]{\mymathboxwidth}{\large\begin{math}
    \hat{A}=\frac{n-2}{n}\ensuremath{\times}{180}^{\circ }\end{math}}\hfill
    \parbox[t]{48pt}{\raggedleft 
    (13.5)}
    \end{minipage}\vspace{12pt}\par
    }% end of conditional for this bit of math
    \typeout{math as usual width = \the\mymathboxwidth}
    
          
          \label{m38380*id319678}where \begin{math}n\end{math} is the number of
sides and \begin{math}\hat{A}\end{math} is any angle.\vspace{\rubberspace}\par
\label{m38380*eip-804}\vspace{.5cm} 
      
      \noindent
      \hspace*{-30pt}\includegraphics[width=0.5in]{col11306.imgs/pspencil2.png} 
 \raisebox{25mm}{   
      \begin{mdframed}[linewidth=4, leftmargin=40, rightmargin=40]  
      \begin{exercise}
    \noindent\textbf{Exercise 13.4}\label{m38380*eip-708}
  \label{m38380*eip-618}
    Find the size of the interior angles of a regular octagon.
  \par 
\vspace{5pt}

\label{m38380*eip-937}\noindent\textbf{Solution to Exercise }
\label{m38380*id7432}\begin{enumerate}[noitemsep, label=\textbf{Step}
\textbf{\arabic*}. ] 
            \leftskip=20pt\rightskip=\leftskip\item 
An octagon has 8 sides.
\item 
\label{m38380*id7344}\nopagebreak\noindent{}
\settowidth{\mymathboxwidth}{\begin{equation}
    \begin{array}{ccc}\hfill \hat{A}& =&
\frac{n-2}{n}\ensuremath{\times}{180}^{\circ }\hfill \\ \hfill \hat{A}& =&
\frac{8-2}{8}\ensuremath{\times}{180}^{\circ }\hfill \\ \hfill \hat{A}& =&
\frac{6}{2}\ensuremath{\times}{180}^{\circ }\hfill \\ \hfill \hat{A}& =&
{135}^{\circ }\hfill \end{array}\tag{13.6}
      \end{equation}
    }
    \typeout{Columnwidth = \the\columnwidth}\typeout{math as usual width =
\the\mymathboxwidth}
    \ifthenelse{\lengthtest{\mymathboxwidth < \columnwidth}}{% if the math fits,
do it again, for real
    \begin{equation}
    \begin{array}{ccc}\hfill \hat{A}& =&
\frac{n-2}{n}\ensuremath{\times}{180}^{\circ }\hfill \\ \hfill \hat{A}& =&
\frac{8-2}{8}\ensuremath{\times}{180}^{\circ }\hfill \\ \hfill \hat{A}& =&
\frac{6}{2}\ensuremath{\times}{180}^{\circ }\hfill \\ \hfill \hat{A}& =&
{135}^{\circ }\hfill \end{array}\tag{13.6}
      \end{equation}
    }{% else, if it doesn't fit
    \setlength{\mymathboxwidth}{\columnwidth}
      \addtolength{\mymathboxwidth}{-48pt}
    \par\vspace{12pt}\noindent\begin{minipage}{\columnwidth}
    \parbox[t]{\mymathboxwidth}{\large\begin{math}
    \hat{A}=\frac{n-2}{n}\ensuremath{\times}{180}^{\circ
}\hat{A}=\frac{8-2}{8}\ensuremath{\times}{180}^{\circ
}\hat{A}=\frac{6}{2}\ensuremath{\times}{180}^{\circ }\hat{A}={135}^{\circ
}\end{math}}\hfill
    \parbox[t]{48pt}{\raggedleft 
    (13.6)}
    \end{minipage}\vspace{12pt}\par
    }% end of conditional for this bit of math
    \typeout{math as usual width = \the\mymathboxwidth}
    

\end{enumerate}
        


    \end{exercise}
    \end{mdframed}
    }
    \noindent
  
       
    \section{ Summary}
        \nopagebreak
        \label{m38380} $ \hspace{-5pt}\begin{array}{cccccccccccc}   \end{array}
$ \hspace{2 pt}\raisebox{-5
pt}{\includegraphics[width=0.5cm]{col11306.imgs/summary_www.png}} {(section
shortcode: P10119 )} \par \label{m38380*eip-439}\begin{itemize}[noitemsep]
            \item Make sure you know what the following terms mean:
quadrilaterals, vertices, sides, angles, parallel lines, perpendicular
lines,diagonals, bisectors and transversals.\item The properties of triangles
has been covered.\item Congruency and similarity of triangles\item Angles can be
classified as acute, right, obtuse, straight, reflex or revolution\item Theorem
of Pythagoras which is used to calculate the lengths of sides of a right-angled
triangle\item Angles: \label{m38380*id98732}\begin{itemize}[noitemsep]
            \item Acute angle: An angle \begin{math}{0}^{\circ }\end{math} and
\begin{math}{90}^{\circ }\end{math}\item Right angle: An angle measuring
\begin{math}{90}^{\circ }\end{math}\item Obtuse angle: An angle
\begin{math}{90}^{\circ }\end{math} and \begin{math}{180}^{\circ
}\end{math}\item Straight angle: An angle measuring \begin{math}{180}^{\circ
}\end{math}\item Reflex angle: An angle \begin{math}{180}^{\circ }\end{math} and
\begin{math}{360}^{\circ }\end{math}\item Revolution: An angle measuring
\begin{math}{360}^{\circ }\end{math}\end{itemize}
        \item There are several properties of angles and some special names for
these\item There are four types of triangles: Equilateral, isoceles,
right-angled and scalene\item The angles in a triangle add up to
\begin{math}{180}^{\circ }\end{math}\end{itemize}
\pagebreak
        \section{ Exercises}
        \nopagebreak
        \label{m38380} $ \hspace{-5pt}\begin{array}{cccccccccccc}   \end{array}
$ \hspace{2 pt}\raisebox{-5
pt}{\includegraphics[width=0.5cm]{col11306.imgs/summary_www.png}} {(section
shortcode: P10120 )} \par \label{m38380*id320135}\begin{enumerate}[noitemsep,
label=\textbf{\arabic*}. ] 
            \label{m38380*uid112}\item Find all the pairs of parallel lines in
the following figures, giving reasons in each case.

%\label{m38380*eip-78}\begin{enumerate}[noitemsep, label=\textbf{\alph*}. ] 
 %           \label{m38380*uid113}\item 
          
    \setcounter{subfigure}{0}


	\begin{figure}[H] % horizontal\label{m38380*id320164}
    \begin{center}
   
\label{m38380*id320164!!!underscore!!!media}\label{
m38380*id320164!!!underscore!!!printimage}\includegraphics[width=0.7\textwidth]{
col11306.imgs/m38380_MG10C13_054.png} % m38380;MG10C13\_054.png;;;6.0;8.5;
        
   %   \vspace{2pt}
   % \vspace{.1in}
    
    \end{center}

 \end{figure}   

    \addtocounter{footnote}{-0}
    
%        \label{m38380*uid114}\item 
          
 %   \setcounter{subfigure}{0}


%	\begin{figure}[H] % horizontal\label{m38380*id320183}
  %  \begin{center}
 %  
\label{m38380*id320183!!!underscore!!!media}\label{
m38380*id320183!!!underscore!!!printimage}\includegraphics[width=0.3\textwidth]{
col11306.imgs/m38380_MG10C13_055.png} % m38380;MG10C13\_055.png;;;6.0;8.5;
        
   %   \vspace{2pt}
   % \vspace{.1in}
    
   % \end{center}

 %\end{figure}   

  %  \addtocounter{footnote}{-0}
    
  %      \label{m38380*uid115}\item 
          
  %  \setcounter{subfigure}{0}


%	\begin{figure}[H] % horizontal\label{m38380*id320201}
%    \begin{center}
%   
\label{m38380*id320201!!!underscore!!!media}\label{
m38380*id320201!!!underscore!!!printimage}\includegraphics[width=0.3\textwidth]{
col11306.imgs/m38380_MG10C13_056.png} % m38380;MG10C13\_056.png;;;6.0;8.5;
        
  %    \vspace{2pt}
 %   \vspace{.1in}
    
  %  \end{center}

% \end{figure}   

 %   \addtocounter{footnote}{-0}
    
  %      \end{enumerate}
        
        \label{m38380*uid116}\item Find angles \begin{math}a\end{math},
\begin{math}b\end{math}, \begin{math}c\end{math} and \begin{math}d\end{math} in
each case, giving reasons.
%\label{m38380*id320255}\begin{enumerate}[noitemsep, label=\textbf{\alph*}. ] 
 %           \label{m38380*uid117}\item 
    \setcounter{subfigure}{0}


	\begin{figure}[H] % horizontal\label{m38380*id320271}
    \begin{center}
   
\label{m38380*id320271!!!underscore!!!media}\label{
m38380*id320271!!!underscore!!!printimage}\includegraphics[width=0.7\textwidth]{
col11306.imgs/m38380_MG10C13_057.png} % m38380;MG10C13\_057.png;;;6.0;8.5;
        
   %   \vspace{2pt}
   % \vspace{.1in}
    
    \end{center}

 \end{figure}   

    \addtocounter{footnote}{-0}
 %   \label{m38380*uid118}\item 
  %  \setcounter{subfigure}{0}


%	\begin{figure}[H] % horizontal\label{m38380*id320290}
%    \begin{center}
 %  
\label{m38380*id320290!!!underscore!!!media}\label{
m38380*id320290!!!underscore!!!printimage}\includegraphics[width=0.2\textwidth]{
col11306.imgs/m38380_MG10C13_058.png} % m38380;MG10C13\_058.png;;;6.0;8.5;
        
  %    \vspace{2pt}
  %  \vspace{.1in}
    
   % \end{center}

 %\end{figure}   

  %  \addtocounter{footnote}{-0}
  %  \label{m38380*uid119}\item 
  %  \setcounter{subfigure}{0}


%	\begin{figure}[H] % horizontal\label{m38380*id320310}
 %   \begin{center}
  % 
\label{m38380*id320310!!!underscore!!!media}\label{
m38380*id320310!!!underscore!!!printimage}\includegraphics[width=0.2\textwidth]{
col11306.imgs/m38380_MG10C13_059.png} % m38380;MG10C13\_059.png;;;6.0;8.5;
        
   %   \vspace{2pt}
  %  \vspace{.1in}
    
   % \end{center}

 %\end{figure}   

  %  \addtocounter{footnote}{-0}
  %  \end{enumerate}
        
        
\label{m38380*uid126}\item Say which of the following pairs of triangles are
congruent with reasons.
%\label{m38380*id320498}\begin{enumerate}[noitemsep, label=\textbf{\alph*}. ] 
 %           \label{m38380*uid127}\item 
    \setcounter{subfigure}{0}


	\begin{figure}[H] % horizontal\label{m38380*id320512}
    \begin{center}
   
\label{m38380*id320512!!!underscore!!!media}\label{
m38380*id320512!!!underscore!!!printimage}\includegraphics[width=0.85\textwidth]
{col11306.imgs/m38380_MG10C13_060.png} % m38380;MG10C13\_060.png;;;6.0;8.5;
        
      \vspace{2pt}
    \vspace{.1in}
    
    \end{center}

 \end{figure}   

    \addtocounter{footnote}{-0}
  %  \label{m38380*uid128}\item 
   % \setcounter{subfigure}{0}


	%\begin{figure}[H] % horizontal\label{m38380*id320530}
   % \begin{center}
   %
\label{m38380*id320530!!!underscore!!!media}\label{
m38380*id320530!!!underscore!!!printimage}\includegraphics[width=0.4\textwidth]{
col11306.imgs/m38380_MG10C13_061.png} % m38380;MG10C13\_061.png;;;6.0;8.5;
        
   %   \vspace{2pt}
   % \vspace{.1in}
    
   % \end{center}

 %\end{figure}   

  %  \addtocounter{footnote}{-0}
  %  \label{m38380*uid129}\item 
  %  \setcounter{subfigure}{0}


%	\begin{figure}[H] % horizontal\label{m38380*id320548}
%    \begin{center}
%   
\label{m38380*id320548!!!underscore!!!media}\label{
m38380*id320548!!!underscore!!!printimage}\includegraphics[width=0.4\textwidth]{
col11306.imgs/m38380_MG10C13_062.png} % m38380;MG10C13\_062.png;;;6.0;8.5;
        
%      \vspace{2pt}
%    \vspace{.1in}
    
 %   \end{center}

 %\end{figure}   

  %  \addtocounter{footnote}{-0}
  %  \label{m38380*uid130}\item 
  %  \setcounter{subfigure}{0}


%	\begin{figure}[H] % horizontal\label{m38380*id320565}
%    \begin{center}
 %  
\label{m38380*id320565!!!underscore!!!media}\label{
m38380*id320565!!!underscore!!!printimage}\includegraphics[width=0.4\textwidth]{
col11306.imgs/m38380_MG10C13_063.png} % m38380;MG10C13\_063.png;;;6.0;8.5;
        
 %    \vspace{2pt}
 %   \vspace{.1in}
    
  %  \end{center}

 %\end{figure}   

  %  \addtocounter{footnote}{-0}
  %  \end{enumerate}
        

        \label{m38380*uid131}\item Identify the types of angles shown below
(e.g. acute/obtuse etc):

          
    \setcounter{subfigure}{0}


	\begin{figure}[H] % horizontal\label{m38380*id401231}
    \begin{center}
   
\label{m38380*id401231!!!underscore!!!media}\label{
m38380*id401231!!!underscore!!!printimage}\includegraphics[width=0.6\textwidth]{
col11306.imgs/m38380_MG10C13_066.png} % m38380;MG10C13\_066.png;;;6.0;8.5;
        
  %    \vspace{2pt}
   
   % \vspace{.1in}
    
    \end{center}

 \end{figure}   

    \addtocounter{footnote}{-0}
    
        
\label{m38380*uid140}\item Calculate the size of the third angle (x) in each of
the diagrams below:

          
    \setcounter{subfigure}{0}


	\begin{figure}[H] % horizontal\label{m38380*id401232}
    \begin{center}
   
\label{m38380*id401232!!!underscore!!!media}\label{
m38380*id401232!!!underscore!!!printimage}\includegraphics[width=0.5\textwidth]{
col11306.imgs/m38380_MG10C13_067.png} % m38380;MG10C13\_067.png;;;6.0;8.5;
        
 %     \vspace{2pt}
  %  \vspace{.1in}
    
    \end{center}

 \end{figure}   

    \addtocounter{footnote}{-0}
 \pagebreak   
        
\label{m38380*uid141}\item Name each of the shapes/polygons, state how many
sides each has and whether it is regular (equiangular and equilateral) or not:

          
    \setcounter{subfigure}{0}


	\begin{figure}[H] % horizontal\label{m38380*id401233}
    \begin{center}
   
\label{m38380*id401233!!!underscore!!!media}\label{
m38380*id401233!!!underscore!!!printimage}\includegraphics[width=0.6\textwidth]{
col11306.imgs/m38380_MG10C13_068.png} % m38380;MG10C13\_068.png;;;6.0;8.5;
        
 %     \vspace{2pt}
  %  \vspace{.1in}
    
    \end{center}

 \end{figure}   

    \addtocounter{footnote}{-0}
    
        
\label{m38380*uid142}\item Assess whether the following statements are true or
false. If the statement is false, explain why:
\label{m38380*id401234}\begin{enumerate}[noitemsep, label=\textbf{\alph*}. ] 
            \item An angle is formed when two straight lines meet at a point.
\item The smallest angle that can be drawn is 5\ensuremath{{\,}^{\circ}}.\item
An angle of 90\ensuremath{{\,}^{\circ}} is called a square angle.\item Two
angles whose sum is 180\ensuremath{{\,}^{\circ}} are called supplementary
angles.\item Two parallel lines will never intersect.\item A regular polygon has
equal angles but not equal sides.\item An isoceles triangle has three equal
sides.\item If three sides of a triangle are equal in length to the same sides
of another triangle, then the two triangles are incongruent.\item If three pairs
of corresponding angles in two triangles are equal, then the triangles are
similar.\end{enumerate}
        
        
\label{m38380*uid143}\item Name the type of angle (e.g. acute/obtuse etc) based
on it's size:
\label{m38380*id401235}\begin{enumerate}[noitemsep, label=\textbf{\alph*}. ] 
            \item  30\ensuremath{{\,}^{\circ}}\item 
47\ensuremath{{\,}^{\circ}}\item  90\ensuremath{{\,}^{\circ}}\item 
91\ensuremath{{\,}^{\circ}}\item  191\ensuremath{{\,}^{\circ}}\item 
360\ensuremath{{\,}^{\circ}}\item  180\ensuremath{{\,}^{\circ}}\end{enumerate}
        
 \pagebreak       
\label{m38380*uid144}\item Using Pythagoras' theorem for right-angled triangles,
calculate the length of x:

          
    \setcounter{subfigure}{0}


	\begin{figure}[H] % horizontal\label{m38380*id401236}
    \begin{center}
   
\label{m38380*id401236!!!underscore!!!media}\label{
m38380*id401236!!!underscore!!!printimage}\includegraphics[width=0.8\textwidth]{
col11306.imgs/m38380_MG10C13_070.png} % m38380;MG10C13\_070.png;;;6.0;8.5;
        
      \vspace{2pt}
    \vspace{.1in}
    
    \end{center}

 \end{figure}   

    \addtocounter{footnote}{-0}
    
        
\end{enumerate}
        
      \label{m38380*uid132}
\par \raisebox{-5
pt}{\includegraphics[width=0.5cm]{col11306.imgs/summary_www.png}} Find the
answers with the shortcodes:
 \par \begin{tabular}[h]{cccccc}
 (1.) lxh  &  (2.) laq  &  (3.) lai  &  (4.) lTb  &  (5.) lTj  &  (6.) lTD  & 
(7.) lTZ  &  (8.) lTB  &  (9.) lTK  & \end{tabular}

\pagebreak

        \subsection{ Challenge Problem}
        \nopagebreak
        
        
        \label{m38380*id320611}\begin{enumerate}[noitemsep,
label=\textbf{\arabic*}. ] 
            \label{m38380*uid133}\item Using the figure below, show that the sum
of the three angles in a triangle is 180\begin{math}{}^{\circ }\end{math}. Line
\begin{math}DE\end{math}\hspace{1ex} is parallel to \begin{math}BC\end{math}.

    \setcounter{subfigure}{0}


	\begin{figure}[H] % horizontal\label{m38380*id320668}
    \begin{center}
   
\label{m38380*id320668!!!underscore!!!media}\label{
m38380*id320668!!!underscore!!!printimage}\includegraphics[width=0.7\textwidth]{
col11306.imgs/m38380_MG10C13_065.png} % m38380;MG10C13\_065.png;;;6.0;8.5;
        
      \vspace{2pt}
    \vspace{.1in}
    
    \end{center}

 \end{figure}   

    \addtocounter{footnote}{-0}
    \newline
            \end{enumerate}
        
      
    
  \label{m38380**end}
    
\par \raisebox{-5
pt}{\includegraphics[width=0.5cm]{col11306.imgs/summary_www.png}} Find the
answers with the shortcodes:
 \par \begin{tabular}[h]{cccccc}
 (1.) laO  & \end{tabular}








\begin{tabular}{|l|l|}\hline
Sides &
Name% make-rowspan-placeholders
\\ \hline
%--------------------------------------------------------------------
5 &
pentagon% make-rowspan-placeholders
\\ \hline
6 &
hexagon%
\\ \hline
7 &
heptagon% make-rowspan-placeholders
\\ \hline
8 &
octagon% make-rowspan-placeholders
\\ \hline
10 &
decagon% make-rowspan-placeholders
\\ \hline
15 &
pentadecagon% make-rowspan-placeholders
\\ \hline
\end{tabular}
\end{center}
% \begin{center}{\small\bfseries Table 12.7}: Table of some polygons and their
number of sides.\end{center}
% \begin{caption}{\small\bfseries Table 12.7}: Table of some polygons and their
number of sides.\end{caption}
\end{table}
\par
\setcounter{subfigure}{0}
\begin{figure}[H] % horizontal\label{m39368*uid93}
\begin{center}
\rule[.1in]{\figurerulewidth}{.005in} \\
\label{m39368*uid93!!!underscore!!!media}\label{
m39368*uid93!!!underscore!!!printimage}\includegraphics{
col11306.imgs/m39368_MG10C13_046.png} % m39368;MG10C13\_046.png;;;6.0;8.5;
\vspace{2pt}
\vspace{\rubberspace}\par \begin{cnxcaption}
\small \textbf{Figure 12.47: }Examples of other polygons.
\end{cnxcaption}
\vspace{.1in}
\rule[.1in]{\figurerulewidth}{.005in} \\
\end{center}
\end{figure}       

\subsubsection{ Angles of Regular Polygons}
\nopagebreak
Polygons need not have all sides the same. When they do, they are called regular
polygons. You can calculate the size of the interior angle of a regular polygon
by using:\par 


\begin{equation*}
\hat{A}=\frac{n-2}{n}\ensuremath{\times}{180}^{\circ }
\end{equation*}
where $n$ is the number of sides and $\hat{A}$ is any angle.
\begin{wex}{}
{

Find the size of the interior angles of a regular octagon.}
{
\westep{}
An octagon has 8 sides.
\westep{}

\begin{equation*}
\begin{array}{ccc}\hfill \hat{A}& =&
\frac{n-2}{n}\ensuremath{\times}{180}^{\circ }\hfill \\
 \hfill \hat{A}& =& \frac{8-2}{8}\ensuremath{\times}{180}^{\circ }\hfill \\
 \hfill \hat{A}& =& \frac{6}{2}\ensuremath{\times}{180}^{\circ }\hfill \\
 \hfill \hat{A}& =& {135}^{\circ }\hfill 
\end{array}
\end{equation*}
}
\end{wex}



\subsection{ Quadrilaterals}
\nopagebreak
In this section we will look at the properties of some special quadrilaterals.
We will then use these properties to solve geometrical problems. It should be
noted that although all the properties of a figure are given, we only need one
unique property of the quadrilateral to prove that it is that quadrilateral. For
example, if we have a quadrilateral with two pairs of opposite sides parallel,
then that quadrilateral is a parallelogram. We can then prove the other
properties of the quadrilateral using what we have learnt about parallel lines
and triangles.\par 

\subsubsection{ Trapezium}
A trapezium is a quadrilateral with one pair of parallel opposite sides. It may
also be called a trapezoid. A special type of trapezium is the isosceles
trapezium, where one pair of opposite sides is parallel, the other pair of sides
is equal in length and the angles at the ends of each parallel side are equal.
An isosceles trapezium has one line of symmetry and its diagonals are equal in
length.\par 
Note: The term trapezoid is predominantly used in North America and refers to
what we call a trapezium. Rather confusingly, they use the term 'trapezium' to
refer to a general irregular quadrilateral, that is a quadrilateral with no
parallel sides!\par 
\setcounter{subfigure}{0}
\begin{figure}[H] % horizontal\label{m39354*uid55}
\begin{center}
\rule[.1in]{\figurerulewidth}{.005in} \\
\label{m39354*uid55!!!underscore!!!media}\label{
m39354*uid55!!!underscore!!!printimage}\includegraphics{
col11306.imgs/m39354_MG10C13_040.png} % m39354;MG10C13\_040.png;;;6.0;8.5;
\vspace{2pt}
\vspace{\rubberspace}\par \begin{cnxcaption}
\small \textbf{Figure 13.1: }Examples of trapeziums.
\end{cnxcaption}
\vspace{.1in}
\rule[.1in]{\figurerulewidth}{.005in} \\
\end{center}
\end{figure}       

\subsubsection{ Parallelogram}
A trapezium with both sets of opposite sides parallel is called a parallelogram.
A summary of the properties of a parallelogram is:\par 
\begin{itemize}[noitemsep]
\item Both pairs of opposite sides are parallel.
\item Both pairs of opposite sides are equal in length.
\item Both pairs of opposite angles are equal.
\item Both diagonals bisect each other (i.e. they cut each other in half).
\end{itemize}
\setcounter{subfigure}{0}
\begin{figure}[H] % horizontal\label{m39354*uid61}
\begin{center}
\rule[.1in]{\figurerulewidth}{.005in} \\
\label{m39354*uid61!!!underscore!!!media}\label{
m39354*uid61!!!underscore!!!printimage}\includegraphics{
col11306.imgs/m39354_MG10C13_041.png} % m39354;MG10C13\_041.png;;;6.0;8.5;
\vspace{2pt}
\vspace{\rubberspace}\par \begin{cnxcaption}
\small \textbf{Figure 13.2: }An example of a parallelogram.
\end{cnxcaption}
\vspace{.1in}
\rule[.1in]{\figurerulewidth}{.005in} \\
\end{center}
\end{figure}       

\subsubsection{ Rectangle}
A rectangle is a parallelogram that has all four angles equal to ${90}^{\circ
}$. A summary of the properties of a rectangle is:\par 
\begin{itemize}[noitemsep]
\item Both pairs of opposite sides are parallel.
\item Both pairs of opposite sides are of equal length.
\item Both diagonals bisect each other.
\item Diagonals are equal in length.
\item All angles at the corners are right angles.
\end{itemize}
\setcounter{subfigure}{0}
\begin{figure}[H] % horizontal\label{m39354*uid68}
\begin{center}
\rule[.1in]{\figurerulewidth}{.005in} \\
\label{m39354*uid68!!!underscore!!!media}\label{
m39354*uid68!!!underscore!!!printimage}\includegraphics{
col11306.imgs/m39354_MG10C13_042.png} % m39354;MG10C13\_042.png;;;6.0;8.5;
\vspace{2pt}
\vspace{\rubberspace}\par \begin{cnxcaption}
\small \textbf{Figure 13.3: }Example of a rectangle.
\end{cnxcaption}
\vspace{.1in}
\rule[.1in]{\figurerulewidth}{.005in} \\
\end{center}
\end{figure}       

\subsubsection{ Rhombus}
A rhombus is a parallelogram that has all four sides of equal length. A summary
of the properties of a rhombus is:\par 
\begin{itemize}[noitemsep]
\item Both pairs of opposite sides are parallel.
\item All sides are equal in length.
\item Both pairs of opposite angles are equal.
\item Both diagonals bisect each other at ${90}^{\circ }$.
\item Diagonals of a rhombus bisect both pairs of opposite angles.
\end{itemize}
\setcounter{subfigure}{0}
\begin{figure}[H] % horizontal\label{m39354*uid75}
\begin{center}
\rule[.1in]{\figurerulewidth}{.005in} \\
\label{m39354*uid75!!!underscore!!!media}\label{
m39354*uid75!!!underscore!!!printimage}\includegraphics{
col11306.imgs/m39354_MG10C13_043.png} % m39354;MG10C13\_043.png;;;6.0;8.5;
\vspace{2pt}
\vspace{\rubberspace}\par \begin{cnxcaption}
\small \textbf{Figure 13.4: }An example of a rhombus. A rhombus is a
parallelogram with all sides equal.
\end{cnxcaption}
\vspace{.1in}
\rule[.1in]{\figurerulewidth}{.005in} \\
\end{center}
\end{figure}       

\subsubsection{ Square}
A square is a rhombus that has all four angles equal to 90$^{\circ }$.\par 
A summary of the properties of a square is:\par 
\begin{itemize}[noitemsep]
\item Both pairs of opposite sides are parallel.
\item All sides are equal in length.
\item All angles are equal to ${90}^{\circ }$.
\item Both pairs of opposite angles are equal.
\item Both diagonals bisect each other at ${90}^{\circ }$.
\item Diagonals are equal in length.
\item Diagonals bisect both pairs of opposite angles (ie. all ${45}^{\circ }$).
\end{itemize}
\setcounter{subfigure}{0}
\begin{figure}[H] % horizontal\label{m39354*uid84}
\begin{center}
\rule[.1in]{\figurerulewidth}{.005in} \\
\label{m39354*uid84!!!underscore!!!media}\label{
m39354*uid84!!!underscore!!!printright
prismsimage}\includegraphics{col11306.imgs/m39354_MG10C13_044.png} %
m39354;MG10C13\_044.png;;;6.0;8.5;
\vspace{2pt}
\vspace{\rubberspace}\par \begin{cnxcaption}
\small \textbf{Figure 13.5: }An example of a square. A square is a rhombus with
all angles equal to 90$^{\circ }$.
\end{cnxcaption}
\vspace{.1in}
\rule[.1in]{\figurerulewidth}{.005in} \\
\end{center}
\end{figure}       

\subsubsection{ Kite}
A kite is a quadrilateral with two pairs of adjacent sides equal.\par 
A summary of the properties of a kite is:\par 
\begin{itemize}[noitemsep]
\item Two pairs of adjacent sides are equal in length.
\item One pair of opposite angles are equal where the angles are between unequal
sides.
\item One diagonal bisects the other diagonal and one diagonal bisects one pair
of opposite angles.
\item Diagonals intersect at right-angles.
\end{itemize}
\setcounter{subfigure}{0}
\begin{figure}[H] % horizontal\label{m39354*uid90}
\begin{center}
\rule[.1in]{\figurerulewidth}{.005in} \\
\label{m39354*uid90!!!underscore!!!media}\label{
m39354*uid90!!!underscore!!!printimage}\includegraphics{
col11306.imgs/m39354_MG10C13_045.png} % m39354;MG10C13\_045.png;;;6.0;8.5;
\vspace{2pt}
\vspace{\rubberspace}\par \begin{cnxcaption}
\small \textbf{Figure 13.6: }An example of a kite.
\end{cnxcaption}
\vspace{.1in}
\rule[.1in]{\figurerulewidth}{.005in} \\
\end{center}
\end{figure}       
Rectangles are a special case (or a subset) of parallelograms. Rectangles are
parallelograms that have all angles equal to 90. Squares are a special case (or
subset) of rectangles. Squares are rectangles that have all sides equal in
length. So all squares are parallelograms and rectangles. So if you are asked to
prove that a quadrilateral is a parallelogram, it is enough to show that both
pairs of opposite sides are parallel. But if you are asked to prove that a
quadrilateral is a square, then you must also show that the angles are all right
angles and the sides are equal in length.
\par 

%     

\subsection{ Proofs and conjectures in geometry}

You have seen how to use geometry and the properties of polygons to help you
find unknown lengths and angles in various quadrilaterals and polygons. We will
now extend this work to proving some of the properties and to solving riders. A
conjecture is the mathematicians way of saying I believe that this is true, but
I have no proof. The following worked examples will help make this clearer. 
\par 

\begin{wex}
{Proofs - 1}
{Given quadrilateral ABCD, with $AB\parallel CD$ and $AD\parallel BC$, prove
that $B\hat{A}D=B\hat{C}A$ and $A\hat{B}C=A\hat{D}C$.}
{
\westep{ We draw the following diagram and construct the diagonals.}

\setcounter{subfigure}{0}
\begin{figure}[H] % horizontal\label{m39352*uid310}
\begin{center}\label{m39352*uid310!!!underscore!!!printimage}\includegraphics{
col11306.imgs/m39352_geomproof1.png} % ;geomproof1.png;;;6.0;8.5;
\vspace{2pt}
\vspace{.1in}
\end{center}
\end{figure}   
\westep{}    
Given: $AB\parallel CD$ and $AD\parallel BC$. We need to prove $A=C$ and $B=D$.
In the formal language of maths we say that we are required to prove (RTP)
$B\hat{A}D=B\hat{C}A$ and $A\hat{B}C=A\hat{D}C$. 

\westep{}
\begin{equation*}
\begin{array}{cccc}\hfill B\hat{A}C& =& A\hat{C}D\hfill &
(\mathrm{corresponding\; angles})\\ \hfill D\hat{A}C& =& B\hat{C}A\hfill &
(\mathrm{corresponding\; angles})\\ \hfill B\hat{A}D& =& B\hat{C}A\hfill &
\end{array}
\end{equation*}
Similarly we find that: 
\begin{equation*}
A\hat{B}C=A\hat{D}C}
\end{equation*}
}
\end{wex}

 
\begin{wex}{Proofs - 2}{
In parallelogram ABCD, the bisectors of the angles (AW, BX, CY and DZ) have been
constructed:
\setcounter{subfigure}{0}
\begin{figure}[H] % horizontal\label{m39352*uid4140}
\begin{center}
\label{m39352*uid4140!!!underscore!!!media}\label{
m39352*uid4140!!!underscore!!!printimage}\includegraphics[width=300px]{
col11306.imgs/m39352_geomproof2.png} % m39352;geomproof2.png;;;6.0;8.5;
\vspace{2pt}
\vspace{.1in}
\end{center}
\end{figure}       
You are also given that $\mathrm{AB}=\mathrm{CD}$, $\mathrm{AD}=\mathrm{BC}$,
$\mathrm{AB}\parallel \mathrm{CD}$, $\mathrm{AD}\parallel \mathrm{BC}$, 
$\hat{A}=\hat{C}$, and $\hat{B}=\hat{D}$. 
Prove that MNOP is a parallelogram.} 
{

\westep{} 
Given: $\mathrm{AB}=\mathrm{CD}$, $\mathrm{AD}=\mathrm{BC}$,
$\mathrm{AB}\parallel \mathrm{CD}$, $\mathrm{AD}\parallel \mathrm{BC}$, 
$\hat{A}=\hat{C}$, and $\hat{B}=\hat{D}$. RTP: MNOP is a parallelogram.

\westep{} 
\begin{equation*}
\begin{array}{cc}\hfill
In\phantom{\rule{2pt}{0ex}}▵\phantom{\rule{2pt}{0ex}}\mathrm{ADW}\phantom{\rule{
2pt}{0ex}}\mathrm{and}\phantom{\rule{2pt}{0ex}}▵\mathrm{CBY}\phantom{\rule{2pt}{
0ex}}\\ \hfill D\hat{A}W& =&
B\hat{C}Y\phantom{\rule{2pt}{0ex}}(\mathrm{given})\hfill \\ \hfill A\hat{D}C& =&
A\hat{B}C\phantom{\rule{2pt}{0ex}}(\mathrm{given})\hfill \\ \hfill \mathrm{AD}&
=& \mathrm{BC}\phantom{\rule{2pt}{0ex}}\mathrm{(given)}\hfill \\ \hfill
\therefore \phantom{\rule{2pt}{0ex}}▵\mathrm{ADW}& =&
▵\mathrm{CBY}\phantom{\rule{2pt}{0ex}}\mathrm{(AAS)}\hfill \\ \hfill \therefore
\phantom{\rule{2pt}{0ex}}\mathrm{DW}& =& \mathrm{BY}\hfill
\end{array}\end{equation*}

\begin{equation*}
\begin{array}{cc}\hfill
In\phantom{\rule{2pt}{0ex}}▵\phantom{\rule{2pt}{0ex}}\mathrm{ABX}\phantom{\rule{
2pt}{0ex}}\mathrm{and}\phantom{\rule{2pt}{0ex}}▵\mathrm{CDZ}\phantom{\rule{2pt}{
0ex}}\\ \hfill D\hat{C}Z& =&
B\hat{A}X\phantom{\rule{2pt}{0ex}}(\mathrm{given})\hfill \\ \hfill Z\hat{D}C& =&
X\hat{B}A\phantom{\rule{2pt}{0ex}}(\mathrm{given})\hfill \\ \hfill \mathrm{DC}&
=& \mathrm{AB}\phantom{\rule{2pt}{0ex}}\mathrm{(given)}\hfill \\ \hfill
\therefore \phantom{\rule{2pt}{0ex}}▵\mathrm{ABX}& \equiv &
▵\mathrm{CDZ}\phantom{\rule{2pt}{0ex}}\mathrm{(AAS)}\hfill \\ \hfill \therefore
\phantom{\rule{2pt}{0ex}}\mathrm{AX}& =& \mathrm{CZ}\hfill \end{array}
\end{equation*}

\begin{equation*}
\begin{array}{cc}\hfill
In\phantom{\rule{2pt}{0ex}}▵\phantom{\rule{2pt}{0ex}}\mathrm{XAM}\phantom{\rule{
2pt}{0ex}}\mathrm{and}\phantom{\rule{2pt}{0ex}}▵\mathrm{ZCO}\phantom{\rule{2pt}{
0ex}}\\ \hfill X\hat{A}M& =&
Z\hat{C}O\phantom{\rule{2pt}{0ex}}(\mathrm{given})\hfill \\ \hfill A\hat{X}M& =&
C\hat{Z}O\phantom{\rule{2pt}{0ex}}(\mathrm{proven\; above})\hfill \\ \hfill
\mathrm{AX}& =& \mathrm{CZ}\phantom{\rule{2pt}{0ex}}\mathrm{(proven\;
above)}\hfill \\ \hfill \therefore \phantom{\rule{2pt}{0ex}}▵\mathrm{XAM}&
\equiv & ▵\mathrm{COZ}\phantom{\rule{2pt}{0ex}}\mathrm{(AAS)}\hfill \\ \hfill
\therefore \phantom{\rule{2pt}{0ex}}A\hat{O}C& =& A\hat{M}X\hfill \end{array}
\end{equation*}

\begin{equation*}
\begin{array}{ccc}\hfill A\hat{M}X& =&
P\hat{M}N\phantom{\rule{2pt}{0ex}}\mathrm{(vert.\; opp.\; \angle
\text{'}s)}\hfill \\ \hfill C\hat{O}Z& =&
N\hat{O}P\phantom{\rule{2pt}{0ex}}\mathrm{(vert.\; opp.\; \angle
\text{'}s)}\hfill \\ \hfill \therefore \phantom{\rule{2pt}{0ex}}P\hat{M}N& =&
N\hat{O}P\hfill \end{array}
\end{equation*}

\begin{equation*}
\begin{array}{cc}\hfill
In\phantom{\rule{2pt}{0ex}}▵\phantom{\rule{2pt}{0ex}}\mathrm{BYN}\phantom{\rule{
2pt}{0ex}}\mathrm{and}\phantom{\rule{2pt}{0ex}}▵\mathrm{DWP}\phantom{\rule{2pt}{
0ex}}\\ \hfill Y\hat{B}N& =&
W\hat{D}P\phantom{\rule{2pt}{0ex}}(\mathrm{given})\hfill \\ \hfill B\hat{Y}N& =&
W\hat{D}P\phantom{\rule{2pt}{0ex}}(\mathrm{proven\; above})\hfill \\ \hfill
\mathrm{DW}& =& \mathrm{BY}\phantom{\rule{2pt}{0ex}}\mathrm{(proven\;
above)}\hfill \\ \hfill \therefore \phantom{\rule{2pt}{0ex}}▵\mathrm{YBN}&
\equiv & ▵\mathrm{WDP}\phantom{\rule{2pt}{0ex}}\mathrm{(AAS)}\hfill \\ \hfill
\therefore \phantom{\rule{2pt}{0ex}}B\hat{N}Y& =& D\hat{P}W\hfill \end{array}
\end{equation*}

\begin{equation*}
\begin{array}{ccc}\hfill D\hat{P}W& =&
M\hat{P}O\phantom{\rule{2pt}{0ex}}\mathrm{(vert.\; opp.\; \angle
\text{'}s)}\hfill \\ \hfill B\hat{N}Y& =&
O\hat{N}M\phantom{\rule{2pt}{0ex}}\mathrm{(vert.\; opp.\; \angle
\text{'}s)}\hfill \\ \hfill \therefore \phantom{\rule{2pt}{0ex}}M\hat{P}O& =&
O\hat{N}M\hfill \end{array}
\end{equation*}
$\therefore $ MNOP is a parallelogram (both pairs opp. $\angle $'s $=$, and
therefore both pairs opp. sides parallel too)

\end{wex}



\Note{It is very important to note that a single counter example disproves a
conjecture. Also numerous specific supporting examples do not prove a
conjecture. }


\subsection{ Summary}
\begin{itemize}[noitemsep]
\item The properties of kites, rhombuses, parallelograms, squares, rectangles
and trapeziums was covered. These figures are all known as quadrilaterals\item
You should know the formulae for surface area of rectangular and triangular
prisms as well as cylinders\item 
The volume of a right prism is calculated by multiplying the area of the base by
the height. So, for a square prism of side length $a$ and height $h$ the volume
is $a\ensuremath{\times}a\ensuremath{\times}h={a}^{2}h$.\item 
Two polygons are similar if:\begin{itemize}[noitemsep]
\item their corresponding angles are equal\item the ratios of corresponding
sides are equal\end{itemize}
. All squares are similar\end{itemize}

\subsection{ End of Chapter Exercises}
\nopagebreak
\begin{enumerate}[noitemsep, label=\textbf{\arabic*}. ] 
\item Assess whether the following statements are true or false. If the
statement is false, explain why:
\begin{enumerate}[noitemsep, label=\textbf{\alph*}. ] 
\item  A trapezium is a quadrilateral with two pairs of parallel opposite
sides.\item  Both diagonals of a parallelogram bisect each other.\item  A
rectangle is a parallelogram that has all four corner angles equal to
60\ensuremath{{\,}^{\circ}}.\item  The four sides of a rhombus have different
lengths.\item  The diagonals of a kite intersect at right angles.\item  Two
polygons are similar if only their corresponding angles are
equal.\end{enumerate}
\item Calculate the area of each of the following shapes:
\setcounter{subfigure}{0}
\begin{figure}[H] % horizontal\label{m39358*id320600}
\begin{center}
\label{m39358*id320600!!!underscore!!!media}\label{
m39358*id320600!!!underscore!!!printimage}\includegraphics[height=300px]{
col11306.imgs/m39358_MG10C14_041.png} % m39358;MG10C14\_041.png;;;6.0;8.5;
\vspace{2pt}
\vspace{.1in}
\end{center}
\end{figure}              \item Calculate the surface area and volume of each of
the following objects (assume that all faces/surfaces are solid -- e.g. surface
area of cylinder will include circular areas at top and bottom):
\setcounter{subfigure}{0}
\begin{figure}[H] % horizontal\label{m39358*id320601}
\begin{center}
\label{m39358*id320601!!!underscore!!!media}\label{
m39358*id320601!!!underscore!!!printimage}\includegraphics[width=300px]{
col11306.imgs/m39358_MG10C14_042.png} % m39358;MG10C14\_042.png;;;6.0;8.5;
\vspace{2pt}
\vspace{.1in}
\end{center}
\end{figure}              \item Calculate the surface area and volume of each of
the following objects (assume that all faces/surfaces are solid):
\setcounter{subfigure}{0}
\begin{figure}[H] % horizontal\label{m39358*id320602}
\begin{center}
\label{m39358*id320602!!!underscore!!!media}\label{
m39358*id320602!!!underscore!!!printimage}\includegraphics[height=300px]{
col11306.imgs/m39358_MG10C14_043.png} % m39358;MG10C14\_043.png;;;6.0;8.5;
\vspace{2pt}
\vspace{.1in}
\end{center}
\end{figure}               \item Using the rules given, identify the type of
transformation and draw the image of the shapes.
\label{m39358*id73963}\begin{enumerate}[noitemsep, label=\textbf{\alph*}. ] 
\label{m39358*uid126}\item  (x;y)$\to $(x+3;y-3)
\setcounter{subfigure}{0}
\begin{figure}[H] % horizontal\label{m39358*id73991}
\begin{center}
\label{m39358*id73991!!!underscore!!!media}\label{
m39358*id73991!!!underscore!!!printimage}\includegraphics[width=300px]{
col11306.imgs/m39358_MG10C14_037.png} % m39358;MG10C14\_037.png;;;6.0;8.5;
\vspace{2pt}
\vspace{.1in}
\end{center}
\end{figure}     \item  (x;y)$\to $(x-4;y)
\setcounter{subfigure}{0}
\begin{figure}[H] % horizontal\label{m39358*id74022}
\begin{center}
\label{m39358*id74022!!!underscore!!!media}\label{
m39358*id74022!!!underscore!!!printimage}\includegraphics[width=300px]{
col11306.imgs/m39358_MG10C14_038.png} % m39358;MG10C14\_038.png;;;6.0;8.5;
\vspace{2pt}
\vspace{.1in}
\end{center}
\end{figure}       \item (x;y)$\to $(y;x)
\setcounter{subfigure}{0}
\begin{figure}[H] % horizontal\label{m39358*id74053}
\begin{center}
\label{m39358*id74053!!!underscore!!!media}\label{
m39358*id74053!!!underscore!!!printimage}\includegraphics[width=300px]{
col11306.imgs/m39358_MG10C14_039.png} % m39358;MG10C14\_039.png;;;6.0;8.5;
\vspace{2pt}
\vspace{.1in}
\end{center}
\end{figure}      \item (x;y)$\to $(-x;-y)
\setcounter{subfigure}{0}
\begin{figure}[H] % horizontal\label{m39358*id74084}
\begin{center}
\label{m39358*id74084!!!underscore!!!media}\label{
m39358*id74084!!!underscore!!!printimage}\includegraphics[width=300px]{
col11306.imgs/m39358_MG10C14_040.png} % m39358;MG10C14\_040.png;;;6.0;8.5;
\vspace{2pt}
\vspace{.1in}
\end{center}
\end{figure}       \end{enumerate}
  \item  
PQRS is a quadrilateral with points P(0; −3) ; Q(−2;5) ; R(3;2) and S(3;--2)  in
the Cartesian plane.
\begin{enumerate}[noitemsep, label=\textbf{\alph*}. ] 
\item Find the length of QR.\item Find the gradient of PS.\item Find the
midpoint of PR.\item Is PQRS a parallelogram?  Give reasons for your answer.
\end{enumerate}
  \item A(--2;3) and B(2;6) are points in the Cartesian plane.  C(a;b) is the
midpoint of AB. Find the values of a and b.\newline
\item 
Consider: Triangle ABC with vertices A (1; 3) B (4; 1) and C (6; 4):
\begin{enumerate}[noitemsep, label=\textbf{\alph*}. ] 
\item Sketch triangle ABC on the Cartesian plane. \item Show that ABC is an
isoceles triangle.\item Determine the co-ordinates of M, the midpoint of
AC.\item Determine the gradient of AB.\item Show that the following points are
collinear: A, B and D(7;-1)\end{enumerate}
\item In the diagram, A is the point (-6;1) and B is the point (0;3)
\setcounter{subfigure}{0}
\begin{figure}[H] % horizontal\label{m39358*id740344}
\begin{center}
\label{m39358*id740344!!!underscore!!!media}\label{
m39358*id740344!!!underscore!!!printimage}\includegraphics[width=.7\columnwidth]
{col11306.imgs/m39358_MG10C14_5.png} % m39358;MG10C14\_5.png;;;6.0;8.5;
\vspace{2pt}
\vspace{.1in}
\end{center}
\end{figure}       \begin{enumerate}[noitemsep, label=\textbf{\alph*}. ] 
\item Find the equation of line AB \item Calculate the length of AB\item  A' is
the image of A and B' is the image of B. Both these images are obtain by
applying the transformation: (x;y)$\to $(x-4;y-1). Give the coordinates of both
A' and B'\item Find the equation of A'B'\item Calculate the length of A'B'\item
Can you state with certainty that AA'B'B is a parallelogram? Justify your
answer.\end{enumerate}
    \item The vertices of triangle PQR have co-ordinates as shown in the
diagram.
\setcounter{subfigure}{0}
\begin{figure}[H] % horizontal\label{m39358*id743344}
\begin{center}
\label{m39358*id743344!!!underscore!!!media}\label{
m39358*id743344!!!underscore!!!printimage}\includegraphics[width=300px]{
col11306.imgs/m39358_mg10c14_6.png} % m39358;mg10c14\_6.png;;;6.0;8.5;
\vspace{2pt}
\vspace{.1in}
\end{center}
\end{figure}       
\begin{enumerate}[noitemsep, label=\textbf{\alph*}. ] 
\item Give the co-ordinates of  P', Q' and R', the images of P, Q and R when P,
Q and R are reflected in the line y=x.\item Determine the area of triangle
PQR.\end{enumerate}
\item Which of the following claims are true? Give a counter-example for those
that are incorrect.
\begin{enumerate}[noitemsep, label=\textbf{\alph*}. ] 
\item All equilateral triangles are similar.
\item All regular quadrilaterals are similar.
\item In any $▵ABC$ with $\angle ABC={90}^{\circ }$ we have
$A{B}^{3}+B{C}^{3}=C{A}^{3}$.
\item All right-angled isosceles triangles with perimeter 10 cm are congruent.
\item All rectangles with the same area are similar.
\end{enumerate}
\item For each pair of figures state whether they are similar or not. Give
reasons.
\setcounter{subfigure}{0}
\begin{figure}[H] % horizontal\label{m39358*id320590}
\begin{center}
\label{m39358*id320590!!!underscore!!!media}\label{
m39358*id320590!!!underscore!!!printimage}\includegraphics[width=.8\columnwidth]
{col11306.imgs/m39358_MG10C13_053.png} % m39358;MG10C13\_053.png;;;6.0;8.5;
\vspace{2pt}
\vspace{.1in}
\end{center}
\end{figure}               \end{enumerate}

\par \raisebox{-5
pt}{\includegraphics[width=0.5cm]{col11306.imgs/summary_www.png}} Find the
answers with the shortcodes:
\par \begin{tabular}[h]{cccccc}
(1.) lbr  &  (2.) lT9  &  (3.) lTX  &  (4.) lTI  &  (5.) la7  &  (6.) laY  & 
(7.) laq  &  (8.) la4  &  (9.) l4o  &  (10.) laG  &  (11.) lal  &  (12.) la3  &
\end{tabular}

\subsection{ Summary}
\begin{itemize}[noitemsep]
\item Make sure you know what the following terms mean: quadrilaterals,
vertices, sides, angles, parallel lines, perpendicular lines,diagonals,
bisectors and transversals.\item The properties of triangles has been
covered.\item Congruency and similarity of triangles\item Angles can be
classified as acute, right, obtuse, straight, reflex or revolution\item Theorem
of Pythagoras which is used to calculate the lengths of sides of a right-angled
triangle\item Angles: 
\begin{itemize}[noitemsep]
\item Acute angle: An angle ${0}^{\circ }$ and ${90}^{\circ }$\item Right angle:
An angle measuring ${90}^{\circ }$\item Obtuse angle: An angle ${90}^{\circ }$
and ${180}^{\circ }$\item Straight angle: An angle measuring ${180}^{\circ
}$\item Reflex angle: An angle ${180}^{\circ }$ and ${360}^{\circ }$\item
Revolution: An angle measuring ${360}^{\circ }$\end{itemize}
\item There are several properties of angles and some special names for
these\item There are four types of triangles: Equilateral, isoceles,
right-angled and scalene\item The angles in a triangle add up to ${180}^{\circ
}$\end{itemize}

\subsection{ Exercises}
\nopagebreak
\begin{enumerate}[noitemsep, label=\textbf{\arabic*}. ] 
\item Find all the pairs of parallel lines in the following figures, giving
reasons in each case.
\begin{enumerate}[noitemsep, label=\textbf{\alph*}. ] 
\item 
\setcounter{subfigure}{0}
\begin{figure}[H] % horizontal\label{m39368*id320164}
\begin{center}
\label{m39368*id320164!!!underscore!!!media}\label{
m39368*id320164!!!underscore!!!printimage}\includegraphics{
col11306.imgs/m39368_MG10C13_054.png} % m39368;MG10C13\_054.png;;;6.0;8.5;
\vspace{2pt}
\vspace{.1in}
\end{center}
\end{figure}       
\item 
\setcounter{subfigure}{0}
\begin{figure}[H] % horizontal\label{m39368*id320183}
\begin{center}
\label{m39368*id320183!!!underscore!!!media}\label{
m39368*id320183!!!underscore!!!printimage}\includegraphics{
col11306.imgs/m39368_MG10C13_055.png} % m39368;MG10C13\_055.png;;;6.0;8.5;
\vspace{2pt}
\vspace{.1in}
\end{center}
\end{figure}       
\item 
\setcounter{subfigure}{0}
\begin{figure}[H] % horizontal\label{m39368*id320201}
\begin{center}
\label{m39368*id320201!!!underscore!!!media}\label{
m39368*id320201!!!underscore!!!printimage}\includegraphics{
col11306.imgs/m39368_MG10C13_056.png} % m39368;MG10C13\_056.png;;;6.0;8.5;
\vspace{2pt}
\vspace{.1in}
\end{center}
\end{figure}       
\end{enumerate}
\item Find angles $a$, $b$, $c$ and $d$ in each case, giving reasons.
}\begin{enumerate}[noitemsep, label=\textbf{\alph*}. ] 
\item 
\setcounter{subfigure}{0}
\begin{figure}[H] % horizontal\label{m39368*id320271}
\begin{center}
\label{m39368*id320271!!!underscore!!!media}\label{
m39368*id320271!!!underscore!!!printimage}\includegraphics{
col11306.imgs/m39368_MG10C13_057.png} % m39368;MG10C13\_057.png;;;6.0;8.5;
\vspace{2pt}
\vspace{.1in}
\end{center}
\end{figure}         
\section{ Polygons and quadrilaterals}
\nopagebreak
 $ \hspace{-5pt}\begin{array}{cccccccccccc}  
\includegraphics[width=0.75cm]{col11306.imgs/summary_fullmarks.png} &  
\end{array} $ \hspace{2 pt}\raisebox{-5 pt}{} {(section shortcode: MG10093 )}
\par 
\item 
\setcounter{subfigure}{0}
\begin{figure}[H] % horizontal\label{m39368*id320290}
\begin{center}
\label{m39368*id320290!!!underscore!!!media}\label{
m39368*id320290!!!underscore!!!printimage}\includegraphics{
col11306.imgs/m39368_MG10C13_058.png} % m39368;MG10C13\_058.png;;;6.0;8.5;
\vspace{2pt}
\vspace{.1in}
\end{center}
\section{ Polygons and quadrilaterals}

 $ \hspace{-5pt}\begin{array}{cccccccccccc}  
\includegraphics[width=0.75cm]{col11306.imgs/summary_fullmarks.png} &  
\end{array} $ \hspace{2 pt}\raisebox{-5 pt}{} {(section shortcode: MG10093 )}
\par 
\end{figure}       

\item 
\setcounter{subfigure}{0}
\begin{figure}[H] % horizontal\label{m39368*id320310}
\begin{center}
\label{m39368*id320310!!!underscore!!!media}\label{
m39368*id320310!!!underscore!!!printimage}\includegraphics{
col11306.imgs/m39368_MG10C13_059.png} % m39368;MG10C13\_059.png;;;6.0;8.5;
\vspace{2pt}
\vspace{.1in}
\end{center}
\end{figure}       \end{enumerate}
\item Say which of the following pairs of triangles are congruent with reasons.
\begin{enumerate}[noitemsep, label=\textbf{\alph*}. ] 
\item 
\setcounter{subfigure}{0}
\begin{figure}[H] % horizontal\label{m39368*id320512}
\begin{center}
\label{m39368*id320512!!!underscore!!!media}\label{
m39368*id320512!!!underscore!!!printimage}\includegraphics{
col11306.imgs/m39368_MG10C13_060.png} % m39368;MG10C13\_060.png;;;6.0;8.5;
\vspace{2pt}
\vspace{.1in}
\end{center}
\end{figure}       
\item 
\setcounter{subfigure}{0}
\begin{figure}[H] % horizontal\label{m39368*id320530}
\begin{center}
\label{m39368*id320530!!!underscore!!!media}\label{
m39368*id320530!!!underscore!!!printimage}\includegraphics{
col11306.imgs/m39368_MG10C13_061.png} % m39368;MG10C13\_061.png;;;6.0;8.5;
\vspace{2pt}
\vspace{.1in}
\end{center}
\end{figure}     
\item 
\setcounter{subfigure}{0}
\begin{figure}[H] % horizontal\label{m39368*id320548}
\begin{center}
\label{m39368*id320548!!!underscore!!!media}\label{
m39368*id320548!!!underscore!!!printimage}\includegraphics{
col11306.imgs/m39368_MG10C13_062.png} % m39368;MG10C13\_062.png;;;6.0;8.5;
\vspace{2pt}
\vspace{.1in}
\end{center}
\end{figure}       

\item 
\setcounter{subfigure}{0}
\begin{figure}[H] % horizontal\label{m39368*id320565}
\begin{center}
\label{m39368*id320565!!!underscore!!!media}\label{
m39368*id320565!!!underscore!!!printimage}\includegraphics{
col11306.imgs/m39368_MG10C13_063.png} % m39368;MG10C13\_063.png;;;6.0;8.5;
\vspace{2pt}
\vspace{.1in}
\end{center}
\end{figure}       \end{enumerate}
\item Identify the types of angles shown below (e.g. acute/obtuse etc):
\setcounter{subfigure}{0}
\begin{figure}[H] % horizontal\label{m39368*id401231}
\begin{center}
\label{m39368*id401231!!!underscore!!!media}\label{
m39368*id401231!!!underscore!!!printimage}\includegraphics[width=300px]{
col11306.imgs/m39368_MG10C13_066.png} % m39368;MG10C13\_066.png;;;6.0;8.5;
\vspace{2pt}
\vspace{.1in}
\end{center}
\end{figure}       
\item Calculate the size of the third angle (x) in each of the diagrams below:
\setcounter{subfigure}{0}
\begin{figure}[H] % horizontal\label{m39368*id401232}
\begin{center}
\label{m39368*id401232!!!underscore!!!media}\label{
m39368*id401232!!!underscore!!!printimage}\includegraphics[width=300px]{
col11306.imgs/m39368_MG10C13_067.png} % m39368;MG10C13\_067.png;;;6.0;8.5;
\vspace{2pt}
\vspace{.1in}
\end{center}
\end{figure}       

\item Name each of the shapes/polygons, state how many sides each has and
whether it is regular (equiangular and equilateral) or not:
\setcounter{subfigure}{0}
\begin{figure}[H] % horizontal\label{m39368*id401233}
\begin{center}
\label{m39368*id401233!!!underscore!!!media}\label{
m39368*id401233!!!underscore!!!printimage}\includegraphics[width=300px]{
col11306.imgs/m39368_MG10C13_068.png} % m39368;MG10C13\_068.png;;;6.0;8.5;
\vspace{2pt}
\vspace{.1in}
\end{center}
\end{figure}       
\item Assess whether the following statements are true or false. If the
statement is false, explain why:
\begin{enumerate}[noitemsep, label=\textbf{\alph*}. ] 
\item An angle is formed when two straight lines meet at a point.	\item
The smallest angle that can be drawn is 5\ensuremath{{\,}^{\circ}}.\item An
angle of 90\ensuremath{{\,}^{\circ}} is called a square angle.\item Two angles
whose sum is 180\ensuremath{{\,}^{\circ}} are called supplementary angles.\item
Two parallel lines will never intersect.\item A regular polygon has equal angles
but not equal sides.\item An isoceles triangle has three equal sides.\item If
three sides of a triangle are equal in length to the same sides of another
triangle, then the two triangles are incongruent.\item If three pairs of
corresponding angles in two triangles are equal, then the triangles are
similar.\end{enumerate}
\item Name the type of angle (e.g. acute/obtuse etc) based on it's size:
\begin{enumerate}[noitemsep, label=\textbf{\alph*}. ] 
\item  30\ensuremath{{\,}^{\circ}}\item  47\ensuremath{{\,}^{\circ}}\item 
90\ensuremath{{\,}^{\circ}}\item  91\ensuremath{{\,}^{\circ}}\item 
191\ensuremath{{\,}^{\circ}}\item  360\ensuremath{{\,}^{\circ}}\item 
180\ensuremath{{\,}^{\circ}}\end{enumerate}
\item Using Pythagoras' theorem for right-angled triangles, calculate the length
of x:
\setcounter{subfigure}{0}
\begin{figure}[H] % horizontal\label{m39368*id401236}
\begin{center}
\label{m39368*id401236!!!underscore!!!media}\label{
m39368*id401236!!!underscore!!!printimage}\includegraphics[width=300px]{
col11306.imgs/m39368_MG10C13_070.png} % m39368;MG10C13\_070.png;;;6.0;8.5;
\vspace{2pt}
\vspace{.1in}
\end{center}
\end{figure}       
\end{enumerate}

\par \raisebox{-5
pt}{\includegraphics[width=0.5cm]{col11306.imgs/summary_www.png}} Find the
answers with the shortcodes:
\par \begin{tabular}[h]{cccccc}
(1.) lxh  &  (2.) laq  &  (3.) lai  &  (4.) lTb  &  (5.) lTj  &  (6.) lTD  & 
(7.) lTZ  &  (8.) lTB  &  (9.) lTK  & \end{tabular}
\subsubsection{ Challenge Problem}
\begin{enumerate}[noitemsep, label=\textbf{\arabic*}. ] 
\item Using the figure below, show that the sum of the three angles in a
triangle is 180$^{\circ }$. Line $DE$ is parallel to $BC$.
\setcounter{subfigure}{0}
\begin{figure}[H] % horizontal\label{m39368*id320668}
\begin{center}
\label{m39368*id320668!!!underscore!!!media}\label{
m39368*id320668!!!underscore!!!printimage}\includegraphics{
col11306.imgs/m39368_MG10C13_065.png} % m39368;MG10C13\_065.png;;;6.0;8.5;
\vspace{2pt}
\vspace{.1in}
\end{center}
\end{figure}       \newline
\end{enumerate}

\par \raisebox{-5
pt}{\includegraphics[width=0.5cm]{col11306.imgs/summary_www.png}} Find the
answers with the shortcodes:
\par \begin{tabular}[h]{cccccc}
(1.) laO  & \end{tabular}
