       \def\leftmark{GLOSSARY}
      \def\rightmark{GLOSSARY}
      \begin{indexheading}
\chapter{Glossary}
      \label{col11306*Glossary}
      \end{indexheading}
      \vspace{.3cm}
    \begin{description}\setlength{\topsep}{0cm}\setlength{\itemsep}{0cm}
    \setlength{\parskip}{0cm}\setlength{\parsep}{0cm}
    \setlength{\partopsep}{0cm}
    \setlength{\labelwidth}{.6cm}\setlength{\labelsep}{0cm}
    \setlength{\leftmargin}{1cm}
	    \vspace{.3cm}
	    \item[{\large \bfseries C}]\noindent\raggedright
	    {\bf Common difference}\\\begin{description}\item{\hspace{.3cm}}\hspace{.3cm}The common difference is the difference between successive terms and is denoted by d.\\\end{description}
	    \item[] \noindent\raggedright {\bf  Compound Interest }\\\begin{description}\item{\hspace{.3cm}}\hspace{.3cm}
      \label{m39334*id72635}Compound interest is the interest payable on the principal and its accumulated interest. \par 
      \\\end{description}
	    \vspace{.3cm}
	    \item[{\large \bfseries D}]\noindent\raggedright
	    {\bf  Data }\\\begin{description}\item{\hspace{.3cm}}\hspace{.3cm}
          \label{m39403*id200337}Data refers to the pieces of information that have been observed and recorded, from an experiment or a survey. There are two types of data: primary and secondary. The word "data" is the plural of the word "datum", and therefore one should say, "the data are" and not "the data is". \par 
          \\\end{description}
	    \vspace{.3cm}
	    \item[{\large \bfseries E}]\noindent\raggedright
	    {\bf  Equality for Exponential Functions }\\\begin{description}\item{\hspace{.3cm}}\hspace{.3cm}
        \label{m39253*id155201}If \begin{math}a\end{math} is a positive number such
that \begin{math}a\greatthan{}0\end{math}, (except when \begin{math}a=1\end{math}\hspace{1ex}) then:\par 
        \label{m39253*id155230}\nopagebreak\noindent{}
          \settowidth{\mymathboxwidth}{\begin{equation}
    {a}^{x}={a}^{y}\tag{9.40}
      \end{equation}
    }
    \typeout{Columnwidth = \the\columnwidth}\typeout{math as usual width = \the\mymathboxwidth}
    \ifthenelse{\lengthtest{\mymathboxwidth < \columnwidth}}{% if the math fits, do it again, for real
    \begin{equation}
    {a}^{x}={a}^{y}\tag{9.40}
      \end{equation}
    }{% else, if it doesn't fit
    \setlength{\mymathboxwidth}{\columnwidth}
      \addtolength{\mymathboxwidth}{-48pt}
    \par\vspace{12pt}\noindent\begin{minipage}{\columnwidth}
    \parbox[t]{\mymathboxwidth}{\large\begin{math}
    {a}^{x}={a}^{y}\end{math}}\hfill
    \parbox[t]{48pt}{\raggedleft 
    (9.40)}
    \end{minipage}\vspace{12pt}\par
    }% end of conditional for this bit of math
    \typeout{math as usual width = \the\mymathboxwidth}
        \label{m39253*id155257}if and only if:\par 
        \label{m39253*id155263}\nopagebreak\noindent{}
          \settowidth{\mymathboxwidth}{\begin{equation}
    x=y\tag{9.41}
      \end{equation}
    }
    \typeout{Columnwidth = \the\columnwidth}\typeout{math as usual width = \the\mymathboxwidth}
    \ifthenelse{\lengthtest{\mymathboxwidth < \columnwidth}}{% if the math fits, do it again, for real
    \begin{equation}
    x=y\tag{9.41}
      \end{equation}
    }{% else, if it doesn't fit
    \setlength{\mymathboxwidth}{\columnwidth}
      \addtolength{\mymathboxwidth}{-48pt}
    \par\vspace{12pt}\noindent\begin{minipage}{\columnwidth}
    \parbox[t]{\mymathboxwidth}{\large\begin{math}
    x=y\end{math}}\hfill
    \parbox[t]{48pt}{\raggedleft 
    (9.41)}
    \end{minipage}\vspace{12pt}\par
    }% end of conditional for this bit of math
    \typeout{math as usual width = \the\mymathboxwidth}
        (If \begin{math}a=1\end{math}, then \begin{math}x\end{math} and \begin{math}y\end{math} can differ)
        \\\end{description}
	    \item[] \noindent\raggedright {\bf  Exponential Notation }\\\begin{description}\item{\hspace{.3cm}}\hspace{.3cm}
      \label{m38359*id62672}Exponential notation means a number written like\par 
      \label{m38359*id62677}\nopagebreak\noindent{}
        \settowidth{\mymathboxwidth}{\begin{equation}
    {a}^{n}\tag{5.1}
      \end{equation}
    }
    \typeout{Columnwidth = \the\columnwidth}\typeout{math as usual width = \the\mymathboxwidth}
    \ifthenelse{\lengthtest{\mymathboxwidth < \columnwidth}}{% if the math fits, do it again, for real
    \begin{equation}
    {a}^{n}\tag{5.1}
      \end{equation}
    }{% else, if it doesn't fit
    \setlength{\mymathboxwidth}{\columnwidth}
      \addtolength{\mymathboxwidth}{-48pt}
    \par\vspace{12pt}\noindent\begin{minipage}{\columnwidth}
    \parbox[t]{\mymathboxwidth}{\large\begin{math}
    {a}^{n}\end{math}}\hfill
    \parbox[t]{48pt}{\raggedleft 
    (5.1)}
    \end{minipage}\vspace{12pt}\par
    }% end of conditional for this bit of math
    \typeout{math as usual width = \the\mymathboxwidth}
      \label{m38359*id62692}where \begin{math}n\end{math} is an integer and \begin{math}a\end{math} can be any real number. \begin{math}a\end{math} is called the \textsl{base} and \begin{math}n\end{math} is called the \textsl{exponent} or \textsl{index}. \par 
      \\\end{description}
	    \vspace{.3cm}
	    \item[{\large \bfseries I}]\noindent\raggedright
	    {\bf  Inter-quartile Range }\\\begin{description}\item{\hspace{.3cm}}\hspace{.3cm}
          \label{m39400*id214019}The inter quartile range is a measure which provides information about the spread of a data set, and is calculated by subtracting the first quartile from the third quartile, giving the range of the middle half of the data set, trimming off the lowest and highest quarters, i.e. \begin{math}{Q}_{3}-{Q}_{1}\end{math}. \par 
          \\\end{description}
	    \vspace{.3cm}
	    \item[{\large \bfseries M}]\noindent\raggedright
	    {\bf  Mean }\\\begin{description}\item{\hspace{.3cm}}\hspace{.3cm}
          \label{m39400*id211228}The mean of a data set, \begin{math}x\end{math}, denoted by \begin{math}\overline{x}\end{math}, is the average of the data values, and is calculated as:\par 
          \label{m39400*uid61}\nopagebreak\noindent{}\settowidth{\mymathboxwidth}{\begin{equation}
    \overline{x}\phantom{\rule{4pt}{0ex}}=\phantom{\rule{4pt}{0ex}}\frac{\mathrm{sum\; of\; all\; values}}{\mathrm{number\; of\; all\; values}}\phantom{\rule{4pt}{0ex}}=\phantom{\rule{4pt}{0ex}}\frac{{x}_{1}+{x}_{2}+{x}_{3}+...+{x}_{n}}{n}\tag{16.1}
      \end{equation}
    }
    \typeout{Columnwidth = \the\columnwidth}\typeout{math as usual width = \the\mymathboxwidth}
    \ifthenelse{\lengthtest{\mymathboxwidth < \columnwidth}}{% if the math fits, do it again, for real
    \begin{equation}
    \overline{x}\phantom{\rule{4pt}{0ex}}=\phantom{\rule{4pt}{0ex}}\frac{\mathrm{sum\; of\; all\; values}}{\mathrm{number\; of\; all\; values}}\phantom{\rule{4pt}{0ex}}=\phantom{\rule{4pt}{0ex}}\frac{{x}_{1}+{x}_{2}+{x}_{3}+...+{x}_{n}}{n}\tag{16.1}
      \end{equation}
    }{% else, if it doesn't fit
    \setlength{\mymathboxwidth}{\columnwidth}
      \addtolength{\mymathboxwidth}{-48pt}
    \par\vspace{12pt}\noindent\begin{minipage}{\columnwidth}
    \parbox[t]{\mymathboxwidth}{\large\begin{math}
    \overline{x}\phantom{\rule{4pt}{0ex}}=\phantom{\rule{4pt}{0ex}}\frac{\mathrm{sum\; of\; all\; values}}{\mathrm{number\; of\; all\; values}}\phantom{\rule{4pt}{0ex}}=\phantom{\rule{4pt}{0ex}}\frac{{x}_{1}+{x}_{2}+{x}_{3}+...+{x}_{n}}{n}\end{math}}\hfill
    \parbox[t]{48pt}{\raggedleft 
    (16.1)}
    \end{minipage}\vspace{12pt}\par
    }% end of conditional for this bit of math
    \typeout{math as usual width = \the\mymathboxwidth}
          \\\end{description}
	    \item[] \noindent\raggedright {\bf  Median }\\\begin{description}\item{\hspace{.3cm}}\hspace{.3cm}
          \label{m39400*id211687}The median of a set of data is the data value in the central position, when the data set has been arranged from highest to lowest or from lowest to highest. There are an equal number of data values on either side of the median value. \par 
          \\\end{description}
	    \item[] \noindent\raggedright {\bf  Mode }\\\begin{description}\item{\hspace{.3cm}}\hspace{.3cm}
          \label{m39400*id212182}The mode is the data value that occurs most often, i.e. it is the most frequent value or most common value in a set. \par 
          \\\end{description}
	    \vspace{.3cm}
	    \item[{\large \bfseries P}]\noindent\raggedright
	    {\bf  Percentiles }\\\begin{description}\item{\hspace{.3cm}}\hspace{.3cm}
          \label{m39400*id214611}Percentiles are the 99 data values that divide a data set into 100 groups. \par 
          \\\end{description}
	    \vspace{.3cm}
	    \item[{\large \bfseries Q}]\noindent\raggedright
	    {\bf  Quartiles }\\\begin{description}\item{\hspace{.3cm}}\hspace{.3cm}
          \label{m39400*id212998}Quartiles are the three data values that divide an ordered data set into four groups containing equal numbers of data values. The median is the second quartile. \par 
          \\\end{description}
	    \vspace{.3cm}
	    \item[{\large \bfseries R}]\noindent\raggedright
	    {\bf  Range }\\\begin{description}\item{\hspace{.3cm}}\hspace{.3cm}
          \label{m39400*id212688}The range of a data set is the difference between the lowest value and the highest value in the set. \par 
          \\\end{description}
	    \item[] \noindent\raggedright {\bf  Rational Number }\\\begin{description}\item{\hspace{.3cm}}\hspace{.3cm}
      \label{m38348*id62709}A rational number is any number which can be written as:\par 
      \label{m38348*uid6}\nopagebreak\noindent{}
        \settowidth{\mymathboxwidth}{\begin{equation}
    \frac{a}{b}\tag{4.2}
      \end{equation}
    }
    \typeout{Columnwidth = \the\columnwidth}\typeout{math as usual width = \the\mymathboxwidth}
    \ifthenelse{\lengthtest{\mymathboxwidth < \columnwidth}}{% if the math fits, do it again, for real
    \begin{equation}
    \frac{a}{b}\tag{4.2}
      \end{equation}
    }{% else, if it doesn't fit
    \setlength{\mymathboxwidth}{\columnwidth}
      \addtolength{\mymathboxwidth}{-48pt}
    \par\vspace{12pt}\noindent\begin{minipage}{\columnwidth}
    \parbox[t]{\mymathboxwidth}{\large\begin{math}
    \frac{a}{b}\end{math}}\hfill
    \parbox[t]{48pt}{\raggedleft 
    (4.2)}
    \end{minipage}\vspace{12pt}\par
    }% end of conditional for this bit of math
    \typeout{math as usual width = \the\mymathboxwidth}
      \label{m38348*id62732}where \begin{math}a\end{math} and \begin{math}b\end{math} are integers and \begin{math}b\ne 0\end{math}. \par 
      \\\end{description}
	    \vspace{.3cm}
	    \item[{\large \bfseries S}]\noindent\raggedright
	    {\bf  Similar Polygons }\\\begin{description}\item{\hspace{.3cm}}\hspace{.3cm}
        \label{m39354*id65247}Two polygons are similar if:\par 
        \label{m39354*id65253}\begin{itemize}[noitemsep]
            \label{m39354*uid30}  
            \item their corresponding angles are equal, \textbf{and}\label{m39354*uid31}  
            \item the ratios of corresponding sides are equal.
\end{itemize}
        \\\end{description}
	    \item[] \noindent\raggedright {\bf  Simple Interest }\\\begin{description}\item{\hspace{.3cm}}\hspace{.3cm}
      \label{m39332*id69230}Simple interest is where you earn interest on the initial amount that you invested, but not interest on interest. \par 
      \\\end{description}
    \end{description}
      \newpage 
      \def\leftmark{INDEX}
      \def\rightmark{INDEX}
      \begin{indexheading}
