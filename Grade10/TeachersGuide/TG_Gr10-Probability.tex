\chapter{Probability}

\begin{exercises}{}
{
  \begin{enumerate}[itemsep=5pt, label=\textbf{\arabic*}. ]
  \item 
A bag contains $6$ red, $3$ blue, $2$ green and $1$ white
    balls. A ball is picked at random. Determine the probability that it
    is:
    \begin{enumerate}[noitemsep, label=\textbf{(\alph*)} ]
    \item red
    \item blue or white
    \item not green
    \item not green or red
    \end{enumerate}
  \item 
A playing card is selected randomly from a pack of $52$
    cards. Determine the probability that it is:
    \begin{enumerate}[noitemsep, label=\textbf{(\alph*)} ]
    \item the $2$ of hearts
    \item a red card
    \item a picture card
    \item an ace
    \item a number less than $4$?
    \end{enumerate}
\item Even numbers in the range $2$--$100$ are written on cards. 
  What is
    the probability of selecting a multiple of $5$, if a card is drawn
    at random?
\end{enumerate}
}
\end{exercises}


 \begin{solutions}{}{
\begin{enumerate}[itemsep=5pt, label=\textbf{\arabic*}. ] 
\item %solution 1
    \begin{enumerate}[noitemsep, label=\textbf{(\alph*)} ]
    \item $\frac{6}{12} &=& \frac{1}{2}$
    \item $\frac{(3 + 1)}{12} &=& \frac{1}{3}$
    \item $1 - (\frac{2}{12}) &=& \frac{5}{6}$
    \item $1 - \frac{(2 + 6)}{12} &= \frac{4}{12}\\
				  &= \frac{1}{3}$Bcc   
    \end{enumerate}
\item %solution 2
    \begin{enumerate}[noitemsep, label=\textbf{(\alph*)} ]
    \item $\frac{1}{52}$ (only one in the deck)
    \item $\frac{1}{2}$ (half the cards are red, half are black)
    \item $\frac{3}{13}$ (for each suite of 13 cards, there are three picture cards: J, Q, K)
    \item $\frac{4}{52} = \frac{1}{13}$ (four aces in the deck)
    \item $\frac{3}{13}$ (for each suite of 13 cards, there are three cards less than 4: A, 2 and 3)
    \end{enumerate}
\item %solution 3
    There are 50 cards.  They are all even.\\
    All even numbers that are also multiples of 5 are multiples of 10 (10, 20,...., 100).\\
    There are 10 of them.\\
    Therefore, the probability is $\frac{10}{50} = \frac{1}{5}$.
\end{enumerate}}
\end{solutions}


% \section{Relative frequency}
% \section{Venn diagrams}
\begin{exercises}{}
{
    \begin{enumerate}[itemsep=5pt, label=\textbf{\arabic*}. ]
   \item Let $S$ denote the set of whole numbers from $1$ to $16$, $X$
    denote the set of even numbers from $1$ to $16$ and $Y$ denote the
    set of prime numbers from $1$ to $16$. Draw a Venn diagram accurately depicting $S$, $X$ and $Y$.
\item
  There are $79$ Grade $10$ learners at school. All of these
    take some combination of Maths, Geography and History. The number who take
    Geography is $41$, those who take History is $36$, and $30$ take
    Maths. The number who take Maths and History is $16$; the number
    who take Geography and History is $6$, and there are $8$ who take
    Maths only and $16$ who take only History.
    \begin{enumerate}[noitemsep, label=\textbf{(\alph*)} ]
    \item Draw a Venn diagram to illustrate all this information.
    \item How many learners take Maths and Geography but not History?
    \item How many learners take Geography only?
    \item How many learners take all three subjects?
    \end{enumerate}
 \item Pieces of paper labelled with the numbers $1$ to $12$ are
    placed in a box and the box is shaken. One piece of paper is taken
    out and then replaced.
    \begin{enumerate}[noitemsep, label=\textbf{(\alph*)} ]
    \item What is the sample space, $S$?
    \item Write down the set $A$, representing the event of taking a
      piece of paper labelled with a factor of $12$.
    \item Write down the set $B$, representing the event of taking a
      piece of paper labelled with a prime number.
    \item Represent $A$, $B$ and $S$ by means of a Venn diagram.
    \item Find
      \begin{enumerate}[noitemsep, label=\textbf{\roman*.} ]
      \item $n\left(S\right)$
      \item $n\left(A\right)$
      \item $n\left(B\right)$
      \end{enumerate}
    \end{enumerate}
  \end{enumerate}
% \practiceinfo
%   \begin{tabularx}{\textwidth}{XXXX}
%     (1.) AAA & (2.) AAA & (3.) AAA \\
%   \end{tabularx}
}
\end{exercises}


 \begin{solutions}{}{
\begin{enumerate}[itemsep=5pt, label=\textbf{\arabic*}. ] 
\item %solution 1
	  \scalebox{0.8} % Change this value to rescale the drawing.
	  {
	  \begin{pspicture}(0,-2.2)(4.5590625,2.2)
	  \pscircle[linewidth=0.04,dimen=outer](2.2,0.0){2.2}
	  \pscircle[linewidth=0.04,dimen=outer](1.45,-0.03){1.29}
	  \pscircle[linewidth=0.04,dimen=outer](2.82,0.48){1.18}
	  \usefont{T1}{ppl}{m}{n}
	  \rput(4.184531,1.89){\LARGE$S$}
	  \usefont{T1}{ppl}{m}{n}
	  \rput(2.1845312,0.27){\LARGE$2$}
	  \usefont{T1}{ppl}{m}{n}
	  \rput(1.2445313,0.89){\LARGE$4$}
	  \usefont{T1}{ppl}{m}{n}
	  \rput(0.64453125,0.51){\LARGE$6$}
	  \usefont{T1}{ppl}{m}{n}
	  \rput(1.3245312,0.19){\LARGE$8$}
	  \usefont{T1}{ppl}{m}{n}
	  \rput(0.6745312,-0.15){\LARGE$10$}
	  \usefont{T1}{ppl}{m}{n}
	  \rput(2.6045313,1.19){\LARGE$3$}
	  \usefont{T1}{ppl}{m}{n}
	  \rput(3.3245313,1.13){\LARGE $5$}
	  \usefont{T1}{ppl}{m}{n}
	  \rput(2.9645312,0.63){\LARGE$7$}
	  \usefont{T1}{ppl}{m}{n}
	  \rput(2.9045312,-1.41){\LARGE$9$}
	  \usefont{T1}{ppl}{m}{n}
	  \rput(1.0045313,1.5){\LARGE$X$}
	  \usefont{T1}{ppl}{m}{n}
	  \rput(2.4945312,1.9){\LARGE$Y$}
	  \usefont{T1}{ppl}{m}{n}
	  \rput(0.75453126,-0.59){\LARGE$12$}
	  \usefont{T1}{ppl}{m}{n}
	  \rput(1.7545313,-0.61){\LARGE$14$}
	  \usefont{T1}{ppl}{m}{n}
	  \rput(1.3945312,-1.03){\LARGE$16$}
	  \usefont{T1}{ppl}{m}{n}
	  \rput(3.5145311,0.35){\LARGE$11$}
	  \usefont{T1}{ppl}{m}{n}
	  \rput(3.2545311,-0.17){\LARGE$13$}
	  \usefont{T1}{ppl}{m}{n}
	  \rput(1.8445313,-1.69){\LARGE$1$}
	  \usefont{T1}{ppl}{m}{n}
	  \rput(3.6745312,-0.87){\LARGE$15$}
	  \end{pspicture} 
	  }
\item%solution 2
    \begin{enumerate}[noitemsep, label=\textbf{(\alph*)} ]
    \item
		      \scalebox{0.8} % Change this value to rescale the drawing.
		      {
		      \begin{pspicture}(0,-2.393125)(4.7790623,2.393125)
		      \pscircle[linewidth=0.04,dimen=outer](1.64,0.7196875){1.18}
		      \pscircle[linewidth=0.04,dimen=outer](1.16,-0.8203125){1.16}
		      \pscircle[linewidth=0.04,dimen=outer](2.64,-0.2603125){1.18}
		      \usefont{T1}{ppl}{m}{n}
		      \rput(4.3,0.6096875){\LARGE$G:41$}
		      \usefont{T1}{ppl}{m}{n}
		      \rput(0.77453125,-2.2){\LARGE$H:36$}
		      \usefont{T1}{ppl}{m}{n}
		      \rput(1.9145312,2.1896875){\LARGE$M:30$}
		      \usefont{T1}{ptm}{m}{n}
		      \rput(0.81453127,-1.1503125){\LARGE$16$}
		      \usefont{T1}{ptm}{m}{n}
		      \rput(1.7445313,1.3096875){\LARGE$8$}
		      \usefont{T1}{ptm}{m}{n}
		      \rput(1.9245312,-0.7503125){\LARGE$6$}
		      \usefont{T1}{ptm}{m}{n}
		      \rput(1.1945312,-0.0303125){\LARGE$16$}
		      \end{pspicture} 
		      }

    \item Each student must do exactly one of the following:
    \begin{itemize}
	\item Take only geography;
	\item Only take maths and/or history;
    \end{itemize}
    There are $30 + 36 -16 =50$ students doing the second one, therefore the must be $79 - 50 = 29$ only doing geography.\\
    \newline
    Each student must do exactly one of:
    \begin{itemize}
	\item Only take geography;
	\item Only take maths;
	\item Take history;
	\item Take geography and maths, but not history;
    \end{itemize}
    There are $29, 8$ and $36$ of the first three, so the answer to B is:\\
    $79 - 29 -8 - 36 = 6$ people

    \item Calculated already: $29$

    \item Each student must do exactly one of:
    \begin{itemize}
	\item Do geography
	\item Only do maths
	\item Only do history
	\item Do maths and history but not geography
    \end{itemize}
    Using the same method as before, the number of people in the last group is:\\
 
    $79 - 41 - 8 - 16 = 14$\\

    But, $16$ people do maths and history, so there must be $16 - 14$ people who do all three.
    \end{enumerate}

\item %solution 3
    \begin{enumerate}[noitemsep, label=\textbf{(\alph*)} ]
    \item $S = \{1; 2;\ldots; 12\}$
    
    \item $A = \{1; 2; 3; 4; 6; 12\}$

    \item $B = \{2; 3; 5; 7; 11\}$

    \item

    		  \scalebox{0.8} % Change this value to rescale the drawing.

		  {
		  \begin{pspicture}(0,-2.2)(4.5590625,2.2)
		  \pscircle[linewidth=0.04,dimen=outer](2.2,0.0){2.2}
		  \pscircle[linewidth=0.04,dimen=outer](1.45,-0.03){1.29}
		  \pscircle[linewidth=0.04,dimen=outer](2.82,0.48){1.18}
		  \usefont{T1}{ppl}{m}{n}
		  \rput(4.184531,1.89){\LARGE $S$}
		  \usefont{T1}{ppl}{m}{n}
		  \rput(2.0045311,0.65){\LARGE$2$}
		  \usefont{T1}{ppl}{m}{n}
		  \rput(0.8645313,0.19){\LARGE$4$}
		  \usefont{T1}{ppl}{m}{n}
		  \rput(1.4645313,-0.27){\LARGE$6$}
		  \usefont{T1}{ppl}{m}{n}
		  \rput(1.9245312,-1.67){\LARGE$8$}
		  \usefont{T1}{ppl}{m}{n}
		  \rput(3.5345314,-1.03){\LARGE$10$}
		  \usefont{T1}{ppl}{m}{n}
		  \rput(2.2245312,0.07){\LARGE$3$}
		  \usefont{T1}{ppl}{m}{n}
		  \rput(2.7645311,1.17){\LARGE$5$}
		  \usefont{T1}{ppl}{m}{n}
		  \rput(3.3045313,0.57){\LARGE$7$}
		  \usefont{T1}{ppl}{m}{n}
		  \rput(2.8045313,-1.47){\LARGE$9$}
		  \usefont{T1}{ppl}{m}{n}
		  \rput(0.9945313,1.41){\LARGE$A$}
		  \usefont{T1}{ppl}{m}{n}
		  \rput(2.4945312,1.83){\LARGE$B$}
		  \usefont{T1}{ppl}{m}{n}
		  \rput(1.6145313,-0.95){\LARGE$12$}
		  \usefont{T1}{ppl}{m}{n}
		  \rput(3.1945312,-0.01){\LARGE$11$}
		  \usefont{T1}{ppl}{m}{n}
		  \rput(0.9845312,0.77){\LARGE$1$}
		  \end{pspicture} 
		  }

    \item
	\begin{enumerate}[noitemsep, label=\textbf{\roman*.} ]
	\item $12$
	\item $6$
	\item $5$
	
	\end{enumerate}
    

    \end{enumerate}
\end{enumerate}}
\end{solutions}


% \section{Union and intersection}
% \section{Probability identities}
% \section{Mutually exclusive events}
% \section{Complementary events}
\begin{exercises}{}
{
\begin{enumerate}[itemsep=6pt, label=\textbf{\arabic*}. ] 
  \item A box contains coloured blocks. The number of each colour is
    given in the following table.
    \begin{center}
      \begin{tabular}{|l|c|c|c|c|}
        \hline
        \textbf{Colour} & Purple & Orange & White & Pink \\ \hline
        \textbf{Number of blocks} & $24$ & $32$ & $41$ & $19$ \\ \hline
      \end{tabular}
    \end{center}
   A block is selected randomly. What is the probability that the block will be:
  \begin{enumerate}[noitemsep, label=\textbf{(\alph*)} ]
    \item purple
    \item purple or white
    \item pink and orange
    \item not orange?
    \end{enumerate}
  \item A small school has a class with children of various ages. The
    table gives the number of pupils of each age in the class.
    \begin{center}
      \begin{tabular}{|l|c|c|c|}
        \hline
               & $3$ years old & $4$ years old & $5$ years old \\\hline
        \textbf{male}   & $2$ & $7$ & $6$ \\\hline
        \textbf{female} & $6$ & $5$ & $4$ \\\hline
      \end{tabular}
    \end{center}
    If a pupil is selected at random what is the probability that the
    pupil will be:
    \begin{enumerate}[noitemsep, label=\textbf{(\alph*)} ]
    \item a female
    \item a $4$ year old male
    \item aged $3$ or $4$
    \item aged $3$ and $4$
    \item not $5$
    \item either $3$ or female?
    \end{enumerate}
  \item Fiona has $85$ labelled discs, which are numbered from $1$ to
    $85$. If a disc is selected at random what is the probability that
    the disc number:
    \begin{enumerate}[noitemsep, label=\textbf{(\alph*)} ]
    \setcounter{enumi}{10}
    \item ends with $5$
    \item is a multiple of $3$
    \item is a multiple of $6$
    \item is number $65$
    \item is not a multiple of $5$
    \item is a multiple of $4$ or $3$
    \item is a multiple of $2$ and $6$
    \item is number $1$?
    \end{enumerate}
 \end{enumerate}
% \practiceinfo
%   \begin{tabularx}{\textwidth}{XXXX}
%     (1.) AAA & (2.) AAA & (3.) AAA \\
%   \end{tabularx}
}
\end{exercises}


\begin{solutions}{}{
\begin{enumerate}[itemsep=5pt, label=\textbf{\arabic*}. ] 

\item %solution 1
    \begin{enumerate}[noitemsep, label=\textbf{(\alph*)} ]
    Before we answer the questions we first work out how many blocks there are in total. This gives us the sample space.\\
    $n(S) = 24 + 32 + 41 + 19$\\
	 $= 116$\\
    \item The probability that a block is purple is:\\
    $P(\text{purple}) = \frac{n(E)}{n(S)}\\
     P(\text{purple}) = \frac{24}{116}\\
     P(\text{purple}) = 0,21$

    \item The probability that a block is either purple or white is:\\
    $P(\text{purple} \cup \text{white}) = P(\text{purple}) + P(\text{white}) - P(\text{purple} \cap \text{white})\\
    = \frac{24}{116} + \frac{41}{116} - \frac{24}{116} \times \frac{41}{116}\\
    = 0,64$

    \item Since one block cannot be two colours the probabiBcc   lity of this event is $0$.

    \item We first work out the probability that a block is orange:\\
    $P(\text{orange}) = \frac{32}{116}
	       = 0,28$\\
    The probability that a block is not orange is:\\
    $P(\text{not orange}) = 1 - 0,28\\
		   = 0,72$
    \end{enumerate}
    
\item %solution 2
We calculate the total number of pupils at the school:\\
$6 + 2 + 5 + 7 + 4 + 6 = 30$
    \begin{enumerate}[noitemsep, label=\textbf{(\alph*)} ]
    \item The total number of female children is $6 + 5 + 4 = 15$.\\
    The probability of a randomly selected child being female is:\\
    $P(\text{female}) = \frac{n(E)}{n(S)}\\
     P(\text{female}) = \frac{15}{30}\\
     P(\text{female}) = 0,5$

    \item The probability of a randomly selected child being a 4 year old male is:\\
    $P(\text{male}) = \frac{7}{30}\\
	      = 0,23$

    \item There are $6 + 2 + 5 + 7 = 20$ children aged 3 or 4.\\
    The probability of a randomly selected child being either 3 or 4 is:\\
    $\frac{20}{30} = 0,67$

    \item A child cannot be both 3 and 4, so the probability is $0$.

    \item This is the same as a randomly selected child being either 3 or 4 and so is $0,67$.

    \item The probability of a child being either 3 or female is:\\
    $P(3 \cup \text{female}) = P(3) + P(\text{female}) - P(3 \cap \text{female})\\
		      = \frac{10}{30} + 0,5 - \frac{10}{30} \times \frac{15}{30}\\
		      = 0,67$
    \end{enumerate}
    
\item %solution 3
    \begin{enumerate}[noitemsep, label=\textbf{(\alph*)} ]
    \item The set of all discs ending with $5$ is: $\{5$; $15$; $25$; $35$; $45$; $55$; $65$; $75$; $85\}$. This has $9$ elements.\\
    \newline
    The probability of drawing a disc that ends with $5$ is:\\
    $P(5) = \frac{n(E)}{n(S)}\\
     P(5) = \frac{9}{85}\\
     P(5) = 0,11$

    \item The set of all discs that are multiples of $3$ is: $\{3$; $6$; $ 9$; $ 12$; $ 15$; $ 18$; $ 21$; $ 24$; $ 27$; $ 30$; $ 33$; $ 36$; $ 39$; $ 42$; $ 45$; $ 48$; $ 51$; $ 54$; $ 57$; $ 60$; $ 63$; $ 66$; $ 69$; $ 72$; $ 75$; $ 78$; $ 81$; $ 84\}$. This has $28$ elements.\\
    \newline
    The probability of drawing a disc that is a multiple of $3$ is:
    $P(3_m) = \frac{28}{85}
	    = 0,33$

    \item The set of all discs that are multiples of $6$ is: $\{6$; $ 12$; $ 18$; $ 24$; $ 30$; $ 36$; $ 42$; $ 48$; $ 54$; $ 60$; $ 66$; $ 72$; $ 78$; $ 84\}$. This set has $14$ elements.\\
    \newline
    The probability of drawing a disc that is a multiple of $6$ is:\\
    $P(6_m) = \frac{14}{85}\\
	    = 0,16$

    \item There is only one element in this set and so the probability of drawing $65$ is:\\
    $P(65) = \frac{1}{85}\\
	   = 0,01$

    \item The set of all discs that is a multiple of $5$ is: $\{5$; $ 10$; $ 15$; $ 20$; $ 25$; $ 30$; $ 35$; $ 40$; $ 45$; $ 50$; $ 55$; $ 60$; $ 65$; $ 70$; $ 75$; $ 80$; $ 85\}$. This set contains $17$ elements. Therefore the number of discs that are not multiples of $5$ is: $85 − 17 = 68$.\\
   
    The probability of drawing a disc that is not a multiple of $5$ is:\\
    $P(\text{not }5_m) = \frac{68}{85}\\
	       = 0,80$

    \item In part b, we worked out the probability for a disc that is a multiple of $3$. Now we work out the number of elements in the set of all discs that are multiples of $4$: $\{4$; $ 8$; $ 12$; $ 16$; $ 20$; $ 24$; $ 28$; $ 32$; $ 36$; $ 40$; $ 44$; $ 48$; $ 52$; $ 56$; $ 60$; $ 64$; $ 68$; $ 72$; $ 76$; $ 80$; $ 84\}$. This has $28$ elements.\\
    
    The probability that a disc is a multiple of either $3$ or $4$ is:\\
    $P(3_m \cup 4_m) = P(3_m) + P(4_m) - P(3_m \cap 4_m)\\
    = 0,33 + \frac{28}{85} - 0,33 \times \frac{28}{85}\\
    = 0,55$
    
    \item The set of all discs that are a multiples of $2$ and $6$ is the same as the set of all discs that are a multiple of $6$. Therefore the probability of drawing a disc that is both a multiple of $2$ and $6$ is:\\
    \newline
    $0,16$
    
    \item There is only $1$ element in this set and so the probability is $0,01$.
    \end{enumerate}

\end{enumerate}}
\end{solutions}


\begin{eocexercises}{}
  \begin{enumerate}[itemsep=5pt, label=\textbf{\arabic*}. ]
  \item A group of $45$ children were asked if they eat Frosties and/or
    Strawberry Pops. $31$ eat both and $6$ eat only Frosties. What is the
    probability that a child chosen at random will eat only Strawberry
    Pops?
  \item In a group of $42$ pupils, all but $3$ had a packet of chips
    or a Fanta or both. If $23$ had a packet of chips and $7$ of these
    also had a Fanta, what is the probability that one pupil chosen at
    random has:
    \begin{enumerate}[noitemsep, label=\textbf{(\alph*)} ]
    \item both chips and Fanta
    \item only Fanta?
    \end{enumerate}
  \item Use a Venn diagram to work out the following probabilities
    from a die being rolled:
    \begin{enumerate}[noitemsep, label=\textbf{(\alph*)} ]
    \item a multiple of $5$ and an odd number
    \item a number that is neither a multiple of $5$ nor an odd
      number
    \item a number which is not a multiple of $5$, but is odd
    \end{enumerate}
  \item A packet has yellow and pink sweets. The probability of taking
    out a pink sweet is $\frac{7}{12}$.What is the probability of taking out a yellow sweet?
  \item In a car park with $300$ cars, there are $190$ Opels. What is the
    probability that the first car to leave the car park is:
    \begin{enumerate}[noitemsep, label=\textbf{(\alph*)} ]
    \item an Opel
    \item not an Opel
    \end{enumerate}
  \item Tamara has $18$ loose socks in a drawer. Eight of these are
    orange and two are pink. Calculate the probability that the first
    sock taken out at random is:
    \begin{enumerate}[noitemsep, label=\textbf{(\alph*)} ]
    \item orange
    \item not orange
    \item pink
    \item not pink
    \item orange or pink
    \item neither orange nor pink
    \end{enumerate}
  \item A plate contains $9$ shortbread cookies, $4$ ginger biscuits,
    $11$ chocolate chip cookies and $18$ Jambos. If a biscuit is
    selected at random, what is the probability that:
    \begin{enumerate}[noitemsep, label=\textbf{(\alph*)} ]
    \item it is either a ginger biscuit of a Jambo
    \item it is not a shortbread cookie
    \end{enumerate}
  \item $280$ tickets were sold at a raffle. Ingrid bought $15$
    tickets. What is the probability that Ingrid:
    \begin{enumerate}[noitemsep, label=\textbf{(\alph*)} ]
    \item wins the prize
    \item does not win the prize
    \end{enumerate}
  \item The children in a nursery school were classified by hair and
    eye colour. $44$ had red hair and not brown eyes, $14$ had brown eyes
    and red hair, $5$ had brown eyes but not red hair and $40$ did not
    have brown eyes or red hair.
    \begin{enumerate}[noitemsep, label=\textbf{(\alph*)} ]
    \item how many children were in the school?
    \item What is the probability that a child chosen at random has:
      \begin{enumerate}
      \item brown eyes
      \item red hair
      \end{enumerate} 
    \item A child with brown eyes is chosen randomly. What is the
      probability that this child will have red hair?
    \end{enumerate}
  \item A jar has purple, blue and black sweets in it. The probability
    that a sweet chosen at random will be purple is $\frac{1}{7}$
    and the probability that it will be black is $\frac{3}{5}$.
    \begin{enumerate}[noitemsep, label=\textbf{(\alph*)} ]
    \item If I choose a sweet at random what
      is the probability that it will be:
      \begin{enumerate}
      \item purple or blue
      \item black
      \item purple
      \end{enumerate}
    \item If there are $70$ sweets in the jar how many purple ones are
      there?
    \item $\frac{2}{5}$ of the purple sweets in b) have streaks on
      them and the rest do not. How many purple sweets have streaks?
    \end{enumerate}
\item For each of the following, draw a Venn diagram to represent
    the situation and find an example to illustrate the situation.
    \begin{enumerate}[noitemsep, label=\textbf{(\alph*)} ]
    \item a sample space in which there are two events that are not
      mutually exclusive
    \item a sample space in which there are two events that are
      complementary
    \end{enumerate}
\item Use a Venn diagram to prove that the probability of either
    event $A$ or $B$ occurring is given by: ($A$ and $B$ are not
    exclusive)
    \[P(A \cup B) = P(A) + P(B) - P(A \cap B)\]
\item All the clubs are taken out of a pack of cards. The remaining
    cards are then shuffled and one card chosen. After being chosen,
    the card is replaced before the next card is chosen.
    \begin{enumerate}[noitemsep, label=\textbf{(\alph*)} ]
    \item What is the sample space?
    \item Find a set to represent the event, $P$, of drawing a picture
      card.
    \item Find a set for the event, $N$, of drawing a numbered card.
    \item Represent the above events in a Venn diagram.
    \item What description of the sets $P$ and $N$ is suitable?
      (Hint: Find any elements of $P$ in $N$ and of $N$ in $P$.)
    \end{enumerate}
  \end{enumerate}
% \practiceinfo
%   \begin{tabularx}{\textwidth}{XXXXX}
%     (1.) AAA & (2.) AAA & (3.) AAA& (4.) AAA& (5.) AAA\\
%     (6.) AAA& (7.) AAA& (8.) AAA& (9.) AAA& (10.) AAA\\
%     (11.) AAA& (12.) AAA& (13.) AAA \\
%   \end{tabularx}
\end{eocexercises}


 \begin{solutions}{}{
\begin{enumerate}[itemsep=5pt, label=\textbf{\arabic*}. ] 


\item %solution 1

$45($All$) - 6($only Frosties$) - 31 ($both$) = 8 ($only Strawberry Pops$)$

\item %solution 2
    \begin{enumerate}[noitemsep, label=\textbf{(\alph*)} ]
    \item $\frac{7}{42} = \frac{1}{6}$

    \item Since $42 - 3 = 39$ had at least one, and $23 + 7$ had a packet of chips, then $39 - 30 = 9$ only had Fanta.\\
    $\frac{9}{42} = \frac{3}{14}$
    \end{enumerate}

\item %solution 3
	    \scalebox{0.8} % Change this value to rescale the drawing.
	    {
	    \begin{pspicture}(0,-1.8767188)(3.62,1.9167187)
	    \pscircle[linewidth=0.04,dimen=outer](1.81,-0.06671875){1.81}
	    \pscircle[linewidth=0.04,dimen=outer](1.08,-0.03671875){0.94}
	    \pscircle[linewidth=0.04,dimen=outer](2.01,1.2132813){0.45}
	    \usefont{T1}{ppl}{m}{n}
	    \rput(3.1245313,1.7132813){\LARGE$S$}
	    \usefont{T1}{ppl}{m}{n}
	    \rput(1.0045313,0.5132812){\LARGE$2$}
	    \usefont{T1}{ppl}{m}{n}
	    \rput(0.6045312,0.03328125){\LARGE$4$}
	    \usefont{T1}{ppl}{m}{n}
	    \rput(1.3245312,-0.34671876){\LARGE$6$}
	    \usefont{T1}{ppl}{m}{n}
	    \rput(2.8445313,-0.50671875){\LARGE$3$}
	    \usefont{T1}{ppl}{m}{n}
	    \rput(1.9845313,1.1732812){\LARGE$5$}
	    \usefont{T1}{ppl}{m}{n}
	    \rput(2.7645311,0.03328125){\LARGE$1$}
	    \pscircle[linewidth=0.04,dimen=outer](2.81,-0.20671874){0.69}
	    \end{pspicture}
	    }\\
    Multiples of $5 :5$\\
    Odd number: $1, 3, 5$\\
    Neither: $2, 4, 6$\\
    Both: $5$
    \begin{enumerate}[noitemsep, label=\textbf{(\alph*)} ]
    \item $\frac{1}{6}$
    \item $\frac{3}{6} = \frac{1}{2}$
    \item $\frac{2}{6} = \frac{1}{3}$
    \end{enumerate}
\item %solution 4\\
$1 - \frac{7}{12} = \frac{5}{12}$
\item %solution 5
    \begin{enumerate}[noitemsep, label=\textbf{(\alph*)} ]
    \item $\frac{190}{300} = \frac{19}{30}$
    \item $1 - \frac{19}{30} = \frac{11}{30}$
    \end{enumerate}
\item %solution 6
    \begin{enumerate}[noitemsep, label=\textbf{(\alph*)} ]
    \item $\frac{8}{18} = \frac{4}{9}$
    \item $1 - \frac{4}{9} = \frac{5}{9}$
    \item $\frac{2}{18} = \frac{1}{9}$
    \item $1 - \frac{1}{9} = \frac{8}{9}$
    \item $\frac{1}{9} + \frac{4}{9} = \frac{5}{9}$
    \item $1 - \frac{5}{9} = \frac{4}{9}$
    \end{enumerate}
\item %solution 7
    \begin{enumerate}[noitemsep, label=\textbf{(\alph*)} ]
    \item Total number of biscuits is $9 + 4 + 11 + 18 = 42$\\
    $\frac{4}{42} + \frac{18}{42} = \frac{22}{42}$\\
    $\frac{11}{21}$
    \item $1 - \frac{9}{42} = 1 - \frac{3}{14}$\\
    $\frac{11}{14}$
    \end{enumerate}
\item %solution 8
    \begin{enumerate}[noitemsep, label=\textbf{(\alph*)} ]
    \item $\frac{15}{280} = \frac{3}{56}$
    \item $1 - \frac{3}{56} = \frac{53}{56}$
    \end{enumerate}
\item %solution 9
    \begin{enumerate}[noitemsep, label=\textbf{(\alph*)} ]
    \item All $4$ groups are mutually exclusive, so total number of children is $44 + 14 + 5 + 40 = 103$.
    \item
	\begin{enumerate}[itemsep=1pt,  label=\textbf{\roman*}. ]
	\item $\frac{19}{103}$
	\item $\frac{58}{103}$
	\end{enumerate}
    \item $\frac{14}{(14 + 5)} = \frac{14}{19}$
    \end{enumerate}
\item %solution 10
    \begin{enumerate}[noitemsep, label=\textbf{(\alph*)} ]
    \item 
	\begin{enumerate}[itemsep=1pt,  label=\textbf{\roman*}. ]
	\item Same as not black: $1 - \frac{3}{5} = \frac{2}{5}$
	\item $\frac{3}{5}$
	\item $\frac{1}{7}$
	\end{enumerate}
    \item $\frac{1}{7} \times 70 = 10$
    \item $10 \times \frac{2}{5} = 4$
    \end{enumerate}
\item %solution 11
    \begin{enumerate}[noitemsep, label=\textbf{(\alph*)} ]
    \item
    			\scalebox{0.8} % Change this value to rescale the drawing.
			{
			\begin{pspicture}(0,-1.8767188)(3.62,1.9167187)
			\pscircle[linewidth=0.04,dimen=outer](1.81,-0.06671875){1.81}
			\pscircle[linewidth=0.04,dimen=outer](1.2,-0.07671875){0.94}
			\usefont{T1}{ppl}{m}{n}
			\rput(3.1245313,1.7132813){\LARGE$S$}
			\pscircle[linewidth=0.04,dimen=outer](2.45,-0.06671875){0.87}
			\end{pspicture}
			}
    \item
			\scalebox{0.8} % Change this value to rescale the drawing.
			{
			\begin{pspicture}(0,-1.8767188)(3.62,1.9167187)
			\pscircle[linewidth=0.04,dimen=outer](1.81,-0.06671875){1.81}
			\usefont{T1}{ppl}{m}{n}
			\rput(3.1245313,1.7132813){\LARGE$S$}
			\psline[linewidth=0.04cm](2.78,1.4032812)(0.88,-1.5767188)
			\end{pspicture}
			}
    \end{enumerate}

\item %solution 12

\scalebox{0.8} % Change this value to rescale the drawing.
{
    \begin{pspicture}(-2,-2)(3.5,2)
    \usefont{T1}{ppl}{m}{n}
    \rput(0.0,0.0){\LARGE$A$ not $B$}
    \rput(1.0,2.3){\LARGE$A$}
    \rput(3.0,2.3){\LARGE$B$}
    \pscircle[linewidth=0.04,dimen=outer](1.0,0.0){2.0}
    \usefont{T1}{ppl}{m}{n}
    \rput(4.0,0.0){\LARGE$B$ not $A$}
    \pscircle[linewidth=0.04,dimen=outer](3,0.0){2.0}
    \usefont{T1}{ppl}{m}{n}
    \rput(2.0,0.0){\LARGE$A$ and $B$}
    \end{pspicture}
}
\item 
    \begin{enumerate}[noitemsep, label=\textbf{(\alph*)} ]
    \item $\{$deck of cards without clubs$\}$
    \item $P = \{J$; $Q$; $K$ of hearts diamonds or spades$\}$
    \item $N = \{A$; $2$; $ 3$; $ 4$; $ 5$; $ 6$; $ 7$; $ 8$; $ 9$; $ 10$ of hearts, diamonds or spades$\}$
    \item $\{$Mutually exclusive and complementary$\}$
    \item Complementary
    \end{enumerate}
\end{enumerate}}
\end{solutions}


