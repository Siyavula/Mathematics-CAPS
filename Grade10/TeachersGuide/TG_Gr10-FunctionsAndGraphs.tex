\chapter{Functions}
\section{Functions in the real world}
\begin{exercises}{}
{
\begin{enumerate}[noitemsep, label=\textbf{\arabic*}. ] 
\item Write the following in set notation:
\begin{enumerate}[noitemsep, label=\textbf{(\alph*)} ] 
 \item $(-\infty; 7]$
\item $[13; 4)$
\item $(35; \infty)$
\item $[\frac{3}{4}; 21)$
\item $[-\frac{1}{2}; \frac{1}{2}]$
\item $(-\sqrt{3}; \infty)$
\end{enumerate}
\item Write the following in interval notation:
\begin{enumerate}[noitemsep, label=\textbf{(\alph*)} ] 
 \item $\{p: p \in \mathbb{R},~ p \leq 6\}$
 \item $\{k: k \in \mathbb{R},~ -5 < k < 5\}$
 \item $\{x: x \in \mathbb{R},~ x > \frac{1}{5}\}$
 \item $\{z: z \in \mathbb{R},~ 21 \leq z < 41\}$
\end{enumerate}
\end{enumerate}
\practiceinfo
\par 
\par \begin{tabular}[h]{cccccc}
(1.) lTN  &  (2.) lTR  &  (3.) lTp  &  (4.) lTn  &  (5.) lTy  &  (6.) lx8  &  (7.) lxX  &  (8.) lx9  & \end{tabular}
} 
\end{exercises}


 \begin{solutions}{}{
\begin{enumerate}[itemsep=5pt, label=\textbf{\arabic*}. ] 


\item solution 1
\item solution 2

\end{enumerate}}
\end{solutions}


\begin{exercises}{}
{
\begin{enumerate}[noitemsep, label=\textbf{\arabic*}. ] 
\item List the $x$ and $y$-intercepts for the following straight line graphs. Indicate whether the graph is increasing or decreasing:
      \begin{enumerate}[noitemsep, label=\textbf{(\alph*)} ] 
      \item $y=x+1$
      \item $y=x-1$
      \item $h(x)=2x-1$
      \item $y+3x=1$
      \item $3y-2x=6$
      \item$k(x)=-3$
      \item $x=3y$
      \item $\frac{x}{2} - \frac{y}{3} = 1$
      \end{enumerate}
\item For the functions in the diagram below, give the equation and the domain and range:
  \begin{enumerate}[noitemsep, label=\textbf{(\alph*)} ]  
  \item $a(x)$
  \item $b(x)$
  \item $p(x)$
  \item $d(x)$
  \end{enumerate} 
\setcounter{subfigure}{0}
\begin{figure}[H]
\begin{center}
\scalebox{1} % Change this value to rescale the drawing.
{
\begin{pspicture}(0,-4.1467185)(9.493593,4.1867185)
\rput(4.0,-0.1467186){\psaxes[linewidth=0.03,arrowsize=0.05291667cm 2.0,arrowlength=1.4,arrowinset=0.4,tickstyle=bottom,labels=none,ticks=none,ticksize=0.08cm]{<->}(0,0)(-4,-4)(4,4)}
\psline[linewidth=0.04cm](2.74,1.7332813)(7.72,-0.9467186)
\usefont{T1}{ptm}{m}{n}
\rput(4.1768746,3.9832811){$y$}
\usefont{T1}{ptm}{m}{n}
\rput(8.234531,-0.036718614){$x$}
\usefont{T1}{ptm}{m}{n}
\rput(4.4442186,1.3432813){$(0;3)$}
\usefont{T1}{ptm}{m}{n}
\rput(6.9642186,0.043281388){$(4;0)$}
\usefont{T1}{ptm}{m}{n}
\rput(8.274531,-0.9){$a(x)$}
\psline[linewidth=0.04cm](3.2,-3.6332815)(8.42,2.3067186)
\usefont{T1}{ptm}{m}{n}
\rput(4.6742187,-2.8167186){$(0;-6)$}
\usefont{T1}{ptm}{m}{n}
\rput(8.869062,2.4967186){$b(x)$}
\psline[linewidth=0.04cm](0.96,1.0667186)(7.86,1.0467187)
\usefont{T1}{ptm}{m}{n}
\rput(8.274531,1.1567186){$p(x)$}
\psline[linewidth=0.04cm](7.42,-2.0332813)(1.1,1.4667186)
\usefont{T1}{ptm}{m}{n}
\rput(7.8745313,-2.1032813){$d(x)$}
\usefont{T1}{ptm}{m}{n}
\rput(3.7745314,-0.39999998){$0$}
\psline[linewidth=0.04cm,arrowsize=0.113cm 4.0,arrowlength=1.4,arrowinset=0.4]{>>-}(3.16,0.30671862)(2.92,0.4467186)
\psline[linewidth=0.04cm,arrowsize=0.113cm 4.0,arrowlength=1.4,arrowinset=0.4]{>>-}(5.14,0.4067186)(4.9,0.5467186)
\end{pspicture} 
}
\end{center}
\end{figure}  
\item Sketch the following functions on the same set of axes, using the dual intercept method. Clearly indicate the intercepts and the point of intersection of the two graphs: $x+2y-5=0$ and $3x-y-1=0$
\item On the same set of axes, draw the graphs of $f(x)=3-3x$ and $g(x)=\frac{1}{3}x+1$ using the gradient--intercept method.
\end{enumerate}
\practiceinfo
\par 
\par \begin{tabular}[h]{cccccc}
(1a-h.) AAA  &  (2a-d.) AAA &  (3.) AAA & (4.) AAA \end{tabular}
}
\end{exercises}


 \begin{solutions}{}{
\begin{enumerate}[itemsep=5pt, label=\textbf{\arabic*}. ] 


\item solution 1
\item solution 2
\item solution 3
\item solution 4

\end{enumerate}}
\end{solutions}


\section{Quadratic functions of the form $f(x)=a{x}^{2}+q$}
\begin{exercises}{}
{
\begin{enumerate}[noitemsep, label=\textbf{\arabic*}. ] 
\item Show that if $a<0$ the range of $f(x)=ax^{2}+q$ is $\{f(x):f(x) \leq q \}$.
\item Draw the graph of the function $y=-x^{2}+4$ showing all intercepts with the axes.
\item Two parabolas are drawn: $g:y=ax^{2}+p$ and $h:y=bx^{2}+q$.
\begin{center}
\begin{pspicture}(-5,-5)(5,1)
\psset{yunit=0.2,xunit=0.5}
\psaxes[arrows=<->,dx=2,Dx=2,dy=2,Dy=2, labels=none, ticks=none](0,0)(-10,-15)(10,28)
\psplot[plotstyle=curve,arrows=<->]{-5.5}{5.5}{x 2 exp  9 sub}
\psplot[plotstyle=curve,arrows=<->]{-5.5}{5.5}{x 2 exp 1 mul neg 23 add}
\rput(0.6,28){$y$}
\rput(0.6,-10){$-9$}
\rput(-5.5,7.5){$(-4;7)$} 
\rput(5.1,7.5){$(4;7)$}
\rput(5.3,-0.85){$3$}
\rput(5.6,17){$g$}
\rput(5.6,-4){$h$}
\rput(0.5, 24){$23$}
\rput (10.4, 0.2){$x$}
\rput(-0.5,-0.85){$0$}
\end{pspicture}
\end{center}
\begin{enumerate}[noitemsep, label=\textbf{(\alph*)} ] 
    \item Find the values of $a$ and $p$.
    \item Find the values of $b$ and $q$.
    \item Find the values of $x$ for which $g(x)\ge h(x)$.
    \item For what values of $x$ is $g$ increasing?
\end{enumerate}
\end{enumerate}
\practiceinfo
\par 
\par \begin{tabular}[h]{cccccc}
(1.) lxD &  (2.) lxW & (3a-d) lxZ \end{tabular}
}
\end{exercises}   


 \begin{solutions}{}{
\begin{enumerate}[itemsep=5pt, label=\textbf{\arabic*}. ] 


\item solution 1
\item solution 2
\item solution 3

\end{enumerate}}
\end{solutions}


\section{Hyperbolic functions of the form $f(x)=\frac{a}{x}+q$}
\begin{exercises}{}
{
   \begin{enumerate}[noitemsep, label=\textbf{\arabic*}. ] 
\item Using graph paper, draw the graph of $xy=-6$.
    \begin{enumerate}[noitemsep, label=\textbf{(\alph*)} ] 
    \item Does the point $(-2; 3)$ lie on the graph ? Give a reason for your answer.
    \item If the $x$-value of a point on the drawn graph is $0,25$ what is the corresponding $y$-value?
    \item What happens to the $y$-values as the $x$-values become very large?
\item Calculate the shortest distance from the origin to the graph.
\item Give the equations of the asymptotes.
    \item With the line $y=-x$ as line of symmetry, what is the point symmetrical to $(-2; 3)$?
    \end{enumerate}
\item Draw the graph of $h(x)=\frac{8}{x}$.
    \begin{enumerate}[noitemsep, label=\textbf{(\alph*)} ] 
    \item How would the graph $g(x)=\frac{8}{x}+3$ compare with that of $h(x)=\frac{8}{x}$? Explain your answer fully.
    \item Draw the graph of $y=\frac{8}{x}+3$ on the same set of axes, showing asymptotes, axes of symmetry and the coordinates of one point on the graph.
    \end{enumerate}
 \end{enumerate}
\practiceinfo
\par 
\par \begin{tabular}[h]{cccccc}
(1.) lxB  &  (2.) lxK  & \end{tabular}
}
\end{exercises}


 \begin{solutions}{}{
\begin{enumerate}[itemsep=5pt, label=\textbf{\arabic*}. ] 


\item solution 1
\item solution 2

\end{enumerate}}
\end{solutions}


\begin{exercises}{ }
 {
 \begin{enumerate}[noitemsep, label=\textbf{\arabic*}. ] 
\item Draw the graphs of $y=2^{x}$ and $y=(\frac{1}{2})^{x}$ on the same set of axes.
 \begin{enumerate}[noitemsep, label=\textbf{(\alph*)} ]
\item Is the $x$-axis an asymptote or an axis of symmetry to both graphs? Explain your answer.
\item Which graph is represented by the equation $y=2^{-x}$ ? Explain your answer.
\item Solve the equation $2^{x}=(\frac{1}{2})^{x}$ graphically and check that your answer is correct by using substitution.
\end{enumerate}
\item The curve of the exponential function $f$ in the accompanying diagram cuts the $y$-axis at the point $A(0; 1)$ and passes through the point $B(2; 9)$.
\begin{center}
\begin{pspicture}(-3,-1)(4,4)
\psset{yunit=0.75,xunit=0.75}
\psaxes[arrows=<->](0,0)(-5,-1)(5,10)
\psplot[plotstyle=curve,arrows=<->]{-2}{2.1}{3 x exp}
\psdots(0,1)(2,9)
\rput(1,1){$A(0;1)$}
\rput(3,9){$B(2;9)$}
\rput(5.2,0.2){$x$}
\rput(0.2,10.2){$y$}
\rput(-0.3,-0.3){$0$}
\end{pspicture}
\end{center}
 \begin{enumerate}[noitemsep, label=\textbf{(\alph*)} ]
\item Determine the equation of the function $f$.
\item Determine the equation of $h$, the reflection of $f$ in the $x$-axis.
\item Determine the range of $h$.
\item Determine the equation of $g$, the reflection of $g$ in the $y$-axis.
\item Determine the equation of $j$ if $j$ is a vertical stretch of $f$ by $+2$ units.
\item Determine the equation of $k$ if $k$ is a vertical shift of $f$ by $-3$ units.
\end{enumerate}
\end{enumerate}
\practiceinfo
\par 
\par \begin{tabular}[h]{cccccc}
(1.) lxk  &  (2.) lx0  & \end{tabular}
}
\end{exercises}


 \begin{solutions}{}{
\begin{enumerate}[itemsep=5pt, label=\textbf{\arabic*}. ] 


\item solution 1
\item solution 2

\end{enumerate}}
\end{solutions}


\section{Trigonometric functions}
\subsection{Sine functions of the form $f(\theta)=a~sin~\theta+q$}
\subsection{Cosine functions of the form $f(\theta)=a\cos\theta +q$}
\subsection{Tangent functions of the form $f(\theta)=a\tan\theta+q$}
\begin{exercises}{}
 {
\begin{enumerate}[noitemsep, label=\textbf{\arabic*}. ] 
\item Using your knowledge of the effects of $a$ and $q$, sketch each of the following graphs, without using a table of values, for $\theta \in [~{0}^{\circ };{360}^{\circ }]$.
 \begin{enumerate}[noitemsep, label=\textbf{(\alph*)} ]
\item $y=2\sin\theta $
\item $y=-4\cos\theta $
\item $y=-2\cos\theta +1$
\item $y=\sin\theta -3$
\item $y=\tan\theta -2$\item $y=2\cos\theta -1$
\end{enumerate}
\item Give the equations of each of the following graphs:
 \begin{enumerate}[noitemsep, label=\textbf{(\alph*)} ]
\item
\begin{pspicture}(-2.5,-2)(5,2)
\psset{yunit=0.25}
\psaxes[Dx=180, dx=2, Dy=2, dy=4]{<->}(0,0)(-0.5,-5.1)(4.5,5.1)
\psplot[xunit=0.0111, plotpoints=500, arrows=->]{0}{360}{x cos -4 mul }
\uput[d](5.4,0.1){$x$ (in degrees)}
\uput[r](0,5.1){$y$}
\rput(-0.2,-0.7){$0$}
\end{pspicture}
\item
\begin{pspicture}(-0.3,-2)(5,2)
\psset{yunit=0.25}
\psaxes[Dx=90, dx=1, Dy=2, dy=4]{<->}(0,0)(-0.5,-5.1)(4.5,5.1)
\psplot[xunit=0.0111, plotpoints=500, arrows=->]{0}{360}{x sin 1 add 2 mul}
\uput[d](5.4,0.1){$x$ (in degrees)}
\uput[r](0,5.1){$y$}
\rput(-0.2,-0.7){$0$}
\end{pspicture}
\end{enumerate}
\end{enumerate}
\practiceinfo
\par 
\par \begin{tabular}[h]{cccccc}
(1a-f.) la8  &  (2a-b.) la0  & \end{tabular}
}
\end{exercises}


 \begin{solutions}{}{
\begin{enumerate}[itemsep=5pt, label=\textbf{\arabic*}. ] 


\item solution 1
\item solution 2

\end{enumerate}}
\end{solutions}


\section{Interpretation of graphs}
\begin{eocexercises}{}
  \begin{enumerate}[itemsep=9pt, label=\textbf{\arabic*}. ] 
  \item Sketch the graphs of the following: 
    \begin{enumerate}[noitemsep, label=\textbf{(\alph*)} ]
    \item $y=2x+4$ 
    \item $y-3x=0$ 
    \item $2y=4-x$
    \end{enumerate}
  \item Sketch the following functions: 
    \begin{enumerate}[noitemsep, label=\textbf{(\alph*)} ] % \setcounter{enumi}{3} 
    \item $y=x^{2}+3$ 
    \item $y=\frac{1}{2}x^{2}+4$
    \item $y=2x^{2}-4$
    \end{enumerate}
  \item Sketch the following functions and identify the asymptotes: 
    \begin{enumerate}[noitemsep, label=\textbf{(\alph*)} ]  % \setcounter{enumi}{6} 
    \item $y=3^{x}+2$ 
    \item $y=-4 \times 2^{x}$ 
    \item $y=\Big(\dfrac{1}{3}\Big)^{x}-2$ 
    \end{enumerate}
  \item Sketch the following functions and identify the asymptotes: 
    \begin{enumerate}[noitemsep, label=\textbf{(\alph*)} ] % \setcounter{enumi}{9} 
    \item $y=\frac{3}{x}+4$ 
    \item $y=\frac{1}{x}$ 
    \item $y=\frac{2}{x}-2$ 
    \end{enumerate}
  \item Determine whether the following statements are true or false. If the statement is false, give reasons why:
    \begin{enumerate}[noitemsep, label=\textbf{(\alph*)} ] % \setcounter{enumi}{12} 
    \item The given or chosen $y$-value is known as the independent variable.
    \item A graph is said to be congruent if there are no breaks in the graph.
    \item Functions of the form $y=ax+q$ are straight lines.
    \item Functions of the form $y=\frac{a}{x}+q$ are exponential functions.
    \item  An asymptote is a straight line which a graph will intersect at least once.
    \item Given a function of the form $y=ax+q$, to find the $y$-intercept let $x=0$ and solve for $y$.
    \end{enumerate}
  \item Given the functions $f(x)=2{x}^{2}-5$ and $g(x)=-2x+6$
    \begin{enumerate}[noitemsep, label=\textbf{(\alph*)} ] % \setcounter{enumi}{18} 
    \item Draw $f$ and $g$ on the same set of axes.
    \item Calculate the points of intersection of $f$ and $g$.
    \item Use your graphs and the points of intersection to solve for $x$ when:
      \begin{enumerate}[noitemsep, label=\textbf{\roman*}. ] % \setcounter{enumi}{20} 
      \item $f(x)>0$
      \item $\dfrac{f(x)}{g(x)}\le 0$
      \end{enumerate}
    \item Give the equation of the reflection of $f$ in the $x$-axis.
    \end{enumerate}
  \item After a ball is dropped, the rebound height of each bounce
    decreases. The equation $y=5{(0,8)}^{x}$ shows the relationship
    between the number of bounces $x$ and the height of the bounce $y$
    for a certain ball.  What is the approximate height of the fifth
    bounce of this ball to the nearest tenth of a unit?
  \item Mark had $15$ coins in R$~5$ and R$~2$ pieces. He had $3$ more
    R$~2$ coins than R$~5$ coins. He wrote a system of equations to
    represent this situation, letting $x$ represent the number of R$~5$
    coins and $y$ represent the number of R$~2$ coins. Then he solved
    the system by graphing.
    \begin{enumerate}[noitemsep, label=\textbf{(\alph*)} ] % \setcounter{enumi}{24} 
    \item Write down the system of equations.
    \item Draw their graphs on the same set of axes.
    \item Use your sketch to determine how many R$~5$ and R$~2$ pieces Mark had.
    \end{enumerate}
  \item Sketch graphs of the following trigonometric functions for
    $\theta \in[~0^{\circ};360^{\circ}]$. Show intercepts and
    asymptotes.
    \begin{enumerate}[noitemsep, label=\textbf{(\alph*)} ]  % \setcounter{enumi}{27} 
    \item $y=-4\cos\theta$
    \item $y=\sin\theta -2$
    \item $y=-2\sin\theta +1$
    \item $y=\tan\theta+2$
    \item $y=\dfrac{\cos\theta}{2}$
    \end{enumerate}
  \item Given the general equations in this chapter, determine the
    specific equations for each of the following graphs:\vspace{20pt}\\
    \begin{center}
      \begin{table}[H]
        \begin{tabular}{|m{6cm}|m{6cm}|}
          \hline
          \begin{center}
            \scalebox{0.8}{
              \begin{pspicture}(-5,-5)(5,1)
                \psset{yunit=0.5,xunit=0.5}
                \psaxes[arrows=<->, labels=none, ticks=none](0,0)(-6,-7)(6,6)
                \psline[linewidth=0.02, linestyle=dashed](-2,0)(-2,-6)
                \psline[linewidth=0.02, linestyle=dashed](-0,-6)(-2,-6)
                \psplot[plotstyle=curve,arrows=<->]{-2.2}{2}{x 3 mul}
                \rput(-5,5){\textbf{(a)}}
                \psdots(-2,-6)
                \rput(0.3, 6.3){$y$}
                \rput(6.2, 0.2){$x$}
                \rput(-0.37,-0.3){$0$}
                \rput(-3.5,-6){$(-2;-6)$}
              \end{pspicture}
            }
          \end{center}
          &
          \begin{center}
            \scalebox{0.8}{
              \begin{pspicture}(-5,-5)(5,1)
                \psset{yunit=0.5,xunit=0.5}
                \psaxes[arrows=<->, labels=none, ticks=none](0,0)(-5,-5)(5,5)
                \psplot[plotstyle=curve,arrows=<->]{-1.7}{1.7}{x 2 exp -2 mul 3 add}
                \rput(-5,5){\textbf{(b)}}
                \psdots(1,1)
                \rput(0.3, 5.3){$y$}
                \rput(5.2, 0.2){$x$}
                \rput(-0.37,-0.3){$0$}
                \rput(1.5,1.5){$(1;1)$}
            \end{pspicture}}
          \end{center}
          \\ \hline
          \begin{center}
            \scalebox{0.8}{
              \begin{pspicture}(-5,-5)(5,1)
                \psset{yunit=0.5,xunit=0.5}
                \psaxes[arrows=<->, labels=none, ticks=none](0,0)(-5,-5)(5,5)
                \psplot[plotstyle=curve,arrows=<->]{-4.5}{-0.6}{x -1 exp -3 mul}
                \psplot[plotstyle=curve,arrows=<->]{0.6}{4.5}{x -1 exp -3 mul }
                \psdots(3,-1)
                \rput(-5,5){\textbf{(c)}}
                \rput(0.3, 5.3){$y$}
                \rput(5.2, 0.2){$x$}
                \rput(-0.37,-0.3){$0$}
                \rput(3.6,-1.5){$(3;-1)$}
            \end{pspicture}}
          \end{center}
          &
          \begin{center}
            \scalebox{0.8}{
              \begin{pspicture}(-5,-5)(5,1)
                \psset{yunit=0.5,xunit=0.5}
                \psaxes[arrows=<->, labels=none, ticks=none](0,0)(-6,-2)(6,7)
                \psplot[plotstyle=curve,arrows=<->]{-3.5}{5}{x 2 add}
                \rput(-5,7){\textbf{(d)}}
                \psdots(0,2)(4,6)
                \rput(0.3, 7.3){$y$}
                \rput(6.2, 0.2){$x$}
                \rput(-0.37,-0.3){$0$}
                \rput(1,2){$(0;2)$}
                \rput(5,6){$(4;6)$}
            \end{pspicture}}
          \end{center}
          \\ \hline
          \begin{center}
            \scalebox{0.8}{
              \begin{pspicture}(-6,-5)(5,6)
                \psset{yunit=0.4, xunit=1}
                \psset{xunit=0.01111}
                \psaxes[dx=30,Dx=30, labels=none, ticks=none]{->}(0,0)(-45,-4.5)(400,6.5)
                \psplot[plotstyle=curve, plotpoints=300, linewidth=1pt, arrows=->]{0}{360}{x sin 5 mul 1 add}  
                \rput(415, 0.2){$x$}
                \rput(15, 6.5){$y$}
                \rput(-90,5){\textbf{(e)}}
                \psline[linewidth=0.02, linestyle=dashed](0,6)(360,6)
                \psline[linewidth=0.02, linestyle=dashed](0,1)(360,1)
                \psline[linewidth=0.02, linestyle=dashed](0,-4)(360,-4)
                \rput(-30,6){$6$}
                \rput(-30,1){$1$}
                \rput(-30,-4){$-4$}
                \psdots(195,0)(350,0)
                \rput(160,-0.8){$180^{\circ}$}
                \rput(340,-0.8){$360^{\circ}$}
                \rput(-20,-0.4){$0$}
              \end{pspicture}
            }
          \end{center}
          &
          \begin{center}
            \scalebox{0.8}{
              \begin{pspicture}(-5,-5)(5,1)
                \psset{yunit=0.5,xunit=0.5}
                \psaxes[arrows=<->, labels=none, ticks=none](0,0)(-5,-3)(5,5)
                \psplot[plotstyle=curve,arrows=<->]{-4}{0.8}{2 x exp 2 mul 1 add}
                \psdots(0,3)
                \psline[linewidth=0.02,linestyle=dashed](-5,1)(5,1)
                \rput(-5,5){\textbf{(f)}}
                \rput(0.3, 5.3){$y$}
                \rput(5.2, 0.2){$x$}
                \rput(-0.37,-0.3){$0$}
                \rput(6,1){$y=1$}
                \rput(0.8,3){$(0;3)$}
            \end{pspicture}}
          \end{center}
          \\ \hline
          \begin{center}
            \scalebox{0.8}{
              \begin{pspicture}(0,-2.5)(4,2)
                \psaxes[Dx=180, dx=2, Dy=1, dy=0.5, labels=none, ticks=x]{<->}(0,0)(-1,-2)(4.5,2)
                \rput(-1,3){\textbf{(g)}}
                \psline[linewidth=0.02,linestyle=dashed](1,-2.5)(1,2)
                \psline[linewidth=0.02,linestyle=dashed](3,-2.5)(3,2)
                \psline[linewidth=0.02,linestyle=dashed](0,-1)(4.5,-1)
                \psplot[xunit=0.0111,yunit=0.5, plotpoints=500, arrows=->]{0}{70}{x sin x cos div -1 mul 2 sub}
                \psplot[xunit=0.0111,yunit=0.5,plotpoints=500, arrows=<->]{100}{250}{x sin x cos div -1 mul 2 sub}
                \psplot[xunit=0.0111,yunit=0.5,plotpoints=500, arrows=<-]{280}{360}{x sin x cos div -1 mul 2 sub}
                \rput(0.2,2.3){$y$}
                \rput(4.5,0.2){$x$}
                \rput(-0.5,-1){$-2$}
  \rput(2,-0.4){$180^{\circ}$}
                \rput(4,-0.4){$360^{\circ}$}
                \psdots(1.5,-0.5)
                \rput(1.65,-0.8){\footnotesize$(135^{\circ};-1)$}
              \end{pspicture}
            }
          \end{center}
          \\ \hline
        \end{tabular}
      \end{table}
    \end{center}
  \item $y=2^x$ and $y=-2^x$ are sketched below. Answer the questions that follow:
\begin{center}
    \scalebox{0.9}{
      \begin{pspicture}(-5,-5)(5,1)
        \psset{yunit=1.3,xunit=1.3}
        \psaxes[arrows=<->, labels=none, ticks=none](0,0)(-2.5,-2.5)(2,2.5)
        \psplot[plotstyle=curve,arrows=<->]{-1.7}{0.6}{2 x exp}
        \psplot[plotstyle=curve,arrows=<->]{-1.7}{0.6}{2 x exp -1 mul}
        \psdots(0,1)(0,-1)(-1,0.5)(-1,-0.5)
        \rput(0.2, 2.6){$y$}
        \rput(2.1, 0.2){$x$}
        \rput(-0.2,-0.2){$0$}
        \rput(0.3,1){$M$}
        \rput(0.3,-1){$N$}
        \rput(-1,0.8){$P$}
        \rput(-1,-0.8){$Q$}
        \rput(-1.2,0.15){$R$}
        \psline[linewidth=0.01,linestyle=dashed, dash=0.10cm 0.10cm](-1,0.5)(-1,-0.5)
        \psline[linewidth=0.01](-1,0.2)(-0.8,0.2)
        \psline[linewidth=0.01](-0.8,0.2)(-0.8,0)
    \end{pspicture}}
\end{center}
    \\
    \begin{enumerate}[noitemsep, label=\textbf{(\alph*)} ]
    \item Calculate the coordinates of $M$ and $N$.
    \item Calculate the length of $MN$.
    \item Calculate length $PQ$ if $OR=1$ unit.
    \item Give the equation of $y=2^x$ reflected about the $y$-axis.
    \item Give the range of both graphs.
    \end{enumerate}
  \item$f(x)=4^x$ and $g(x)=4x^2+q$ are sketched below. The points $A(0;1)$, $B(1;4)$ and$C$ are given. Answer the questions that follow:
  \begin{center}  
    \scalebox{1}{
      \begin{pspicture}(-5,-5)(5,1)
        \psset{yunit=0.5,xunit=1}
        \psaxes[arrows=<->, labels=none, ticks=none](0,0)(-2.5,-5)(2,5)
        \psplot[plotstyle=curve,arrows=<->]{-1}{1.2}{4 x exp}
        \psplot[plotstyle=curve,arrows=<->]{-1.2}{1.2}{x 2 exp -4 mul 1 add}
        \psdots(0,1)(1,4)(1,-3)
        \rput(0.2, 5.2){$y$}
        \rput(2.2, 0.1){$x$}
        \rput(-0.3,1.3){$A$}
        \rput(1.2,4){$B$}
        \rput(1.3,-3){$C$}
        \rput(-0.2,-0.4){$0$}
        \rput(2,5){$f(x)=4^x$}
        \rput(2.5,-4){$g(x)=-4x^2+q$}
        \psline[linewidth=0.01,linestyle=dashed, dash=0.10cm 0.10cm](1,4)(1,-3)
        \psline[linewidth=0.02](1,0.4)(1.2,0.4)
        \psline[linewidth=0.02](1.2,0.4)(1.2,0)
    \end{pspicture}}
\end{center}
    \\
    \begin{enumerate}[noitemsep, label=\textbf{(\alph*)} ]
    \item Determine the value of $q$.
    \item Calculate the length of $BC$.
    \item Give the equation of $f(x)$ reflected about the $x$-axis.
    \item Give the equation of $f(x)$ shifted vertically upwards by $1$ unit.
    \item Give the equation of the asymptote of $f(x)$.
\item Give the ranges of $f(x)$ and $g(x)$.
    \end{enumerate}
  \item Sketch the graphs $h(x)=x^2-4$ and $k(x)=-x^2+4$ on the same set of axes and answer the questions that follow: 
    \begin{enumerate}[noitemsep, label=\textbf{(\alph*)} ]
    \item Describe the relationship between $h$ and $k$.
    \item Give the equation of $k(x)$ reflected about the line $y=4$.
    \item Give the domain and range of $h$.
    \end{enumerate}
  \item Sketch the graphs $f(\theta)=2\sin\theta$ and $g(\theta)=\cos\theta-1$ on the same set of axes. Use your sketch to determine:
    \begin{enumerate}[noitemsep, label=\textbf{(\alph*)} ]
    \item $f(180^{\circ})$
    \item $g(180^{\circ})$
    \item $g(270^{\circ}) -f(270^{\circ})$
    \item The domain and range of $g$.
    \item The amplitude and period of $f$.
    \end{enumerate}
  \end{enumerate}
\practiceinfo
\par 
\par \begin{tabular}[h]{cccccc}
(1.) lTN  &  (2.) lTR  &  (3.) lTp  &  (4.) lTn  &  (5.) lTy  &  (6.) lx8  &  (7.) lxX  &  (8.) lx9  & (9.) AAA & (10.)AAA & (11.) AAA & (12.) AAA & (13.) AAA & (14.) AAA\end{tabular}
\end{eocexercises}


 \begin{solutions}{}{
\begin{enumerate}[itemsep=5pt, label=\textbf{\arabic*}. ] 


\item solution 1
\item solution 2
\item solution 3
\item solution 4
\item solution 5
\item solution 6
\item solution 7
\item solution 8
\item solution 9
\item solution 10
\item solution 11
\item solution 12
\item solution 13
\item solution 14
\item solution 15
\item solution 16

\end{enumerate}}
\end{solutions}


