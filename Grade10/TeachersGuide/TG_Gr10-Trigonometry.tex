\chapter{Trigonometry}

\begin{exercises}{}
{
\begin{enumerate}[itemsep=5pt, label=\textbf{\arabic*}. ]
\item In each of the following triangles, state whether $a$, $b$ and $c$ are the hypotenuse, opposite or adjacent sides of the triangle with respect to $\theta$. 
\begin{center}
\scalebox{0.85} % Change this value to rescale the drawing.
{
\begin{pspicture}(0,-3.7754679)(20.177345,3.9279697)
\rput (1,2){\textbf{(a)}}
\rput (5.5,2){\textbf{(b)}}
\rput (10.5,2){\textbf{(c)}}
\rput (1.5,-1.5){\textbf{(d)}}
\rput (6,-1.5){\textbf{(e)}}
\rput (9.5,-1.5){\textbf{(f)}}
\psdots[dotsize=0.027999999](2.9173439,-1.5720303)
\psline[linewidth=0.04cm](5.517344,2.5079696)(7.9173436,0.107969664)
\psline[linewidth=0.04cm](7.9173436,0.107969664)(9.0173435,1.2079697)
\psline[linewidth=0.04cm](5.517344,2.5079696)(9.0173435,1.2079697)
\psline[linewidth=0.04cm](7.717344,0.30796966)(7.9173436,0.5079697)
\psline[linewidth=0.04cm](7.9173436,0.5079697)(8.117344,0.30796966)
\psline[linewidth=0.04cm](11.0573435,2.2679696)(11.0573435,0.067969665)
\psline[linewidth=0.04cm](11.0573435,0.067969665)(14.757344,0.067969665)
\psline[linewidth=0.04cm](11.0573435,2.2679696)(14.757344,0.067969665)
\psline[linewidth=0.04cm](1.2373447,-3.3120303)(3.9373438,-0.71203035)
\psline[linewidth=0.04cm](3.9373438,-0.71203035)(3.9373438,-3.3120303)
\psline[linewidth=0.04cm](3.9373438,-3.3120303)(1.2373447,-3.3120303)
\psline[linewidth=0.04cm](7.477345,-1.2120303)(5.7773438,-3.3120303)
\psline[linewidth=0.04cm](7.477345,-1.2120303)(8.9773445,-2.4120302)
\psline[linewidth=0.04cm](8.9773445,-2.4120302)(5.7773438,-3.3120303)
\psline[linewidth=0.04cm](7.257343,-1.5120304)(7.537344,-1.7320304)
\psline[linewidth=0.04cm](7.517344,-1.7120303)(7.717344,-1.4320303)
\psline[linewidth=0.04cm](11.097343,0.38796967)(11.397344,0.38796967)
\psline[linewidth=0.04cm](11.397344,0.38796967)(11.397344,0.08796967)
\psline[linewidth=0.04cm](3.9173427,-3.0320303)(3.6173437,-3.0320303)
\psline[linewidth=0.04cm](3.6173437,-3.0320303)(3.6173437,-3.3320303)
\psline[linewidth=0.04cm](10.0173435,-1.7320304)(11.417343,-3.2320304)
\psline[linewidth=0.04cm](11.417343,-3.2320304)(13.0173435,-1.7320304)
\psline[linewidth=0.04cm](10.0173435,-1.7320304)(13.0173435,-1.7320304)
\psline[linewidth=0.04cm](11.217343,-3.0320303)(11.417343,-2.8320303)
\psline[linewidth=0.04cm](11.417343,-2.8320303)(11.617344,-3.0320303)
\usefont{T1}{ptm}{m}{n}
\rput(8.662344,1.1179696){$\theta$}
\usefont{T1}{ptm}{m}{n}
\rput(13.842344,0.29796967){$\theta$}
\usefont{T1}{ptm}{m}{n}
\rput(3.682343,-1.3020303){$\theta$}
\usefont{T1}{ptm}{m}{n}
\rput(6.362344,-2.9020302){$\theta$}
\usefont{T1}{ptm}{m}{n}
\rput(10.502343,-1.9420303){$\theta$}
\psline[linewidth=0.04cm](2.2973437,2.2879696)(0.09734377,0.18796967)
\psline[linewidth=0.04cm](0.09734377,0.18796967)(4.4973435,0.18796967)
\psline[linewidth=0.04cm](2.2973437,2.2879696)(4.4973435,0.18796967)
\psline[linewidth=0.04cm](2.0973437,2.0879698)(2.2973437,1.8879696)
\psline[linewidth=0.04cm](2.2973437,1.8879696)(2.4973438,2.0879698)
\usefont{T1}{ptm}{m}{n}
\rput(0.54234374,0.37796965){$\theta$}
\usefont{T1}{ptm}{m}{n}
\rput(1.1764063,1.5979697){$a$}
\usefont{T1}{ptm}{m}{n}
\rput(2.1778126,-0.10203034){$b$}
\usefont{T1}{ptm}{m}{n}
\rput(3.759219,1.2979697){$c$}
\usefont{T1}{ptm}{m}{n}
\rput(6.596406,1.0179696){$a$}
\usefont{T1}{ptm}{m}{n}
\rput(7.479219,2.1179698){$c$}
\usefont{T1}{ptm}{m}{n}
\rput(8.797812,0.41796967){$b$}
\usefont{T1}{ptm}{m}{n}
\rput(13.076406,1.4979696){$a$}
\usefont{T1}{ptm}{m}{n}
\rput(10.777813,1.1979697){$b$}
\usefont{T1}{ptm}{m}{n}
\rput(12.659219,-0.20203033){$c$}
\usefont{T1}{ptm}{m}{n}
\rput(2.896407,-3.6220303){$a$}
\usefont{T1}{ptm}{m}{n}
\rput(2.2978134,-1.8220303){$b$}
\usefont{T1}{ptm}{m}{n}
\rput(4.2792187,-2.3220303){$c$}
\usefont{T1}{ptm}{m}{n}
\rput(6.596406,-1.8220303){$a$}
\usefont{T1}{ptm}{m}{n}
\rput(7.7978134,-3.1220303){$b$}
\usefont{T1}{ptm}{m}{n}
\rput(8.579218,-1.7220303){$c$}
\usefont{T1}{ptm}{m}{n}
\rput(10.496406,-2.6220303){$a$}
\usefont{T1}{ptm}{m}{n}
\rput(12.397812,-2.8220303){$b$}
\usefont{T1}{ptm}{m}{n}
\rput(11.579218,-1.4220303){$c$}
\rput{-66.69996}(-0.02258055,0.761629){\psarc[linewidth=0.04](0.5673438,0.39796966){0.35}{25.785578}{138.26501}}
\rput{106.83747}(12.386083,-6.856496){\psarc[linewidth=0.04](8.737344,1.1679693){0.34}{32.561234}{147.21713}}
\rput{93.56388}(15.110441,-13.562675){\psarc[linewidth=0.04](13.927344,0.31796986){0.39}{24.333387}{127.3505}}
\psarc[linewidth=0.04](14.697344,3.1279697){0.0}{0.0}{180.0}
\psarc[linewidth=0.04](14.837344,3.2279696){0.0}{0.0}{180.0}
\rput{153.95303}(6.6359906,-3.7989676){\psarc[linewidth=0.04](3.757346,-1.1320316){0.5}{50.00946}{137.84839}}
\psarc[linewidth=0.04](20.097343,3.9079697){0.0}{0.0}{180.0}
\psarc[linewidth=0.04](20.117344,3.8879697){0.0}{0.0}{180.0}
\psarc[linewidth=0.04](20.157345,3.8279696){0.0}{0.0}{180.0}
\rput{-46.143967}(4.180272,3.6097476){\psarc[linewidth=0.04](6.3273435,-3.1020286){0.57}{58.40874}{128.71255}}
\rput{-116.039185}(16.474606,6.6625834){\psarc[linewidth=0.04](10.317342,-1.8120308){0.52}{49.23777}{123.63383}}
\end{pspicture} 
}
\end{center}
\item Use your calculator to determine the value of the following (correct to $2$ decimal places):
\begin{enumerate}[noitemsep, label=\textbf{(\alph*)} ]
\begin{multicols}{2} 
\item $tan~ 65^{\circ}$
\item $sin~ 38^{\circ}$
\item $cos~ 74^{\circ}$
\item $sin~ 12^{\circ}$
\item $cos~ 26^{\circ}$
\item $tan~ 49^{\circ}$
\item $\frac{1}{4} cos~ 20^{\circ}$
\item $3 tan~ 40^{\circ}$
\item $\frac{2}{3} sin~ 90^{\circ}$
\end{multicols}
\end{enumerate}
\item If $x=39^{\circ}$ and $y=21^{\circ}$, use a calculator to determine whether the following statements are true or false:
\begin{enumerate}[noitemsep, label=\textbf{(\alph*)} ]
\begin{multicols}{2} 
\item $cos~ x + 2 cos~ x = 3 cos~ x$
\item $cos~ 2y = cos~ y + cos~ y$
\item $tan~ x = \dfrac{sin~ x}{cos~ x}$
\item $cos~ (x+y) = cos~ x + cos~ y$
\end{multicols}
\end{enumerate}
\item Complete each of the following (the first example has been done
  for you):
\begin{center}
\setcounter{subfigure}{0}
\scalebox{1}{
\begin{pspicture}(0,-2.0990624)(4.21625,2.0990624)
\psline[linewidth=0.04cm](0.4775,-1.5590625)(3.7775,1.5409375)
\psline[linewidth=0.04cm](3.7775,1.5409375)(3.7775,-1.5590625)
\psline[linewidth=0.04cm](0.4775,-1.5590625)(3.7775,-1.5590625)
\psline[linewidth=0.04cm](3.4775,-1.5590625)(3.4775,-1.2590625)
\psline[linewidth=0.04cm](3.4775,-1.2590625)(3.7775,-1.2590625)
\rput(4.025,1.8759375){$A$}
\rput(3.874375,-1.9240625){ $B$}
\rput(0.10625,-1.8240625){$C$}
\end{pspicture} 
}
\end{center}
\begin{enumerate}[noitemsep, label=\textbf{(\alph*)} ]
\begin{multicols}{2}
\item $sin~ \hat{A} = \frac{\mbox{opposite}}{\mbox{hypotenuse}}=\dfrac{CB}{AC}$
\item $cos~ \hat{A} = $
\item $tan~ \hat{A} = $
\item $sin~ \hat{C} = $
\item $cos~ \hat{C} = $
\item $tan~ \hat{C} = $
\end{multicols}
\end{enumerate}
\item Use the triangle below to complete the following:
\begin{center}
\scalebox{1} % Change this value to rescale the drawing.
{
\begin{pspicture}(0,-1.9864062)(4.449063,1.9664062)
\psline[linewidth=0.04cm](0.66,-1.4935937)(3.1364062,-1.4935937)
\psline[linewidth=0.04cm](3.1364062,-1.4935937)(3.1,1.9464062)
\psline[linewidth=0.04cm](3.08,1.9464062)(0.68,-1.4935937)
\psline[linewidth=0.04cm](2.7364063,-1.4935937)(2.7364063,-1.0935937)
\psline[linewidth=0.04cm](2.7364063,-1.0935937)(3.1364062,-1.0935937)
\usefont{T1}{ptm}{m}{n}
\rput(1.4889063,0.49640626){$2$}
\usefont{T1}{ptm}{m}{n}
\rput(2.0890625,-1.7835938){$1$}
\usefont{T1}{ptm}{m}{n}
\rput(1.3,-1.2){$60^{\circ}$}
\usefont{T1}{ptm}{m}{n}
\rput(2.85,1.1){$30^{\circ}$}
\usefont{T1}{ptm}{m}{n}
\rput(3.4545314,0.01640625){$\sqrt{3}$}
\end{pspicture} 
}
\end{center}
\\
\begin{enumerate}[noitemsep, label=\textbf{(\alph*)} ]
\begin{multicols}{2}
\item $sin~ 60^{\circ} = $
\item $cos~ 60^{\circ} = $
\item $tan~ 60^{\circ}= $
\item $sin~ 30^{\circ}= $
\item $cos~ 30^{\circ}= $
\item $tan~ 30^{\circ}= $
\end{multicols}
\end{enumerate}
\item Use the triangle below to complete the following:
\begin{center}
\scalebox{1} % Change this value to rescale the drawing.
{
\begin{pspicture}(0,-2.24)(5.0271873,2.22)
\psline[linewidth=0.04cm](0.361875,-1.7)(4.061875,-1.7)
\psline[linewidth=0.04cm](4.061875,-1.7)(4.061875,2.2)
\psline[linewidth=0.04cm](4.061875,2.2)(0.361875,-1.7)
\psline[linewidth=0.04cm](3.661875,-1.7)(3.661875,-1.3)
\psline[linewidth=0.04cm](3.661875,-1.3)(4.061875,-1.3)
\rput(2.2314062,-2.09){$1$}
\rput(1.1554687,-1.39){$45^{\circ}$}
\rput(3.7554688,1.41){$45^{\circ}$}
\rput(1.6054688,0.51){$\sqrt{2}$}
\rput(4.331406,0.21){$1$}
\end{pspicture} 
}
\end{center}
\begin{enumerate}[noitemsep, label=\textbf{(\alph*)} ]
\item $sin~ 45^{\circ} = $
\item $cos~ 45^{\circ} = $
\item $tan~ 45^{\circ}= $
\end{enumerate}
\end{enumerate}

}
\end{exercises}


 \begin{solutions}{}{
\begin{enumerate}[itemsep=5pt, label=\textbf{\arabic*}. ] 


\item %solution 1
\begin{multicols}{2}
%In each of the following triangles, state whether $a$, $b$ and $c$ are the hypotenuse, opposite or adjacent sides of the triangle with respect to $\theta$. 
% \setcounter{subfigure}{0}
\begin{enumerate}[noitemsep, label=\textbf{(\alph*)} ]
\item $a=$adj; $b=$hyp; $c=$opp
\item $a=$opp; $b=$adj; $c=$hyp
\item $a=$hyp; $b=$opp; $c=$adj
\item $a=$opp; $b=$hyp; $c=$adj
\item $a=$adj; $b=$hyp; $c=$opp
\item $a=$adj; $b=$opp; $c=$hyp
\end{enumerate}
\end{multicols}
\item %solution 2
\begin{multicols}{2}
\begin{enumerate}[noitemsep, label=\textbf{(\alph*)} ]
\item $2,14$%$tan~65^{\circ}$
\item $0,62$%$sin~38^{\circ}$
\item $0,28$%$cos~74^{\circ}$
\item $0,21$%$sin~12^{\circ}$
\item $0,90$%$cos~26^{\circ}$
\item $1,15$%$tan~49^{\circ}$
\item $0,23$%$\frac{1}{4}~cos~20^{\circ}$
\item $2,52$%$3~tan~40^{\circ}$
\item $0,67$%$\frac{2}{3}~sin~90^{\circ}$
\end{enumerate}

\end{multicols}
\item %solution 3
\begin{enumerate}[noitemsep, label=\textbf{(\alph*)} ]
\item %$cos~x + 2~cos~x=3~cos~x$
    LHS $= cos~39^{\circ} + 2~cos~^{\circ}\\
	 = 2,33 \\$
    RHS $= 3~cos~^{\circ}\\
	= 2,33$ \\
LHS=RHS, therefore statement is true.
\item %$cos~2y = cos~y+cos~y$
    LHS $= cos~2(21^{\circ})\\
	 = 0,74 \\$
    RHS $= cos~21^{\circ}+cos~21^{\circ}\\
	= 1,87$ \\
LHS $\ne$ RHS, therefore statement is false.
\item %$tan~x=\dfrac{sin~x}{cos~x}$
    LHS $= tan~39^{\circ}\\
	 = 0,81 \\$
    RHS $= \frac{sin~39^{\circ}}{cos~39^{\circ}}\\
	= 0,81$ \\
LHS=RHS, therefore statement is true.
\item %$cos~(x+y) = cos~x+cos~y$
LHS $= cos~(39^{\circ}+21^{\circ})\\
	 = 0,5 \\$
    RHS $= cos~39^{\circ}+cos~21^{\circ}\\
	= 1,71$ \\
LHS $\ne$ RHS, therefore statement is false.
\end{enumerate}

\item %solution 4
\begin{multicols}{2}
\begin{enumerate}[itemsep=1pt, label=\textbf{(\alph*)} ]
\item $sin~\hat{A} = \frac{CB}{AC}$
\item $cos~\hat{A} = \frac{AB}{AC}$%$cos~\hat{A} = $
\item $tan~\hat{A} = \frac{CB}{AB}$%$tan~\hat{A}= $
\item $sin~\hat{C} = \frac{AB}{AC}$%$sin~\hat{C}= $
\item $cos~\hat{C} = \frac{CB}{AC}$%$cos~\hat{C}= $
\item $tan~\hat{C} = \frac{AB}{CB}$%$tan~\hat{C}= $
\end{enumerate}

\end{multicols}
\item %solution 5
\begin{multicols}{2}
\begin{enumerate}[itemsep=1pt, label=\textbf{(\alph*)} ]
\item $\frac{\sqrt{3}}{2}$%$sin~60^{\circ} = $
\item $\frac{1}{2}$%$cos~60^{\circ} = $
\item $\sqrt{3}$%$tan~60^{\circ}= $
\item $\frac{1}{2}$%$sin~30^{\circ}= $
\item $\frac{\sqrt{3}}{2}$%$cos~30^{\circ}= $
\item $\frac{1}{\sqrt{3}}$%$tan~30^{\circ}= $
\end{enumerate}

\end{multicols}
\item %solution 6
\begin{multicols}{2}
\begin{enumerate}[itemsep=1pt, label=\textbf{(\alph*)} ]
\item $\frac{1}{\sqrt{2}}$%$sin~45^{\circ} = $
\item $\frac{1}{\sqrt{2}}$%$cos~45^{\circ} = $
\item $1$%$tan~45^{\circ}= $
\end{enumerate}
\end{multicols}
\end{enumerate}}
\end{solutions}

\begin{exercises}{}{
\begin{enumerate}[itemsep=6pt, label=\textbf{\arabic*}. ] 
\item Calculate the value of the following without using a calculator:
\begin{enumerate}[noitemsep, label=\textbf{(\alph*)} ]
\item $sin~ 45^{\circ} \times cos~ 45^{\circ}$
\item $cos~ 60^{\circ} + tan~ 45^{\circ}$
\item $sin~ 60^{\circ} - cos~ 60^{\circ}$
\end{enumerate}
\item Use the table to  show that:
\begin{enumerate}[itemsep=5pt, label=\textbf{(\alph*)} ]
\item $\dfrac{sin~ 60^{\circ}}{cos~ 60^{\circ}} = tan~ 60^{\circ} $
\item $sin^{2} 45^{\circ}+ cos^{2} 45^{\circ} =1 $
\item $cos~ 30^{\circ} =\sqrt{1- sin^{2} 30^{\circ}}$
\end{enumerate}
\item Use the definitions of the trigonometric rations to answer the following questions:
\begin{enumerate}[noitemsep, label=\textbf{(\alph*)} ]
\item Explain why $sin ~\alpha \leq 1$ for all values of $\alpha$.
\item Explain why $cos~ \beta$ has a maximum value of $1$.
\item Is there a maximum value for $tan~\gamma$?
\end{enumerate}
\end{enumerate}
}
\end{exercises}


 \begin{solutions}{}{
\begin{enumerate}[itemsep=7pt, label=\textbf{\arabic*}. ] 

\item %solution 1
\begin{multicols}{2}
\begin{enumerate}[noitemsep, label=\textbf{(\alph*)} ]
\item $sin~ 45^{\circ} \times cos~ 45^{\circ}\\
      =\frac{1}{\sqrt{2}}\times \frac{1}{\sqrt{2}} \\
      =\frac{1}{2}$\\
\item $cos~ 60^{\circ} + tan~ 45^{\circ}\\
      =\frac{1}{2}+1 \\
      =1\frac{1}{2}$ \\
\item $sin~ 60^{\circ} - cos~ 60^{\circ} \\
      =\frac{\sqrt{3}}{2}-\frac{1}{2} \\
      =\frac{\sqrt{3}-1}{2}$
\end{enumerate}
\end{multicols}
\item %solution 2
\begin{multicols}{2}
\begin{enumerate}[itemsep=3pt, label=\textbf{(\alph*)} ]
 \item   LHS$ = \dfrac{sin~60^{\circ}}{cos~60^{\circ}}\\[4pt]
	      = \dfrac{\frac{\sqrt{3}}{2}}{\frac{1}{2}}\\[4pt]
	      = \frac{\sqrt{3}}{2}\times {\frac{2}{1}}\\
	      = \sqrt{3}\\
	  $RHS $= tan~60^{\circ}\\
	      = \sqrt{3}$\\
Therefore LHS$=$RHS.
 \item  LHS$ = sin^{2}~45^{\circ}+ cos^{2}~45^{\circ}\\
	     = (\frac{1}{\sqrt{2}})^2+(\frac{1}{\sqrt{2}})^2\\
	     = \frac{1}{2}+\frac{1}{2}\\
	     = 1\\
	$RHS$ = 1$\\
Therefore LHS$=$RHS.
 \item  LHS$ = cos~30^{\circ}\\
	     = \frac{\sqrt{3}}{2}\\
	$RHS$ = \sqrt{1- sin^{2}~30^{\circ}}\\
	    = \sqrt{1- (\frac{1}{2})^2}\\
	    = \sqrt{1- (\frac{1}{4})}\\
	    = \sqrt{\frac{3}{4}}\\
	    = \frac{\sqrt{3}}{2}$\\
Therefore LHS$=$RHS.
 \end{enumerate}
\end{multicols}
\item %solution 3
\begin{enumerate}[noitemsep, label=\textbf{(\alph*)} ]
 \item  The sine ratio is defined as $\frac{\mbox{\scriptsize opposite}}{\mbox{\scriptsize hypotenuse}}$.  In any right-angled triangle, the hypotenuse is the side of longest length.  Therefore the maximum length of the opposite side is equal to the length of the hypotenuse.  The maximum value of the sine ratio is then $\frac{\mbox{\scriptsize hypotenuse}}{\mbox{\scriptsize hypotenuse}}=1$.
 \item Similarly, the cosine ratio is defined as $\frac{\mbox{\scriptsize adjacent}}{\mbox{\scriptsize hypotenuse}}$.  In any right-angled triangle, the hypotenuse is the side of longest length.  Therefore the maximum length of the adjacent side is equal to the length of the hypotenuse.  The maximum value of the cosine ratio is then $\frac{\mbox{\scriptsize hypotenuse}}{\mbox{\scriptsize hypotenuse}}=1$.
 \item For the tangent ratio $\frac{\mbox{\scriptsize opposite}}{\mbox{\scriptsize adjacent}}$, there is no maximum value. Notice that there is a restriction; the ratio is undefined if the length of the adjacent side equals $0$.



 \end{enumerate}
\end{enumerate}}
\end{solutions}

\begin{exercises}{}
{
\begin{enumerate}[itemsep=5pt, label=\textbf{\arabic*}. ]
\item In each triangle find the length of the side marked with a letter. Give answers correct to $2$ decimal places.
\begin{center}
\scalebox{0.85} % Change this value to rescale the drawing.
{
\begin{pspicture}(0,-4.11)(11.1198435,5)
\psline[linewidth=0.04](4.974369,2.612122)(2.6350498,3.9603012)(0.6377475,0.49464345)(4.974369,2.612122)
\psline[linewidth=0.04cm](2.8949742,3.8105037)(2.7451768,3.550579)
\psline[linewidth=0.04cm](2.7451768,3.550579)(2.4852521,3.7003772)
\psline[linewidth=0.04](4.557088,-3.2752385)(4.589271,-0.5754303)(0.5895552,-0.52775127)(4.557088,-3.2752385)
\psline[linewidth=0.04cm](4.5856953,-0.8754088)(4.2857165,-0.87183315)
\psline[linewidth=0.04cm](4.2857165,-0.87183315)(4.2892923,-0.5718545)
\psline[linewidth=0.04](7.0587263,3.3569472)(6.922255,0.6603981)(10.917143,0.45821914)(7.0587263,3.3569472)
\psline[linewidth=0.04cm](6.9374194,0.9600146)(7.2370353,0.94485116)
\psline[linewidth=0.04cm](7.2370353,0.94485116)(7.2218714,0.64523476)
\psline[linewidth=0.04](7.9354,-3.9493167)(10.598035,-3.9603012)(10.613694,-0.16448934)(7.9354,-3.9493167)
\psline[linewidth=0.04cm](10.340361,-3.959238)(10.341382,-3.7116852)
\psline[linewidth=0.04cm](10.341382,-3.7116852)(10.599056,-3.7127483)
\rput(0.12453125,3.7196066){\textbf{(a)}}
\rput(6.133125,3.7396066){\textbf{(b)}}
\rput(0.11421875,-0.26039344){\textbf{(c)}}
\rput(6.1343746,-0.28039345){\textbf{(d)}}
\usefont{T1}{ptm}{m}{n}
\rput(4.197969,2.6346066){$37^\circ$}
\usefont{T1}{ptm}{m}{n}
\rput(3.0853126,1.2){$62$}
\usefont{T1}{ptm}{m}{n}
\rput(1.3824998,2.5846066){$a$}
\usefont{T1}{ptm}{m}{n}
\rput(9.804531,0.8146064){$23^\circ$}
\usefont{T1}{ptm}{m}{n}
\rput(8.595937,0.21460645){$21$}
\usefont{T1}{ptm}{m}{n}
\rput(6.663907,2.0646067){$b$}
\usefont{T1}{ptm}{m}{n}
\rput(4.2134376,-2.6453934){$ 55^\circ$}
\usefont{T1}{ptm}{m}{n}
\rput(2.166719,-2.1653934){$19$}
\usefont{T1}{ptm}{m}{n}
\rput(4.8759375,-1.8353934){$c$}
\usefont{T1}{ptm}{m}{n}
\rput(8.953907,-1.9653934){$33$}
\usefont{T1}{ptm}{m}{n}
\rput(10.22,-1.1453935){ $49^\circ$}
\usefont{T1}{ptm}{m}{n}
\rput(10.90422,-2.0153935){$d$}
\end{pspicture}    
}
\end{center}
\begin{center}
 \scalebox{0.85} % Change this value to rescale the drawing.
{
\begin{pspicture}(0,-3.8416457)(11.207031,3.8216705)
\psline[linewidth=0.04](2.3197074,3.8016455)(1.0997375,1.3929795)(4.668132,-0.4143837)(2.3197074,3.8016455)
\psline[linewidth=0.04cm](1.2352896,1.6606089)(1.5029193,1.5250567)
\psline[linewidth=0.04cm](1.5029193,1.5250567)(1.3673671,1.2574271)
\rput(0.11234375,3.5211668){\textbf{(e)}}
\rput(6.084844,3.5211668){\textbf{(f)}}
\rput(0.14375,-0.39883313){\textbf{(g)}}
\rput(6.1403127,-0.498833){\textbf{(h)}}
\psline[linewidth=0.04](1.0870312,-0.6888331)(1.0870312,-3.3888335)(5.0870314,-3.3888335)(1.0870312,-0.6888331)
\psline[linewidth=0.04cm](1.0870312,-3.0888333)(1.3870313,-3.0888333)
\psline[linewidth=0.04cm](1.3870313,-3.0888333)(1.3870313,-3.3888335)
\psline[linewidth=0.04](10.705838,-3.6336784)(10.738021,-0.93386954)(6.738305,-0.88619083)(10.705838,-3.6336784)
\psline[linewidth=0.04cm](10.734446,-1.2338486)(10.434467,-1.2302725)
\psline[linewidth=0.04cm](10.434467,-1.2302725)(10.438043,-0.9302939)
\psline[linewidth=0.04](11.200404,0.87534267)(9.325028,2.8177538)(6.4473815,0.039416883)(11.200404,0.87534267)
\psline[linewidth=0.04cm](9.533403,2.6019304)(9.31758,2.393555)
\psline[linewidth=0.04cm](9.31758,2.393555)(9.109204,2.6093786)
\usefont{T1}{ptm}{m}{n}
\rput(3.5887504,1.9261669){$e$}
\usefont{T1}{ptm}{m}{n}
\rput(3.8853126,0.3161668){$17^\circ$}
\usefont{T1}{ptm}{m}{n}
\rput(1.3671876,2.6761668){ $12$}
\usefont{T1}{ptm}{m}{n}
\rput(7.533282,0.51616687){$22^\circ$}
\usefont{T1}{ptm}{m}{n}
\rput(9.139219,0.2861668){$f$}
\usefont{T1}{ptm}{m}{n}
\rput(7.758125,1.7361668){ $31$}
\usefont{T1}{ptm}{m}{n}
\rput(3.028594,-1.683833){$32$}
\usefont{T1}{ptm}{m}{n}
\rput(4.1132817,-3.123833){$23^\circ$}
\usefont{T1}{ptm}{m}{n}
\rput(2.949844,-3.6138332){$g$}
\usefont{T1}{ptm}{m}{n}
\rput(10.928438,-2.2538335){$h$}
\usefont{T1}{ptm}{m}{n}
\rput(7.7667193,-1.2438331){ $30^\circ$}
\usefont{T1}{ptm}{m}{n}
\rput(8.245001,-2.3038335){$20$}
\end{pspicture} 
}
\end{center}  
\item Write down two ratios for each of the following in terms of the
  sides: $AB$; $BC$; $BD$; $AD$; $DC$ and $AC$:
\begin{center}
\scalebox{1} % Change this value to rescale the drawing.
{
\begin{pspicture}(0,-1.2515883)(4.02125,1.2515885)
\psline[linewidth=0.04,fillstyle=solid](0.019002499,-1.2315885)(0.009700612,1.2315885)(4.00125,-1.1902716)(0.039492033,-1.2120903)(0.039492033,-1.2120903)(0.0,-1.1911051)
\psline[linewidth=0.04](0.009700612,-0.9884115)(0.20970061,-0.9884115)(0.20970061,-1.2284116)
\psline[linewidth=0.04,fillstyle=solid](1.3697007,0.4315885)(0.049700614,-1.1684115)(0.049700614,-1.1484115)
\psline[linewidth=0.04](1.2297006,0.2715885)(1.0697006,0.3515885)(1.1897006,0.5115885)
\usefont{T1}{ptm}{m}{n}
\rput(0,-1.5){$A$}
\rput(0, 1.5){$B$}
\rput(1.4, 0.7){$C$}
\rput(4, -1.5){$D$}
\end{pspicture} 
}
\end{center}
     \begin{enumerate}[noitemsep, label=\textbf{(\alph*)} ]
    \item $sin~ \hat{B}$
    \item $cos~ \hat{D}$
    \item $tan~ \hat{B}$
    \end{enumerate}
\vspace{10pt}
\item In $\triangle MNP$, $\hat{N}=90^{\circ}$, $MP=20$ and $\hat{P}=40^{\circ}$. Calculate $NP$ and $MN$ (correct to $2$ decimal places).
\end{enumerate}
\end{enumerate}

}
\end{exercises}


 \begin{solutions}{}{
\begin{enumerate}[itemsep=5pt, label=\textbf{\arabic*}. ] 


\item 
\begin{multicols}{2}
      \begin{enumerate}[noitemsep, label=\textbf{(\alph*)} ]
      
\item $sin~37^{\circ} = \frac{a}{62}\\ a = 62 \times sin~ 37^{\circ} \\ a=37,31$ units
\item $tan~23^{\circ} = \frac{b}{21}\\ b = 21 \times tan~23^{\circ} \\b=8,91$ units
\item $cos~55^{\circ}  = \frac{c}{19}\\ c = 19 \times cos~55^{\circ} \\c=10,90$ units
\item $cos~49^{\circ} =\frac{d}{33} \\ d = 33 \times cos~49^{\circ} \\d=21,65$ units
\item $sin~17^{\circ} = \frac{12}{e}\\ e = \frac{12}{sin~17^{\circ} } \\e=41,04$ units
\item $cos ~22^{\circ}  = \frac{31}{f}\\ f = \frac{31}{cos~22^{\circ} }\\f=33,43$ units
\item $cos~ 23^{\circ}=\frac{g}{32}\\ g = 32 \times cos~23^{\circ} \\ g=29,46$ units
\item $sin~30^{\circ}  = \frac{h}{20}\\ h = 20 \times sin~30^{\circ} \\h=10,00$ units
      \end{enumerate}
\end{multicols}
\item 

 \begin{enumerate}[itemsep=1pt, label=\textbf{(\alph*)} ]
    \item $sin~\hat{B}=\frac{AC}{AB}=\frac{AD}{BD}$%$sin~\hat{B}$
    \item $cos~\hat{D}=\frac{AD}{BD}=\frac{CD}{AD}$%$cos~\hat{D}$
    \item $tan~\hat{B}=\frac{AC}{BC}=\frac{AD}{AB}$%$tan~\hat{B}$
    \end{enumerate}
\item 
$sin~\hat{P}=\frac{MN}{MP}\\
sin~40^{\circ} = \frac{MN}{20}\\
\therefore MN = 20~sin~40^{\circ}\\
=12,86$ units\\
$cos~\hat{P} = \frac{NP}{MP}\\
cos~40^^{\circ}=\frac{NP}{20}\\
\therefore NP = 20~cos~40^{\circ}\\
=15,32$ units

\end{enumerate}}
\end{solutions}


% \section{Finding an angle}
\begin{exercises}{}
{
   \begin{enumerate}[itemsep=5pt, label=\textbf{\arabic*}. ] 
\item Determine the angle (correct to $1$ decimal place):
    \begin{enumerate}[itemsep=3pt, label=\textbf{(\alph*)} ]
\begin{multicols}{2}
 \item $tan~ \theta = 1,7$
\item $sin~ \theta = 0,8$
\item $cos~ \alpha = 0,32$
\item $tan~ \theta = 5\frac{3}{4}$
\item $sin~ \theta = \frac{2}{3}$
\item $cos~ \gamma = 1,2$
\item $4 cos~ \theta = 3$
\item $cos~ 4\theta = 0,3$
\item $sin~ \beta + 2= 2,65$
\item $sin~ \theta = 0,8$
\item $3 tan~ \beta = 1$
\item $sin~ 3\alpha = 1,2$
\item $tan~ \frac{\theta}{3} = sin~ 48^{\circ}$
\item $\frac{1}{2} cos~ 2\beta = 0,3$
\item $2 sin~ 3\theta +1= 2,6$
\end{multicols}
\end{enumerate}
\item Determine $\alpha$ in the following right-angled triangles:
\begin{center}
\scalebox{1} % Change this value to rescale the drawing.
{
\begin{pspicture}(0,-4.032389)(7.3778124,4.17849)
\rput(0, 3.5){\textbf{(a)}}
\rput(3.5, 3.5){\textbf{(b)}}
\rput(0, 0.5){\textbf{(c)}}
\rput(3.5, 0.5){\textbf{(d)}}
\rput(0, -2.0){\textbf{(e)}}
\rput(3.5, -2.0){\textbf{(f)}}
\psline[linewidth=0.04,fillstyle=solid](4.128174,1.9359901)(4.129576,3.691884)(6.1603127,1.9559901)(4.142015,1.9498184)(4.142015,1.9498184)(4.115519,1.9649143)
\psline[linewidth=0.04,fillstyle=solid](0.5403125,3.7206802)(2.3103564,3.7223134)(0.5605087,1.1162983)(0.5542749,3.7072453)(0.5542749,3.7072453)(0.5694489,3.7330344)
\psline[linewidth=0.04](0.8203125,3.71599)(0.8203125,3.41599)(0.5403125,3.41599)
\psline[linewidth=0.04](4.132775,2.2306225)(4.429114,2.2301245)(4.4286456,1.9555349)
\psline[linewidth=0.04,fillstyle=solid](0.58817375,-1.2040099)(0.5895763,0.5518841)(2.6203125,-1.1840099)(0.60201496,-1.1901817)(0.60201496,-1.1901817)(0.57551914,-1.1750857)
\psline[linewidth=0.04](0.59277505,-0.90937746)(0.889114,-0.9098756)(0.88864577,-1.184465)
\psline[linewidth=0.04,fillstyle=solid](0.57534015,-1.9803641)(2.331222,-1.9737015)(0.6046739,-4.0123897)(0.5892318,-1.9941416)(0.5892318,-1.9941416)(0.6042059,-1.9675767)
\psline[linewidth=0.04](0.8699906,-1.983612)(0.8708536,-2.2799501)(0.5962649,-2.2807431)
\psline[linewidth=0.04,fillstyle=solid](6.6596694,0.2178756)(6.6803126,-0.6840099)(3.9803126,0.21599011)(6.645639,0.20423959)(6.645639,0.20423959)(6.6719236,0.18877967)
\psline[linewidth=0.04](6.6710052,-0.076665334)(6.3803124,-0.06400988)(6.3789563,0.20247632)
\psline[linewidth=0.04,fillstyle=solid](5.604427,-1.7804683)(6.7567835,-3.1053228)(4.0853443,-3.1304228)(5.603074,-1.7999867)(5.603074,-1.7999867)(5.632968,-1.7939644)
\psline[linewidth=0.04](5.794496,-2.005642)(5.570731,-2.199926)(5.3907113,-1.9925796)
\rput{-34.695152}(-0.9369147,0.7926677){\psarc[linewidth=0.024](0.8003125,1.8959901){0.3}{39.289406}{180.0}}
\usefont{T1}{ptm}{m}{n}
\rput(1.4359375,3.9229398){$4$}
\usefont{T1}{ptm}{m}{n}
\rput(0.29078126,2.4629397){$9$}
\usefont{T1}{ptm}{m}{n}
\rput(5.470625,3.02294){$13$}
\usefont{T1}{ptm}{m}{n}
\rput(5.034219,1.5829399){$7,5$}
\usefont{T1}{ptm}{m}{n}
\rput(1.75375,-0.057060193){$2,2$}
\usefont{T1}{ptm}{m}{n}
\rput(0.2096875,-0.4770602){$1,7$}
\usefont{T1}{ptm}{m}{n}
\rput(5.716875,0.4829398){$9,1$}
\usefont{T1}{ptm}{m}{n}
\rput(7.0603123,-0.3170602){$4,5$}
\usefont{T1}{ptm}{m}{n}
\rput(0.26703125,-2.8770602){$12$}
\usefont{T1}{ptm}{m}{n}
\rput(1.9092188,-3.1370602){$15$}
\usefont{T1}{ptm}{m}{n}
\rput(4.5784373,-2.3170602){$1$}
\usefont{T1}{ptm}{m}{n}
\rput(5.54375,-3.47706){$\sqrt{2}$}
\usefont{T1}{ptm}{m}{n}
\rput(6.2634373,-2.93706){$\alpha$}
\usefont{T1}{ptm}{m}{n}
\rput(0.7834375,-3.4970603){$\alpha$}
\usefont{T1}{ptm}{m}{n}
\rput(1.9234375,-0.9970602){$\alpha$}
\usefont{T1}{ptm}{m}{n}
\rput(5.1634374,-0.017060194){$\alpha$}
\usefont{T1}{ptm}{m}{n}
\rput(4.3034377,3.1429398){$\alpha$}
\usefont{T1}{ptm}{m}{n}
\rput(0.7634375,1.8029398){$\alpha$}
\rput{-162.53017}(7.4652467,7.66313){\psarc[linewidth=0.024](4.3213253,3.258065){0.34310362}{39.289406}{156.30495}}
\rput{-89.91282}(5.095528,5.0594177){\psarc[linewidth=0.024](5.081325,-0.02193504){0.34310362}{53.61859}{134.10115}}
\rput{-319.68652}(-0.17349519,-1.5915118){\psarc[linewidth=0.024](2.0811038,-1.0320795){0.42839816}{67.16096}{156.30495}}
\rput{-30.447884}(1.9160283,-0.06363264){\psarc[linewidth=0.024](0.8411037,-3.5520794){0.42839816}{67.16096}{156.30495}}
\rput{-319.68652}(-0.41087502,-4.8648257){\psarc[linewidth=0.024](6.4211035,-2.9920795){0.42839816}{67.16096}{156.30495}}
\end{pspicture} 
}
\end{center}
\end{enumerate}

}
\end{exercises}


 \begin{solutions}{}{
\begin{enumerate}[itemsep=5pt, label=\textbf{\arabic*}. ] 


\item 
\begin{multicols}{2}
    \begin{enumerate}[noitemsep, label=\textbf{(\alph*)} ]
\item $\theta = 59,5^{\circ}$%$tan~\theta = 1,7$
\item $\theta =53,1^{\circ}$%$sin~\theta = 0,8$
\item $\theta =71,3^{\circ}$%$cos~\alpha = 0,32$
\item $\theta =80,1^{\circ}$%$tan~\theta = 5\frac{3}{4}$
\item $\theta =41,8^{\circ}$%$sin~\theta = \frac{2}{3}$
\item No solution%$cos~\gamma = 1,2$
\item $\theta =41,4^{\circ}$%$4~cos~\theta = 3$
\item $\theta =18,1^{\circ}$%$cos~4\theta = 0,3$
\item $\theta =40,5^{\circ}$%$sin~\beta + 2= 2,65$
\item $\theta =53,1^{\circ}$%$sin~\theta = 0,8$
\item $\theta =18,4^{\circ}$%$3 ~tan~\beta = 1$
\item No solution%$sin~3\alpha = 1,2$
\item $\theta =109,8^{\circ}$%$tan~(\frac{\theta}{3}) = sin~48^{\circ}$
\item $\theta =26,6^{\circ}$%$\frac{1}{2}~cos~2\beta = 0,3$
\item $\theta =17,7^{\circ}$%$2~sin~3\theta +1= 2,6$
\end{enumerate}
\end{multicols}
\item 
\begin{multicols}{2}
    \begin{enumerate}[noitemsep, label=\textbf{(\alph*)} ]
\item $tan~ \alpha = \frac{4}{9}=24,0^{\circ}$
\item $sin~\alpha = \frac{7,5}{13}=35,2^{\circ}$
\item $sin~\alpha= \frac{1,7}{2,2}=50,6^{\circ}$
\item $tan~\alpha=\frac{4,5}{9,1}=26,3^{\circ}$
\item $cos~\alpha=\frac{12}{15}=37,0^{\circ}$
\item $sin~\alpha=\frac{1}{\sqrt{2}}=45,0^{\circ}$
      \end{enumerate}
\end{multicols}
\end{enumerate}}
\end{solutions}


% \section{Two-dimensional problems}
\begin{exercises}{}
{
\begin{enumerate}[noitemsep, label=\textbf{\arabic*}. ] 
\item A boy flying a kite is standing $30~$m from a point directly under the kite. If the kite's string is $50~$m long, find the angle of elevation of the kite.
\item What is the angle of elevation of the sun when a tree $7,15$ m tall casts a shadow $10,1$ m long?
\item From a distance of $300$ m, Susan looks up at the top of a lighthouse. The angle of elevation is $5^{\circ}$. Determine the height of the lighthouse to the nearest metre.
\item A ladder of length $25$ m is resting against a wall, the ladder makes an angle $37^{\circ}$ to the wall. Find the distance between the wall and the base of the ladder. 
\end{enumerate}

}
\end{exercises} 


 \begin{solutions}{}{
\begin{enumerate}[itemsep=5pt, label=\textbf{\arabic*}. ] 


\item 
$cos~x=\frac{30}{50}\\ \therefore x=53,13^{\circ}$
\item $tan~x = \frac{7,15}{10,1}\ \therefore x=35,3^{\circ}$\\
\item 
\scalebox{1} % Change this value to rescale the drawing.
{
\begin{pspicture}(0,-1.383125)(4.8690624,1.383125)
\psline[linewidth=0.04](4.17,1.2696875)(4.17,-0.9503125)(0.69,-0.9703125)(4.15,1.2496876)(4.15,1.2496876)(3.85,1.0496875)
% \usefont{T1}{ptm}{m}{n}
\rput(0.27453125,-1.0003124){$S$}
% \usefont{T1}{ptm}{m}{n}
\rput(4.4845314,-1.0203125){$T$}
% \usefont{T1}{ptm}{m}{n}
\rput(4.5045314,1.1796875){$L$}
\psframe[linewidth=0.04,dimen=outer](4.19,-0.6503125)(3.89,-0.9503125)
% \usefont{T1}{ptm}{m}{n}
\rput(2.545,-1.1803125){$300$ m}
% \usefont{T1}{ptm}{m}{n}
\rput(1.3745313,-0.7403125){$5^{\circ}$}
\end{pspicture} 
}\\
In $\triangle LTS\\
tan~\hat{S} = \frac{LT}{ST}\\
\therefore LT = 300~tan~5^{\circ}\\
=26$ m
\item 
\scalebox{1} % Change this value to rescale the drawing.
{
\begin{pspicture}(0,-1.3664062)(3.58,1.3464062)
\psline[linewidth=0.04](0.6954687,1.3264062)(0.6954687,-0.8935937)(3.56,-0.8801563)(0.6754687,1.3064064)
\usefont{T1}{ptm}{m}{n}
\rput(2.635,0.49640635){$25$ m}
\psframe[linewidth=0.04,dimen=outer](0.9754687,-0.61359376)(0.6754687,-0.91359377)
\usefont{T1}{ptm}{m}{n}
\rput(2.1054688,-1.1635938){$x$ }
\usefont{T1}{ptm}{m}{n}
\rput(1,0.79640627){$37^{\circ}$}
\end{pspicture} }
\\
$sin~37^{\circ}=\frac{x}{ 25}\\
x=25~sin~37^{\circ}\\
x=15$ m
\end{enumerate}
}
\end{solutions}


% \section{Defining ratios in the Cartesian plane}
\begin{exercises}{}
{
  \begin{enumerate}[itemsep=5pt, label=\textbf{\arabic*}. ]
\item $B$ is a point in the Cartesian plane. Determine without using a calculator:
\begin{enumerate}[noitemsep, label=\textbf{(\alph*)} ]
\item $OB$
\item $cos~ \beta$
\item $cosec~ \beta$
\item $tan~ \beta$
\end{enumerate}

\scalebox{0.8} % Change this value to rescale the drawing.
{
\begin{pspicture}(0,-3.1567187)(6.5990624,3.1967187)
\rput(3.0,-0.15671875){\psaxes[linewidth=0.04,arrowsize=0.05291667cm 2.0,arrowlength=1.4,arrowinset=0.4,labels=none,ticks=none,ticksize=0.10583333cm]{<->}(0,0)(-2.5,-2.5)(2.5,2.5)}
\usefont{T1}{ptm}{m}{n}
\rput(6.224531,0.01328125){$x$}
\usefont{T1}{ptm}{m}{n}
\rput(3.1245313,2.9932814){$y$}
\psline[linewidth=0.04cm](3.0,-0.17671876)(4.0,-2.2167187)
\psline[linewidth=0.04cm,linestyle=dashed,dash=0.16cm 0.16cm](3.98,-2.2167187)(4.02,-0.09671875)
\psdots[dotsize=0.12](4.0,-2.2367187)
\psarc[linewidth=0.04,arrowsize=0.05291667cm 2.0,arrowlength=1.4,arrowinset=0.4]{->}(3.02,-0.15671875){0.4}{0.0}{284.03625}
\psframe[linewidth=0.04,dimen=outer](4.04,-0.13671875)(3.76,-0.41671875)
\usefont{T1}{ptm}{m}{n}
\rput(3.9645312,0.1){$X$}
\usefont{T1}{ptm}{m}{n}
\rput(3,-0.1){\LARGE $O$}
\usefont{T1}{ptm}{m}{n}
\rput(2.5445313,0.33328125){\LARGE$\beta$}
\usefont{T1}{ptm}{m}{n}
\rput(4.1445312,-2.5067186){$B(1;-3)$}
\end{pspicture} 
}

\item If $sin~ \theta= 0,4$ and $\theta$ is an obtuse angle, determine:
\begin{enumerate}[noitemsep, label=\textbf{(\alph*)} ]
 \item $cos~ \theta$
\item $\sqrt{21} tan~ \theta$
\end{enumerate}


\item Given $tan~ \theta = \frac{t}{2}$, where $0^{\circ} \leq \theta \leq 90^{\circ}$.
Determine the following in terms of $t$:
\begin{enumerate}[noitemsep, label=\textbf{(\alph*)} ]
\item $sec~ \theta $
\item $cot~ \theta $  
\item $cos^2~ \theta $
\item $tan^2~ \theta-sec^2~ \theta $
\end{enumerate}


}
\end{exercises}


 \begin{solutions}{}{
\begin{enumerate}[itemsep=5pt, label=\textbf{\arabic*}. ] 
\item
\begin{multicols}{2}
\begin{enumerate}[itemsep=1pt, label=\textbf{(\alph*)} ]
\item In $\triangle BOX\\
OB^2 = OX^2+XB^2\\
OB^2 = 1^2+3^2\\
OB=\sqrt{10}$ units%$OB$
\item $cos~\beta = \frac{x}{r} =\frac{1}{\sqrt{10}}$%$cos~\beta$
\item $cosec~\beta= \frac{r}{y}=\frac{\sqrt{10}}{-3}$%$cosec~\beta$
\item $tan~\beta = \frac{y}{x} = \frac{-3}{1}=-3$%$tan~\beta$
\end{enumerate}
\end{multicols}
\item $sin~\theta = 0,4 = \frac{4}{10} = \frac{2}{5}\\
\therefore y=2;~ r =5\\
\therefore x^2=r^2-y^2 = 5^2-2^2=21\\
\therefore x= \pm \sqrt{21}\\$
but angle is in Quadrant III, therefore $x$ is negative
$\therefore x=-\sqrt{21}$\\
\scalebox{1} % Change this value to rescale the drawing.
{
\begin{pspicture}(0,-3.1567187)(6.5990624,3.1967187)
\rput(3.0,-0.15671875){\psaxes[linewidth=0.04,arrowsize=0.05291667cm 2.0,arrowlength=1.4,arrowinset=0.4,labels=none,ticks=none,ticksize=0.10583333cm]{<->}(0,0)(-2,-2)(2,2)}
\usefont{T1}{ptm}{m}{n}
\rput(5.224531,0.01328125){$x$}
\usefont{T1}{ptm}{m}{n}
\rput(3.1245313,1.9932814){$y$}
\psline[linewidth=0.04cm](3.0,-0.17671876)(1.38,0.68328124)
\psline[linewidth=0.04cm,linestyle=dashed,dash=0.16cm 0.16cm](1.4,-0.13671875)(1.4,0.7032812)
\psdots[dotsize=0.12](1.4,0.66328126)
\psframe[linewidth=0.04,dimen=outer](1.62,0.06328125)(1.38,-0.17671876)
\usefont{T1}{ptm}{m}{n}
\rput(2.8045313,-0.34671876){$0$}
\usefont{T1}{ptm}{m}{n}

\psarc[linewidth=0.04,arrowsize=0.05291667cm 2.0,arrowlength=1.4,arrowinset=0.4]{->}(3.03,-0.10671875){0.49}{356.9872}{150.52411}
\usefont{T1}{ptm}{m}{n}
\rput(3.2345312,0.11328125){$\theta$}
\end{pspicture} 
}

\begin{enumerate}[itemsep=1pt, label=\textbf{(\alph*)} ]
\item $cos~\theta = \frac{x}{r} = \frac{-\sqrt{21}}{5}$%$cos~\theta$
\item $\sqrt{21}~tan~\theta= \sqrt{21}\left(\frac{y}{x}\right) = \sqrt{21}\left(\frac{2}{-\sqrt{21}}\right) = -2$%$\sqrt{21}~tan~\theta$
\end{enumerate}
\item
\begin{enumerate}[itemsep=4pt, label=\textbf{(\alph*)} ]
\item $sec~ \theta = \frac{\sqrt{t^2+4}}{2}$
\item $cot~ \theta = \frac{2}{t}$  
\item $cos^2~ \theta = (\frac{2}{\sqrt{t^2+4}})^2 = \frac{4}{t^2+4}$
\item $tan^2~ \theta-sec^2~ \theta \\[4pt]
= \frac{t^2}{4} - (\frac{\sqrt{t^2+4}}{2})^2\\[4pt]
= \frac{t^2}{4} - (\frac{t^2+4}{4})\\[4pt]
= \frac{t^2 -t^2 -4}{4}\\
=-1$
\end{enumerate}}
\end{solutions}


\begin{eocexercises}{}
\begin{enumerate}[itemsep=6pt, label=\textbf{\arabic*}. ] 
\item Without using a calculator determine the value of 
\begin{equation*}
sin~ 60^{\circ} cos~ 30^{\circ}-cos~ 60^{\circ}sin~ 30^{\circ} + tan~ 45^{\circ}
\end{equation*}
\item If $3 tan~ \alpha = -5$ and $0^{\circ} < \alpha < 270^{\circ}$, use a sketch to determine:
    \begin{enumerate}[noitemsep, label=\textbf{(\alph*)} ]
    \item $cos~ \alpha$
    \item $tan~^{2}~\alpha - sec~^{2}~\alpha$
    \end{enumerate}
\item Solve for $\theta$ if $\theta$ is a positive, acute angle:
    \begin{enumerate}[noitemsep, label=\textbf{(\alph*)} ]
    \item $2 sin~ \theta = 1,34$
    \item $1 - tan~ \theta = -1$
    \item $cos~ 2\theta = sin~ 40^{\circ}$ 
    \item $\dfrac{sin~ \theta}{cos~ \theta}= 1$
    \end{enumerate}
\item Calculate the unknown lengths in the diagrams below:
\begin{center}
\scalebox{1}  
{ 
\begin{pspicture}(0,-2.0390613)(8.035,2.0253136) 
\psline[linewidth=0.04cm](0.02,0.02093862)(3.06,0.02093862) 
\psline[linewidth=0.04cm](0.02,0.02093862)(0.02,-2.0190613) 
\psline[linewidth=0.04cm](0.04,-1.9990613)(3.04,0.02093862) 
\psframe[linewidth=0.04,dimen=outer](0.26,0.04093862)(0.0,-0.21906137) 
\psline[linewidth=0.04cm](0.9944844,1.4531701)(3.04,0.04093862) 
\psline[linewidth=0.04cm](0.96,1.4209386)(0.024381146,0.02542873) 
\rput{-33.90198}(-0.5512336,0.79249614){\psframe[linewidth=0.04,dimen=outer](1.1544285,1.4305158)(0.8944285,1.1705158)} 
\psline[linewidth=0.04cm](2.2384837,1.992771)(3.04,0.04093862) 
\psline[linewidth=0.04cm](2.238,1.9929386)(1.0103352,1.4765087) 
\rput{-66.64198}(-0.3803531,3.117777){\psframe[linewidth=0.04,dimen=outer](2.3111107,1.9781737)(2.0511107,1.7181737)} 
\rput(2.2601562,-0.21406138){$30^{\circ}$} 
\rput(2.3034375,0.22593862){$25^{\circ}$} 
\rput(2.4634376,0.77){$20^{\circ}$} 
\rput(1.8576562,-1.2540613){$16$ cm} 
\rput(1.3398438,0.20593862){$a$} 
\rput(1.9,1.1){$b$} 
\rput(1.58,1.9059386){$c$} 
\psline[linewidth=0.04cm](6.689009,-1.6241995)(6.0199666,-1.4501117) 
\rput{-90.46057}(4.6635084,7.686732){\psframe[linewidth=0.04,dimen=outer](6.2543488,1.6402806)(6.0343485,1.4202806)} 
\rput{-90.46057}(4.4806204,7.508202){\psframe[linewidth=0.04,dimen=outer](6.0743546,1.6417276)(5.8543544,1.4217275)} 
\rput{-104.9971}(9.7950115,4.4890895){\psframe[linewidth=0.04,dimen=outer](6.7298956,-1.4036404)(6.509896,-1.6236403)} 
\psline[linewidth=0.04cm](7.375193,1.6303898)(4.2549725,1.6154709) 
\psline[linewidth=0.04cm](6.055236,1.6410005)(6.030317,-1.4588993) 
\psline[linewidth=0.04cm](6.0304775,-1.4389)(4.2748113,1.5953108) 
\psline[linewidth=0.04cm](7.375193,1.6303898)(6.030317,-1.4588993) 
\psline[linewidth=0.04cm](7.375193,1.6303898)(6.689009,-1.6241995) 
\rput(4.880625,0.005938619){$d$} 
\rput(6.7414064,1.8259386){$e$} 
\rput(5.2226562,1.8259386){$5$ m} 
\rput(6.2709374,-1.7940614){$f$} 
\rput(7.2148438,-0.07406138){$g$} 
\rput(4.756875,1.3859386){ $50^{\circ}$} 
\rput(6.9460936,1.4059386){$60^{\circ}$} 
\rput(6.45,-1.2740613){$80^{\circ}$} 
\end{pspicture} 
}
\end{center}
\item In $\triangle PQR$, $PR=20$ cm, $QR=22$ cm and $P\hat{R}Q = 30^{\circ}$. The perpendicular line from $P$ to $QR$ intersects $QR$ at $X$. Calculate 
\begin{enumerate}[noitemsep, label=\textbf{(\alph*)} ]
\item the length $XR$ 
\item the length $PX$
\item the angle $Q\hat{P}X$ 
\end{enumerate} 
\item A ladder of length $15$ m is resting against a wall, the base of the ladder is $5$ m from the wall. Find the angle between the wall and the ladder. 
\item In the following triangle find the angle $A\hat{B}C$:
\begin{center}
\begin{pspicture}(0,-2.4701562)(5.49875,2.4701562) 
\pspolygon[linewidth=0.04](0.1665625,-1.7301563)(3.3665626,1.9698437)(5.1665626,-1.7301563)(4.1665626,-1.7301563) 
\psline[linewidth=0.04cm](3.3665626,1.9698437)(3.3665626,-1.7301563) 
\rput(3.3871875,2.2798438){$A$} 
\rput(5.3459377,-2.0201561){$B$} 
\rput(3.371875,-2.0201561){$C$} 
\rput(0.07546875,-2.0201561){$D$} 
\rput(3.6,0){$9$} 
\rput(2.7525,-2.3201563){$17$} 
\psline[linewidth=0.04cm,arrowsize=0.05291667cm 2.0,arrowlength=1.4,arrowinset=0.4]{->}(3.0665624,-2.3301563)(5.2665625,-2.3301563) 
\psline[linewidth=0.04cm,arrowsize=0.05291667cm 2.0,arrowlength=1.4,arrowinset=0.4]{->}(2.4665625,-2.3301563)(0.0665625,-2.3301563) 
\psline[linewidth=0.04cm](3.3665626,-1.5301563)(3.5665624,-1.5301563) 
\psline[linewidth=0.04cm](3.5665624,-1.5301563)(3.5665624,-1.7301563) 
\rput(0.8,-1.48){$41^{\circ}$} 
\end{pspicture} 
\end{center}
\item In the following triangle find the length of side $CD$:
\begin{center}
\begin{pspicture}(0,-2.2234375)(6.091875,2.2234375) 
\pspolygon[linewidth=0.04](0.1665625,-1.776875)(5.1665626,-1.776875)(5.1665626,1.823125) 
\psline[linewidth=0.04cm](3.4665625,-1.776875)(5.1665626,1.823125) 
\rput(5.2871876,2.033125){$A$} 
\rput(5.3459377,-2.066875){$B$} 
\rput(3.471875,-2.066875){$C$} 
\rput(0.07546875,-2.066875){$D$} 
\rput(5.4,0){$9$} 
\rput(4.490156,0.95){$15^{\circ}$} 
\rput(3.960156,-1.5){$35^{\circ}$} 
\psline[linewidth=0.04cm](4.9665626,-1.576875)(5.1665626,-1.576875) 
\psline[linewidth=0.04cm](4.9665626,-1.576875)(4.9665626,-1.776875) 
\end{pspicture}
\end{center} 
\item $A(5;0)$ and $B(11;4)$. Find the angle between the line through $A$ and $B$ and the $x$-axis. 
\item $C(0;-13)$ and $D(-12;14)$. Find the angle between the line through $C$ and $D$ and the $y$-axis. 
\item A right-angled triangle has hypotenuse $13$ mm. Find the length of the other two sides if one of the angles of the triangle is $50^{\circ}$.
\item One of the angles of a rhombus with perimeter $20$ cm is $30^{\circ}$. 
\begin{enumerate}[noitemsep, label=\textbf{(\alph*)} ]
\item Find the sides of the rhombus. 
\item Find the length of both diagonals. 
\end{enumerate} 
\item Captain Jack is sailing towards a cliff with a height of $10$ m. 
\begin{enumerate}[noitemsep, label=\textbf{(\alph*)} ] 
\item The distance from the boat to the top of the cliff is $30$ m. Calculate the angle of elevation from the boat to the top of the cliff (correct to the nearest integer).
\item If the boat sails $7$ m closer to the cliff, what is the new angle of elevation from the boat to the top of the cliff? 
\end{enumerate} 
\item Given the points: $E(5;0)$, $F(6;2)$ and $G(8;-2)$. Find the angle $F\hat{E}G$. 
\item  A triangle with angles $40^{\circ}$, $40^{\circ}$ and $100^{\circ}$ has a perimeter of $20$ cm. Find the length of each side of the triangle. 
\end{enumerate}

\end{eocexercises}


 \begin{eocsolutions}{}{
\begin{enumerate}[itemsep=6pt, label=\textbf{\arabic*}. ] 


\item 
$sin~ 60^{\circ} cos~ 30^{\circ}-cos~ 60^{\circ}sin~ 30^{\circ} + tan~ 45^{\circ}\\[4pt]
=\left(\dfrac{\sqrt{3}}{2} \times \dfrac{\sqrt{3}}{2}\right)  - \left(\dfrac{1}{2} \times \dfrac{1}{2}\right) +1\\[4pt]
=\dfrac{3}{4} - \dfrac{1}{4} +1\\[4pt]
= \dfrac{2}{4} +1\\[4pt]
=1\frac{1}{2}$
\item 
\scalebox{1} % Change this value to rescale the drawing.
{
\begin{pspicture}(0,-2.1567187)(4.5735936,2.196719)
\rput(2.0,-0.15671882){\psaxes[linewidth=0.04,arrowsize=0.05291667cm 2.0,arrowlength=1.4,arrowinset=0.4,labels=none,ticks=none,ticksize=0.10583333cm]{<->}(0,0)(-2,-2)(2,2)}
\usefont{T1}{ptm}{m}{n}
\rput(4.1990623,0.013281175){$x$}
\usefont{T1}{ptm}{m}{n}
\rput(2.0990624,1.9932814){$y$}
\psline[linewidth=0.04cm](2.0,-0.17671883)(1.0,1.2567189)
\psline[linewidth=0.04cm,linestyle=dashed,dash=0.16cm 0.16cm](0.96,-0.083281174)(0.96,1.1967188)
\psdots[dotsize=0.12](1.0,1.2432811)
\psframe[linewidth=0.04,dimen=outer](1.18,0.06328118)(0.94,-0.17671883)
\usefont{T1}{ptm}{m}{n}
\rput(1.7790625,-0.34671885){$0$}
\usefont{T1}{ptm}{m}{n}
\rput(0.94906247,1.5332813){$(-3;-5)$}
\psarc[linewidth=0.04,arrowsize=0.05291667cm 2.0,arrowlength=1.4,arrowinset=0.4]{->}(2.03,-0.10671882){0.49}{356.9872}{129.05815}
\usefont{T1}{ptm}{m}{n}
\rput(2.2990625,0.13328117){$\alpha$}
\end{pspicture} 
}\\
$3~tan ~\alpha = -5$\\
$tan~\alpha =\frac{-5}{3}\\
\therefore x = -3;~ y = 5\\
\therefore r^2 = x^2+y^2\\
=(-3)^2 + (5)^2\\
=34\\
\therefore r=\sqrt{34}\\$

    \begin{enumerate}[itemsep=1pt, label=\textbf{(\alph*)} ]
\item $cos~\alpha = \frac{x}{r} = \frac{-3}{\sqrt{34}}$%$cos~\alpha$
    \item $tan^2\alpha - sec^2\alpha \\
=\left(\frac{y}{x}\right)^2 - \left(\frac{r}{x}\right)^2\\
=\left(\frac{5}{-3}\right)^2 - \left(\frac{\sqrt{34}}{-3}\right)^2\\
=\frac{25}{9} - \frac{34}{9}\\
= \frac{-9}{9}\\
=-1$%$tan^{2}~\alpha - sec^{2}~\alpha$
    \end{enumerate}

\item    
\begin{multicols}{2}
 \begin{enumerate}[itemsep=3pt, label=\textbf{(\alph*)} ]
     \item $2 ~sin~\theta= 1,34\\sin~\theta=\frac{1,34}{2}\\\therefore \theta=42,07^{\circ}$%$2~sin~\theta = 1,34$
    \item $1-tan~\theta = -1\\ -tan~\theta = -2\\tan~\theta = 2\\ \therefore \theta=63,43^{\circ}$%$1 - tan~\theta = -1$
    \item $cos~2\theta = sin~40^{\circ}\\ cos~2\theta = 0,64\\ \therfore 2\theta = 50^{\circ}\\ \theta = 25^{\circ}$%$cos~2\theta = sin~40^{\circ}$ 
    \item $\dfrac{sin~\theta}{cos~\theta} = 1\\ \therefore tan~\theta = 1\\ \therefore \theta=45^{\circ}$%$\dfrac{sin~\theta}{cos~\theta}= 1$
    \end{enumerate}
\end{multicols}
\item 
\begin{multicols}{2}
    \begin{enumerate}[itemsep=3pt, label=\textbf{(\alph*)} ]
\item$cos~30^{\circ} =\frac{a}{16}\\ a = 16~cos~30^{\circ} \\ \therfore a=13,86$ cm
\item$cos~25^{\circ} =\frac{b}{13,86}\\\therefore b=12,56$ cm
\item$sin~20^{\circ} =\frac{c}{12,56}\\ \therefore c=4,30$ cm
\item$cos~50^{\circ} =\frac{5}{d} \\ \therefore d=7,78$ cm
\item By Pythagoras the third side is \\$\sqrt{5^2 + 7,779^2} = 9,246$ cm\\

$\therefore tan~60^{\circ} = \frac{9,247}{e}\\ \therefore e=5,34$ cm
\item By Pythagoras the third side is \\$\sqrt{5,339^2 + 7,779^2} = 9,435$ cm\\

$\therefore f = \sqrt{9,345^2-1,648^2} = 9,2$ cm
\item$tan~80^{\circ}=\frac{9,435}{g}\\ \therefore g=1,65$ cm
\end{multicols}
      \end{enumerate}
\item First draw a diagram:\\
\scalebox{1} % Change this value to rescale the drawing.
{
\begin{pspicture}(0,-1.94375)(4.2078123,1.94375)
\psline[linewidth=0.04](2.1246874,1.48125)(0.2646875,-1.11875)(3.8646874,-1.09875)(2.1246874,1.46125)
\psline[linewidth=0.04cm](2.1046875,1.48125)(2.1046875,-1.13875)
% \usefont{T1}{ptm}{m}{n}
\rput(2.149375,1.75125){$P$}
% \usefont{T1}{ptm}{m}{n}
\rput(4.026094,-1.32875){$Q$}
% \usefont{T1}{ptm}{m}{n}
\rput(0.066875,-1.36875){$R$}
% \usefont{T1}{ptm}{m}{n}
\rput(2.1170313,-1.36875){$X$}
% \usefont{T1}{ptm}{m}{n}
\rput(0.7415625,0.61125){$20$ cm}
% \usefont{T1}{ptm}{m}{n}
\rput(2.0615625,-1.78875){$22$ cm}
% \usefont{T1}{ptm}{m}{n}
\rput(0.7834375,-0.92875){$30^{\circ}$}
\end{pspicture} 
}
\begin{multicols}{2}
\begin{enumerate}[itemsep=3pt, label=\textbf{(\alph*)} ]
\item $cos~30^{\circ} = \frac{XR}{20}\\ XR = 20~cos~30^{\circ}\\ XR= 17,32$ cm%the length $XR$ 
\item $sin~30^{\circ} = \frac{PX}{20} \\ PX = 20~sin~30^{\circ}\\ PX = 10$ cm%the length $PX$
\item $tan~ Q\hat{P}X =\frac{4,68}{10}\\ Q\hat{P}X= 25,08^{\circ}$%the angle $Q\hat{P}X$ 
\end{enumerate} 
\end{multicols}
\item First draw a diagram:\\
\scalebox{1} % Change this value to rescale the drawing.
{
\begin{pspicture}(0,-1.6146259)(2.544252,1.230374)
\psline[linewidth=0.04](0.64425194,1.210374)(0.62425196,-1.129626)(2.504252,-1.129626)(0.62425196,1.190374)
\psframe[linewidth=0.04,dimen=outer](0.94425195,-0.809626)(0.62425196,-1.129626)
% \usefont{T1}{ptm}{m}{n}
\rput(2.0444083,0.260374){$15$ m}
% \usefont{T1}{ptm}{m}{n}
\rput(1.3803457,-1.459626){$5$ m}
\rput{28.266916}(0.49617285,-0.22972818){\psarc[linewidth=0.04](0.70425195,0.870374){0.52}{231.11343}{296.74106}}
% \usefont{T1}{ptm}{m}{n}
\rput(0.7997207,0.640374){$x$}
\end{pspicture} 
}\\
$sin~x = \frac{5}{15}\\
\therefore x = 19,47^{\circ}$
\item 
$tan~41^{\circ} = \frac{9}{DC}\\
DC = 9~tan~41^{\circ}\\
= 7,82$ units\\
$BC=BD-DC = 9,18$ units\\
$tan~A\hat{B}C = \frac{9}{9,18}\\
\therefore A\hat{B}C = 44,43^{\circ}$
\item $C\hat{A}B=180^{\circ} - 90^{\circ}-35^{\circ}=55^{\circ}\\
\therefore D\hat{A}B = 15^{\circ}+55^{\circ}=70^{\circ}\\
tan~35^{\circ} = \frac{9}{BC}\\
\therefore BC=12,85$ units\\
$tan~70^{\circ}=\frac{BD}{9}\\
BD=24,73$ units\\
$CD=BD-BC=11,88$ units
\item %question 9
First draw a diagram:\\
\scalebox{1} % Change this value to rescale the drawing.
{
\begin{pspicture}(0,-2.0329165)(7.2703643,2.0663543)
\rput(0.74583334,-1.3470832){\psaxes[linewidth=0.04,arrowsize=0.05291667cm 2.0,arrowlength=1.4,arrowinset=0.4,ticks=all,dx=0.5cm,dy=0.5cm]{<->}(0,0)(-0.3,-0.3)(6,3)}
\psline[linewidth=0.04cm](3.0058334,-1.5670834)(6.6058335,1.6929166)
\psline[linewidth=0.04cm](6.2458334,2.0329165)(6.2458334,-1.6470833)
\psframe[linewidth=0.04,dimen=outer](6.2658334,-1.1136458)(6.005833,-1.3670833)
\psarc[linewidth=0.04](3.4958334,-1.3170834){0.65}{-3.0127876}{52.495857}
\usefont{T1}{ppl}{m}{n}
\rput(3.7558331,-1.1170832){$x$}
\usefont{T1}{ppl}{m}{n}
\rput(6.895833,-1.0970832){$x$}
\usefont{T1}{ppl}{m}{n}
\rput(0.97614586,1.8629167){$y$}
\psdots[dotsize=0.12](6.2658334,1.3929168)
\psdots[dotsize=0.12](3.2658334,-1.3470832)
\usefont{T1}{ppl}{m}{n}
\rput(2.9,-1.0570834){$A(5;0)$}
\usefont{T1}{ppl}{m}{n}
\rput(5.6,1.6){$B(11;4)$}
\end{pspicture} 
}\\
Note that the distance from $B$ to the $x$-axis is $4$ units and that the distance $AC$ ia $11-5=6$ units.\\
$tan~x = \frac{4}{6}\\
\therefore x= 33,69^{\circ}$
\item 
First draw a diagram:\\%question 10
\scalebox{0.7} % Change this value to rescale the drawing.
{
\begin{pspicture}(0,-5.1267185)(7.6245313,5.166719)

\rput(6.0,-0.12671874){\psaxes[linewidth=0.04,arrowsize=0.05291667cm 2.0,arrowlength=1.4,arrowinset=0.4,ticks=all,dx=1cm,dy=1cm,Dx=2,Dy=4]{<->}(0,0)(-6,-5)(1,5)}
\psline[linewidth=0.04cm](0.35416666,3.3867188)(6.3941665,3.4067187)
\psline[linewidth=0.04cm](0.49416664,4.206719)(6.574167,-4.173281)
\psframe[linewidth=0.04,dimen=outer](6.02,3.4267187)(5.6941667,3.1067188)
\psarc[linewidth=0.04](5.75,-2.6167188){0.65}{64.76399}{149.26799}
\usefont{T1}{ppl}{m}{n}
\rput(5.69,-2.4567187){\LARGE$x$}
\usefont{T1}{ppl}{m}{n}
\rput(7.25,0.02328125){\LARGE$x$}
\usefont{T1}{ppl}{m}{n}
\rput(6.2503123,4.963281){\LARGE$y$}
\psdots[dotsize=0.18](6.0,-3.3667188)
\psdots[dotsize=0.18](1.0600001,3.3932812)
\usefont{T1}{ppl}{m}{n}
\rput(2.024844,3.6832812){\LARGE$D(-12;14)$}
\usefont{T1}{ppl}{m}{n}
\rput(7,-3.3767188){\LARGE$C(0;-13)$}
\end{pspicture} 
}\\
The distance from $d$ to the $y$-axis is $12$ units. The distance from $C$ to the point where the perpendicular line from $d$ intercepts the $y$-axis is $14-(-13)=27$ units.\\
$tan~x = \frac{12}{27}\\
\therefore x = 23,96^{\circ}$
\item First draw a diagram:\\ %question 11
\scalebox{1} % Change this value to rescale the drawing.
{
\begin{pspicture}(0,-1.6425)(3.0071876,1.6225)
\psline[linewidth=0.04](0.4271875,1.6025)(0.4271875,-1.2375)(2.3871875,-1.2375)(0.4471875,1.5225)
% \usefont{T1}{ptm}{m}{n}
\rput(1.8,0.5325){$13$ mm}
% \usefont{T1}{ptm}{m}{n}
\rput(1.2103125,-1.4875){$a$}
% \usefont{T1}{ptm}{m}{n}
\rput(0.05515625,0.4325){$b$}
\psframe[linewidth=0.04,dimen=outer](0.7071875,-0.9575)(0.4071875,-1.2575)
\psarc[linewidth=0.04](0.5671875,0.9825){0.44}{250.01689}{333.43494}
% \usefont{T1}{ptm}{m}{n}
\rput(0.68,0.8725){\scriptsize$50^{\circ}$}
\end{pspicture} 
}\\
$sin~50^{\circ} = \frac{a}{13}\\
a=13~sin~50^{\circ}\\
\therefore a = 9,96$ mm\\
Use the theorem of Pythagoras to find the other side:\\
$b^2=c^2-a^2\\
=13^2-9,96^2\\
\therefore b = \sqrt{69,7984}\\
=8,35$ mm

\item First draw a diagram:\\%question 12

\scalebox{1} % Change this value to rescale the drawing.
{
\begin{pspicture}(0,-1.1795312)(5.45125,1.4267187)
\definecolor{color6923b}{rgb}{0.996078431372549,0.996078431372549,0.996078431372549}
\psline[linewidth=0.04](0.49125,-0.84671867)(2.63125,1.0732813)(5.43125,1.0532813)(3.51125,-0.80671877)(0.51125,-0.84671867)
\psline[linewidth=0.04cm](5.39125,1.0532813)(0.53125,-0.84671867)
\psline[linewidth=0.04cm](2.6285787,1.0788563)(3.51125,-0.8267187)
\rput{24.625969}(0.40700373,-1.2780948){\psframe[linewidth=0.04,dimen=outer](3.27125,0.43328127)(2.99125,0.15328126)}
\psarc[linewidth=0.04,fillstyle=solid,fillcolor=color6923b](0.83125,-0.7867187){0.6}{355.23636}{59.036243}
\usefont{T1}{ptm}{m}{n}
\rput(3.9089062,1.2232813){$a$}
\usefont{T1}{ptm}{m}{n}
\rput(2.0137498,-0.0567187){$b$}
\usefont{T1}{ptm}{m}{n}
\rput(2.9635937,0.6632813){$c$}
\usefont{T1}{ptm}{m}{n}
\rput(4.648906,0.0632813){$a$}
\usefont{T1}{ptm}{m}{n}
\rput(2.1889062,-0.97671866){$a$}
\usefont{T1}{ptm}{m}{n}
\rput(1.6089063,0.4632813){$a$}
\psline[linewidth=0.04,linecolor=color6923b,fillstyle=solid](0.66,-0.75328124)(1.2312195,-0.41328126)(1.36,-0.73328125)(0.58731705,-0.7982195)(0.58731705,-0.7982195)
\usefont{T1}{ptm}{m}{n}
\rput(1.1245313,-0.6367187){$30^{\circ}$}
\end{pspicture} 
}\\
\begin{enumerate}[noitemsep, label=\textbf{(\alph*)} ]
\item The perimeter is found by adding each side together. All the sides are equal in length therefore perimeter$=4a$\\
$20=4a\\
\therefore a = 5$ cm

\item The diagonals of a rhombus bisect the angle, so working in one of the small triangles we can use trigonometric identities to find $b$:
\\ $cos~15^{\circ} = \frac{b}{5}\\
\therefore b = 4,83$ cm\\
By Pythagoras $c^2=a^2-b^2\\
\therefore c= \sqrt{25-23,33}\\
c= 1,29$ cm\\
Since the diagonals bisect each other we know that the total length of each diagonal is either $2b$ or $2c$, depending on which diagonal we examine.\\
The one diagonal $=4,8 \times 2 = 9,6$ cm\\
The other diagonal $=1,29 \times 2 = 2,58$ cm
\end{enumerate} 

\item First draw a diagram:\\ %solution 13
\scalebox{1} % Change this value to rescale the drawing.
{
\begin{pspicture}(0,-1.5825)(4.3603125,1.5625)
\psline[linewidth=0.04](0.9803125,1.5025)(0.9603125,-0.6775)(4.3203125,-0.6575)(0.9603125,1.5425)
\psline[linewidth=0.04cm,linestyle=dashed,dash=0.16cm 0.16cm](1.0403125,1.4825)(2.7803125,-0.5775)
% \usefont{T1}{ptm}{m}{n}
\rput(3.6157813,-0.4275){$x$}
% \usefont{T1}{ptm}{m}{n}
\rput(2.3760939,-0.5){$y$}
% \usefont{T1}{ptm}{m}{n}
\rput(0.38046876,0.4525){$10$ m}
% \usefont{T1}{ptm}{m}{n}
\rput(3.5954688,-0.9275){$7$ m}
\psframe[linewidth=0.04,dimen=outer](1.3203125,-0.3375)(0.9603125,-0.6975)
% \usefont{T1}{ptm}{m}{n}
\rput(2.7265625,-1.4275){$30$ m}
\psline[linewidth=0.04cm,arrowsize=0.05291667cm 2.0,arrowlength=1.4,arrowinset=0.4]{<->}(0.9803125,-1.1575)(4.3403125,-1.1375)
\psarc[linewidth=0.04](3.8503125,-0.4475){0.51}{137.72632}{205.70996}
\psarc[linewidth=0.04](2.6703124,-0.5475){0.55}{124.215706}{195.75117}
\end{pspicture} 
}
\begin{enumerate}[noitemsep, label=\textbf{(\alph*)} ] 
\item $tan~x = \frac{10}{30}\\
\therefore x = 18^{\circ}$%The distance from the boat to the top of the cliff is $30$ m, calculate the angle of elevation from the boat to the top of the cliff (correct to the nearest integer).
\item $tan~y=\frac{10}{23}\\
\therefore y= 23^{\circ}$%If the boat sails $7$ m closer to the cliff, what is the new angle of elevation from the boat to the top of the cliff? 
\end{enumerate} 

\item First draw a diagram:\\%solution 14
\scalebox{0.7} % Change this value to rescale the drawing.
{
\begin{pspicture}(0,-3.2484374)(10.167084,3.2884376)
\rput(0.8858333,-0.2484375){\psaxes[linewidth=0.04,ticksize=0.10583333cm]{<->}(0,0)(-0.5,-3)(9,3)}
% \usefont{T1}{ptm}{m}{n}
\rput(0.9816146,3.1615624){\LARGE$y$}
% \usefont{T1}{ptm}{m}{n}
\rput(10.021302,0.1415625){\LARGE$x$}
\psline[linewidth=0.04](5.8658333,-0.2284375)(6.8658333,1.6115625)(8.905833,-2.2284374)(5.8258333,-0.2484375)
\psdots[dotsize=0.12](8.925834,-2.2484374)
\psdots[dotsize=0.12](5.9058332,-0.2484375)
\psdots[dotsize=0.12](6.8658333,1.5915625)
% \usefont{T1}{ptm}{m}{n}
\rput(7.1928644,1.9015625){\LARGE$F(6;2)$}
% \usefont{T1}{ptm}{m}{n}
\rput(5,0.0615625){\LARGE$E(5;0)$}
% \usefont{T1}{ptm}{m}{n}
\rput(9.410364,-2.4584374){\LARGE$G(8;-2)$}
% \usefont{T1}{ptm}{m}{n}
\rput(7.1,-0.0615625){\LARGE$A$}
% \usefont{T1}{ptm}{m}{n}
\rput(8.913802,0.0615625){\LARGE$B$}
\psline[linewidth=0.04cm,linestyle=dashed,dash=0.16cm 0.16cm](8.885834,-0.2684375)(8.925834,-2.2084374)
\psline[linewidth=0.04cm,linestyle=dashed,dash=0.16cm 0.16cm](6.8658333,1.5715625)(6.8858333,-0.2084375)
\end{pspicture} 
}\\
To find $F\hat{E}G$ we inspect triangles $FEA$ and $GEB$ in turn. \\
In $\triangle FEA$\\
$tan~F\hat{E}A = \frac{2}{1}\\
\therefore F\hat{E}A = 63,43^{\circ}$\\
In $\triangle GEB\\
tan~ G\hat{E}B = \frac{2}{3}\\
\therefore G\hat{E}B = 33,69^{\circ}\\$
Add these two angles together to get the value of $F\hat{E}G$:\\
$F\hat{E}G = F\hat{E}B + G\hat{E}B\\=33,69^{\circ} + 63,43^{\circ}\\
\therefore F\hat{E}G = 97,12^{\circ}$
\item First draw a diagram:\\ % solution 15
\scalebox{1} % Change this value to rescale the drawing.
{
\begin{pspicture}(0,-1.0655637)(6.423424,1.0344363)
\psline[linewidth=0.04](0.2234236,-1.0455636)(2.9234235,1.0144364)(6.403424,0.034436367)(0.2434236,-1.0455636)
\psline[linewidth=0.04cm](2.9434235,0.9944364)(3.2834237,-0.52556366)
\rput{10.167775}(-0.009474515,-0.60462916){\psframe[linewidth=0.04,dimen=outer](3.5634236,-0.18556364)(3.2234237,-0.52556366)}
\psarc[linewidth=0.04](5.4534235,0.0044363677){0.67}{138.81407}{201.0375}
\rput{-144.89656}(2.154564,-0.90965706){\psarc[linewidth=0.04](0.9334236,-0.79556364){0.67}{143.16086}{201.0375}}
% \usefont{T1}{ptm}{m}{n}
\rput(5.1271734,0.06443637){$40^{\circ}$}
% \usefont{T1}{ptm}{m}{n}
\rput(4.626549,0.82443637){$a$}
% \usefont{T1}{ptm}{m}{n}
\rput(3.02608,0.6844364){$100^{\circ}$}
% \usefont{T1}{ptm}{m}{n}
\rput(3.3713923,-0.8755636){$b$}
\rput{3.2352896}(0.064875826,-0.16837043){\psarc[linewidth=0.04](3.0134237,1.0644363){0.63}{216.36922}{332.61945}}
% \usefont{T1}{ptm}{m}{n}
\rput(1.1271735,-0.6355636){$40^{\circ}$}
% \usefont{T1}{ptm}{m}{n}
\rput(1.7065486,0.42443636){$a$}
\end{pspicture} 
}\\
Construct a perpendicular bisector to create a right-angled triangle. \\
Therefore  we have: \\
$2a+b = 20\\
b=2(10-a)\\
\therefore cos~50^{\circ} = \dfrac{\frac{b}{2}}{a}\\[4pt]
0,77 = \dfrac{\frac{2(10-a)}{2}}{a}\\[4pt]
0,77=\dfrac{10-a}{a}\\[4pt]
0,77a = 10-a\\
\therefore a=5,56$ cm\\
$\therefore b = 2(10-5,56) = 8,7$ cm


\end{enumerate}}
\end{eocsolutions}


