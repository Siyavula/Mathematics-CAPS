\chapter{Trigonometry}

\begin{exercises}{}
{
\begin{enumerate}[itemsep=5pt, label=\textbf{\arabic*}. ]
\item In each of the following triangles, state whether $a$, $b$ and $c$ are the hypotenuse, opposite or adjacent sides of the triangle with respect to $\theta$. 
\begin{center}
\scalebox{0.85} % Change this value to rescale the drawing.
{
\begin{pspicture}(0,-3.7754679)(20.177345,3.9279697)
\rput (1,2){\textbf{(a)}}
\rput (5.5,2){\textbf{(b)}}
\rput (10.5,2){\textbf{(c)}}
\rput (1.5,-1.5){\textbf{(d)}}
\rput (6,-1.5){\textbf{(e)}}
\rput (9.5,-1.5){\textbf{(f)}}
\psdots[dotsize=0.027999999](2.9173439,-1.5720303)
\psline[linewidth=0.04cm](5.517344,2.5079696)(7.9173436,0.107969664)
\psline[linewidth=0.04cm](7.9173436,0.107969664)(9.0173435,1.2079697)
\psline[linewidth=0.04cm](5.517344,2.5079696)(9.0173435,1.2079697)
\psline[linewidth=0.04cm](7.717344,0.30796966)(7.9173436,0.5079697)
\psline[linewidth=0.04cm](7.9173436,0.5079697)(8.117344,0.30796966)
\psline[linewidth=0.04cm](11.0573435,2.2679696)(11.0573435,0.067969665)
\psline[linewidth=0.04cm](11.0573435,0.067969665)(14.757344,0.067969665)
\psline[linewidth=0.04cm](11.0573435,2.2679696)(14.757344,0.067969665)
\psline[linewidth=0.04cm](1.2373447,-3.3120303)(3.9373438,-0.71203035)
\psline[linewidth=0.04cm](3.9373438,-0.71203035)(3.9373438,-3.3120303)
\psline[linewidth=0.04cm](3.9373438,-3.3120303)(1.2373447,-3.3120303)
\psline[linewidth=0.04cm](7.477345,-1.2120303)(5.7773438,-3.3120303)
\psline[linewidth=0.04cm](7.477345,-1.2120303)(8.9773445,-2.4120302)
\psline[linewidth=0.04cm](8.9773445,-2.4120302)(5.7773438,-3.3120303)
\psline[linewidth=0.04cm](7.257343,-1.5120304)(7.537344,-1.7320304)
\psline[linewidth=0.04cm](7.517344,-1.7120303)(7.717344,-1.4320303)
\psline[linewidth=0.04cm](11.097343,0.38796967)(11.397344,0.38796967)
\psline[linewidth=0.04cm](11.397344,0.38796967)(11.397344,0.08796967)
\psline[linewidth=0.04cm](3.9173427,-3.0320303)(3.6173437,-3.0320303)
\psline[linewidth=0.04cm](3.6173437,-3.0320303)(3.6173437,-3.3320303)
\psline[linewidth=0.04cm](10.0173435,-1.7320304)(11.417343,-3.2320304)
\psline[linewidth=0.04cm](11.417343,-3.2320304)(13.0173435,-1.7320304)
\psline[linewidth=0.04cm](10.0173435,-1.7320304)(13.0173435,-1.7320304)
\psline[linewidth=0.04cm](11.217343,-3.0320303)(11.417343,-2.8320303)
\psline[linewidth=0.04cm](11.417343,-2.8320303)(11.617344,-3.0320303)
\usefont{T1}{ptm}{m}{n}
\rput(8.662344,1.1179696){$\theta$}
\usefont{T1}{ptm}{m}{n}
\rput(13.842344,0.29796967){$\theta$}
\usefont{T1}{ptm}{m}{n}
\rput(3.682343,-1.3020303){$\theta$}
\usefont{T1}{ptm}{m}{n}
\rput(6.362344,-2.9020302){$\theta$}
\usefont{T1}{ptm}{m}{n}
\rput(10.502343,-1.9420303){$\theta$}
\psline[linewidth=0.04cm](2.2973437,2.2879696)(0.09734377,0.18796967)
\psline[linewidth=0.04cm](0.09734377,0.18796967)(4.4973435,0.18796967)
\psline[linewidth=0.04cm](2.2973437,2.2879696)(4.4973435,0.18796967)
\psline[linewidth=0.04cm](2.0973437,2.0879698)(2.2973437,1.8879696)
\psline[linewidth=0.04cm](2.2973437,1.8879696)(2.4973438,2.0879698)
\usefont{T1}{ptm}{m}{n}
\rput(0.54234374,0.37796965){$\theta$}
\usefont{T1}{ptm}{m}{n}
\rput(1.1764063,1.5979697){$a$}
\usefont{T1}{ptm}{m}{n}
\rput(2.1778126,-0.10203034){$b$}
\usefont{T1}{ptm}{m}{n}
\rput(3.759219,1.2979697){$c$}
\usefont{T1}{ptm}{m}{n}
\rput(6.596406,1.0179696){$a$}
\usefont{T1}{ptm}{m}{n}
\rput(7.479219,2.1179698){$c$}
\usefont{T1}{ptm}{m}{n}
\rput(8.797812,0.41796967){$b$}
\usefont{T1}{ptm}{m}{n}
\rput(13.076406,1.4979696){$a$}
\usefont{T1}{ptm}{m}{n}
\rput(10.777813,1.1979697){$b$}
\usefont{T1}{ptm}{m}{n}
\rput(12.659219,-0.20203033){$c$}
\usefont{T1}{ptm}{m}{n}
\rput(2.896407,-3.6220303){$a$}
\usefont{T1}{ptm}{m}{n}
\rput(2.2978134,-1.8220303){$b$}
\usefont{T1}{ptm}{m}{n}
\rput(4.2792187,-2.3220303){$c$}
\usefont{T1}{ptm}{m}{n}
\rput(6.596406,-1.8220303){$a$}
\usefont{T1}{ptm}{m}{n}
\rput(7.7978134,-3.1220303){$b$}
\usefont{T1}{ptm}{m}{n}
\rput(8.579218,-1.7220303){$c$}
\usefont{T1}{ptm}{m}{n}
\rput(10.496406,-2.6220303){$a$}
\usefont{T1}{ptm}{m}{n}
\rput(12.397812,-2.8220303){$b$}
\usefont{T1}{ptm}{m}{n}
\rput(11.579218,-1.4220303){$c$}
\rput{-66.69996}(-0.02258055,0.761629){\psarc[linewidth=0.04](0.5673438,0.39796966){0.35}{25.785578}{138.26501}}
\rput{106.83747}(12.386083,-6.856496){\psarc[linewidth=0.04](8.737344,1.1679693){0.34}{32.561234}{147.21713}}
\rput{93.56388}(15.110441,-13.562675){\psarc[linewidth=0.04](13.927344,0.31796986){0.39}{24.333387}{127.3505}}
\psarc[linewidth=0.04](14.697344,3.1279697){0.0}{0.0}{180.0}
\psarc[linewidth=0.04](14.837344,3.2279696){0.0}{0.0}{180.0}
\rput{153.95303}(6.6359906,-3.7989676){\psarc[linewidth=0.04](3.757346,-1.1320316){0.5}{50.00946}{137.84839}}
\psarc[linewidth=0.04](20.097343,3.9079697){0.0}{0.0}{180.0}
\psarc[linewidth=0.04](20.117344,3.8879697){0.0}{0.0}{180.0}
\psarc[linewidth=0.04](20.157345,3.8279696){0.0}{0.0}{180.0}
\rput{-46.143967}(4.180272,3.6097476){\psarc[linewidth=0.04](6.3273435,-3.1020286){0.57}{58.40874}{128.71255}}
\rput{-116.039185}(16.474606,6.6625834){\psarc[linewidth=0.04](10.317342,-1.8120308){0.52}{49.23777}{123.63383}}
\end{pspicture} 
}
\end{center}
\item Use your calculator to determine the value of the following (correct to $2$ decimal places):
\begin{enumerate}[noitemsep, label=\textbf{(\alph*)} ]
\begin{multicols}{2} 
\item $tan~ 65^{\circ}$
\item $sin~ 38^{\circ}$
\item $cos~ 74^{\circ}$
\item $sin~ 12^{\circ}$
\item $cos~ 26^{\circ}$
\item $tan~ 49^{\circ}$
\item $\frac{1}{4} cos~ 20^{\circ}$
\item $3 tan~ 40^{\circ}$
\item $\frac{2}{3} sin~ 90^{\circ}$
\end{multicols}
\end{enumerate}
\item If $x=39^{\circ}$ and $y=21^{\circ}$, use a calculator to determine whether the following statements are true or false:
\begin{enumerate}[noitemsep, label=\textbf{(\alph*)} ]
\begin{multicols}{2} 
\item $cos~ x + 2 cos~ x = 3 cos~ x$
\item $cos~ 2y = cos~ y + cos~ y$
\item $tan~ x = \dfrac{sin~ x}{cos~ x}$
\item $cos~ (x+y) = cos~ x + cos~ y$
\end{multicols}
\end{enumerate}
\item Complete each of the following (the first example has been done
  for you):
\begin{center}
\setcounter{subfigure}{0}
\scalebox{1}{
\begin{pspicture}(0,-2.0990624)(4.21625,2.0990624)
\psline[linewidth=0.04cm](0.4775,-1.5590625)(3.7775,1.5409375)
\psline[linewidth=0.04cm](3.7775,1.5409375)(3.7775,-1.5590625)
\psline[linewidth=0.04cm](0.4775,-1.5590625)(3.7775,-1.5590625)
\psline[linewidth=0.04cm](3.4775,-1.5590625)(3.4775,-1.2590625)
\psline[linewidth=0.04cm](3.4775,-1.2590625)(3.7775,-1.2590625)
\rput(4.025,1.8759375){$A$}
\rput(3.874375,-1.9240625){ $B$}
\rput(0.10625,-1.8240625){$C$}
\end{pspicture} 
}
\end{center}
\begin{enumerate}[noitemsep, label=\textbf{(\alph*)} ]
\begin{multicols}{2}
\item $sin~ \hat{A} = \frac{\mbox{opposite}}{\mbox{hypotenuse}}=\dfrac{CB}{AC}$
\item $cos~ \hat{A} = $
\item $tan~ \hat{A} = $
\item $sin~ \hat{C} = $
\item $cos~ \hat{C} = $
\item $tan~ \hat{C} = $
\end{multicols}
\end{enumerate}
\item Use the triangle below to complete the following:
\begin{center}
\scalebox{1} % Change this value to rescale the drawing.
{
\begin{pspicture}(0,-1.9864062)(4.449063,1.9664062)
\psline[linewidth=0.04cm](0.66,-1.4935937)(3.1364062,-1.4935937)
\psline[linewidth=0.04cm](3.1364062,-1.4935937)(3.1,1.9464062)
\psline[linewidth=0.04cm](3.08,1.9464062)(0.68,-1.4935937)
\psline[linewidth=0.04cm](2.7364063,-1.4935937)(2.7364063,-1.0935937)
\psline[linewidth=0.04cm](2.7364063,-1.0935937)(3.1364062,-1.0935937)
\usefont{T1}{ptm}{m}{n}
\rput(1.4889063,0.49640626){$2$}
\usefont{T1}{ptm}{m}{n}
\rput(2.0890625,-1.7835938){$1$}
\usefont{T1}{ptm}{m}{n}
\rput(1.3,-1.2){$60^{\circ}$}
\usefont{T1}{ptm}{m}{n}
\rput(2.85,1.1){$30^{\circ}$}
\usefont{T1}{ptm}{m}{n}
\rput(3.4545314,0.01640625){$\sqrt{3}$}
\end{pspicture} 
}
\end{center}
\\
\begin{enumerate}[noitemsep, label=\textbf{(\alph*)} ]
\begin{multicols}{2}
\item $sin~ 60^{\circ} = $
\item $cos~ 60^{\circ} = $
\item $tan~ 60^{\circ}= $
\item $sin~ 30^{\circ}= $
\item $cos~ 30^{\circ}= $
\item $tan~ 30^{\circ}= $
\end{multicols}
\end{enumerate}
\item Use the triangle below to complete the following:
\begin{center}
\scalebox{1} % Change this value to rescale the drawing.
{
\begin{pspicture}(0,-2.24)(5.0271873,2.22)
\psline[linewidth=0.04cm](0.361875,-1.7)(4.061875,-1.7)
\psline[linewidth=0.04cm](4.061875,-1.7)(4.061875,2.2)
\psline[linewidth=0.04cm](4.061875,2.2)(0.361875,-1.7)
\psline[linewidth=0.04cm](3.661875,-1.7)(3.661875,-1.3)
\psline[linewidth=0.04cm](3.661875,-1.3)(4.061875,-1.3)
\rput(2.2314062,-2.09){$1$}
\rput(1.1554687,-1.39){$45^{\circ}$}
\rput(3.7554688,1.41){$45^{\circ}$}
\rput(1.6054688,0.51){$\sqrt{2}$}
\rput(4.331406,0.21){$1$}
\end{pspicture} 
}
\end{center}
\begin{enumerate}[noitemsep, label=\textbf{(\alph*)} ]
\item $sin~ 45^{\circ} = $
\item $cos~ 45^{\circ} = $
\item $tan~ 45^{\circ}= $
\end{enumerate}
\end{enumerate}

}
\end{exercises}


 \begin{solutions}{}{
\begin{enumerate}[itemsep=5pt, label=\textbf{\arabic*}. ] 


\item %solution 1
%In each of the following triangles, state whether $a$, $b$ and $c$ are the hypotenuse, opposite or adjacent sides of the triangle with respect to $\theta$. 
% \setcounter{subfigure}{0}
\begin{enumerate}[noitemsep, label=\textbf{(\alph*)} ]
\item $a=$adj; $b=$hyp; $c=$hyp
\item $a=$opp; $b=$adj; $c=$adj
\item $a=$hyp; $b=$opp; $c=$adj
\item $a=$opp; $b=$hyp; $c=$opp
\item $a=$adj; $b=$hyp; $c=$opp
\item $a=$adj; $b=$opp; $c=$hyp
\end{enumerate}

\item %solution 2
\begin{enumerate}[noitemsep, label=\textbf{(\alph*)} ]
\item $2,14$%$tan~65^{\circ}$
\item $0,62$%$sin~38^{\circ}$
\item $0,28$%$cos~74^{\circ}$
\item $0,21$%$sin~12^{\circ}$
\item $0,90$%$cos~26^{\circ}$
\item $1,15$%$tan~49^{\circ}$
\item $0,23$%$\frac{1}{4}~cos~20^{\circ}$
\item $2,52$%$3~tan~40^{\circ}$
\item $0,67$%$\frac{2}{3}~sin~90^{\circ}$
\end{enumerate}

\item %solution 3
\begin{enumerate}[noitemsep, label=\textbf{(\alph*)} ]
\item %$cos~x + 2~cos~x=3~cos~x$
    LHS $= cos~39^{\circ} + 2~cos~^{\circ}\\
	 = 2,33 \\$
    RHS $= 3~cos~^{\circ}\\
	= 2,33$ \\
LHS=RHS, therefore statement is true.
\item %$cos~2y = cos~y+cos~y$
    LHS $= cos~2(21^{\circ})\\
	 = 0,74 \\$
    RHS $= cos~21^{\circ}+cos~21^{\circ}\\
	= 1,87$ \\
LHS $\ne$ RHS, therefore statement is false.
\item %$tan~x=\dfrac{sin~x}{cos~x}$
    LHS $= tan~39^{\circ}\\
	 = 0,81 \\$
    RHS $= \frac{sin~39^{\circ}}{cos~39^{\circ}}\\
	= 0,81$ \\
LHS=RHS, therefore statement is true.
\item %$cos~(x+y) = cos~x+cos~y$
LHS $= cos~(39^{\circ}+21^{\circ})\\
	 = 0,5 \\$
    RHS $= cos~39^{\circ}+cos~21^{\circ}\\
	= 1,71$ \\
LHS $\ne$ RHS, therefore statement is false.
\end{enumerate}

\item %solution 4
\begin{enumerate}[itemsep=1pt, label=\textbf{(\alph*)} ]
\item $sin~\hat{A} = \frac{CB}{AC}$
\item $cos~\hat{A} = \frac{AB}{AC}$%$cos~\hat{A} = $
\item $tan~\hat{A} = \frac{CB}{AB}$%$tan~\hat{A}= $
\item $sin~\hat{C} = \frac{AB}{AC}$%$sin~\hat{C}= $
\item $cos~\hat{C} = \frac{CB}{AC}$%$cos~\hat{C}= $
\item $tan~\hat{C} = \frac{AB}{CB}$%$tan~\hat{C}= $
\end{enumerate}

\item %solution 5
\begin{enumerate}[itemsep=1pt, label=\textbf{(\alph*)} ]
\item $\frac{\sqrt{3}}{2}$%$sin~60^{\circ} = $
\item $\frac{1}{2}$%$cos~60^{\circ} = $
\item $\sqrt{3}$%$tan~60^{\circ}= $
\item $\frac{1}{2}$%$sin~30^{\circ}= $
\item $\frac{\sqrt{3}}{2}$%$cos~30^{\circ}= $
\item $\frac{1}{\sqrt{3}}$%$tan~30^{\circ}= $
\end{enumerate}

\item %solution 6
\begin{enumerate}[itemsep=1pt, label=\textbf{(\alph*)} ]
\item $\frac{1}{\sqrt{2}}$%$sin~45^{\circ} = $
\item $\frac{1}{\sqrt{2}}$%$cos~45^{\circ} = $
\item $1$%$tan~45^{\circ}= $
\end{enumerate}

\end{enumerate}}
\end{solutions}

\begin{exercises}{}{
\begin{enumerate}[itemsep=6pt, label=\textbf{\arabic*}. ] 
\item Calculate the value of the following without using a calculator:
\begin{enumerate}[noitemsep, label=\textbf{(\alph*)} ]
\item $sin~ 45^{\circ} \times cos~ 45^{\circ}$
\item $cos~ 60^{\circ} + tan~ 45^{\circ}$
\item $sin~ 60^{\circ} - cos~ 60^{\circ}$
\end{enumerate}
\item Use the table to  show that:
\begin{enumerate}[itemsep=5pt, label=\textbf{(\alph*)} ]
\item $\dfrac{sin~ 60^{\circ}}{cos~ 60^{\circ}} = tan~ 60^{\circ} $
\item $sin^{2} 45^{\circ}+ cos^{2} 45^{\circ} =1 $
\item $cos~ 30^{\circ} =\sqrt{1- sin^{2} 30^{\circ}}$
\end{enumerate}
\item Use the definitions of the trigonometric rations to answer the following questions:
\begin{enumerate}[noitemsep, label=\textbf{(\alph*)} ]
\item Explain why $sin ~\alpha \leq 1$ for all values of $\alpha$.
\item Explain why $cos~ \beta$ has a maximum value of $1$.
\item Is there a maximum value for $tan~\gamma$?
\end{enumerate}
\end{enumerate}
}
\end{exercises}


 \begin{solutions}{}{
\begin{enumerate}[itemsep=5pt, label=\textbf{\arabic*}. ] 

\item %solution 1
\begin{enumerate}[noitemsep, label=\textbf{(\alph*)} ]
\item $sin~ 45^{\circ} \times cos~ 45^{\circ}\\
      =\frac{1}{\sqrt{2}}\times \frac{1}{\sqrt{2}} \\
      =\frac{1}{2}$\\
\item $cos~ 60^{\circ} + tan~ 45^{\circ}\\
      =\frac{1}{2}+1 \\
      =1\frac{1}{2}$ \\
\item $sin~ 60^{\circ} - cos~ 60^{\circ} \\
      =\frac{\sqrt{3}}{2}-\frac{1}{2} \\
      =\frac{\sqrt{3}-1}{2}$
\end{enumerate}

\item %solution 2
\begin{enumerate}[noitemsep, label=\textbf{(\alph*)} ]
 \item   $LHS = \dfrac{sin~60^{\circ}}{cos~60^{\circ}}\\
	      = \dfrac{\frac{\sqrt{3}}{2}}{\frac{1}{2}}\\
	      = \frac{\sqrt{3}}{2}\times {\frac{2}{1}}\\
	      = \sqrt{3}}{2}\\
	  RHS = tan~60^{\circ}\\
	      = \sqrt{3}$\\
Therefore $LHS=RHS$.
 \item  $LHS = sin^{2}~45^{\circ}+ cos^{2}~45^{\circ}\\
	     = (\frac{1}{\sqrt{2}})^2+(\frac{1}{\sqrt{2}})^2\\
	     = \frac{1}{2}+\frac{1}{2}\\
	     = 1\\
	RHS = 1$\\
Therefore $LHS=RHS$.
 \item  $LHS = cos~30^{\circ}\\
	     = \frac{\sqrt{3}}{2}\\
	RHS = \sqrt{1- sin^{2}~30^{\circ}}\\
	    = \sqrt{1- (\frac{1}{2})^2}\\
	    = \sqrt{1- (\frac{1}{4})}\\
	    = \sqrt{\frac{3}{4}}\\
	    = \frac{\sqrt{3}}{2}$\\
Therefore $LHS=RHS$.
 \end{enumerate}

\item %solution 3
\begin{enumerate}[noitemsep, label=\textbf{(\alph*)} ]
 \item  The sine ratio is defined as $\frac{\mbox{opposite}}{\mbox{hypotenuse}}$.  From the Theorem of Pythagorus, we know that the hypotenuse is the longest side in any right-angled. So the maximum length of the opposite side is the length of the hypotenuse, which would make the ratio equal to $1$.  Therefore for any value of $\alpha$, $sin~\alpha\leq 1$.
 \item  Explain why $cos~\beta$ has a maximum value of $1$.
 \item  Is there a maximum value for $tan~\gamma$ ?
 \end{enumerate}
\end{enumerate}}
\end{solutions}

\begin{exercises}{}
{
\begin{enumerate}[itemsep=5pt, label=\textbf{\arabic*}. ]
\item In each triangle find the length of the side marked with a letter. Give answers correct to $2$ decimal places.
\begin{center}
\scalebox{0.85} % Change this value to rescale the drawing.
{
\begin{pspicture}(0,-4.11)(11.1198435,5)
\psline[linewidth=0.04](4.974369,2.612122)(2.6350498,3.9603012)(0.6377475,0.49464345)(4.974369,2.612122)
\psline[linewidth=0.04cm](2.8949742,3.8105037)(2.7451768,3.550579)
\psline[linewidth=0.04cm](2.7451768,3.550579)(2.4852521,3.7003772)
\psline[linewidth=0.04](4.557088,-3.2752385)(4.589271,-0.5754303)(0.5895552,-0.52775127)(4.557088,-3.2752385)
\psline[linewidth=0.04cm](4.5856953,-0.8754088)(4.2857165,-0.87183315)
\psline[linewidth=0.04cm](4.2857165,-0.87183315)(4.2892923,-0.5718545)
\psline[linewidth=0.04](7.0587263,3.3569472)(6.922255,0.6603981)(10.917143,0.45821914)(7.0587263,3.3569472)
\psline[linewidth=0.04cm](6.9374194,0.9600146)(7.2370353,0.94485116)
\psline[linewidth=0.04cm](7.2370353,0.94485116)(7.2218714,0.64523476)
\psline[linewidth=0.04](7.9354,-3.9493167)(10.598035,-3.9603012)(10.613694,-0.16448934)(7.9354,-3.9493167)
\psline[linewidth=0.04cm](10.340361,-3.959238)(10.341382,-3.7116852)
\psline[linewidth=0.04cm](10.341382,-3.7116852)(10.599056,-3.7127483)
\rput(0.12453125,3.7196066){\textbf{(a)}}
\rput(6.133125,3.7396066){\textbf{(b)}}
\rput(0.11421875,-0.26039344){\textbf{(c)}}
\rput(6.1343746,-0.28039345){\textbf{(d)}}
\usefont{T1}{ptm}{m}{n}
\rput(4.197969,2.6346066){$37^\circ$}
\usefont{T1}{ptm}{m}{n}
\rput(3.0853126,1.2){$62$}
\usefont{T1}{ptm}{m}{n}
\rput(1.3824998,2.5846066){$a$}
\usefont{T1}{ptm}{m}{n}
\rput(9.804531,0.8146064){$23^\circ$}
\usefont{T1}{ptm}{m}{n}
\rput(8.595937,0.21460645){$21$}
\usefont{T1}{ptm}{m}{n}
\rput(6.663907,2.0646067){$b$}
\usefont{T1}{ptm}{m}{n}
\rput(4.2134376,-2.6453934){$ 55^\circ$}
\usefont{T1}{ptm}{m}{n}
\rput(2.166719,-2.1653934){$19$}
\usefont{T1}{ptm}{m}{n}
\rput(4.8759375,-1.8353934){$c$}
\usefont{T1}{ptm}{m}{n}
\rput(8.953907,-1.9653934){$33$}
\usefont{T1}{ptm}{m}{n}
\rput(10.22,-1.1453935){ $49^\circ$}
\usefont{T1}{ptm}{m}{n}
\rput(10.90422,-2.0153935){$d$}
\end{pspicture}    
}
\end{center}
\begin{center}
 \scalebox{0.85} % Change this value to rescale the drawing.
{
\begin{pspicture}(0,-3.8416457)(11.207031,3.8216705)
\psline[linewidth=0.04](2.3197074,3.8016455)(1.0997375,1.3929795)(4.668132,-0.4143837)(2.3197074,3.8016455)
\psline[linewidth=0.04cm](1.2352896,1.6606089)(1.5029193,1.5250567)
\psline[linewidth=0.04cm](1.5029193,1.5250567)(1.3673671,1.2574271)
\rput(0.11234375,3.5211668){\textbf{(e)}}
\rput(6.084844,3.5211668){\textbf{(f)}}
\rput(0.14375,-0.39883313){\textbf{(g)}}
\rput(6.1403127,-0.498833){\textbf{(h)}}
\psline[linewidth=0.04](1.0870312,-0.6888331)(1.0870312,-3.3888335)(5.0870314,-3.3888335)(1.0870312,-0.6888331)
\psline[linewidth=0.04cm](1.0870312,-3.0888333)(1.3870313,-3.0888333)
\psline[linewidth=0.04cm](1.3870313,-3.0888333)(1.3870313,-3.3888335)
\psline[linewidth=0.04](10.705838,-3.6336784)(10.738021,-0.93386954)(6.738305,-0.88619083)(10.705838,-3.6336784)
\psline[linewidth=0.04cm](10.734446,-1.2338486)(10.434467,-1.2302725)
\psline[linewidth=0.04cm](10.434467,-1.2302725)(10.438043,-0.9302939)
\psline[linewidth=0.04](11.200404,0.87534267)(9.325028,2.8177538)(6.4473815,0.039416883)(11.200404,0.87534267)
\psline[linewidth=0.04cm](9.533403,2.6019304)(9.31758,2.393555)
\psline[linewidth=0.04cm](9.31758,2.393555)(9.109204,2.6093786)
\usefont{T1}{ptm}{m}{n}
\rput(3.5887504,1.9261669){$e$}
\usefont{T1}{ptm}{m}{n}
\rput(3.8853126,0.3161668){$17^\circ$}
\usefont{T1}{ptm}{m}{n}
\rput(1.3671876,2.6761668){ $12$}
\usefont{T1}{ptm}{m}{n}
\rput(7.533282,0.51616687){$22^\circ$}
\usefont{T1}{ptm}{m}{n}
\rput(9.139219,0.2861668){$f$}
\usefont{T1}{ptm}{m}{n}
\rput(7.758125,1.7361668){ $31$}
\usefont{T1}{ptm}{m}{n}
\rput(3.028594,-1.683833){$32$}
\usefont{T1}{ptm}{m}{n}
\rput(4.1132817,-3.123833){$23^\circ$}
\usefont{T1}{ptm}{m}{n}
\rput(2.949844,-3.6138332){$g$}
\usefont{T1}{ptm}{m}{n}
\rput(10.928438,-2.2538335){$h$}
\usefont{T1}{ptm}{m}{n}
\rput(7.7667193,-1.2438331){ $30^\circ$}
\usefont{T1}{ptm}{m}{n}
\rput(8.245001,-2.3038335){$20$}
\end{pspicture} 
}
\end{center}  
\item Write down two ratios for each of the following in terms of the
  sides: $AB$; $BC$; $BD$; $AD$; $DC$ and $AC$:
\begin{center}
\scalebox{1} % Change this value to rescale the drawing.
{
\begin{pspicture}(0,-1.2515883)(4.02125,1.2515885)
\psline[linewidth=0.04,fillstyle=solid](0.019002499,-1.2315885)(0.009700612,1.2315885)(4.00125,-1.1902716)(0.039492033,-1.2120903)(0.039492033,-1.2120903)(0.0,-1.1911051)
\psline[linewidth=0.04](0.009700612,-0.9884115)(0.20970061,-0.9884115)(0.20970061,-1.2284116)
\psline[linewidth=0.04,fillstyle=solid](1.3697007,0.4315885)(0.049700614,-1.1684115)(0.049700614,-1.1484115)
\psline[linewidth=0.04](1.2297006,0.2715885)(1.0697006,0.3515885)(1.1897006,0.5115885)
\usefont{T1}{ptm}{m}{n}
\rput(0,-1.5){$A$}
\rput(0, 1.5){$B$}
\rput(1.4, 0.7){$C$}
\rput(4, -1.5){$D$}
\end{pspicture} 
}
\end{center}
     \begin{enumerate}[noitemsep, label=\textbf{(\alph*)} ]
    \item $sin~ \hat{B}$
    \item $cos~ \hat{D}$
    \item $tan~ \hat{B}$
    \end{enumerate}
\vspace{10pt}
\item In $\triangle MNP$, $\hat{N}=90^{\circ}$, $MP=20$ and $\hat{P}=40^{\circ}$. Calculate $NP$ and $MN$ (correct to $2$ decimal places).
\end{enumerate}
\end{enumerate}

}
\end{exercises}


 \begin{solutions}{}{
\begin{enumerate}[itemsep=5pt, label=\textbf{\arabic*}. ] 


\item solution 1
\item solution 2
\item solution 3

\end{enumerate}}
\end{solutions}


% \section{Finding an angle}
\begin{exercises}{}
{
   \begin{enumerate}[itemsep=5pt, label=\textbf{\arabic*}. ] 
\item Determine the angle (correct to $1$ decimal place):
    \begin{enumerate}[itemsep=3pt, label=\textbf{(\alph*)} ]
\begin{multicols}{2}
 \item $tan~ \theta = 1,7$
\item $sin~ \theta = 0,8$
\item $cos~ \alpha = 0,32$
\item $tan~ \theta = 5\frac{3}{4}$
\item $sin~ \theta = \frac{2}{3}$
\item $cos~ \gamma = 1,2$
\item $4 cos~ \theta = 3$
\item $cos~ 4\theta = 0,3$
\item $sin~ \beta + 2= 2,65$
\item $sin~ \theta = 0,8$
\item $3 tan~ \beta = 1$
\item $sin~ 3\alpha = 1,2$
\item $tan~ \frac{\theta}{3} = sin~ 48^{\circ}$
\item $\frac{1}{2} cos~ 2\beta = 0,3$
\item $2 sin~ 3\theta +1= 2,6$
\end{multicols}
\end{enumerate}
\item Determine $\alpha$ in the following right-angled triangles:
\begin{center}
\scalebox{1} % Change this value to rescale the drawing.
{
\begin{pspicture}(0,-4.032389)(7.3778124,4.17849)
\rput(0, 3.5){\textbf{(a)}}
\rput(3.5, 3.5){\textbf{(b)}}
\rput(0, 0.5){\textbf{(c)}}
\rput(3.5, 0.5){\textbf{(d)}}
\rput(0, -2.0){\textbf{(e)}}
\rput(3.5, -2.0){\textbf{(f)}}
\psline[linewidth=0.04,fillstyle=solid](4.128174,1.9359901)(4.129576,3.691884)(6.1603127,1.9559901)(4.142015,1.9498184)(4.142015,1.9498184)(4.115519,1.9649143)
\psline[linewidth=0.04,fillstyle=solid](0.5403125,3.7206802)(2.3103564,3.7223134)(0.5605087,1.1162983)(0.5542749,3.7072453)(0.5542749,3.7072453)(0.5694489,3.7330344)
\psline[linewidth=0.04](0.8203125,3.71599)(0.8203125,3.41599)(0.5403125,3.41599)
\psline[linewidth=0.04](4.132775,2.2306225)(4.429114,2.2301245)(4.4286456,1.9555349)
\psline[linewidth=0.04,fillstyle=solid](0.58817375,-1.2040099)(0.5895763,0.5518841)(2.6203125,-1.1840099)(0.60201496,-1.1901817)(0.60201496,-1.1901817)(0.57551914,-1.1750857)
\psline[linewidth=0.04](0.59277505,-0.90937746)(0.889114,-0.9098756)(0.88864577,-1.184465)
\psline[linewidth=0.04,fillstyle=solid](0.57534015,-1.9803641)(2.331222,-1.9737015)(0.6046739,-4.0123897)(0.5892318,-1.9941416)(0.5892318,-1.9941416)(0.6042059,-1.9675767)
\psline[linewidth=0.04](0.8699906,-1.983612)(0.8708536,-2.2799501)(0.5962649,-2.2807431)
\psline[linewidth=0.04,fillstyle=solid](6.6596694,0.2178756)(6.6803126,-0.6840099)(3.9803126,0.21599011)(6.645639,0.20423959)(6.645639,0.20423959)(6.6719236,0.18877967)
\psline[linewidth=0.04](6.6710052,-0.076665334)(6.3803124,-0.06400988)(6.3789563,0.20247632)
\psline[linewidth=0.04,fillstyle=solid](5.604427,-1.7804683)(6.7567835,-3.1053228)(4.0853443,-3.1304228)(5.603074,-1.7999867)(5.603074,-1.7999867)(5.632968,-1.7939644)
\psline[linewidth=0.04](5.794496,-2.005642)(5.570731,-2.199926)(5.3907113,-1.9925796)
\rput{-34.695152}(-0.9369147,0.7926677){\psarc[linewidth=0.024](0.8003125,1.8959901){0.3}{39.289406}{180.0}}
\usefont{T1}{ptm}{m}{n}
\rput(1.4359375,3.9229398){$4$}
\usefont{T1}{ptm}{m}{n}
\rput(0.29078126,2.4629397){$9$}
\usefont{T1}{ptm}{m}{n}
\rput(5.470625,3.02294){$13$}
\usefont{T1}{ptm}{m}{n}
\rput(5.034219,1.5829399){$7,5$}
\usefont{T1}{ptm}{m}{n}
\rput(1.75375,-0.057060193){$2,2$}
\usefont{T1}{ptm}{m}{n}
\rput(0.2096875,-0.4770602){$1,7$}
\usefont{T1}{ptm}{m}{n}
\rput(5.716875,0.4829398){$9,1$}
\usefont{T1}{ptm}{m}{n}
\rput(7.0603123,-0.3170602){$4,5$}
\usefont{T1}{ptm}{m}{n}
\rput(0.26703125,-2.8770602){$12$}
\usefont{T1}{ptm}{m}{n}
\rput(1.9092188,-3.1370602){$15$}
\usefont{T1}{ptm}{m}{n}
\rput(4.5784373,-2.3170602){$1$}
\usefont{T1}{ptm}{m}{n}
\rput(5.54375,-3.47706){$\sqrt{2}$}
\usefont{T1}{ptm}{m}{n}
\rput(6.2634373,-2.93706){$\alpha$}
\usefont{T1}{ptm}{m}{n}
\rput(0.7834375,-3.4970603){$\alpha$}
\usefont{T1}{ptm}{m}{n}
\rput(1.9234375,-0.9970602){$\alpha$}
\usefont{T1}{ptm}{m}{n}
\rput(5.1634374,-0.017060194){$\alpha$}
\usefont{T1}{ptm}{m}{n}
\rput(4.3034377,3.1429398){$\alpha$}
\usefont{T1}{ptm}{m}{n}
\rput(0.7634375,1.8029398){$\alpha$}
\rput{-162.53017}(7.4652467,7.66313){\psarc[linewidth=0.024](4.3213253,3.258065){0.34310362}{39.289406}{156.30495}}
\rput{-89.91282}(5.095528,5.0594177){\psarc[linewidth=0.024](5.081325,-0.02193504){0.34310362}{53.61859}{134.10115}}
\rput{-319.68652}(-0.17349519,-1.5915118){\psarc[linewidth=0.024](2.0811038,-1.0320795){0.42839816}{67.16096}{156.30495}}
\rput{-30.447884}(1.9160283,-0.06363264){\psarc[linewidth=0.024](0.8411037,-3.5520794){0.42839816}{67.16096}{156.30495}}
\rput{-319.68652}(-0.41087502,-4.8648257){\psarc[linewidth=0.024](6.4211035,-2.9920795){0.42839816}{67.16096}{156.30495}}
\end{pspicture} 
}
\end{center}
\end{enumerate}

}
\end{exercises}


 \begin{solutions}{}{
\begin{enumerate}[itemsep=5pt, label=\textbf{\arabic*}. ] 


\item solution 1
\item solution 2

\end{enumerate}}
\end{solutions}


% \section{Two-dimensional problems}
\begin{exercises}{}
{
\begin{enumerate}[noitemsep, label=\textbf{\arabic*}. ] 
\item A boy flying a kite is standing $30~$m from a point directly under the kite. If the kite's string is $50~$m long, find the angle of elevation of the kite.
\item What is the angle of elevation of the sun when a tree $7,15$ m tall casts a shadow $10,1$ m long?
\item Susan is $1,4$ m tall. From a distance of $300$ m, she looks up at the top of a lighthouse. The angle of elevation is $5^{\circ}$. Determine the height of the lighthouse to the nearest metre.
\item A ladder of length $25$ m is resting against a wall, the ladder makes an angle $37^{\circ}$ to the wall. Find the distance between the wall and the base of the ladder. 
\end{enumerate}

}
\end{exercises} 


 \begin{solutions}{}{
\begin{enumerate}[itemsep=5pt, label=\textbf{\arabic*}. ] 


\item solution 1
\item solution 2
\item solution 3
\item solution 4

\end{enumerate}}
\end{solutions}


% \section{Defining ratios in the Cartesian plane}
\begin{exercises}{}
{
  \begin{enumerate}[itemsep=5pt, label=\textbf{\arabic*}. ]
   \item $B$ is a point in the Cartesian plane. Determine without using a calculator:
\begin{enumerate}[noitemsep, label=\textbf{(\alph*)} ]
 \item $OB$
\item $cos~ \beta$
\item $cosec~ \beta$
\item $tan~ \beta$
\end{enumerate}
\item If $sin~ \theta= 0,4$ and $\theta$ is an obtuse angle, determine:
\begin{enumerate}[noitemsep, label=\textbf{(\alph*)} ]
 \item $cos~ \theta$
\item $\sqrt{21} tan~ \theta$
\end{enumerate}
\end{enumerate}
\insertpracticeinfo{2}
}
\end{exercises}


 \begin{solutions}{}{
\begin{enumerate}[itemsep=5pt, label=\textbf{\arabic*}. ] 


\item solution 1
\item solution 2

\end{enumerate}}
\end{solutions}


\begin{eocexercises}{}
\begin{enumerate}[itemsep=6pt, label=\textbf{\arabic*}. ] 
\item Without using a calculator determine the value of 
\begin{equation*}
sin~ 60^{\circ} cos~ 30^{\circ}-cos~ 60^{\circ}sin~ 30^{\circ} + tan~ 45^{\circ}
\end{equation*}
\item If $3 tan~ \alpha = -5$ and $0^{\circ} < \alpha < 270^{\circ}$, use a sketch to determine:
    \begin{enumerate}[noitemsep, label=\textbf{(\alph*)} ]
    \item $cos~ \alpha$
    \item $tan~^{2}~\alpha - sec~^{2}~\alpha$
    \end{enumerate}
\item Solve for $\theta$ if $\theta$ is a positive, acute angle:
    \begin{enumerate}[noitemsep, label=\textbf{(\alph*)} ]
    \item $2 sin~ \theta = 1,34$
    \item $1 - tan~ \theta = -1$
    \item $cos~ 2\theta = sin~ 40^{\circ}$ 
    \item $\dfrac{sin~ \theta}{cos~ \theta}= 1$
    \end{enumerate}
\item Calculate the unknown lengths in the diagrams below:
\begin{center}
\scalebox{1}  
{ 
\begin{pspicture}(0,-2.0390613)(8.035,2.0253136) 
\psline[linewidth=0.04cm](0.02,0.02093862)(3.06,0.02093862) 
\psline[linewidth=0.04cm](0.02,0.02093862)(0.02,-2.0190613) 
\psline[linewidth=0.04cm](0.04,-1.9990613)(3.04,0.02093862) 
\psframe[linewidth=0.04,dimen=outer](0.26,0.04093862)(0.0,-0.21906137) 
\psline[linewidth=0.04cm](0.9944844,1.4531701)(3.04,0.04093862) 
\psline[linewidth=0.04cm](0.96,1.4209386)(0.024381146,0.02542873) 
\rput{-33.90198}(-0.5512336,0.79249614){\psframe[linewidth=0.04,dimen=outer](1.1544285,1.4305158)(0.8944285,1.1705158)} 
\psline[linewidth=0.04cm](2.2384837,1.992771)(3.04,0.04093862) 
\psline[linewidth=0.04cm](2.238,1.9929386)(1.0103352,1.4765087) 
\rput{-66.64198}(-0.3803531,3.117777){\psframe[linewidth=0.04,dimen=outer](2.3111107,1.9781737)(2.0511107,1.7181737)} 
\rput(2.2601562,-0.21406138){$30^{\circ}$} 
\rput(2.3034375,0.22593862){$25^{\circ}$} 
\rput(2.4634376,0.77){$20^{\circ}$} 
\rput(1.8576562,-1.2540613){$16$ cm} 
\rput(1.3398438,0.20593862){$a$} 
\rput(1.9,1.1){$b$} 
\rput(1.58,1.9059386){$c$} 
\psline[linewidth=0.04cm](6.689009,-1.6241995)(6.0199666,-1.4501117) 
\rput{-90.46057}(4.6635084,7.686732){\psframe[linewidth=0.04,dimen=outer](6.2543488,1.6402806)(6.0343485,1.4202806)} 
\rput{-90.46057}(4.4806204,7.508202){\psframe[linewidth=0.04,dimen=outer](6.0743546,1.6417276)(5.8543544,1.4217275)} 
\rput{-104.9971}(9.7950115,4.4890895){\psframe[linewidth=0.04,dimen=outer](6.7298956,-1.4036404)(6.509896,-1.6236403)} 
\psline[linewidth=0.04cm](7.375193,1.6303898)(4.2549725,1.6154709) 
\psline[linewidth=0.04cm](6.055236,1.6410005)(6.030317,-1.4588993) 
\psline[linewidth=0.04cm](6.0304775,-1.4389)(4.2748113,1.5953108) 
\psline[linewidth=0.04cm](7.375193,1.6303898)(6.030317,-1.4588993) 
\psline[linewidth=0.04cm](7.375193,1.6303898)(6.689009,-1.6241995) 
\rput(4.880625,0.005938619){$d$} 
\rput(6.7414064,1.8259386){$e$} 
\rput(5.2226562,1.8259386){$5$ m} 
\rput(6.2709374,-1.7940614){$f$} 
\rput(7.2148438,-0.07406138){$g$} 
\rput(4.756875,1.3859386){ $50^{\circ}$} 
\rput(6.9460936,1.4059386){$60^{\circ}$} 
\rput(6.45,-1.2740613){$80^{\circ}$} 
\end{pspicture} 
}
\end{center}
\item In $\triangle PQR$, $PR=20$ cm, $QR=22$ cm and $P\hat{R}Q = 30^{\circ}$. The perpendicular line from $P$ to $QR$ intersects $QR$ at $X$. Calculate 
\begin{enumerate}[noitemsep, label=\textbf{(\alph*)} ]
\item the length $XR$ 
\item the length $PX$
\item the angle $Q\hat{P}X$ 
\end{enumerate} 
\item A ladder of length $15$ m is resting against a wall, the base of the ladder is $5$ m from the wall. Find the angle between the wall and the ladder. 
\item In the following triangle find the angle $A\hat{B}C$:
\begin{center}
\begin{pspicture}(0,-2.4701562)(5.49875,2.4701562) 
\pspolygon[linewidth=0.04](0.1665625,-1.7301563)(3.3665626,1.9698437)(5.1665626,-1.7301563)(4.1665626,-1.7301563) 
\psline[linewidth=0.04cm](3.3665626,1.9698437)(3.3665626,-1.7301563) 
\rput(3.3871875,2.2798438){$A$} 
\rput(5.3459377,-2.0201561){$B$} 
\rput(3.371875,-2.0201561){$C$} 
\rput(0.07546875,-2.0201561){$D$} 
\rput(3.6,0){$9$} 
\rput(2.7525,-2.3201563){$17$} 
\psline[linewidth=0.04cm,arrowsize=0.05291667cm 2.0,arrowlength=1.4,arrowinset=0.4]{->}(3.0665624,-2.3301563)(5.2665625,-2.3301563) 
\psline[linewidth=0.04cm,arrowsize=0.05291667cm 2.0,arrowlength=1.4,arrowinset=0.4]{->}(2.4665625,-2.3301563)(0.0665625,-2.3301563) 
\psline[linewidth=0.04cm](3.3665626,-1.5301563)(3.5665624,-1.5301563) 
\psline[linewidth=0.04cm](3.5665624,-1.5301563)(3.5665624,-1.7301563) 
\rput(0.8,-1.48){$41^{\circ}$} 
\end{pspicture} 
\end{center}
\item In the following triangle find the length of side $CD$:
\begin{center}
\begin{pspicture}(0,-2.2234375)(6.091875,2.2234375) 
\pspolygon[linewidth=0.04](0.1665625,-1.776875)(5.1665626,-1.776875)(5.1665626,1.823125) 
\psline[linewidth=0.04cm](3.4665625,-1.776875)(5.1665626,1.823125) 
\rput(5.2871876,2.033125){$A$} 
\rput(5.3459377,-2.066875){$B$} 
\rput(3.471875,-2.066875){$C$} 
\rput(0.07546875,-2.066875){$D$} 
\rput(5.4,0){$9$} 
\rput(4.490156,0.95){$15^{\circ}$} 
\rput(3.960156,-1.5){$35^{\circ}$} 
\psline[linewidth=0.04cm](4.9665626,-1.576875)(5.1665626,-1.576875) 
\psline[linewidth=0.04cm](4.9665626,-1.576875)(4.9665626,-1.776875) 
\end{pspicture}
\end{center} 
\item $A(5;0)$ and $B(11;4)$. Find the angle between the line through $A$ and $B$ and the $x$-axis. 
\item $C(0;-13)$ and $D(-12;14)$. Find the angle between the line through $C$ and $D$ and the $y$-axis. 
\item A right-angled triangle has hypotenuse $13$ mm. Find the length of the other two sides if one of the angles of the triangle is $50^{\circ}$.
\item One of the angles of a rhombus with perimeter $20$ cm is $30^{\circ}$. 
\begin{enumerate}[noitemsep, label=\textbf{(\alph*)} ]
\item Find the sides of the rhombus. 
\item Find the length of both diagonals. 
\end{enumerate} 
\item Captain Jack is sailing towards a cliff with a height of $10$ m. 
\begin{enumerate}[noitemsep, label=\textbf{(\alph*)} ] 
\item The distance from the boat to the top of the cliff is $30$ m. Calculate the angle of elevation from the boat to the top of the cliff (correct to the nearest integer).
\item If the boat sails $7$ m closer to the cliff, what is the new angle of elevation from the boat to the top of the cliff? 
\end{enumerate} 
\item Given the points: $E(5;0)$, $F(6;2)$ and $G(8;-2)$. Find the angle $F\hat{E}G$. 
\item  A triangle with angles $40^{\circ}$, $40^{\circ}$ and $100^{\circ}$ has a perimeter of $20$ cm. Find the length of each side of the triangle. 
\end{enumerate}

\end{eocexercises}


 \begin{solutions}{}{
\begin{enumerate}[itemsep=5pt, label=\textbf{\arabic*}. ] 


\item solution 1
\item solution 2
\item solution 3
\item solution 4
\item solution 5
\item solution 6
\item solution 7
\item solution 8
\item solution 9
\item solution 10
\item solution 11
\item solution 12
\item solution 13
\item solution 14
\item solution 15

\end{enumerate}}
\end{solutions}


