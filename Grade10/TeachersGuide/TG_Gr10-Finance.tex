\chapter{Finance and growth}
% \section{Being interested in interest}
% \section{Simple interest}
% \subsection{When the time period is not in years}
\begin{exercises}{}{
    \begin{enumerate}[itemsep=6pt, label=\textbf{\arabic*}.]
	\item An amount of R~$3~500$ is invested in a savings account which pays simple interest at a rate of $7,5\%$ per annum. Calculate the balance accumulated by the end of $2$ years.
	\item Calculate the accumulated amount in the following situations:
	\begin{enumerate}[noitemsep, label=\textbf{(\alph*)} ]
	    \item A loan of R~$300$ at a rate of $8\%$ for $1$ year.
	    \item An investment of R~$2~250$ at a rate of $12,5\%$p.a.\@ for $6$ years.
	\end{enumerate}
	\item Sally wanted to calculate the number of years she needed to invest R~$1~000$ for in order to accumulate R~$2~500$. She has been offered a simple interest rate of $8,2\%$ p.a. How many years will it take for the money to grow to R~$2~500$?
	\item Joseph made a deposit of R~$5~000$ in the bank for his $5$ year old son's $21^{\mbox{st}}$ birthday. He has given his son the amount of R~$18~000$ on his birthday. At what rate was the money invested, if simple interest was calculated?
    \end{enumerate}
}
\end{exercises}


 \begin{solutions}{}{
\begin{enumerate}[itemsep=5pt, label=\textbf{\arabic*}. ] 


\item  $P=3~500\\
i=0,075\\
n=2\\
A=?\\
 A=P(1 + i . n) \\
A = 3~500(1+ (0,075)(2))\\
A = 3~500(1.15)\\
A = \text{R }4~025$
\item \begin{enumerate}[noitemsep, label=\textbf{(\alph*)} ]
\item $P=300\\
i=0,08\\
n=1\\
A=?\\
 A=P(1 + i . n) \\
A = 300(1+ (0,08)(1))\\
A = 300(1,08)\\
A = \text{R }324$
\item $ P=2~250\\
i=0,125\\
n=6\\
A=?\\
A=P(1 + i . n) \\
A = 2~250(1+ (0,125)(6))\\
A = 2~250(1.75)\\
A =\text{R }3~937,5$
\end{enumerate}
\item $ A=2~500\\
P=1~000\\
i=0,082\\
n=?\\
A=P(1 + i . n) \\
2~500 = 1~000(1+ (0,082)(n))\\
\dfrac{2~500}{1~000} = 1 + 0,082n\\
\dfrac{2~500}{1~000} - 1 = 0,082n \\
(\dfrac{2~500}{1~000} - 1) \divide 0,082 = n\\
n = 18,3 $\\
It would take 19 years for $\text{R }1~000$ to become $\text{R }2~500$ at $8,2\%$ p.a.
\item $ A=18~000\\
P=5~000\\
i=?\\
n=21-5=16\\
A=P(1 + i . n) \\
18~000 = 5~000(1+ (i)(16))\\
\dfrac{18~000}{5~000} = 1 + 16i\\
\dfrac{18~000}{5~000} - 1 = 16i \\
(\dfrac{18~000}{5~000} - 1) \divide 16 = i\\
i = 0,0125 $\\
The interest rate at which the money was invested was $1,25\%$.
\end{enumerate}}
\end{solutions}


% \subsection{Hire purchase}
\begin{exercises}{}{
    \begin{enumerate}[label=\textbf{\arabic*}.]
	\item Vanessa wants to buy a fridge on a hire purchase agreement. The cash price of the fridge is R~$4~500$. She is required to pay a deposit of $15\%$ and pay the remaining loan amount off over $24$ months at an interest rate of $12\%$ p.a.
	\begin{enumerate}[noitemsep, label=\textbf{(\alph*)} ]
	    \item What is the principal loan amount?
	    \item What is the accumulated loan amount?
	    \item What are Vanessa's monthly repayments?
	    \item What is the total amount she has paid for the fridge?
	\end{enumerate}
	\item Bongani buys a dining room table costing R~$8~500$ on a
          hire purchase agreement. He is charged an interest rate of
          $17,5\%$ p.a.\@ over $3$ years.
	\begin{enumerate}[noitemsep, label=\textbf{(\alph*)} ]
	    \item How much will Bongani pay in total?
	    \item How much interest does he pay?
	    \item What is his monthly instalment?
	\end{enumerate}
	\item A lounge suite is advertised on TV, to be paid off over $36$ months at R~$150$ per month.
	\begin{enumerate}[noitemsep, label=\textbf{(\alph*)} ]
	    \item Assuming that no deposit is needed, how much will the buyer pay for the lounge suite once it has been paid off?
	    \item If the interest rate is $9\%$ p.a., what is the cash price of the suite?\\
	\end{enumerate}
    \end{enumerate}
}
\end{exercises}


 \begin{solutions}{}{
\begin{enumerate}[itemsep=5pt, label=\textbf{\arabic*}. ] 


\item \begin{enumerate}[noitemsep, label=\textbf{(\alph*)} ]
  \item  $P=4~500 - (4~500 \times 0,15) \\
= 4~500 - 675 \\
= 3~825$
  \item $P = 3~825\\
i=0,12\\
n=\frac{24}{12}=2\\
A=P(1 + i.n)\\
A = 3~825(1 + (0,12)(2))\\
A=4~743$
  \item $\dfrac{4~743}{24} = 197,63$
  \item $4~743+675=5~418$
  \end{enumerate}
\item \begin{enumerate}[noitemsep, label=\textbf{(\alph*)} ]
  \item $A=?
P = 8~500\\
i=0,175\\
n=3\\
A=P(1 + i.n)\\
A = 8~500(1 + (0,175)(3))\\
A=12~962,50$
  \item $12~962,50-8~500=4~462,50$
  \item $\dfrac{12~962,50}{36}=360,07$
  \end{enumerate}
\item \begin{enumerate}[noitemsep, label=\textbf{(\alph*)} ]
  \item $36 \times 150 = 5~400$
  \item $A=5~400\\
P = ?\\
i=0,09\\
n=3\\
A=P(1 + i.n)\\
5~400 = P(1 + (0,09)(3))\\
\dfrac{5~400}{1,27} = P\\
P=4~251,97$
  \end{enumerate}

\end{enumerate}}
\end{solutions}


% \section{Compound interest}
% \subsection{The power of compound interest}
\begin{exercises}{}{
    \begin{enumerate}[label=\textbf{\arabic*}.]
	\item An amount of R~$3~500$ is invested in a savings account which pays a compound interest rate of $7,5\%$ p.a. Calculate the balance accumulated by the end of $2$ years.
	\item Morgan invests R~$5~000$ into an account which pays out a lump sum at the end of $5$ years. If he gets R~$7~500$ at the end of the period, what compound interest rate did the bank offer him?
	\item Nicola wants to invest some money at a compound interest rate of $11\%$ p.a. How much money (to the nearest Rand) should be invested if she wants to reach a sum of R~$100~000$ in five years time?\\
    \end{enumerate}
}
\end{exercises}


 \begin{solutions}{}{
\begin{enumerate}[itemsep=5pt, label=\textbf{\arabic*}. ] 


\item $P = 3~500\\
i=0,075\\
n=2\\
A=P(1 + i)^{n}\\
A = 3~500(1 + 0,075)^{2}\\
A = 4~044,69$
\item $A = 7~500
P = 5~000\\
i=?\\
n=5\\
A=P(1 + i)^{n}\\
7~500 = 5~000(1 + i)^{5}\\
\dfrac{7~500}{5~500} = (1+i)^5\\
\sqrt[5]{\dfrac{7~500}{5~500}} = (1+i)\\
\sqrt[5]{\dfrac{7~500}{5~500}} - 1 = i\\
i = 0,0844717712$\\
interest rate is $8,45 \%$
\item $A = 100~000\\
P=?\\
i=0,11\\
n=5\\
A=P(1 + i)^{n}\\
100~000 = P(1 + 0,11)^{5}\\
\dfrac{100~000}{(1,11)^5} = P\\
P = 59~345,13$

\end{enumerate}}
\end{solutions}


% \subsection{Inflation}
% \subsection{Population growth}
\begin{exercises}{}{
    \begin{enumerate}[label=\textbf{\arabic*}.]
	\item If the average rate of inflation for the past few years was $7,3\%$ p.a.\@ and your water and electricity account is R~$1~425$ on average, what would you expect to pay in $6$ years time?
	\item The price of popcorn and a coke at the movies is now R~$60$. If the average rate of inflation is $9,2\%$ p.a. What was the price of popcorn and coke $5$ years ago?
	\item A small town in Ohio, USA is experiencing a huge increase in births. If the average growth rate of the population is $16\%$ p.a., how many babies will be born to the $1~600$ residents in the next $2$ years?\\
    \end{enumerate}
}
\end{exercises}


 \begin{solutions}{}{
\begin{enumerate}[itemsep=5pt, label=\textbf{\arabic*}. ] 


\item $A = ?\\
P=1~425\\
i=0,073\\
n=6\\
A=P(1 + i)^{n}\\
A = 1~425(1 + 0,073)^{6}\\
A = 2~174,77$
\item $A = 60\\
P=?\\
i=0,092\\
n=5\\
A=P(1 + i)^{n}\\
60 = P(1 + 0,092)^{5}\\
\dfrac{60}{(1,092)^5}=P\\
P = 2~152,96$
\item $A = ?\\
P=1~600\\
i=0,16\\
n=2\\
A=P(1 + i)^{n}\\
A = 1~600(1 + 0,16)^{2}\\
A = 2~152,96\\
2~153-1~600=553$
There will be roughly $553$ babies born in the next two years.
\end{enumerate}}
\end{solutions}


% \subsection{Foreign exchange rates}
\begin{exercises}{}
{
    \begin{enumerate}[itemsep=6pt, label=\textbf{\arabic*}.]
	\item Bridget wants to buy an iPod that costs £~$100$, with the exchange rate currently at £~$1$ $=$ R~$14$. She estimates that the exchange rate will drop to R~$12$ in a month.
	\begin{enumerate}[noitemsep, label=\textbf{(\alph*)} ]
	    \item How much will the iPod cost in Rands, if she buys it now?
	    \item How much will she save if the exchange rate drops to R~$12$?
	    \item How much will she lose if the exchange rate moves to R~$15$?
	\end{enumerate}
	\item Study the following exchange rate table:\\
	\begin{center}
	    \begin{tabular}{ |l|c|c| }
		\hline
		\textbf{Country}	&	\textbf{Currency}	&	\textbf{Exchange Rate}\\ \hline
		United Kingdom (UK)	&	Pounds (£)	&	R~$14,13$\\ \hline
		United States (USA)	&	Dollars (\$)	&	R~$7,04$\\ \hline
	    \end{tabular}
	\end{center}
	\vspace{8pt}\\
	\begin{enumerate}[noitemsep, label=\textbf{(\alph*)} ]
	    \item In South Africa the cost of a new Honda Civic is R~$173~400$. In England the same vehicle costs £~$12~200$ and in the USA \$~$21~900$. In which country is the car the cheapest?
	    \item Sollie and Arinda are waiters in a South African restaurant attracting many tourists from abroad. Sollie gets a £~$6$ tip from a tourist and Arinda gets \$~$12$. Who got the better tip?
	\end{enumerate}
    \end{enumerate}
}
\end{exercises}


 \begin{solutions}{}{
\begin{enumerate}[itemsep=5pt, label=\textbf{\arabic*}. ] 


\item \begin{enumerate}[noitemsep, label=\textbf{(\alph*)} ]
\item Cost in rands = (cost in pounds) times exchange rate\\
$\text{Cost in rands}=100 \times \frac{14}{1} = \text{R }1~400$
\item $\text{Cost in rands}=100 \times \frac{12}{1} = \text{R }1~200$\\
So she will save $\text{R }200$ ($\text{Saving} = \text{R }1~400 - \text{R }1~200$)
\item $\text{Cost in rands}=100 \times \frac{15}{1} = \text{R }1~500$ \\
So she will lose $\text{R }100$ ($\text{Loss} = \text{R }1~400 - \text{R }1~500$)
\end{enumerate}
\item \begin{enumerate}[noitemsep, label=\textbf{(\alph*)} ]
\item To answer this question we work out the cost of the car in rand for each country and then compare the three answers to see which is the cheapest. Cost in rands = cost in currency times exchange rate.\\
Cost in UK: $12~200 \times \frac{14,13}{1} = \text{R }172~386$\\
Cost in USA: $21~900 \times \frac{7,04}{1} = \text{R }154~400$\\
Comparing the three costs we find that the car is the cheapest in the USA.
\item Sollie: $6 \times \frac{14,31}{1} = \text{R }84,78$\\
Arinda: $12 \times \frac{7,04}{1} = \text{R }84,48$.
\end{enumerate}

\end{enumerate}}
\end{solutions}


\begin{eocexercises}{}
    \begin{enumerate}[label=\textbf{\arabic*}.]
	\item Alison is going on holiday to Europe. Her hotel will cost €~$200$ per night. How much will she need in Rands to cover her hotel bill, if the exchange rate is €~$1$ = R~$9,20$?
	\item Calculate how much you will earn if you invested R~$500$ for 1 year at the following interest rates:
	\begin{enumerate}[noitemsep, label=\textbf{(\alph*)} ]
	    \item $6,85\%$ simple interest
	    \item $4,00\%$ compound interest
	\end{enumerate}
	\item Bianca has R~$1~450$ to invest for 3 years. Bank A offers a savings account which pays simple interest at a rate of $11\%$ per annum, whereas Bank B offers a savings account paying compound interest at a rate of $10,5\%$ per annum. Which account would leave Bianca with the highest accumulated balance at the end of the 3 year period?
	\item How much simple interest is payable on a loan of R~$2~000$ for a year, if the interest rate is $10\%$ p.a.?
	\item How much compound interest is payable on a loan of R~$2~000$ for a year, if the interest rate is $10\%$ p.a.?
	\item Discuss:
	\begin{enumerate}[noitemsep, label=\textbf{(\alph*)} ]
	    \item Which type of interest would you like to use if you are the borrower?
	    \item Which type of interest would you like to use if you were the banker?
	\end{enumerate}
	\item Calculate the compound interest for the following problems.
	\begin{enumerate}[noitemsep, label=\textbf{(\alph*)} ]
	    \item A R~$2~000$ loan for 2 years at $5\%$ p.a.
	    \item A R~$1~500$ investment for 3 years at $6\%$ p.a.
	    \item A R~$800$ loan for 1 year at $16\%$ p.a.
	\end{enumerate}
	\item If the exchange rate to the Rand for Japanese Yen is
          ¥~$100$ = R~$6,2287$ and for Australian Dollar is $1$~AUD =
          R~$5,1094$, determine the exchange rate between the
          Australian Dollar and the Japanese Yen.
	\item Bonnie bought a stove for R~$3~750$. After $3$ years she had finished paying for it and the R~$956,25$ interest that was charged for hire purchase. Determine the rate of simple interest that was charged.
    \end{enumerate}
\end{eocexercises}


 \begin{solutions}{}{
\begin{enumerate}[itemsep=5pt, label=\textbf{\arabic*}. ] 


\item $\text{cost in rands}=\text{cost in euros} \times \text{exchange rate}\\
=200 \times 9,201\\
 = \text{R }1~840$
\item \begin{enumerate}[noitemsep, label=\textbf{(\alph*)} ]
	    \item  $P=500\\
i=0,685\\
n=1\\
A=?\\
 A=P(1 + i . n) \\
A = 500(1+ (0,685)(1))\\
A = 500(1,685)\\
A = \text{R }534,25$
	    \item  $P = 500\\
i=0,04\\
n=1\\
A=?\\
A=P(1 + i)^{n}\\
A = 500(1 + 0,04)^{1}\\
A = \text{R }520$
	\end{enumerate}
\item Bank A:\\
$P=1~450\\
i=0,11\\
n=3\\
A=?\\
 A=P(1 + i . n) \\
A = 1~450(1+ (0,11)(3))\\
A = 1~450(1,33)\\
A = \text{R }1~925,50$\\
Bank B:\\
 $P = 1~450\\
i=0,150\\
n=3\\
A=?\\
A=P(1 + i)^{n}\\
A = 1~450(1 + 0,150)^{3}\\
A = \text{R }1956,39$\\
She should choose Bank B as it will give her more money after 3 years.
\item $P=2~000\\
i=0,10\\
n=1\\
A=?\\
 A=P(1 + i . n) \\
A = 2~000(1+ (0,10)(1))\\
A = 2~000(1,10)\\
A = \text{R }2~200$\\
So the amount of interest is:\\
$2~200 - 2~000 = 200$
\item  $P = 2~000\\
i=0,10\\
n=1\\
A=?\\
A=P(1 + i)^{n}\\
A = 2~000(1 + 0,10)^{1}\\
A = \text{R }2~200$\\
So the amount of interest is:\\
$2~200 - 2~000 = 200$
\item 	\begin{enumerate}[noitemsep, label=\textbf{(\alph*)} ]
	    \item Simple interest. Interest is only calculated on the principal amount and not on the interest earned during prior periods. This will lead to the borrower paying less interest.
	    \item Compound interest. Interest is calculated from the principal amount as well as interest earned from prior periods. This will lead to the banker getting more money for the bank.
	\end{enumerate}
\item 	\begin{enumerate}[noitemsep, label=\textbf{(\alph*)} ]
	    \item  $P = 2~000\\
i=0,05\\
n=2\\
A=?\\
A=P(1 + i)^{n}\\
A = 2~000(1 + 0,05)^{2}\\
A = \text{R }2~205$\\
So the amount of interest is:\\
$2~205 - 2~000 = 205$
	    \item $P = 1~500\\
i=0,06\\
n=3\\
A=?\\
A=P(1 + i)^{n}\\
A = 1~500(1 + 0,06)^{3}\\
A = \text{R }1~786,524$\\
So the amount of interest is:\\
$1~786,524 - 1~500 = 286,524$
	    \item $P = 800\\
i=0,16\\
n=1\\
A=?\\
A=P(1 + i)^{n}\\
A = 800(1 + 0,16)^{1}\\
A = \text{R }928$\\
So the amount of interest is:\\
$928 - 800 = 128$
	\end{enumerate}
\item $\frac{\text{AUD}}{\text{Yen}}=\frac{\text{ZAR}}{\text{Yen}} \times \frac{\text{AUD}}{\text{Yen}} \\
= \frac{6,2287}{100} \times 15,1094 \\
= \frac{0,01219}{0,00219} \text{ AUD} \\
= 1 \text{ Yen} \\
\text{or } 1 \text{ AUD} = 82,03 \text{ Yen}$
\item $\text{Total paid } = 3750 + 956,25 = 4706,25$\\
$P=3~750\\
i=?\\
n=3\\
A=4~706,25\\
 A=P(1 + i . n) \\
4~706,25 = 3~750(1+ i(3))\\
1,255 = (1 + 3i)\\
0,255 = 3i \\
i = 0,085$\\
So the interest rate is $8,5 \%$
\end{enumerate}}
\end{solutions}


