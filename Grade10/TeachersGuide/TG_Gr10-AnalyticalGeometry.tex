\chapter{Analytical geometry}
\begin{exercises}{}{
\begin{enumerate}[label=\textbf{\arabic*}.]
\item Find the length of $AB$ if:
 \begin{enumerate}[noitemsep, label=\textbf{(\alph*)} ] 
\item $A(2;7)$ and $B(-3;5)$.
\item $A(-3;5)$ and $B(-9;1)$.
\item $A(x;y)$ and $B(x+4;y-1)$.
\end{enumerate}
\item The length of $CD=5$. Find the missing coordinate if:
 \begin{enumerate}[noitemsep, label=\textbf{(\alph*)} ] 
\item $C(6;-2)$ and $D(x;2)$.
\item $C(4;y)$ and $D(1;-1)$.
\end{enumerate}
\end{enumerate}
}
\end{exercises}


\begin{solutions}{}{
\begin{enumerate}[itemsep=5pt, label=\textbf{\arabic*}. ] 
 \item
% \begin{multicols}{2}
  \begin{enumerate}[noitemsep, label=\textbf{(\alph*)} ] 
\item
\begin{array*}
  $d_{AB} &=& \sqrt{(x_1-x_2)^2+(y_1-y_2)^2}\\
   &=& \sqrt{(2-(-3))^2+(7-5)^2}\\
   &=& \sqrt{(5)^2+(2)^2}\\
   &=& \sqrt{29}\\$
\end{array*}
\item
\begin{array*}
  $d_{AB} &=& \sqrt{(x_1-x_2)^2+(y_1-y_2)^2}\\
   &=& \sqrt{(-3-(-9))^2+(5-1)^2}\\
   &=& \sqrt{(6)^2+(4)^2}\\
   &=& \sqrt{52}\\$
\end{array*}
\item
\begin{array*}
  $d_{AB} &=& \sqrt{(x_1-x_2)^2+(y_1-y_2)^2}\\
   &=& \sqrt{(x-(x+4))^2+(y-(y-1))^2}\\
   &=& \sqrt{(x-x-4)^2+(y-y+1)^2}\\
   &=& \sqrt{(-4)^2+(1)^2}\\
   &=& \sqrt{17}\\$
\end{array*}
\end{enumerate}
% \end{multicols}
\item

\begin{enumerate}[noitemsep, label=\textbf{(\alph*)} ] 
\item
\begin{array*}
  $d_{CD} &=& \sqrt{(x_1-x_2)^2+(y_1-y_2)^2}\\
  5 &=& \sqrt{(6-x)^2+(-2-2))^2}\\
  5^2 &=& 36-12x+x^2+16\\
  0 &=& x^2-12x+36-25+16\\
  0 &=& x^2-12x+27\\
   &=& (x-3)(x-9)\\$
\end{array*}
Therefore $x=3$ or $x=9$\\

Check solution for $x=3$:\\
\begin{array*}
  $d_{CD} &=& \sqrt{(x_1-x_2)^2+(y_1-y_2)^2}\\
  &=& \sqrt{(6-3)^2+(-2-2))^2}\\
  &=& \sqrt{(3)^2+(-4)^2}\\
  &=& \sqrt{25}\\
  &=& 5\\$
\end{array*}
Solution is valid.\\
Check solution for $x=9$:\\
\begin{array*}
  $d_{CD} &=& \sqrt{(x_1-x_2)^2+(y_1-y_2)^2}\\
  &=& \sqrt{(6-9)^2+(-2-2))^2}\\
  &=& \sqrt{(-3)^2+(-4)^2}\\
  &=& \sqrt{25}\\
  &=& 5\\$
\end{array*}
Solution is valid.
\item
\begin{array*}
  $d_{CD} &=& \sqrt{(x_1-x_2)^2+(y_1-y_2)^2}\\
  5 &=& \sqrt{(4-1)^2+(y+1))^2}\\
  5^2 &=& 9+y^2+2y+1\\
  0 &=& y^2+2y+1+9-25\\
  0 &=& y^2+2y-15\\
   &=& (y-3)(y+5)\\$
\end{array*}
Therefore $x=3$ or $x=-5$\\

Check solution for $x=3$:\\
\begin{array*}
  $d_{CD} &=& \sqrt{(x_1-x_2)^2+(y_1-y_2)^2}\\
  &=& \sqrt{(4-1)^2+(3+1))^2}\\
  &=& \sqrt{3^2+4^2}\\
  &=& \sqrt{25}\\
  &=& 5\\$
\end{array*}
Solution is valid.\\
Check solution for $x=-5$:\\
\begin{array*}
  $d_{CD} &=& \sqrt{(x_1-x_2)^2+(y_1-y_2)^2}\\
  &=& \sqrt{(4-1)^2+(-5+1)^2}\\
  &=& \sqrt{(3)^2+(-4)^2}\\
  &=& \sqrt{25}\\
  &=& 5\\$
\end{array*}
Solution is valid.
\end{enumerate}
\end{enumerate}}
\end{solutions}

\begin{exercises}{}{
\begin{enumerate}[noitemsep, label=\textbf{\arabic*}. ]
\begin{multicols}{2}

\item Find the gradient of $AB$ if
 \begin{enumerate}[noitemsep, label=\textbf{(\alph*)} ] 
\item $A(7;10)$ and $B(-4;1)$
\item $A(-5;-9)$ and $B(3;2)$
\item $A(x-3;y)$ and $B(x;y+4)$
\end{enumerate}
\item If the gradient of $CD=\frac{2}{3}$, find $P$ given
\begin{enumerate}[noitemsep, label=\textbf{(\alph*)} ] 
\item $C(16;2)$ and $D(8;P)$
\item $C(3;2P)$ and $D(9;14)$
\end{enumerate}
\end{multicols}
\end{enumerate}
}
\end{exercises}


 \begin{solutions}{}{
\begin{enumerate}[itemsep=5pt, label=\textbf{\arabic*}. ] 


\item\begin{multicols}{2}
\begin{enumerate}[noitemsep, label=\textbf{(\alph*)} ] 
\item
\begin{array*}
  $m_{AB} &=& \frac{y_2-y_1}{x_2-x_1}\\
  &=& \frac{-1-10}{-4-7}\\
  &=& \frac{-11}{-11}\\
  &=& 1\\$
\end{array*}
\item
$m_{AB} &=& \frac{y_2-y_1}{x_2-x_1}\\
  =& \frac{2-(-9)}{3-(-5)}\\
  =& \frac{11}{8}\\$
\item 
$m_{AB} &=& \frac{y_2-y_1}{x_2-x_1}\\
  =& \frac{y+4-y)}{x-(x-3)}\\
  =& \frac{4}{3}\\$
\end{enumerate}
\end{multicols}
\item\begin{multicols}{2}
\begin{enumerate}[noitemsep, label=\textbf{(\alph*)} ] 
\item
$m_{CD} = \frac{y_2-y_1}{x_2-x_1}\\
  \frac{2}{3} = \frac{p-2}{8-16}\\
  \frac{2}{3}\times(-8)= p-2\\
  \frac{-16+6}{3} = p\\
  \frac{-10}{3} = p\\$

\item
$m_{CD} = \frac{y_2-y_1}{x_2-x_1}\\
  \frac{2}{3} = \frac{14-2p}{9-3}\\
  \frac{2}{3}\times(6)= 14-2p\\
    4-14 = -2p\\
  \frac{-10}{-2} = p\\
    5 = p\\$
\end{enumerate}
\end{multicols}
\end{enumerate}}
\end{solutions}


\begin{exercises}{}{
  \begin{enumerate}[itemsep=5pt, label=\textbf{\arabic*}. ]
\item Determine whether $AB$ and $CD$ are parallel, perpendicular or neither if:
\begin{enumerate}[noitemsep, label=\textbf{(\alph*)} ]
\item $A(3;-4)$, $B(5;2)$, $C(-1;-1)$, $D(7;23)$.
\item $A(3;-4)$, $B(5;2)$, $C(-1;-1)$, $D(0;-4)$.
\item $A(3;-4)$, $B(5;2)$, $C(-1;-1)$, $D(1;2)$.
\end{enumerate}
\item Determine whether the following points are collinear:
\begin{enumerate}[noitemsep, label=\textbf{(\alph*)} ]
\item $E(0;3)$, $F(-2;5)$, $G(2;1)$.
\item $H(-3;-5)$, $I(-0;0)$, $J(6;10)$.
\item $K(-6;2)$, $L(-3;1)$, $M(1;-1)$.
\end{enumerate}
\item Points $P(-6;2)$, $Q(2;-2)$ and $R(-3;r)$ lie on a straight line. Find the value of $r$.
\item Line $PQ$ with $P(-1;-7)$ and $Q(q;0)$ has a gradient of $1$. Find $q$.
\end{enumerate}
\end{enumerate}
}
\end{exercises}


 \begin{solutions}{}{
\begin{enumerate}[itemsep=5pt, label=\textbf{\arabic*}. ] 
\item 
\begin{enumerate}[noitemsep, label=\textbf{(\alph*)} ] 
\item $m_{AB} &=& \frac{y_2-y_1}{x_2-x_1}\\
  = \frac{2-(-4)}{5-3}\\
  = \frac{6}{2}\\
  = 3\\$
And,\\
$m_{CD} &=& \frac{y_2-y_1}{x_2-x_1}\\
  = \frac{23-(-1)}{7-(3)}\\
  = \frac{24}{8}\\
  = 3\\$
So $m_{AB} = m_{CD}$\\
Therefore $AB\parallel CD$.
\item $m_{AB} &=& 3\\$
And,\\
$m_{CD} &=& \frac{y_2-y_1}{x_2-x_1}\\
  = \frac{-4-(-1)}{0-(-1)}\\
  = \frac{-3}{1}\\
  = -3\\$
So $m_{AB}\ne m_{CD}$\\
Therefore $AB$ is not parallel to $CD$.\\
And $m_AB\times\frac{1}{m_CD} \ne -1$\\
Therefore $AB$ and $CD$ are not perpendicular.
\item $m_{AB} &=& 3\\$
And,\\
$m_{CD} &=& \frac{y_2-y_1}{x_2-x_1}\\
  = \frac{2-(-1)}{1-(-1)}\\
  = \frac{3}{2}\\$
So $m_{AB}\ne m_{CD}$\\
Therefore $AB$ is not parallel to $CD$.\\
And $m_AB\times\frac{1}{m_CD} \ne -1$\\
Therefore $AB$ and $CD$ are not perpendicular.
\end{enumerate}
\item
\begin{enumerate}[noitemsep, label=\textbf{(\alph*)} ] 
\item $m_{EF} &=& \frac{y_2-y_1}{x_2-x_1}\\
  = \frac{5-3}{-2-0}\\
  = \frac{2}{-2}\\
  = -1\\$
And,\\
$m_{FG} &=& \frac{y_2-y_1}{x_2-x_1}\\
  = \frac{1-5}{2-(-2)}\\
  = \frac{-4}{4}\\
  = -1\\$
So $m_{HI} = m_{FG}$ and $F$ is a common point,\\ 
Therefore $E, F$ and $G$ are collinear.
\item $m_{EF} &=& \frac{y_2-y_1}{x_2-x_1}\\
  = \frac{0-(-5)}{0-(-3)}\\
  = \frac{5}{3}$\\
And,\\
$m_{IJ} &=& \frac{y_2-y_1}{x_2-x_1}\\
  = \frac{10-0}{6-0}\\
  = \frac{10}{6}\\
  = \frac{5}{3}$\\
So $m_{HI} = m_{IJ}$ and $I$ is a common point, \\
Therefore $H, I$ and $J$ are collinear.
\item $m_{KL} &=& \frac{y_2-y_1}{x_2-x_1}\\
  = \frac{1-2)}{-3-(-6)}\\
  = -\frac{1}{3}$\\
And,\\
$m_{LM} &=& \frac{y_2-y_1}{x_2-x_1}\\
  = \frac{-1-1}{1-(-3)}\\
  = \frac{-2}{4}\\
  = -\frac{1}{2}$\\
So $m_{HI}\ne m_{IJ}$, therefore $H, I$ and $J$ are not collinear.
\end{enumerate}
\item $m_{PQ} = \frac{y_2-y_1}{x_2-x_1}\\
  = \frac{-2-5}{2-(-6)}\\
  = \frac{-4}{8}\\
  = -\frac{1}{2}\\$
And, \\
  $m_{QR} = \frac{y_2-y_1}{x_2-x_1}\\
  = \frac{r-(-2)}{-3-2}\\
  = \frac{r+2}{-5}\\
  = -\frac{1}{2}\\$
So,\\
$-\frac{1}{2} = \frac{r+2}{-5}\\
  (-1)\times(-5) = 2(r+2)\\
  5 = 2r+4\\
  5-4 = 2r\\
  1 = 2r\\
  r = \frac{1}{2}$

\item $m_{PQ} = \frac{y_2-y_1}{x_2-x_1}\\
  1 = \frac{0-(-7)}{q-(-1)}\\
  q+1 = 7\\
  q = 6\\$
\end{enumerate}}
\end{solutions}

\begin{exercises}{}{
\begin{enumerate}[itemsep=5pt, label=\textbf{\arabic*}. ]
\item Find the mid-points of the following lines:
  \begin{enumerate}[noitemsep, label=\textbf{(\alph*)} ]
\item $A(2;5)$, $B(-4;7)$
\item $C(5;9)$, $D(23;55)$
\item $E(x+2;y-1)$, $F(x-5;y-4)$
\end{enumerate}
\begin{exercises}{}{
    \begin{enumerate}[itemsep=5pt, label=\textbf{\arabic*}. ]
    \item Find the mid-points of the lines between the following points:
      \begin{enumerate}[noitemsep, label=\textbf{(\alph*)} ]
      \item $A(2;5)$, $B(-4;7)$
      \item $C(5;9)$, $D(23;55)$
      \item $E(x+2;y-1)$, $F(x-5;y-4)$
      \end{enumerate}
    \item The mid-point $M$ of $PQ$ is $(3;9)$. Find $P$ if $Q$ is $(-2;5)$.
    \item $PQRS$ is a parallelogram with the points $P(5;3)$, $Q(2;1)$ and $R(7;-3)$. Find point $S$.
    \end{enumerate}
}
\end{exercises}    

 \begin{solutions}{}{
\begin{enumerate}[itemsep=5pt, label=\textbf{\arabic*}. ] 
\item
  \begin{enumerate}[noitemsep, label=\textbf{(\alph*)} ]
\item $M_{AB} = (\frac{x_1+x_2}{2};\frac{y_1+y_2}{2}) \\
    &= (\frac{2-4}{2};\frac{5+7}{2}) \\
    &= (\frac{-2}{2};\frac{12}{2}) \\
    &= (-1;6)$
\item $M_{CD} = (\frac{x_1+x_2}{2};\frac{y_1+y_2}{2}) \\
    = (\frac{5+23}{2};\frac{9+55}{2}) \\
    = (\frac{28}{2};\frac{64}{2}) \\
    = (14;32)$ 
\item $M_{EF} = (\frac{x_1+x_2}{2};\frac{y_1+y_2}{2}) \\
    = (\frac{x+2+x-5}{2};\frac{y-1+y-4}{2}) \\
    = (\frac{2x-3}{2};\frac{2y-5}{2}) \\ $
\end{enumerate}
\item $M_{PQ} = (\frac{x_1+x_2}{2};\frac{y_1+y_2}{2}) \\
   (3;9) = (\frac{x-2}{2};\frac{y+5}{2}) \\ $
Solve for $x$: \\
  $3 = \frac{x-2}{2} \\
  6 = x-2 \\
  x=8$ \\
Solve for $y$: \\
  $9 = \frac{y+5}{2} \\
  18 = y+5 \\
  y=13$ \\  
Therefore $P(8;13)$ \\
\item $M_{QR} = (\frac{x_1+x_2}{2};\frac{y_1+y_2}{2})\\
   = (\frac{2+7}{2};\frac{1-3}{2}) \\ 
   = (\frac{9}{2};\frac{-2}{2}) \\ 
   = (\frac{9}{2};-1)$\\
Use mid-point $M$ to find the coordinates of $S$: \\
 $M_{QR} = (\frac{x_1+x_2}{2};\frac{y_1+y_2}{2}) \\
  (\frac{9}{2};-1) = (\frac{x+5}{2};\frac{y+3}{2}) \\ $
Solve for $x$: \\
  $\frac{9}{2} = \frac{x+5}{2} \\ 
  9 = x+5 \\ 
  x = 4 \\ $
Solve for $y$:
  $-1 = \frac{y+3}{2} \\ 
  -2 = y+3 \\ 
  y = -5 \\ $
Therefore $S(4;-5)$ \\
\end{enumerate}}
\end{solutions}
%End of Chapter Exercises
\begin{eocexercises}{}
  \begin{enumerate}[noitemsep, label=\textbf{\arabic*}. ] 
  \item Represent the following figures in the Cartesian plane: 
    \begin{enumerate}[noitemsep, label=\textbf{(\alph*)} ]
    \item Triangle $DEF$ with $D(1;2)$, $E(3;2)$ and $F(2;4)$ 
    \item Quadrilateral $GHIJ$ with $G(2;-1)$, $H(0;2)$, $I(-2;-2)$ and $J(1;-3)$
    \item Quadrilateral $MNOP$ with $M(1;1)$, $N(-1;3)$, $O(-2;3)$ and $P(-4;1)$ 
    \item Quadrilateral $WXYZ$ with $W(1;-2)$, $X(-1;-3)$, $Y(2;-4)$ and $Z(3;-2)$
    \end{enumerate}
  \item In the diagram below, the vertices of the quadrilateral are $F(2;0)$, $G(1;5)$, $H(3;7)$ and $I(7;2)$.
    \setcounter{subfigure}{0}
    \begin{figure}[H] % horizontal\label{m39107*id63458}
      \begin{center}
        \scalebox{.8}{
          \begin{pspicture}(-5,-5)(5.5,5.5)
            \psaxes[linewidth=1pt,labels=all,ticks=all]{<->}(0,0)(-1,-1)(7.5,7.5)
            \pspolygon[linewidth=1pt](2,0)(1,5)(3,7)(7,2)(2,0)
            \psdots(2,0)(1,5)(3,7)(7,2)
            \uput[d](1.3,.6){\normalsize{$F (2;0)$}}
            \uput[ul](1.2,5){\normalsize{$G (1;5)$}}
            \uput[u](3,7){\normalsize{$H (3;7)$}}
            \uput[r](7,2){\normalsize{$I (7;2)$}}
            \uput[l](8,0){\Large{$x$}}
            \uput[d](0,8){\Large{$y$}}
            \uput[d](-0.2,0){\Large{$0$}}
          \end{pspicture}
        }
      \end{center}
    \end{figure}  
    \begin{enumerate}[noitemsep, label=\textbf{(\alph*)} ]
    \item Calculate the lengths of the opposite sides of $FGHI$?
    \item Are the opposite sides of $FGHI$ parallel?
    \item Do the diagonals of $FGHI$ bisect each other?
    \item Can you state what type of quadrilateral $FGHI$ is? Give reasons for your answer.
    \end{enumerate}
  \item Consider a quadrilateral $ABCD$ with vertices $A(3;2)$, $B(4;5)$, $C(1;7)$ and $D(1;3)$.
    \begin{enumerate}[noitemsep, label=\textbf{(\alph*)} ]
    \item  Draw the quadrilateral.
    \item  Find the lengths of the sides of the quadrilateral.
    \end{enumerate}
  \item $ABCD$ is a quadrilateral with vertices $A(0;3)$, $B(4;3)$, $C(5;-1)$ and $D(-1;-1)$.
    \begin{enumerate}[noitemsep, label=\textbf{(\alph*)} ]
    \item Show that:
      \begin{enumerate}[noitemsep, label=\textbf{\roman*}. ] 
      \item $AD = BC$
      \item $AB \parallel DC$
      \end{enumerate}
    \item What type of quadrilateral is $ABCD$?
    \item Show that the diagonals $AC$ and $BD$ do not bisect each other.
    \end{enumerate}
  \item $P$, $Q$, $R$ and $S$ are the points $(-2;0)$, $(2;3)$, $(5;3)$, $(-3;-3)$ respectively.
    \begin{enumerate}[noitemsep, label=\textbf{(\alph*)} ]
    \item Show that:
      \begin{enumerate}[noitemsep, label=\textbf{\roman*}. ] 
      \item $SR = 2PQ$
      \item $SR \parallel PQ$
      \end{enumerate}
    \item Calculate:
      \begin{enumerate}[noitemsep, label=\textbf{\roman*}. ] 
      \item $PS$
      \item $QR$
      \end{enumerate}
    \item What kind of quadrilateral is $PQRS$? Give reasons for your answer.
    \end{enumerate}
  \item $EFGH$ is a parallelogram with vertices $E(-1;2)$, $F(-2;-1)$ and $G(2;0)$. Find the coordinates of $H$ by using the fact that the diagonals of a parallelogram bisect each other.
  \item  $PQRS$ is a quadrilateral with points $P(0;-3)$, $Q(-2;5)$, $R(3;2)$ and $S(3;-2)$  in the Cartesian plane.
    \begin{enumerate}[noitemsep, label=\textbf{(\alph*)} ]
    \item Find the length of $QR$.
    \item Find the gradient of $PS$.
    \item Find the mid-point of $PR$.
    \item Is $PQRS$ a parallelogram?  Give reasons for your answer.
    \end{enumerate}
  \item $A(-2;3)$ and $B(2;6)$ are points in the Cartesian plane. $C(a;b)$ is the mid-point of $AB$. Find the values of $a$ and $b$.
  \item Consider triangle $ABC$ with vertices $A(1; 3)$, $B(4;1)$ and $C(6; 4)$.
    \begin{enumerate}[noitemsep, label=\textbf{(\alph*)} ]
    \item Sketch triangle $ABC$ on the Cartesian plane. 
    \item Show that $ABC$ is an isosceles triangle.
    \item Determine the coordinates of $M$, the mid-point of $AC$.
    \item Determine the gradient of $AB$.
    \item Show that the following points are collinear: $A$, $B$ and $D(7;-1)$.
    \end{enumerate}
  \item In the diagram, $A$ is the point $(-6;1)$ and $B$ is the point $(0;3)$
    \setcounter{subfigure}{0}
    \begin{figure}[H] % horizontal\label{m39107*id63458}
      \begin{center}
        \scalebox{.8}{
          \begin{pspicture}(-5,-5)(5.5,5.5)
            \psaxes[linewidth=1pt,labels=all,ticks=all]{<->}(0,0)(-10,-1)(1,7)
            \psline[linewidth=.05cm](-6,1)(0,3)
            \uput[d](-6,1){\Large{$A (-6;1)$}}
            \uput[d](-5.9,1.2){\qdisk(0,0){3pt}}
            \uput[ul](-.2,3){\Large{$B (0;3)$}}
            \uput[d](0,3.2){\qdisk(0,0){3pt}}
            \uput[l](1.5,0){\Large{$x$}}
            \uput[d](0,7.5){\Large{$y$}}
            \uput[d](-0.2,0.1){\Large{$0$}}
          \end{pspicture}
        }
      \end{center}
    \end{figure} 
    \begin{enumerate}[noitemsep, label=\textbf{(\alph*)} ]
    \item Find the equation of line $AB$ .
    \item Calculate the length of $AB$.
    \end{enumerate}
  \end{enumerate}

\end{eocexercises}

 \begin{eocsolutions}{}{
\begin{enumerate}[itemsep=5pt, label=\textbf{\arabic*}. ] 

 \item  %solution1

\scalebox{.8} % Change this value to rescale the drawing.
{
\begin{pspicture}(0,-5.0)(10.0,5.0)
\definecolor{color2279}{rgb}{0.25882352941176473,0.24313725490196078,0.24313725490196078}
\definecolor{color3500}{rgb}{0.5529411764705883,0.5450980392156862,0.5450980392156862}
\rput(5.0,0.0){\psaxes[linewidth=1pt,ticksize=0.10583333cm]{<->}(0,0)(-5,-5)(5,5)}
\psline[linewidth=0.04](6.0,2.02)(7.02,4.04)(7.96,2.04)(6.02,2.04)
\psline[linewidth=0.04](6.04,0.98)(4.0,3.0)(2.94,3.0)(0.94,1.0218182)(6.0,1.0)
\psline[linewidth=0.04,linecolor=color2279](4.96,1.98)(3.0,-1.98)(5.84,-3.0)(6.96,-1.0)(4.96,2.0)
\psline[linewidth=0.04,linecolor=color3500](8.02,-2.02)(8.02,-2.02)(5.96,-2.0)(3.92,-3.0)(6.96,-4.04)(8.0,-2.02)
% \usefont{T1}{ptm}{m}{n}
\rput(5.7703123,1.97){\LARGE$D$}
% \usefont{T1}{ptm}{m}{n}
\rput(8.124531,1.99){\LARGE$E$}
% \usefont{T1}{ptm}{m}{n}
\rput(6.9954686,4.23){\LARGE$F$}
% \usefont{T1}{ptm}{m}{n}
\rput(4.079375,3.15){\LARGE$N$}
% \usefont{T1}{ptm}{m}{n}
\rput(2.8614063,3.15){\LARGE$O$}
% \usefont{T1}{ptm}{m}{n}
\rput(0.7646875,0.85){\LARGE$P$}
% \usefont{T1}{ptm}{m}{n}
\rput(6.22,0.89){\LARGE$M$}
% \usefont{T1}{ptm}{m}{n}
\rput(5.22,2.31){\LARGE$H$}
% \usefont{T1}{ptm}{m}{n}
\rput(7.21,-0.97){\LARGE$G$}
% \usefont{T1}{ptm}{m}{n}
\rput(2.8378124,-2.05){\LARGE$I$}
% \usefont{T1}{ptm}{m}{n}
\rput(5.864844,-3.17){\LARGE$J$}
% \usefont{T1}{ptm}{m}{n}
\rput(5.8660936,-1.87){\LARGE$W$}
% \usefont{T1}{ptm}{m}{n}
\rput(8.167031,-1.91){\LARGE$Z$}
% \usefont{T1}{ptm}{m}{n}
\rput(6.990625,-4.29){\LARGE$Y$}
% \usefont{T1}{ptm}{m}{n}
\rput(3.7123437,-3.09){\LARGE$X$}
% \usefont{T1}{ptm}{m}{n}
\rput(6.9692187,2.63){\LARGE\textbf{(a)}}
% \usefont{T1}{ptm}{m}{n}
\rput(5.3392186,-0.67){\LARGE\textbf{(b)}}
% \usefont{T1}{ptm}{m}{n}
\rput(3.4392188,2.01){\LARGE\textbf{(c)}}
% \usefont{T1}{ptm}{m}{n}
\rput(6.559219,-2.77){\LARGE\textbf{(d)}}
\rput(10.2,0.2){\LARGE$x$}
\rput(5.2,5.2){\LARGE$y$}
\end{pspicture} 
}

\item  %solution2
    \begin{enumerate}[itemsep=5pt, label=\textbf{(\alph*)} ] 
\item
$d_{FG} = \sqrt{(x_1-x_2)^2+(y_1-y_2)^2}\\
    = \sqrt{(1-2)^2+(5-0)^2}\\
    = \sqrt{(-1)^2+(5)^2}\\
    = \sqrt{26}$\\

$d_{IH} = \sqrt{(x_1-x_2)^2+(y_1-y_2)^2}\\
    = \sqrt{(7-3)^2+(2-7)^2}\\
    = \sqrt{(4)^2+(-5)^2}\\
    = \sqrt{41}$\\
Opposite sides $FG$ and $IH$ are not equal.\\
$d_{GH} = \sqrt{(x_1-x_2)^2+(y_1-y_2)^2}\\
    = \sqrt{(3-1)^2+(7-5)^2}\\
    = \sqrt{(2)^2+(2)^2}\\
    = \sqrt{8}$\\

$d_{FI} = \sqrt{(x_1-x_2)^2+(y_1-y_2)^2}\\
    = \sqrt{(2-7)^2+(0-2)^2}\\
    = \sqrt{(-5)^2+(-2)^2}\\
    = \sqrt{29}$\\
Opposite sides $GH$ and $FI$ are not equal.\\

\item
$m_{FG} = \dfrac{y_2-y_1}{x_2-x_1}\\
    = \frac{0-5}{2-1}\\
    = \frac{-5}{1}\\
    = -5$\\

$m_{IH} = \dfrac{y_2-y_1}{x_2-x_1}\\
    = \frac{2-7}{7-3}\\
    = \frac{-5}{4}\\
    = -\frac{-5}{4}$\\
Therefore $m_{FG}\ne m_{IH}$ and opposite sides are not parallel.\\

$m_{GH} = \dfrac{y_2-y_1}{x_2-x_1}\\
    = \dfrac{5-7}{1-3}\\
    = \dfrac{-2}{-2}\\
    = 1$\\
$m_{FI} = \dfrac{y_2-y_1}{x_2-x_1}\\
    = \frac{0-2}{2-7}\\
    = \frac{-2}{-5}\\
    = \frac{2}{5}$\\
Therefore $m_{GH}\ne m_{FI}$ and opposite sides are not parallel.\\

\item
$M_{GI} = (\dfrac{x_1+x_2}{2};\dfrac{y_1+y_2}{2})\\
    = (\frac{1+7}{2};\frac{5+2}{2})\\
    = (\frac{8}{2};\frac{7}{2})\\
    = (4;\frac{7}{2})$\\

$M_{FH} = (\dfrac{x_1+x_2}{2};\dfrac{y_1+y_2}{2})\\
    = (\frac{3+2}{2};\frac{7+0}{2})\\
    = (\frac{5}{2};\frac{7}{2})$\\
Therefore $M_{GI}\ne M_{FH}$ and diagonals do not bisect each other.\\

\item This is an ordinary quadrilateral.
\end{enumerate}

\item  %solution3
    \begin{enumerate}[itemsep=5pt, label=\textbf{(\alph*)} ] 
\item % Draw the quadrilateral.
% \begin{center}
\scalebox{.8}{
\begin{pspicture}(-5,-5)(5.5,5.5)
% \psaxes{<->}(0,0)(5,5)
% \psgrid[gridcolor=lightgray,linecolor=lightgray,subgriddiv=1](0,0)(-1,-1)(7,7)
\psaxes[linewidth=1pt,labels=all,ticks=all]{<->}(0,0)(-1,-1)(6.5,7.5)
\pspolygon[linewidth=1pt](3,2)(4,5)(1,7)(1,3)(3,2)
\psdots(3,2)(4,5)(1,7)(1,3)
% \psline[linewidth=.05cm](0,0)(3,3.5)
% \psline[linewidth=.05cm](0,0)(3,2)
% \psline[linewidth=.05cm](0,0)(3,0.5)
% \uput[ur](.9,3.5){\Large{$T$}}
\uput[d](3.3,2){\LARGE{$A (3;2)$}}
\uput[ul](1.6,7){\LARGE{$B (1;7)$}}
\uput[u](4.3,5){\LARGE{$C (4;5)$}}
\uput[l](1.6,2.5){\LARGE{$D (1;3)$}}
\uput[l](7,0){\LARGE{$x$}}
\uput[d](0,8){\LARGE{$y$}}
\uput[d](-0.2,0){\LARGE{$0$}}
\end{pspicture}
}
% \end{center}
\item
$d_{AB} = \sqrt{(x_1-x_2)^2+(y_1-y_2)^2}\\
    = \sqrt{(3-4)^2+(2-5)^2}\\
    = \sqrt{(-1)^2+(-3)^2}\\
    = \sqrt{10}$\\

$d_{BC} = \sqrt{(x_1-x_2)^2+(y_1-y_2)^2}\\
    = \sqrt{(4-1)^2+(5-7)^2}\\
    = \sqrt{(3)^2+(-2)^2}\\
    = \sqrt{13}$\\

$d_{CD} = \sqrt{(x_1-x_2)^2+(y_1-y_2)^2}\\
    = \sqrt{(1-1)^2+(7-3)^2}\\
    = \sqrt{0+(4)^2}\\
    = \sqrt{16}\\
    = 4$\\

$d_{DA} = \sqrt{(x_1-x_2)^2+(y_1-y_2)^2}\\
    = \sqrt{(1-3)^2+(3-2)^2}\\
    = \sqrt{(-2)^2+(1)^2}\\
    = \sqrt{5}$\\
\end{enumerate}
\item  %solution4
    \begin{enumerate}[itemsep=5pt, label=\textbf{(\alph*)} ] 
\item 
\begin{enumerate}[noitemsep, label=\textbf{\roman*}. ] 
\item
$d_{AD} = \sqrt{(x_1-x_2)^2+(y_1-y_2)^2}\\
    = \sqrt{(0-(-1))^2+(3-(-1))^2}\\
    = \sqrt{(1)^2+(4)^2}\\
    = \sqrt{17}$\\

$d_{BC} = \sqrt{(x_1-x_2)^2+(y_1-y_2)^2}\\
    = \sqrt{(4-5)^2+(3-(-1))^2}\\
    = \sqrt{(-1)^2+(4)^2}\\
    = \sqrt{17}$\\
Therefore opposite sides $AD$ and $BC$ are equal.\\
\item
$m_{AB} = \dfrac{y_2-y_1}{x_2-x_1}\\
    = \frac{3-3}{0-4}\\
    = \frac{0}{-4}\\
    = 0$\\
$m_{DC} = \dfrac{y_2-y_1}{x_2-x_1}\\
    = \frac{-1+1}{-1-5}\\
    = \frac{0}{-6}\\
    = 0$\\
Gradients are equal therefore opposite sides $AB$ and $DC$ are parallel.\\
\end{enumerate}
\item An isosceles trapezium; one pair of opposite sides equal in length and one pair of opposite sides parallel.
\item
$M_{AC} = (\dfrac{x_1+x_2}{2};\dfrac{y_1+y_2}{2})\\
    = (\frac{0+5}{2};\frac{3-1}{2})\\
    = (\frac{5}{2};\frac{2}{2})\\
    = (\frac{5}{2};1)$\\

$M_{BD} = (\dfrac{x_1+x_2}{2};\dfrac{y_1+y_2}{2})\\
    = (\frac{4-1}{2};\frac{3-1}{2})\\
    = (\frac{3}{2};\frac{2}{2})\\
    = (\frac{3}{2};1)$\\

Therefore diagonals do not bisect each other.
\end{enumerate}

\item  %solution5 
 \begin{enumerate}[itemsep=5pt, label=\textbf{(\alph*)} ] 
  \item 
    \begin{enumerate}[noitemsep, label=\textbf{\roman*}. ] 
\item 
$d_{PQ} = \sqrt{(x_1-x_2)^2+(y_1-y_2)^2}\\
    = \sqrt{(-2-2)^2+(0-3)^2}\\
    = \sqrt{(-4)^2+(-3)^2}\\
    = \sqrt{25}\\
    = 5$\\

$d_{SR} = \sqrt{(x_1-x_2)^2+(y_1-y_2)^2}\\
    = \sqrt{(-3-5)^2+(-3-3)^2}\\
    = \sqrt{(-8)^2+(-6)^2}\\
    = \sqrt{100}\\
    = 10$\\
Therefore $SR=2PQ$.
\item
$m_{PQ} = \dfrac{y_2-y_1}{x_2-x_1}\\
    = \frac{3-0}{2-(-2)}\\
    = \frac{3}{4}$\\

$m_{SR} = \dfrac{y_2-y_1}{x_2-x_1}\\
    = \frac{-3-3}{-3-5}\\
    = \frac{-6}{-8}\\
    = \frac{3}{4}$\\
Therefore $m_{PQ}=m_{SR}$.
\end{enumerate}

\item

\begin{enumerate}[noitemsep, label=\textbf{\roman*}. ] 
 \item 
$d_{PS} = \sqrt{(x_1-x_2)^2+(y_1-y_2)^2}\\
    = \sqrt{(-2-(-3))^2+(0-(-3))^2}\\
    = \sqrt{(1)^2+(3)^2}\\
    = \sqrt{10}$

\item
$d_{QR} = \sqrt{(x_1-x_2)^2+(y_1-y_2)^2}\\
    = \sqrt{(2-5)^2+(3-3)^2}\\
    = \sqrt{(-3)^2+0}\\
    = \sqrt{9}\\
    = 3$
\end{enumerate}

\item Trapezium.%Need reasons why!
\end{enumerate}

\item  %solution6
$M_{EG} = (\dfrac{x_1+x_2}{2};\dfrac{y_1+y_2}{2})\\
    = (\frac{-1+2}{2};\frac{2+0}{2})\\
    = (\frac{1}{2};1)$\\

$M_{FH} = (\dfrac{x_1+x_2}{2};\dfrac{y_1+y_2}{2})\\
	= (\frac{-2+x}{2};\frac{-1+y}{2})\\
 (\frac{1}{2};1) = (\frac{-2+x}{2};\frac{-1+y}{2})$\\

Solve for $x$: \\
  $\frac{1}{2} = \frac{-2+x}{2} \\ 
  1 = -2+x\\ 
  x = 3 \\ $
Solve for $y$: \\
  $1 = \frac{-1+y}{2} \\ 
  2 = -1+y \\ 
  y = 3 \\ $
Therefore $H(3;3)$ \\
\item  %solution7
\begin{enumerate}[itemsep=5pt, label=\textbf{(\alph*)} ] 
\item
$d_{QR} = \sqrt{(x_1-x_2)^2+(y_1-y_2)^2}\\
    = \sqrt{(-2-3)^2+(5-2)^2}\\
    = \sqrt{(-5)^2+(3)^2}\\
    = \sqrt{34}$
\item
$m_{PS} = \dfrac{y_2-y_1}{x_2-x_1}\\
    = \frac{-3+2}{0-3}\\
    = \frac{-1}{-3}\\
    = \frac{1}{3}$
\item
$M_{QR} = (\dfrac{x_1+x_2}{2};\dfrac{y_1+y_2}{2})\\
    = (\frac{0+3}{2};\frac{-3+2}{2})\\
    = (\frac{3}{2};\frac{-1}{2})$
\item
$m_{RS} = \dfrac{y_2-y_1}{x_2-x_1}\\
    = \frac{-2-2}{3-3}\\
    = \frac{-4}{0}\\
    = \mbox{undefined}$\\
And, \\
$m_{QR} = \dfrac{y_2-y_1}{x_2-x_1}\\
    = \frac{2-5}{3-(-2)}\\
    = \frac{-3}{5}$\\
Therefore $PQRS$ is not a parallelogram.
\end{enumerate}
\item  %solution8
$M_{AB} = (\dfrac{x_1+x_2}{2};\dfrac{y_1+y_2}{2})\\
	= (\frac{-2-2}{2};\frac{3+6}{2})\\
 (a;b) = (0;\frac{9}{2})$\\
Therefore $a=0$ and $b=\frac{9}{2})$.
\item  %solution9
\begin{enumerate}[itemsep=5pt, label=\textbf{(\alph*)} ] 
\item
\scalebox{.8}{
\begin{pspicture}(-5,-5)(5.5,5.5)
% \psaxes{<->}(0,0)(5,5)
% \psgrid[gridcolor=lightgray,linecolor=lightgray,subgriddiv=1](0,0)(-1,-1)(7,7)
\psaxes[linewidth=1pt,labels=all,ticks=all]{<->}(0,0)(-1,-1)(6.5,5)
\pspolygon[linewidth=1pt](4,1)(6,4)(1,3)(4,1)
\psdots(4,1)(6,4)(1,3)(4,1)
% \psline[linewidth=.05cm](0,0)(3,3.5)
% \psline[linewidth=.05cm](0,0)(3,2)
% \psline[linewidth=.05cm](0,0)(3,0.5)
% \uput[ur](.9,3.5){\Large{$T$}}
\uput[d](0.8,3){\LARGE{$A (1;3)$}}
\uput[d](4,1){\LARGE{$B (4;1)$}}
\uput[u](6,4){\LARGE{$C (6;4)$}}

\uput[l](7,0){\LARGE{$x$}}
\uput[d](0,5.5){\LARGE{$y$}}
\uput[d](-0.2,0){\LARGE{$0$}}
\end{pspicture}
}
\item
$d_{AB} = \sqrt{(x_1-x_2)^2+(y_1-y_2)^2}\\
    = \sqrt{(1-4)^2+(3-1)^2}\\
    = \sqrt{(-3)^2+(2)^2}\\
    = \sqrt{13}$\\
$d_{BC} = \sqrt{(x_1-x_2)^2+(y_1-y_2)^2}\\
    = \sqrt{(4-6)^2+(1-4)^2}\\
    = \sqrt{(-2)^2+(-3)^2}\\
    = \sqrt{13}$\\
$d_{AC} = \sqrt{(x_1-x_2)^2+(y_1-y_2)^2}\\
    = \sqrt{(1-6)^2+(3-4)^2}\\
    = \sqrt{(-5)^2+(-1)^2}\\
    = \sqrt{26}$
\item
Two sides of the triangle are equal in length, therefore $\triangle ABC$ is isosceles.
\item
$M_{AC} = (\dfrac{x_1+x_2}{2};\dfrac{y_1+y_2}{2})\\
    = (\frac{1+6}{2};\frac{3+4}{2})\\
    = (\frac{7}{2};\frac{7}{2})$
\item
$m_{AB} = \dfrac{y_2-y_1}{x_2-x_1}\\
    = \frac{1-3}{4-1}\\
    = \frac{-2}{3}$\\
$m_{BD} = \dfrac{y_2-y_1}{x_2-x_1}\\
    = \frac{-1-1}{7-4}\\
    = \frac{-2}{3}$\\
$m_{AD} = \dfrac{y_2-y_1}{x_2-x_1}\\
    = \frac{-1-3}{7-1}\\
    = \frac{-2}{3}$\\
Therefore $A,B$ and $D$ are collinear.
\end{enumerate}
\item  %solution10
\begin{enumerate}[itemsep=5pt, label=\textbf{(\alph*)} ] 
\item
$m_{AB} = \dfrac{y_2-y_1}{x_2-x_1}\\
    = \frac{1-3}{-6-0}\\
    = \frac{-2}{-6}\\
    = \frac{1}{3}$\\

Therefore equation of the line $AB$ is $y=\frac{1}{3}+3$.
\item
$d_{AB} = \sqrt{(x_1-x_2)^2+(y_1-y_2)^2}\\
    = \sqrt{(-6+0)^2+(1-3)^2}\\
    = \sqrt{(-6)^2+(-2)^2}\\
    = \sqrt{40}$\\
\end{enumerate}

\end{enumerate}}
\end{eocsolutions}


