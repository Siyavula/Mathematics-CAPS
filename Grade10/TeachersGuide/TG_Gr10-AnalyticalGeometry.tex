\chapter{Analytical geometry}
\section{Distance between two points}
\begin{exercises}{}{
\begin{enumerate}[label=\textbf{\arabic*}.]
\item Find the length of $AB$ if:
 \begin{enumerate}[noitemsep, label=\textbf{(\alph*)} ] 
\item $A(2;7)$ and $B(-3;5)$.
\item $A(-3;5)$ and $B(-9;1)$.
\item $A(x;y)$ and $B(x+4;y-1)$.
\end{enumerate}
\item The length of $CD=5$. Find the missing coordinate if:
 \begin{enumerate}[noitemsep, label=\textbf{(\alph*)} ] 
\item $C(6;-2)$ and $D(x;2)$.
\item $C(4;y)$ and $D(1;-1)$.
\end{enumerate}
\end{enumerate}
\practiceinfo
\par 
\par \begin{tabular}[h]{cccccc}
(1.) lgv  &  (2.) liZ  &  (3.) liB  &  (4.) lac  &  (5.) lax  &  (6.) laa  &  (7.) laY  &  (8.) lag  &  (9.) la4  &  (10.) l4o  & \end{tabular}
}
\end{exercises}


 \begin{solutions}{}{
\begin{enumerate}[itemsep=5pt, label=\textbf{\arabic*}. ] 


\item solution 1
\item solution 2

\end{enumerate}
}
\end{solutions}


\section{Gradient between two points}
\begin{exercises}{}{
\begin{enumerate}[noitemsep, label=\textbf{\arabic*}. ]
\item Find the gradient of $AB$ if
 \begin{enumerate}[noitemsep, label=\textbf{(\alph*)} ] 
\item $A(7;10)$ and $B(-4;1)$
\item $A(-5;-9)$ and $B(3;2)$
\item $A(x-3;y)$ and $B(x;y+4)$
\end{enumerate}
\item If the gradient of $CD=\frac{2}{3}$, find $P$ given
\begin{enumerate}[noitemsep, label=\textbf{(\alph*)} ] 
\item $C(16;2)$ and $D(8;P)$
\item $C(3;2P)$ and $D(9;14)$
\end{enumerate}
\end{enumerate}
\practiceinfo
\par 
\par \begin{tabular}[h]{cccccc}
(1.) lgv  &  (2.) liZ  &  (3.) liB  &  (4.) lac  &  (5.) lax  &  (6.) laa  &  (7.) laY  &  (8.) lag  &  (9.) la4  &  (10.) l4o  & \end{tabular}
}
\end{exercises}


 \begin{solutions}{}{
\begin{enumerate}[itemsep=5pt, label=\textbf{\arabic*}. ] 


\item solution 1
\item solution 2

\end{enumerate}}
\end{solutions}


\begin{exercises}{}{
  \begin{enumerate}[itemsep=5pt, label=\textbf{\arabic*}. ]
\item Determine whether $AB$ and $CD$ are parallel, perpendicular or neither if:
\begin{enumerate}[noitemsep, label=\textbf{(\alph*)} ]
\item $A(3;-4)$, $B(5;2)$, $C(-1;-1)$, $D(7;23)$.
\item $A(3;-4)$, $B(5;2)$, $C(-1;-1)$, $D(0;-4)$.
\item $A(3;-4)$, $B(5;2)$, $C(-1;-1)$, $D(1;2)$.
\end{enumerate}
\item Determine whether the following points are collinear:
\begin{enumerate}[noitemsep, label=\textbf{(\alph*)} ]
\item $E(0;3)$, $F(-2;5)$, $G(2;1)$.
\item $H(-3;-5)$, $I(-0;0)$, $J(6;10)$.
\item $K(-6;2)$, $L(-3;1)$, $M(1;-1)$.
\end{enumerate}
\item Points $P(-6;2)$, $Q(2;-2)$ and $R(-3;r)$ lie on a straight line. Find the value of $r$.
\item Line $PQ$ with $P(-1;-7)$ and $Q(q;0)$ has a gradient of $1$. Find $q$.
\end{enumerate}
\end{enumerate}
\practiceinfo
\par 
\par \begin{tabular}[h]{cccccc}
(1.) lgv  &  (2.) liZ  &  (3.) liB  &  (4.) lac  &  (5.) lax  &  (6.) laa  &  (7.) laY  &  (8.) lag  &  (9.) la4  &  (10.) l4o  & \end{tabular}
}
\end{exercises}


 \begin{solutions}{}{
\begin{enumerate}[itemsep=5pt, label=\textbf{\arabic*}. ] 


\item solution 1
\item solution 2
\item solution 3
\item solution 4

\end{enumerate}}
\end{solutions}


\section{Mid-point of a line}
\begin{exercises}{}{
\begin{enumerate}[itemsep=5pt, label=\textbf{\arabic*}. ]
\item Find the mid-points of the following lines:
  \begin{enumerate}[noitemsep, label=\textbf{(\alph*)} ]
\item $A(2;5)$, $B(-4;7)$
\item $C(5;9)$, $D(23;55)$
\item $E(x+2;y-1)$, $F(x-5;y-4)$
\end{enumerate}
\Tip{Remember to draw sketches!}
\begin{exercises}{}{
    \begin{enumerate}[itemsep=5pt, label=\textbf{\arabic*}. ]
    \item Find the mid-points of the lines between the following points:
      \begin{enumerate}[noitemsep, label=\textbf{(\alph*)} ]
      \item $A(2;5)$, $B(-4;7)$
      \item $C(5;9)$, $D(23;55)$
      \item $E(x+2;y-1)$, $F(x-5;y-4)$
      \end{enumerate}
    \item The mid-point $M$ of $PQ$ is $(3;9)$. Find $P$ if $Q$ is $(-2;5)$.
    \item $PQRS$ is a parallelogram with the points $P(5;3)$, $Q(2;1)$ and $R(7;-3)$. Find point $S$.
    \item $MPNO$ is a parallelogram with the points $M(5;3)$, $N(2;1)$ and $O(7;-3)$. Find point $P$.
    \end{enumerate}
\practiceinfo
\par 
\par \begin{tabular}[h]{cccccc}
(1.) lgv  &  (2.) liZ  &  (3.) liB  &  (4.) lac  &  (5.) lax  &  (6.) laa  &  (7.) laY  &  (8.) lag  &  (9.) la4  &  (10.) l4o  & \end{tabular}
}
\end{exercises}    


 \begin{solutions}{}{
\begin{enumerate}[itemsep=5pt, label=\textbf{\arabic*}. ] 


\item solution 1
\item solution 2
\item solution 3
\item solution 4

\end{enumerate}}
\end{solutions}


\begin{eocexercises}{}
  \begin{enumerate}[noitemsep, label=\textbf{\arabic*}. ] 
  \item Represent the following figures in the Cartesian plane: 
    \begin{enumerate}[noitemsep, label=\textbf{(\alph*)} ]
    \item Triangle $DEF$ with $D(1;2)$, $E(3;2)$ and $F(2;4)$ 
    \item Quadrilateral $GHIJ$ with $G(2;-1)$, $H(0;2)$, $I(-2;-2)$ and $J(1;-3)$
    \item Quadrilateral $MNOP$ with $M(1;1)$, $N(-1;3)$, $O(-2;3)$ and $P(-4;1)$ 
    \item Quadrilateral $WXYZ$ with $W(1;-2)$, $X(-1;-3)$, $Y(2;-4)$ and $Z(3;-2)$
    \end{enumerate}
  \item In the diagram below, the vertices of the quadrilateral are $F(2;0)$, $G(1;5)$, $H(3;7)$ and $I(7;2)$.
    \setcounter{subfigure}{0}
    \begin{figure}[H] % horizontal\label{m39107*id63458}
      \begin{center}
        \scalebox{.8}{
          \begin{pspicture}(-5,-5)(5.5,5.5)
            \psaxes[linewidth=1pt,labels=all,ticks=all]{<->}(0,0)(-1,-1)(7.5,7.5)
            \pspolygon[linewidth=1pt](2,0)(1,5)(3,7)(7,2)(2,0)
            \psdots(2,0)(1,5)(3,7)(7,2)
            \uput[d](1.3,.6){\normalsize{$F (2;0)$}}
            \uput[ul](1.2,5){\normalsize{$G (1;5)$}}
            \uput[u](3,7){\normalsize{$H (3;7)$}}
            \uput[r](7,2){\normalsize{$I (7;2)$}}
            \uput[l](8,0){\Large{$x$}}
            \uput[d](0,8){\Large{$y$}}
            \uput[d](-0.2,0){\Large{$0$}}
          \end{pspicture}
        }
      \end{center}
    \end{figure}  
    \begin{enumerate}[noitemsep, label=\textbf{(\alph*)} ]
    \item Calculate the lengths of the opposite sides of $FGHI$?
    \item Are the opposite sides of $FGHI$ parallel?
    \item Do the diagonals of $FGHI$ bisect each other?
    \item Can you state what type of quadrilateral $FGHI$ is? Give reasons for your answer.
    \end{enumerate}
  \item Consider a quadrilateral $ABCD$ with vertices $A(3;2)$, $B(1;7)$, $C(4;5)$ and $D(1;3)$.
    \begin{enumerate}[noitemsep, label=\textbf{(\alph*)} ]
    \item  Draw the quadrilateral.
    \item  Find the lengths of the sides of the quadrilateral.
    \end{enumerate}
  \item $ABCD$ is a quadrilateral with vertices $A(0;3)$, $B(4;3)$, $C(5;-1)$ and $D(-1;-1)$.
    \begin{enumerate}[noitemsep, label=\textbf{(\alph*)} ]
    \item Show that:
      \begin{enumerate}[noitemsep, label=\textbf{\roman*}. ] 
      \item $AD = BC$
      \item $AB \parallel DC$
      \end{enumerate}
    \item What type of quadrilateral is $ABCD$?
    \item Show that the diagonals $AC$ and $BD$ do not bisect each other.
    \end{enumerate}
  \item $P$, $Q$, $R$ and $S$ are the points $(-2;0)$, $(2;3)$, $(5;3)$, $(-3;-3)$ respectively.
    \begin{enumerate}[noitemsep, label=\textbf{(\alph*)} ]
    \item Show that:
      \begin{enumerate}[noitemsep, label=\textbf{\roman*}. ] 
      \item $SR = 2PQ$
      \item $SR \parallel PQ$
      \end{enumerate}
    \item Calculate:
      \begin{enumerate}[noitemsep, label=\textbf{\roman*}. ] 
      \item $PS$
      \item $QR$
      \end{enumerate}
    \item What kind of quadrilateral is $PQRS$? Give reasons for your answer.
    \end{enumerate}
  \item $EFGH$ is a parallelogram with vertices $E(-1;2)$, $F(-2;-1)$ and $G(2;0)$. Find the coordinates of $H$ by using the fact that the diagonals of a parallelogram bisect each other.
  \item  $PQRS$ is a quadrilateral with points $P(0;-3)$, $Q(-2;5)$, $R(3;2)$ and $S(3;-2)$  in the Cartesian plane.
    \begin{enumerate}[noitemsep, label=\textbf{(\alph*)} ]
    \item Find the length of $QR$.
    \item Find the gradient of $PS$.
    \item Find the mid-point of $PR$.
    \item Is $PQRS$ a parallelogram?  Give reasons for your answer.
    \end{enumerate}
  \item $A(-2;3)$ and $B(2;6)$ are points in the Cartesian plane. $C(a;b)$ is the mid-point of $AB$. Find the values of $a$ and $b$.
  \item Consider triangle $ABC$ with vertices $A(1; 3)$, $B(4;1)$ and $C(6; 4)$.
    \begin{enumerate}[noitemsep, label=\textbf{(\alph*)} ]
    \item Sketch triangle $ABC$ on the Cartesian plane. 
    \item Show that $ABC$ is an isosceles triangle.
    \item Determine the coordinates of $M$, the mid-point of $AC$.
    \item Determine the gradient of $AB$.
    \item Show that the following points are collinear: $A$, $B$ and $D(7;-1)$.
    \end{enumerate}
  \item In the diagram, $A$ is the point $(-6;1)$ and $B$ is the point $(0;3)$
    \setcounter{subfigure}{0}
    \begin{figure}[H] % horizontal\label{m39107*id63458}
      \begin{center}
        \scalebox{.8}{
          \begin{pspicture}(-5,-5)(5.5,5.5)
            \psaxes[linewidth=1pt,labels=all,ticks=all]{<->}(0,0)(-10,-1)(1,7)
            \psline[linewidth=.05cm](-6,1)(0,3)
            \uput[d](-6,1){\Large{$A (-6;1)$}}
            \uput[d](-5.9,1.2){\qdisk(0,0){3pt}}
            \uput[ul](-.2,3){\Large{$B (0;3)$}}
            \uput[d](0,3.2){\qdisk(0,0){3pt}}
            \uput[l](1.5,0){\Large{$x$}}
            \uput[d](0,7.5){\Large{$y$}}
            \uput[d](-0.2,0.1){\Large{$0$}}
          \end{pspicture}
        }
      \end{center}
    \end{figure} 
    \begin{enumerate}[noitemsep, label=\textbf{(\alph*)} ]
    \item Find the equation of line $AB$ .
    \item Calculate the length of $AB$.
    \end{enumerate}
  \end{enumerate}
\practiceinfo
\par 
\par \begin{tabular}[h]{cccccc}
(1.) lgv  &  (2.) liZ  &  (3.) liB  &  (4.) lac  &  (5.) lax  &  (6.) laa  &  (7.) laY  &  (8.) lag  &  (9.) la4  &  (10.) l4o  & \end{tabular}
\end{eocexercises}


 \begin{solutions}{}{
\begin{enumerate}[itemsep=5pt, label=\textbf{\arabic*}. ] 


\item solution 1
\item solution 2
\item solution 3
\item solution 4
\item solution 5
\item solution 6
\item solution 7
\item solution 8
\item solution 9
\item solution 10
\item solution 11
\item solution 12
\item solution 13
\item solution 14
\item solution 15
\item solution 16

\end{enumerate}}
\end{solutions}


