\chapter{Front matter}

\section{Support for educators}
Science education is about more than physics, chemistry and mathematics... It's about learning to think and to solve problems, which are valuable skills that can be applied through all spheres of life. Teaching these skills to our next generation is crucial in the current global environment where methodologies, technology and tools are rapidly evolving. Education should benefit from these fast moving developments. In our simplified model there are three layers to how technology can significantly influence your teaching and teaching environment. 

\subsubsection{First Layer: educator collaboration}
There are many tools that help educators collaborate more effectively. We know that communities of practice are powerful tools for the refinement of methodology, content and knowledge and are also superb for providing support to educators. One of the challenges facing community formation is the time and space to have sufficient meetings to build real communities and exchange practices, content and learnings effectively. Technology allows us to streamline this very effectively by transcending space and time. It is now possible to collaborate over large distances (transcending space) and when it is most appropriate for each individual (transcending time) by working virtually (email, mobile, online etc.).\par

Our textbooks have been re-purposed from content available on the Connexions website\\ (\underline{http://cnx.org/lenses/fhsst}). The content on this website is easily accessible and adaptable  as it is under an open licence, stored in an open format, based on an open standard, on an open-source platform, for free, where everyone can produce their own books. The content on Connexions is available under an open copyright license - CC-BY. This Creative Commons By Attribution Licence allows others to legally distribute, remix, tweak, and build upon the work available, even commercially, as long as the original author is credited for the original creation. This means that learners and educators are able to download, copy, share and distribute this content legally at no cost. It also gives educators the freedom to edit, adapt, translate and contextualise it, to better suit their teaching needs. \par

Connexions is a tool where individuals can share, but more importantly communities can form around the collaborative, online development of resources. Your community of educators can therefore:
\begin{itemize}[noitemsep]
\item form an online workgroup around the content;
\item make your own copy of the content;
\item edit sections of your own copy;
\item add your own content or replace existing content with your own content;
\item use other content that has been shared on the platform in your own work;
\item create your own notes / textbook / course material as a community.
\end{itemize}
Educators often want to share assessment items as this helps reduce workload, increase variety and improve quality. Currently all the solutions to the exercises contained in the textbooks have been uploaded onto our free and open online assessment bank called Monassis \underline{(www.monassis.com)}, with each exercise having a shortcode link to its solution on Monassis. To access the solution simply go to \underline{www.everythingmaths.co.za},  enter the shortcode, and you will be redirected to the solution on Monassis.\par

Monassis is similar to Connexions but is focused on the sharing of assessment items. Monassis contains a selection of test and exam questions with solutions, openly shared by educators. Educators can further search and browse the database by subject and grade and add relevant items to a test. The website automatically generates a test or exam paper with the corresponding memorandum for download.\par

By uploading all the end-of-chapter exercises and solutions to this open assessment bank, the larger community of educators in South Africa are provided with a wide selection of items to use in setting their tests and exams. More details about the use of Monassis as a collaboration tool are included in the Monassis section.\par

\subsubsection{Second Layer: classroom engagement}
In spite of the impressive array of rich media open educational resources available freely online (such as videos, simulations, exercises and presentations), only a small number of educators actively make use of them. Our investigations revealed that the overwhelming quantity, the predominant international context, and difficulty in correctly aligning them with the local curriculum level acts as deterrents. The opportunity here is that, if used correctly, they can make the classroom environment more engaging.\par

Presentations can be a first step to bringing material to life in ways that are more compelling than are possible with just a blackboard and chalk. There are opportunities to:
\begin{itemize}[noitemsep]
\item create more graphical representations of the content;
\item control timing of presented content more effectively;
\item allow learners to relive the lesson later if constructed well;
 \item supplement the slides with notes for later use;
\item embed key assessment items in advance to promote discussion; and
\item embed other rich media like videos.
\end{itemize}
Videos have been shown to be potentially both engaging and effective. They provide opportunities to:
\begin{itemize}[noitemsep]
\item present an alternative explanation;
\item challenge misconceptions without challenging an individual in the class; and
\item show an environment or experiment that cannot be replicated in the class which could be far away, too expensive or too dangerous.
\end{itemize}
Simulations are also very useful and can allow learners to:
\begin{itemize}[noitemsep]
\item have increased freedom to explore, rather than reproduce a fixed experiment or process;
\item explore expensive or dangerous environments more effectively; and
\item overcome implicit misconceptions.
\end{itemize}
We realised the opportunity for embedding a selection of rich media resources such as presentations, simulations, videos and links into the online version of Everything Maths and Everything Science at the relevant sections. This will not only present them with a selection of locally relevant and curriculum aligned resources, but also position these resources within the appropriate grade and section. Links to these online resources are recorded in the print or PDF versions of the  books, making them a tour-guide or credible pointer to the world of online rich media available. \par
\subsubsection{Third Layer: beyond the classroom}
The internet has provided many opportunities for self-learning and participation which were never before possible. There are huge stand-alone archives of videos like the Khan Academy which covers most Mathematics for Grades 1 - 12 and Science topics required in FET. These videos, if not used in class, provide opportunities for the learners to:
\begin{itemize}[noitemsep]
\item look up content themselves;
\item get ahead of class;
\item independently revise and consolidate their foundation; and
\item explore a subject to see if they find it interesting.
\end{itemize}
There are also many opportunities for learners to participate in science projects online as real participants (see the section on citizen cyberscience “On the web, everyone can be a scientist”). Not just simulations or tutorials but real science so that:
\begin{itemize}[noitemsep]
\item learners gain an appreciation of how science is changing;
\item safely and easily explore subjects that they would never have encountered before university;
\item contribute to real science (real international cutting edge science programmes);
\item have the possibility of making real discoveries even from their school computer laboratory; and
\item find active role models in the world of science.
\end{itemize}
In our book we've embedded opportunities to help educators and learners take advantage of all these resources, without becoming overwhelmed at all the content that is available online. 

\subsubsection{Tell us how to improve the book}
If you have any comments, thoughts or suggestions on the books, you can visit  \underline{www.everythingmaths.co.za} and in educator mode (you will see on the website how to do this), capture these in the text. These can range from sharing tips and ideas on the content in the textbook with your fellow educators, to discussing how to better explain concepts in class. Also, if you have picked up any errors in the book you can make a note of them here, and we will correct them in time for the next print run.\par

\section{On the Web, everyone can be a scientist}
Did you know that you can fold protein molecules, hunt for new planets around distant suns or simulate how malaria spreads in Africa, all from an ordinary PC or laptop connected to the Internet? And you don’t need to be a certified scientist to do this. In fact some of the most talented contributors are teenagers. The reason this is possible is that scientists are learning how to turn simple scientific tasks into competitive online games. \par

This is the story of how a simple idea of sharing scientific challenges on the Web turned into a global trend, called citizen cyberscience. And how you can be a scientist on the Web, too.

\subsubsection{Looking for Little Green Men}
A long time ago, in 1999, when the World Wide Web was barely ten years old and no one had heard of Google, Facebook or Twitter, a researcher at the University of California at Berkeley, David Anderson, launched an online project called SETI@home. SETI stands for Search for Extraterrestrial Intelligence. Looking for life in outer space.\par

Although this sounds like science fiction, it is a real and quite reasonable scientific project. The idea is simple enough. If there are aliens out there on other planets, and they are as smart or even smarter than us, then they almost certainly have invented the radio already. So if we listen very carefully for radio signals from outer space, we may pick up the faint signals of intelligent life.\par

Exactly what radio broadcasts aliens would produce is a matter of some debate. But the idea is that if they do, it would sound quite different from the normal hiss of background radio noise produced by stars and galaxies. So if you search long enough and hard enough, maybe you’ll find a sign of life. 
\par
It was clear to David and his colleagues that the search was going to require a lot of computers. More than scientists could afford. So he wrote a simple computer program which broke the problem down into smaller parts, sending bits of radio data collected by a giant radio-telescope to volunteers around the world. The volunteers agreed to download a programme onto their home computers that would sift through the bit of data they received, looking for signals of life, and send back a short summary of the result to a central server in California. \par

The biggest surprise of this project was not that they discovered a message from outer space. In fact, after over a decade of searching, no sign of extraterrestrial life has been found, although there are still vast regions of space that have not been looked at. The biggest surprise was the number of people willing to help such an endeavour. Over a million people have downloaded the software, making the total computing power of SETI@home rival that of even the biggest supercomputers in the world.

David was deeply impressed by the enthusiasm of people to help this project. And he realized that searching for aliens was probably not the only task that people would be willing to help with by using the spare time on their computers. So he set about building a software platform that would allow many other scientists to set up similar projects. You can read more about this platform, called BOINC, and the many different kinds of volunteer computing projects it supports today, at \underline{http://boinc.berkeley.edu/}. \par

There’s something for everyone, from searching for new prime numbers (PrimeGrid) to simulating the future of the Earth’s climate (ClimatePrediction.net). One of the projects, MalariaControl.net, involved researchers from the University of Cape Town as well as from universities in Mali and Senegal. \par

The other neat feature of BOINC is that it lets people who share a common interest in a scientific topic share their passion, and learn from each other. BOINC even supports teams – groups of people who put their computer power together, in a virtual way on the Web, to get a higher score than their rivals. So BOINC is a bit like Facebook and World of Warcraft combined – part social network, part online multiplayer game.\par

Here’s a thought: spend some time searching around BOINC for a project you’d like to participate in, or tell your class about. 

\subsubsection{You are a computer, too}
Before computers were machines, they were people. Vast rooms full of hundreds of government employees used to calculate the sort of mathematical tables that a laptop can produce nowadays in a fraction of a second. They used to do those calculations laboriously, by hand. And because it was easy to make mistakes, a lot of the effort was involved in double-checking the work done by others. \par

Well, that was a long time ago. Since electronic computers emerged over 50 years ago, there has been no need to assemble large groups of humans to do boring, repetitive mathematical tasks. Silicon chips can solve those problems today far faster and more accurately. But there are still some mathematical problems where the human brain excels.\par

Volunteer computing is a good name for what BOINC does: it enables volunteers to contribute computing power of their PCs and laptops. But in recent years, a new trend has emerged in citizen cyberscience that is best described as volunteer thinking. Here the computers are replaced by brains, connected via the Web through an interface called eyes. Because for some complex problems – especially those that involve recognizing complex patterns or three-dimensional objects – the human brain is still a lot quicker and more accurate than a computer.\par

Volunteer thinking projects come in many shapes and sizes. For example, you can help to classify millions of images of distant galaxies (GalaxyZoo), or digitize hand-written information associated with museum archive data of various plant species (Herbaria@home).  This is laborious work, which if left to experts would take years or decades to complete. But thanks to the Web, it’s possible to distribute images so that hundreds of thousands of people can contribute to the search. \par

Not only is there strength in numbers, there is accuracy, too. Because by using a technique called validation – which does the same sort of double-checking that used to be done by humans making mathematical tables – it is possible to practically eliminate the effects of human error. This is true even though each volunteer may make quite a few mistakes. So projects like Planet Hunters have already helped astronomers pinpoint new planets circling distant stars. The game FoldIt invites people to compete in folding protein molecules via a simple mouse-driven interface. By finding the most likely way a protein will fold, volunteers can help understand illnesses like Alzheimer’s disease, that depend on how proteins fold. \par

Volunteer thinking is exciting. But perhaps even more ambitious is the emerging idea of volunteer sensing: using  your laptop or even your mobile phone to collect data – sounds, images, text you type in – from any point on the planet, helping scientists to create global networks of sensors that can pick up the first signs of an outbreak of a new disease (EpiCollect), or the initial tremors associated with an earthquake (QuakeCatcher.net), or the noise levels around a new airport (NoiseTube).\par

There are about a billion PCs and laptops on the planet, but already 5 billion mobile phones. The rapid advance of computing technology, where the power of a ten-year old PC can easily be packed into a smart phone today, means that citizen cyberscience has a bright future in mobile phones. And this means that more and more of the world’s population can be part of citizen cyberscience projects. Today there are probably a few million participants in a few hundred citizen cyberscience initiatives. But there will soon be seven billion brains on the planet. That is a lot of potential citizen cyberscientists. 

You can explore much more about citizen cyberscience on the Web. There’s a great list of all sorts of projects, with brief summaries of their objectives, at \underline{http://distributedcomputing.info/} . BBC Radio 4 produced a short series on citizen science \underline{http://www.bbc.co.uk/radio4/science/citizenscience.shtml}  and you can subscribe to a newsletter about the latest trends in this field at \underline{http://scienceforcitizens.net/}  . The Citizen Cyberscience Centre, \underline{www.citizencyberscience.net}  which is sponsored by the South African Shuttleworth Foundation, is promoting citizen cyberscience in Africa and other developing regions.

\section{Blog posts}
\subsubsection{General blogs}
\begin{itemize}
\item Teachers Monthly - Education News and Resources
    \begin{itemize}[noitemsep]
      \item “We eat, breathe and live education! “
      \item “Perhaps the most remarkable yet overlooked aspect of the South African teaching community is its enthusiastic, passionate spirit. Every day, thousands of talented, hard-working teachers gain new insight from their work and come up with brilliant, inventive and exciting ideas. Teacher’s Monthly aims to bring teachers closer and help them share knowledge and resources.
      \item Our aim is twofold...
	    \begin{itemize}[noitemsep]
	      \item To keep South African teachers updated and informed.
	    \item To give teachers the opportunity to express their views and cultivate their interests.”
	    \end{itemize}
      \item \underline{http://www.teachersmonthly.com }
    \end{itemize}

\item Head Thoughts – Personal Reflections of a School Headmaster
    \begin{itemize}[noitemsep]
	\item blog by Arthur Preston
	\item “Arthur is currently the headmaster of a growing independent school in Worcester, in the Western Cape province of South Africa. His approach to primary education is progressive and is leading the school through an era of new development and change.”
\item \underline{http://headthoughts.co.za/ }
    \end{itemize}
\end{itemize}

\subsubsection{Maths blogs}
\begin{itemize}

\item CEO: Circumspect Education Officer - Educating The Future
\begin{itemize} [noitemsep]
 \item blog by Robyn Clark
\item “Mathematics teacher and inspirer.”
\item \underline{http://clarkformaths.tumblr.com/ }
\end{itemize}

\item dy/dan - Be less helpful
\begin{itemize} [noitemsep]
\item blog by Dan Meyer
\item “I'm Dan Meyer. I taught high school math between 2004 and 2010 and I am currently studying at Stanford University on a doctoral fellowship. My specific interests include curriculum design (answering the question, "how we design the ideal learning experience for students?") and teacher education (answering the questions, "how do teachers learn?" and "how do we retain more teachers?" and "how do we teach teachers to teach?").”
\item \underline{http://blog.mrmeyer.com }
\end{itemize}

\item Without Geometry, Life is Pointless - Musings on Math, Education, Teaching, and Research

\begin{itemize}[noitemsep]
 \item blog by Avery
\item “I've been teaching some permutation (or is that combination?) of math and science to third through twelfth graders in private and public schools for 11 years. I'm also pursuing my EdD in education and will be both teaching and conducting research in my classroom this year.”
\item \underline{ http://mathteacherorstudent.blogspot.com/ }
\end{itemize}

\item Overthinking my teaching - The Mathematics I Encounter in Classrooms
\begin{itemize}[noitemsep]
\item blog by Christopher Danielson
\item “I think a lot about my math teaching. Perhaps too much. This is my outlet. I hope you find it interesting and that you’ll let me know how it’s going.”
\item \underline{http://christopherdanielson.wordpress.com}
\end{itemize}

\item A Recursive Process - Math Teacher Seeking Patterns
\begin{itemize}[noitemsep]
\item blog by Dan
\item “I am a High School math teacher in upstate NY. I currently teach Geometry, Computer Programming (Alice and Java), and two half year courses: Applied and Consumer Math. This year brings a new 21st century classroom (still not entirely sure what that entails) and a change over to standards based grades (#sbg).”
\item \underline{http://dandersod.wordpress.com }
\end{itemize}

\item Think Thank Thunk – Dealing with the Fear of Being a Boring Teacher 
\begin{itemize} [noitemsep]
\item blog by Shawn Cornally
\item “I am Mr. Cornally. I desperately want to be a good teacher. I teach Physics, Calculus, Programming, Geology, and Bioethics. Warning: I have problem with using colons. I proof read, albeit poorly.”
\item \underline{http://101studiostreet.com/wordpress/}
\end{itemize}
\end{itemize}

\section{Overview}
\subsubsection{Curriculum overview}
Before 1994 there existed a number of education departments and subsequent curriculum according to the segregation that was so evident during the apartheid years. As a result, the curriculum itself became one of the political icons of freedom or suppression. Since then the government and political leaders have sought to try and develop one curriculum that is aligned with our national agenda of democratic freedom and equality for all, in fore-grounding the knowledge, skills and values our country believes our learners need to acquire and apply, in order to participate meaningfully in society as citizens of a free country. The National Curriculum Statement (NCS) of Grades R – 12 (DBE, 2012) therefore serves the purposes of: 
\begin{itemize}
\item equipping learners, irrespective of their socio-economic background, race, gender, physical ability or intellectual ability, with the knowledge, skills and values necessary for self-fulfilment, and meaningful participation in society as citizens of a free country; 
\item providing access to higher education; 
\item facilitating the transition of learners from education institutions to the workplace; and 
\item providing employers with a sufficient profile of a learner’s competencies. 
\end{itemize}
Although elevated to the status of political icon, the curriculum remains a tool that requires the skill of an educator in interpreting and operationalising this tool within the classroom. The curriculum itself cannot accomplish the purposes outlined above without the community of curriculum specialists, material developers, educators and assessors contributing to and supporting the process, of the intended curriculum becoming the implemented curriculum. A curriculum can succeed or fail, depending on its implementation, despite its intended principles or potential on paper. It is therefore important that stakeholders of the curriculum are familiar with and aligned to the following principles that the NCS is based on: 

\begin{table}[H]
\begin{center}
\begin{tabular}{|p{6.5cm}|p{6.5cm}|} \hline
\textbf{Principle} & \textbf{Implementation} \\ \hline
Social Transformation & Redressing imbalances of the past.\par Providing equal opportunities for all.\\ \hline
Active and Critical Learning & Encouraging an active and critical approach to learning. \par Avoiding excessive rote and uncritical learning of given truths.\\ \hline
High Knowledge and Skills & Learners achieve minimum standards of knowledge and skills specified for each grade in each subject. \\ \hline
Progression & Content and context shows progression from simple to complex. \\ \hline
Social and Environmental Justice and Human Rights & These practices as defined in the Constitution are infused into the teaching and learning of each of the subjects. \\ \hline
Valuing Indigenous Knowledge Systems & Acknowledging the rich history and heritage of this country. \\ \hline
Credibility, Quality and Efficiency & Providing an education that is globally comparable in quality. \\ \hline


\end{tabular}
\end{center}
\end{table}

This guide is intended to add value and insight to the existing National Curriculum for Grade 10 Mathematics, in line with its purposes and principles. It is hoped that this will assist you as the educator in optimising the implementation of the intended curriculum. 

\subsubsection{Curriculum requirements and objectives}
The main objectives of the curriculum relate to the learners that emerge from our educational system. While educators are the most important stakeholders in the implementation of the intended curriculum, the quality of learner coming through this curriculum will be evidence of the actual attained curriculum from what was intended and then implemented. \par

These purposes and principles aim to produce learners that are able to: 
\begin{itemize}[noitemsep]
\item identify and solve problems and make decisions using critical and creative thinking; 
\item work effectively as individuals and with others as members of a team; 
\item organise and manage themselves and their activities responsibly and effectively; 
\item collect, analyse, organise and critically evaluate information; 
\item communicate effectively using visual, symbolic and/or language skills in various modes; 
\item use science and technology effectively and critically showing responsibility towards the environment and the health of others; and 
\item demonstrate an understanding of the world as a set of related systems by recognising that problem solving contexts do not exist in isolation. 
\end{itemize}
The above points can be summarised as an independent learner who can think critically and analytically, while also being able to work effectively with members of a team and identify and solve problems through effective decision making. This is also the outcome of what educational research terms the “reformed” approach rather than the “traditional” approach many educators are more accustomed to. Traditional practices have their role and cannot be totally abandoned in favour of only reform practices. However, in order to produce more independent and mathematical thinkers, the reform ideology needs to be more embraced by educators within their instructional behaviour. Here is a table that can guide you to identify your dominant instructional practice and try to assist you in adjusting it (if necessary) to be more balanced and in line with the reform approach being suggested by the NCS. 

\begin{table}[H]
\begin{center}
 \begin{tabular}{|p{3.5cm}|p{8.5cm}|} \hline 
  
& \textbf{Traditional Versus Reform Practices} \\ \hline
Values & \textbf{Traditional}– values content, correctness of learners’ responses and mathematical validity of methods. \par
\textbf{Reform} – values finding patterns, making connections, communicating mathematically and problem-solving. \\ \hline
Teaching Methods &
\textbf{Traditional} – expository, transmission, lots of drill and practice, step by step mastery of algorithms.\par
\textbf{Reform} – hands-on guided discovery methods, exploration, modelling. High level reasoning processes are central. \\ \hline
Grouping Learners & \textbf{Traditional} – dominantly same grouping approaches. \par
\textbf{Reform} – dominantly mixed grouping and abilities. \\ \hline


 \end{tabular}
\end{center}
\end{table}

The subject of mathematics, by the nature of the discipline, provides ample opportunities to meet the reformed objectives. In doing so, the definition of mathematics needs to be understood and embraced by educators involved in the teaching and the learning of the subject. In research it has been well documented that, as educators, our conceptions of what mathematics is, has an influence on our approach to the teaching and learning of the subject. \par

Three possible views of mathematics can be presented. The instrumentalist view of mathematics assumes the stance that mathematics is an accumulation of facts, rules and skills that need to be used as a means to an end, without there necessarily being any relation between these components. The Platonist view of mathematics sees the subject as a static but unified body of certain knowledge, in which mathematics is discovered rather than created. The problem solving view of mathematics is a dynamic, continually expanding and evolving field of human creation and invention that is in itself a cultural product. Thus mathematics is viewed as a process of enquiry, not a finished product. The results remain constantly open to revision. It is suggested that a hierarchical order exists within these three views, placing the instrumentalist view at the lowest level and the problem solving view at the highest.
\subsubsection{According to the NCS:}
Mathematics is the study of quantity, structure, space and change. Mathematicians seek out patterns, formulate new conjectures, and establish axiomatic systems by rigorous deduction from appropriately chosen axioms and definitions. Mathematics is a distinctly human activity practised by all cultures, for thousands of years. Mathematical problem solving enables us to understand the world (physical, social and economic) around us, and, most of all, to teach us to think creatively.\par

This corresponds well to the problem solving view of mathematics and may challenge some of our instrumentalist or Platonistic views of mathematics as a static body of knowledge of accumulated facts, rules and skills to be learnt and applied. The NCS is trying to discourage such an approach and encourage mathematics educators to dynamically and creatively involve their learners as mathematicians engaged in a process of study, understanding, reasoning, problem solving and communicating mathematically. \par

Below is a check list that can guide you in actively designing your lessons in an attempt to embrace the definition of mathematics from the NCS and move towards a problem solving conception of the subject. Adopting such an approach to the teaching and learning of mathematics will in turn contribute to the intended curriculum being properly implemented and attained through the quality of learners coming out of the education system. \par

\begin{table}[H]
\begin{center}
 \begin{tabular}{|p{6.5cm}|p{6.5cm}|} \hline 
  \textbf{Practice} & \textbf{Example} \\ \hline
Learners engage in solving contextual problems related to their lives that require them to interpret a problem and then find a suitable mathematical solution. &
Learners are asked to work out which bus service is the cheapest given the fares they charge and the distance they want to travel. \\ \hline
Learners engage in solving problems of a purely mathematical nature, which require higher order thinking and application of knowledge (non-routine problems). &
Learners are required to draw a graph; they have not yet been given a specific technique on how to draw (for example a parabola), but have learnt to use the table method to draw straight-line graphs. \\ \hline
Learners are given opportunities to negotiate meaning. &
Learners discuss their understanding of concepts and strategies for solving problems with each other and the educator. \\ \hline
Learners are shown and required to represent situations in various but equivalent ways (mathematical modelling). &
Learners represent data using a graph, a table and a formula to represent the same data. \\ \hline
Learners individually do mathematical investigations in class, guided by the educator where necessary.  &
Each learner is given a paper containing the mathematical problem (for instance to find the number of prime numbers less than 50) that needs to be investigated and the solution needs to be written up. Learners work independently. \\ \hline
Learners work together as a group/team to investigate or solve a mathematical problem. &
A group is given the task of working together to solve a problem that requires them investigating patterns and working through data to make conjectures and find a formula for the pattern. \\ \hline
Learners do drill and practice exercises to consolidate the learning of concepts and to master various skills. &
Completing an exercise requiring routine procedures. \\ \hline
Learners are given opportunities to see the interrelatedness of the mathematics and to see how the different outcomes are related and connected.  &
While learners work through geometry problems, they are encouraged to make use of algebra. \\ \hline
Learners are required to pose problems for their educator and peer learners. &
Learners are asked to make up an algebraic word problem (for which they also know the solution) for the person sitting next to them to solve. \\ \hline


 \end{tabular}
\end{center}
\end{table}

\subsection{Overview of topics}
Summary of topics and their relevance:

\begin{table}[H]
\begin{center} 
\begin{tabular}{|p{8.5cm}|p{3.5cm}|} \hline
\textbf{1. Functions – linear, quadratic, exponential, rational} &\textbf{Relevance}  \\ \hline  

Relationships between variables in terms of graphical, verbal and symbolic representations of functions (tables, graphs, words and formulae). Generating graphs and generalising effects of parameters of vertical shifts and stretches and reflections about the x-axis.\par
Problem solving and graph work involving prescribed functions.& Functions form a core part of learners’ mathematical understanding and reasoning processes in algebra. This is also an excellent opportunity for contextual mathematical modelling questions. \\ \hline

 \end{tabular}
\end{center}
\end{table}

\begin{table}[H]
\begin{center} 
\begin{tabular}{|p{8.5cm}|p{3.5cm}|} \hline
\textbf{2. Number Patterns, Sequences and Series}&\textbf{Relevance} \\ \hline  

Number patterns with constant difference.& Much of mathematics revolves around the identification of patterns.
\\ \hline

 \end{tabular}
\end{center}
\end{table}

\begin{table}[H]
\begin{center} 
\begin{tabular}{|p{8.5cm}|p{3.5cm}|} \hline
\textbf{3. Finance, Growth and Decay}& \textbf{Relevance} \\ \hline  

Use simple and compound growth formulae. Implications of fluctuating exchange rates.& The mathematics of finance is very relevant to daily and long-term financial decisions learners will need to make in terms of investing, taking loans, saving and understanding exchange rates and their influence more globally.
\\ \hline

 \end{tabular}
\end{center}
\end{table}

\begin{table}[H]
\begin{center} 
\begin{tabular}{|p{8.5cm}|p{3.5cm}|} \hline
\textbf{4. Algebra}&\textbf{Relevance}  \\ \hline  

Understand that real numbers can be irrational or rational. \par
Simplify expressions using the laws of exponents for rational exponents.\par
Identifying and converting forms of rational numbers.\par
Working with simple surds that are not rational.\par
Working with laws of integral exponents.\par
Establish between which two integers a simple surd lies.\par
Appropriately rounding off real numbers.\par
Manipulating and simplifying algebraic expressions (including multiplication and factorisation).
Solving linear, quadratic, literal and exponential equations.\par
Solving linear inequalities in one and two variables algebraically and graphically.& Algebra provides the basis for mathematics learners to move from numerical calculations to generalising operations, simplifying expressions, solving equations and using graphs and inequalities in solving contextual problems.
\\ \hline

 \end{tabular}
\end{center}
\end{table}

\begin{table}[H]
 \begin{center} 
\begin{tabular}{|p{8.5cm}|p{3.5cm}|} \hline
\textbf{5. Differential Calculus}& \textbf{Relevance}\\ \hline  

Investigate average rate of change between two independent values of a function.& The central aspect of rate of change to differential calculus is a basis to further understanding of limits, gradients and calculations and formulae necessary for work in engineering fields, e.g. designing roads, bridges etc. 
\\ \hline

 \end{tabular}
\end{center}
\end{table}

\begin{table}[H]
 \begin{center} 
\begin{tabular}{|p{8.5cm}|p{3.5cm}|} \hline
\textbf{6. Probability}& \textbf{Relevance} \\ \hline  

Compare relative frequency and theoretical probability.\par
Use Venn diagrams to solve probability problems.\par
Mutually exclusive and complementary events.\par
Identity for any two events A and B. \par
& This topic is helpful in developing good logical reasoning in learners and for educating them in terms of real-life issues such as gambling and the possible pitfalls thereof. 
\\ \hline

 \end{tabular}
\end{center}
\end{table}


\begin{table}[H]
 \begin{center} 
\begin{tabular}{|p{8.5cm}|p{3.5cm}|} \hline
\textbf{7. Euclidean Geometry and Measurement}& \textbf{Relevance}\\ \hline  

Investigate, form and try to prove conjectures about properties of special triangles, quadrilaterals and other polygons.\par
Disprove false conjectures using counter-examples.\par
Investigate alternative definitions of various polygons.\par
Solve problems involving surface area and volumes of solids and combinations thereof.
& The thinking processes and mathematical skills of proving conjectures and identifying false conjectures is more the relevance here than the actual content studied. The surface area and volume content studied in real-life contexts of designing kitchens, tiling and painting rooms, designing packages, etc. is relevant to the current and future lives of learners. 
\\ \hline

 \end{tabular}
\end{center}
\end{table}

\begin{table}[H]
 \begin{center} 
\begin{tabular}{|p{8.5cm}|p{3.5cm}|} \hline
\textbf{8. Trigonometry}& \textbf{Relevance}\\ \hline  

Definitions of trig functions.\par
Derive values for special angles.\par
Take note of names for reciprocal functions.\par
Solve problems in 2 dimensions.\par
Extend definition of basic trig functions to all four quadrants and know graphs of these functions.\par
Investigate and know the effects of a and q on the graphs of basic trig functions.\par
Solve problems involving surface area and volumes of solids and combinations thereof.
& Trigonometry has several uses within society, including within navigation, music, geographical locations and building design and construction.
\\ \hline

 \end{tabular}
\end{center}
\end{table}

\begin{table}[H]
 \begin{center} 
\begin{tabular}{|p{8.5cm}|p{3.5cm}|} \hline
\textbf{9. Analytical Geometry}&  \textbf{Relevance} \\ \hline  

Represent geometric figures on a Cartesian coordinate system.\par
For any two points, derive and apply formula for calculating distance, gradient of line segment and coordinates of mid-point.
& This section provides a further application point for learners’ algebraic and trigonometric interaction with the Cartesian plane. Artists and design and layout industries often draw on the content and thought processes of this mathematical topic.
\\ \hline

 \end{tabular}
\end{center}
\end{table}

\begin{table}[H]
 \begin{center} 
\begin{tabular}{|p{8.5cm}|p{3.5cm}|} \hline
\textbf{10. Statistics}& \textbf{Relevance}\\ \hline  

Collect, organise and interpret univariate numerical data to determine mean, median, mode, percentiles, quartiles, deciles, interquartile and semi-interquartile range.\par
Identify possible sources of bias and errors in measurements.
& Citizens are daily confronted with interpreting data presented from the media. Often this data may be biased or misrepresented within a certain context. In any type of research, data collection and handling is a core feature. This topic also educates learners to become more socially and politically educated with regards to the media.
\\ \hline

 \end{tabular}
\end{center}
\end{table}

Mathematics educators also need to ensure that the following important specific aims and general principles are applied in mathematics activities across all grades:
\begin{itemize}[noitemsep]
\item Calculators should only be used to perform standard numerical computations and verify calculations done by hand.
\item Real-life problems should be incorporated into all sections to keep mathematical modelling as an important focal point of the curriculum.
\item Investigations give learners the opportunity to develop their ability to be more methodical, to generalise and to make and justify and/or prove conjectures.
\item Appropriate approximation and rounding skills should be taught and continuously included and encouraged in activities.
\item The history of mathematics should be incorporated into projects and tasks where possible, to illustrate the human aspect and developing nature of mathematics. 
\item Contextual problems should include issues relating to health, social, economic, cultural, scientific, political and environmental issues where possible. 
\item Conceptual understanding of when and why should also feature in problem types.
\item Mixed ability teaching requires educators to challenge able learners and provide remedial support where necessary. 
\item Misconceptions exposed by assessment need to be dealt with and rectified by questions designed by educators. 
\item Problem solving and cognitive development should be central to all mathematics teaching and learning so that learners can apply the knowledge effectively. 
\end{itemize}

\subsubsection{Allocation of teaching time}
Time allocation for Mathematics per week: 4 hours and 30 minutes e.g. six forty-five minute periods per week.
\begin{table}[H]
 \begin{center} 
\begin{tabular}{|p{2cm}|p{6cm}|p{2cm}|} \hline
\textbf{Term}& \textbf{Topic} & \textbf{No. of weeks} \\ \hline  
\textbf{Term 1} & Algebraic expressions \par
Exponents\par
Number patterns \par
Equations and inequalities\par
Trigonometry

&
3\par
2\par
1\par
2\par
3 \\ \hline
\textbf{Term 2} & Functions \par
Trigonometric functions  \par
Euclidean Geometry  \par
MID-YEAR EXAMS &
4  \par
1 \par
3  \par
3  \\ \hline

\textbf{Term 3} & Analytical Geometry \par
Finance and growth \par
Statistics \par
Trigonometry \par
Euclidean Geometry\par
Measurement &
2 \par
2\par
2\par
2\par
1\par
1 \\ \hline
\textbf{Term 4} & Probability \par Revision \par EXAMS &
2 \par 
4 \par 
3 \\ \hline


 \end{tabular}
\end{center}
\end{table}

Please see page 18 of the Curriculum and Assessment Policy Statement for the sequencing and pacing of topics.

\section{Assessment}
“Educator assessment is part of everyday teaching and learning in the classroom. Educators discuss with learners, guide their work, ask and answer questions, observe, help, encourage and challenge. In addition, they mark and review written and other kinds of work. Through these activities they are continually finding out about their learners’ capabilities and achievements. This knowledge then informs plans for future work. It is this continuous process that makes up educator assessment. It should not be seen as a separate activity necessarily requiring the use of extra tasks or tests.”  \par 

As the quote above suggests, assessment should be incorporated as part of the classroom practice, rather than as a separate activity. Research during the past ten years indicates that learners get a sense of what they do and do not know, what they might do about this and how they feel about it, from frequent and regular classroom assessment and educator feedback. The educator’s perceptions of and approach to assessment (both formal and informal assessment) can have an influence on the classroom culture that is created with regard to the learners’ expectations of and performance in assessment tasks. Literature on classroom assessment distinguishes between two different purposes of assessment; assessment of learning and assessment for learning. \par 

Assessment of learning tends to be a more formal assessment and assesses how much learners have learnt or understood at a particular point in the annual teaching plan. The NCS provides comprehensive guidelines on the types of and amount of formal assessment that needs to take place within the teaching year to make up the school-based assessment mark. The school-based assessment mark contributes 25\% of the final percentage of a learner’s promotion mark, while the end-of-year examination constitutes the other 75\% of the annual promotion mark. Learners are expected to have 7 formal assessment tasks for their school-based assessment mark. The number of tasks and their weighting in the Grade 10 Mathematics curriculum is summarised below: 
\begin{table}[H]
\begin{center}
\begin{tabular} {|p{4.5cm}|p{1.5cm}|p{3cm}|p{2cm}|} \hline
	  & 			& \textbf{Tasks} 			& \textbf{Weight (\%)} \\ \hline
School-Based Assessment & Term 1 & Test \par Project/Investigation 	& 10 \par 20 \\ \hline
			& Term 2 & Assignment/Test \par Examination 	& 10 \par 30 \\ \hline
			& Term 3 & Test \par Test 			& 10 \par 10 \\ \hline
			& Term 4 & Test 				&  10 \\ \hline
School-Based Assessment Mark&    & 					& 100 \\ \hline
School-Based Assessment Mark  \par
(as a \% of Promotion Mark)
			&	 & 					&  25 \% \\ \hline

End-of-Year Examination & 	& 					&75 \% \\ \hline
Promotion Mark 		&       & 					& 100 \% \\ \hline


\end{tabular}
 \end{center}
\end{table}

The following provides a brief explanation of each of the assessment tasks included in the assessment programme above. 
\subsubsection{Tests}
All mathematics educators are familiar with this form of formal assessment. Tests include a variety of items/questions covering the topics that have been taught prior to the test. The new NCS also stipulates that mathematics tests should include questions that cover the following four types of cognitive levels in the stipulated weightings: 
\begin{table}[H]
\begin{center}
\begin{tabular} {|p{3cm}|p{7cm}|p{1.5cm}|} \hline
\textbf{Cognitive Levels} & \textbf{Description} & \textbf{Weighting (\%)} \\ \hline
Knowledge & 
Estimation and appropriate rounding of numbers. \par 
Proofs of prescribed theorems.\par 
Derivation of formulae.\par 
Straight recall.\par 
Identification and direct use of formula on information sheet (no changing of the subject).
Use of mathematical facts.\par 
Appropriate use of mathematical vocabulary. & 
20 \\ \hline

Routine Procedures & 
Perform well known procedures.\par 
Simple applications and calculations.\par 
Derivation from given information.\par 
Identification and use (including changing the subject) of correct formula.\par 
Questions generally similar to those done in class. &
45 \\ \hline
Complex Procedures &
Problems involve complex calculations and/or higher reasoning. \par 
There is often not an obvious route to the solution.\par 
Problems need not be based on real world context.\par 
Could involve making significant connections between different representations.\par 
Require conceptual understanding. &
25 \\ \hline
Problem Solving & 
Unseen, non-routine problems (which are not necessarily difficult). \par 
Higher order understanding and processes are often involved.\par 
Might require the ability to break the problem down into its constituent parts.&
10 \\ \hline

\end{tabular}
 \end{center}
\end{table}

The breakdown of the tests over the four terms is summarised from the NCS assessment programme as follows: \\
\textbf{Term 1}:	One test of at least 50 marks, and one hour or two/three tests of at least 40 minutes each.\\
\textbf{Term 2}:	Either one test (of at least 50 marks) or an assignment.\\
\textbf{Term 3}:	Two tests, each of at least 50 marks and one hour.\\
\textbf{Term 4}:	One test of at least 50 marks.\\

\subsubsection{Projects/Investigations}

Investigations and projects consist of open-ended questions that initiate and expand thought processes. Acquiring and developing problem-solving skills are an essential part of doing investigations and projects. These tasks provide learners with the opportunity to investigate, gather information, tabulate results, make conjectures and justify or prove these conjectures.  Examples of investigations and projects and possible marking rubrics are provided in the next section on assessment support. The NCS assessment programme indicates that only one project or investigation (of at least 50 marks) should be included per year. Although the project/investigation is scheduled in the assessment programme for the first term, it could also be done in the second term. 
\subsubsection{Assignments}
The NCS includes the following tasks as good examples of assignments: 
\begin{itemize}[noitemsep]
\item Open book test
\item Translation task
\item Error spotting and correction
\item Shorter investigation
\item Journal entry
\item Mind-map (also known as a metacog)
\item Olympiad (first round)
\item Mathematics tutorial on an entire topic
\item Mathematics tutorial on more complex/problem solving questions
\end{itemize}
The NCS assessment programme requires one assignment in term 2 (of at least 50 marks) which could also be a combination of some of the suggested examples above. More information on these suggested examples of assignments and possible rubrics are provided in the following section on assessment support. 
\subsubsection{Examinations}
Educators are also all familiar with this summative form of assessment that is usually completed twice a year: mid-year examinations and end-of-year examinations. These are similar to the tests but cover a wider range of topics completed prior to each examination. The NCS stipulates that each examination should also cover the four cognitive levels according to their recommended weightings as summarised in the section above on tests. The following table summarises the requirements and information from the NCS for the two examinations.

\begin{table}[H]
\begin{center}
\begin{tabular} {|p{2cm}|p{1.5cm}|p{3cm}|p{4.5cm}|} \hline
\textbf{Examination}	 	 & \textbf{Marks}			& \textbf{Breakdown} 			& \textbf{Content and Mark Distribution} \\ \hline
Mid-Year Exam & 100 \par 50 + 50 & One paper: 2 hours\par \textbf{or} \par
				    Two papers: each of 1 hour  	& Topics completed\\ \hline
End-of-Year Exam		& 100 + & Paper 1: 2 hours	& Number patterns (±10) \par 
Algebraic expressions, equations and inequalities (±25)\par 
Functions (±35)\par 
Exponents (±10)\par 
Finance (±10)\par 
Probability (±10)
 \\ \hline
			&100  & Paper 2: 2 hours	& Trigonometry (±45) \par 
Analytical geometry (±15)\par 
Euclidean geometry and measurement (±25)\par 
Statistics (±15)
 \\ \hline

\end{tabular}
 \end{center}
\end{table}

In the annual teaching plan summary of the NCS in Mathematics for Grade 10, the pace setter section provides a detailed model of the suggested topics to be covered each week of each term and the accompanying formal assessment.

Assessment \textbf{for} learning tends to be more informal and focuses on using assessment in and of daily classroom activities that can include: 
\begin{itemize}[noitemsep]
\item Marking homework
\item Baseline assessments
\item Diagnostic assessments
\item Group work
\item Class discussions
\item Oral presentations
\item Self-assessment
\item Peer-assessment
\end{itemize}
These activities are expanded on in the next section on assessment support and suggested marking rubrics are provided. Where formal assessment tends to restrict the learner to written assessment tasks, the informal assessment is necessary to evaluate and encourage the progress of the learners in their verbal mathematical reasoning and communication skills. It also provides a less formal assessment environment that allows learners to openly and honestly assess themselves and each other, taking responsibility for their own learning, without the heavy weighting of the performance (or mark) component. The assessment for learning tasks should be included in the classroom activities at least once a week (as part of a lesson) to ensure that the educator is able to continuously evaluate the learners’ understanding of the topics covered as well as the effectiveness, and identify any possible deficiencies in his or her own teaching of the topics. 

\subsection{Assessment support}
A selection of explanations, examples and suggested marking rubrics for the assessment of learning (formal) and the assessment for learning (informal) forms of assessment discussed in the preceding section are provided in this section. 
\subsubsection{Baseline assessment}
Baseline assessment is a means of establishing:
\begin{itemize}[noitemsep]
\item What prior knowledge a learner possesses 
\item What the extent of knowledge is that they have regarding a specific learning area
\item The level they demonstrate regarding various skills and applications
\item The learner’s level of understanding of various learning areas
\end{itemize}
It is helpful to educators in order to assist them in taking learners from their individual point of departure to a more advanced level and to thus make progress. This also helps avoid large "gaps” developing in the learners’ knowledge as the learner moves through the education system. Outcomes-based education is a more learner-centered approach than we are used to in South Africa, and therefore the emphasis should now be on the level of each individual learner rather than that of the whole class. \par 

The baseline assessments also act as a gauge to enable learners to take more responsibility for their own learning and to view their own progress. In the traditional assessment system, the weaker learners often drop from a 40\% average in the first term to a 30\% average in the fourth term due to an increase in workload, thus demonstrating no obvious progress. Baseline assessment, however, allows for an initial assigning of levels which can be improved upon as the learner progresses through a section of work and shows greater knowledge, understanding and skill in that area.

\subsubsection{Diagnostic assessments}
These are used to specifically find out if any learning difficulties or problems exist within a section of work in order to provide the learner with appropriate additional help and guidance. The assessment helps the educator and the learner identify problem areas, misunderstandings, misconceptions and incorrect use and interpretation of notation. \par

Some points to keep in mind:
\begin{itemize}[noitemsep]
\item Try not to test too many concepts within one diagnostic assessment.
\item Be selective in the type of questions you choose. 
\item Diagnostic assessments need to be designed with a certain structure in mind. As an educator, you should decide exactly what outcomes you will be assessing and structure the content of the assessment accordingly. 
\item The assessment is marked differently to other tests in that the mark is not the focus but rather the type of mistakes the learner has made.
\end{itemize}
An example of an understanding rubric for educators to record results is provided below:\\

0: indicates that the learner has not grasped the concept at all and that there appears to be a fundamental mathematical problem.\\
1: indicates that the learner has gained some idea of the content, but is not demonstrating an understanding of the notation and concept.\\
2: indicates evidence of some understanding by the learner but further consolidation is still required.\\
3:indicates clear evidence that the learner has understood the concept and is using the notation correctly.\par

\subsubsection{Calculator worksheet - diagnostic skills assessment}
\begin{enumerate}[itemsep=7pt, label=\textbf{\arabic*}. ] 
 \item Calculate:
\begin{enumerate}[itemsep=6pt,label=\textbf{(\alph*)}]
\item $ 242 + 63=$   ~~~\underline{~~~~~~~~~~~~~}
\item $2-36 \times (114 + 25)=$~~~\underline{~~~~~~~~~~~~~}
\item $\sqrt{144+25}=$~~~\underline{~~~~~~~~~~~~~}
\item $\sqrt[4]{729}=$~~~\underline{~~~~~~~~~~~~~}
\item $-312 + 6 + 879 -321 + 18~ 901=$ ~~~\underline{~~~~~~~~~~~~~}
\end{enumerate}

\item Calculate:
\begin{enumerate}[itemsep=6pt,label=\textbf{(\alph*)}]
\item $\frac{2}{7} + \frac{1}{3}=$  ~~~\underline{~~~~~~~~~~~~~}
\item $2\frac{1}{5} - \frac{2}{9}=$ ~~~\underline{~~~~~~~~~~~~~}
\item $-2\frac{5}{6} + \frac{3}{8}=$ ~~~\underline{~~~~~~~~~~~~~}
\item $ 4 - \frac{3}{4} \times \frac{5}{7}=$ ~~~\underline{~~~~~~~~~~~~~}
\item $\left(\frac{9}{10} - \frac{8}{9}\right) \div \frac{3}{5}=$ ~~~\underline{~~~~~~~~~~~~~}
\item $2\times \left(\frac{4}{5}\right)^2 - \left(\frac{19}{25}\right)=$ ~~~\underline{~~~~~~~~~~~~~}
\item $\sqrt{\frac{9}{4} - \frac{4}{16}} =$ ~~~\underline{~~~~~~~~~~~~~}
\end{enumerate}
\end{enumerate}

Self-Assesment Rubric: \\
Name: \underline{~~~~~~~~~~~~~~~~~~~~~~~~~~~~~~}
\begin{table}[H]
 \begin{center}
  \begin{tabular}{|p{1.5cm}|p{1.5cm}|p{1cm}|p{1cm}|p{6cm}|} \hline

\textbf{Question} & \textbf{Answer} & \textbf{√} & \textbf{X} & \textbf{If X, write down sequence of keys pressed} \\ \hline
1a) &&&&\\ \hline
1b)&&&&\\ \hline
1c)&&&&\\ \hline
1d)&&&&\\ \hline
1e)&&&&\\ \hline
\textbf{Subtotal}&&&&\\ \hline
2a)&&&&\\ \hline
2b)&&&&\\ \hline
2c)&&&&\\ \hline
2d)&&&&\\ \hline
2e)&&&&\\ \hline
2f)&&&&\\ \hline
2g)&&&&\\ \hline
\textbf{Subtotal}&&&&\\ \hline
\textbf{Total}&&&& \\ \hline


   
  \end{tabular}

 \end{center}

\end{table}

Educator Assessment Rubric:
\begin{table}[H]
 \begin{center}
  \begin{tabular}{|p{4.5cm}|p{1.5cm}|p{3cm}|p{1.5cm}|} \hline

\textbf{Type of Skill} & \textbf{Competent} & \textbf{Needs Practice} & \textbf{Problem}   \\ \hline
Raising to a Power &&&\\ \hline
Finding a Root&&&\\ \hline
Calculations with Fractions&&&\\ \hline
Brackets and Order of Operations&&&\\ \hline
Estimation and Mental Control&&&\\ \hline
   
  \end{tabular}

 \end{center}

\end{table}
Guidelines for Calculator Skills Assessment:
\begin{table}[H]
 \begin{center}
  \begin{tabular}{|p{5cm}|p{4cm}|p{3cm}|} \hline

\textbf{Type of Skill} & \textbf{Sub-Division} & \textbf{Questions}   \\ \hline
Raising to a Power & Squareing and cubing\par Higher order powers&1a, 2f \par1b \\ \hline
Finding a Root&Square and cube roots \par Higher order roots & 1c, 2g \par 1d\\ \hline
Calculations with Fractions&Basic operations \par Mixed numbers \par Negative numbers \par Squaring fractions \par Square rooting fractions&2a,  2d\par
2b, 2c\par
1e, 2c\par
2f\par
2g
\\ \hline
Brackets and Order of Operations&Correct use of brackets or order of operations&1b, 1c, 2e, 2f, 2g\\ \hline
Estimation and Mental Control&Overall&All\\ \hline
   
  \end{tabular}

 \end{center}

\end{table}

\subsubsection{Suggested guideline to allocation of overall levels}
\textbf{Level 1}
\begin{itemize}[noitemsep]
\item Learner is able to do basic operations on calculator.
\item Learner is able to do simple calculations involving fractions.
\item Learner does not display sufficient mental estimation and control techniques.
\end{itemize}
\textbf{Level 2}\begin{itemize}[noitemsep]
\item Learner is able to do basic operations on calculator.
\item Learner is able to square and cube whole numbers as well as find square and cube roots of numbers.
\item Learner is able to do simple calculations involving fractions as well as correctly execute calculations involving mixed numbers.
\item Learner displays some degree of mental estimation awareness.
\end{itemize}
\textbf{Level 3}\begin{itemize}[noitemsep]
\item Learner is able to do basic operations on calculator.
\item Learner is able to square and cube rational numbers as well as find square and cube roots of numbers.
\item Learner is also able to calculate higher order powers and roots.
\item Learner is able to do simple calculations involving fractions as well as correctly execute calculations involving mixed numbers.
\item Learner works correctly with negative numbers.
\item Learner is able to use brackets in certain calculations but has still not fully understood the order of operations that the calculator has been programmed to execute, hence the need for brackets.
\item Learner is able to identify possible errors and problems in their calculations but needs assistance solving the problem.
\end{itemize}
\textbf{Level 4}\begin{itemize}[noitemsep]
\item Learner is able to do basic operations on calculator.
\item Learner is able to square and cube rational numbers as well as find square and cube roots.
\item Learner is also able to calculate higher order powers and roots.
\item Learner is able to do simple calculations involving fractions as well as correctly execute calculations involving mixed numbers.
\item Learner works correctly with negative numbers.
\item Learner is able to work with brackets correctly and understands the need and use of brackets and the “= key” in certain calculations due to the nature of a scientific calculator.
\item Learner is able to identify possible errors and problems in their calculations and to find solutions to these in order to arrive at a “more viable” answer.
\end{itemize}

\subsubsection{Other short diagnostic tests }
These are short tests that assess small quantities of recall knowledge and application ability on a day-to-day basis. Such tests could include questions on one or a combination of the following:
\begin{itemize}[noitemsep]
\item Definitions
\item Theorems
\item Riders (geometry)
\item Formulae
\item Applications
\item Combination questions
\end{itemize}
Here is a selection of model questions that can be used at Grade 10 level to make up short diagnostic tests. They can be marked according to a memorandum drawn up by the educator. \par
\textbf{Geometry}
\begin{enumerate}[itemsep=0pt, label=\textbf{\arabic*}. ] 
\item Points $A(-5 ; -3)$, $B (-1 ; 2)$ and $C (9 ; -6)$ are the vertices of $∆ABC$. 
\begin{enumerate}[itemsep=0pt,label=\textbf{(\alph*)}]
\item Calculate the gradients of $AB$ and $BC$ and hence show that angle $ABC$ is 
equal to $90^{\circ}$.				
\begin{flushright}(5)\end{flushright}
\item State the distance formula and use it to calculate the lengths of the sides 
$AB$, $BC$ and $AC$ of $∆ABC$. (Leave your answers in surd form). 
\begin{flushright}(5)\end{flushright}
\end{enumerate}
\end{enumerate}

\textbf{Algebra}
\begin{enumerate}[itemsep=0pt, label=\textbf{\arabic*}. ] 
\item Write down the formal definition of an exponent as well as the exponent laws for integral exponents.\begin{flushright}(6)\end{flushright}

\item Simplify: $\dfrac{2x^4y^8z^3}{4xy} \times \dfrac{x^7}{y^3z^0}$	\begin{flushright}(4)\end{flushright}
\end{enumerate}

\textbf{Trigonometry}
\begin{enumerate}[itemsep=0pt, label=\textbf{\arabic*}. ] 
\item A jet leaves an airport and travels $578$ km in a direction of $50^{\circ}$ E of N. The pilot then changes direction and travels $321$ km $10^{\circ}$ W of N.
\begin{enumerate}[itemsep=0pt,label=\textbf{(\alph*)}]
\item  How far away from the airport is the jet? (To the nearest kilometre) \begin{flushright}(5)\end{flushright}
\item Determine the jet’s bearing from the airport.\begin{flushright}(5)\end{flushright}
\end{enumerate}
\end{enumerate}

\subsubsection{Exercises}
This entails any work from the textbook or other source that is given to the learner, by the educator, to complete either in class or at home. Educators should encourage learners not to copy each other’s work and be vigilant when controlling this work. It is suggested that such work be marked/controlled by a check list (below) to speed up the process for the educator. \par

The marks obtained by the learner for a specific piece of work need not be based on correct and/or incorrect answers but preferably on the following:
\begin{itemize}[noitemsep]
\item the effort of the learner to produce answers.
\item the quality of the corrections of work that was previously incorrect.
\item the ability of the learner to explain the content of some selected examples (whether in writing or orally).
\end{itemize}
The following rubric can be used to assess exercises done in class or as homework: 

\begin{table}[H]
 \begin{center}
  \begin{tabular}{|p{3cm}|p{3cm}|p{3cm}|p{3cm}|} \hline
   \textbf{Criteria} & \textbf{Performance Indicators} &&\\ \hline
Work Done & 2 \par All the work & 1 \par Partially completed & 0 \par No work done \\ \hline
Work Neatly Done & 2 \par Work neatly done & 1 \par Some work not neatly done & 0 \par Messy and muddled\\ \hline
Corrections Done & 2 \par All corrections done consistently & 1 \par At least half of the corrections done & 0 \par No corrections done \\ \hline
Correct Mathematical Method & 2 \par Consistently & 1 \par Sometimes & 0 \par Never \\ \hline
Understanding of Mathematical Techniques and Processes & 2 \par Can explain concepts and processes precisely & 1 \par Explanations are ambiguous or not focused & 0 \par Explanations are confusing or irrelevant \\ \hline
  \end{tabular}

 \end{center}

\end{table}

\subsubsection{Journal entries}
A journal entry is an attempt by a learner to express in the written word what is happening in Mathematics. It is important to be able to articulate a mathematical problem, and its solution in the written word. \par

This can be done in a number of different ways:
\begin{itemize}[noitemsep]
\item Today in Maths we learnt \underline{~~~~~~~~~~~~~~~~~~~~~~~~~~~~~~~} 
\item Write a letter to a friend, who has been sick, explaining what was done in class today.
\item Explain the thought process behind trying to solve a particular maths problem, e.g. sketch the graph of $ y = x^2 - 2x^2 + 1$ and explain how to sketch such a graph.
\item Give a solution to a problem, decide whether it is correct and if not, explain the possible difficulties experienced by the person who wrote the incorrect solution. 
\end{itemize}
A journal is an invaluable tool that enables the educator to identify any mathematical misconceptions of the learners. The marking of this kind of exercise can be seen as subjective but a marking rubric can simplify the task. 

The following rubric can be used to mark journal entries. The learners must be given the marking rubric before the task is done. 
\begin{table}[H]
 \begin{center}
  \begin{tabular}{|p{4cm}|p{2cm}|p{2.5cm}|p{3cm}|} \hline
  \textbf{Task} & \textbf{Competent \newline(2 Marks)} & \textbf{Still Developing \newline(1 Mark)}& \textbf{Not Yet Developed\newline (1 Mark)}\\ \hline
Completion in Time Limit? &&&\\ \hline
Correctness of the Explanation? &&&\\ \hline
Correct and Relevant use of Mathematical Language? &&&\\ \hline
Is the Mathematics Correct? &&&\\ \hline
Has the Concept Been Interpreted Correctly?&&&\\ \hline

  \end{tabular}

 \end{center}

\end{table}


\subsubsection{Translations}
Translations assess the learner’s ability to translate from words into mathematical notation or to give an explanation of mathematical concepts in words. Often when learners can use mathematical language and notation correctly, they demonstrate a greater understanding of the concepts. \par 

For example: \\
Write the letter of the correct expression next to the matching number:\\
\begin{table}[H]
 \begin{center}
  \begin{tabular}{lrl} 
$x$ increased by $10$&					a)	&$xy$ \\
The product of  $x$ and $ y$		 &		 b)	&$x^2$\\
The sum of a certain number and 	&		c)&	$x^2$\\
double that number&					d)&	$29x$	\\
Half of a certain number multiplied by itself	&	e)&	$\frac{1}{2} \times 2$\\
Two less than  $x$&					f)&	$x + x + 2  $\\
A certain number multiplied by itself	&		g)&	$x^ 2$\\
  \end{tabular}
 \end{center}
\end{table}

\subsubsection{Group work}
One of the principles in the NCS is to produce learners who are able to work effectively within a group. Learners generally find this difficult to do. Learners need to be encouraged to work within small groups. Very often it is while learning under peer assistance that a better understanding of concepts and processes is reached. Clever learners usually battle with this sort of task, and yet it is important that they learn how to assist and communicate effectively with other learners. \par

\subsubsection{Mind maps or metacogs}
A metacog or “mind map” is a useful tool. It helps to associate ideas and make connections that would otherwise be too unrelated to be linked. A metacog can be used at the beginning or end of a section of work in order to give learners an overall perspective of the work covered, or as a way of recalling a section already completed. It must be emphasised that it is not a summary. Whichever way you use it, it is a way in which a learner is given the opportunity of doing research in a particular field and can show that he/she has an understanding of the required section. \par 

This is an open book form of assessment and learners may use any material they feel will assist them. It is suggested that this activity be practised, using other topics, before a test metacog is submitted for portfolio assessment purposes. \par

On completion of the metacog, learners must be able to answer insightful questions on the metacog. This is what sets it apart from being just a summary of a section of work. Learners must refer to their metacog when answering the questions, but may not refer to any reference material. Below are some guidelines to give to learners to adhere to when constructing a metacog as well as two examples to help you get learners started. A marking rubric is also provided. This should be made available to learners before they start constructing their metacogs. On the next page is a model question for a metacog, accompanied by some sample questions that can be asked within the context of doing a metacog about analytical geometry. \par

A basic metacog is drawn in the following way:
\begin{itemize}[noitemsep]
\item Write the title/topic of the subject in the centre of the page and draw a circle around it.
\item For the first main heading of the subject, draw a line out from the circle in any direction, and write the heading above or below the line.
\item For sub-headings of the main heading, draw lines out from the first line for each subheading and label each one. 
\item For individual facts, draw lines out from the appropriate heading line. 
\end{itemize}
Metacogs are one’s own property. Once a person understands how to assemble the basic structure they can develop their own coding and conventions to take things further, for example to show linkages between facts. The following suggestions may assist educators and learners to enhance the effectiveness of their metacogs:
\begin{itemize}[noitemsep]
\item Use single words or simple phrases for information. Excess words just clutter the metacog and take extra time to write down.
\item Print words – joined up or indistinct writing can be more difficult to read and less attractive to look at. 
\item Use colour to separate different ideas – this will help your mind separate ideas where it is necessary, and helps visualisation of the metacog for easy recall. Colour also helps to show organisation.
\item Use symbols and images where applicable. If a symbol means something to you, and conveys more information than words, use it. Pictures also help you to remember information.
\item Use shapes, circles and boundaries to connect information – these are additional tools to help show the grouping of information.
\end{itemize}
Use the concept of analytical geometry as your topic and construct a mind map (or metacog) containing all the information (including terminology, definitions, formulae and examples) that you know about the topic of analytical geometry. \par 

Possible questions to ask the learner on completion of their metacog: 
\begin{itemize}
\item Briefly explain to me what the mathematics topic of analytical geometry entails.
\item Identify and explain the distance formula, the derivation and use thereof for me on your metacog.
\item How does the calculation of gradient in analytical geometry differ (or not) from the approach used to calculate gradient in working with functions? 
\end{itemize}
A suggested simple rubric for marking a metacog:
\begin{table}[H]
 \begin{center}
  \begin{tabular}{|p{3cm}|p{2.5cm}|p{2.5cm}|p{3cm}|} \hline
  \textbf{Task} & \textbf{Competent \newline(2 Marks)} & \textbf{Still Developing \newline(1 Mark)}& \textbf{Not Yet Developed \newline 1 Mark)}\\ \hline
Completion in Time Limit &&&\\ \hline
Main Headings&&&\\ \hline
Correct Theory (Formulae, Definitions, Terminology etc.) &&&\\ \hline
Explanation &&&\\ \hline
“Readability”&&&\\ \hline

  \end{tabular}

 \end{center}

\end{table}

10 marks for the questions, which are marked using the following scale:\\
0	-	no attempt or a totally incorrect attempt has been made \\
1	-	a correct attempt was made, but the learner did not get the correct answer \\
2	-	a correct attempt was made and the answer is correct\\

\subsubsection{Investigations}
Investigations consist of open-ended questions that initiate and expand thought processes. Acquiring and developing problem-solving skills are an essential part of doing investigations. \par 

It is suggested that 2 – 3 hours be allowed for this task.  During the first 30 – 45 minutes learners could be encouraged to talk about the problem, clarify points of confusion, and discuss initial conjectures with others. The final written-up version should be done individually though and should be approximately four pages. \par 


Assessing investigations may include feedback/ presentations from groups or individuals on the results keeping the following in mind:
\begin{itemize}[noitemsep]
\item following of a logical sequence in solving the problems
\item pre-knowledge required to solve the problem
\item correct usage of mathematical language and notation
\item purposefulness of solution
\item quality of the written and oral presentation 
\end{itemize}
Some examples of suggested marking rubrics are included on the next few pages, followed by a selection of topics for possible investigations. \par

The following guidelines should be provided to learners before they begin an investigation: \par

\textbf{General Instructions Provided to Learners}
\begin{itemize}[noitemsep]
\item You may choose any one of the projects/investigations given (see model question on investigations)
\item You should follow the instructions that accompany each task as these describe the way in which the final product must be presented.
\item You may discuss the problem in groups to clarify issues, but each individual must write-up their own version.
\item Copying from fellow learners will cause the task to be disqualified.
\item Your educator is a resource to you, and though they will not provide you with answers / solutions, they may be approached for hints.
	\end{itemize}	
\textbf{The Presentation}
The investigation is to be handed in on the due date, indicated to you by your educator. It should have as a minimum:
\begin{itemize}[noitemsep]
\item A description of the problem.
\item A discussion of the way you set about dealing with the problem.
\item A description of the final result with an appropriate justification of its validity.
\item Some personal reflections that include mathematical or other lessons learnt, as well as the feelings experienced whilst engaging in the problem.
\item The written-up version should be attractively and neatly presented on about four A4 pages.
\item Whilst the use of technology is encouraged in the presentation, the mathematical content and processes must remain the major focus.
\end{itemize}	
Below are some examples of possible rubrics to use when marking investigations:\par

Example 1:
\begin{table}[H]
 \begin{center}
  \begin{tabular}{|p{3cm}|p{8.5cm}|} \hline
\textbf{Level of Performance}& \textbf{Criteria} \\ \hline
4 &
\begin{itemize}[noitemsep]
\item Contains a complete response.
\item Clear, coherent, unambiguous and elegant explanation.
\item Includes clear and simple diagrams where appropriate.
\item Shows understanding of the question’s mathematical ideas and processes.
\item Identifies all the important elements of the question.
\item Includes examples and counter examples.
\item Gives strong supporting arguments.
\item Goes beyond the requirements of the problem. 
   \end{itemize} \\ \hline
3 & 
\begin{itemize}[noitemsep]
\item Contains a complete response.
\item Explanation less elegant, less complete.
\item Shows understanding of the question’s mathematical ideas and processes.
\item Identifies all the important elements of the question.
\item Does not go beyond the requirements of the problem.
\end{itemize} \\ \hline
2 &
\begin{itemize}[noitemsep]
\item Contains an incomplete response.
\item Explanation is not logical and clear.
\item Shows some understanding of the question’s mathematical ideas and processes.
\item Identifies some of the important elements of the question.
\item Presents arguments, but incomplete.
\item Includes diagrams, but inappropriate or unclear.
\end{itemize} \\ \hline
1 &
\begin{itemize}[noitemsep]
\item Contains an incomplete response.
\item Omits significant parts or all of the question and response.
\item Contains major errors.
\item Uses inappropriate strategies.
\end{itemize} \\ \hline
0 &
\begin{itemize}[noitemsep]
\item No visible response or attempt
\end{itemize} \\ \hline
  \end{tabular}

 \end{center}

\end{table}

\subsubsection{Orals}
An oral assessment involves the learner explaining to the class as a whole, a group or the educator his or her understanding of a concept, a problem or answering specific questions. The focus here is on the correct use of mathematical language by the learner and the conciseness and logical progression of their explanation as well as their communication skills.\par

Orals can be done in a number of ways:
\begin{itemize}[noitemsep]
\item A learner explains the solution of a homework problem chosen by the educator.
\item The educator asks the learner a specific question or set of questions to ascertain that the learner understands, and assesses the learner on their explanation.
\item The educator observes a group of learners interacting and assesses the learners on their contributions and explanations within the group.
\item A group is given a mark as a whole, according to the answer given to a question by any member of a group.
\end{itemize}
An example of a marking rubric for an oral:\\
1	-	the learner has understood the question and attempts to answer it\\
2	-	the learner uses correct mathematical language\\
2	-	the explanation of the learner follows a logical progression\\
2	-	the learner’s explanation is concise and accurate\\
2	-	the learner shows an understanding of the concept being explained\\
1	-	the learner demonstrates good communication skills\\
Maximum mark = 10 \par

An example of a peer-assessment rubric for an oral: \\
My name: \underline{~~~~~~~~~~~~~~~~~~~~~~~~~~~~~~~~~~}\\
Name of person I am assessing: \underline{~~~~~~~~~~~~~~~~~~~~~~~~~~~~~~~~~~}\\

\begin{table}[H]
 \begin{center}
  \begin{tabular}{|p{5cm}|p{2.5cm}|p{2.5cm}|} \hline
  \textbf{Criteria} & \textbf{Mark Awarded} & \textbf{Maximum Mark}\\ \hline
Correct Answer &&2\\ \hline
Clarity of Explanation&&3\\ \hline
Correctness of Explanation  &&3\\ \hline
Evidence of Understanding &&3\\ \hline
\textbf{Total} &&10\\ \hline

  \end{tabular}

 \end{center}

\end{table}

\section{Chapter Contexts}
\subsubsection{Algebraic expressions}
Algebra provides the basis for mathematics learners to move from numerical calculations to generalising operations, simplifying expressions, solving equations and using graphs and inequalities in solving contextual problems. Being able to multiply out and factorise are core skills in the process of simplifying expressions and solving equations in mathematics. Identifying irrational numbers and knowing their estimated position on a number line or graph is an important part of any mathematical calculations and processes that move beyond the basic number system of whole numbers and integers. Rounding off irrational numbers (such as the value of $\pi$) when needed, allows mathematics learners to work more efficiently with numbers that would otherwise be difficult to “pin down”, use and comprehend.  \par

Once learners have grasped the basic number system of whole numbers and integers, it is vital that their understanding of the numbers between integers is also expanded. This incorporates their dealing with fractions, decimals and surds which form a central part of most mathematical calculations in real-life contextual issues. Estimation is an extremely important component within mathematics. It allows learners to work with a calculator or present possible solutions while still being able to gauge how accurate and realistic their answers may be, which is relevant for other subjects too. 

\subsubsection{Equations and inequalities}
If learners are to later work competently with functions and the graphing and interpretation thereof, their understanding and skills in solving equations and inequalities will need to be developed. 
\subsubsection{Exponents}
Exponential notation is a central part of mathematics in numerical calculations as well as algebraic reasoning and simplification. It is also a necessary component for learners to understand and appreciate certain financial concepts such as compound interest and growth and decay. 
\subsubsection{Number patterns}
Much of mathematics revolves around the identification of patterns. In earlier grades learners saw patterns in the form of pictures and numbers. In this chapter  we look at the mathematics of patterns. Patterns are repetitive sequences and can be found in nature, shapes, events, sets of numbers and almost everywhere you care to look. For example, seeds in a sunflower, snowflakes, geometric designs on quilts or tiles, the number sequence $0; ~4;~ 8;~ 12; ~16; \ldots$
\subsubsection{Functions}
Functions form a core part of learners’ mathematical understanding and reasoning processes in algebra. This is also an excellent opportunity for contextual mathematical modelling questions. Functions are mathematical building blocks for designing machines, predicting natural disasters, curing diseases, understanding world economies and for keeping aeroplanes in the air. A useful advantage of functions is that they allow us to visualise relationships in terms of a graph. Functions are much easier to read and interpret than lists of numbers. In addition to their use in the problems facing humanity, functions also appear on a day-to-day level, so they are worth learning about. A function is always dependent on one or more variables, like time, distance or a more abstract  quantity.
\subsubsection{Finance and Growth }
The mathematics of finance is very relevant to daily and long-term financial decisions learners will need to take in terms of investing, taking loans, saving and understanding exchange rates and their influence more globally.
\subsubsection{Trigonometry}
There are many applications of trigonometry. Of particular value is the technique of triangulation, which is used in astronomy to measure the distances to nearby stars, in geography to measure distances between landmarks, and in satellite navigation systems. GPS (the global positioning system) would not be possible without trigonometry. Other fields which make use of trigonometry include acoustics, optics, analysis of financial markets, electronics, probability theory, statistics, biology, medical imaging (CAT scans and ultrasound), chemistry, cryptology, meteorology, oceanography, land surveying, architecture, phonetics, engineering, computer graphics and game development.
\subsubsection{Analytical geometry}
This section provides a further application point for learners’ algebraic and trigonometric interaction with the Cartesian plane. Artists and design and layout industries often draw on the content and thought processes of this mathematical topic.
\subsubsection{Statistics}
Citizens are daily confronted with interpreting data presented from the media. Often this data may be biased or misrepresented within a certain context. In any type of research, data collection and handling is a core feature. This topic also educates learners to become more socially and politically educated with regards to the media.
\subsubsection{Probability}
This topic is helpful in developing good logical reasoning in learners and for educating them in terms of real-life issues such as gambling and the possible pitfalls thereof. We use probability to describe uncertain events: when you accidentally drop a slice of bread, you don’t know if it’s going to fall with the buttered side facing upwards or downwards. When your favourite sports team plays a game, you don’t know whether they will win or not. When the weatherman says that there is a $40\%$ chance of rain tomorrow, you may or may not end up getting wet. Uncertainty presents itself to some degree in every event that occurs around us and in every decision that we make.
\subsubsection{Euclidean geometry }
The thinking processes and mathematical skills of proving conjectures and identifying false conjectures is more the relevance here than the actual content studied. The surface area and volume content studied in real-life contexts of designing kitchens, tiling and painting rooms, designing packages, etc. is relevant to the current and future lives of learners. Euclidean geometry deals with space and shape using a system of logical deductions.
\subsubsection{Measurement}
This chapter revises the volume and surface areas of three-dimensional objects, otherwise known as solids. The chapter covers the volume and surface area of prisms and cylinders, Many exercises cover finding the surface area and volume of polygons, prisms, pyramids, cones and spheres, as well as a complex object. The effect on volume and surface area when multiplying a dimension of a factor of $k$ is also explored.
\subsubsection{Exercise solutions}
This chapter includes the solutions to the exercises covered in each chapter of the book.