
\chapter{Front matter stuff}

\section{Support for educators}
Science education is about more than physics, chemistry and mathematics... It's about learning to think and to solve problems, which are valuable skills that can be applied through all spheres of life. Teaching these skills to our next generation is crucial in the current global environment where methodologies, technology and tools are rapidly evolving. Education should benefit from these fast moving developments. In our simplified model there are three layers to how technology can significantly influence your teaching and teaching environment. 

\subsubsection{First Layer: Educator Collaboration}
There are many tools that help educators collaborate more effectively. We know that communities of practice are powerful tools for the refinement of methodology, content and knowledge and are also superb for providing support to educators. One of the challenges facing community formation is the time and space to have sufficient meetings to build real communities and exchange practices, content and learnings effectively. Technology allows us to streamline this very effectively by transcending space and time. It is now possible to collaborate over large distances (transcending space) and when it is most appropriate for each individual (transcending time) by working virtually (email, mobile, online etc.).\par

Our textbooks have been re-purposed from content available on the Connexions website (\url{http://cnx.org/lenses/fhsst}). The content on this website is easily accessible and adaptable  as it is under an open licence, stored in an open format, based on an open standard, on an open-source platform, for free, where everyone can produce their own books. The content on Connexions is available under an open copyright license - CC-BY. This Creative Commons By Attribution Licence allows others to legally distribute, remix, tweak, and build upon the work available, even commercially, as long as the original author is credited for the original creation. This means that learners and educators are able to download, copy, share and distribute this content legally at no cost. It also gives educators the freedom to edit, adapt, translate and contextualise it, to better suit their teaching needs. \par

Connexions is a tool where individuals can share, but more importantly communities can form around the collaborative, online development of resources. Your community of educators can therefore:
\begin{itemize}
\item form an online workgroup around the content;
\item make your own copy of the content;
\item edit sections of your own copy;
\item add your own content or replace existing content with your own content;
\item use other content that has been shared on the platform in your own work;
\item create your own notes / textbook / course material as a community.
\end{itemize}
Educators often want to share assessment items as this helps reduce workload, increase variety and improve quality. Currently all the solutions to the exercises contained in the textbooks have been uploaded onto our free and open online assessment bank called Monassis (www.monassis.com), with each exercise having a shortcode link to its solution on Monassis. To access the solution simply go to \url{www.everythingmaths.co.za},  enter the shortcode, and you will be redirected to the solution on Monassis.\par

Monassis is similar to Connexions but is focused on the sharing of assessment items. Monassis contains a selection of test and exam questions with solutions, openly shared by educators. Educators can further search and browse the database by subject and grade and add relevant items to a test. The website automatically generates a test or exam paper with the corresponding memorandum for download.\par

By uploading all the end-of-chapter exercises and solutions to this open assessment bank, the larger community of educators in South Africa are provided with a wide selection of items to use in setting their tests and exams. More details about the use of Monassis as a collaboration tool are included in the Monassis section.\par

\subsubsection{Second Layer: Classroom Engagement}
In spite of the impressive array of rich media open educational resources available freely online (such as videos, simulations, exercises and presentations), only a small number of educators actively make use of them. Our investigations revealed that the overwhelming quantity, the predominant international context, and difficulty in correctly aligning them with the local curriculum level acts as deterrents. The opportunity here is that, if used correctly, they can make the classroom environment more engaging.\par

Presentations can be a first step to bringing material to life in ways that are more compelling than are possible with just a blackboard and chalk. There are opportunities to:
\begin{itemize}
\item create more graphical representations of the content;
\item control timing of presented content more effectively;
\item allow learners to relive the lesson later if constructed well;
 \item supplement the slides with notes for later use;
\item embed key assessment items in advance to promote discussion; and
\item embed other rich media like videos.
\end{itemize}
Videos have been shown to be potentially both engaging and effective. They provide opportunities to:
\begin{itemize}
\item present an alternative explanation;
\item challenge misconceptions without challenging an individual in the class; and
\item show an environment or experiment that cannot be replicated in the class which could be far away, too expensive or too dangerous.
\end{itemize}
Simulations are also very useful and can allow learners to:
\begin{itemize}
\item have increased freedom to explore, rather than reproduce a fixed experiment or process;
\item explore expensive or dangerous environments more effectively; and
\item overcome implicit misconceptions.
\end{itemize}
We realised the opportunity for embedding a selection of rich media resources such as presentations, simulations, videos and links into the online version of Everything Maths and Everything Science at the relevant sections. This will not only present them with a selection of locally relevant and curriculum aligned resources, but also position these resources within the appropriate grade and section. Links to these online resources are recorded in the print or PDF versions of the  books, making them a tour-guide or credible pointer to the world of online rich media available. \par
\subsubsection{Third Layer: Beyond the Classroom}
The internet has provided many opportunities for self-learning and participation which were never before possible. There are huge stand-alone archives of videos like the Khan Academy which covers most Mathematics for Grades 1 - 12 and Science topics required in FET. These videos, if not used in class, provide opportunities for the learners to:
\begin{itemize}
\item look up content themselves;
\item get ahead of class;
\item independently revise and consolidate their foundation; and
\item explore a subject to see if they find it interesting.
\end{itemize}
There are also many opportunities for learners to participate in science projects online as real participants (see the section on citizen cyberscience “On the web, everyone can be a scientist”). Not just simulations or tutorials but real science so that:
\begin{itemize}
\item learners gain an appreciation of how science is changing;
\item safely and easily explore subjects that they would never have encountered before university;
\item contribute to real science (real international cutting edge science programmes);
\item have the possibility of making real discoveries even from their school computer laboratory; and
\item find active role models in the world of science.
\end{itemize}
In our book we've embedded opportunities to help educators and learners take advantage of all these resources, without becoming overwhelmed at all the content that is available online. 

\subsubsection{Annotation of the book}
If you have any comments, thoughts or suggestions on the books, you can visit  \url{www.everythingmaths.co.za} and in educator mode, capture these in the text. These can range from sharing tips and ideas with your fellow educators, to discussing how to better explain concepts in class. Also, if you have picked up any errors in the book you can make a note of them here, and we will correct them in time for the next print run.\par

\section{On the Web, Everyone can be a Scientist}
Did you know that you can fold protein molecules, hunt for new planets around distant suns or simulate how malaria spreads in Africa, all from an ordinary PC or laptop connected to the Internet? And you don’t need to be a certified scientist to do this. In fact some of the most talented contributors are teenagers. The reason this is possible is that scientists are learning how to turn simple scientific tasks into competitive online games. \par

This is the story of how a simple idea of sharing scientific challenges on the Web turned into a global trend, called citizen cyberscience. And how you can be a scientist on the Web, too.\par

\subsubsection{Looking for Little Green Men}
A long time ago, in 1999, when the World Wide Web was barely ten years old and no one had heard of Google, Facebook or Twitter, a researcher at the University of California at Berkeley, David Anderson, launched an online project called SETI@home. SETI stands for Search for Extraterrestrial Intelligence. Looking for life in outer space.\par

Although this sounds like science fiction, it is a real and quite reasonable scientific project. The idea is simple enough. If there are aliens out there on other planets, and they are as smart or even smarter than us, then they almost certainly have invented the radio already. So if we listen very carefully for radio signals from outer space, we may pick up the faint signals of intelligent life.\par

Exactly what radio broadcasts aliens would produce is a matter of some debate. But the idea is that if they do, it would sound quite different from the normal hiss of background radio noise produced by stars and galaxies. So if you search long enough and hard enough, maybe you’ll find a sign of life. 

It was clear to David and his colleagues that the search was going to require a lot of computers. More than scientists could afford. So he wrote a simple computer program which broke the problem down into smaller parts, sending bits of radio data collected by a giant radio-telescope to volunteers around the world. The volunteers agreed to download a programme onto their home computers that would sift through the bit of data they received, looking for signals of life, and send back a short summary of the result to a central server in California. 

The biggest surprise of this project was not that they discovered a message from outer space. In fact, after over a decade of searching, no sign of extraterrestrial life has been found, although there are still vast regions of space that have not been looked at. The biggest surprise was the number of people willing to help such an endeavour. Over a million people have downloaded the software, making the total computing power of SETI@home rival that of even the biggest supercomputers in the world.

David was deeply impressed by the enthusiasm of people to help this project. And he realized that searching for aliens was probably not the only task that people would be willing to help with by using the spare time on their computers. So he set about building a software platform that would allow many other scientists to set up similar projects. You can read more about this platform, called BOINC, and the many different kinds of volunteer computing projects it supports today, at http://boinc.berkeley.edu/ . 

There’s something for everyone, from searching for new prime numbers (PrimeGrid) to simulating the future of the Earth’s climate (ClimatePrediction.net). One of the projects, MalariaControl.net, involved researchers from the University of Cape Town as well as from universities in Mali and Senegal. 

The other neat feature of BOINC is that it lets people who share a common interest in a scientific topic share their passion, and learn from each other. BOINC even supports teams – groups of people who put their computer power together, in a virtual way on the Web, to get a higher score than their rivals. So BOINC is a bit like Facebook and World of Warcraft combined – part social network, part online multiplayer game.

Here’s a thought: spend some time searching around BOINC for a project you’d like to participate in, or tell your class about. 
You are a Computer, too
Before computers were machines, they were people. Vast rooms full of hundreds of government employees used to calculate the sort of mathematical tables that a laptop can produce nowadays in a fraction of a second. They used to do those calculations laboriously, by hand. And because it was easy to make mistakes, a lot of the effort was involved in double-checking the work done by others. 

Well, that was a long time ago. Since electronic computers emerged over 50 years ago, there has been no need to assemble large groups of humans to do boring, repetitive mathematical tasks. Silicon chips can solve those problems today far faster and more accurately. But there are still some mathematical problems where the human brain excels.

Volunteer computing is a good name for what BOINC does: it enables volunteers to contribute computing power of their PCs and laptops. But in recent years, a new trend has emerged in citizen cyberscience that is best described as volunteer thinking. Here the computers are replaced by brains, connected via the Web through an interface called eyes. Because for some complex problems – especially those that involve recognizing complex patterns or three-dimensional objects – the human brain is still a lot quicker and more accurate than a computer.

Volunteer thinking projects come in many shapes and sizes. For example, you can help to classify millions of images of distant galaxies (GalaxyZoo), or digitize hand-written information associated with museum archive data of various plant species (Herbaria@home).  This is laborious work, which if left to experts would take years or decades to complete. But thanks to the Web, it’s possible to distribute images so that hundreds of thousands of people can contribute to the search. 

Not only is there strength in numbers, there is accuracy, too. Because by using a technique called validation – which does the same sort of double-checking that used to be done by humans making mathematical tables – it is possible to practically eliminate the effects of human error. This is true even though each volunteer may make quite a few mistakes. So projects like Planet Hunters have already helped astronomers pinpoint new planets circling distant stars. The game FoldIt invites people to compete in folding protein molecules via a simple mouse-driven interface. By finding the most likely way a protein will fold, volunteers can help understand illnesses like Alzheimer’s disease, that depend on how proteins fold. 

Volunteer thinking is exciting. But perhaps even more ambitious is the emerging idea of volunteer sensing: using  your laptop or even your mobile phone to collect data – sounds, images, text you type in – from any point on the planet, helping scientists to create global networks of sensors that can pick up the first signs of an outbreak of a new disease (EpiCollect), or the initial tremors associated with an earthquake (QuakeCatcher.net), or the noise levels around a new airport (NoiseTube).

There are about a billion PCs and laptops on the planet, but already 5 billion mobile phones. The rapid advance of computing technology, where the power of a ten-year old PC can easily be packed into a smart phone today, means that citizen cyberscience has a bright future in mobile phones. And this means that more and more of the world’s population can be part of citizen cyberscience projects. Today there are probably a few million participants in a few hundred citizen cyberscience initiatives. But there will soon be seven billion brains on the planet. That is a lot of potential citizen cyberscientists. 

You can explore much more about citizen cyberscience on the Web. There’s a great list of all sorts of projects, with brief summaries of their objectives, at http://distributedcomputing.info/ . BBC Radio 4 produced a short series on citizen science http://www.bbc.co.uk/radio4/science/citizenscience.shtml  and you can subscribe to a newsletter about the latest trends in this field at http://scienceforcitizens.net/  . The Citizen Cyberscience Centre, www.citizencyberscience.net  which is sponsored by the South African Shuttleworth Foundation, is promoting citizen cyberscience in Africa and other developing regions.