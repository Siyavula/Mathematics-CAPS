         \chapter{Number patterns}
% \section{Number pattern examples}
% \section{Describing sequences}
\begin{exercises}{}
{ 
\begin{enumerate}[noitemsep, label=\textbf{\arabic*}. ] 
\item Write down the next three terms in each of the following sequences:
  \begin{enumerate} [noitemsep, label=\textbf{(\alph*)} ]
  \item $5;~15;~25;~\ldots$
  \item $-8;-3;~2:`\ldots$
  \item $30;~27;~24;~\ldots$
  \end{enumerate}
 \item The general term is given for each sequence below. Calculate the missing terms.
  \begin{enumerate} [noitemsep, label=\textbf{(\alph*)} ]
  \item $0;3;~\ldots;~15;~24$\hspace{2.2cm}$T_{n}={n}^{2}-1$
  \item $3;~2;~1;~0;~\ldots;~-2$\hspace{2cm}$T_{n}=-n+4$
  \item $-11;~\ldots;~-7;~\ldots;~-3$\hspace{1.5cm}$T_{n}=-13+2n$
  \end{enumerate}
\item Find the general formula for the following sequences and then find ${T}_{10}$, ${T}_{50}$ and ${T}_{100}$
  \begin{enumerate}[noitemsep, label=\textbf{(\alph*)} ]
  \item $2;~5;~8;~11;~14;~\ldots$
  \item $0;~4;~8;~12;~16;~\ldots$
  \item $2;~-1;~-4;~-7;~-10;~\ldots$
  \end{enumerate}
\end{enumerate}

}%\End of exercise
\end{exercises}


 \begin{solutions}{}{
\begin{multicols}{2}
\begin{enumerate}[noitemsep, label=\textbf{\arabic*}. ] 
\item 
  \begin{enumerate} [noitemsep, label=\textbf{(\alph*)} ]
  \item $35;45;55$
  \item $7;12;7$
  \item $21;18;15$
  \end{enumerate}
 \item 
  \begin{enumerate} [noitemsep, label=\textbf{(\alph*)} ]
  \item $T_n = n^{2}-1\\
T_3 = 3^2 - 1\\
= 9 -1\\
= 8$
  \item $T_n = -n + 4\\
T_5 = -5 + 4\\
= -1$
  \item $T_n = -13 + 2n\\
T_2 = -13 + 2(2)\\
= -13 + 4\\
= -9$ \\
$T_n = -13 + 2n\\
T_4 = -13 + 2(4)\\
= -13 + 8\\
= -5$
  \end{enumerate}
\item 
  \begin{enumerate}[noitemsep, label=\textbf{(\alph*)} ]
  \item $a = 2\\
d = 3\\
T_n = 3n - 1\\
T_{10} = 3(10) - 1 = 29\\
T_{50} = 3(50) - 1 = 149\\
T_{100} = 3(100) - 1 = 299$ 
  \item $a = 0\\
d = 4\\
T_n = 4n - 4\\
T_{10} = 4(10) - 4 = 36\\
T_{50} = 4(50) - 4 = 196\\
T_{100} = 4(100) - 4 = 396$ 
  \item $a = 2\\
d = -3\\
T_n = 5 - 3n\\
T_{10} = 5 - 3(10) = -25\\
T_{50} = 5 - 3(50) = -145\\
T_{100} = 5 - 3(100) = -295$ 
  \end{enumerate}
\end{enumerate}
\end{multicols}
}
\end{solutions}


% \section{Patterns and conjecture}
\begin{eocexercises}{}
\begin{enumerate}[noitemsep, label=\textbf{\arabic*}. ] 
\item Find the $6^{\mathrm{th}}$ term in each of the following sequences:
  \begin{enumerate}[noitemsep, label=\textbf{(\alph*)} ]
  \item $4;~13;~22;~31;~\ldots$
  \item $5;~2;~-1;~-4;~\ldots$
  \item $7,4;~9,7;~12;~14,3;~\ldots$
  \end{enumerate}
\item Find the general term of the following sequences:
  \begin{enumerate}[noitemsep, label=\textbf{(\alph*)} ]
  \item $3;~7;~11;~15;~\ldots$
  \item $-2;~1;~4;~7;~\ldots$
  \item $11;~15;~19;~23;~\ldots$
  \item $\dfrac{1}{3};~\dfrac{2}{3};~1;~1\dfrac{1}{3};~\ldots$
  \end{enumerate}
\item The seating of a sports stadium is arranged so that the first row has $15$ seats, the second row has $19$ seats, the third row has $23$ seats and so on. Calculate how many seats are in the twenty-fifth row.
\item A single square is made from $4$ matchsticks. Two squares in a row need $7$ matchsticks and three squares in a row need $10$ matchsticks. For this sequence determine:
  \begin{enumerate}[noitemsep, label=\textbf{(\alph*)} ]
  \item the first term;
  \item the common difference;
  \item the general formula;
  \item how many matchsticks there are in a row of twenty-five squares.
  \end{enumerate}
\setcounter{subfigure}{0}
\begin{figure}[H] 
\begin{center}
\begin{pspicture}(0,0)(8,2)
\def\match{\psline(0,0)(2,0)\psellipse*(1.8,0)(0.2,0.1)}
\rput(0,0){\match}
\rput{90}(2,0){\match}
\rput{180}(2,2){\match}
\rput{270}(0,2){\match}
\rput(2,0){\rput(0,0){\match}
\rput{90}(2,0){\match}
\rput{180}(2,2){\match}}
\rput(4,0){\rput(0,0){\match}
\rput{90}(2,0){\match}
\rput{180}(2,2){\match}}
\rput(6,0){\rput(0,0){\match}
\rput{90}(2,0){\match}
\rput{180}(2,2){\match}}
\end{pspicture}
\vspace{2pt}
\vspace{.1in}
\end{center}
\end{figure}       
\item You would like to start saving some money, but because you have never tried to save money before, you decide to start slowly. At the end of the first week you deposit R$~5$ into your bank account. Then at the end of the second week you deposit R$~10$ and at the end of the third week, R$~15$. After how many weeks will you deposit R$~50$ into your bank account?
\item A horizontal line intersects a piece of string at $4$ points and divides it into five parts, as shown below.
\setcounter{subfigure}{0}
\begin{figure}[H] 
\begin{center}
\begin{pspicture}(-1,-2)(6,2)
\psplot[xunit=0.00556, linewidth=1pt]{90}{810}{x sin}
\psline[linestyle=dashed](-1,0)(6,0)
\psdots[dotsize=5pt](1,0)(2,0)(3,0)(4,0)
\rput(0.5,1.5){\psframebox{1}}
\rput(1.5,-1.5){\psframebox{2}}
\rput(2.5,1.5){\psframebox{3}}
\rput(3.5,-1.5){\psframebox{4}}
\rput(4.5,1.5){\psframebox{5}}
\end{pspicture}
\end{center}
\end{figure}  
If the piece of string is intersected in this way by $19$ parallel lines, each of which intersects it at $4$ points, determine the number
of parts into which the string will be divided.
\item Consider what happens when you add $9$ to a two-digit number:
  \begin{equation*}
    \begin{array}{ccl}\hfill 9+16&=& 25\\ 9+28 &=& 37\\9+43&=& 52\end{array}
  \end{equation*} 
  \begin{enumerate}[noitemsep, label=\textbf{(\alph*)} ]
  \item What pattern do you see?
  \item Make a conjecture and express it in words.
  \item Generalise your conjecture algebraically.
  \item Prove that your conjecture is true.
  \end{enumerate}
\end{enumerate}

\end{eocexercises}


 \begin{eocsolutions}{}{
\begin{enumerate}[itemsep=5pt, label=\textbf{\arabic*}. ] 
 \item \begin{multicols}{2}



  \begin{enumerate}[noitemsep, label=\textbf{(\alph*)} ]
  \item $T_6 = 49$
  \item $T_6 = -10$
  \item $T_6 = 18,9$
  \end{enumerate}
\end{multicols}
\item \begin{multicols}{2}
  \begin{enumerate}[noitemsep, label=\textbf{(\alph*)} ]
  \item $a = 3\\
d = 4\\
T_n = 3 + (n - 1)(4)\\
T_n = 4n - 1$ 
  \item $a = -2\\
d = 3\\
T_n = -2 + (n-1)(3) \\
T_n = 3n -5$ 
  \item $a = 11\\
d = 4\\
T_n = 11 + 4(n-1)\\
T_n = 4n + 7 $ 
  \item $a = \frac{1}{3}\\
d = \frac{1}{3}\\
T_n = \frac{1}{3} + \frac{1}{3}(n - 1)\\
T_n = \frac{n}{3} $ 
  \end{enumerate}
\end{multicols}
\item Extract the relevant information which is a sequence: $15; 19 ; 23 \ldots$\\
 $a = 4\\
d = 15\\
T_n = 15 + 4(n - 1)\\
T_n = 4n + 11 $ \\
So the number of seats in row 25 is:\\
$T_{25} = 4(25) + 11 = 111$ 
\item 
  \begin{enumerate}[noitemsep, label=\textbf{(\alph*)} ]
  \item The first term is 4 since this is the number of matches needed for 1 square.
  \item The common difference between the terms is $3$. ($7 -4 = 3$ ; $10 - 7 = 3$)
  \item $T_n = 4 + 3(n -1)\\
T_n = 3n + 1$
  \item $T_25 = 3(25) + 1 \\
= 76$
  \end{enumerate}   
\item Write down the sequence:\\
$5 ; 10 ; 15 ; \ldots$
We first find the general formula:\\
$a = 5\\
d = 5\\
T_n = 5 + 5(n - 1)\\
T_n = 5n  \\
50 = 5n \\
n = 10 $ weeks
\item With one line intersecting at four points we get five parts. If we add a second line it is now broken up into 9 parts. And if we add a third line it is now broken up into $13$ parts. So we see that for each line added we add four parts. The sequence is: $5 ; 9 ; 13 \ldots$\\
$a = 5\\
d = 4\\
T_n = 5 + 4(n - 1)\\
T_n = 4n + 1  \\
T_{19} = 4(19) + 1 \\
T_{19} = 77 $
\item 
  \begin{enumerate}[noitemsep, label=\textbf{(\alph*)} ]
  \item The first digit of the answer increases by one number and the second digit decreases by one number.
  \item When $9$ is added to any two-digit number, the answer is the two-digit number, with its first ('tens') digit increased by one number and its second ('ones') digit decreased by one number.
  \item $9 + (10x + y) = 10(x+1)+(y-1)$
  \item $9 + 16 = 25\\
9 + (10x + y) = 10(x + 1) + (y-1)\\
9 + (10 + 6) = 10(2) + (6-1)\\
9 + 16 = 20 + 5 = 25\\
9 + (10x + y) = 10x + 10 + y - 1\\
9 + (10x + y) = (10 - 1) + (10x + y) 
= 9 + (10x + y)$
  \end{enumerate}
\end{enumerate}}
\end{eocsolutions}


