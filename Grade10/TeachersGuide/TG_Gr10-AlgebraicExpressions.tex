\chapter{Algebraic expressions}
%Ex 1-1 Questions
\begin{exercises}{}{
\begin{enumerate}[itemsep=5pt, label=\textbf{\arabic*}. ] 
\item If $a$ is an integer, $b$ is an integer and $c$ is irrational, which of the following are rational numbers? 
  \begin{enumerate}[itemsep=5pt, label=\textbf{(\alph*)} ] 
    \item $\dfrac{5}{6}$
    \item $\dfrac{a}{3}$
    \item $\dfrac{-2}{b}$
    \item $\dfrac{1}{c}$
    \end{enumerate}
\item If $\dfrac{a}{1}$ is a rational number, which of the following are valid values for $a$?
    \begin{enumerate}[itemsep=5pt, label=\textbf{(\alph*)} ] 
    \item $1$
    \item $-10$
    \item $\sqrt{2}$
    \item $2,1$
    \end{enumerate}
\item Write the following as fractions:
    \begin{enumerate}[itemsep=5pt, label=\textbf{(\alph*)} ] 
    \item $0,1$
    \item $0,12$
    \item $0,58$
    \item $0,2589$
    \end{enumerate}
\item Write the following using the recurring decimal notation:
    \begin{enumerate}[itemsep=5pt, label=\textbf{(\alph*)} ] 
    \item $0,11111111\ldots$
    \item $0,1212121212\ldots$
    \item $0,123123123123\ldots$
    \item $0,11414541454145\ldots$
    \end{enumerate}
\item Write the following in decimal form, using the recurring decimal notation:
    \begin{enumerate}[itemsep=5pt, label=\textbf{(\alph*)} ] 
    \item $\dfrac{2}{3}$
    \item $1\dfrac{3}{11}$
    \item $4\dfrac{5}{6}$
    \item $2\dfrac{1}{9}$
    \end{enumerate}
\item Write the following decimals in fractional form:
    \begin{enumerate}[itemsep=5pt, label=\textbf{(\alph*)} ] 
    \item $0,\dot{5}$
    \item $0,6\dot{3}$
    \item $5,\overline{31}$
    \end{enumerate}
\end{enumerate}
}
\end{exercises}


 \begin{solutions}{}{
%Exercise 1-1 Solutions
\begin{enumerate}[itemsep=5pt, label=\textbf{\arabic*}. ] 
\item   %Question 1
\begin{enumerate}[itemsep=5pt, label=\textbf{(\alph*)} ] 
   \item Rational
\item Rational
\item Rational
\item Irrational 
  \end{enumerate}

\item %Question 2
\begin{enumerate}[itemsep=5pt, label=\textbf{(\alph*)} ] 
 \item Rational
\item Rational
\item Irrational
\item Rational
\end{enumerate}

\item %Question 3
\begin{enumerate}[itemsep=5pt, label=\textbf{(\alph*)} ] 
\item $0,1 = \frac{1}{10}$
\item $0,12 = \frac{12}{100} = \frac{3}{25}$
\item $0,58 = \frac{58}{100} = \frac{29}{50}$
\item $0,2589 = \frac{2589}{10000}$
\end{enumerate}

\item %Question 4
\begin{enumerate}[itemsep=5pt, label=\textbf{(\alph*)} ] 
\item $0,\dot{1}$
\item $0,\overline{12}$
\item $0,\overline{123}$
\item $0,11\overline{4145}$
\end{enumerate}

\item %Question 5
\begin{enumerate}[itemsep=5pt, label=\textbf{(\alph*)} ] 
\item $\frac{2}{3} = 2(\frac{1}{3}) = 2(0,333333\ldots) = 0,666666\ldots = 0,\dot{6}$
\item $1\frac{3}{11} = 1 + 3(\frac{1}{11} = 1 + 3(0,090909\ldots)= 1 + 0,27272727\ldots = 1,\overline{27}$
\item $4\frac{5}{6} = 4 + 5(\frac{1}{6}) = 4= 5(0,1666666\ldots) = 4 + 0,833333\ldots = 4,8\dot{3}$
\item $2\frac{1}{9} = 2 + 0,1111111\ldots = 2,\dot{1}$
\end{enumerate}

\item %Question 6
\begin{enumerate}[itemsep=5pt, label=\textbf{(\alph*)} ] 
 \item $x=0,55555$ and $10x = 5,55555$. \\$\therefore 10x-x = 9x= 5$, $\therefore x=\frac{5}{9}$
\item $10x = 6,3333$ and $100x= 63,3333$. \\$\therefore 100x-10x = 90x = 57$, $\therefore x=\frac{57}{90}$
\item $x = 5,313131$ and $100x=531,313131$.\\ $\therefore 100x-x=99x=526$, $\therefore x=\frac{526}{99}$
\end{enumerate}
\end{enumerate}
}
\end{solutions}

%Exercise 1-2 Questions
\begin{exercises}{}
{
Use your calculator to write the following in decimal form, rounded to $3$ decimal places:
\begin{enumerate}[itemsep=5pt, label=\textbf{\arabic*}. ]
 \item $4\pi$
\item $\sqrt{11}$
\item $\dfrac{0,8}{3}$
\item $\sqrt[3]{7}$
\item $2\sqrt{10}$
\item $\dfrac{1}{18}$
\end{enumerate}

}
\end{exercises}

% Ex 1-2 Solutions

 \begin{solutions}{}{
\begin{enumerate}[itemsep=5pt, label=\textbf{\arabic*}. ] 
 \item $12,5666$%
\item $3,317$%
\item $0,267$%
\item $1,913$
\item $6,325$%
\item $0,056$%
\end{enumerate}}
\end{solutions}


% Ex1-3 Questions
\begin{exercises}{}
 {
Determine between which two consecutive integers the following numbers lie, without using a calculator:
\begin{enumerate}[itemsep=5pt, label=\textbf{\arabic*}. ]
\item $\sqrt{18}$
\item $\sqrt{29}$
\item $\sqrt[3]{5}$
\item $\sqrt[3]{79}$
\end{enumerate}

}
\end{exercises}

% Ex 1-3 solutions
 \begin{solutions}{}{
\begin{enumerate}[noitemsep, label=\textbf{\arabic*}. ]
\item $4$ and $5$, $(4^2 = 16$ and $5^2=25)$%
\item $5$ and $6$, $(5^2 = 25$ and $6^2=36)$%
\item $1$ and $2$, $(1^3 = 1$ and $2^3=8)$%
\item $4$ and $5$, $(4^3 = 64$ and $5^3=125)$%
\end{enumerate}
}
\end{solutions}


% Ex 1-4 questions
\begin{exercises}{}
{
Expand the following products:
\begin{multicols}{2}
\begin{enumerate}[label=\textbf{\arabic*}., itemsep=5pt]
\item $2y(y+4)$ 
\item $(y+5)(y+2) $
\item $(2-t)(1-2t)$
\item $(x-4)(x+4)$
\item $ (2p+9)(3p+1)$
\item $(3k-2)(k+6)$
\item $(s+6)^2$
\item $-(7-x)(7+x)$
\item $(3x-1)(3x+1)$
\item $(7k+2)(3-2k)$
\item $(1-4x)^2$
\item $(-3-y)(5-y)$
\item $(8-x)(8+x)$
\item $(9+x)^2$
\item $(-2{y}^{2}-4y+11)(5y-12)$ 
\item $(7{y}^{2}-6y-8)(-2y+2)$% make-rowspan-placeholders
\item $(10{y}+3)(-2{y}^{2}-11y+2)$ 
\item $(-12y-3)(12{y}^{2}-11y+3)$% make-rowspan-placeholders
\item $(-10)(2{y}^{2}+8y+3)$ 
\item $(2{y}^{6}+3{y}^{5})(-5y-12)$% make-rowspan-placeholders
\item $(-7y+11)(-12y+3)$% make-rowspan-placeholders
\item $(7y+3)(7{y}^{2}+3y+10)$% make-rowspan-placeholders
\item $(9)(8{y}^{2}-2y+3)$ 
\item $(-6{y}^{4}+11{y}^{2}+3y)(y+4)(y-4)$ 
\end{enumerate}
\end{multicols}

}
\end{exercises}

% Ex1-4 solutions
 \begin{solutions}{}{
\begin{enumerate}[noitemsep, label=\textbf{\arabic*}. ]
 \item $2y^2 + 8y$%
\item $y^2 + 2y+5y+10 =y^2 + 7y + 10$%$(y+5)(y+2) $
\item $2+4t-t+2t^2=2 + 3t +2t^2$%$(2-t)(1-2t)$
\item $x^2+4x-4x-16=x^2 - 16$%$(x-4)(x+4)$
\item $6p^2+2p+27p+9=6p^2 + 29p + 9$%$ (2p+9)(3p+1)$
\item $3k^2+18k-2k-12=3k^2 +16k - 12$%$(3k-2)(k+6)$
\item $s^2 + 6s+6s+36=s^2 + 12s +36$%$(s+6)^2$
\item $-(49+7x-7x-x^2)=-(49-x^2)=-49 + x^2$%$-(7-x)(7+x)$
\item $9x^2+3x-3x-1=9x^2 - 1$%$(3x-1)(3x+1)$
\item $21k-14k^2 + 6-4k=14k^2 + 17k + 6$%$(7k+2)(3-2k)$
\item $(1-4x)(1-4x)=1-4x-4x+16x^2=1 -8x + 16x^2$%$(1-4x)^2$
\item $-15+3y-5y+y^2=y^2 - 2y - 15$%$(-3-y)(5-y)$
\item $16+8x-8x-x^2=16 - x^2$%$(8-x)(8+x)$
\item $(9+x)(9+x)=81+9x+9x+x^2=81 + 18x + x^2$%$(9+x)^2$
\item $-10y^3+24y^2-20y^2+48y+55y-132=-10y^3 + 4y^2 + 103y - 132$% $(-2{y}^{2}-4y+11)(5y-12)$ 
\item $-14y^3+14y^2+12y^2-12y+16y-16=-14y^3 + 26y^2 + 4y -16$ % $(7{y}^{2}-6y-8)(-2y+2)$% 
\item $-20y^3 - 110y^2 +20y-6y^2-33y+6=-20y^3 -116y^2 -13y +6$%$(10{y}^{5}+3)(-2{y}^{2}-11y+2)$ 
\item $-144y^3+132y^2-36y-36y^2+33y-9=-144y^3 + 96y^2 -3y -9$%$(-12y-3)(12{y}^{2}-11y+3)$% 
\item $-20y^2 - 80y - 30$%$(-10)(2{y}^{2}+8y+3)$ 
\item $-10y^7-24y^6-15y^6-36y^5=-10y^7 - 39y^6 - 36y^5$%$(2{y}^{6}+3{y}^{5})(-5y-12)$% 
\item $84y^2-21y-132y+33=84y^2 - 153y +33$%$(-7y+11)(-12y+3)$% 
\item $49y^3+21y^2+70y+21y^2+9y+10=49y^3 + 42y^2 + 79y + 10$%$(7y+3)(7{y}^{2}+3y+10)$% m
\item $72y^2 - 18y + 27$%$(9)(8{y}^{2}-2y+3)$ 
\item $(-6y^4+11y^2+3y)(y^2+16)=-6y^6-96y^4+11y^4+176y^2+3y^3+48y=-6y^6 - 85y^4 + 3y^3 + 176y^2 + 48y$%$(-6{y}^{4}+11{y}^{2}+3y)(10y+4)(4y-4)$ 
\end{enumerate}}
\end{solutions}


% Ex 1-5 questions
\begin{exercises}{}
{
Find the highest common factors of the
following pairs of terms:\par
\begin{multicols}{2}
\begin{enumerate}[label=\textbf{\arabic*}., itemsep=5pt]
\item $6y;~18x$
\item $12mn;~8n$
\item $3st;~4su$ 
\item $18kl;~9kp$
\item $abc;~ac$% 
\item $2xy;~4xyz$
\item $3uv;~6u$ 
\item $9xy;~15xz$
\item $24xyz;~16yz$
\item $3m;~45n$
\end{enumerate}
\end{multicols}

}
\end{exercises}

% Ex 1-5 solutions
 \begin{solutions}{}{
\begin{enumerate}[label=\textbf{\arabic*}., noitemsep]
\item $6$%$6y;~18x$
\item $4n$%$12mn;~8n$
\item $s$%$3st;~4su$ 
\item $9k$%$18kl;~9kp$
\item $ac$%$abc;~ac$% 
\item $2xy$%$2xy;~4xyz$
\item $3u$%$3uv;~6u$ 
\item $3x$%$9xy;~15xz$
\item $4yz$%$24xyz;~16yz$
\item $3$%$3m;~45n$
\end{enumerate}
}
\end{solutions}


% Ex 1-6 questions
\begin{exercises}{}{
Factorise:
\begin{multicols}{2}
\begin{enumerate}[itemsep=5pt, label=\textbf{\arabic*}. ] 
\item $2l+2w$
\item $12x+32y$
\item $6{x}^{2}+2x+10{x}^{3}$
\item $2x{y}^{2}+x{y}^{2}z+3xy$
\item $-2a{b}^{2}-4{a}^{2}b$
\item $7a+4$ 
\item $20a-10$ 
\item $18ab-3bc$
\item $12kj+18kq$ 
\item $16{k}^{2}-4$ 
\item $3{a}^{2}+6a-18$
\item $-12a+24a^3$ 
\item $-2ab-8a$ 
\item $24kj-16{k}^{2}j$
\item $-{a}^{2}b-{b}^{2}a$ 
\item $12{k}^{2}j+24{k}^{2}{j}^{2}$ 
\item $72{b}^{2}q-18{b}^{3}{q}^{2}$
\item $4(y-3)+k(3-y)$ 
\item $a^2(a-1)-25(a-1)$ 
\item $bm(b+4)-6m(b+4)$
\item ${a}^{2}(a+7)+9(a+7)$ 
\item $3b(b-4)-7(4-b)$ 
\item ${a}^{2}{b}^{2}{c}^{2}-1$
\end{enumerate}
\end{multicols}

}
\end{exercises}

% Ex 1-6 solutions
 \begin{solutions}{}{
\begin{enumerate}[noitemsep, label=\textbf{\arabic*}. ] 
\item $2(l + w)$%$2l+2w$
\item $4(3x + 8y)$%$12x+32y$
\item $2x(3x + 1 +5x^2)$%$6{x}^{2}+2x+10{x}^{3}$
\item $xy(2y^2 + yz + 3)$%$2x{y}^{2}+x{y}^{2}z+3xy$
\item $-2ab(b + a)$%$-2a{b}^{2}-4{a}^{2}b$
\item $7a + 4$%$7a+4$ 
\item $10(2a - 1)$%$20a-10$ 
\item $3b(6a - c)$%$18ab-3bc$
\item $6k(2j + 3q)$%$12kj+18kq$ 
\item $(4k - 2)(4k + 2)$%$16{k}^{2}-4$ 
\item $3(a^2 + 2a - 9)$%$3{a}^{2}+6a-18$
\item $12a( 2a^2 -1)$%$-12a+24a^3$ 
\item $-2a(b + 4)$%$-2ab-8a$ 
\item $8kj(3 - 2k)$%$24kj-16{k}^{2}j$
\item $-ab(a + b)$%$-{a}^{2}b-{b}^{2}a$ 
\item $12jk^2(2j+1)$%$12{k}^{2}j+24{k}^{2}{j}^{2}$ 
\item $18b^2q(4 - bq)$%$72{b}^{2}q-18{b}^{3}{q}^{2}$
\item $(3 - y)(-4 + k)$%$4(y-3)+k(3-y)$ 
\item $(a-1)(a^2-25)=(a - 1)(a - 5)(a + 5)$%$a^2(a-1)-25(a-1)$ 
\item $(b+4)(bm-6m)=(b + 4)(m)(b - 6)$%$bm(b+4)-6m(b+4)$
\item $(a + 7)(a^2 + 9)$ %${a}^{2}(a+7)+9(a+7)$ 
\item $(b - 4)(3b + 7)$%$3b(b-4)-7(4-b)$ 
\item $(abc - 1)(abc + 1)$%${a}^{2}{b}^{2}{c}^{2}-1$
\end{enumerate}}
\end{solutions}

% ex 1-7 questions
\begin{exercises}{}
{
\begin{enumerate}[itemsep=5pt, label=\textbf{\arabic*}. ] 
\item Factorise the following:
\begin{multicols}{2}
\begin{enumerate}[itemsep=5pt, label=\textbf{(\alph*)} ] 
\item ${x}^{2}+8x+15$
\item ${x}^{2}+10x+24$
\item ${x}^{2}+9x+8$
\item ${x}^{2}+9x+14$
\item ${x}^{2}+15x+36$
\item ${x}^{2}+12x+36$
\end{enumerate}
\end{multicols}
\item Write the following expressions in factorised form:
\begin{multicols}{2}
\begin{enumerate}[itemsep=5pt, label=\textbf{(\alph*)} ]  
\item ${x}^{2}-2x-15$
\item ${x}^{2}+2x-3$
\item ${x}^{2}+2x-8$
\item ${x}^{2}+x-20$
\item ${x}^{2}-x-20$
\item $2{x}^{2}+22x+20$
\end{enumerate}
\end{multicols}
\item Find the factors of the following trinomial expressions:
\begin{multicols}{2}
\begin{enumerate}[itemsep=5pt, label=\textbf{(\alph*)} ] 
\item $3{x}^{2}+19x+6$
\item $6{x}^{2}+7x+1$
\item $12{x}^{2}+8x+1$
\item $8{x}^{2}+6x+1$
\end{enumerate}
\end{multicols}
\item Factorise completely:
\begin{multicols}{2}
\begin{enumerate}[itemsep=5pt, label=\textbf{(\alph*)} ] 
\item $3{x}^{2}+17x-6$
\item $7{x}^{2}-6x-1$
\item $8{x}^{2}-6x+1$
\item $6{x}^{2}-15x-9$
\end{enumerate}
\end{multicols}
\end{enumerate}

}
\end{exercises} 

% ex 1-7 solutions
 \begin{solutions}{}{
\begin{enumerate}[noitemsep, label=\textbf{\arabic*}. ] 
\item %%Factorise the following:

\begin{enumerate}[noitemsep, label=\textbf{(\alph*)} ] 
\item $(x + 5)(x + 3)$%${x}^{2}+8x+15$
\item $(x + 6)(x + 4)$%${x}^{2}+10x+24$
\item $(x + 8)(x + 1)$%${x}^{2}+9x+8$
\item $(x + 7)(x + 2)$%${x}^{2}+9x+14$
\item $(x + 12)(x + 3)$%${x}^{2}+15x+36$
\item $(x + 6)(x + 6)$%${x}^{2}+12x+36$
\end{enumerate}



\item %%Write the following expressions in factorised form:

\begin{enumerate}[noitemsep, label=\textbf{(\alph*)} ]  
% \setcounter{enumi}{6}
\item $(x + 5)(x - 3)$%${x}^{2}-2x-15$
\item $(x + 3)(x - 1)$%${x}^{2}+2x-3$
\item $(x + 4)(x - 2)$%${x}^{2}+2x-8$
\item $(x + 5)(x - 4)$%${x}^{2}+x-20$
\item $(x - 5)(x + 4)$%${x}^{2}-x-20$
\item $2(x^2+11x+10)=2(x + 10)(x + 1)$%$2{x}^{2}+22x+20$
\end{enumerate}



\item %%Find the factors of the following trinomial expressions:

\begin{enumerate}[noitemsep, label=\textbf{(\alph*)} ] 


\item $(3x + 1)(x + 6)$%$3{x}^{2}+19x+6$
\item $(6x + 1)(x + 1)$%$6{x}^{2}+7x+2$
\item $(6x + 1)(2x + 1)$%$12{x}^{2}+8x+1$
\item $(2x + 1)(4x + 1)$%$8{x}^{2}+6x+1$
\end{enumerate}


\item %%Factorise completely:

\begin{enumerate}[noitemsep, label=\textbf{(\alph*)} ] 
% \setcounter{enumi}{16}
\item $(3x + 1)(x + 6)$%$3{x}^{2}+17x-6$
\item $(7x + 1)(x - 1)$%$7{x}^{2}-6x-1$
\item $(4x - 1)(2x - 1)$%$8{x}^{2}-6x+1$
\item $(6x + 3)(x - 3)$%$6{x}^{2}-15x-9$
\end{enumerate}

\end{enumerate}}
\end{solutions}

% Ex 1-8 questions
\begin{exercises}{}{

Factorise the following:
\begin{multicols}{2}
\begin{enumerate}[itemsep=5pt, label=\textbf{\arabic*}. ] 
\item $6x+a+2ax+3$
\item ${x}^{2}-6x+5x-30$
\item $5x+10y-ax-2ay$
\item ${a}^{2}-2a-ax+2x$
\item $5xy-3y+10x-6$
\item $ab - a^{2} - a + b$
\end{enumerate}
\end{multicols}

}
\end{exercises}

% Ex 1-8 solutions
 \begin{solutions}{}{
\begin{enumerate}[noitemsep, label=\textbf{\arabic*}. ] 
\item $3(2x+1)+a(2x+1)=(3 + a)(2x + 1)$%$6x+a+2ax+3$
\item $x(x-6)+5(x-6)=(x + 5)(x - 6)$%${x}^{2}-6x+5x-30$
\item $5(x+2y)-a(x+2y)=(5 - a)(x + 2y)$%$5x+10y-ax-2ay$
\item $a(a-2)-x(a-2)=(a - x)(a - 2)$%${a}^{2}-2a-ax+2x$
\item $y(5x-3)+2(5x-3)=(y + 2)(5x - 3)$%$5xy-3y+10x-6$
\item $(-a + b)(a + 1)$%$ab - a^{2} - a + b$
\end{enumerate}}
\end{solutions}

% Ex 1-9 questions
\begin{exercises}{}
{
Factorise completely:
\begin{multicols}{2}
\begin{enumerate}[itemsep=5pt, label=\textbf{\arabic*}. ] 
\item ${x}^{3}+8$
\item $27-m^{3}$
\item $2x^{3}-2y^{3}$
\item $3k^{3} + 27q^{3}$
\item $64t^{3}-1$
\item $64x^{2} -1$
\item $125x^{3} +1$
\item $25x^{2} +1$
\item $z-125z^4{}$
\item $8m^{6} + n^{9}$
\end{enumerate}
\end{multicols}

}
\end{exercises}

% Ex 1-9 solutions
 \begin{solutions}{}{
\begin{enumerate}[noitemsep, label=\textbf{\arabic*}. ] 
\item $(x + 2)(x^2 - 2x + 4)$%${x}^{3}+8$
\item $(3 - m)(9 + 3m + m^2)$%$27-m^{3}$
\item $2(x^3-y^3)=2(x - y)(x^2 + xy + y^2)$%$2x^{3}-2y^{3}$
\item $3(k^3+27q^3)=3(k + 3q)(k^2 - 3kq + 9q^2)$%$3k^{3} + 27q^{3}$
\item $(4t - 1)(16t^2 + 4t + 1)$%$64t^{3}-1$
\item $(8x - 1)(8x + 1)$%$64x^{2} -1$
\item $(5x + 1)(25x^2 - 5x + 1)$%$125x^{3} +1$
\item $(5x + 1)(5x - 1)$%$25x^{2} +1$
\item $z(1-125z^3)=z(1 - 5z)(1 + 5z + 25z^2)$%$z-125z^4{}$
\item $(2m^2)^3 + (n^3)^3=(2m^2 + n^3)(4m^4 - 2m^2n^3 + n^6)$%$8m^{6} + n^{9}$
\end{enumerate}}
\end{solutions}


% Ex 1-10 questions
\begin{exercises}{}
{
Simplify (assume all denominators are non-zero):
\begin{multicols}{2}
\begin{enumerate}[itemsep=5pt, label=\textbf{\arabic*}. ] 
\item$\dfrac{3a}{15}$
\item $\dfrac{2a+10}{4}$
\item $\dfrac{5a+20}{a+4}$
\item $\dfrac{{a}^{2}-4a}{a-4}$
\item $\dfrac{3{a}^{2}-9a}{2a-6}$
\item $\dfrac{9a+27}{9a+18}$
\item $\dfrac{6ab+2a}{2b}$
\item $\dfrac{16{x}^{2}y-8xy}{12x-6}$
\item $\dfrac{4xyp-8xp}{12xy}$
\item $\dfrac{3a+9}{14}÷\dfrac{7a+21}{a+3}$
\item $\dfrac{{a}^{2}-5a}{2a+10} \times \dfrac{4a}{3a+15}$
\item $\dfrac{3xp+4p}{8p}÷\dfrac{12{p}^{2}}{3x+4}$
\item $\dfrac{24a-8}{12}÷\dfrac{9a-3}{6}$
\item $\dfrac{{a}^{2}+2a}{5}÷\dfrac{2a+4}{20}$
\item $\dfrac{{p}^{2}+pq}{7p} \times \dfrac{21q}{8p+8q}$
\item $\dfrac{5ab-15b}{4a-12}÷\dfrac{6{b}^{2}}{a+b}$
\item $\dfrac{{f}^{2}a-f{a}^{2}}{f-a}$
\item $\dfrac{2}{xy} + \dfrac{4}{xz}+\dfrac{3}{yz}$
\item $\dfrac{5}{t-2} - \dfrac{1}{t-3}$
\item $\dfrac{k+2}{k^{2} +2} - \dfrac{1}{k+2}$
\item $\dfrac{t+2}{3q} + \dfrac{t+1}{2q}$
\item $\dfrac{3}{p^{2}-4}+\dfrac{2}{(p-2)^{2}}$
\item $\dfrac{x}{x+y}+\dfrac{x^{2}}{y^{2} - x^{2}}$
\item $\dfrac{1}{m+n} + \dfrac{3mn}{m^{3} + n^{3}}$
\item $\dfrac{h}{h^{3}-f^{3}} - \dfrac{1}{h^{2} + hf + f^{2}}$
\item $\dfrac{{x}^{2}-1}{3}\times\dfrac{1}{x-1}-\dfrac{1}{2}$
\end{enumerate}
\end{multicols}

}
\end{exercises}

% Ex 1-10 solutions
 \begin{solutions}{}{
\begin{enumerate}[noitemsep, label=\textbf{\arabic*}. ] 
\item $\frac{a}{5}$%$\dfrac{3a}{15}$
\item $\frac{2(+5)}{4}=frac{a + 5}{2}$%$\dfrac{2a+10}{4}$
\item $5$%$\dfrac{5a+20}{a+4}$
\item $a$%$\dfrac{{a}^{2}-4a}{a-4}$
\item $\frac{3a(a-3)}{2(a-3)}=\frac{3a}{2}$%$\dfrac{3{a}^{2}-9a}{2a-6}$
\item $\frac{9(a+3)}{9(a+2)}=\frac{a + 3}{a + 2}$%$\dfrac{9a+27}{9a+18}$
\item $\frac{2a(3b+1)}{2b}=\frac{a(3b + 1)}{b}$%$\dfrac{6ab+2a}{2b}$
\item $\frac{4xy}{3}$%$\dfrac{16{x}^{2}y-8xy}{12x-6}$
\item $\frac{p(y - 2)}{3y}$%$\dfrac{4xyp-8xp}{12xy}$
\item $\frac{3(a + 3)}{98}$%$\dfrac{3a+9}{14}÷\dfrac{7a+21}{a+3}$
\item $\frac{4a^2(a - 5)}{6(a + 5)^2}$%$\dfrac{{a}^{2}-5a}{2a+10} \times \dfrac{4a}{3a+15}$
\item $\frac{(3x + 4)^2}{96p^2}$%$\dfrac{3xp+4p}{8p}÷\dfrac{12{p}^{2}}{3x+4}$
\item $\frac{4}{3}$%$\dfrac{24a-8}{12}÷\dfrac{9a-3}{6}$
\item $2a$%$\dfrac{{a}^{2}+2a}{5}÷\dfrac{2a+4}{20}$
\item $\frac{3q}{8}$%$\dfrac{{p}^{2}+pq}{7p} \times \dfrac{21q}{8p+8q}$
\item $\frac{5(a + b)}{24b}$%$\dfrac{5ab-15b}{4a-12}÷\dfrac{6{b}^{2}}{a+b}$
\item $\frac{af(f-a)}{f-a}=af$%$\dfrac{{f}^{2}a-f{a}^{2}}{f-a}$
\item $\ffrac{2z}{xyz} + \frac{4y}{xyz} + \frac{3x}{xyz}=\frac{2z + 4y + 3x}{xyz}$%$\dfrac{2}{xy} + \dfrac{4}{xz}+\dfrac{3}{yz}$
\item $\frac{5(t-3)}{(t-2)(t-3)}-\frac{t-2}{(t-2)(t-3)}= \frac{5(t-2)(t-3)}{(t-2)(t-3)}=5$%$\dfrac{5}{t-2} - \dfrac{1}{t-3}$
\item $\frac{(k+2)(k+2)-(k^2+2)}{(k^2+2)(k+2)} = \frac{k^2+4k+4-k^2-2}{k^3+2k^2+2k+4}=\frac{2(k - 1)}{(k^2 + 2)(k + 2)}$%$\dfrac{k+2}{k^{2} +2} - \dfrac{1}{k+2}$
\item $\frac{2(t+2)+3(t+1)}{6q}=\frac{5t + 7}{6q}$%$\dfrac{t+2}{3q} + \dfrac{t+1}{2q}$
\item $\frac{3(p-2)=2(p+2)}{(p-2)(p+2)(p-2)} =\frac{(p-2)+2(p+2)}{(p+2)(p^2 -4)}=\frac{(5p - 2)}{(p - 2)^2(p+ 2)}$%$\dfrac{3}{p^{2}-4}+\dfrac{2}{(p-2)^{2}}$
\item $\frac{x(x-y)-x^2}{(x+y)(x-y)}=\frac{-xy}{x^2 - y^2}$%$\dfrac{x}{x+y}+\dfrac{x^{2}}{y^{2} - x^{2}}$
\item $\frac{1}{m+n} + \frac{3mn}{(m+n)(m^2-mn+n^2)} = \frac{m^2-mn+n^2+3mn}{m^3-m^3}=\frac{m^2+2mn+n^2}{m^3-n^3}=\frac{m+1}{m^2 - mn + n^2}$%$\dfrac{1}{m+n} + \dfrac{3mn}{m^{3} + n^{3}}$
\item $\frac{h}{(h-f)(h^2+hf+f^2)}-\frac{1}{h^2+hf+f^2}=\frac{h-(h-f)}{h^3-f^3}=\frac{f}{h^3 - f^3}$%$\dfrac{h}{h^{3}-f^{3}} - \dfrac{1}{h^{2} + hf + f^{2}}$
\item $\frac{2x - 1}{6}$%$\dfrac{{x}^{2}-1}{3}\times\dfrac{1}{x-1}-\dfrac{1}{2}$
\end{enumerate}
}
\end{solutions}

% EOC exercise questions
\begin{eocexercises}{}
\begin{enumerate}[itemsep=5pt, label=\textbf{\arabic*}. ] 
\item If $a$ is an integer, $b$ is an integer and $c$ is irrational, which of the following are rational numbers?
    \begin{enumerate}[itemsep=5pt, label=\textbf{(\alph*)} ] 
    \item $\dfrac{-b}{a}$
    \item $c \div c$
    \item $\dfrac{a}{c}$
    \item $\dfrac{1}{c}$
    \end{enumerate}
\item Write each decimal as a simple fraction.
    \begin{enumerate}[itemsep=5pt, label=\textbf{(\alph*)} ] 
    \item $0,12$
    \item $0,006$
    \item $1,59$
    \item $12,27\dot{7}$
    \end{enumerate}
\item Show that the decimal $3,21\dot{1}\dot{8}$ is a rational number.
\item Express $0,7\dot{8}$ as a fraction $\dfrac{a}{b}$ where $a,b\in \mathbb{Z}$ (show all working).
\item Write the following rational numbers to $2$ decimal places.
    \begin{enumerate}[itemsep=5pt, label=\textbf{(\alph*)} ]  
    \item $\dfrac{1}{2}$
    \item $1$
    \item $0,11111\overline{1}$
    \item $0,99999\overline{1}$
    \end{enumerate}
\item Round off the following irrational numbers to $3$ decimal places.
\begin{multicols}{2}
    \begin{enumerate}[itemsep=5pt, label=\textbf{(\alph*)} ] 
    \item $3,141592654\ldots$
    \item $1,618033989\ldots$
    \item $1,41421356\ldots$
    \item $2,71828182845904523536\ldots$
    \end{enumerate}
\end{multicols}
\item Use your calculator and write the following irrational numbers to $3$ decimal places.
\begin{multicols}{2}
    \begin{enumerate}[itemsep=5pt, label=\textbf{(\alph*)} ] 
    \item $\sqrt{2}$
    \item $\sqrt{3}$
    \item $\sqrt{5}$
    \item $\sqrt{6}$
    \end{enumerate}
\end{multicols}
\item Use your calculator (where necessary) and write the following numbers to $5$ decimal places. State whether the numbers are irrational or rational.
\begin{multicols}{2}
    \begin{enumerate}[itemsep=5pt, label=\textbf{(\alph*)} ] 
    \item $\sqrt{8}$
    \item $\sqrt{768}$
    \item $\sqrt{0,49}$
    \item $\sqrt{0,0016}$
    \item $\sqrt{0,25}$
    \item $\sqrt{36}$
    \item $\sqrt{1960}$
    \item $\sqrt{0,0036}$
    \item $-8\sqrt{0,04}$
    \item $5\sqrt{80}$
    \end{enumerate}
\end{multicols}
\item Write the following irrational numbers to $3$ decimal places and then write each one as a rational number to get an approximation to the irrational number.
\begin{multicols}{2}
\begin{enumerate}[itemsep=5pt, label=\textbf{(\alph*)} ] 
    \item $3,141592654\ldots$
    \item $1,618033989\ldots$
    \item $1,41421356\ldots$
    \item $2,71828182845904523536\ldots$
    \end{enumerate}
\end{multicols}
\item Determine between which two consecutive integers the following irrational numbers lie, without using a calculator.
\begin{multicols}{2}
    \begin{enumerate}[itemsep=5pt, label=\textbf{(\alph*)} ] 
    \item $\sqrt{5}$ 
    \item $\sqrt{10}$ 
    \item $\sqrt{20}$ 
    \item $\sqrt{30}$ 
    \item $\sqrt[3]{5}$ 
    \item $\sqrt[3]{10}$ 
    \item $\sqrt[3]{20}$ 
    \item $\sqrt[3]{30}$ 
    \end{enumerate}
\end{multicols}
\item  Find two consecutive integers such that $\sqrt{7}$ lies between them.          
\item  Find two consecutive integers such that $\sqrt{15}$ lies between them.          
\item Factorise:
\begin{multicols}{2}
\begin{enumerate}[itemsep=5pt, label=\textbf{(\alph*)} ] 
\item ${a}^{2}-9$
\item ${m}^{2}-36$
\item $9{b}^{2}-81$
\item $16{b}^{6}-25{a}^{2}$
\item ${m}^{2}-\frac{1}{9}$
\item $5-5{a}^{2}{b}^{6}$
\item $16b{a}^{4}-81b$
\item ${a}^{2}-10a+25$
\item $16{b}^{2}+56b+49$
\item $2{a}^{2}-12ab+18{b}^{2}$
\item $-4{b}^{2}-144{b}^{8}+48{b}^{5}$
\item $(16-{x}^{4})$
\item ${7x}^{2}-14x+7xy-14y$
\item ${y}^{2}-7y-30$
\item $1-x-{x}^{2}+{x}^{3}$
\item $-3(1-{p}^{2})+p+1$
\item $x-x^{3} + y - y^{3}$
\item $x^{2} - 2x + 1 - y^{4}$
\item $4b(x^{3} - 1) + x(1-x^{3})$
\item $3p^{3} - \frac{1}{9}$
\end{enumerate}
\end{multicols}
\item Simplify the following:
\begin{multicols}{2}
\begin{enumerate}[itemsep=5pt, label=\textbf{(\alph*)} ] 
\item ${(a-2)}^{2}-a(a+4)$
\item $(5a-4b)(25{a}^{2}+20ab+16{b}^{2})$
\item $(2m-3)(4{m}^{2}+9)(2m+3)$
\item $(a+2b-c)(a+2b+c)$
\item $\dfrac{{p}^{2}-{q}^{2}}{p}÷\dfrac{p+q}{{p}^{2}-pq}$
\item $\dfrac{2}{x}+\dfrac{x}{2}-\dfrac{2x}{3}$
\item $\dfrac{1}{a+7}-\dfrac{a+7}{a^{2}-49}$
\item $\dfrac{x+2}{2x^{3}} + 16$
\item $\dfrac{1-2a}{4a^{2} -1} - \dfrac{a-1}{2a^{2}-3a+1} - \dfrac{1}{1-a}$
\item $\dfrac{x^{2} + 2x}{x^{2}+ x + 6} \times \dfrac{x^{2} + 2x + 1}{x^{2} + 3x +2}$
\end{enumerate}
\end{multicols}
\item Show that ${(2x-1)}^{2}-{(x-3)}^{2}$ can be simplified to $(x+2)(3x-4)$.
\item What must be added to ${x}^{2}-x+4$ to make it equal to ${(x+2)}^{2}$ ?
\item Evaluate $\dfrac{x^{3}+1}{x^{2}-x+1}$ if $x=7,85$ without using a calculator. Show your work.
\end{enumerate}

\end{eocexercises}

% EOC solutions
 \begin{solutions}{}{
\begin{enumerate}[noitemsep, label=\textbf{\arabic*}. ] 
\item% If $a$ is an integer, $b$ is an integer and $c$ is irrational, which of the following are rational numbers?
    \begin{enumerate}[noitemsep, label=\textbf{(\alph*)} ] 
    \item Rational%$\dfrac{-b}{a}$
    \item Irrational%$c \div c$
    \item Irrational%$\dfrac{a}{c}$
    \item Irrational%$\dfrac{1}{c}$
    \end{enumerate}

%Question 2
\item %% Write each decimal as a simple fraction:
    \begin{enumerate}[noitemsep, label=\textbf{(\alph*)} ] 
    \item $\frac{1}{10}+\frac{2}{100} = \frac{12}{100} = \frac{6}{10}\frac{3}{5}$%$0,12$
    \item $\frac{6}{1000}=\frac{3}{500}$%$0,006$
    \item $1+\frac{5}{10} + \frac{9}{100}=1\frac{59}{100}$%$1,59$
    \item $x=12,2\dot{7}$, $10x=122,\dot{7}$, and $100x=1227,\dot{7}$, \\$\therefore 100x-10x=90x=1105$, \\$\therefore x=\frac{1105}{90}=\frac{221}{18}$%$12,27\dot{7}$
    \end{enumerate}

%Question 3
 \item $x=3,21\overline{18}$ and $10~000x=32118,\overline{18}$,\\ $\therefore 10~000x-x=9~999x=32115$, \\$\therefore x=\frac{32~115}{9~999}$.\\This is a rational number because both the numerator and denominator are integers.

  %%Show that the decimal $3,21\dot{1}\dot{8}$ is a rational number.

% Question 4
% \setcounter{enumi}{3}
\item $\frac{71}{90}$%%Express $0,7\dot{8}$ as a fraction $\dfrac{a}{b}$ where $a,b\in \mathbb{Z}$ (show all working).
$x=0,\overline{78}$ and $100x=78,\overline{78}$\\ $\therefore 100x-x=99x=\frac{78}{99}$
%Question 5
\item %%Write the following rational numbers to $2$ decimal places:
    \begin{enumerate}[noitemsep, label=\textbf{(\alph*)} ]  
    \item  To write to two decimal places we must convert to decimal: $\frac{1}{2}=0,5$%$\dfrac{1}{2}$
    \item To write to two decimal places just add a comma and two $0$'s: $1,00$%$1$
    \item We mark where the cut off point is, determine if it has to be rounded up or not and then write the answer. In this case there is a $1$ 
after the cut off point so we do not round up. The final answer is: $0,11111\overline{1} \sim 0,11$%$0,11111\overline{1}$
    \item Repeat the steps in c) but this time we round up: $0,99999\overline{1} \sim 1,00$%$0,99999\overline{1}$
    \end{enumerate}

%Question 6
\item %%Round off the following irrational numbers to $3$ decimal places:
We mark where the cut off point is, determine if it has to be rounded up or not and then write the answer. 
    \begin{enumerate}[noitemsep, label=\textbf{(\alph*)} ] 
\item $3,142$ (round up as there is a 5 after the cut off point) 
\item $1,618$ (no rounding as there is a 0 after the cut off point) 
\item $1,414$ (no rounding as there is a 2 after the cut off point) 
\item $2,718$ (round up as there is a 2 after the cut off point) 
    \end{enumerate}

%Question 7
\item %%Use your calculator and write the following irrational numbers to $4$ decimal places:
% 
    \begin{enumerate}[noitemsep, label=\textbf{(\alph*)} ] 
    \item $1,414$ %$\sqrt{2}$
    \item $1,732$%$\sqrt{3}$
    \item $2,236$%%$\sqrt{5}$
    \item $2,449$%$\sqrt{6}$
    \end{enumerate}
% 
% 
\item %Use your calculator (where necessary) and write the following numbers to $5$ decimal places. State whether the numbers are irrational or rational.

    \begin{enumerate}[noitemsep, label=\textbf{(\alph*)} ] 
    \item $2,82843$%$\sqrt{8}$
    \item $27,71281$ - irrational %$\sqrt{768}$
    \item $10,00000$ - rational 
    \item $0,70000$ - rational  %$\sqrt{0,49}$
    \item $0,04000$ - rational %%$\sqrt{0,0016}$
    \item $0,500000$ - rational  %$\sqrt{0,25}$
    \item $6,00000$ - rational  %$\sqrt{36}$
    \item $44,27189$ - irrational  %$\sqrt{1960}$
    \item $0,06000$ - rational n %$\sqrt{0,0036}$
    \item $-8(0,2) = -4,00000$ - rational  %$-8\sqrt{0,04}$
    \item $44,72136$ - irrational %$5\sqrt{80}$
    \end{enumerate}
% \setcounter{enumi}{8}
\item %Write the following irrational numbers to $3$ decimal places and then write each one as a rational number to get an approximation to the irrational number.

\begin{enumerate}[noitemsep, label=\textbf{(\alph*)} ] 
\item $3 \frac{142}{1~000} =\frac{1~571}{500}$
\item $1\frac{618}{1000}=\frac{809}{500}$
\item $1\frac{414}{1000}\frac{707}{500}$
\item $2\frac{718}{1000}=\frac{1~359}{500}$
    \end{enumerate}


\item %Determine between which two consecutive integers the following irrational numbers lie, without using a calculator:

    \begin{enumerate}[noitemsep, label=\textbf{(\alph*)} ] 
\item $2$ and $3$
\item $3$ and $4$
\item $4$ and $5$
\item $5$ and $6$
\item $1$ and $2$
\item $2$ and $3$
\item $2$ and $3$
\item $3$ and $4$
    \end{enumerate}


\item $2$ and $3$% Find two consecutive integers such that $\sqrt{7}$ lies between them.          
\item $3$ and $4$% Find two consecutive integers such that $\sqrt{15}$ lies between them.          

\item %%Factorise:

\begin{enumerate}[noitemsep, label=\textbf{(\alph*)} ] 
\item $(a - 3)(a + 3)$%%${a}^{2}-9$
\item $(m + 6)(m - 6)$%%${m}^{2}-36$
\item $(3b - 9)(3b + 9)$%%$9{b}^{2}-81$
\item $(4b + 5a)(4b - 5a)$%%$16{b}^{6}-25{a}^{2}$
\item $(m +\frac{1}{3})(m -\frac{1}{3})$%%${m}^{2}-\frac{1}{9}$
\item $5(1-a^2b^6)=5(1 - ab^3)(1 + ab^3)$%%$5-5{a}^{2}{b}^{6}$
\item $b(4a^2+0)(4a^2+9)=b(4a^2 - 9)(2a + 3)(2a - 3)$%%$16b{a}^{4}-81b$
\item $(a - 5)(a - 5)$%%${a}^{2}-10a+25$
\item $(4b + 7)(4b + 7)$%%$16{b}^{2}+56b+49$
\item $(2a - 6b)(a - 3b)$%%$2{a}^{2}-12ab+18{b}^{2}$
\item $-4b^2(6b^3 - 1)(6b^3 - 1)$%%$-4{b}^{2}-144{b}^{8}+48{b}^{5}$
\item $(4-x^2)(4+x^2)=(2 - x)(2 + x)(4 + x^2)$%%$(16-{x}^{4})$
\item $7x(x-2)+7y(x-2)=(x-2)(7x+7y)=7(x - 2)(x + y)$%%${7x}^{2}-14x+7xy-14y$
\item $(y - 10)(y + 3)$%%${y}^{2}-7y-30$
\item $(1-x)-x^2(1-x)=(1-x)(1-x^2)=(1 - x)^{2}(1 +x)$%%$1-x-{x}^{2}+{x}^{3}$
\item $-3(1-p)(1+p)+(p+1)=(p+1)[-3(1-p)+1]=(p - 1)(-3p-2)$%%$-3(1-{p}^{2})+p+1$
\item $x(1-x^2)+y(1-y^2)=x(1-x)(1+x)+y(1-y)(1+y)=(x+y)(1-x^2+xy-y^2)$
\item $x(x-2)+(1-y^2)(1+y^2)=x(x-2)+(1+y)(1-y)(1+y^2)=(x-1-y^2)(x-1+y^2)$
\item $(x^3-1)(4b-x)=(x-1)(x^2+x+1)(4b-x)=4b(x^3-1)(1-x)$
\item $3(p-\frac{1}{3})(p^2+\frac{p}{3}+\frac{1}{9})$
\end{enumerate}


\item %Simplify the following:

\begin{enumerate}[noitemsep, label=\textbf{(\alph*)} ] 

\item$a^2-4a+4-a^2-4a=-8a + 4$%${(a-2)}^{2}-a(a+4)$
\item $125a^3 - 64b^3$ %$(5a-4b)(25{a}^{2}+20ab+16{b}^{2})$
\item $(4m^2-9)(4m^2+9)=16m^4 - 81$%$(2m-3)(4{m}^{2}+9)(2m+3)$
\item $(a+2b)^2-c^2=a^2 + 4ab + 4b^2 - c^2$%$(a+2b-c)(a+2b+c)$
\item $\frac{(p-q)(p+q)}{p} \times \frac{p(p-q)}{p+q}= (piq)^2=p^2 - 2pq +q^2$%$\dfrac{{p}^{2}-{q}^{2}}{p}÷\dfrac{p+q}{{p}^{2}-pq}$
\item $\frac{12+3x^2-4x^2}{6x}=\frac{12 - x^2}{6x}$%$\dfrac{2}{x}+\dfrac{x}{2}-\dfrac{2x}{3}$
\item $\frac{1}{a+7}-\frac{a+7}{(a+7)(a-7)= \frac{-14}{(a+7)(a-7)}$
\item $\frac{(x+2)+16(2x^3)}{2x^3}=\frac{32x^3+x+2}{2x^3}$
\item $\frac{1-2a}{(2a-1)(2a+1)} - \frac{a-1}{(2a-1)(2a+1)} + \frac{1}{a-1}= -\frac{(2a-1)}{(2a-1)(2a+1)}-\frac{1}{2a-1}+\frac{1}{a-1}=\frac{-(2a-1)(a-1)-(2a+1)(a-1)+(2a+1)(2a-1)}{(2a+1)(2a-1)(a-1)}=\frac{4a-1}{(2a+1)(2a-1)(a-1)}$
\item $\frac{x(x+2)}{x^2+x+6} \times \frac{(x+1)(x+1)}{(x+2)(x+1)}=\frac{x(x+1)}{x^2+x+6}$
\end{enumerate}


\item $(2x-1)(2x-1)-(x-3)(x-3)= 4x^2-2x-2x+1-(x^2-3x-3x-9)=(4-1)x^2+(-4+6)x+(1-9)=3x^2 +2x-8=(3x - 4)(x + 2)$%Show that ${(2x-1)}^{2}-{(x-3)}^{2}$ can be simplified to $(x+2)(3x-4)$
\item Suppose $A$ must be added to the expression to get the desired result. $\therefore (x^2-x+4) +A = (x+2)^2\\ 
\therefore A = (x+2)(x+2)-(x^2-x+4)=x^2+2x+2x+4-x^2+x-4=(1-1)x^2+(2+2+1)x +(4-4) =5x$.\\
Therefore $5x$ must be added. %What must be added to ${x}^{2}-x+4$ to makeit equal to ${(x+2)}^{2}$ ?
\item First simplify the expression: $\frac{(x+1)(x^2-x+1)}{x^2-x+1} = x+1$.\\
Now substitute the value of $x$: $7,85+1=8,85$ %Evaluate $\dfrac{x^{3}+1}{x^{2}-x+1}$ if $x=7,85$ without using a calculator. Show your working.
\end{enumerate}

\end{enumerate}
}
\end{solutions}


