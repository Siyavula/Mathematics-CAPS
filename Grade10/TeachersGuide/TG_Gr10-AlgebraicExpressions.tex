\chapter{Algebraic expressions}
\section{Rational and irrational numbers}
\subsection{Converting terminating decimals into rational numbers}
\subsection{Converting recurring decimals into rational numbers}
\subsection{Irrational numbers}
\begin{exercises}{}{
\begin{enumerate}[itemsep=5pt, label=\textbf{\arabic*}. ] 
\item If $a$ is an integer, $b$ is an integer and $c$ is irrational, which of the following are rational numbers? 
  \begin{enumerate}[itemsep=5pt, label=\textbf{(\alph*)} ] 
    \item $\dfrac{5}{6}$
    \item $\dfrac{a}{3}$
    \item $\dfrac{-2}{b}$
    \item $\dfrac{1}{c}$
    \end{enumerate}
\item If $\dfrac{a}{1}$ is a rational number, which of the following are valid values for $a$?
    \begin{enumerate}[itemsep=5pt, label=\textbf{(\alph*)} ] 
    \item $1$
    \item $-10$
    \item $\sqrt{2}$
    \item $2,1$
    \end{enumerate}
\item Write the following as fractions:
    \begin{enumerate}[itemsep=5pt, label=\textbf{(\alph*)} ] 
    \item $0,1$
    \item $0,12$
    \item $0,58$
    \item $0,2589$
    \end{enumerate}
\item Write the following using the recurring decimal notation:
    \begin{enumerate}[itemsep=5pt, label=\textbf{(\alph*)} ] 
    \item $0,11111111\ldots$
    \item $0,1212121212\ldots$
    \item $0,123123123123\ldots$
    \item $0,11414541454145\ldots$
    \end{enumerate}
\item Write the following in decimal form, using the recurring decimal notation:
    \begin{enumerate}[itemsep=5pt, label=\textbf{(\alph*)} ] 
    \item $\dfrac{2}{3}$
    \item $1\dfrac{3}{11}$
    \item $4\dfrac{5}{6}$
    \item $2\dfrac{1}{9}$
    \end{enumerate}
\item Write the following decimals in fractional form:
    \begin{enumerate}[itemsep=5pt, label=\textbf{(\alph*)} ] 
    \item $0,\dot{5}$
    \item $0,6\dot{3}$
    \item $5,\overline{31}$
    \end{enumerate}
\end{enumerate}
\practiceinfo 
\par 
 \par \begin{tabular}[h]{cccccc}
 (1.) lD1  &  (2.) l3N  &  (3.) l3R  & (4.) l3U & (5.) l3n & (6.) lDr \end{tabular}
}
\end{exercises}


 \begin{solutions}{}{
\begin{enumerate}[itemsep=5pt, label=\textbf{\arabic*}. ] 


\item solution 1
\item solution 2
\item solution 3
\item solution 4
\item solution 5
\item solution 6

\end{enumerate}}
\end{solutions}


\section{Rounding off}
\begin{exercises}{}
{
Use your calculator to write the following in decimal form, rounded to $3$ decimal places:
\begin{enumerate}[itemsep=5pt, label=\textbf{\arabic*}. ]
 \item $4\pi$
\item $\sqrt{11}$
\item $\dfrac{0,8}{3}$
\item $\sqrt[3]{7}$
\item $2\sqrt{10}$
\item $\dfrac{1}{18}$
\end{enumerate}
\practiceinfo 
\par 
 \par \begin{tabular}[h]{c}
 (1.-6.) lDY \end{tabular}
}
\end{exercises}


 \begin{solutions}{}{
\begin{enumerate}[itemsep=5pt, label=\textbf{\arabic*}. ] 


\item solution 1
\item solution 2
\item solution 3
\item solution 4
\item solution 5
\item solution 6

\end{enumerate}}
\end{solutions}


\section{Estimating surds}
\begin{exercises}{}
 {
Determine between which two consecutive integers the following numbers lie, without using a calculator:
\begin{enumerate}[itemsep=5pt, label=\textbf{\arabic*}. ]
\item $\sqrt{18}$
\item $\sqrt{29}$
\item $\sqrt[3]{5}$
\item $\sqrt[3]{79}$
\end{enumerate}
\practiceinfo 
\par 
 \par \begin{tabular}[h]{c}
 (1.-4.) lDg \end{tabular}
}
\end{exercises}


 \begin{solutions}{}{
\begin{enumerate}[itemsep=5pt, label=\textbf{\arabic*}. ] 


\item solution 1
\item solution 2
\item solution 3
\item solution 4

\end{enumerate}}
\end{solutions}


\section{Products}
\begin{exercises}{}
{
Expand the following products:
\begin{multicols}{2}
\begin{enumerate}[label=\textbf{\arabic*}., itemsep=5pt]
\item $2y(y+4)$ 
\item $(y+5)(y+2) $
\item $(2-t)(1-2t)$
\item $(x-4)(x+4)$
\item $ (2p+9)(3p+1)$
\item $(3k-2)(k+6)$
\item $(s+6)^2$
\item $-(7-x)(7+x)$
\item $(3x-1)(3x+1)$
\item $(7k+2)(3-2k)$
\item $(1-4x)^2$
\item $(-3-y)(5-y)$
\item $(8-x)(8+x)$
\item $(9+x)^2$
\item $(-2{y}^{2}-4y+11)(5y-12)$ 
\item $(7{y}^{2}-6y-8)(-2y+2)$% make-rowspan-placeholders
\item $(10{y}+3)(-2{y}^{2}-11y+2)$ 
\item $(-12y-3)(12{y}^{2}-11y+3)$% make-rowspan-placeholders
\item $(-10)(2{y}^{2}+8y+3)$ 
\item $(2{y}^{6}+3{y}^{5})(-5y-12)$% make-rowspan-placeholders
\item $(-7y+11)(-12y+3)$% make-rowspan-placeholders
\item $(7y+3)(7{y}^{2}+3y+10)$% make-rowspan-placeholders
\item $(9)(8{y}^{2}-2y+3)$ 
\item $(-6{y}^{4}+11{y}^{2}+3y)(y+4)(y-4)$ 
\end{enumerate}
\end{multicols}
\practiceinfo 
\par 
 \par \begin{tabular}[h]{cccccc}
 (1.-6.) lD4  &  (7.-12.) lD2  &  (13.-18.) lDT & (19.-24.) lDb   \end{tabular}
}
\end{exercises}


 \begin{solutions}{}{
\begin{enumerate}[itemsep=5pt, label=\textbf{\arabic*}. ] 


\item solution 1
\item solution 2
\item solution 3
\item solution 4
\item solution 5
\item solution 6
\item solution 7
\item solution 8
\item solution 9
\item solution 10
\item solution 11
\item solution 12
\item solution 13
\item solution 14
\item solution 15
\item solution 16
\item solution 17
\item solution 18
\item solution 19
\item solution 20
\item solution 21
\item solution 22
\item solution 23
\item solution 24

\end{enumerate}}
\end{solutions}


\section{Factorisation}
\begin{exercises}{}
{
Find the highest common factors of the
following pairs of terms:\par
\begin{multicols}{2}
\begin{enumerate}[label=\textbf{\arabic*}., itemsep=5pt]
\item $6y;~18x$
\item $12mn;~8n$
\item $3st;~4su$ 
\item $18kl;~9kp$
\item $abc;~ac$% 
\item $2xy;~4xyz$
\item $3uv;~6u$ 
\item $9xy;~15xz$
\item $24xyz;~16yz$
\item $3m;~45n$
\end{enumerate}
\end{multicols}
\practiceinfo 
\par 
 \par \begin{tabular}[h]{cccccc}
 (1.-5.) lDj  &  (6.-10.) lDD  \end{tabular}
}
\end{exercises}


 \begin{solutions}{}{
\begin{enumerate}[itemsep=5pt, label=\textbf{\arabic*}. ] 


\item solution 1
\item solution 2
\item solution 3
\item solution 4
\item solution 5
\item solution 6
\item solution 7
\item solution 8
\item solution 9
\item solution 10

\end{enumerate}}
\end{solutions}


\begin{exercises}{}{
Factorise:
\begin{multicols}{2}
\begin{enumerate}[itemsep=5pt, label=\textbf{\arabic*}. ] 
\item $2l+2w$
\item $12x+32y$
\item $6{x}^{2}+2x+10{x}^{3}$
\item $2x{y}^{2}+x{y}^{2}z+3xy$
\item $-2a{b}^{2}-4{a}^{2}b$
\item $7a+4$ 
\item $20a-10$ 
\item $18ab-3bc$
\item $12kj+18kq$ 
\item $16{k}^{2}-4$ 
\item $3{a}^{2}+6a-18$
\item $-12a+24a^3$ 
\item $-2ab-8a$ 
\item $24kj-16{k}^{2}j$
\item $-{a}^{2}b-{b}^{2}a$ 
\item $12{k}^{2}j+24{k}^{2}{j}^{2}$ 
\item $72{b}^{2}q-18{b}^{3}{q}^{2}$
\item $4(y-3)+k(3-y)$ 
\item $a^2(a-1)-25(a-1)$ 
\item $bm(b+4)-6m(b+4)$
\item ${a}^{2}(a+7)+9(a+7)$ 
\item $3b(b-4)-7(4-b)$ 
\item ${a}^{2}{b}^{2}{c}^{2}-1$
\end{enumerate}
\end{multicols}
\practiceinfo 
\par 
 \par \begin{tabular}[h]{cccccc}
 (1.7.) lDW  &  (7.-12.) lDZ  & (13.-18.) lDB  &  (19.-23.) lDK    \end{tabular}
}
\end{exercises}


 \begin{solutions}{}{
\begin{enumerate}[itemsep=5pt, label=\textbf{\arabic*}. ] 


\item solution 1
\item solution 2
\item solution 3
\item solution 4
\item solution 5
\item solution 6
\item solution 7
\item solution 8
\item solution 9
\item solution 10
\item solution 11
\item solution 12
\item solution 13
\item solution 14
\item solution 15
\item solution 16
\item solution 17
\item solution 18
\item solution 19
\item solution 20
\item solution 21
\item solution 22
\item solution 23

\end{enumerate}}
\end{solutions}


\begin{exercises}{}
{
\begin{enumerate}[itemsep=5pt, label=\textbf{\arabic*}. ] 
\item Factorise the following:
\begin{multicols}{2}
\begin{enumerate}[itemsep=5pt, label=\textbf{(\alph*)} ] 
\item ${x}^{2}+8x+15$
\item ${x}^{2}+10x+24$
\item ${x}^{2}+9x+8$
\item ${x}^{2}+9x+14$
\item ${x}^{2}+15x+36$
\item ${x}^{2}+12x+36$
\end{enumerate}
\end{multicols}
\item Write the following expressions in factorised form:
\begin{multicols}{2}
\begin{enumerate}[itemsep=5pt, label=\textbf{(\alph*)} ]  
\item ${x}^{2}-2x-15$
\item ${x}^{2}+2x-3$
\item ${x}^{2}+2x-8$
\item ${x}^{2}+x-20$
\item ${x}^{2}-x-20$
\item $2{x}^{2}+22x+20$
\end{enumerate}
\end{multicols}
\item Find the factors of the following trinomial expressions:
\begin{multicols}{2}
\begin{enumerate}[itemsep=5pt, label=\textbf{(\alph*)} ] 
\item $3{x}^{2}+19x+6$
\item $6{x}^{2}+7x+1$
\item $12{x}^{2}+8x+1$
\item $8{x}^{2}+6x+1$
\end{enumerate}
\end{multicols}
\item Factorise completely:
\begin{multicols}{2}
\begin{enumerate}[itemsep=5pt, label=\textbf{(\alph*)} ] 
\item $3{x}^{2}+17x-6$
\item $7{x}^{2}-6x-1$
\item $8{x}^{2}-6x+1$
\item $6{x}^{2}-15x-9$
\end{enumerate}
\end{multicols}
\end{enumerate}
\practiceinfo 
\par 
 \par \begin{tabular}[h]{cccccc}
 (1.) liY  &  (2.) lDk  &  (3.) lD0  &  (4.) lD8  & \end{tabular}
}
\end{exercises} 


 \begin{solutions}{}{
\begin{enumerate}[itemsep=5pt, label=\textbf{\arabic*}. ] 


\item solution 1
\item solution 2
\item solution 3
\item solution 4

\end{enumerate}}
\end{solutions}


\subsection{Factorising by grouping}
\begin{exercises}{}{
\nopagebreak
Factorise the following:
\begin{multicols}{2}
\begin{enumerate}[itemsep=5pt, label=\textbf{\arabic*}. ] 
\item $6x+a+2ax+3$
\item ${x}^{2}-6x+5x-30$
\item $5x+10y-ax-2ay$
\item ${a}^{2}-2a-ax+2x$
\item $5xy-3y+10x-6$
\item $ab - a^{2} - a + b$
\end{enumerate}
\end{multicols}
\practiceinfo 
\par 
\par \begin{tabular}[h]{cccccc}
(1.) lih  &  (2.) liS  &  (3.) liJ  &  (4.) liu  &  (5.) liz  & (6.) lD9 \end{tabular}
}
\end{exercises}


 \begin{solutions}{}{
\begin{enumerate}[itemsep=5pt, label=\textbf{\arabic*}. ] 


\item solution 1
\item solution 2
\item solution 3
\item solution 4
\item solution 5
\item solution 6

\end{enumerate}}
\end{solutions}


\begin{exercises}{}
{
Factorise completely:
\begin{multicols}{2}
\begin{enumerate}[itemsep=5pt, label=\textbf{\arabic*}. ] 
\item ${x}^{3}+8$
\item $27-m^{3}$
\item $2x^{3}-2y^{3}$
\item $3k^{3} + 27q^{3}$
\item $64t^{3}-1$
\item $64x^{2} -1$
\item $125x^{3} +1$
\item $25x^{2} +1$
\item $z-125z^4{}$
\item $8m^{6} + n^{9}$
\end{enumerate}
\end{multicols}
\practiceinfo 
\par 
 \par \begin{tabular}[h]{cccccc}
 (1.-5.) lDX   & (6.-10.) lD5 &  \end{tabular}
}
\end{exercises}


 \begin{solutions}{}{
\begin{enumerate}[itemsep=5pt, label=\textbf{\arabic*}. ] 


\item solution 1
\item solution 2
\item solution 3
\item solution 4
\item solution 5
\item solution 6
\item solution 7
\item solution 8
\item solution 9
\item solution 10

\end{enumerate}}
\end{solutions}


\section{Simplification of fractions}
\begin{exercises}{}
{
Simplify (assume all denominators are non-zero):
\begin{multicols}{2}
\begin{enumerate}[itemsep=5pt, label=\textbf{\arabic*}. ] 
\item$\dfrac{3a}{15}$
\item $\dfrac{2a+10}{4}$
\item $\dfrac{5a+20}{a+4}$
\item $\dfrac{{a}^{2}-4a}{a-4}$
\item $\dfrac{3{a}^{2}-9a}{2a-6}$
\item $\dfrac{9a+27}{9a+18}$
\item $\dfrac{6ab+2a}{2b}$
\item $\dfrac{16{x}^{2}y-8xy}{12x-6}$
\item $\dfrac{4xyp-8xp}{12xy}$
\item $\dfrac{3a+9}{14}÷\dfrac{7a+21}{a+3}$
\item $\dfrac{{a}^{2}-5a}{2a+10} \times \dfrac{4a}{3a+15}$
\item $\dfrac{3xp+4p}{8p}÷\dfrac{12{p}^{2}}{3x+4}$
\item $\dfrac{24a-8}{12}÷\dfrac{9a-3}{6}$
\item $\dfrac{{a}^{2}+2a}{5}÷\dfrac{2a+4}{20}$
\item $\dfrac{{p}^{2}+pq}{7p} \times \dfrac{21q}{8p+8q}$
\item $\dfrac{5ab-15b}{4a-12}÷\dfrac{6{b}^{2}}{a+b}$
\item $\dfrac{{f}^{2}a-f{a}^{2}}{f-a}$
\item $\dfrac{2}{xy} + \dfrac{4}{xz}+\dfrac{3}{yz}$
\item $\dfrac{5}{t-2} - \dfrac{1}{t-3}$
\item $\dfrac{k+2}{k^{2} +2} - \dfrac{1}{k+2}$
\item $\dfrac{t+2}{3q} + \dfrac{t+1}{2q}$
\item $\dfrac{3}{p^{2}-4}+\dfrac{2}{(p-2)^{2}}$
\item $\dfrac{x}{x+y}+\dfrac{x^{2}}{y^{2} - x^{2}}$
\item $\dfrac{1}{m+n} + \dfrac{3mn}{m^{3} + n^{3}}$
\item $\dfrac{h}{h^{3}-f^{3}} - \dfrac{1}{h^{2} + hf + f^{2}}$
\item $\dfrac{{x}^{2}-1}{3}\times\dfrac{1}{x-1}-\dfrac{1}{2}$
\end{enumerate}
\end{multicols}
\practiceinfo 
\par 
\par \begin{tabular}[h]{cccc}
  (1.-7.) lDR  &  (8.-13.) lDN   &  (14.-20.) lDn  &  (21.-26) lDQ \end{tabular}
}
\end{exercises}


 \begin{solutions}{}{
\begin{enumerate}[itemsep=5pt, label=\textbf{\arabic*}. ] 


\item solution 1
\item solution 2
\item solution 3
\item solution 4
\item solution 5
\item solution 6
\item solution 7
\item solution 8
\item solution 9
\item solution 10
\item solution 11
\item solution 12
\item solution 13
\item solution 14
\item solution 15
\item solution 16
\item solution 17
\item solution 18
\item solution 19
\item solution 20
\item solution 21
\item solution 22
\item solution 23
\item solution 24
\item solution 25
\item solution 26

\end{enumerate}}
\end{solutions}


\begin{eocexercises}{}
\begin{enumerate}[itemsep=5pt, label=\textbf{\arabic*}. ] 
\item If $a$ is an integer, $b$ is an integer and $c$ is irrational, which of the following are rational numbers?
    \begin{enumerate}[itemsep=5pt, label=\textbf{(\alph*)} ] 
    \item $\dfrac{-b}{a}$
    \item $c \div c$
    \item $\dfrac{a}{c}$
    \item $\dfrac{1}{c}$
    \end{enumerate}
\item Write each decimal as a simple fraction.
    \begin{enumerate}[itemsep=5pt, label=\textbf{(\alph*)} ] 
    \item $0,12$
    \item $0,006$
    \item $1,59$
    \item $12,27\dot{7}$
    \end{enumerate}
\item Show that the decimal $3,21\dot{1}\dot{8}$ is a rational number.
\item Express $0,7\dot{8}$ as a fraction $\dfrac{a}{b}$ where $a,b\in \mathbb{Z}$ (show all working).
\item Write the following rational numbers to $2$ decimal places.
    \begin{enumerate}[itemsep=5pt, label=\textbf{(\alph*)} ]  
    \item $\dfrac{1}{2}$
    \item $1$
    \item $0,11111\overline{1}$
    \item $0,99999\overline{1}$
    \end{enumerate}
\item Round off the following irrational numbers to $3$ decimal places.
\begin{multicols}{2}
    \begin{enumerate}[itemsep=5pt, label=\textbf{(\alph*)} ] 
    \item $3,141592654\ldots$
    \item $1,618033989\ldots$
    \item $1,41421356\ldots$
    \item $2,71828182845904523536\ldots$
    \end{enumerate}
\end{multicols}
\item Use your calculator and write the following irrational numbers to $3$ decimal places.
\begin{multicols}{2}
    \begin{enumerate}[itemsep=5pt, label=\textbf{(\alph*)} ] 
    \item $\sqrt{2}$
    \item $\sqrt{3}$
    \item $\sqrt{5}$
    \item $\sqrt{6}$
    \end{enumerate}
\end{multicols}
\item Use your calculator (where necessary) and write the following numbers to $5$ decimal places. State whether the numbers are irrational or rational.
\begin{multicols}{2}
    \begin{enumerate}[itemsep=5pt, label=\textbf{(\alph*)} ] 
    \item $\sqrt{8}$
    \item $\sqrt{768}$
    \item $\sqrt{0,49}$
    \item $\sqrt{0,0016}$
    \item $\sqrt{0,25}$
    \item $\sqrt{36}$
    \item $\sqrt{1960}$
    \item $\sqrt{0,0036}$
    \item $-8\sqrt{0,04}$
    \item $5\sqrt{80}$
    \end{enumerate}
\end{multicols}
\item Write the following irrational numbers to $3$ decimal places and then write each one as a rational number to get an approximation to the irrational number.
\begin{multicols}{2}
\begin{enumerate}[itemsep=5pt, label=\textbf{(\alph*)} ] 
    \item $3,141592654\ldots$
    \item $1,618033989\ldots$
    \item $1,41421356\ldots$
    \item $2,71828182845904523536\ldots$
    \end{enumerate}
\end{multicols}
\item Determine between which two consecutive integers the following irrational numbers lie, without using a calculator.
\begin{multicols}{2}
    \begin{enumerate}[itemsep=5pt, label=\textbf{(\alph*)} ] 
    \item $\sqrt{5}$ 
    \item $\sqrt{10}$ 
    \item $\sqrt{20}$ 
    \item $\sqrt{30}$ 
    \item $\sqrt[3]{5}$ 
    \item $\sqrt[3]{10}$ 
    \item $\sqrt[3]{20}$ 
    \item $\sqrt[3]{30}$ 
    \end{enumerate}
\end{multicols}
\item  Find two consecutive integers such that $\sqrt{7}$ lies between them.          
\item  Find two consecutive integers such that $\sqrt{15}$ lies between them.          
\item Factorise:
\begin{multicols}{2}
\begin{enumerate}[itemsep=5pt, label=\textbf{(\alph*)} ] 
\item ${a}^{2}-9$
\item ${m}^{2}-36$
\item $9{b}^{2}-81$
\item $16{b}^{6}-25{a}^{2}$
\item ${m}^{2}-\frac{1}{9}$
\item $5-5{a}^{2}{b}^{6}$
\item $16b{a}^{4}-81b$
\item ${a}^{2}-10a+25$
\item $16{b}^{2}+56b+49$
\item $2{a}^{2}-12ab+18{b}^{2}$
\item $-4{b}^{2}-144{b}^{8}+48{b}^{5}$
\item $(16-{x}^{4})$
\item ${7x}^{2}-14x+7xy-14y$
\item ${y}^{2}-7y-30$
\item $1-x-{x}^{2}+{x}^{3}$
\item $-3(1-{p}^{2})+p+1$
\item $x-x^{3} + y - y^{3}$
\item $x^{2} - 2x + 1 - y^{4}$
\item $4b(x^{3} - 1) + x(1-x^{3})$
\item $3p^{3} - \frac{1}{9}$
\end{enumerate}
\end{multicols}
\item Simplify the following:
\begin{multicols}{2}
\begin{enumerate}[itemsep=5pt, label=\textbf{(\alph*)} ] 
\item ${(a-2)}^{2}-a(a+4)$
\item $(5a-4b)(25{a}^{2}+20ab+16{b}^{2})$
\item $(2m-3)(4{m}^{2}+9)(2m+3)$
\item $(a+2b-c)(a+2b+c)$
\item $\dfrac{{p}^{2}-{q}^{2}}{p}÷\dfrac{p+q}{{p}^{2}-pq}$
\item $\dfrac{2}{x}+\dfrac{x}{2}-\dfrac{2x}{3}$
\item $\dfrac{1}{a+7}-\dfrac{a+7}{a^{2}-49}$
\item $\dfrac{x+2}{2x^{3}} + 16$
\item $\dfrac{1-2a}{4a^{2} -1} - \dfrac{a-1}{2a^{2}-3a+1} - \dfrac{1}{1-a}$
\item $\dfrac{x^{2} + 2x}{x^{2}+ x + 6} \times \dfrac{x^{2} + 2x + 1}{x^{2} + 3x +2}$
\end{enumerate}
\end{multicols}
\item Show that ${(2x-1)}^{2}-{(x-3)}^{2}$ can be simplified to $(x+2)(3x-4)$.
\item What must be added to ${x}^{2}-x+4$ to make it equal to ${(x+2)}^{2}$ ?
\item Evaluate $\dfrac{x^{3}+1}{x^{2}-x+1}$ if $x=7,85$ without using a calculator. Show your work.
\end{enumerate}
\practiceinfo 
\par 
 \par \begin{tabular}[h]{cccccc}
 (1.) lDU  &  (2.) lDP  &  (3.) l3G  & (4.) lOf & (5.) llN & (6.) lDE &
 (7.) lln  &  (8.) llQ  &  (9.) lDm  & (10.) lDy & (11.) lqW & (12.) lq1 &
 (13a-k.)  liM  &  (13l-p.) lTY  &  (13q-t.) lDV  & (14a-d.) lTg & (14e-f.) lT4 & (14g-j.) lDp &
 (15.) lib  &  (16.) liT  &  (17.) lDd    \end{tabular}
\end{eocexercises}


 \begin{solutions}{}{
\begin{enumerate}[itemsep=5pt, label=\textbf{\arabic*}. ] 


\item solution 1
\item solution 2
\item solution 3
\item solution 4
\item solution 5
\item solution 6
\item solution 7
\item solution 8
\item solution 9
\item solution 10
\item solution 11
\item solution 12
\item solution 13
\item solution 14
\item solution 15
\item solution 16
\item solution 17

\end{enumerate}}
\end{solutions}


