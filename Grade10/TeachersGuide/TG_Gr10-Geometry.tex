\chapter{Euclidean geometry}
\begin{exercises}{}
{
        \nopagebreak \noindent
\begin{enumerate}[label=\textbf{\arabic*}.]
\item Use adjacent, corresponding, co-interior and alternate angles to fill in all the angles labelled with letters in the diagram:\\
    \begin{pspicture}(0,-1.7981373)(6.5116725,1.7981373)
    \psline[linewidth=0.04cm](0.34542254,0.6181373)(5.2654223,0.6181373)
    \psline[linewidth=0.04cm](0.34542254,-0.7418627)(5.2654223,-0.7418627)
    \psline[linewidth=0.04cm](0.0,-1.7781373)(5.5254226,1.7781373)
    \rput(4.4,0.8){\footnotesize$42^\circ$}
    \rput(3.664485,0.8881373){$a$}
    \rput(3.0458913,0.3881373){$b$}
    \rput(3.8072975,0.4081373){$c$}
    \rput(1.5262038,-0.5518627){$d$}
    \rput(2.2315164,-0.5518627){$e$}
    \rput(1.6212038,-1.0518627){$f$}
    \rput(1.05,-0.95){$g$}
    \rput{-66.70425}(1.9155159,4.522671){\psarc[linewidth=0.032]{-}(4.3934965,0.8061751){0.24084859}{20.012312}{156.69316}}
    \psline[linewidth=0.04](3.9254224,-0.6018627)(4.2254224,-0.7418627)(3.9454224,-0.8818627)
    \psline[linewidth=0.04](3.6854224,-0.6018627)(3.9854226,-0.7418627)(3.7054226,-0.8818627)
    \psline[linewidth=0.04](2.2454226,0.7581373)(2.5454226,0.6181373)(2.2654226,0.47813728)
    \psline[linewidth=0.04](2.0054226,0.7581373)(2.3054225,0.6181373)(2.0254226,0.47813728)
    \end{pspicture}
\\


\item Find all the unknown angles in the figure: \\
\\
\scalebox{1.2} {
    \begin{pspicture}(0,-1.6967187)(5.5376563,1.7167188)
    \psline[linewidth=0.04cm](3.1065626,1.4032813)(5.0465627,-1.4567187)
    \psline[linewidth=0.04cm](0.4665625,1.1832813)(2.4065626,-1.6767187)
    \psline[linewidth=0.01cm,arrowsize=0.2cm 2.0,arrowlength=1.4,arrowinset=0.5]{->>}(0.49,1.1417187)(1.3465625,-0.11671878)
    \psline[linewidth=0.01cm,arrowsize=0.2cm 2.0,arrowlength=1.4,arrowinset=0.5]{->>}(3.3065624,1.1032813)(4.29,-0.29828128)
    \psline[linewidth=0.04cm](1.5065625,-0.3367187)(4.4465623,-0.5567187)
    \psline[linewidth=0.04cm](0.8665625,0.6232813)(3.8065624,0.40328127)
    \psline[linewidth=0.01cm,arrowsize=0.2cm 2.0,arrowlength=1.4,arrowinset=0.5]{->}(1.5265626,-0.33671877)(3.1265626,-0.47671878)
    \psline[linewidth=0.01cm,arrowsize=0.2cm 2.0,arrowlength=1.4,arrowinset=0.5]{->}(1.0065625,0.62328136)(2.77,0.48171872)
    \psline[linewidth=0.04cm](3.7665625,0.40328127)(1.5065625,-0.3367187)
    \psline[linewidth=0.04cm](4.4065623,-0.5567187)(2.1465626,-1.2967187)
    \psline[linewidth=0.01cm,arrowsize=0.2cm 2.0,arrowlength=1.4,arrowinset=0.5]{->}(3.2065625,0.22328123)(2.3665626,-0.09671878)
    \psline[linewidth=0.01cm,arrowsize=0.2cm 2.0,arrowlength=1.4,arrowinset=0.5]{->}(3.4065626,0.30328122)(2.5665624,-0.01671878)
    \psline[linewidth=0.01cm,arrowsize=0.2cm 2.0,arrowlength=1.4,arrowinset=0.5]{->}(3.6265626,0.38328123)(2.7865624,0.06328122)
    \psline[linewidth=0.01cm,arrowsize=0.2cm 2.0,arrowlength=1.4,arrowinset=0.5]{->}(3.7465625,-0.7767188)(2.89,-1.0782813)
    \psline[linewidth=0.01cm,arrowsize=0.2cm 2.0,arrowlength=1.4,arrowinset=0.5]{->}(3.9465625,-0.69671875)(3.1065626,-1.0167189)
    \psline[linewidth=0.01cm,arrowsize=0.2cm 2.0,arrowlength=1.4,arrowinset=0.5]{->}(4.1665626,-0.61671877)(3.3265624,-0.93671876)
    \usefont{T1}{ptm}{m}{n}
    \rput(0.28453124,1.1932813){$A$}
    \usefont{T1}{ptm}{m}{n}
    \rput(0.60765624,0.5132813){$B$}
    \usefont{T1}{ptm}{m}{n}
    \rput(1.3045312,-0.54627085){$C$}
    \usefont{T1}{ptm}{m}{n}
    \rput(1.9399999,-1.3867188){$D$}
    \usefont{T1}{ptm}{m}{n}
    \rput(3.2096875,1.5132812){$E$}
    \usefont{T1}{ptm}{m}{n}
    \rput(3.9235938,0.49328125){$F$}
    \usefont{T1}{ptm}{m}{n}
    \rput(4.5775,-0.46671876){$G$}
    \usefont{T1}{ptm}{m}{n}
    \rput(5.1331253,-1.3267187){$H$}
    \usefont{T1}{ptm}{m}{n}
    \rput(1.25,0.41){\tiny $70^\circ$}
    \usefont{T1}{ptm}{m}{n}
    \rput(2.3,-1.0917187){\tiny $80^\circ$}
    \usefont{T1}{ptm}{m}{n}
    \rput(0.87921876,0.7532813){\tiny $1$}
    \usefont{T1}{ptm}{m}{n}
    \rput(3.4192188,0.61328125){\tiny $1$}
    \usefont{T1}{ptm}{m}{n}
    \rput(3.144219,0.33328128){\tiny $2$}
    \usefont{T1}{ptm}{m}{n}
    \rput(3.6359375,0.21328127){\tiny $3$}
    \usefont{T1}{ptm}{m}{n}
    \rput(4.119219,-0.44671872){\tiny $1$}
    \usefont{T1}{ptm}{m}{n}
    \rput(3.7442188,-0.6667187){\tiny $2$}
    \usefont{T1}{ptm}{m}{n}
    \rput(4.2959375,-0.7467187){\tiny $3$}
    \usefont{T1}{ptm}{m}{n}
    \rput(1.5392189,-0.12671873){\tiny $1$}
    \usefont{T1}{ptm}{m}{n}
    \rput(2.1442187,-0.28671873){\tiny $2$}
    \usefont{T1}{ptm}{m}{n}
    \rput(1.7959374,-0.5667187){\tiny $3$}
    \usefont{T1}{ptm}{m}{n}
    \rput(2.3192186,-1.3667188){\tiny $1$}
    \rput{-165.35927}(1.9916399,1.1648247){\psarc[linewidth=0.04](1.0706391,0.4544851){0.34608197}{92.73279}{185.53572}}
    \rput{-45.10457}(1.5105618,1.2407944){\psarc[linewidth=0.04](2.2491949,-1.1983162){0.29357287}{50.13503}{194.957}}
    \end{pspicture}
}  \\ 


\item Find the value of $x$ in the figure: \\
\scalebox{1.2} % Change this value to rescale the drawing.
{
    \begin{pspicture}(0,-2.203125)(4.2173834,2.203125)
    \psline[linewidth=0.04cm](0.84234375,1.8096874)(0.84234375,-1.7303125)
    \psline[linewidth=0.04cm](1.8423437,1.8096874)(1.8423437,-1.7303125)
    \psline[linewidth=0.04cm](0.38,0.9631251)(3.36,-0.7768749)
    \psline[linewidth=0.04cm](3.94,0.6631251)(0.12,-0.4368749)
    \usefont{T1}{ptm}{m}{n}
    \rput(0.81859374,1.9996876){$A$}
    \usefont{T1}{ptm}{m}{n}
    \rput(0.77796876,-2.0003126){$B$}
    \usefont{T1}{ptm}{m}{n}
    \rput(1.7839063,-2.0003126){$C$}
    \usefont{T1}{ptm}{m}{n}
    \rput(1.7903125,1.9996876){$D$}
    \usefont{T1}{ptm}{m}{n}
    \rput(0.29453126,0.73968744){$X$}
    \usefont{T1}{ptm}{m}{n}
    \rput(1.6604688,-0.2610938){$Y$}
    \usefont{T1}{ptm}{m}{n}
    \rput(0.63609374,-0.5403126){$Z$}
    \psline[linewidth=0.04cm](1.8423437,1.5096875)(1.9423437,1.6896875)
    \psline[linewidth=0.04cm](1.8423437,1.5096875)(1.7423438,1.6696874)
    \usefont{T1}{ptm}{m}{n}
    \rput(1.1,-0.0){\scriptsize $60^{\circ}$}
    \usefont{T1}{ptm}{m}{n}
    \rput(0.9571874,0.45468745){\scriptsize $x$}
    \usefont{T1}{ptm}{m}{n}
    \rput{-0.80275255}(0.0,0.046365827){\rput(2.9,-0.006999445){\scriptsize $x-20^\circ$}}
    \rput{-122.44731}(3.35288,1.925009){\psarc[linewidth=0.04](2.205064,0.041774992){0.20109077}{59.366795}{168.11142}}
    \psline[linewidth=0.04cm](0.84234375,1.5296875)(0.9423437,1.7096875)
    \psline[linewidth=0.04cm](0.84234375,1.5296875)(0.7423437,1.6896874)
    \end{pspicture} 
} \\



\item Determine whether the pairs of lines in the following figures are parallel:
    \begin{enumerate}[itemsep=10pt, label=\textbf{(\alph*)} ] 
    \item 
    \scalebox{1} % Change this value to rescale the drawing.
    {
	\begin{pspicture}(0,-2.451875)(6.3617187,2.451875)
	\psline[linewidth=0.04cm](0.0,1.6081251)(6.1,-1.751875)
	\psline[linewidth=0.04cm](1.42,2.171875)(1.74,-1.808125)
	\psline[linewidth=0.04cm](4.72,2.188125)(5.08,-2.211875)
	\rput(2,0.85812503){$115^{\circ}$}
	\rput(4.583125,-0.621875){$55^{\circ}$}
	\rput(1.2482814,2.278125){$O$}
	\rput(1.5804688,-2.021875){$P$}
	\rput(4.9214063,2.178125){$Q$}
	\rput(5.2339063,-2.301875){$R$}
	\rput(0.3,1.678125){$S$}
	\rput(6.210781,-1.7818749){$T$}
	\rput(1.3671875,0.21812505){$A$}
	\rput(4.5465627,-1.361875){$B$}
	\rput(1.3509375,0.97812504){\tiny $1$}
	\rput(1.3359375,0.65812504){\tiny $2$}
	\rput(1.6876563,0.51812506){\tiny $3$}
	\rput(5.1876564,-1.5218749){\tiny $3$}
	\rput(5.215937,-1.0618749){\tiny $2$}
	\rput(4.8,-1.3018749){\tiny $1$}
	\end{pspicture} 
    }
\\
    \item 
    \scalebox{1} % Change this value to rescale the drawing.
    {
	\begin{pspicture}(0,-2.8392186)(7.70125,2.8392186)
	\psline[linewidth=0.04cm](0.0625,-0.05921875)(7.4625,-0.05921875)
	\psline[linewidth=0.04cm](0.6625,-1.9592187)(4.74,2.5192187)
	\psline[linewidth=0.04cm](6.1225,2.4007812)(2.0025,-2.6592188)
	\rput(4.8440623,0.25078127){$45^{\circ}$}
	\rput(2.5667188,-0.28921875){$124^{\circ}$}
	\rput(4.7973437,2.6707811){$M$}
	\rput(0.601875,-1.7292187){$N$}
	\rput(6.250781,2.4707813){$O$}
	\rput(2.2229688,-2.6892188){$P$}
	\rput(0.10390625,0.19078125){$Q$}
	\rput(7.536406,0.13078125){$R$}
	\rput(2.21875,0.49078128){$K$}
	\rput(4.4520316,-0.5492188){$L$}
	\rput(2.3734376,0.13078125){\tiny $1$}
	\rput(2.7584374,0.09078125){\tiny $2$}
	\rput(2.0901563,-0.18921876){\tiny $3$}
	\rput(4.170156,-0.28921875){\tiny $3$}
	\rput(4.0184374,0.09078125){\tiny $2$}
	\rput(3.7334375,-0.22921875){\tiny $1$}
	\end{pspicture} 
    }
    \item 
\scalebox{1} % Change this value to rescale the drawing.
{
    \begin{pspicture}(0,-3.1192186)(6.325469,3.1192186)
    \psline[linewidth=0.04cm](0.1525,1.0007812)(6.1125,0.98078126)
    \psline[linewidth=0.04cm](0.1525,-1.0392187)(6.1125,-1.0192188)
    \psline[linewidth=0.04cm](3.0125,3.0007813)(3.3125,-3.0192187)
    \rput(2.7432813,0.7307812){$95^{\circ}$}
    \rput(3.6395311,-1.3692188){$85^{\circ}$}
    \rput(3.24875,2.950781){$K$}
    \rput(3.5020313,-2.9692187){$L$}
    \rput(0.14734375,-0.88921875){$M$}
    \rput(6.0718746,-0.82921875){$N$}
    \rput(0.12328125,1.1707813){$T$}
    \rput(6.1564064,1.1507812){$Y$}
    \rput(3.3034375,0.79078126){\tiny $1$}
    \rput(3.3084376,1.2107813){\tiny $2$}
    \rput(2.9001563,1.1907812){\tiny $3$}
    \rput(3.3234375,-0.88921875){\tiny $1$}
    \rput(2.9884377,-0.88921875){\tiny $2$}
    \rput(3.0201561,-1.2692188){\tiny $3$}
% \rput(2.973594,1.4596876){$U$}
    \end{pspicture} 
}
    \end{enumerate}
\\

\item If $AB$ is parallel to $CD$ and $AB$ is parallel to $EF$, explain why $CD$ must be parallel to $EF$.\vspace{8pt}\\
\begin{pspicture}(0,-0.81328124)(3.8328125,0.81328124)
\psline[linewidth=0.04cm](0.2415625,0.6067188)(3.4415624,0.38671875)
\psline[linewidth=0.04cm](0.3015625,-0.09328125)(3.4215624,-0.09328125)
\psline[linewidth=0.04cm](0.3815625,-0.6532813)(3.5215626,-0.55328125)
\rput(0.1221875,-0.10328125){$A$}
\rput(3.6009376,-0.08328125){$B$}
\rput(3.654375,-0.5832813){$F$}
\rput(0.178125,-0.66328126){$E$}
\rput(0.086875,0.61671877){$C$}
\rput(3.6504688,0.33671874){$D$}
\end{pspicture}  
\end{enumerate}

}
\end{exercises}


 \begin{solutions}{}{
\begin{enumerate}[itemsep=5pt, label=\textbf{\arabic*}. ] 


\item \begin{multicols}{2} %solution 1
    $a=138^{\circ}$\\ $b=42^{\circ}$\\ $c=138^{\circ}$\\ $d=138^{\circ}$\\ $e=42^{\circ}$\\ $f=138^{\circ}$\\ $g=42^{\circ}$
\end{multicols}
\item \begin{multicols}{2}%solution 2
 $\hat{B}_1=110^{\circ}$\\ $\hat{C}_1=80^{\circ}$\\ $\hat{C}_2=30^{\circ}$\\ $\hat{C}_3=70^{\circ}$\\ $\hat{D}_1=100^{\circ}$\\ $\hat{F}_1=70^{\circ}$\\ $\hat{F}_2=30^{\circ}$\\ $\hat{F}_3=80^{\circ}$\\ $\hat{G}_1=70^{\circ}$\\ $\hat{G}_2=30^{\circ}$\\ $\hat{G}_3=80^{\circ}$
\end{multicols}
\item %solution 3
$x + 60^\circ + x - 20^\circ = 180^\circ\\
2x = 180^\circ - 40^\circ\\
2x = 140^\circ\\
\therefore x = 70^\circ$

\item %solution 4
\begin{enumerate}[itemsep=3pt, label=\textbf{(\alph*)} ]
\item For interior angles $115^\circ + 55^\circ = 170^\circ$\\
if $\parallel$ the sum would be $180^\circ \therefore$ the lines are not parallel.

\item $K_2 = 180^{\circ} - 124^\circ  - 56^\circ$\\
if $MN \parallel OP$ then $\hat{K}_2 = 56$ would be equal to $\hat{L} = 45^\circ$\\
$\therefore MN$ is not parallel to $OP$.

\item Let $\hat{U}$ be point of intersection of lines $KL$ and $TY$ and $\hat{V}$ be the point of intersection of lines $KL$ and $MN$.\\
$\hat{U_4} = 96^{\circ}\\
\therefore \hat{U_1} = 180^{\circ} - 95^{\circ}$~~~~~~~(Angles on a straight line)\\
$\therefore \hat{U_1} = 85^{\circ}$\\
Also, $\hat{U_1} = 85^{\circ} = \hat{V_4}$\\
These are corresponding angles\\
$\therefore TY \parallel MN$
\end{enumerate}

\item %solution 5
if $a = 2$\\
and $b = a$\\
the we know that $b = 2$\\
Similarly if $AB \parallel CD$\\
and $EF \parallel AB$\\
then we know that $EF \parallel CD$
\end{enumerate}}
\end{solutions}


\begin{exercises}{}{
\begin{enumerate}[noitemsep,label=\textbf{\arabic*}. ] 
\item 
Calculate the unknown variables in each of the following figures. 
\begin{center}
\scalebox{1}
{
\begin{pspicture}(0,-5.071875)(12.643125,5.071875)
\rput{-110.476}(-0.30693707,5.6482253){\pstriangle[linewidth=0.04,dimen=outer](1.8065645,1.4546672)(2.2230408,2.9519162)}
\psline[linewidth=0.04cm](1.6210938,2.2534375)(1.6410937,2.6534376)
\psline[linewidth=0.04cm](2.1375413,3.566676)(1.8839829,3.2154255)
% % \usefont{T1}{ptm}{m}{n}
\rput(2.4582813,2.6234374){$36^{\circ}$}
% % \usefont{T1}{ptm}{m}{n}
\rput(0.8920312,4.0634375){$x$}
% % \usefont{T1}{ptm}{m}{n}
\rput(0.42671874,2.6834376){$y$}
% % \usefont{T1}{ptm}{m}{n}
\rput(1.0107813,4.6434374){$N$}
% % \usefont{T1}{ptm}{m}{n}
\rput(0.11265625,2.1234374){$P$}
% % \usefont{T1}{ptm}{m}{n}
\rput(3.2645311,2.2034376){$O$}
% % \usefont{T1}{ptm}{m}{n}
\rput(0.11328125,4.7234373){\textbf{(a)}}
% % \usefont{T1}{ptm}{m}{n}
\rput(6.0982814,4.1){$30^{\circ}$}
% % \usefont{T1}{ptm}{m}{n}
\rput(7.052031,2.6834376){$x$}
% % \usefont{T1}{ptm}{m}{n}
\rput(5.637344,2.6634376){$68^{\circ}$}
% % \usefont{T1}{ptm}{m}{n}
\rput(5.8707814,4.7634373){$N$}
% % \usefont{T1}{ptm}{m}{n}
\rput(5.2726564,2.2434375){$P$}
% % \usefont{T1}{ptm}{m}{n}
\rput(7.024531,2.2234375){$O$}
% % \usefont{T1}{ptm}{m}{n}
\rput(4.497031,4.6434374){\textbf{(b)}}
\psline[linewidth=0.04cm](5.201094,2.4934375)(8.101094,2.4934375)
\psline[linewidth=0.04cm](6.101094,4.7934375)(6.901094,2.4934375)
\psline[linewidth=0.04cm](6.101094,4.7934375)(5.201094,2.4934375)
\psline[linewidth=0.04cm](9.341094,2.5334375)(10.241094,4.8534374)
\psline[linewidth=0.04cm](10.241094,4.8934374)(11.541093,1.9934375)
\psline[linewidth=0.04cm](9.341094,2.5334375)(12.001093,2.5334375)
% % \usefont{T1}{ptm}{m}{n}
\rput(11.757343,2.2834375){$68^{\circ}$}
% % \usefont{T1}{ptm}{m}{n}
\rput(9.757343,2.7034376){$68^{\circ}$}
% % \usefont{T1}{ptm}{m}{n}
\rput(11.164531,2.2834375){$O$}
% % \usefont{T1}{ptm}{m}{n}
\rput(9.172656,2.2834375){$P$}
% % \usefont{T1}{ptm}{m}{n}
\rput(9.990782,4.9034376){$N$}
% % \usefont{T1}{ptm}{m}{n}
\rput(10.272031,4.4634376){$x$}
% % \usefont{T1}{ptm}{m}{n}
\rput(11.446719,2.8034375){$y$}
% % \usefont{T1}{ptm}{m}{n}
\rput(9.233907,4.6434374){\textbf{(c)}}
% \usefont{T1}{ptm}{m}{n}
\rput(0.6439063,-2.9565625){$12$}
\psline[linewidth=0.04cm](0.82109374,-4.2665625)(2.5210938,-4.2665625)
\psline[linewidth=0.04cm](1.0210937,-1.6665626)(2.5210938,-4.2665625)
\psarc[linewidth=0.04](0.87109375,-4.3165627){0.31}{3.0127876}{93.01279}
\psarc[linewidth=0.04](1.0610937,-1.7265625){0.38}{260.5377}{303.69006}
\psline[linewidth=0.04cm](1.0210937,-1.6665626)(0.82109374,-4.2665625)
\psarc[linewidth=0.04](1.0910938,-1.7565625){0.41}{258.35403}{303.69006}
% \usefont{T1}{ptm}{m}{n}
\rput(1.9464062,-2.8565626){$14$}
% \usefont{T1}{ptm}{m}{n}
\rput(2.4045312,-4.5165625){$O$}
% \usefont{T1}{ptm}{m}{n}
\rput(0.85078126,-1.5965625){$N$}
% \usefont{T1}{ptm}{m}{n}
\rput(0.67265624,-4.4565625){$P$}
\psline[linewidth=0.04cm](3.9605203,-2.7353694)(3.1151874,-4.210297)
\psline[linewidth=0.04cm](6.1168413,-4.2017527)(3.1151874,-4.210297)
\rput{-119.818535}(8.217016,-0.7463117){\psarc[linewidth=0.04](3.892278,-2.753887){0.31}{3.0127876}{93.01279}}
\rput{-119.818535}(12.700432,-1.0538051){\psarc[linewidth=0.04](6.044895,-4.206622){0.38}{260.5377}{303.69006}}
\psline[linewidth=0.04cm](6.1168413,-4.2017527)(3.9605203,-2.7353694)
\rput{-119.818535}(12.648766,-1.1059643){\psarc[linewidth=0.04](6.0039496,-4.217732){0.41}{258.35403}{303.69006}}
% \usefont{T1}{ptm}{m}{n}
\rput(6.181094,-4.4565625){$T$}
% \usefont{T1}{ptm}{m}{n}
\rput(3.3395312,-4.4365625){$S$}
% \usefont{T1}{ptm}{m}{n}
\rput(4.4492188,-4.4565625){$21$}
% \usefont{T1}{ptm}{m}{n}
\rput(3.8982813,-2.5365624){$R$}
% \usefont{T1}{ptm}{m}{n}
\rput(5.112031,-3.1965625){$x$}
% \usefont{T1}{ptm}{m}{n}
\rput(0.50421876,0.7834375){\textbf{(d)}}
\psline[linewidth=0.04cm](1.9010937,1.1334375)(6.501094,-1.5665625)
\psline[linewidth=0.04cm](6.501094,-1.5665625)(6.501094,0.4334375)
\psline[linewidth=0.04cm](6.501094,0.4334375)(1.4010937,0.3334375)
\psline[linewidth=0.04cm](1.9010937,1.1334375)(1.4010937,0.3334375)
\psline[linewidth=0.04cm](2.1010938,1.0334375)(2.0010939,0.8334375)
\psline[linewidth=0.04cm](1.8010937,0.9334375)(2.0010939,0.8334375)
\psline[linewidth=0.04cm](6.301094,0.4334375)(6.301094,0.2334375)
\psline[linewidth=0.04cm](6.301094,0.2334375)(6.501094,0.2334375)
% \usefont{T1}{ptm}{m}{n}
\rput(3.1045313,0.1234375){$O$}
% \usefont{T1}{ptm}{m}{n}
\rput(1.9107813,1.2834375){$N$}
% \usefont{T1}{ptm}{m}{n}
\rput(1.2326562,0.2834375){$P$}
% \usefont{T1}{ptm}{m}{n}
\rput(6.7382812,0.5834375){$R$}
% \usefont{T1}{ptm}{m}{n}
\rput(6.599531,-1.6765625){$S$}
% \usefont{T1}{ptm}{m}{n}
\rput(2.7320313,0.8434375){$x$}
% \usefont{T1}{ptm}{m}{n}
\rput(1.3435937,0.7834375){$19$}
% \usefont{T1}{ptm}{m}{n}
\rput(6.7139063,-0.4565625){$76$}
% \usefont{T1}{ptm}{m}{n}
\rput(4.655156,0.5834375){$116$}
% \usefont{T1}{ptm}{m}{n}
\rput(7.7326565,0.7634375){\textbf{(e)}}
\psline[linewidth=0.04cm](8.821094,1.1334375)(8.821094,-1.1665626)
\psline[linewidth=0.04cm](8.821094,-1.1665626)(10.621093,-1.1665626)
\psline[linewidth=0.04cm](8.821094,1.1334375)(10.621093,-1.1665626)
\psline[linewidth=0.04cm](8.821094,-0.9665625)(9.021094,-0.9665625)
\psline[linewidth=0.04cm](9.021094,-0.9665625)(9.021094,-1.1665626)
% \usefont{T1}{ptm}{m}{n}
\rput(8.535155,-0.0565625){$15$}
% \usefont{T1}{ptm}{m}{n}
\rput(9.780469,-1.3765625){$20$}
% \usefont{T1}{ptm}{m}{n}
\rput(9.872031,0.1434375){$x$}
% \usefont{T1}{ptm}{m}{n}
\rput(10.644532,-1.4365625){$O$}
% \usefont{T1}{ptm}{m}{n}
\rput(8.652656,-1.3765625){$P$}
% \usefont{T1}{ptm}{m}{n}
\rput(8.790781,1.2634375){$N$}
\psline[linewidth=0.04cm](7.3610935,-2.5665624)(12.261094,-2.5665624)
\psline[linewidth=0.04cm](7.3610935,-2.5665624)(11.161094,-4.6665626)
\psline[linewidth=0.04cm](12.261094,-2.5665624)(11.161094,-4.6665626)
\psline[linewidth=0.04cm](7.3610935,-2.5665624)(6.7610936,-3.7665625)
\psline[linewidth=0.04cm](6.7610936,-3.7665625)(11.161094,-4.6665626)
\psline[linewidth=0.04cm](11.061094,-4.3665624)(11.261094,-4.4665623)
\psline[linewidth=0.04cm](11.061094,-4.3665624)(10.961093,-4.5665627)
\psline[linewidth=0.04cm](7.5610933,-2.6665626)(7.4610934,-2.8665626)
\psline[linewidth=0.04cm](7.4610934,-2.8665626)(7.2610936,-2.7665625)
% \usefont{T1}{ptm}{m}{n}
\rput(11.921407,-3.6365626){$9$}
% \usefont{T1}{ptm}{m}{n}
\rput(9.915155,-2.8365624){$15$}
% \usefont{T1}{ptm}{m}{n}
\rput(7.210156,-3.2965624){$5$}
% \usefont{T1}{ptm}{m}{n}
\rput(8.792031,-3.0965624){$x$}
% \usefont{T1}{ptm}{m}{n}
\rput(8.286718,-4.3365627){$y$}
% \usefont{T1}{ptm}{m}{n}
\rput(6.572656,-3.7965624){$P$}
% \usefont{T1}{ptm}{m}{n}
\rput(7.2507815,-2.4165626){$N$}
% \usefont{T1}{ptm}{m}{n}
\rput(11.164531,-4.9165626){$O$}
% \usefont{T1}{ptm}{m}{n}
\rput(12.478281,-2.5165625){$R$}
% \usefont{T1}{ptm}{m}{n}
\rput(0.23234375,-2.5){\textbf{(f)}}
% \usefont{T1}{ptm}{m}{n}
%\rput(3.1,-2.7165625){\textbf{(g)}}
\rput(6.5154686,-2.7165625){\textbf{(g)}}
% \usefont{T1}{ptm}{m}{n}
\rput(1.5467187,-4.3965626){$y$}
% \usefont{T1}{ptm}{m}{n}
\rput(3.3759375,-3.2765625){$6$}
\end{pspicture}  
} 
} 
\end{center}
\item 
State whether the following pairs of triangles are congruent or not. Give reasons for your answers. If there is not enough information to make a decision, explain why.
\begin{center}
\scalebox{0.75} % Change this value to rescale the drawing.
{
\begin{pspicture}(0,-5.6992188)(12.713437,5.6992188)
\psline[linewidth=0.04cm](0.5671875,5.480781)(1.1671875,1.9807812)
\psline[linewidth=0.04cm](0.5671875,5.480781)(6.6671877,1.4807812)
\psline[linewidth=0.04cm](1.1671875,1.9807812)(6.5671873,5.2807813)
\psline[linewidth=0.04cm](6.5671873,5.2807813)(6.6671877,1.4807812)
\rput(3.654375,3.1707811){$C$}
\rput(0.954375,5.5307813){$B$}
\rput(0.97375,1.7907813){$A$}
\rput(6.849219,5.3707814){$E$}
\rput(6.9271874,1.3907813){$D$}
\psline[linewidth=0.04cm](1.9671875,4.7807813)(1.7671875,4.480781)
\psline[linewidth=0.04cm](2.0671875,4.6807814)(1.8671875,4.380781)
\psline[linewidth=0.04cm](5.1671877,2.6807814)(4.9671874,2.3807812)
\psline[linewidth=0.04cm](5.2671876,2.5807812)(5.0671873,2.2807813)
\psline[linewidth=0.04cm](5.0671873,4.5807815)(5.2671876,4.2807813)
\psline[linewidth=0.04cm](2.3271875,2.8807812)(2.5271876,2.5807812)
\psline[linewidth=0.04cm](10.3671875,5.480781)(11.967188,1.8807813)
\psline[linewidth=0.04cm](10.3671875,5.480781)(8.767187,1.8807813)
\psline[linewidth=0.04cm](8.767187,1.8807813)(11.967188,1.8807813)
\psline[linewidth=0.04cm](10.3671875,5.480781)(10.3671875,1.8807813)
\psline[linewidth=0.04cm](11.3671875,3.7807813)(10.967188,3.5807812)
\psline[linewidth=0.04cm](9.3671875,3.7807813)(9.767187,3.5807812)
\psarc[linewidth=0.04](11.967188,1.8807813){0.3}{111.80141}{180.0}
\rput{-90.0}(6.9864063,10.747969){\psarc[linewidth=0.04](8.8671875,1.8807813){
0.3}{90.0}{172.87498}}
\psline[linewidth=0.04cm](7.8471875,0.26078126)(12.547188,-2.4392188)
\psline[linewidth=0.04cm](12.547188,-2.4392188)(12.547188,-0.13921875)
\psline[linewidth=0.04cm](12.547188,-0.13921875)(7.8471875,-2.9392188)
\psline[linewidth=0.04cm](7.8471875,-2.9392188)(7.8471875,0.26078126)
\psline[linewidth=0.04cm](11.547188,-1.6392188)(11.347187,-1.9392188)
\psline[linewidth=0.04cm](11.247188,-0.7392188)(11.447187,-1.0392188)
\psline[linewidth=0.04cm](9.327188,-0.33921874)(9.087188,-0.6592187)
\psline[linewidth=0.04cm](9.427188,-0.41921875)(9.207188,-0.71921873)
\psline[linewidth=0.04cm](9.087188,-1.9792187)(9.287188,-2.2792187)
\psline[linewidth=0.04cm](8.987187,-2.0592186)(9.187187,-2.3592188)
\psline[linewidth=0.04cm](0.7071875,-1.8592187)(3.2071874,-1.4592187)
\psline[linewidth=0.04cm](3.2071874,-1.4592187)(5.7071877,-1.8592187)
\psline[linewidth=0.04cm](4.5071874,0.54078126)(5.7071877,-1.8592187)
\psline[linewidth=0.04cm](4.5071874,0.54078126)(3.2071874,-1.4592187)
\psline[linewidth=0.04cm](1.9071875,0.54078126)(3.2071874,-1.4592187)
\psline[linewidth=0.04cm](1.9071875,0.54078126)(0.7071875,-1.8592187)
\psline[linewidth=0.04cm](3.8071876,-0.35921875)(4.1071873,-0.55921876)
\psline[linewidth=0.04cm](2.7071874,-0.35921875)(2.4071875,-0.55921876)
\psline[linewidth=0.04cm](1.9071875,-1.5192188)(2.0071876,-1.8192188)
\psline[linewidth=0.04cm](2.0071876,-1.4792187)(2.1071875,-1.7792188)
\psline[linewidth=0.04cm](4.4271874,-1.4792187)(4.3271875,-1.7792188)
\psline[linewidth=0.04cm](4.5271873,-1.4992187)(4.4271874,-1.7992188)
\psline[linewidth=0.04cm](1.3071876,-4.159219)(6.3071876,-4.159219)
\psline[linewidth=0.04cm](2.5071876,-2.9592187)(1.3071876,-4.159219)
\psline[linewidth=0.04cm](2.5071876,-2.9592187)(6.3071876,-4.159219)
\psline[linewidth=0.04cm](1.3071876,-4.159219)(2.5071876,-5.3592186)
\psline[linewidth=0.04cm](6.3071876,-4.159219)(2.5071876,-5.3592186)
\psdots[dotsize=0.12](1.7071875,-3.9592187)
\psdots[dotsize=0.12](1.7071875,-4.3592186)
\psarc[linewidth=0.04](2.5271876,-2.9792187){0.3}{220.85538}{342.21613}
\psarc[linewidth=0.04](2.5671875,-5.3392186){0.3}{14.036243}{139.39871}
\rput(5.8271875,-2.1092188){$D$}
\rput(0.73375,-2.1692188){$A$}
\rput(1.774375,0.8307812){$B$}
\rput(3.214375,-1.6892188){$C$}
\rput(4.709219,0.7707813){$E$}
\rput(12.547188,-2.7692187){$D$}
\rput(8.07375,-3.1692188){$A$}
\rput(8.194375,0.37078124){$B$}
\rput(10.554375,-1.6492188){$C$}
\rput(12.309218,0.09078125){$E$}
\rput(10.714375,5.3707814){$B$}
\rput(8.83375,1.5707812){$A$}
\rput(10.394375,1.6107812){$C$}
\rput(12.107187,1.5907812){$D$}
\rput(6.594375,-4.0492187){$C$}
\rput(1.13375,-4.229219){$A$}
\rput(2.8871875,-5.5492187){$D$}
\rput(2.834375,-2.7492187){$B$}
\rput(0.11328125,4.9307814){\LARGE \textbf{(a)}}
\rput(9.050157,5.090781){\LARGE \textbf{(b)}}
\rput(0.25359374,0.29078126){\LARGE \textbf{(c)}}
\rput(7.36375,-0.04921875){\LARGE \textbf{(d)}}
\rput(0.45296875,-3.3292189){\LARGE \textbf{(e)}}
\end{pspicture} 
}
\end{center}
\end{enumerate}     

\end{exercises}


 \begin{solutions}{}{
\begin{enumerate}[itemsep=5pt, label=\textbf{\arabic*}. ] 


\item %solution 1
      \begin{enumerate}[noitemsep, label=\textbf{(\alph*)} ]
\item The triangle is isosceles,\\ $\therefore x=y$\\
$180^{\circ} = 36^{\circ}-2x\\
2x=144^{\circ}\\
\therefore x=y=72^{\circ}$
\item $x$ is an exterior angle\\
$\therefore x=30^{\circ}+68^{\circ} \\
=98^{\circ}$
\item $y$ is an exterior angle\\
$\therefore y = x+68^{\circ}= 112^{\circ}\\
x=112^{\circ}-68^{\circ} = 44^{\circ}$
\item Interior angles of a triangle add up to $180^{\circ}$\\
$\therefore \hat{P} = 180^{\circ} - \hat{N} - N\hat{O}P\\
$and $\hat{S} = 180^{\circ} - \hat{R} - R\hat{O}S\\
\hat{N} = \hat{R} = 90^{\circ}$ and $N\hat{O}P = R\hat{O}P$ (opposite angles)\\
$\therfore \hat{P} = \hat{S}\\$
Therefore $\triangle NPO$ and $\triangle ROS$ are similar because they have the same angles.\\
Similar triangles have proportional sides\\
$\therefore \frac{NP}{RS} = \frac{NO}{OR}\\
\frac{19}{76}=\frac{x}{116}\\
\therefore x = 19$ units

 
\item From the theorem of Pythagoras we have\\
$x^2 = 15^2 + 20^2\\
\therefore x = \sqrt{625}\\
=25$ units
\item $\triangle NPO ||| \triangle TSR$ (AAA)\\
$\therefore \frac{OP}{NP} = \frac{SR}{TR}\\
\frac{y}{12}=\frac{6}{x}\\
\therefore xy=72\\
$and $\frac{NO}{OP} = \frac{TS}{TR}\\
\frac{14}{12} = \frac{21}{x}\\
\therefore x = \frac{21 \times 12}{14}\\
\therefore x = 18$ units\\
$\therefore y(18)=72\\
\therefore y=4$ units


\item From the theorem of Pythagoras:\\
$x^2 = 15^2 - 9^2\\
x=\sqrt{144}
x=12$ units \\
And\\
$y^2 = x^2+5^2\\
y^2=144+25\\
y=\sqrt{169}\\
y=13$ units
      \end{enumerate}
\item %solution 2
      \begin{enumerate}[itemsep=3pt, label=\textbf{(\alph*)} ]
      \item 
$\triangle ABC \equiv \triangle EDC$ (SAS: $BC=CD$, $EC=AC$ and $B\hat{C}A = E\hat{C}D$)
      \item We have two equal sides ($AB = BD$ and $BC$ is common to both triangles) and one equal angle ($\hat{A} = \hat{D}$) but the sides do not include the known angle. The triangles therefore do not have a SAS  and are therefore not congruent. (Note: $A\hat{C}B$ is not necessarily equal to $D\hat{C}B$ because it is not given that $BC \perp AD$)
      \item There is not enough information given. We need at least three facts about the triangles and in this example we only know two sides in each triangle.

 

\item There is not enough information given. Although we can work out which angles are equal we are not given any sides as equal. All we know is that we have two isosceles triangles. Note how this question differs from part a). In part a) we were given equal sides in both triangles, in this question we are only given that sides in the same triangle are equal.
      \item $\triangle ABD ||| \triangle ADC$ (AAS: $CA$ is a common side, and two angles are given as being equal.)
      \end{enumerate}
\end{enumerate}}
\end{solutions}


% \section{Quadrilaterals}
% \subsection{Parallelogram}
\begin{exercises}{}{
\  \begin{enumerate}[itemsep=5pt, label=\textbf{\arabic*}. ]
 \item Prove that the diagonals of the parallelogram $MNRS$ bisect one another at $P$. \\
\begin{center}
\scalebox{1} % Change this value to rescale the drawing.
{
\begin{pspicture}(0,-1.821875)(3.6975,1.821875)
\pspolygon[linewidth=0.04](0.3540625,1.5034375)(1.3540626,-1.4765625)(3.3540626,-1.4765625)(2.3540626,1.5034375)
\psline[linewidth=0.04cm](0.3940625,1.4834375)(3.3340626,-1.4765625)
\rput(2.0848436,0.0334375){$P$}
\rput(0.14734375,1.6134375){$M$}
\rput(2.6439064,1.6534375){$N$}
\rput(3.5326562,-1.6465625){$R$}
\rput(1.1932813,-1.6665626){$S$}
\psline[linewidth=0.01cm,arrowsize=0.2cm 2.0,arrowlength=1.4,arrowinset=0.5]{->}(2.2140625,-1.4965625)(2.4740624,-1.4965625)
\psline[linewidth=0.01cm,arrowsize=0.2cm 2.0,arrowlength=1.4,arrowinset=0.5]{->}(1.1740625,1.5034375)(1.4340625,1.5034375)
\psline[linewidth=0.01cm,arrowsize=0.2cm 2.0,arrowlength=1.4,arrowinset=0.5]{->>}(0.8340625,0.0834375)(0.7540625,0.3634375)
\psline[linewidth=0.01cm,arrowsize=0.2cm 2.0,arrowlength=1.4,arrowinset=0.5]{->>}(2.8340626,0.2034375)(2.7140625,0.4834375)
\psline[linewidth=0.04cm](1.3340625,-1.4565625)(2.3540626,1.5234375)
\end{pspicture} 
}
\end{center}
\\
Hint: Use congruency.
\end{enumerate}

}
\end{exercises}


 \begin{solutions}{}{
\begin{enumerate}[itemsep=5pt, label=\textbf{\arabic*}. ] 


\item First number each angle on the given diagram:\\

\scalebox{1} % Change this value to rescale the drawing.
{
\begin{pspicture}(0,-1.8631251)(4.104375,1.8631251)
\pspolygon[linewidth=0.04](0.55671877,1.5096875)(1.5567188,-1.4703125)(3.5567188,-1.4703125)(2.5567188,1.5096875)
\psline[linewidth=0.04cm](0.5967187,1.4896876)(3.5367188,-1.4703125)
\usefont{T1}{ptm}{m}{n}
\rput(1.9420311,0.5396876){$P$}
\usefont{T1}{ptm}{m}{n}
\rput(0.32453126,1.6196876){$M$}
\usefont{T1}{ptm}{m}{n}
\rput(2.8210938,1.6596875){$N$}
\usefont{T1}{ptm}{m}{n}
\rput(3.7098436,-1.6403124){$R$}
\usefont{T1}{ptm}{m}{n}
\rput(1.3704689,-1.6603125){$S$}
\psline[linewidth=0.01cm,arrowsize=0.2cm 2.0,arrowlength=1.4,arrowinset=0.5]{->}(2.4167187,-1.4903125)(2.6767187,-1.4903125)
\psline[linewidth=0.01cm,arrowsize=0.2cm 2.0,arrowlength=1.4,arrowinset=0.5]{->}(1.3767188,1.5096875)(1.6367188,1.5096875)
\psline[linewidth=0.01cm,arrowsize=0.2cm 2.0,arrowlength=1.4,arrowinset=0.5]{->>}(1.0367187,0.08968755)(0.95671874,0.36968756)
\psline[linewidth=0.01cm,arrowsize=0.2cm 2.0,arrowlength=1.4,arrowinset=0.5]{->>}(3.0367188,0.20968755)(2.9167187,0.48968756)
\psline[linewidth=0.04cm](1.5367187,-1.4503125)(2.5567188,1.5296875)
\usefont{T1}{ptm}{m}{n}
\rput(0.9678125,1.273125){\tiny $1$}
\usefont{T1}{ptm}{m}{n}
\rput(2.5678124,1.1131251){\tiny $1$}
\usefont{T1}{ptm}{m}{n}
\rput(1.9478126,0.29312506){\tiny $1$}
\usefont{T1}{ptm}{m}{n}
\rput(3.1878126,-1.306875){\tiny $1$}
\usefont{T1}{ptm}{m}{n}
\rput(1.4878125,-1.0268749){\tiny $1$}
\usefont{T1}{ptm}{m}{n}
\rput(2.2507813,0.09312505){\tiny $2$}
\usefont{T1}{ptm}{m}{n}
\rput(2.3107812,1.3531251){\tiny $2$}
\usefont{T1}{ptm}{m}{n}
\rput(0.8307812,1.053125){\tiny $2$}
\usefont{T1}{ptm}{m}{n}
\rput(1.7507813,-1.286875){\tiny $2$}
\usefont{T1}{ptm}{m}{n}
\rput(3.2907813,-1.0068749){\tiny $2$}
\usefont{T1}{ptm}{m}{n}
\rput(2.0728126,-0.20687495){\tiny $3$}
\usefont{T1}{ptm}{m}{n}
\rput(1.8584375,0.01312505){\tiny $4$}
\end{pspicture} 
}\\
In $\triangle MNP$ and $\triangle RSP$\\
$\hat{M_1} = \hat{R_1}$ (alt $\angle$'s)\\
$\hat{P_1} = \hat{P_3}$ ($MN \parallel SR$, vert. opp. $\angle$'s)\\
$MN=RS$ (opp. sides of parallelogram)\\
$\therefore \triangle MNP ||| \triangle RSP$ (AAS)\\
$\therefore MP = RP$\\
$\therefore P$ is the mid-point of $MR$\\
\\
Similarly, in $\triangle MSP$ and $\triangle RNP$\\
$\hat{M_2} = \hat{R_2}$ (alt $\angle$'s)\\
$\hat{P_4} = \hat{P_2}$ ($MN \parallel SR$, vert. opp. $\angle$'s)\\
$MS=RN$ (opp. sides of parallelogram)\\
$\therefore \triangle MSP ||| \triangle RNP$ (AAS)\\
$\therefore SP=NP$\\
$\therefore P$ is the mid-point of $NS$\\
Therefore the diagonals bisect each other at point $P$. 

\end{enumerate}}
\end{solutions}


% \subsection{Rectangle}
\begin{exercises}{}
{
  \begin{enumerate}[itemsep=5pt, label=\textbf{\arabic*}. ]
   \item
$ABCD$ is a quadrilateral. Diagonals $AC$ and $BD$ intersect at $T$. $AC = BD$, $AT=TC$, $DT=TB$. Prove that:
\begin{enumerate}[noitemsep, label=\textbf{(\alph*)} ]
\item $ABCD$ is a parallelogram.
\item $ABCD$ is a rectangle.
\end{enumerate}
\scalebox{.8} % Change this value to rescale the drawing.
{
\begin{pspicture}(0,-2.8645313)(3.7246876,2.8645313)
\psframe[linewidth=0.04,dimen=outer](3.3771875,2.4607813)(0.2571875,-2.4992187)
\rput(3.564375,-2.7092187){$C$}
\rput(0.12375,2.6907814){$A$}
\rput(0.1571875,-2.7092187){$D$}
\psline[linewidth=0.04cm](0.2771875,2.4407814)(3.3571875,-2.4992187)
\psline[linewidth=0.04cm](0.2971875,-2.4792187)(3.3371875,2.4407814)
\rput(3.424375,2.6907814){$B$}
\rput(1.4171875,0.01078125){$T$}
\end{pspicture} 
}
}
\end{enumerate}

}
\end{exercises}


 \begin{solutions}{}{
\begin{enumerate}[itemsep=5pt, label=\textbf{\arabic*}. ] 


\item 
      \begin{enumerate}[noitemsep, label=\textbf{(\alph*)} ]
      \item 
$AT=TC$ (given)\\
$\therefore DB$ bisects $AC$ at $T$\\
and $DT=TB$ (given)\\
$\therefore AC$ bisects $DB$ at $T$\\
therefore quadrilateral $ABCD$ is a parallelogram (diagonals bisect each other)
%$ABCD$ is a parallelogram.
      \item 
$AC=BD$ (given)\\
Therefore $ABCD$ is a rectangle (diagonals are of equal length)
%$ABCD$ is a rectangle.
      \end{enumerate}

\end{enumerate}}
\end{solutions}


% \subsection{Rhombus}
% \subsection{Square}
% \subsection{Trapezium}
% \subsection{Kite}
\begin{exercises}{}
{
\begin{enumerate}[itemsep=10pt, label=\textbf{\arabic*}.]
 \item Use the sketch of kite $ABCD$ to prove the diagonals are perpendicular to one another.\\
 \scalebox{1} % Change this value to rescale the drawing.
{
\begin{pspicture}(0,-1.3692187)(4.6646876,1.3692187)
\psline[linewidth=0.04cm](0.2971875,0.05078125)(4.2571874,0.01078125)
\psline[linewidth=0.04cm](1.2571875,-0.96921873)(1.2771875,0.99078125)
\psline[linewidth=0.04cm](0.7571875,-0.34921876)(0.6371875,-0.46921876)
\psline[linewidth=0.04cm](0.6771875,0.57078123)(0.7771875,0.41078126)
\psline[linewidth=0.04cm](2.3947396,-0.5207606)(2.4519124,-0.696064)
\rput(1.2339063,0.02078125){$O$}
\pspolygon[linewidth=0.04](0.2571875,0.05078125)(1.2771875,1.0107813)(4.2971873,0.01078125)(1.2571875,-1.0092187)
\rput(0.12375,0.06078125){$A$}
\rput(1.264375,1.2007812){$B$}
\rput(4.504375,0.02078125){$C$}
\rput(1.2571875,-1.2192187){$D$}
\psline[linewidth=0.04cm](2.3424625,-0.54237354)(2.3996353,-0.7176769)
\psline[linewidth=0.04cm](2.4469252,0.7103273)(2.359111,0.5481894)
\psline[linewidth=0.04cm](2.395264,0.73337305)(2.3074498,0.5712352)
\psdots[dotsize=0.12](0.5371875,0.17078125)
\psdots[dotsize=0.12](0.5371875,-0.06921875)
\rput(3.589375,0.13){\footnotesize $x$}
\rput(3.589375,-0.08421875){\footnotesize $x$}
\end{pspicture} 
}
\item
Explain why quadrilateral $WXYZ$ is a kite. Write down all the properties of quadrilateral $WXYZ$.\\
\scalebox{1} % Change this value to rescale the drawing.
{
\begin{pspicture}(0,-1.3292187)(4.79,1.3292187)
\psline[linewidth=0.04cm](2.4753125,-0.42921874)(2.4553125,0.99078125)
\pspolygon[linewidth=0.04](0.4953125,-1.0092187)(2.4353125,0.99078125)(4.4353123,-1.0092187)(2.4753125,-0.44921875)
\rput(0.16140625,-1.1192187){$W$}
\rput(2.481875,1.1607813){$X$}
\rput(2.4179688,-0.77921873){$Z$}
\psline[linewidth=0.04cm](3.4050503,-0.6296727)(3.3172362,-0.7918106)
\psline[linewidth=0.04cm](3.3533888,-0.6066269)(3.2655747,-0.7687648)
\psline[linewidth=0.04cm,tbarsize=0.07055555cm 5.0]{-|*}(1.4753125,-0.00921875)(1.5553125,0.09078125)
\psline[linewidth=0.04cm,tbarsize=0.07055555cm 5.0]{-|*}(3.3953125,0.03078125)(3.2753124,0.15078124)
\psline[linewidth=0.04cm](1.7328646,-0.5807606)(1.7900374,-0.756064)
\psline[linewidth=0.04cm](1.6805876,-0.60237354)(1.7377604,-0.7776769)
\rput(4.6209373,-1.1792188){$Y$}
\rput(2.3439062,0.6357812){\tiny $1$}
\rput(2.6034374,0.6357812){\tiny $2$}
\rput(3.9367187,-0.68421876){\tiny $3$}
\rput(2.6648438,-0.28421876){\tiny $4$}
\rput(2.2978125,-0.28421876){\tiny $5$}
\rput(1.0414063,-0.68421876){\tiny $6$}
\end{pspicture} 
}
\end{enumerate}

}
\end{exercises}


 \begin{solutions}{}{
\begin{enumerate}[itemsep=5pt, label=\textbf{\arabic*}. ] 


\item 
First number the angles:\\
\scalebox{1} % Change this value to rescale the drawing.
{
\begin{pspicture}(0,-1.4131249)(5.0596876,1.4131249)
\psline[linewidth=0.04cm](0.4834375,0.0596875)(4.4434376,0.0196875)
\psline[linewidth=0.04cm](1.4434375,-0.9603125)(1.4634376,0.9996875)
\psline[linewidth=0.04cm](0.9434375,-0.3403125)(0.8234375,-0.46031252)
\psline[linewidth=0.04cm](0.8634375,0.5796875)(0.9634375,0.4196875)
\psline[linewidth=0.04cm](2.5809896,-0.51185435)(2.6381624,-0.68715775)
\usefont{T1}{ptm}{m}{n}
\rput(1.7946875,0.3696875){$O$}
\pspolygon[linewidth=0.04](0.4434375,0.0596875)(1.4634376,1.0196875)(4.4834375,0.0196875)(1.4434375,-1.0003124)
\usefont{T1}{ptm}{m}{n}
\rput(0.28453124,0.0696875){$A$}
\usefont{T1}{ptm}{m}{n}
\rput(1.4251562,1.2096875){$B$}
\usefont{T1}{ptm}{m}{n}
\rput(4.6651564,0.0296875){$C$}
\usefont{T1}{ptm}{m}{n}
\rput(1.4179688,-1.2103125){$D$}
\psline[linewidth=0.04cm](2.5287125,-0.5334673)(2.5858853,-0.70877063)
\psline[linewidth=0.04cm](2.6331751,0.7192336)(2.545361,0.55709565)
\psline[linewidth=0.04cm](2.581514,0.7422793)(2.4936998,0.5801414)
\psdots[dotsize=0.12](0.7234375,0.1796875)
\psdots[dotsize=0.12](0.7234375,-0.0603125)
\usefont{T1}{ptm}{m}{n}
\rput(3.7509375,0.13890626){\footnotesize $x$}
\usefont{T1}{ptm}{m}{n}
\rput(3.7509375,-0.0753125){\footnotesize $x$}
\usefont{T1}{ptm}{m}{n}
\rput(1.5678124,0.16312495){\tiny $1$}
\usefont{T1}{ptm}{m}{n}
\rput(1.5707812,-0.09687505){\tiny $2$}
\usefont{T1}{ptm}{m}{n}
\rput(1.3328125,-0.09687505){\tiny $3$}
\usefont{T1}{ptm}{m}{n}
\rput(1.3584375,0.18312494){\tiny $4$}
\end{pspicture} 
}
\\
In $\triangle ADB$\\
let $\hat{A_1} = \hat{A_2} = t\\
$and let $\hat{D} = \hat{B} = p\\
\therefore $ in $ \triangle ADB\\
2t + 2p = 180^{\circ}\\
\therefore t+p = 90^{\circ}$\\
But $\hat{O_1} = \hat{B} + \hat{A_1}$ (ext. $\angle = $ sum of two opp. int. $\angle$'s)\\
$\hat{O_1} = p +t\\
= 90^{\circ}\\
\therefore AC \perp BD$
\item 
\scalebox{1} % Change this value to rescale the drawing.
{
\begin{pspicture}(0,-1.3731251)(5.1985936,1.3731251)
\psline[linewidth=0.04cm](2.6939063,-0.4203125)(2.6739063,0.9996875)
\pspolygon[linewidth=0.04](0.7139062,-1.0003124)(2.6539063,0.9996875)(4.653906,-1.0003124)(2.6939063,-0.4403125)
\usefont{T1}{ptm}{m}{n}
\rput(0.35453126,-1.1103125){$W$}
\usefont{T1}{ptm}{m}{n}
\rput(2.675,1.1696875){$X$}
\usefont{T1}{ptm}{m}{n}
\rput(2.8310938,-0.6503125){$Z$}
\psline[linewidth=0.04cm](3.623644,-0.62076646)(3.53583,-0.7829043)
\psline[linewidth=0.04cm](3.5719826,-0.5977206)(3.4841685,-0.75985855)
\psline[linewidth=0.04cm,tbarsize=0.07055555cm 5.0]{-|*}(1.6939063,0.0)(1.7739062,0.0996875)
\psline[linewidth=0.04cm,tbarsize=0.07055555cm 5.0]{-|*}(3.6139061,0.0396875)(3.4939063,0.15968749)
\psline[linewidth=0.04cm](1.9514583,-0.57185435)(2.0086312,-0.74715775)
\psline[linewidth=0.04cm](1.8991814,-0.5934673)(1.9563541,-0.76877064)
\usefont{T1}{ptm}{m}{n}
\rput(4.814062,-1.1703125){$Y$}
\usefont{T1}{ptm}{m}{n}
\rput(2.5442188,0.6446875){\tiny $1$}
\usefont{T1}{ptm}{m}{n}
\rput(2.8037498,0.6446875){\tiny $2$}
\usefont{T1}{ptm}{m}{n}
\rput(4.137031,-0.6753125){\tiny $3$}
\usefont{T1}{ptm}{m}{n}
\rput(2.8451562,-0.2753125){\tiny $4$}
\usefont{T1}{ptm}{m}{n}
\rput(2.498125,-0.2753125){\tiny $5$}
\usefont{T1}{ptm}{m}{n}
\rput(1.2417188,-0.6753125){\tiny $6$}
\psline[linewidth=0.04cm,linestyle=dashed,dash=0.16cm 0.1cm](0.7,-1.0268749)(4.66,-1.0068749)
\psline[linewidth=0.04cm,linestyle=dashed,dash=0.16cm 0.1cm](2.7,-0.42687494)(2.7,-0.96687496)
\psframe[linewidth=0.02,dimen=outer](2.86,-0.826875)(2.68,-1.0068749)
\psline[linewidth=0.03](1.8,-1.166875)(1.8733333,-0.946875)(1.9466667,-1.166875)(2.02,-0.946875)(2.02,-0.946875)
\psline[linewidth=0.03](3.38,-1.1468749)(3.4533334,-0.92687494)(3.5266666,-1.1468749)(3.6,-0.92687494)(3.6,-0.92687494)
\end{pspicture} 
}\\
Quadrilateral $WXYZ$ is a kite because is has two pairs of adjacent sides that are equal in length.\\
\begin{itemize}[noitemsep]
\item $\hat{W_1} = \hat{Y_1}$ (opp. angles equal)
\item Diagonal between equal sides bisects the other diagonal: $WP+PY$
\item $\hat{X_1} = \hat{X_2}$ (diagonal bisects int. $\angle$)\\
$\hat{Y_1} = \hat{Y_2}$ (diagonal bisects int. $\angle$)
\item $WY \perp PX$ \end{itemize}
\end{enumerate}}
\end{solutions}


% \section{The mid-point theorem}
\begin{exercises}{}{
\begin{enumerate}[itemsep=6pt,label=\textbf{\arabic*}.]
\item 
Find $x$ and $y$ in each of the following:\\
\begin{tabular}{c m{3cm} c m{3cm}} 
\textbf{(a)} &
\raisebox{-1.5\height}{\scalebox{1} % Change this value to rescale the drawing.
{
\begin{pspicture}(0,-1.06)(3.06,1.06)
\pspolygon[linewidth=0.04](0.0,-0.92)(2.02,1.04)(3.04,-0.94)
\psline[linewidth=0.04cm](1.02,0.08)(1.58,-0.92)
\psline[linewidth=0.04cm](0.72,-0.82)(0.72,-1.02)
\psline[linewidth=0.04cm](0.78,-0.82)(0.78,-1.02)
\psline[linewidth=0.04cm](2.18,-0.84)(2.18,-1.04)
\psline[linewidth=0.04cm](2.24,-0.86)(2.24,-1.04)
\psline[linewidth=0.04cm](0.44,-0.34)(0.54,-0.5)
\psline[linewidth=0.04cm](1.32,0.48)(1.44,0.34)
\rput(1.1153125,-0.53){$7$}
\rput(2.4654686,-0.15){$x$}
\end{pspicture} 
}}
& \textbf{(b)} &
\raisebox{-1.5\height}{\scalebox{1} % Change this value to rescale the drawing.
{
\begin{pspicture}(0,-1.1392188)(3.0,1.1592188)
\pspolygon[linewidth=0.04](1.42,-1.1192187)(0.0,0.78078127)(2.98,0.7607812)
\psline[linewidth=0.04cm](0.74,-0.21921875)(2.16,-0.21921875)
\psline[linewidth=0.04cm](1.168575,-0.61497647)(1.0252811,-0.7545)
\psline[linewidth=0.04cm](1.1961027,-0.671917)(1.0671381,-0.7974882)
\psline[linewidth=0.04cm](1.7,-0.63921875)(1.8,-0.7992188)
\psline[linewidth=0.04cm](2.42,0.24078125)(2.54,0.10078125)
\rput(1.4953125,0.99078125){$7$}
\rput(1.393125,-0.02921875){$x$}
\psline[linewidth=0.04cm](0.48857507,0.28502357)(0.34528103,0.14550002)
\psline[linewidth=0.04cm](0.51610273,0.228083)(0.3871381,0.1025118)
\end{pspicture} 
}}
\\
\textbf{(c)} & 
\raisebox{-1.5\height}{\scalebox{1} % Change this value to rescale the drawing.
{
\begin{pspicture}(0,-1.2821875)(2.58625,1.2621875)
\pspolygon[linewidth=0.04](0.02,-0.7378125)(2.02,1.2221875)(2.04,-0.7378125)
\psline[linewidth=0.04cm](1.0,-0.7178125)(2.04,0.2221875)
\psline[linewidth=0.04cm](2.114656,0.7046623)(1.9148401,0.6960835)
\psline[linewidth=0.04cm](2.097248,0.6438596)(1.9174137,0.6361387)
\psline[linewidth=0.04cm](1.52,-0.6378125)(1.52,-0.8178125)
\psline[linewidth=0.04cm](0.5,-0.6578125)(0.5,-0.8378125)
\rput(2.4553125,0.2121875){$6$}
\rput(1.3854687,-0.0478125){$x$}
\psline[linewidth=0.04cm](2.154234,-0.31388846)(1.9545573,-0.32525736)
\psline[linewidth=0.04cm](2.137677,-0.37492836)(1.957968,-0.38516036)
\rput(0.94875,-1.1278125){$8$}
\psline[linewidth=0.04cm](1.88,-0.7178125)(1.88,-0.5778125)
\psline[linewidth=0.04cm](1.88,-0.5778125)(2.02,-0.5778125)
\psline[linewidth=0.04cm,tbarsize=0.07055555cm 5.0]{|-|}(0.0,-0.9578125)(2.02,-0.9578125)
\psline[linewidth=0.04cm,tbarsize=0.07055555cm 5.0]{|-|}(2.3,1.2421875)(2.3,-0.7378125)
\end{pspicture} 
}}
& \textbf{(d)} &
\raisebox{-1.5\height}{\scalebox{1} % Change this value to rescale the drawing.
{
\begin{pspicture}(0,-1.3092188)(4.5684376,1.3092188)
\pspolygon[linewidth=0.04](0.2375,-1.0092187)(2.2375,0.9507812)(4.2575,-1.0092187)
\psline[linewidth=0.04cm](1.2375,-0.00921875)(3.2375,-0.00921875)
\psline[linewidth=0.04cm](1.8122703,0.40506744)(1.6488547,0.5203731)
\psline[linewidth=0.04cm](1.761337,0.36757335)(1.614263,0.47134843)
\psline[linewidth=0.04cm](3.804056,-0.43863574)(3.670944,-0.55980176)
\psline[linewidth=0.04cm](2.784056,0.56136423)(2.650944,0.44019824)
\rput(2.1301563,-0.21921875){$14$}
\rput(0.6653125,-0.77921873){$y$}
\psline[linewidth=0.04cm](0.83256006,-0.5741614)(0.6680387,-0.4604392)
\psline[linewidth=0.04cm](0.7819913,-0.6121456)(0.6339221,-0.5097956)
\rput(2.2029688,-1.1592188){$x$}
\rput(2.2482812,1.1407813){$P$}
\rput(0.89421874,0.08078125){$Q$}
\rput(3.4360938,0.08078125){$R$}
\rput(0.05671875,-1.0192188){$S$}
\rput(4.4175,-1.0792187){T}
\rput(2.229375,0.66578126){\footnotesize $40^{\circ}$}
\rput(2.825,0.12578125){\footnotesize $60^{\circ}$}
\end{pspicture} 
}}
\\
\textbf{(e)}&
\raisebox{-1.5\height}{\scalebox{1} % Change this value to rescale the drawing.
{
\begin{pspicture}(0,-1.1789062)(4.5721874,1.1589062)
\pspolygon[linewidth=0.04](0.5521875,-0.8410938)(0.5321875,1.1389062)(4.5521874,-0.86109376)
\psline[linewidth=0.04cm](0.5321875,0.13890626)(2.5321875,0.13890626)
\psline[linewidth=0.04cm](0.6321875,-0.42109376)(0.4456315,-0.41167676)
\psline[linewidth=0.04cm](0.6121875,0.5389063)(0.4656315,0.5483232)
\psline[linewidth=0.04cm](0.5521875,-0.70109373)(0.6921875,-0.70109373)
\psline[linewidth=0.04cm](0.6921875,-0.70109373)(0.6921875,-0.8410938)
\psline[linewidth=0.04cm](1.3721875,0.63890624)(1.4521875,0.7789062)
\psline[linewidth=0.04cm](1.4321876,0.61890626)(1.4921875,0.73890626)
\psline[linewidth=0.04cm](3.1521876,-0.26109374)(3.2321875,-0.10109375)
\psline[linewidth=0.04cm](3.2121875,-0.30109376)(3.3121874,-0.12109375)
\rput{61.486862}(1.2470814,-1.7252281){\psarc[linewidth=0.04](2.073843,0.18573713){0.20604318}{39.82307}{132.54938}}
\rput(2.0529687,0.25890625){\small $x$}
\rput(0.20640625,0.7889063){$2,5$}
\rput(3.8025,-0.23109375){$6,5$}
\rput(2.32,-1.2){$y$}
\rput(0.71921873,0.8539063){\footnotesize $66^{\circ}$}
\end{pspicture} 
}} & & \\
\end{tabular}
\item Show that $M$ is the mid-point of $AB$ and that $MN=RC$.\\
\scalebox{1} % Change this value to rescale the drawing.
{
\begin{pspicture}(0,-1.331875)(2.8815625,1.331875)
\pspolygon[linewidth=0.04](0.4740625,-0.951875)(0.4540625,1.028125)(2.5140624,-0.951875)
\psline[linewidth=0.04cm](0.4540625,0.028125)(1.4740624,0.028125)
\psline[linewidth=0.04cm](2.1006186,-0.42129198)(1.9675065,-0.542458)
\psline[linewidth=0.04cm](1.0806185,0.578708)(0.9475065,0.457542)
\rput(2.72125,-1.041875){$C$}
\rput(1.4326563,-1.181875){$R$}
\rput(0.300625,1.158125){$A$}
\rput(0.14734375,0.038125){$M$}
\rput(0.32125,-1.041875){$B$}
\rput(1.6439062,0.158125){$N$}
\psline[linewidth=0.04cm](1.4540625,0.048125)(1.4540625,-0.951875)
\psline[linewidth=0.04cm](0.4540625,0.168125)(0.6140625,0.168125)
\psline[linewidth=0.04cm](0.5940625,0.168125)(0.5940625,0.048125)
\psline[linewidth=0.04cm](0.4740625,-0.811875)(0.6140625,-0.811875)
\psline[linewidth=0.04cm](0.6140625,-0.811875)(0.6140625,-0.951875)
\psline[linewidth=0.04cm](1.4540625,-0.831875)(1.5940624,-0.831875)
\psline[linewidth=0.04cm](1.5940624,-0.831875)(1.5940624,-0.931875)
\end{pspicture} 
}
\item $ABCD$ is a rhombus with $AM = MO$ and $AN = NO$. Prove $ANOM$ is also a rhombus.\\
\\
\scalebox{1} % Change this value to rescale the drawing.
{
\begin{pspicture}(0,-1.2990191)(4.0170274,1.2994184)
\psdiamond[linewidth=0.04,dimen=outer,gangle=-30.75696](2.1794336,0.0)(1.8712169,1.1173581)
\psline[linewidth=0.04cm](0.58952755,0.9609809)(3.7495275,-0.9390191)
\psline[linewidth=0.04cm](1.6095276,-0.9390191)(2.7295275,0.9409809)
\psline[linewidth=0.04cm](2.1495275,0.0)(1.0895276,0.0)
\psline[linewidth=0.04cm](2.1495275,0.0)(1.7095276,0.9609809)
\psline[linewidth=0.04cm](1.3295276,1.0409809)(1.3495276,0.86098087)
\psline[linewidth=0.04cm](1.8295276,0.5009809)(2.0095277,0.5409809)
\psline[linewidth=0.04cm](0.7895276,0.3609809)(0.9895276,0.4009809)
\psline[linewidth=0.04cm](0.82952756,0.3009809)(1.0095276,0.3409809)
\psline[linewidth=0.04cm](1.4895276,0.0809809)(1.5295275,-0.07901911)
\psline[linewidth=0.04cm](1.5695276,0.0809809)(1.5895276,-0.07901911)
\rput(0.85812134,0.6459809){\tiny $1$}
\rput(1.0376526,0.8059809){\tiny $2$}
\rput(3.3181214,-0.8340191){\tiny $1$}
\rput(3.4176526,-0.5940191){\tiny $2$}
\rput(0.45609006,1.0709809){$A$}
\rput(2.8967152,1.1309808){$B$}
\rput(3.856715,-1.0290191){$C$}
\rput(1.3895276,-1.1490191){$D$}
\rput(0.8228088,-0.049019106){$M$}
\rput(1.7193713,1.1309808){$N$}
\rput(2.1462464,-0.009019107){$O$}
\end{pspicture} 
}
\end{enumerate}

}
\end{exercises}


 \begin{solutions}{}{
\begin{enumerate}[itemsep=5pt, label=\textbf{\arabic*}. ] 


\item 
	    \begin{enumerate}[noitemsep, label=\textbf{(\alph*)} ]
		\item$x=14$ units
\item$x=3,5$ units
\item$x=5$ units
\item$x=28$ units; $y=80^{\circ}$
\item$x=24^{\circ}$;$y=12$ units
	    \end{enumerate}
\item $N$ is the mid-point of $AC$ (given: $AN=NC$)\\
and $\hat{B} = \hat{M} = 90^\circ$ (given)\\
But these are equal, corresponding angles\\
$\therfore MN \parallel BR$\\
$BM \parallel RN\\
\therefore M$ is the mid-point of line $AB$\\
\\
$\therefore MN = \frac{1}{2} BC$ (mid-point theorem)\\
but $MN=BR$ (opp. sides of parallelogram $MNRB$)\\
and $BL = BR+RC$\\
$\therefore MN = \frac{1}{2}(BR + RC)\\
2MN = MN + RC\\
\therefore MN=RC$
\item %solution 3
In $\triangle AMO$ and $\triangle ANO\\
\hat{A_1} = \hat{A_2}$ (given rhombus $ABCD$, diagonal $AC$ bisects $\hat{A}$)\\
$\therefore \hat{A_1} = \hat{O_1}$ ($\angle$'s opp. equal sides, given $AM=MO$ and $AN=NO$)\\
and $\hat{A_2} = \hat{O_2}$\\
$\therefore \hat{A_2} = \hat{O_1}$ and $\hat{A_1} = \hat{O_2}$\\
but these are alt. $\angle$'s\\
$\therefore AN \parallel MO$ and $AM \parallel NO$\\
$\therefore AMON$ is a parallelogram\\
$\therefore AM = NO$ (opp. sides parm equal)\\
$\threrefore AM=MO=ON=NO\\
\therefore AMON$ is a rhombus (all sides equal and parallel)

\end{enumerate}}
\end{solutions}


% \section{Proofs and conjectures}
\begin{eocexercises}{}
\begin{enumerate}[itemsep=20pt, label=\textbf{\arabic*}.]
\item Identify the types of angles shown below:\\
\begin{center}
\begin{tabular}{lm{4cm}lm{3cm}}
\textbf{(a)} & 
\raisebox{-.5\height}{\scalebox{1} % Change this value to rescale the drawing.
{
\begin{pspicture}(0,-0.04)(4.02,0.3)
\psline[linewidth=0.04cm](0.0,-0.02)(4.0,-0.02)
\psarc[linewidth=0.04](1.82,0.0){0.28}{0.0}{180.0}
\end{pspicture}} 
}
& \textbf{(b)} &
\raisebox{-.8\height}{\scalebox{1} % Change this value to rescale the drawing.
{
\begin{pspicture}(0,-0.31)(3.18,0.71)
\psarc[linewidth=0.04](0.98,-0.29){0.4}{0.0}{123.69007}
\psline[linewidth=0.04](0.0,0.69)(1.1,-0.25)(3.16,-0.25)
\end{pspicture} }
}
\\
\textbf{(c)} & 
\raisebox{-.8\height}{\scalebox{1} % Change this value to rescale the drawing.
{
\begin{pspicture}(0,-0.31)(2.36,0.75)
\psarc[linewidth=0.04](0.44,-0.29){0.44}{0.0}{66.80141}
\psline[linewidth=0.04](1.2,0.73)(0.28,-0.29)(2.34,-0.29)
\end{pspicture} }
}
& \textbf{(d)} &
\raisebox{-.8\height}{\scalebox{1} % Change this value to rescale the drawing.
{
\begin{pspicture}(0,-0.66)(2.08,0.66)
\psline[linewidth=0.04](0.0,0.64)(0.0,-0.62)(2.06,-0.62)
\psline[linewidth=0.04cm](0.0,-0.4)(0.2,-0.4)
\psline[linewidth=0.04cm](0.2,-0.4)(0.2,-0.64)
\end{pspicture} 
}}
\\ 
\textbf{(e)} &
\raisebox{-.8\height}{\scalebox{1} % Change this value to rescale the drawing.
{
\begin{pspicture}(0,-0.62)(3.1,0.62)
\psline[linewidth=0.04](0.0,-0.6)(1.02,0.28)(3.08,0.28)
\psarc[linewidth=0.04](1.0,0.3){0.3}{0.0}{225.0}
\end{pspicture} 
}}
& \textbf{(f)} & An angle of $91^\circ$  \\
\textbf{(g)} & An angle of $180^\circ$ & \textbf{(h)} & An angle of $210^\circ$ \\
\end{tabular}\\
\end{center}
\item Assess whether the following statements are true or false. If the
statement is false, explain why:
   \begin{enumerate}[noitemsep, label=\textbf{(\alph*)} ]
\item  A trapezium is a quadrilateral with two pairs of opposite sides that are parallel.
\item  Both diagonals of a parallelogram bisect each other.
\item  A rectangle is a parallelogram that has one corner angles equal to $90^{\circ}$.
\item  Two adjacent sides of a rhombus have different lengths.
\item  The diagonals of a kite intersect at right angles.
\item All squares are parallelograms.
\item A rhombus is a kite with a pair of equal, opposite sides.
\item The diagonals of a parallelogram are axes of symmetry.
\item The diagonals of a rhombus are equal in length.
\item Both diagonals of a kite bisect the interior angles.
\end{enumerate}
\item Calculate the size of the third angle ($x$) in each of the diagrams below:\\
\begin{center}
\begin{tabular}{lm{4.5cm}lm{4cm}}
\textbf{(a)} & \raisebox{-1.5\height}{\scalebox{1} % Change this value to rescale the drawing.
{
\begin{pspicture}(0,-1.0204266)(4.2938037,1.0995734)
\pspolygon[linewidth=0.04](0.23380375,-1.0004267)(0.23380375,1.0595734)(4.2738037,1.0795734)
\psline[linewidth=0.04cm](0.23380375,0.8795734)(0.47380376,0.8795734)
\psline[linewidth=0.04cm](0.47380376,0.8795734)(0.47380376,1.0795734)
\rput(0.5,-0.6854266){\footnotesize $65^{\circ}$}
\rput(3.6792724,0.9){$x$}
\rput{107.26479}(5.5713153,-2.3035204){\psarc[linewidth=0.04](3.6338038,0.8995734){0.26}{32.92963}{140.93379}}
\rput{-20.206701}(0.3015834,0.09886189){\psarc[linewidth=0.04](0.42820147,-0.79682213){0.33352643}{32.92963}{140.93379}}
\end{pspicture} 
} }
& \textbf{(b)} &
\raisebox{-1\height}{\scalebox{1} % Change this value to rescale the drawing.
{
\begin{pspicture}(0,-0.57905805)(4.94,0.440942)
\pspolygon[linewidth=0.04](0.0,-0.559058)(2.92,0.42094198)(4.92,-0.539058)
\rput(0.97078127,-0.40405804){\footnotesize $15^{\circ}$}
\rput(2.9054687,0.26){$x$}
\rput{107.26479}(5.2358394,-4.604439){\psarc[linewidth=0.04](4.3132534,-0.37440655){0.35332903}{27.735203}{103.69896}}
\rput{-20.206701}(0.22287409,0.3201259){\psarc[linewidth=0.04](1.009721,-0.4653292){0.28314972}{1.2905142}{89.806305}}
\rput(4.23,-0.40405804){\footnotesize $20^{\circ}$}
\rput{197.50908}(5.536452,1.6237775){\psarc[linewidth=0.04](2.8932536,0.38559344){0.35332903}{357.98285}{146.57666}}
\end{pspicture} 
} }
\\
\textbf{(c)} &
\raisebox{-1.5\height}{\scalebox{1} % Change this value to rescale the drawing.
{
\begin{pspicture}(0,-1.4)(4.3225,1.38)
\pspolygon[linewidth=0.04](0.3425,-1.04)(0.3425,1.36)(4.3025,-1.0157576)
\rput(0.09328125,0.035){$15^{\circ}$}
\rput(2.7085938,0.275){ $25^{\circ}$}
\pspolygon[linewidth=0.04](0.8425,-0.64)(0.8425,0.4)(2.2625,-0.62949497)
\psline[linewidth=0.04cm](0.8225,-0.46)(1.0025,-0.46)
\psline[linewidth=0.04cm](1.0025,-0.46)(1.0025,-0.66)
\psline[linewidth=0.04cm](0.3425,-0.82)(0.5825,-0.82)
\psline[linewidth=0.04cm](0.5825,-0.82)(0.5825,-1.02)
\rput(2.0564063,-1.25){$2x$}
\rput(1.4184375,-0.465){\footnotesize $x$}
\rput(1.720625,0.055){\footnotesize $y$}
\end{pspicture} 
}}
& \textbf{(d)} &
\raisebox{-1.5\height}{\scalebox{1} % Change this value to rescale the drawing.
{
\begin{pspicture}(0,-1.3538659)(3.72,1.8261342)
\pspolygon[linewidth=0.04](0.0,-1.1938658)(2.04,1.8061342)(3.7,-1.1938658)
\rput(2.0075,1.3411342){\footnotesize $60^{\circ}$}
\rput(3.2854688,-0.9638658){$x$}
\psline[linewidth=0.04cm](2.72,0.26613417)(2.96,0.38613418)
\psline[linewidth=0.04cm](1.84,-1.0538658)(1.84,-1.3338659)
\rput{167.76651}(4.302379,2.382765){\psarc[linewidth=0.04](2.0235155,1.4219141){0.32604876}{32.458344}{174.69376}}
\rput{37.977333}(0.042277616,-2.4176195){\psarc[linewidth=0.04](3.5340343,-1.1473787){0.47183454}{66.86393}{149.30316}}
\end{pspicture} 
}} \\
\textbf{(e)} &
\raisebox{-1\height}{\scalebox{1} % Change this value to rescale the drawing.
{
\begin{pspicture}(0,-0.56560516)(4.1,0.45439485)
\pspolygon[linewidth=0.04](0.0,0.43439484)(4.08,0.43439484)(2.04,-0.5456052)
\rput(2.0515625,-0.28){\footnotesize $3x$}
\rput(0.6454688,0.3){$x$}
\rput{-13.787732}(0.16189782,0.47940308){\psarc[linewidth=0.04](2.0635154,-0.42982522){0.32604876}{25.533592}{192.70609}}
\rput{-131.79404}(0.7000065,0.93765175){\psarc[linewidth=0.04](0.5597484,0.31224003){0.362567}{82.89441}{149.30316}}
\psline[linewidth=0.04cm](1.14,-0.20560516)(1.28,-0.08560516)
\psline[linewidth=0.04cm](2.84,-0.005605158)(2.94,-0.20560516)
\end{pspicture} 
}}
& \textbf{(f)} &
\raisebox{-1.5\height}{\scalebox{1} % Change this value to rescale the drawing.
{
\begin{pspicture}(0,-1.2526648)(4.36,1.2317102)
\pspolygon[linewidth=0.04](0.0,-0.90828985)(2.5,-0.90828985)(1.34,1.2117101)
\pspolygon[linewidth=0.04](2.78,0.6917102)(4.34,0.6917102)(3.572,-0.64828986)
\psarc[linewidth=0.04](3.58,-0.46828985){0.2}{33.690067}{153.43495}
\rput{179.72585}(2.6254327,1.9153188){\psarc[linewidth=0.04](1.3104252,0.9608001){0.25066817}{33.690067}{153.43495}}
\psdots[dotsize=0.12](2.98,0.59171015)
\psdots[dotsize=0.12](2.24,-0.76828986)
\rput(3.5054688,0.90171015){$x$}
\rput(2.1678126,0.22171015){$y$}
\rput(0.31140625,0.14171015){$12$}
\rput(1.26875,-1.0982898){$8$}
\rput(4.135156,-0.058289852){$9$}
\rput(2.8,-0.07828985){$7,5$}
\end{pspicture} 
}}
\\ 
\\
\end{tabular}
\end{center}
\item Find all the pairs of parallel lines in the following figures, giving reasons in each case.
\begin{center}
\begin{tabular}{lm{4.5cm}lm{4.5cm}}
\textbf{(a)} & 
\raisebox{-1.5\height}{
\scalebox{0.8}{
\begin{pspicture}(0,-1.6034375)(5.1659374,1.6034375) 
\psline[linewidth=0.04cm](0.476875,1.203125)(4.716875,1.203125) 
\psline[linewidth=0.04cm](0.476875,-1.096875)(4.716875,-1.096875) 
\psline[linewidth=0.04cm](0.476875,1.203125)(4.676875,-1.096875)
\rput(3.3,-0.766875){$62^{\circ}$}
\rput(2.0,0.853125){$62^{\circ}$}
\rput(0.1175,1.413125){$A$} 
\rput(4.97625,1.313125){$B$}
\rput(0.22578125,-1.406875){$D$} 
\rput(5.0021877,-1.446875){$C$} 
\end{pspicture} } 
 } 
& \textbf{(b)} &
\raisebox{-1.5\height}{
\scalebox{0.65}{
\begin{pspicture}(0,-2.7134376)(8.0,2.7134376) 
\psline[linewidth=0.04cm](0.64,2.153125)(1.94,-2.526875) 
\psline[linewidth=0.04cm](5.8,2.553125)(7.1,-2.126875) 
\psline[linewidth=0.04cm](0.0,1.513125)(7.24,2.153125) 
\psline[linewidth=0.04cm](0.74,-2.126875)(7.98,-1.486875)  
\rput(0.1615625,2.023125){\LARGE$M$} 
\rput(6.3395314,2.523125){\LARGE$N$} 
\rput(1.4215626,-2.556875){\LARGE$O$} 
\rput(7.5796876,-1.956875){\LARGE$P$} 
\rput(6.5,1.743125){\LARGE$57^{\circ}$}
\rput(6.6,-1.956875){\LARGE$123^{\circ}$} 
\rput(1.3,1.983125){\LARGE$137^{\circ}$} 
\end{pspicture}} }\\
\textbf{(c)} & 
\raisebox{-1\height}{\scalebox{0.7}{
\begin{pspicture}(0,-2.66)(7.791875,2.66) 
\psline[linewidth=0.04cm](0.7065625,1.72)(6.1865625,1.72) 
\psline[linewidth=0.04cm](0.0065625,-1.48)(5.5265627,-1.48) 
\psline[linewidth=0.04cm](1.5465626,2.64)(0.4665625,-2.44) 
\psline[linewidth=0.04cm](5.8465624,2.44)(4.7665625,-2.64)  
\rput(6.3,1.27){\LARGE$120^{\circ}$} 
\rput(1.2,-1.15){\LARGE$60^{\circ}$} 
\rput(4.6,-1.91){\LARGE$60^{\circ}$} 
\rput(1.051875,2.17){\LARGE$G$}
\rput(6.1254687,2.05){\LARGE$H$} 
\rput(0.06421875,-1.89){\LARGE$K$} 
\rput(5.443594,-1.97){\LARGE$L$} 
\end{pspicture} }}
& & \\
\end{tabular}
\end{center}
\item Find angles $a$, $b$, $c$ and $d$ in each case, giving reasons:\\
\begin{tabular}{lm{4.5cm}lm{4.5cm}}
\textbf{(a)} & 
\raisebox{-1.5\height}{\begin{pspicture}(0,-1.5101563)(4.01,1.5101563) 
\psline[linewidth=0.04cm](0.0,-0.15015624)(2.98,-1.1301563) 
\psline[linewidth=0.04cm](3.0,0.22984375)(0.96,1.1298437) 
\psline[linewidth=0.04cm](2.98,0.22984375)(2.98,-1.1501563) 
\psline[linewidth=0.04cm](0.98,1.1098437)(1.0,-0.47015625) 
\psline[linewidth=0.02cm](1.7,-0.5701563)(1.9,-0.7501562) 
\psline[linewidth=0.02cm](1.6,-0.85015625)(1.9,-0.79015625) 
\psline[linewidth=0.02cm](1.86,0.90984374)(2.08,0.64984375) 
\psline[linewidth=0.02cm](1.72,0.6298438)(2.02,0.6298438) 
\psline[linewidth=0.02cm](0.84,0.22984375)(0.98,-0.05015625) 
\psline[linewidth=0.02cm](0.98,-0.05015625)(0.98,-0.05015625) 
\psline[linewidth=0.02cm](0.98,-0.01015625)(1.12,0.20984375) 
\psline[linewidth=0.02cm](0.76,0.12984376)(0.98,-0.21015625) 
\psline[linewidth=0.02cm](0.98,-0.21015625)(1.18,0.12984376) 
\psline[linewidth=0.02cm](2.78,-0.37015626)(2.96,-0.71015626) 
\psline[linewidth=0.02cm](2.96,-0.71015626)(3.08,-0.37015626) 
\psline[linewidth=0.02cm](2.72,-0.43015626)(2.96,-0.83015627) 
\psline[linewidth=0.02cm](2.96,-0.83015627)(3.14,-0.43015626) 
\psline[linewidth=0.02cm](4.0,-1.0701562)(4.0,-1.0701562) 
\rput(0.9196875,1.3198438){$P$} 
\rput(3.1346874,0.35984376){$Q$} 
\rput(3.0753126,-1.3601563){$R$} 
\rput(0.89921874,-0.74015623){$S$} 
\rput(2.8010938,0.11484375){\small $a$} 
\rput(1.1614063,0.77484375){\small $b$} 
\rput(1.1696875,-0.34515625){\small $c$} 
\rput(0.76140624,-0.24515624){\small $d$} 
\rput(2.6165626,-0.79015625){\footnotesize $73^{\circ}$} 
\end{pspicture}} 
& \textbf{(b)} &  
\raisebox{-1.5\height}{
\begin{pspicture}(0,-2.0301561)(3.6778126,8.0301561) 
\psline[linewidth=0.04cm](0.2865625,1.1098437)(3.3065624,1.1098437) 
\psline[linewidth=0.04cm](0.2865625,0.08984375)(3.2665625,0.08984375) 
\psline[linewidth=0.04cm](0.2865625,-0.89015627)(3.2865624,-0.89015627) 
\rput(0.1271875,1.0998437){$A$} 
\rput(3.5059376,1.1398437){$B$} 
\rput(0.091875,0.03984375){$C$} 
 \rput(3.4954689,0.11984375){$D$} 
\rput(0.043125,-0.9401562){$E$} 
\rput(3.479375,-0.90015626){$F$} 
\psline[linewidth=0.04cm](2.1665626,1.6898438)(1.2265625,-1.5901562) 
 \rput(2.3242188,1.8398438){$K$} 
 \rput(2.1435938,0.83984375){$L$} 
 \rput(1.868125,-0.16015625){$M$} 
 \rput(1.2660937,-1.8801563){$O$} 
 \rput(1.608125,-1.1401563){$N$} 
 \rput(1.7076563,0.97484374){\small $a$} 
 \rput(1.9679687,0.27484375){\small $b$} 
 \rput(1.67625,-0.6851562){\small $c$} 
\rput(1.1479688,-1.1051563){\small $d$} 
\rput(1.5776563,1.3348438){\small $100^{\circ}$} 
\psline[linewidth=0.04cm](0.5665625,1.2898438)(0.9265625,1.1098437) 
\psline[linewidth=0.04cm](0.6065625,0.88984376)(0.8665625,1.1098437) 
\psline[linewidth=0.04cm](0.5665625,0.26984376)(0.8865625,0.10984375) 
\psline[linewidth=0.04cm](0.5865625,-0.11015625)(0.5865625,-0.11015625) 
\psline[linewidth=0.04cm](0.5865625,-0.11015625)(0.8465625,0.06984375) 
\psline[linewidth=0.04cm](0.5465625,-0.71015626)(0.8465625,-0.8701562) 
\psline[linewidth=0.04cm](0.5865625,-1.0901562)(0.8065625,-0.89015627) 
\end{pspicture}
} \\
\textbf{(c)} &  
\raisebox{-.5\height}{\begin{pspicture}(0,-1.8901563)(3.724375,1.8901563) 
\psline[linewidth=0.04cm](0.256875,0.50984377)(0.256875,-1.4901563) 
\psline[linewidth=0.04cm](3.256875,1.5098437)(3.296875,-0.45015624)
 \psline[linewidth=0.04cm](0.236875,-1.4501562)(3.296875,1.5498438) 
\psline[linewidth=0.04cm](0.256875,0.52984375)(3.276875,-0.45015624) 
\psline[linewidth=0.02cm](0.076875,-0.67015624)(0.236875,-0.21015625) 
\psline[linewidth=0.02cm](0.276875,-0.25015625)(0.416875,-0.67015624) 
\psline[linewidth=0.02cm](3.076875,0.20984375)(3.276875,0.64984375) 
\psline[linewidth=0.02cm](3.256875,0.6298438)(3.476875,0.24984375) 
\rput(0.10765625,0.63984376){$T$} 
\rput(1.7901562,-0.28015625){$U$} 
\rput(3.490625,-0.74015623){$V$} 
\rput(3.5304687,1.6998438){$W$} 
\rput(0.1634375,-1.7401563){$X$} 
\rput(3.0579689,-0.16515625){\small $a$} 
\rput(0.43828124,-0.9451563){\small $b$} 
\rput(1.3065625,-0.06515625){\small $d$} 
\rput(1.6782813,0.29484376){\small $c$} 
\rput(2.9854689,0.9348438){\small $45^{\circ}$} 
\rput(0.5106875,0.18984374){\footnotesize $50^{\circ}$} 
\end{pspicture}} & & \\
\end{tabular}\\
\item Say which of the following pairs of triangles are congruent with reasons. 
  \begin{enumerate}[itemsep=6pt, label=\textbf{(\alph*)} ]
\item 
\begin{center}
\begin{pspicture}(0,-1.7101562)(6.965625,1.7101562) 
\psline[linewidth=0.02cm](1.2465625,1.1498437)(0.2865625,-1.4101562) 
\psline[linewidth=0.02cm](0.3265625,-1.3901563)(2.5665624,-0.97015625) 
\psline[linewidth=0.02cm](1.2465625,1.1298437)(2.5065625,-0.9501563) 
\psline[linewidth=0.02cm](5.4865627,1.3098438)(4.4065623,-1.1101563) 
\psline[linewidth=0.02cm](4.4065623,-1.1101563)(6.6065626,-0.53015625) 
\psline[linewidth=0.02cm](5.4865627,1.2898438)(6.5665627,-0.5101563) 
\psline[linewidth=0.02cm](1.4265625,-1.0701562)(1.4865625,-1.2701563) 
\psline[linewidth=0.02cm](5.5665627,-0.65015626)(5.6465626,-0.97015625) 
\psline[linewidth=0.02cm](0.6065625,-0.07015625)(0.8465625,-0.15015624) 
\psline[linewidth=0.02cm](0.5465625,-0.17015626)(0.8065625,-0.27015626) 
\psline[linewidth=0.02cm](4.8265624,0.16984375)(5.0465627,0.06984375) 
\psline[linewidth=0.02cm](4.8865623,0.30984375)(5.1065626,0.18984374) 
\psline[linewidth=0.02cm](1.6465625,0.14984375)(1.9465625,0.30984375) 
\psline[linewidth=0.02cm](1.7065625,0.06984375)(2.0065625,0.20984375) 
\psline[linewidth=0.02cm](1.7665625,-0.03015625)(2.0665624,0.10984375) 
\psline[linewidth=0.02cm](5.8665624,0.46984375)(6.1065626,0.56984377) 
\psline[linewidth=0.02cm](5.9265623,0.36984375)(6.1465626,0.46984375) 
\psline[linewidth=0.02cm](5.9865627,0.24984375)(6.2065625,0.34984374) 
\rput(1.2671875,1.3598437){$A$} 
\rput(0.0459375,-1.5601562){$B$} 
\rput(2.751875,-1.1201563){$C$} 
\rput(5.4754686,1.5198437){$D$} 
\rput(4.203125,-1.2201562){$E$} 
\rput(6.819375,-0.60015625){$F$} 
\end{pspicture}
\end{center}
\item 
\begin{center}
\begin{pspicture}(0,-1.6132812)(6.9140625,1.6132812) 
\psline[linewidth=0.02cm](0.2665625,1.1867187)(0.2465625,-1.2932812) 
\psline[linewidth=0.02cm](0.2465625,-1.2932812)(2.4665625,-1.2532812)
\psline[linewidth=0.02cm](2.4665625,-1.2732812)(0.2665625,1.1867187) 
\psline[linewidth=0.02cm](3.2665625,0.7867187)(4.6065626,-0.8932812) 
\psline[linewidth=0.02cm](4.5865626,-0.8932812)(6.5865626,0.88671875) 
\psline[linewidth=0.02cm](3.2865624,0.7867187)(6.5665627,0.88671875) 
\psline[linewidth=0.02cm](0.2465625,-0.9532812)(0.6265625,-0.9532812) 
\psline[linewidth=0.02cm](0.6265625,-0.9532812)(0.6265625,-1.2532812) 
\psline[linewidth=0.02cm](4.4265623,-0.69328123)(4.6865625,-0.49328125) 
\psline[linewidth=0.02cm](4.6865625,-0.5132812)(4.8065624,-0.67328125) 
\psline[linewidth=0.02cm](1.3065625,-0.19328125)(1.5265625,0.02671875) 
\psline[linewidth=0.02cm](4.7065625,1.0467187)(4.7065625,0.68671876) 
\psline[linewidth=0.02cm](0.0665625,-0.01328125)(0.4665625,-0.01328125) 
\psline[linewidth=0.02cm](0.0665625,-0.17328125)(0.4865625,-0.17328125) 
\psline[linewidth=0.02cm](4.0865626,0.16671875)(3.7065625,-0.05328125) 
\psline[linewidth=0.02cm](4.1665626,0.06671875)(3.8065624,-0.15328126) 
\rput(0.131875,1.4167187){$G$} 
\rput(0.06546875,-1.4632813){$H$} 
\rput(2.585,-1.3232813){$I$} 
\rput(3.1226563,0.71671873){$J$} 
\rput(4.584219,-1.1832813){$K$} 
\rput(6.7635937,0.8567188){$L$} 
\end{pspicture}
\end{center}
\item 
\begin{center}
\begin{pspicture}(0,-1.6734375)(8.350625,1.6734375) 
\psline[linewidth=0.02cm](1.8665625,1.033125)(0.3465625,-1.346875) 
\psline[linewidth=0.02cm](0.3465625,-1.346875)(3.1265626,-1.286875) 
\psline[linewidth=0.02cm](3.1265626,-1.286875)(1.8665625,1.053125) 
\psline[linewidth=0.02cm](6.4265623,1.273125)(4.8065624,-1.166875) 
\psline[linewidth=0.02cm](4.8065624,-1.166875)(8.026563,-1.146875) 
\psline[linewidth=0.02cm](8.026563,-1.146875)(6.4265623,1.253125) 
\psarc[linewidth=0.02](8.246563,1.073125){0.0}{0.0}{180.0} 
\pscustom[linewidth=0.02] { \newpath \moveto(7.7465625,-0.746875) \lineto(7.6965623,-0.816875) \curveto(7.6715627,-0.851875)(7.6365623,-0.906875)(7.6265626,-0.926875) \curveto(7.6165624,-0.946875)(7.5915623,-0.991875)(7.5765624,-1.016875) \curveto(7.5615625,-1.041875)(7.5415626,-1.081875)(7.5265627,-1.126875) } 
\pscustom[linewidth=0.02] { \newpath \moveto(0.6265625,-0.886875) \lineto(0.6965625,-0.946875) \curveto(0.7315625,-0.976875)(0.7765625,-1.021875)(0.7865625,-1.036875) \curveto(0.7965625,-1.051875)(0.8165625,-1.086875)(0.8265625,-1.106875) \curveto(0.8365625,-1.126875)(0.8515625,-1.171875)(0.8565625,-1.196875) \curveto(0.8615625,-1.221875)(0.8565625,-1.261875)(0.8465625,-1.276875) \curveto(0.8365625,-1.291875)(0.8265625,-1.311875)(0.8265625,-1.326875) } 
\pscustom[linewidth=0.02] { \newpath \moveto(1.6665626,0.773125) \lineto(1.7365625,0.713125) \curveto(1.7715625,0.683125)(1.8315625,0.648125)(1.8565625,0.643125) \curveto(1.8815625,0.638125)(1.9365625,0.633125)(1.9665625,0.633125) \curveto(1.9965625,0.633125)(2.0365624,0.648125)(2.0665624,0.693125) } \pscustom[linewidth=0.02] { \newpath \moveto(1.6065625,0.593125) \lineto(1.6865625,0.543125) \curveto(1.7265625,0.518125)(1.7915626,0.488125)(1.8165625,0.483125) \curveto(1.8415625,0.478125)(1.8865625,0.468125)(1.9065624,0.463125) \curveto(1.9265625,0.458125)(1.9715625,0.453125)(1.9965625,0.453125) \curveto(2.0215626,0.453125)(2.0665624,0.463125)(2.0865624,0.473125) \curveto(2.1065626,0.483125)(2.1315625,0.493125)(2.1465626,0.493125) } 
\pscustom[linewidth=0.02] { \newpath \moveto(6.2465625,1.013125) \lineto(6.3265624,0.963125) \curveto(6.3665624,0.938125)(6.4315624,0.913125)(6.4565625,0.913125) \curveto(6.4815626,0.913125)(6.5315623,0.918125)(6.5565624,0.923125) \curveto(6.5815625,0.928125)(6.6115627,0.933125)(6.6265626,0.933125) } 
\pscustom[linewidth=0.02] { \newpath \moveto(6.1665626,0.833125) \lineto(6.2565627,0.813125) \curveto(6.3015623,0.803125)(6.3715625,0.793125)(6.3965626,0.793125) \curveto(6.4215627,0.793125)(6.4715624,0.793125)(6.4965625,0.793125) \curveto(6.5215626,0.793125)(6.5665627,0.798125)(6.5865626,0.803125) \curveto(6.6065626,0.808125)(6.6415625,0.818125)(6.6865625,0.833125) } \psline[linewidth=0.02cm](1.6865625,-1.086875)(1.6865625,-1.526875) \psline[linewidth=0.02cm](6.2065625,-0.906875)(6.1865625,-1.326875) 
\rput(0.088125,-1.476875){$M$} 
\rput(1.8460938,1.203125){$N$} 
\rput(3.288125,-1.516875){$O$} 
\rput(6.38625,1.483125){$P$} 
\rput(4.56125,-1.276875){$Q$}
 \rput(8.181875,-1.176875){$R$} 
\end{pspicture} 
\end{center}
\item 
\begin{center}
 \begin{pspicture}(0,-1.91625)(9.015312,1.91625) 
\psline[linewidth=0.02cm](2.1815624,0.8959375)(0.3215625,-1.6240625) 
\psline[linewidth=0.02cm](0.3215625,-1.6240625)(3.5215626,-1.2040625) 
\psline[linewidth=0.02cm](2.1815624,0.8759375)(3.4815626,-1.1840625) 
\psline[linewidth=0.02cm](7.1215625,1.5359375)(5.4615626,-1.2040625) 
\psline[linewidth=0.02cm](5.4615626,-1.2040625)(8.621563,-0.2040625) 
\psline[linewidth=0.02cm](8.621563,-0.2040625)(7.1215625,1.5359375) 
\psline[linewidth=0.02cm](1.3015625,-0.0040625)(1.6615624,-0.2040625) 
\psline[linewidth=0.02cm](6.2815623,0.5559375)(6.7615623,0.2559375) 
\psline[linewidth=0.02cm](1.7215625,-1.2040625)(1.7215625,-1.7040625) 
\psline[linewidth=0.02cm](1.9215626,-1.1840625)(1.9215626,-1.6240625) 
\psline[linewidth=0.02cm](6.6215625,-0.6240625)(6.7815623,-0.9640625) 
\psline[linewidth=0.02cm](6.8215623,-0.5640625)(7.0015626,-0.9040625) 
\pscustom[linewidth=0.02] { \newpath \moveto(0.6215625,-1.2040625) \lineto(0.6815625,-1.2540625) \curveto(0.7115625,-1.2790625)(0.7565625,-1.3240625)(0.7715625,-1.3440624) \curveto(0.7865625,-1.3640625)(0.8115625,-1.3990625)(0.8215625,-1.4140625) \curveto(0.8315625,-1.4290625)(0.8515625,-1.4640625)(0.8615625,-1.4840626) \curveto(0.8715625,-1.5040625)(0.8765625,-1.5290625)(0.8615625,-1.5440625) } 
\pscustom[linewidth=0.02] { \newpath \moveto(5.7415624,-0.7640625) \lineto(5.8115625,-0.7740625) \curveto(5.8465624,-0.7790625)(5.9115624,-0.8240625)(5.9415627,-0.8640625) \curveto(5.9715624,-0.9040625)(6.0115623,-0.9640625)(6.0215626,-0.9840625) \curveto(6.0315623,-1.0040625)(6.0415626,-1.0290625)(6.0415626,-1.0440625) } 
\rput(0.11625,-1.7140625){$Q$} 
\rput(2.156875,1.0459375){$R$} 
\rput(3.6407812,-1.2740625){$S$} 
\rput(5.2523437,-1.3140625){$T$} 
\rput(7.034844,1.7259375){$U$} 
\rput(8.855312,-0.2740625){$V$} 
\end{pspicture}
\end{center}
\end{enumerate}
\item Using Pythagoras' theorem for right-angled triangles, calculate the length $x$:
   \begin{enumerate}[itemsep=8pt, label=\textbf{(\alph*)} ]
\item 
\begin{center}
\scalebox{1} % Change this value to rescale the drawing.
{
\begin{pspicture}(0,-1.07)(2.7653124,1.07)
\pspolygon[linewidth=0.04](0.7253125,-0.93)(2.7453125,-0.93)(0.7253125,1.05)
\rput(1.9907813,0.14){$x$}
\rput(0.3371875,-0.06){\small $3$ cm}
\psline[linewidth=0.04cm](0.7053125,-0.67)(0.9653125,-0.67)
\psline[linewidth=0.04cm](0.9653125,-0.67)(0.9653125,-0.91)
\psline[linewidth=0.04cm](0.6253125,0.17)(0.7853125,0.17)
\psline[linewidth=0.04cm](1.7653126,-0.77)(1.7653126,-1.05)
\end{pspicture} 
}
\end{center}
\item 
\begin{center}
\scalebox{1} % Change this value to rescale the drawing.
{
\begin{pspicture}(0,-0.67)(4.6184373,0.65)
\pspolygon[linewidth=0.04](3.78,-0.35)(0.0,-0.35)(3.78,0.63)
\rput(2.1054688,-0.52){$x$}
\rput(4.210625,0.1){\small $5$ cm}
\psline[linewidth=0.04cm](3.62,-0.19)(3.76,-0.19)
\psline[linewidth=0.04cm](3.64,-0.17)(3.64,-0.35)
\rput(1.568125,0.36){\small $13$ cm}
\end{pspicture} 
}
\end{center}
\item 
\begin{center}
\scalebox{1} % Change this value to rescale the drawing.
{
\begin{pspicture}(0,-1.0017188)(1.5984375,1.0217187)
\pspolygon[linewidth=0.04](0.78,0.61828125)(0.78,-0.9817188)(0.0,0.61828125)
\rput(0.18546875,-0.37171876){$x$}
\rput(1.1925,-0.15171875){\small $7$ cm}
\psline[linewidth=0.04cm](0.64,0.43828124)(0.78,0.43828124)
\psline[linewidth=0.04cm](0.64,0.59828126)(0.64,0.41828126)
\rput(0.3953125,0.84828126){\small $2$ cm}
\end{pspicture} 
}
\end{center}
\item 
\begin{center}
\scalebox{1} % Change this value to rescale the drawing.
{
\begin{pspicture}(0,-1.570625)(5.5721874,1.570625)
\pspolygon[linewidth=0.04](1.413125,-0.84625)(1.413125,1.17375)(0.413125,-0.84625)
\rput(3.3585937,0.40375){$x$}
\rput(0.4234375,0.24875){\small $25$ mm}
\pspolygon[linewidth=0.04](1.413125,-0.84625)(1.413125,1.21375)(5.173125,-0.84625)
\psline[linewidth=0.04cm](1.393125,-0.62625)(1.593125,-0.62625)
\psline[linewidth=0.04cm](1.593125,-0.62625)(1.593125,-0.82625)
\rput(0.8609375,-0.99125){\small  $7$ mm}
\psline[linewidth=0.04cm,tbarsize=0.07055555cm 5.0]{|-|}(0.373125,-1.26625)(5.153125,-1.26625)
\rput(2.5003126,-1.43125){\small  $39$ mm}
\rput(1.425625,1.40875){\small  $A$}
\rput(0.2284375,-0.87125){\small $B$}
\rput(1.4685937,-1.05125){\small  $C$}
\rput(5.4198437,-0.87125){\small  $D$}
\end{pspicture} 
}
\end{center}
\end{enumerate}
\item Consider the diagram below. Is $\triangle ABC ||| \triangle DEF$? Give reasons for your answer. \\
\begin{center}
\scalebox{1} % Change this value to rescale the drawing.
{
\begin{pspicture}(0,-1.391875)(7.8615627,1.391875)
\pspolygon[linewidth=0.04](0.3740625,-1.111875)(2.9740624,1.048125)(7.4340625,-1.111875)
\pspolygon[linewidth=0.04](1.8940625,-0.551875)(3.0357056,0.288125)(4.9940624,-0.551875)
\rput(3.000625,1.218125){$A$}
\rput(0.10125,-1.241875){$B$}
\rput(7.70125,-1.161875){$C$}
\rput(2.9940624,0.478125){$D$}
\rput(1.6760937,-0.621875){$E$}
\rput(5.18375,-0.581875){$F$}
\rput(1.4792187,0.278125){$32$}
\rput(5.04,0.338125){$64$}
\rput(2.1626563,-0.021875){$18$}
\rput(4.119844,0.078125){$36$}
\end{pspicture} 
}\end{center}
\item Explain why $\triangle PQR$ is similar to $\triangle TRS$ and calculate the values of $x$ and $y$.\\
\begin{center}
\scalebox{1} % Change this value to rescale the drawing.
{
\begin{pspicture}(0,-1.4445312)(5.99,1.4445312)
\pspolygon[linewidth=0.04](0.574375,-0.99390626)(0.594375,1.0260937)(3.654375,0.00609375)
\pspolygon[linewidth=0.04](3.654375,0.00609375)(5.594375,0.68609375)(5.594375,-0.5939062)
\rput(0.48515624,1.2760937){$P$}
\rput(3.5929687,-0.30390626){$R$}
\rput(5.634375,-0.82390624){$T$}
\rput(0.29109374,-1.2239063){$Q$}
\rput(5.793594,0.81609374){$S$}
\rput(5.8598437,0.03609375){$x$}
\rput(2.5945313,0.67609376){$6,1$}
\rput(0.25125,-0.04390625){$4,8$}
\rput(4.2021875,0){$y$}
\rput(4.525156,-0.62390625){$3,5$}
\psline[linewidth=0.04cm](2.014375,0.46609375)(2.074375,0.6460937)
\psline[linewidth=0.04cm](2.074375,0.44609374)(2.154375,0.62609375)
\psline[linewidth=0.04cm](1.734375,-0.51390624)(1.794375,-0.67390627)
\psline[linewidth=0.04cm](1.774375,-0.49390626)(1.854375,-0.6539062)
\psline[linewidth=0.04cm](4.734375,-0.23390625)(4.654375,-0.37390625)
\psline[linewidth=0.04cm](4.614375,0.48609376)(4.674375,0.28609374)
\rput(3.020625,-0.00390625){$30$}
\rput{86.05961}(2.9856095,-3.1436067){\psarc[linewidth=0.04](3.1766007,0.027365318){0.44857898}{57.686554}{133.2298}}
\rput{-90.71507}(4.118759,4.03831){\psarc[linewidth=0.04](4.0534906,-0.014681857){0.40599707}{57.686554}{133.2298}}
\end{pspicture} 
}\end{center}
\item Calculate $a$ and $b$:\\
\begin{center}
\scalebox{1} % Change this value to rescale the drawing.
{
\begin{pspicture}(0,-2.2531252)(5.7157817,2.2531252)
\pstriangle[linewidth=0.04,dimen=outer](2.8253126,-2.0403123)(4.66,3.9)
\psline[linewidth=0.04cm](2.0353124,0.49968755)(3.6,0.5131251)
\psline[linewidth=0.04](2.7753124,-2.1803124)(3.1153123,-2.0203123)(2.7953124,-1.8403124)(3.1153123,-2.0403123)
\psline[linewidth=0.04](2.6553125,0.35968754)(2.9953127,0.51968753)(2.6753125,0.69968754)(2.9953127,0.49968755)
\usefont{T1}{ptm}{m}{n}
\rput(2.8007812,2.0496876){$T$}
\usefont{T1}{ptm}{m}{n}
\rput(1.7720313,0.50968754){$R$}
\usefont{T1}{ptm}{m}{n}
\rput(3.9,0.48968756){$S$}
\usefont{T1}{ptm}{m}{n}
\rput(0.27453125,-2.0303123){$P$}
\usefont{T1}{ptm}{m}{n}
\rput(5.31125,-2.0503125){$Q$}
\usefont{T1}{ptm}{m}{n}
\rput(2.4729688,0.9296876){$a$}
\usefont{T1}{ptm}{m}{n}
\rput(1.4387499,-0.93031245){$15$}
\usefont{T1}{ptm}{m}{n}
\rput(4.1403127,-0.9503124){$9$}
\usefont{T1}{ptm}{m}{n}
\rput(3.4751563,1.25){$\dfrac{b}{4}$}
\psline[linewidth=0.04cm,arrowsize=0.05291667cm 2.0,arrowlength=1.4,arrowinset=0.4]{<->}(3.5953126,1.6396875)(5.395313,-1.2603126)
\usefont{T1}{ptm}{m}{n}
\rput(4.5978127,0.36968753){$b$}
\end{pspicture} 
}
\end{center}
\item
$ABCD$ is a parallelogram with diagonal $AC$.\\
Given that $AF=HC$, show that:
   \begin{enumerate}[noitemsep, label=\textbf{(\alph*)} ]
 \item $\triangle AFD \equiv \triangle CHB$
\item $DF\parallel HB$
\item $DFBH$ is a parallelogram
\end{enumerate}
\begin{figure}[H]
\begin{center}
\scalebox{1} % Change this value to rescale the drawing.
{
\begin{pspicture}(0,-1.75375)(7.1690626,1.75375)
\psline[linewidth=0.04](0.3371875,1.45125)(1.2971874,-1.32875)(6.8971877,-1.34875)(6.0171876,1.41125)(0.3371875,1.41125)(0.3371875,1.41125)(0.3771875,1.41125)(0.3771875,1.41125)
\psline[linewidth=0.04cm](0.3371875,1.39125)(6.9171877,-1.34875)
\psline[linewidth=0.04](1.2771875,-1.30875)(2.4571874,0.51125)(6.0171876,1.39125)
\psline[linewidth=0.04](1.2971874,-1.30875)(4.6971874,-0.42875)(5.9971876,1.39125)
\psline[linewidth=0.04cm,arrowsize=0.05291667cm 3.0,arrowlength=1.4,arrowinset=0.4]{->>}(3.0971875,1.41125)(3.5971875,1.41125)
\psline[linewidth=0.04cm,arrowsize=0.05291667cm 3.0,arrowlength=1.4,arrowinset=0.4]{->>}(3.9771874,-1.34875)(4.4771876,-1.34875)
\psline[linewidth=0.04cm,arrowsize=0.05291667cm 3.0,arrowlength=1.4,arrowinset=0.4]{<-}(0.8171875,0.11125)(0.9771875,-0.40875)
\psline[linewidth=0.04cm,arrowsize=0.05291667cm 3.0,arrowlength=1.4,arrowinset=0.4]{<-}(6.4171877,0.19125)(6.5771875,-0.32875)
\psline[linewidth=0.04cm](5.7771873,-0.70875)(5.6371875,-0.98875)
\psline[linewidth=0.04cm](1.5571876,1.03125)(1.4171875,0.75125)
\usefont{T1}{ptm}{m}{n}
\rput(0.1175,1.56125){$A$}
\usefont{T1}{ptm}{m}{n}
\rput(6.1703124,1.56125){$B$}
\usefont{T1}{ptm}{m}{n}
\rput(7.002969,-1.59875){$C$}
\usefont{T1}{ptm}{m}{n}
\rput(1.1475,-1.57875){$D$}
\usefont{T1}{ptm}{m}{n}
\rput(2.4726562,0.70125){$F$}
\usefont{T1}{ptm}{m}{n}
\rput(4.6371875,-0.71875){$H$}
\usefont{T1}{ptm}{m}{n}
\rput(0.627,1.06125){\scriptsize $1$}
\usefont{T1}{ptm}{m}{n}
\rput(6.1879687,-1.21875){\scriptsize $2$}
\usefont{T1}{ptm}{m}{n}
\rput(2.047,0.40125){\scriptsize $1$}
\usefont{T1}{ptm}{m}{n}
\rput(6.627,-1.07875){\scriptsize $1$}
\usefont{T1}{ptm}{m}{n}
\rput(4.987,-0.33875){\scriptsize $1$}
\usefont{T1}{ptm}{m}{n}
\rput(4.627969,-0.19875){\scriptsize $2$}
\usefont{T1}{ptm}{m}{n}
\rput(2.5079687,0.30125){\scriptsize $2$}
\usefont{T1}{ptm}{m}{n}
\rput(1.1079688,1.26125){\scriptsize $2$}
\end{pspicture} 
}\end{center}
\end{figure}
\item
$\triangle PQR$ and $\triangle PSR$ are equilateral triangles. Prove that $PQRS$ is a rhombus:\\
\begin{center}
\scalebox{1} % Change this value to rescale the drawing.
{
\begin{pspicture}(0,-1.44125)(4.759375,1.44125)
\psline[linewidth=0.04](0.27625,-1.06125)(1.63625,1.11875)(4.41625,1.09875)(3.17625,-1.06125)(0.29625,-1.06125)
\psline[linewidth=0.04cm](1.65625,1.11875)(3.15625,-1.04125)
\rput(1.4809375,1.24875){$P$}
\rput(4.5776563,1.20875){$Q$}
\rput(3.2584374,-1.29125){$R$}
\rput(0.06859375,-1.17125){$S$}
\rput(1.6240625,0.86875){\scriptsize $1$}
\rput(2.8240626,-0.89125){\scriptsize $1$}
\rput(1.9470313,0.94875){\scriptsize $2$}
\rput(3.1470313,-0.73125){\scriptsize $2$}
\end{pspicture} 
}
\end{center}
\item Given parallelogram $ABCD$ with $AE$ and $FC$, $AE$ bisecting $\hat{A}$ and $FC$ bisecting $\hat{C}$:
   \begin{enumerate}[noitemsep, label=\textbf{(\alph*)} ]
 \item Write all interior angles in terms of $y$.
\item Prove that $AFCE$ is a parallelogram.
\begin{center}
\scalebox{1} % Change this value to rescale the drawing.
{
\begin{pspicture}(0,-1.28125)(6.80625,1.28125)
\psline[linewidth=0.04](0.2646875,-0.86125)(1.5046875,0.85875)(6.4846873,0.85875)(5.4446874,-0.86125)(0.2446875,-0.86125)
\psline[linewidth=0.04cm](1.5246875,0.83875)(3.1646874,-0.86125)
\psline[linewidth=0.04cm](5.4246874,-0.84125)(3.7646875,0.83875)
\rput(1.525,1.04875){$A$}
\rput(6.6378126,1.02875){$B$}
\rput(3.7601562,1.08875){$F$}
\rput(5.490469,-1.09125){$C$}
\rput(3.0892189,-1.13125){$E$}
\rput(0.075,-1.07125){$D$}
\rput(1.5404687,0.48875){$y$}
\end{pspicture} 
}
\end{center}
\end{enumerate}
\item Given that $WZ=ZY=YX$, $\hat{W}=\hat{X}$ and $WZ \parallel ZY$, prove that:
   \begin{enumerate}[noitemsep, label=\textbf{(\alph*)} ]
\item $XZ$ bisects $\hat{X}$
\item $WY=XZ$
\begin{center}
\scalebox{1} % Change this value to rescale the drawing.
{
\begin{pspicture}(0,-1.46125)(5.8434377,1.46125)
\psline[linewidth=0.04](1.348125,-1.06125)(4.208125,-1.06125)(5.528125,1.09875)(0.368125,1.11875)(1.328125,-1.06125)(1.308125,-1.06125)(1.348125,-1.06125)(1.308125,-1.06125)
\psline[linewidth=0.04cm](0.348125,1.13875)(4.208125,-1.06125)
\psline[linewidth=0.04cm](1.308125,-1.04125)(5.488125,1.05875)
\psline[linewidth=0.04](2.948125,1.29875)(3.188125,1.13875)(2.948125,0.97875)
\psline[linewidth=0.04](3.048125,-0.90125)(3.288125,-1.06125)(3.048125,-1.22125)
\psline[linewidth=0.04cm](4.668125,-0.06125)(4.968125,-0.06125)
\psline[linewidth=0.04cm](0.728125,-0.08125)(1.028125,-0.08125)
\psline[linewidth=0.04cm](2.628125,-0.94125)(2.628125,-1.22125)
\usefont{T1}{ptm}{m}{n}
\rput(0.15421875,1.26875){$W$}
\usefont{T1}{ptm}{m}{n}
\rput(5.6804686,1.26875){$X$}
\usefont{T1}{ptm}{m}{n}
\rput(1.0951562,-1.31125){$Z$}
\usefont{T1}{ptm}{m}{n}
\rput(4.37875,-1.25125){$Y$}
\end{pspicture} 
}
\end{center}
\end{enumerate}
\item
$LMNO$ is a quadrilateral with $LM=LO$ and diagonals that intersect at $S$ such that $MS=SO$. Prove that:
   \begin{enumerate}[noitemsep, label=\textbf{(\alph*)} ]
 \item $M\hat{L}S = S\hat{L}O$
\item $\triangle LON \equiv \triangle LMN$
\item $MO \perp LN$
\begin{center}
\scalebox{1} % Change this value to rescale the drawing.
{
\begin{pspicture}(0,-2.04125)(2.3478124,2.04125)
\psline[linewidth=0.04](0.3046875,0.47875)(1.1646875,1.61875)(1.9846874,0.45875)(1.1046875,-1.62125)(0.3046875,0.51875)
\psline[linewidth=0.04cm](0.2846875,0.45875)(1.9846874,0.47875)
\psline[linewidth=0.04cm](1.1446875,1.59875)(1.1046875,-1.56125)
\rput(1.1464063,1.84875){$L$}
\rput(2.1660938,0.46875){$O$}
\rput(1.085,-1.89125){$N$}
\rput(0.0846875,0.44875){$M$}
\rput(0.9170312,0.24875){$S$}
\end{pspicture} 
}
\end{center}
\end{enumerate}
\item
Using the figure below, show that the sum of the three angles in a triangle is 180$^{\circ }$. Line $DE$ is parallel to $BC$.\\
\begin{center}
\begin{pspicture}(0,-0.5)(6,3.5)
\pspolygon(1,0)(5,0)(4,3)
\uput[l](1,0){$B$}
\uput[r](5,0){$C$}
\uput[u](4,3){$A$}
\psline[linestyle=dotted,arrows=<->](1,3)(6,3)
\uput[l](1,3){$D$}
\uput[r](6,3){$E$}
\uput{0.5}[20](1,0){$b$}
\uput{0.6}[145](5.1,-0.2){$c$}
\uput{0.5}[200](4,3){$d$}
\uput{0.6}[325](4,3){$e$}
\uput{0.7}[255](4.2,3.2){$a$}
\end{pspicture}
\end{center}

\item 
$D$ is a point on $BC$, in $\triangle ABC$. $N$ is the mid-point of $AD$. $O$ is the mid-point of $AB$ and $M$ is the mid-point of $BD$. $NR \ parallel AC$. 
\begin{center}
 
\scalebox{1}{

\begin{pspicture}(0,-2.46125)(5.9165626,2.46125)
\psline[linewidth=0.04](3.6046875,2.05875)(5.6046877,-0.62125)(0.2246875,-2.04125)(3.6046875,2.05875)
\psline[linewidth=0.04cm](3.6046875,2.05875)(3.8446875,-1.08125)
\psline[linewidth=0.04cm](3.7046876,0.35875)(4.5446873,-0.88125)
\psline[linewidth=0.04cm](3.7046876,0.39875)(1.7646875,-0.20125)
\psline[linewidth=0.04cm](3.7246876,0.37875)(2.1646874,-1.52125)
\psdots[dotsize=0.12](3.7246876,0.37875)
\psdots[dotsize=0.12](2.1646874,-1.54125)
\psdots[dotsize=0.12](1.7246875,-0.20125)
\psline[linewidth=0.04cm,arrowsize=0.133cm 2.4,arrowlength=1.4,arrowinset=0.4]{<-}(4.731271,0.6030493)(4.858104,0.39445072)
\psline[linewidth=0.04cm,arrowsize=0.133cm 2.4,arrowlength=1.4,arrowinset=0.4]{<-}(4.071271,-0.13695072)(4.198104,-0.3455493)
\psline[linewidth=0.04cm](3.5446875,1.07875)(3.7846875,1.11875)
\psline[linewidth=0.04cm](3.6646874,-0.54125)(3.9046874,-0.50125)
\psline[linewidth=0.04cm](2.6246874,1.07875)(2.8246875,0.89875)
\psline[linewidth=0.04cm](2.5846875,1.01875)(2.7846875,0.83875)
\psline[linewidth=0.04cm](1.0846875,-0.76125)(1.2846875,-0.94125)
\psline[linewidth=0.04cm](1.0446875,-0.82125)(1.2446876,-1.00125)
\psline[linewidth=0.04](1.4246875,-1.86125)(1.4646875,-1.58125)(1.5446875,-1.80125)(1.5446875,-1.84125)
\psline[linewidth=0.04](3.1246874,-1.42125)(3.1646874,-1.14125)(3.2446876,-1.36125)(3.2446876,-1.40125)
% \usefont{T1}{ptm}{m}{n}
\rput(3.625,2.26875){$A$}
% \usefont{T1}{ptm}{m}{n}
\rput(5.7504687,-0.67125){$C$}
% \usefont{T1}{ptm}{m}{n}
\rput(4.606875,-1.13125){$R$}
% \usefont{T1}{ptm}{m}{n}
\rput(3.875,-1.33125){$D$}
% \usefont{T1}{ptm}{m}{n}
\rput(2.1246874,-1.83125){$M$}
% \usefont{T1}{ptm}{m}{n}
\rput(0.0578125,-2.31125){$B$}
% \usefont{T1}{ptm}{m}{n}
\rput(3.9240625,0.44875){$N$}
% \usefont{T1}{ptm}{m}{n}
\rput(1.4460938,-0.19125){$O$}
\end{pspicture} }
\end{center}
\begin{enumerate}[noitemsep, label=\textbf{(\alph*)} ]
\item Prove that $OBMN$ is a parallelogram.
\item Prove that $BC=2MR$.
\end{enumerate}
\item $PQR$ is an isoceles with $PR=QR$. $S$ is the mid-point of $PQ$, $T$ is the mid-point of $PR$ and $U$ is the mid-point of $RQ$.
\begin{center}
 \scalebox{1}{
\scalebox{1} % Change this value to rescale the drawing.
{
\begin{pspicture}(0,-2.2235937)(3.5078125,2.2235937)
\psline[linewidth=0.04](0.1846875,1.8364062)(3.1646874,1.8364062)(1.6446875,-1.8235937)(0.1846875,1.8364062)
\psline[linewidth=0.04](0.8646875,0.17640625)(1.6646875,1.8164062)(2.4646876,0.17640625)(0.8646875,0.17640625)
% \usefont{T1}{ptm}{m}{n}
\rput(0.049375,1.9464062){$P$}
% \usefont{T1}{ptm}{m}{n}
\rput(1.6570313,2.0264063){$S$}
% \usefont{T1}{ptm}{m}{n}
\rput(3.3260937,1.9464062){$Q$}
% \usefont{T1}{ptm}{m}{n}
\rput(0.635625,0.16640624){$T$}
% \usefont{T1}{ptm}{m}{n}
\rput(2.74,0.14640625){$U$}
% \usefont{T1}{ptm}{m}{n}
\rput(1.646875,-2.0735939){$R$}
\end{pspicture} 
}
}
\end{center}
\begin{enumerate}[noitemsep, label=\textbf{(\alph*)} ]
\item Prove $\triangle STU$ is also isoceles.
\item What type of quadrilateral is $STRU$? Motivate your answer.
\item If $R\hat TU = 68^{\circ}$ calculate, with reasons, the size of $T \hat{S} U$.
\end{enumerate}
\item In $\triangle MNP$, $M=90^{\circ}$, $S$ is the mid-point of $MN$ and $T$ is the mid-point of $NR$. 
\begin{center}
 \scalebox{1}{
\scalebox{1} % Change this value to rescale the drawing.
{
\begin{pspicture}(0,-1.82375)(7.229375,1.82375)
\psline[linewidth=0.04](0.3046875,-1.39875)(2.3446875,1.52125)(6.9046874,-1.39875)(0.3046875,-1.39875)
\psline[linewidth=0.04cm](0.3046875,-1.37875)(4.2446876,0.32125)
\psline[linewidth=0.04cm](1.4246875,0.20125)(3.5846875,-1.39875)
\rput{56.97092}(2.159071,-1.4164218){\psframe[linewidth=0.04,dimen=outer](2.5646875,1.46125)(2.2046876,1.10125)}
% \usefont{T1}{ptm}{m}{n}
\rput(2.3046875,1.7){$M$}
% \usefont{T1}{ptm}{m}{n}
\rput(4.386875,0.49125){$R$}
% \usefont{T1}{ptm}{m}{n}
\rput(7.069375,-1.44875){$P$}
% \usefont{T1}{ptm}{m}{n}
\rput(3.6051562,-1.66875){$U$}
% \usefont{T1}{ptm}{m}{n}
\rput(1.3370312,0.39125){$S$}
% \usefont{T1}{ptm}{m}{n}
\rput(2.475625,-0.24875){$T$}
% \usefont{T1}{ptm}{m}{n}
\rput(0.0640625,-1.56875){$N$}
\end{pspicture} 
}

}
\end{center}
\begin{enumerate}[noitemsep, label=\textbf{(\alph*)} ]
\item Prove $M$ is the mid-point of $NP$.
\item If $ST=4$ cm and the area of $\triangle SNT$ is $6$ cm$^2$, calculate the area of $\triangle MNR$.
\item Prove that the area of $\triangle MNR$ will always be four times the area of $\triangle SNT$, let $ST=x$ units and $SN=y$ units.
\end{enumerate}
\end{enumerate}

\end{eocexercises}


 \begin{eocsolutions}{}{
\begin{enumerate}[itemsep=5pt, label=\textbf{\arabic*}. ] 


\item %solution 1
	  \begin{enumerate}[noitemsep, label=\textbf{(\alph*)} ]
		\item Straight angle
\item Obtuse angle
\item Acute angle
\item Right angle
\item Reflex angle
\item Obtuse angle
\item Straight angle
\item Reflex angle
	  \end{enumerate}
\item %solution 2
      \begin{enumerate}[noitemsep, label=\textbf{(\alph*)} ]

\item False - a trapezium only has one pair of opposite parallel sides.% A trapezium is a quadrilateral with two pairs of opposite sides that are parallel.
\item True.% Both diagonals of a parallelogram bisect each other.
\item True.% A rectangle is a parallelogram that has one corner angles equal to $90^{\circ}$.
\item False - two adjacent sides of a rhombus are equal in length.% Two adjacent sides of a rhombus have different lengths.
\item True.% The diagonals of a kite intersect at right angles.
\item True.%All squares are parallelograms.
\item True.%A rhombus is a kite with a pair of equal, opposite sides.
% \item %The diagonals of a parallelogram are axes of symmetry.
\item False - the diagonals of a rhombus are not equal in length.%The diagonals of a rhombus are equal in length.
\item False - only one diagonal of a kite bisects two pairs of interior angles.%Both diagonals of a kite bisect the interior angles.
	  \end{enumerate}
\item %solution 3
	  \begin{enumerate}[noitemsep, label=\textbf{(\alph*)} ]
	\item $x=180^{\circ}-90^{\circ}-65^{\circ}=25^{\circ}$
\item $x=180^{\circ}-20^{\circ}-15^{\circ}=145^{\circ}$
\item $25^2 = 15^2 + (2x)^2\\
4x^2 = 400\\
x^2 = 100\\
\therefore x=10$ units\\
$\frac{x}{2x} = \frac{y}{25}\\
\therefore y=12,5$ units
\item $x=60^{\circ}$
\item $x+x+3x = 180^{\circ}\\
\therefore 5x = 180^{\circ}\\ \therefore x=36^{\circ}$
\item $
\frac{x}{9} = \frac{8}{12}\\
\therefore x=6$ units\\
$\frac{y}{12} = \frac{7.5}{9} \\ 
\therefore y=10$ units
	  \end{enumerate}

\item %solution 4
	  \begin{enumerate}[noitemsep, label=\textbf{(\alph*)} ]
		\item $AB \parallel CD$ (alt. int. $\angle$'s equal)
		\item $NP$ not $\parallel MO$ (corresp. $\angle$'s not equal)\\
$MN \parallel OP$ (corresp. $\angle$'s equal)
		\item $GH \parallel HL$ (corresp. $\angle$'s equal)
	  \end{enumerate}
\item 
\begin{enumerate}[noitemsep, label=\textbf{(\alph*)} ]
		\item $a=180^{\circ} -73^{\circ}=107^{\circ}$ (co-interior $\angle$'s)\\
$ b=180^{\circ} -107^{\circ}=73^{\circ}$ (co-interior $\angle$'s)\\

 $c=180^{\circ} -73^{\circ}=107^{\circ}$ ($\angle$'s on a str. line)\\
$d=73^{\circ}$ (corresp. $\angle$'s)
\item $a=80^{\circ}$ ($\angle$'s on a str. line)\\
 $b=80^{\circ}$ (alt. int. $\angle$'s)\\
 $c=80^{-73\circ}$ (corresp. $\angle$'s)\\
 $d=80^{\circ}$ (opp. $\angle$'s)
\item $a=50^{\circ}$ (alt. int. $\angle$'s)\\
 $ b=45^{\circ}$ (alt. int. $\angle$'s)\\
 $c=95^{\circ}$ (sum of int. $\angle$'s)\\
 $ d=85^{\circ}$ (sum of $\angle$'s in a $\triangle)$
	  \end{enumerate}
\item 
  \begin{enumerate}[noitemsep, label=\textbf{(\alph*)} ]
% 	  % \setcounter{enumi}{30}
		\item Congruent by SSS%
		\item Congruent by RHS%
		\item  Congruent by AAS%
		\item Congruent by SAS
	  \end{enumerate}

\item 	  \begin{enumerate}[noitemsep, label=\textbf{(\alph*)} ]
	\item $x=\sqrt{3^2+3^2} = \sqrt{9+9} =\sqrt{18} =4,24$ cm%

\item $x=\sqrt{13^2+5^2} = \sqrt{139-25} =\sqrt{144} =12$ cm%

\item $x=\sqrt{2^2+7^2} = \sqrt{4+49} =\sqrt{53} =7,28$ cm%

\item $AC=\sqrt{25^2-7^2} = 576\\
\therefore AC=24\\
x^2 = 32^2+24^2\\
\therefore x=40$ mm%
	  \end{enumerate}
\item 
$\frac{ED}{BA} = \frac{18}{32} = \frac{9}{16}\\
\frac{EF}{BC} = \frac{32}{64} = \frac{9}{16}\\$
These pairs of sides are in proportion,\\ $\therefore \triangle ABC ||| \triangle DEF$
\item 
$y=30^{\circ}$ (vert. opp. $\angle$'s)\\
$\frac{x}{4,8} = \frac{3,5}{6,1} \\
\therefore x = 2,75$\\
$\hat{P} = \hat{Q}$ (equal $\angle$\s opp. sides isoceles $\triangle$)\\
and $\hat{S} = \hat{T}$ (equal $\angle$\s opp. sides isoceles $\triangle$)\\
$\therefore \triangle PQR ||| \triangle TRS$ (AAA)
\item 
$\frac{a}{a+15} = \dfrac{\frac{b}{4}}{b}\\
a = (a+15) \left(\frac{1}{4}\right) \\
4a = a+15\\
3a=15\\
\therefore a=5\\
\\
b=\frac{b}{4} +9\\
4b=b+36\\
3b=36\\
\therefore b=12$
\item %solution 11
 \begin{enumerate}[noitemsep, label=\textbf{(\alph*)} ]
      \item 

$\hat{A_1} = \hat{C_1}$ (alt. $\angle$'s, $AD \parallel BC$)\\
$AD = BC$ (opp. sides of parallelogram equal)\\
$AF=HC$ (given)\\
$\therefore \triangle AFD \equiv \triangle CHB$ (SAS)\\
%$\triangle AFD \equiv \triangle CHB$
      \item $\hat{F_1} = \hat{H_1}$ ($\triangle AFD \equiv \triangle CHB$)\\
$\therefore \hat{F_1} +\hat{F_2} = 180^{\circ}$ ($\angle$'s on str. line)\\
and $ \hat{H_1} +\hat{H_2} = 180^{\circ}$ ($\angle$'s on str. line)\\
$\therefore \hat{F_2} +\hat{H_2}$\\
but these are alternate $\angle$'s, \\$\therefore DF\parallel HB$
      \item 
$FD= HB$ ($\triangle AFD \equiv \triangle CHB$)\\
and $DF \parallel HB$ (proved above)\\
$\therefore DFBH$ is a parallelogram (one pair opp. sides equal & parallel)
%$DFBH$ is a parallelogram
      \end{enumerate}


\item %solution 12

Given $\triangle PSR \equiv \triangle PQR$, with common side $PR$\\
$PS=SR=PR=QR$\\
$\therefore$ all sides equal in length\\
$\therefore PQRS$ is a rhombus
\item %solution 13
First number the angles:\\
\scalebox{1} % Change this value to rescale the drawing.
{
\begin{pspicture}(0,-1.313125)(7.251875,1.313125)
\psline[linewidth=0.04](0.5196875,-0.8403125)(1.7596875,0.8796875)(6.7396874,0.8796875)(5.6996875,-0.8403125)(0.4996875,-0.8403125)
\psline[linewidth=0.04cm](1.7796875,0.8596875)(3.4196875,-0.8403125)
\psline[linewidth=0.04cm](5.6796875,-0.8203125)(4.0196877,0.8596875)
% \usefont{T1}{ptm}{m}{n}
\rput(1.7545313,1.0696875){$A$}
% \usefont{T1}{ptm}{m}{n}
\rput(6.867344,1.0496875){$B$}
% \usefont{T1}{ptm}{m}{n}
\rput(3.9896874,1.1096874){$F$}
% \usefont{T1}{ptm}{m}{n}
\rput(5.7200003,-1.0703125){$C$}
% \usefont{T1}{ptm}{m}{n}
\rput(3.3187501,-1.1103125){$E$}
% \usefont{T1}{ptm}{m}{n}
\rput(0.30453125,-1.0503125){$D$}
% \usefont{T1}{ptm}{m}{n}
\rput(1.75,0.3696875){$y$}
% \usefont{T1}{ptm}{m}{n}
\rput(1.7478125,0.603125){\tiny $1$}
% \usefont{T1}{ptm}{m}{n}
\rput(2.9478126,-0.636875){\tiny $1$}
% \usefont{T1}{ptm}{m}{n}
\rput(3.9478126,0.723125){\tiny $1$}
% \usefont{T1}{ptm}{m}{n}
\rput(5.3078127,-0.656875){\tiny $1$}
% \usefont{T1}{ptm}{m}{n}
\rput(5.630781,-0.496875){\tiny $2$}
% \usefont{T1}{ptm}{m}{n}
\rput(4.3707814,0.723125){\tiny $2$}
% \usefont{T1}{ptm}{m}{n}
\rput(3.4507813,-0.596875){\tiny $2$}
% \usefont{T1}{ptm}{m}{n}
\rput(2.1507812,0.643125){\tiny $2$}
\end{pspicture} 
}\\
     \begin{enumerate}[noitemsep, label=\textbf{(\alph*)} ]
\item $\hat{A_2} = y$ (given $AE$ bisects $\hat{A}$) \\
$\hat{E_1} = y$  (alt. $\angle$'s, $AB \parallel DC $ in parm $ABCD$)\\
$\therefore hat{E_2} = 180^{\circ}- y$ ($\angle$'s on str. line)\\
\\
$\hat{C_1} = \hat{C_2}$ (given $FC$ bisects $\hat{C}$)\\
and $\hat{A} = \hat{C}$ (opp. $\angle$'s parm $ABCD$ equal)\\
$\therefore \hat{C_1} = \hat{C_2} =y$\\
\\
$\therefore \hat{F_2} = \hat{C_1} =y$ (alt. $\angle$'s, $AB \parallel DC$)\\
$\therefore \hat{F_1} = 180^{\circ}- y$\\
\\

In $\triangle ADE$\\
$\hat{D} + \hat{A_1} + \hat{E_1} = 180^{\circ}$ (sum of $\angle$'s in $\triangle$)\\
$\therefore \hat{D} + y + y = 180^{\circ}\\
\therefore \hat{D} = 180^{\circ} - 2y$\\
$\hat{D} = 90^{\circ} - y\\
\therefore \hat{B} = 90^{\circ} - y$ (opp. $\angle$'s parm $ABCD$ equal)\\

\item 
$AF\parallel EC$ (opp. sides parm $ABCD$ equal)\\
and $\hat{C_1} + \hat{E_2} = y + (180^{\circ} - y)\\
\therefore $ the sum of the co-interior angles is $180^{\circ}$\\
$\therefore AE \parallel FC\\
\therefore AFCE$ is a parallelogram (both opp. sides parallel)
\end{enumerate}

\item %solution 14
First label the angles:\\
\scalebox{1} % Change this value to rescale the drawing.
{
\begin{pspicture}(0,-1.493125)(6.2753124,1.493125)
\psline[linewidth=0.04](1.5739063,-1.0403125)(4.433906,-1.0403125)(5.7539062,1.1196876)(0.5939062,1.1396875)(1.5539062,-1.0403125)(1.5339062,-1.0403125)(1.5739063,-1.0403125)(1.5339062,-1.0403125)
\psline[linewidth=0.04cm](0.57390624,1.1596875)(4.433906,-1.0403125)
\psline[linewidth=0.04cm](1.5339062,-1.0203125)(5.7139063,1.0796875)
\psline[linewidth=0.04](3.1739063,1.3196875)(3.4139063,1.1596875)(3.1739063,0.9996875)
\psline[linewidth=0.04](3.2739062,-0.8803125)(3.5139062,-1.0403125)(3.2739062,-1.2003125)
\psline[linewidth=0.04cm](4.893906,-0.0403125)(5.1939063,-0.0403125)
\psline[linewidth=0.04cm](0.95390624,-0.0603125)(1.2539062,-0.0603125)
\psline[linewidth=0.04cm](2.8539062,-0.9203125)(2.8539062,-1.2003125)
\usefont{T1}{ptm}{m}{n}
\rput(0.35453126,1.2896875){$W$}
\usefont{T1}{ptm}{m}{n}
\rput(5.880781,1.2896875){$X$}
\usefont{T1}{ptm}{m}{n}
\rput(1.2954687,-1.2903125){$Z$}
\usefont{T1}{ptm}{m}{n}
\rput(4.5790625,-1.2303125){$Y$}
\usefont{T1}{ptm}{m}{n}
\rput(0.9678125,0.703125){\tiny $1$}
\usefont{T1}{ptm}{m}{n}
\rput(1.6278125,-0.756875){\tiny $1$}
\usefont{T1}{ptm}{m}{n}
\rput(3.8278124,-0.916875){\tiny $1$}
\usefont{T1}{ptm}{m}{n}
\rput(5.0678124,0.923125){\tiny $1$}
\usefont{T1}{ptm}{m}{n}
\rput(1.2307812,0.943125){\tiny $2$}
\usefont{T1}{ptm}{m}{n}
\rput(5.210781,0.623125){\tiny $2$}
\usefont{T1}{ptm}{m}{n}
\rput(2.1307812,-0.896875){\tiny $2$}
\usefont{T1}{ptm}{m}{n}
\rput(4.3707814,-0.796875){\tiny $2$}
\psdots[dotsize=0.12](0.84,0.873125)
\psdots[dotsize=0.12](1.0,1.033125)
\psdots[dotsize=0.12](4.04,-0.946875)
\usefont{T1}{ptm}{m}{n}
\rput(5.2984376,1.023125){\scriptsize \textbf{x}}
\usefont{T1}{ptm}{m}{n}
\rput(5.4184375,0.803125){\scriptsize \textbf{x}}
\usefont{T1}{ptm}{m}{n}
\rput(1.9784375,-0.916875){\scriptsize \textbf{x}}
\end{pspicture} 
}\\
 \begin{enumerate}[noitemsep, label=\textbf{(\alph*)} ]
		\item 
IN $\triangle XYZ$\\
$\hat{X_2} = \hat{Z_2}$ ($\angle$'s opp. equal sides of isoceles $\triangle$)\\
and $\hat{X_1} = \hat{Z_2}$ (alt. $\angle$'s, $WX \parallel ZY$)\\
$\therefore \hat{X_1} = \hat{X_2}\\
\therefore XZ$ bisects $\hat{X}$
		\item 
Similarly, $WY$ bisects $\hat{W}\\
\therefore \hat{W_1} = \hat{W_2}\\$
and $\hat{W} = \hat{X}$ (given)\\
$\therefore \hat{W_1} = \hat{W_2} = \hat{X_1} = \hat{X_2}\\
$ and $\hat{W_1} = \hat{Y_1}$ ($\angle$'s opp. equal sides)\\
In $\triangle WZY$ and $\triangle XYZ\\
WZ=XY$ (given)\\
$ZY$ is a common side\\
$\hat{Z} = \hat{Y} $(third $\angle$ in $\triangle$)\\
$\therefore \triangle WZY \equiv  \triangle XYZ$ (SAS)
	

	  \end{enumerate}
\item %solution 15
	  \begin{enumerate}[noitemsep, label=\textbf{(\alph*)} ]

		\item 
In $\triangle LMS$ and $\triangle LOS$\\
$LM = LO$ (given)\\
$MS=SO$ (given)\\
$LS$ is a common side\\
$\therefore \triangle LMS \equiv \triangle LOS$ (SSS)\\
$\therefore \hat{L_1} = \hat{L_2}$
		\item 
In $\triangle LON$ and $\triangle LMN$\\
$LO = LM$ (given)\\
$\hat{L_1} = \hat{L_2}$ (proved above)\\
$LN$ is a common side\\
$\therefore \triangle LON \equiv \triangle LMN$ (SSS)\\
$\therefore \hat{L_1} = \hat{L_2}$
%$\triangle LOA \equiv \triangle LMA$
		\item %$MO \perp LA$
In $\hat{M_1} = \hat{O_1}$ ($\triangle LON \equiv \triangle LMN$)\\
and $\hat{L_1} = \hat{L_2}$ (proved above)\\
$\therefore$ in $\triangle LMO\\
\hat{L_1} + \hat{L_2} + \hat{M_1} + \hat{O_1} = 180^{\circ}$ (sum of $\angle$'s in $\triangle$)\\
$\therefore 2\hat{L_1} + 2\hat{O_1} = 180^{\circ}\\
2(\hat{L_1} + \hat{O_1}) = 180^{\circ}\\
\hat{L_1} + \hat{O_1} = 90^{\circ}\\$
but $\hat{S_1} = \hat{O_1} + \hat{L_2}$ (ext. $\angle$ of $\triangle=$ sum of int. opp. $\angle$'s)\\
$\therefore \hat{S_1} = 90^{\circ}\\
\therefore MO \perp LN$
	  \end{enumerate}
\item % solution 16
$DE \parallel BC$\\
$e=c$ (alt. int. $\angle$'s)\\
$d=b$ (alt. int. $\angle$'s)\\
We know that $d+a+e=180^\circ$\\
And we have shown that $e=c$ and $d=e$ therefore we can replace $d$ and $e$ in the diagram to get:\\
$a+b+c = 180^\circ$\\
Therefore the angles in a triangle do add up to $180^\circ$. 

\item %solution 17
 \begin{enumerate}[noitemsep, label=\textbf{(\alph*)} ]
\item
$AO=OB$ (given)\\
$AN=ND$ (given)\\
$\therefore ON\parallel BD$ (mid-pt theorem)\\
$BM=MD$ (given)\\
$AN=ND$ (given)\\
$\therefore MN \parallel AB$\\
$\therefore OBMN$ is a parallelogram (opp. sides parallel)
\item 
$AN=NC$ (given)\\
$NR \parallel AC$ (given)\\
$\therefore DR=RC$ (mid-pt theorem)\\
$\therefore DR=\frac{1}{2} DC\\
MD=\frac{1}{2}BD$ (given)\\
$\therefore MD + DR = \frac{1}{2} (BD+DC)\\
MR = \frac{1}{2}BC\\
\therefore BC = 2MR$
\end{enumerate}

\item %solution 18
 \begin{enumerate}[noitemsep, label=\textbf{(\alph*)} ]
\item
$PT=\frac{1}{2} PR$ (given)\\
$S$ mid-point of $PQ$\\
$U$ mid-point of $RQ$\\
$SU=\frac{1}{2}PR$\\
$\therefore SU=PT\\
S$ mid-point of $PQ$\\
$T$ mid-point of $PR$\\
$\therefore ST = \frac{1}{2}QR=QU\\$
But $PR=QR$ (given)\\
$\therefore SU=ST\\
\therefore \triangle STU$ is isoceles.
\item $STRU$ is a rhombus. It is a parallelogram ($SU \parallel TR$ and $ST \parallel UR$) with four equal sides ($US=ST=TR=RU$).
\item $R\hat{T}U = 68^{\circ}\\
\therefore T\hat{U}S = 68^{\circ}$ (alt $\angle$'s, $TR \parallel SU$)\\
$\therefore S\hat{T}U = 68^{\circ}$ ($SU=ST$)\\
$\therefore T\hat{S}U = 180^{\circ} - 136^{\circ}\\
=44^{\circ}$
\end{enumerate}
\item %solution 19
 \begin{enumerate}[noitemsep, label=\textbf{(\alph*)} ]
\item $NS=SM$ (given)\\
$NT=TR$ (given)\\
$\therefore ST \parallel MR$ (mid-pt theorem)
\item $N\hat{S}T = 90^{\circ}\\
\therefore $ Area $\triangle SNT=\frac{1}{2} ST(SN)$ (corresp. $\angle$'s, $ST\parallel MR)\\
6 = \frac{1}{2}(4)SN\\
\therefore SN = 3$ cm\\
$\therefore MN = 6$ cm\\
$MR = 2ST = 8$ cm\\
Area $\triangle MNR = \frac{1}{2} MR \times MN\\
=\frac{1}{2} (8)(6)\\
=24$ cm$^2$
\item Let $ST$ be $x$ units\\
$\therefore MR$ will be $2x$\\
Let $SN$ be $y$ units$\\
\therefore $MN will be $2y$\\
Area $\triangle SNT = \frac{1}{2}xy\\
\frac{1}{2} xy\\$
Area $\triangle MNR = \frac{1}{2} (2x)(2y)\\
=2xy\\
\therefore$ Area $ \triangle MNR = 4(\frac{1}{2}xy)\\
=4($Area $\triangle SNT)$
\end{enumerate}
\end{enumerate}}
\end{eocsolutions}


