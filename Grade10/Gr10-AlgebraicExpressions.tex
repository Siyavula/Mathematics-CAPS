     \chapter{Algebraic Expressions}
    \fancyfoot[LO,RE]{Focus Area: Mathematics}
    \section{Rational and Irrational Numbers}
    \setcounter{figure}{1}
    \setcounter{subfigure}{1}
    \label{m38348}
            \nopagebreak
            \label{m38348*cid2} $ \hspace{-5pt}\begin{array}{cccccccccccc}   \includegraphics[width=0.75cm]{col11306.imgs/summary_video.png} &   \end{array} $ \hspace{2 pt}\raisebox{-5 pt}{} {(subsection shortcode: MG10033 )} \par 
      \label{m38348*id62184}A number is a way of representing quantity. The numbers that will be used in high school are all real numbers, but there are many different ways of writing any single real number.\par 
      \label{m38348*id62191}This section describes \textsl{rational numbers}.\par \label{m38348*eip-195}
    \setcounter{subfigure}{0}
	\begin{figure}[H] % horizontal\label{m38348*circuits-1}
    \textnormal{Khan Academy video on Integers and Rational Numbers}\vspace{.1in} \nopagebreak
  \label{m38348*yt-media1}\label{m38348*yt-video1}
            \raisebox{-5 pt}{ \includegraphics[width=0.5cm]{col11306.imgs/summary_www.png}} { (Video:  MG10034 )}
      \vspace{2pt}
    \vspace{.1in}
 \end{figure}       

\par 
    \subsection*{The big picture of numbers}
    \addcontentsline{toc}{subsection}{The big picture of numbers}

            \label{m38348*cid3} $ \hspace{-5pt}\begin{array}{cccccccccccc}   \end{array} $ \hspace{2 pt}\raisebox{-5 pt}{\includegraphics[width=0.5cm]{col11306.imgs/summary_www.png}} {(subsection shortcode: MG10035 )} \par 
      \label{m38348*id62547}
    \setcounter{subfigure}{0}
	\begin{figure}[H] % horizontal\label{m38348*id62548}
    \begin{center}
    \label{m38348*id62548!!!underscore!!!media}\label{m38348*id62548!!!underscore!!!printimage}
%\includegraphics[width=9cm]{col11306.imgs/m38348_MG10C3_001.png} % m38348;MG10C3\_001.png;;;6.0;8.5;
\scalebox{0.6} % Change this value to rescale the drawing.
{
\begin{pspicture}(0,-4.764375)(14.481563,4.804375)
\psellipse[linewidth=0.04,dimen=outer](6.81,-0.484375)(6.81,4.28)
\psline[linewidth=0.04cm](8.18,3.695625)(8.26,-4.684375)
\psellipse[linewidth=0.04,dimen=outer](4.34,-1.364375)(3.8,2.5)
\psellipse[linewidth=0.04,dimen=outer](3.57,-1.854375)(2.57,1.61)
\usefont{T1}{ppl}{b}{n}
\rput(6.735781,4.350625){\Huge REAL $\mathbb{R}$}
\usefont{T1}{ppl}{b}{n}
\rput(11.03875,-0.484375){\Large Irrational $\mathbb{Q'}$}
\usefont{T1}{ppl}{b}{n}
\rput(5.11875,1.975625){\Large Rational $\mathbb{Q}$}
\usefont{T1}{ppl}{b}{n}
\rput(4.06875,0.275625){\Large Integers $\mathbb{Z}$}
\usefont{T1}{ppl}{b}{n}
\rput(3.21875,-2.324375){\Large Natural $\mathbb{N}$}
\psellipse[linewidth=0.04,dimen=outer](3.14,-2.354375)(1.68,0.83)
\usefont{T1}{ptm}{b}{n}
\rput(3.5735939,-1.024375){\Large Whole $\mathbb{N}_0$}
\end{pspicture} 
}
      \vspace{2pt}
    \vspace{.1in}
    \end{center}
 \end{figure}       
      \par 
      We use the following definitions:\par 
      \label{m38348*id62559}\begin{itemize}[itemsep=5pt]
            \label{m38348*uid1}\item natural numbers are $\{1, 2, 3, ...\}$
\label{m38348*uid2}\item whole numbers are $\{0, 1, 2, 3, ...\}$
\label{m38348*uid3}\item integers are $\{... -3, -2, -1, 0, 1, 2, 3, ....\}$
\end{itemize}

            \nopagebreak
            \label{m38348*cid4} $ \hspace{-5pt}\begin{array}{cccccccccccc}   \end{array} $ \hspace{2 pt}\raisebox{-5 pt}{\includegraphics[width=0.5cm]{col11306.imgs/summary_www.png}} {(subsection shortcode: MG10036 )} \par 
  
\Definition{Rational number}{
      \label{m38348*id62709}A rational number is any number which can be written as: 
%       \label{m38348*uid6}\nopagebreak\noindent{}
        
    \begin{equation}
    \frac{a}{b}
      \end{equation}
      \label{m38348*id62732}where $a$ and $b$ are integers and $b\ne 0$. \par 
       } 


    \label{m38348*id62607}The following numbers are all rational numbers.\par 
      \label{m38348*uid4}\nopagebreak\noindent{}
        
    \begin{equation}
    \frac{10}{1},\frac{21}{7},\frac{-1}{-3},\frac{10}{20},\frac{-3}{6}
      \end{equation}
      \label{m38348*id62687}You can see that all denominators and all numerators are integers.\par 
% \label{m38348*fhsst!!!underscore!!!id138}\begin{definition}
% 	  \begin{tabular*}{15 cm}{m{15 mm}m{}}
% 	\hspace*{-50pt}  \includegraphics[width=0.5in]{col11306.imgs/psflag2.png}   & 



%       \end{tabular*}
%       \end{definition}
% \label{m38348*eip-761}Note that because we can write $\dfrac{a}{-b}$
% as $\dfrac{-a}{b}$
% (in other words, one can always find an equivalent rational expression where $b\greatthan{}0$) mathematicians typically define rational numbers not as both $a$ and $b$ being integers, but rather that $a$ is an integer and $b$ is a natural number. This avoids having to worry about zero in the denominator. \par \label{m38348*notfhsst!!!underscore!!!id150}
% % \begin{tabular}{cc}
% % 	   \hspace*{-50pt}\raisebox{-8 mm}{ \includegraphics[width=0.5in]{col11306.imgs/pstip2.png}  }& 
% 	\begin{minipage}{0.85\textwidth}
% 	
%       \Tip{Only fractions which have a numerator and a denominator (that is not 0) that are integers
% are rational numbers.}
% % 	\end{note}
% 	\end{minipage}
% % 	\end{tabular}
	\par
      \label{m38348*id62778}This means that all integers are rational numbers, because they can be written with a denominator of 1.\par 
      \label{m38348*id62782}Therefore\par 
      \label{m38348*uid7}\nopagebreak\noindent{}
        
    \begin{equation}
    \frac{\sqrt{2}}{7} ; \frac{20}{\pi}
      \end{equation}
      \label{m38348*id62817}are \textbf{not examples} of rational numbers, because either the numerator or the denominator is not an integer.\par 
      \label{m38348*id62829}A number may not be written as an integer divided by another integer, but may still
be a rational number. The rule is, if a number \textbf{can be} written
as a fraction of integers, it is rational even if it can be written in other
ways as well. Here are two examples that might not look like rational numbers
at first glance, but are because there are equivalent forms that are expressed as an
integer divided by another integer:\par 
      \label{m38348*uid8}\nopagebreak\noindent{}
    \begin{equation}    
    \frac{-1,33}{-3}=\frac{133}{300}; ~~~~~~\frac{-3}{6,39}=\frac{-300}{639}=\frac{-100}{213}
      \end{equation}
\label{m38348*secfhsst!!!underscore!!!id232}
% begin{exercise}   \subsubsubsection

     


    \subsection*{Decimal numbers}
    \addcontentsline{toc}{subsection}{Decimal numbers}
            \nopagebreak
            \label{m38348*cid5} $ \hspace{-5pt}\begin{array}{cccccccccccc}   \end{array} $ \hspace{2 pt}\raisebox{-5 pt}{\includegraphics[width=0.5cm]{col11306.imgs/summary_www.png}} {(subsection shortcode: MG10037 )} \par 
      \label{m38348*id63345}All integers and fractions with integer numerators and denominators are rational numbers. There are two more forms of rational numbers.\par 
\label{m38348*secfhsst!!!underscore!!!id245}



\begin{Investigation}{Decimal Numbers }
            \nopagebreak
      \label{m38348*id63357}You can write the rational number
$\frac{1}{2}$ as the decimal number 0,5. Write the following numbers as
decimals and investigate them:\par 
      \label{m38348*id63375}\begin{enumerate}[itemsep=5pt, label=\textbf{\arabic*}. ] 
            \label{m38348*uid11}\item 
          $\dfrac{1}{4}$
        \label{m38348*uid12}\item 
          $\dfrac{1}{10}$
        \label{m38348*uid13}\item 
          $\dfrac{2}{5}$
        \label{m38348*uid14}\item 
          $\dfrac{1}{100}$
        \label{m38348*uid15}\item 
          $\dfrac{2}{3}$
        \end{enumerate}
      \label{m38348*id63486}Do the numbers after the decimal comma end or do they continue? If they continue, is there a recurring pattern to the numbers? \par 
\end{Investigation}
      \label{m38348*id63495}You can write any rational number as a decimal number but not all decimal numbers are rational numbers. However, two types of decimal numbers can be written as rational numbers:\par 
      \label{m38348*id63500}\begin{enumerate}[itemsep=5pt, label=\textbf{\arabic*}. ] 
            \label{m38348*uid16}\item decimal numbers that end or \textsl{terminate}, for example the fraction $\dfrac{4}{10}$ can be written as $0,4$.
\label{m38348*uid17}\item decimal numbers that have a recurring pattern of numbers, for example the fraction $\dfrac{1}{3}$ can be written as 
$0,\dot{3}$. 
The dot represents recurring $3$'s i.e.,
$0,333...=0,\dot{3}$.
\end{enumerate}
      \label{m38348*id63576}The rational number $\dfrac{5}{6}$ can be written in decimal notation as $0,8\dot{3}$ and similarly, the decimal number 0,25 can be written as a rational number as $\dfrac{1}{4}$.\par 
\label{m38348*notfhsst!!!underscore!!!id301}
% \begin{tabular}{cc}
% 	   \hspace*{-50pt}\raisebox{-8 mm}{ \includegraphics[width=0.5in]{col11306.imgs/pstip2.png}  }& 
% 	\begin{minipage}{0.85\textwidth}
% 	\begin{note}
      \Tip{You can use a dot or a bar over the repeated numbers to indicate that the decimal is a recurring decimal.}
% 	\end{note}
% 	\end{minipage}
% 	\end{tabular}
	\par

    \subsection*{Converting terminating decimals into rational numbers}
    \addcontentsline{toc}{subsection}{Converting terminating decimals into rational numbers}
%             \nopagebreak
%             \label{m38348*cid6} $ \hspace{-5pt}\begin{array}{cccccccccccc}   \end{array} $ \hspace{2 pt}\raisebox{-5 pt}{\includegraphics[width=0.5cm]{col11306.imgs/summary_www.png}} {(subsection shortcode: MG10038 )} \par 
%       \label{m38348*id63646}A decimal number has an integer part and a fractional part. For example $10,589$ has an integer part of 10 and a fractional part of $0,589$ because $10+0,589=10,589$. The fractional part can be written as a rational number, i.e. with a numerator and a denominator that are integers.\par 
%       \label{m38348*id63704}Each digit after the decimal point is a fraction with a denominator in increasing powers of ten. For example:\par 
% 
% 
%       \label{m38348*id63708}\begin{itemize}[itemsep=5pt]
%             \label{m38348*uid18}\item $\dfrac{1}{10}$ is $0,1$\label{m38348*uid19}\item $\dfrac{1}{100}$ is $0,01$\end{itemize}
%       \label{m38348*id63781}This means that:\par 
%       \label{m38348*id63784}\nopagebreak\noindent{}
%     \begin{equation}
%    \hfill 10,589& =& 10+\frac{5}{10}+\frac{8}{100}+\frac{9}{1000}\hfill \\ & =& 10\frac{589}{1000}\hfill \\ & =& \frac{10589}{1000}\hfill 
%       \end{equation}
% \label{m38348*secfhsst!!!underscore!!!id378}
% 
% 

% 
% 
%     \subsection{ Converting Repeating Decimals into Rational Numbers}
            \nopagebreak
            \label{m38348*cid7} $ \hspace{-5pt}\begin{array}{cccccccccccc}   \end{array} $ \hspace{2 pt}\raisebox{-5 pt}{\includegraphics[width=0.5cm]{col11306.imgs/summary_www.png}} {(subsection shortcode: MG10039 )} \par 


      \label{m38348*id63993}When the decimal is a recurring decimal, a bit more work is needed to write the fractional part of the decimal number as a fraction.\par 
      
\begin{wex}
{%title
Converting decimal numbers to fractions
}
{%question
\\
Write $0,\dot{3}$ in the form $\dfrac{a}{b}$ (where $a$ and $b$ are integers)
}
{%answer

\westep{}
$$x = 0,33333\ldots$$


\westep{Multiply by 10 on both sides}

$$10x = 3,33333\ldots$$


\westep{Subtract the second equation from the first equation}

$$9x = 3 $$

\westep{Simplify}

$$ x = \dfrac{3}{9} = \dfrac{1}{3} $$
}
\end{wex}


\begin{wex}
{%title
Converting decimal numbers to fractions
}
{%question
\\Write $5,\dot{4}\dot{3}\dot{2}$ as a rational fraction
}
{%answer

\westep{}

$$ x = 5,432432432\ldots $$

\westep{Multiply by 1000 on both sides}

$$ 1000x = 5432,432432432\ldots $$

\westep{Subtract the second equation from the first}

$$ 999x = 5427 $$

\westep{Simplify}

$$ x = \dfrac{5427}{999} = \dfrac{201}{37} $$
 
}
\end{wex}





%       \label{m38348*id64237}And another example would be to write 
% $5,\dot{4}\dot{3}\dot{2}$ 
% as a rational fraction.
% \par 
%       \label{m38348*uid22}\nopagebreak\noindent{}
%     \begin{equation}
%     \begin{array}{cccc}\hfill x& =& 5,432432432...\hfill & \\ \hfill 1000x& =& 5432,432432432...\hfill & \text{multiply by\hspace{0.17em}}\phantom{\rule{4.pt}{0ex}}\text{1000}\phantom{\rule{4.pt}{0ex}}\text{on both sides}\\ \hfill \\ \hfill 999x& =& 5427\hfill & \text{(}\text{subtracting the second equation from the first equation}\text{)}\\ \hfill x& =& \frac{5427}{999}=\frac{201}{37}\hfill & \end{array}
%       \end{equation}
      \label{m38348*id64459}For the first example, the decimal was multiplied by 10 and for the second example, the decimal was multiplied by 1000. This is because for the first example there was only one digit (i.e. 3) recurring, while for the second example there were three digits (i.e. 432) recurring.\par 
      \label{m38348*id64465}In general, if you have one digit recurring, then multiply by 10. If you have two digits recurring, then multiply by 100. If you have three digits recurring, then multiply by 1000 and so on.\par

      \label{m38348*id64474}Not all decimal numbers can be written as rational numbers. Why? Irrational decimal numbers like 
$\sqrt{2}=1,4142135...$
cannot be written with an integer numerator and denominator, because they do not have a pattern of recurring digits and they do not terminate. However, when possible, you should try to use rational numbers or fractions instead of decimals.
% \label{m38348*secfhsst!!!underscore!!!id606}




    \section{Irrational numbers}
    \setcounter{figure}{1}
    \setcounter{subfigure}{1}
    \label{m38349}
%     Introduction
            \nopagebreak
            \label{m38349*cid2} $ \hspace{-5pt}\begin{array}{cccccccccccc}   \end{array} $ \hspace{2 pt}\raisebox{-5 pt}{\includegraphics[width=0.5cm]{col11306.imgs/summary_www.png}} {(subsection shortcode: MG10055 )} \par 
      \label{m38349*id324260}You have seen that repeating decimals may take a lot of paper and ink to write out. Not only is that impossible, but writing numbers out to many decimal places or \textsl{a high accuracy} is very inconvenient and rarely gives practical answers. For this reason we often estimate the number to a certain number of decimal places or to a given number of \textsl{significant figures}, which is even better.\par 
    \Definition{Irrational Numbers}{
            \nopagebreak
%             \label{m38349*cid3} $ \hspace{-5pt}\begin{array}{cccccccccccc}   \end{array} $ \hspace{2 pt}\raisebox{-5 pt}{\includegraphics[width=0.5cm]{col11306.imgs/summary_www.png}} {(subsection shortcode: MG10056 )} \par \label{m38349*id324624}
Irrational numbers are numbers that cannot be written as a fraction with the numerator and denominator as integers. This means that any number that is \textsl{not} a terminating or recurring decimal number is irrational. 

}

Examples of irrational numbers are:\par 
%       \label{m38349*id324635}\nopagebreak\noindent{}
        
    \begin{equation}
 \sqrt{2},\sqrt{3},\sqrt[3]{4},\pi ,
\frac{1+\sqrt{5}}{2}\approx 1,618
      \end{equation}


\label{m38349*notfhsst!!!underscore!!!id128}
% \begin{tabular}{cc}
% 	   \hspace*{-50pt}\raisebox{-8 mm}{ \includegraphics[width=0.5in]{col11306.imgs/pstip2.png}  }& 
% 	\begin{minipage}{0.85\textwidth}
% 	\begin{note}
      \Tip{When irrational numbers are written in decimal form, they go on forever and
there is no repeated pattern of digits. The square roots of non-perfect squares and the cube roots of non-perfect cubes are all irrational}
% 	\end{note}
% 	\end{minipage}
% 	\end{tabular}
	\par
      \label{m38349*id324739}If you are asked to identify whether a number is rational or irrational, first write the number in decimal form. If the number terminates then it is rational. If it goes on forever, then look for a repeated pattern of digits. If there is no repeated pattern, then the number is irrational.\par 
      \label{m38349*id324745}When you write irrational numbers in decimal form, you may (if you have a lot of time and paper!) continue writing them for many, many decimal places. However, this is not convenient and it is often necessary to round off.\par 
\label{m38349*secfhsst!!!underscore!!!id133}


\begin{activity}{Irrational Numbers }
            \nopagebreak
      \label{m38349*id324757}Which of the following cannot be
written as a rational number?\par \vspace{0.5cm}
      \label{m38349*id324763}\textbf{Remember}: A rational number is a fraction with numerator and denominator as integers. Terminating decimal numbers or recurring decimal numbers are rational.\par 
      \label{m38349*id324775}\begin{enumerate}[itemsep=5pt, label=\textbf{\arabic*}. ] 
            \label{m38349*uid1}\item 
          $\pi =3,14159265358979323846264338327950288419716939937510...$
        \label{m38349*uid2}\item $1,4$
\label{m38349*uid3}\item 
          $1,618\phantom{\rule{0.166667em}{0ex}}033\phantom{\rule{0.166667em}{0ex}}989\phantom{\rule{0.166667em}{0ex}}...$
        \label{m38349*uid4}\item $100$
	\item $1,7373737373\ldots$
	\item $0,\overline{02}$
\end{enumerate}
\end{activity}


\begin{exercises}{}{
            \nopagebreak
            \label{m38348*id63121}\begin{enumerate}[itemsep=5pt, label=\textbf{\arabic*}. ] 
            \label{m38348*uid9}\item If $a$ is an integer, $b$ is an integer and $c$ is irrational, which of the following are rational numbers? 
\label{m38348*id734}\begin{enumerate}[itemsep=5pt, label=\textbf{\alph*}. ] 
            \item $\frac{5}{6}$\newline
    \item $\dfrac{a}{3}$\newline
    \item $\dfrac{-2}{b}$\newline
    \item $\dfrac{1}{c}$\end{enumerate}
        \label{m38348*uid10}\item If $\dfrac{a}{1}$ is a rational number, which of the following are valid values for $a$?\label{m38348*id7432}\begin{enumerate}[itemsep=5pt, label=\textbf{\alph*}. ] 
            \item 1\item $-10$\item $\sqrt{2}$\item $2,1$\end{enumerate}
%         \end{enumerate}
% \par \raisebox{-5 pt}{\includegraphics[width=0.5cm]{col11306.imgs/summary_www.png}} Find the answers with the shortcodes:
%  \par \begin{tabular}[h]{cccccc}
%  (1.) l35  &  (2.) l3N  & \end{tabular}}
% \end{exercises}
% 
%            \begin{exercises}{Fractions }{
%             \nopagebreak
%       \label{m38348*id63882}\begin{enumerate}[itemsep=5pt, label=\textbf{\arabic*}. ] 
            \label{m38348*uid20}\item Write the following as fractions:\label{m38348*id7322}
	     \begin{enumerate}[itemsep=5pt, label=\textbf{\alph*}. ] 
            \item $0,1$\item $0,12$\item $0,58$\item $0,2589$
	    \end{enumerate}
%         \end{enumerate}
% \par \raisebox{-5 pt}{\includegraphics[width=0.5cm]{col11306.imgs/summary_www.png}} Find the answers with the shortcodes:
%  \par \begin{tabular}[h]{cccccc}
%  (1.) l3R  & \end{tabular}}
% \end{exercises}
% 
% \begin{exercises}{Repeated Decimal Notation }{
% %             \nopagebreak
%       \label{m38348*id64513}\begin{enumerate}[itemsep=5pt, label=\textbf{\arabic*}. ] 
            \label{m38348*uid23}\item Write the following using the recurring decimal notation:
\label{m38348*id64529}\begin{enumerate}[itemsep=5pt, label=\textbf{\alph*}. ] 
            \label{m38348*uid24}\item $0,11111111...$\label{m38348*uid25}\item $0,1212121212...$\label{m38348*uid26}\item $0,123123123123...$\label{m38348*uid27}\item $0,11414541454145...$\end{enumerate}
        \label{m38348*uid28}\item Write the following in decimal form, using the recurring decimal notation:
\label{m38348*id64650}\begin{enumerate}[itemsep=5pt, label=\textbf{\alph*}. ] 
            \label{m38348*uid29}\item $\frac{2}{3}$\label{m38348*uid30}\item $1\frac{3}{11}$\label{m38348*uid31}\item $4\frac{5}{6}$\label{m38348*uid32}\item $2\frac{1}{9}$\end{enumerate}
        \label{m38348*uid33}\item Write the following decimals in fractional form:
\label{m38348*id64767}\begin{enumerate}[itemsep=5pt, label=\textbf{\alph*}. ] 
            \label{m38348*uid34}\item $0,\dot{5}$\label{m38348*uid35}\item $0,6\dot{3}$\label{m38348*uid36}\item $5,\overline{31}$\end{enumerate}
        \end{enumerate}
% \par \raisebox{-5 pt}{\includegraphics[width=0.5cm]{col11306.imgs/summary_www.png}} Find the answers with the shortcodes:
%  \par \begin{tabular}[h]{cccccc}
%  (1.) l3U  &  (2.) l3n  &  (3.) l3Q  & \end{tabular}}
\end{exercises}

    \section{ Rounding off}
            \nopagebreak
            \label{m38349*cid4} $ \hspace{-5pt}\begin{array}{cccccccccccc}   \includegraphics[width=0.75cm]{col11306.imgs/summary_fullmarks.png} &   \end{array} $ \hspace{2 pt}\raisebox{-5 pt}{} {(section shortcode: MG10057 )} \par 
      \label{m38349*id324198}Rounding off or approximating a decimal number to a given number of decimal places is the quickest way to approximate a number. For example, if you wanted to round-off $2,6525272$ to three decimal places, you would first count three places after the decimal and place a $|$ between the third and fourth numbers after the decimal.\par 
      \label{m38349*id325085}\nopagebreak\noindent{}
        
    \begin{equation}
    2,652|5272
      \end{equation}
      \label{m38349*id325105}All numbers to the right of the $|$ are ignored after you have determined whether the number in the third decimal place must be rounded up or rounded down. You \textsl{round up} the final digit if the first digit after the $|$ was greater than or equal to 5 and \textsl{round down} (leave the digit unchanged) otherwise. In the case that the first digit before the $|$ is 9 \textsl{and} you need to round up, then the 9 becomes a 0 and the second digit before the $|$ is rounded up.\par 
      \label{m38349*id325160}So, since the first digit after the $|$ is a 5, we must round up the digit in the third decimal place to a 3 and the final answer of $2,6525272$ rounded to three decimal places is\par 
      \label{m38349*id325186}\nopagebreak\noindent{}
        
    \begin{equation}
    2,653
      \end{equation}
\par
            \label{m38349*secfhsst!!!underscore!!!id199}\vspace{.5cm} 
      \noindent
%       \hspace*{-30pt}\includegraphics[width=0.5in]{col11306.imgs/pspencil2.png}   \raisebox{25mm}{   
 
      \begin{wex}{Rounding-Off }
%       \label{m38349*probfhsst!!!underscore!!!id200}
%       \label{m38349*id325213}
{Round-off the following numbers to the indicated number of decimal places:\par 
      \label{m38349*id325219}\begin{enumerate}[itemsep=5pt, label=\textbf{\arabic*}. ] 
%             \leftskip=20pt\rightskip=\leftskip\label{m38349*uid5}
\item $\frac{120}{99}=1,212121212\dot{1}\dot{2}$ to 3 decimal places
\label{m38349*uid6}\item $\pi =3,141592654...$ to 4 decimal places
\label{m38349*uid7}\item $\sqrt{3}=1,7320508...$ to 4 decimal places
\label{m38349*uid789}\item $2,78974526...$ to 3 decimal places\end{enumerate}}
{
%       \vspace{5pt}
%       \label{m38349*solfhsst!!!underscore!!!id212}\noindent\textbf{Solution to Exercise } \label{m38349*listfhsst!!!underscore!!!id212}[itemsep=5pt, label=\textbf{Step} \textbf{\arabic*}. ] 
%             \leftskip=20pt\rightskip=\leftskip\item  
%       \label{m38349*id325360}[itemsep=5pt, label=\textbf{\alph*}. ] 
%             \leftskip=20pt\rightskip=\leftskip\label{m38349*uid8}

\westep{}

\begin{enumerate}[itemsep=5pt, label=\textbf{\arabic*}. ] 
\item  $\frac{120}{99}=1,212|121212\dot{1}\dot{2}$
%         \label{m38349*uid9}
\item           $\pi =3,1415|92654...$
%         \label{m38349*uid10}
\item           $\sqrt{3}=1,7320|508...$
\item $2,789|74526...$
\newline
\end{enumerate}
\westep{}
\begin{enumerate}[itemsep=5pt, label=\textbf{\arabic*}. ]
%       \item  
%       \label{m38349*id325490}\begin{enumerate}[itemsep=5pt, label=\textbf{\alph*}. ] 
%             \leftskip=20pt\rightskip=\leftskip\label{m38349*uid11}
\item The last digit of $\frac{120}{99}=1,212|121212\dot{1}\dot{2}$  must be rounded down.
\item The last digit of $\pi =3,1415|92654...$ must be rounded up.
\item The last digit of $\sqrt{3}=1,7320|508...$ must be rounded up.
\item  The last digit of $2,789|74526...$ must be rounded up. Since this is a 9, we replace it with a 0 and round up the second last digit.
\newline
\end{enumerate}
\westep{}
\begin{enumerate}[itemsep=5pt, label=\textbf{\arabic*}. ]
   
%       \label{m38349*id325626}[itemsep=5pt, label=\textbf{\alph*}. ] 
%             \leftskip=20pt\rightskip=\leftskip\label{m38349*uid14}
\item $\frac{120}{99}=1,212$ rounded to 3 decimal places
\item $\pi =3,1416$  rounded to 4 decimal places
\item $\sqrt{3}=1,7321$ rounded to 4 decimal places
\item $2,790$
\newline
\end{enumerate}
   
}  
    \end{wex}

    }
    \noindent
         \section{Estimating Surds}
    \setcounter{figure}{1}
    \setcounter{subfigure}{1}
    \label{m38347}
%     \subsection{ Introduction}
            \nopagebreak
            \label{m38347*cid1} $ \hspace{-5pt}\begin{array}{cccccccccccc}   \includegraphics[width=0.75cm]{col11306.imgs/summary_fullmarks.png} &   \end{array} $ \hspace{2 pt}\raisebox{-5 pt}{} {(subsection shortcode: MG10052 )} \par 
      If the ${n}^{\mathrm{th}}$ root of a number cannot be simplified to a rational number, we call it a $\mathit{surd}$. For example, $\sqrt{2}$ and $\sqrt[3]{6}$ are surds, but $\sqrt{4}$ is not a surd because it can be simplified to the rational number 2.\par 
      \label{m38347*id258405}In this chapter we will only look at surds that look like $\sqrt[n]{a}$, where $a$ is any positive number, for example $\sqrt{7}$ or $\sqrt[3]{5}$. It is very common for $n$ to be 2, so we usually do not write $\sqrt[2]{a}$. Instead we write the surd as just $\sqrt{a}$, which is much easier to read.\par 
      \label{m38347*id258479}It is sometimes useful to know the approximate value of a surd without having to use a calculator. For example, we want to be able to estimate where a surd like $\sqrt{3}$ is on the number line. So how do we know where surds lie on the number line? From a calculator we know that $\sqrt{3}$ is equal to $1,73205...$. It is easy to see that $\sqrt{3}$ is above 1 and below 2. But to see this for other surds like $\sqrt{18}$ without using a calculator, you must first understand the following:\par 
\label{m38347*notfhsst!!!underscore!!!id71}
% \begin{tabular}{cc}
% 	\hspace*{-50pt}\raisebox{-8 mm}{\hspace{-0.2in}\includegraphics[width=0.75in]{col11306.imgs/psfact2.png} } & 
% 	\begin{minipage}{0.85\textwidth}
% 	\begin{note}
      \Identity{}{
\begin{center}
If $a$ and $b$ are positive whole numbers, and $a\lessthan{}b$, then $\sqrt[n]{a}\lessthan{}\sqrt[n]{b}$. 
\end{center}}
	\par
%       
      \Note{A perfect square is the number obtained when an integer is squared. For example, 9 is a perfect square since ${3}^{2}=9$. Similarly, a perfect cube is a number which is the cube of an integer. For example, 27 is a perfect cube, because ${3}^{3}=27$.}
% 	\end{note}
% 	\end{minipage}
% 	\end{tabular}
	\par
%       \label{m38347*id258890}To make it easier to use our idea, we will create a list of some of the perfect squares and perfect cubes. The list is shown in Table 6.1.\par 
%     % \textbf{m38347*uid1}\par
%           \begin{table}[H]
%     % \begin{table}[H]
%     % \\ '' '0'
%         \begin{center}
%       \label{m38347*uid1}
%     \noindent
%     \tabletail{%
%         \hline
%         \multicolumn{3}{|p{\mytableboxwidth}|}{\raggedleft \small \sl continued on next page}\\
%         \hline
%       }
%       \tablelasttail{}
%       \begin{xtabular}[t]{|l|l|l|}\hline
%         Integer &
%         Perfect Square &
%         Perfect Cube% make-rowspan-placeholders
%      \tabularnewline\cline{1-1}\cline{2-2}\cline{3-3}
%       %--------------------------------------------------------------------
%         0 &
%         0 &
%         0% make-rowspan-placeholders
%      \tabularnewline\cline{1-1}\cline{2-2}\cline{3-3}
%       %--------------------------------------------------------------------
%         1 &
%         1 &
%         1% make-rowspan-placeholders
%      \tabularnewline\cline{1-1}\cline{2-2}\cline{3-3}
%       %--------------------------------------------------------------------
%         2 &
%         4 &
%         8% make-rowspan-placeholders
%      \tabularnewline\cline{1-1}\cline{2-2}\cline{3-3}
%       %--------------------------------------------------------------------
%         3 &
%         9 &
%         27% make-rowspan-placeholders
%      \tabularnewline\cline{1-1}\cline{2-2}\cline{3-3}
%       %--------------------------------------------------------------------
%         4 &
%         16 &
%         64% make-rowspan-placeholders
%      \tabularnewline\cline{1-1}\cline{2-2}\cline{3-3}
%       %--------------------------------------------------------------------
%         5 &
%         25 &
%         125% make-rowspan-placeholders
%      \tabularnewline\cline{1-1}\cline{2-2}\cline{3-3}
%       %--------------------------------------------------------------------
%         6 &
%         36 &
%         216% make-rowspan-placeholders
%      \tabularnewline\cline{1-1}\cline{2-2}\cline{3-3}
%       %--------------------------------------------------------------------
%         7 &
%         49 &
%         343% make-rowspan-placeholders
%      \tabularnewline\cline{1-1}\cline{2-2}\cline{3-3}
%       %--------------------------------------------------------------------
%         8 &
%         64 &
%         512% make-rowspan-placeholders
%      \tabularnewline\cline{1-1}\cline{2-2}\cline{3-3}
%       %--------------------------------------------------------------------
%         9 &
%         81 &
%         729% make-rowspan-placeholders
%      \tabularnewline\cline{1-1}\cline{2-2}\cline{3-3}
%       %--------------------------------------------------------------------
%         10 &
%         100 &
%         1000% make-rowspan-placeholders
%      \tabularnewline\cline{1-1}\cline{2-2}\cline{3-3}
%       %--------------------------------------------------------------------
%     \end{xtabular}
%       \end{center}
%     \begin{center}{\small\bfseries Table 6.1}: Some perfect squares and perfect cubes\end{center}
%     \begin{caption}{\small\bfseries Table 6.1}: Some perfect squares and perfect cubes\end{caption}
\end{table}
    \par
      \label{m38347*id259412}Consider the surd $\sqrt[3]{52}$, it lies somewhere between 3 and 4, because $\sqrt[3]{27}=3$ and $\sqrt[3]{64}=4$ and 52 is between 27 and 64. Checking on a calculator we see that $\sqrt[3]{52}=3,73...$ which is indeed between 3 and 4.\par 
\label{m38347*secfhsst!!!underscore!!!id162}\vspace{.5cm} 
      \noindent
%       \hspace*{-30pt}\includegraphics[width=0.5in]{col11306.imgs/pspencil2.png}   \raisebox{25mm}{   
  
      \begin{wex}{ Estimating Surds }{
%       \label{m38347*probfhsst!!!underscore!!!id163}
      \label{m38347*id259741}Find the two consecutive integers such that $\sqrt{26}$ lies between them.\par 
      \label{m38347*id259757}(Remember that consecutive numbers are two numbers that follow one another, like 5 and 6 or 8 and 9 in the set of integers) \par }
%       \vspace{5pt}
%       \label{m38347*solfhsst!!!underscore!!!id167}\noindent\textbf{Solution to Exercise } \label{m38347*listfhsst!!!underscore!!!id167}
{
% \begin{enumerate}[itemsep=5pt, label=\textbf{Step} \textbf{\arabic*}. ] 
%             \leftskip=20pt\rightskip=\leftskip\item  
 \westep{}     \label{m38347*id259781}${5}^{2}=25$. Therefore $5\lessthan{}\sqrt{26}$.\par 
%       \item  
     \westep{} \label{m38347*id259824} ${6}^{2}=36$. Therefore $\sqrt{26}\lessthan{}6$.\par 
%       \item  
      \westep{}\label{m38347*id259866}Our answer is $5\lessthan{}\sqrt{26}\lessthan{}6$. \par 
%       \end{enumerate}
    }
\end{wex}



   
\begin{wex}
{
Estimating Surds
}
{%question
Find the two consecutive integers such that $\sqrt[3]{49}$ lies between them:
}
{
If $3\lessthan{}\sqrt[3]{49}\lessthan{}4$ then cubing all terms gives ${3}^{3}\lessthan{}49\lessthan{}{4}^{3}$. Simplifying gives $27\lessthan{}49\lessthan{}64$ which is true. So $\sqrt[3]{49}$ lies between 3 and 4.
}
\end{wex}

    \section{Products}
    \setcounter{figure}{1}
    \setcounter{subfigure}{1}
    \label{d4e6ddcad4e2d9e383c4732da6858c66}
%       Introduction and recap
    \nopagebreak
            \label{m39383} $ \hspace{-5pt}\begin{array}{cccccccccccc}   \includegraphics[width=0.75cm]{col11306.imgs/summary_fullmarks.png} &   \end{array} $ \hspace{2 pt}\raisebox{-5 pt}{} {(section shortcode: MG10060 )} \par 
%   
            \nopagebreak
        \label{m39383*id267144}Mathematical expressions are just like sentences and their parts have special names. You should be familiar with the following names used to describe the parts of  mathematical expressions.\par 
        \label{m39383*uid2}\nopagebreak\noindent{}
          
    \begin{equation}
    3x^2 + 7xy -14 = 0
      \end{equation}



\begin{table}[H]
 \begin{center}
\begin{tabular}{|l|l|}
\hline
\textbf{Name} & \textbf{Examples} \\
\hline
terms & $3x^2, 7xy, -14$\\ \hline
expression & $3x^2 + 7xy -14$\\ \hline
coefficient & $3,7$\\ \hline
exponent & $2,1$\\ \hline
base & $x,y$\\ \hline
constant & $3,7,-14$\\ \hline
variable & $x, y$\\ \hline
equation & $3x^2 + 7xy -14 = 0$\\ \hline
binomial & expression with two terms\\ \hline
trinomial & expression with three terms \\ \hline
 \end{tabular}
 \end{center}
\end{table} 

   \par
      \label{m39383*uid4}
            \subsection*{ Product of two binomials}
	    \addcontentsline{toc}{subsection}{Product of two binomials}
            \nopagebreak
        \label{m39383*id268015}A \textsl{binomial} is a mathematical expression with two terms, e.g. $ax+b$ and $cx+d$. If these two binomials are multiplied (or expanded) we get the following:
        \label{m39383*id268064}\nopagebreak\noindent{}
    \begin{equation}
    \begin{array}{ccc}\hfill \left(ax+b\right)\left(cx+d\right) & =& \left(ax\right)\left(cx\right)+\left(ax\right)d+b\left(cx\right)+bd\hfill \\ & =& ac{x}^{2}+x\left(ad+bc\right)+bd\hfill \end{array}
      \end{equation}

\Note{Remember to use FOIL = F(firsts) O(outers) I(inners) L(lasts)}
\par
            \label{m39383*secfhsst!!!underscore!!!id342}\vspace{.5cm} 
      \noindent
%       \hspace*{-30pt}\includegraphics[width=0.5in]{col11306.imgs/pspencil2.png}   \raisebox{25mm}{   

      \begin{wex}{Product of two binomials }{\label{m39383*probfhsst!!!underscore!!!id343}
        \label{m39383*id268279}Find the product of $\left(3x-2\right)\mbox{ and }\left(5x+8\right)$ \par }{
%         \vspace{5pt}
%         \label{m39383*solfhsst!!!underscore!!!id346}\noindent\textbf{Solution to Exercise } \label{m39383*listfhsst!!!underscore!!!id346}[itemsep=5pt, label=\textbf{Step} \textbf{\arabic*}. ] 
            \leftskip=20pt\rightskip=\leftskip\item  
        \label{m39383*id268333}\nopagebreak\noindent{}
          
    \begin{equation}
    \begin{array}{ccl}\hfill \left(3x-2\right)\left(5x+8\right)& =& \left(3x\right)\left(5x\right)+\left(3x\right)\left(8\right)+\left(-2\right)\left(5x\right)+\left(-2\right)\left(8\right)\hfill \\ & =& 15{x}^{2}+24x-10x-16\hfill \\ & =& 15{x}^{2}+14x-16\hfill \end{array}
      \end{equation}
      }} 
    \end{wex}

 
    }
    \noindent
        \label{m39383*id268534}The product of two identical binomials is known as the \textsl{square of the binomial} and is written as:\par 
        \label{m39383*id268543}\nopagebreak\noindent{}
          
    \begin{equation}
    {\left(ax+b\right)}^{2}={a}^{2}{x}^{2}+2abx+{b}^{2}
      \end{equation}
        \label{m39383*id268608}If the two terms are $ax+b$\hspace{1ex} and $ax-b$\hspace{1ex} then their product is:\par 
        \label{m39383*id268642}\nopagebreak\noindent{}
          
    \begin{align*}
    \left(ax+b\right)\left(ax-b\right) &={a}^{2}{x}^{2}-{b}^{2} \\ &= (ax)^2-b^2
    \end{align*}

        \label{m39383*id268705}This product is known as the \textsl{difference of two squares}.\par 
  
    \label{m39387*cid4}
            \subsection*{ Multiplication of a binomial with a trinomial}
	    \addcontentsline{toc}{subsection}{Multiplication of a binomial with a trinomial}
            \nopagebreak
            \label{m39387*eip-109}
    \setcounter{subfigure}{0}


	\begin{figure}[H] % horizontal\label{m39387*productpolynomials}
    
    
    \textnormal{Khan Academy video on products of polynomials.}\vspace{.1in} \nopagebreak
  \label{m39387*yt-media1}\label{m39387*yt-video1}
            \raisebox{-5 pt}{ \includegraphics[width=0.5cm]{col11306.imgs/summary_www.png}} { (Video:  MG10062 )}
      
      \vspace{2pt}
    \vspace{.1in}
    
    

 \end{figure}   

    \addtocounter{footnote}{-0}
    Now we learn how to multiply a binomial (expression with two terms) by a
trinomial (expression with three terms). We can use the same methods we used to
multiply two binomials to multiply a binomial and a trinomial.\par 
      
      \Tip{ \label{m39387*id271813}If the binomial is \begin{math}A+B\end{math} and the trinomial is \begin{math}C+D+E\end{math}, then the very first step is to apply the distributive law:\par 
      \begin{equation}
    \left(A+B\right)\left(C+D+E\right)=A\left(C+D+E\right)+B\left(C+D+E\right)
      \end{equation}
     
    If you remember this, you will never go wrong!\par }

\begin{wex}
{Multiplication of Binomial with Trinomial 
}
{
Multiply \begin{math}x-1\end{math} with \begin{math}{x}^{2}-2x+1\end{math}.
} 
{
\westep{Apply the distributive law}
$$
(x-1)(x^2-2x+1) &= x(x^2-2x+1)-1(x^2-2x+1)
$$

\westep{Expand the bracket}

$$
\phantom{(x-1)(x^2-2x+1) } = x^3-2x^2+x-x^2+2x-1
$$

\westep{Simplify}

$$
\phantom{(x-1)(x^2-2x+1) } = x^3-3x^2 + 3x1
$$

%     \begin{equation}
%     \begin{array}{cccc}& \phantom{\rule{4pt}{0ex}}& \left(x-1\right)\left({x}^{2}-2x+1\right)\hfill & \\ & =& x\left({x}^{2}-2x+1\right)-1\left({x}^{2}-2x+1\right)\hfill & \left(\mathrm{apply\; distributive\; law}\right)\hfill \\ & =& \left[x\left({x}^{2}\right)+x\left(-2x\right)+x\left(1\right)\right]+\left[-1\left({x}^{2}\right)-1\left(-2x\right)-1\left(1\right)\right]\hfill & \\ & =& {x}^{3}-2{x}^{2}+x-{x}^{2}+2x-1\hfill & \left(\mathrm{expand\; the\; brackets}\right)\hfill & \\ & =& {x}^{3}+\left(-2{x}^{2}-{x}^{2}\right)+\left(x+2x\right)-1\hfill & \left(\mathrm{group\; like\; terms\; to\; simplify}\right)\hfill & \\ & =& {x}^{3}-3{x}^{2}+3x-1\hfill & \left(\mathrm{simplify\; to\; get\; final\; answer}\right)\hfill & \end{array}
%       \end{equation}

%     
      
%       \westep{}  
%       \label{m39387*id272535}The product of \begin{math}x-1\end{math} and \begin{math}{x}^{2}-2x+1\end{math} is \begin{math}{x}^{3}-3{x}^{2}+3x-1\end{math}. \par 
% %       \end{enumerate}
  }       

    \end{wex}

    }
    \noindent
  
\label{m39387*secfhsst!!!underscore!!!id1562}\vspace{.5cm}

\begin{exercises}{Finding Products}
 Find the products:

\begin{multicols}{2}
\begin{enumerate}[label=\textbf{\arabic*}., itemsep=5pt]
\item $2y(y+4)$ 
\item $(y+5)(y+2) $
\item $(2-t)(1-2t)$
\item $(x-4)(x+4)$
\item $ (2p+9)(3p+1)$
\item $(3k-2)(k+6)$
\item $(s+6)^2$
\item $-(7-x)(7+x)$
\item $(3x-1)(3x+1)$
\item $(7k+2)(3-2k)$
\item $(1-4x)^2$
\item $(-3-y)(5-y)$
\item $(8-x)(8+x)$
\item $(9+x)^2$
\item$\left(-2{y}^{2}-4y+11\right)\left(5y-12\right)$ 
\item$\left(7{y}^{2}-6y-8\right)\left(-2y+2\right)$% make-rowspan-placeholders
\item$\left(10{y}^{5}+3\right)\left(-2{y}^{2}-11y+2\right)$ 
\item$\left(-12y-3\right)\left(12{y}^{2}-11y+3\right)$% make-rowspan-placeholders
\item$\left(-10\right)\left(2{y}^{2}+8y+3\right)$ 
\item$\left(2{y}^{6}+3{y}^{5}\right)\left(-5y-12\right)$% make-rowspan-placeholders
\item$\left(-7y+11\right)\left(-12y+3\right)$% make-rowspan-placeholders
\item$\left(7y+3\right)\left(7{y}^{2}+3y+10\right)$% make-rowspan-placeholders
\item$\left(9\right)\left(8{y}^{2}-2y+3\right)$ 
\item$\left(-6{y}^{4}+11{y}^{2}+3y\right)\left(10y+4\right)\left(4y-4\right)$ 
\end{enumerate}
\end{multicols}
\end{exercises}





\section{ Factorisation}

    \nopagebreak
            \label{m39394} $ \hspace{-5pt}\begin{array}{cccccccccccc}   \includegraphics[width=0.75cm]{col11306.imgs/summary_fullmarks.png} &   \includegraphics[width=0.75cm]{col11306.imgs/summary_video.png} &   \end{array} $ \hspace{2 pt}\raisebox{-5 pt}{} {(section shortcode: MG10063 )} \par 
%            \subsubsubsection{ Factorisation}
%             \nopagebreak
        \label{m39383*id268725}Factorisation is the opposite process of expanding brackets. For example expanding brackets would require $2\left(x+1\right)$ to be written as $2x+2$. Factorisation would be to start with $2x+2$\hspace{1ex} and to end up with $2\left(x+1\right)$. 

\Identity{}
{
\begin{center}
\begin{small}\hspace{8pt}expanding\end{small}\\
\begin{Large}
$2(x+1)$ \begin{Huge} $\rightleftharpoons$ \end{Huge} $2x+2$
\end{Large}\\
\begin{small}\hspace{8pt}factorising\end{small}
\end{center}
}

The two expressions $2(x+1)$ and $2x+2$ are equivalent -- they are the same value for all values of $x$



\par
In previous grades, we factorised by taking out a common factor and using difference of squares.\par 

        \label{m39383*uid6}
          \subsection*{ Common Factors}
            \nopagebreak
          \label{m39383*id268808}Factorising based on common factors relies on there being factors common to all the terms. \par
          
	  For example, $2x-6{x}^{2}$\hspace{1ex}can be factorised as follows:\par 
          \label{m39383*id268835}\nopagebreak\noindent{}
            
    \begin{equation}
    2x-6{x}^{2}=2x\left(1-3x\right)
      \end{equation}
\label{m39383*secfhsst!!!underscore!!!id565}


\begin{wex}{Factorisation }
{Factorise completely: ${b}^{2}{y}^{5}-3ab{y}^{3}$}
{
\westep{The highest common factor is $by^3$}  
\begin{equation}
{b}^{2}{y}^{5}-3ab{y}^{3}& =& b{y}^{3}\left(b{y}^{2}-3a\right)
\end{equation}
}
\end{wex}

\begin{wex}{ Factorising using a switch around in brackets }{Factorise $5\left(a-2\right)-b\left(2-a\right)$ }{
\westep{In this case we have to use a ``switch around'' strategy to find the common factor $2-a = -(a-2)$}
\begin{equation}
\begin{array}{ccl}
\hfill 5\left(a-2\right)-b\left(2-a\right)& =& \hfill 5\left(a-2\right)-\left[-b\left(a-2\right)\right]  \\
&= & 5\left(a-2\right)+b\left(a-2\right)\\ 
& =& \left(a-2\right)\left(5+b\right) \hfill
\end{array}
\end{equation}
}
\end{wex}
            
\begin{exercises}{Common factors}

Find the highest common factors of the
following pairs of terms:\par

\begin{multicols}{2}
\begin{enumerate}[label=\textbf{\arabic*}., itemsep=5pt]
\item $6y;18x$
\item $12mn;8n$
\item $3st;4su$ 
\item $18kl;9kp$
\item $abc;ac$% 
\item $2xy;4xyz$
\item $3uv;6u$ 
\item $9xy;15xz$
\item $24xyz;16yz$
\item $3m;45n$
\end{enumerate}
\end{multicols}



\end{exercises}
    \par
        \label{m39383*uid7}
           \subsection* { Difference of two squares}
            \nopagebreak
          \label{m39383*id269179}We have seen that\par 
          \label{m39383*uid8}\nopagebreak\noindent{}
            
    \begin{equation}
    \left(ax+b\right)\left(ax-b\right)~\mbox{ can be expanded to }~{a}^{2}{x}^{2}-{b}^{2}
      \end{equation}

          \label{m39383*id269338}Therefore,\par 
          \label{m39383*id269343}\nopagebreak\noindent{}
            
    \begin{equation}
    {a}^{2}{x}^{2}-{b}^{2}~\mbox{ can be factorised as }~\left(ax+b\right)\left(ax-b\right)
      \end{equation}
          \label{m39383*id269408}For example, ${x}^{2}-16$\hspace{1ex} can be written as $\left({x}^{2}-{4}^{2}\right)$ which is a difference of two squares. Therefore, the factors of ${x}^{2}-16$\hspace{1ex}are $\left(x-4\right)$ and $\left(x+4\right)$.\par 


\Tip{When factorising look for expressions
\begin{itemize}
 \item Consisting of two terms ($a^2-1$)
 \item with terms that have different signs (one positive, one negative) ($4x^2-y^2$)
 \item with each term a perfect square ($p^4-49$)
\end{itemize}
}



\begin{wex}{}
{Factorise completely: $3a\left(a^2-4\right)-7\left(a^2-4\right)$ \par }

{
\westep{$\left(a^2-4\right)$ is the common factor }
     
            
    \begin{equation*}
    \begin{array}{ccc}\hfill 3a\left(a^2-4\right)-7\left(a^2-4\right)& =& \left(a^2-4\right)\left(3a-7\right)\hfill \end{array}
      \end{equation*}

\westep{$(a^2-4)$ is a difference of two squares}

$$
(a^2-4)(3a-7) = (a-2)(a+2)(3a-7)
$$

}
\end{wex}
    

    
    \noindent
\label{m39383*secfhsst!!!underscore!!!id923}
           \begin{exercises}{Factorisation}{
Factorise:
\begin{multicols}{2}
\begin{enumerate}[itemsep=5pt, label=\textbf{\arabic*}. ] 
    \item $2l+2w$\label{m39383*uid12}
    \item $12x+32y$\label{m39383*uid13}
    \item $6{x}^{2}+2x+10{x}^{3}$
    \item $2x{y}^{2}+x{y}^{2}z+3xy$\label{m39383*uid15}
    \item $-2a{b}^{2}-4{a}^{2}b$
    \item $7a+4$ 
    \item $20a-10$ 
    \item $18ab-3bc$% make-rowspan-placeholders
    \item $12kj+18kq$ 
    \item $16{k}^{2}-4$ 
    \item $3{a}^{2}+6a-18$% make-rowspan-placeholders
    \item $-12a+24a^3$ 
    \item $-2ab-8a$ 
    \item $24kj-16{k}^{2}j$% make-rowspan-placeholders
    \item $-{a}^{2}b-{b}^{2}a$ 
    \item $12{k}^{2}j+24{k}^{2}{j}^{2}$ 
    \item $72{b}^{2}q-18{b}^{3}{q}^{2}$% make-rowspan-placeholders
    \item $4\left(y-3\right)+k\left(3-y\right)$ 
    \item $a^2\left(a-1\right)-25\left(a-1\right)$ 
    \item $bm\left(b+4\right)-6m\left(b+4\right)$% make-rowspan-placeholders
    \item ${a}^{2}\left(a+7\right)+9\left(a+7\right)$ 
    \item $3b\left(b-4\right)-7\left(4-b\right)$ 
    \item ${a}^{2}{b}^{2}{c}^{2}-1$% make-rowspan-placeholders
    \end{enumerate}
\end{multicols}
\end{exercises}


    \label{m39394*cid5}
           \subsection* { Factorising a quadratic trinomial }
            \nopagebreak
      \label{m39394*eip-218}
    \setcounter{subfigure}{0}
	\begin{figure}[H] % horizontal\label{m39394*factorisingquadratic}
    \textnormal{Khan Academy video on factorising a quadratic.}\vspace{.1in} \nopagebreak
  \label{m39394*yt-media2}\label{m39394*yt-video2}
            \raisebox{-5 pt}{ \includegraphics[width=0.5cm]{col11306.imgs/summary_www.png}} { (Video:  MG10064 )}
      \vspace{2pt}
    \vspace{.1in}
 \end{figure}       \par \label{m39394*eip-411}Factorisation can be seen as the reverse of calculating the product of factors. In order to factorise a quadratic, we need to find the factors which, when multiplied together, equal the original quadratic.\par 

      \label{m39394*id275057}Let us consider a quadratic that is of the form $a{x}^{2}+bx$\hspace{1ex}. We can see here that $x$ is a common factor of both terms. Therefore,\hspace{1ex}$a{x}^{2}+bx$\hspace{1ex}factorises to $x\left(ax+b\right)$. For example, $8{y}^{2}+4y$\hspace{1ex}factorises to\hspace{1ex}$4y\left(2y+1\right)$.\par 
      \label{m39394*id275188}Another type of quadratic is made up of the difference of squares. We know that:\par 
      \label{m39394*id275192}\nopagebreak\noindent{}
        
    \begin{equation*}
    \left(a+b\right)\left(a-b\right)={a}^{2}-{b}^{2}
      \end{equation*}
So $a^2-b^2$ can be written in factorised form as $(a+b)(a-b)$
   
      \par 
This means that if we ever come across a quadratic that is made up of a difference of squares, we can immediately write down what the factors are.\par 
\label{m39394*secfhsst!!!underscore!!!id2008}\vspace{.5cm} 
      \noindent
%       \hspace*{-30pt}\includegraphics[width=0.5in]{col11306.imgs/pspencil2.png}   \raisebox{25mm}{   


   
    
    \noindent
      \label{m39394*id275654}These types of quadratics are very simple to factorise. However, many quadratics do not fall into these categories and we need a more general method to factorise quadratics.
  \par 
      \label{m39394*id275684}We can learn about how to factorise quadratics by looking at the opposite process where two binomials are multiplied to get a quadratic. For example
        
    \begin{equation*}
    \begin{array}{ccl}\hfill \left(x+2\right)\left(x+3\right)& =& x^2+3x+2x+6 \\ & =& {x}^{2}+5x+6.\hfill \end{array}
      \end{equation*}
      \label{m39394*id275871}We see that the ${x}^{2}$\hspace{1ex}term in the quadratic is the product of the $x$-terms in each bracket. Similarly, the 6 in the quadratic is the product of the 2 and 3 in the brackets. Finally, the middle term is the sum of two terms.\par 
      \label{m39394*id275901}So, how do we use this information to factorise the quadratic?\par 
      \label{m39394*id275905}Let us start with factorising ${x}^{2}+5x+6$ \hspace{1ex}and see if we can decide upon some general rules. Firstly, write down two brackets with an $x$ in each bracket and space for the remaining terms.\par 
      \label{m39394*id275944}\nopagebreak\noindent{}
        
    \begin{equation}
    \left(x\phantom{\rule{2.em}{0ex}}\right)\left(x\phantom{\rule{2.em}{0ex}}\right)
      \end{equation}
      \label{m39394*id275980}Next, decide upon the factors of 6. Since the 6 is positive, possible combinations are:\par 
    % \textbf{m39394*id275986}\par
          \begin{table}[H]
    % \begin{table}[H]
    % \\ '' '0'
        \begin{center}
      \label{m39394*id275986}
    \noindent
    \tabletail{%
%         \hline
%         \multicolumn{2}{|p{\mytableboxwidth}|}{\raggedleft \small \sl continued on next page}\\
%         \hline
      }
      \tablelasttail{}
      \begin{xtabular}[t]{|p{1cm}|p{1cm}|}\hline
    % My position: 0
    % my spanname: 
    % my ct of spanspec: 0
    % my column-count: 2
    \multicolumn{2}{|c|}{Factors of 6}
     \tabularnewline\cline{1-1}\cline{2-2}
      %--------------------------------------------------------------------
        1 &
        6% make-rowspan-placeholders
     \tabularnewline\cline{1-1}\cline{2-2}
      %--------------------------------------------------------------------
        2 &
        3% make-rowspan-placeholders
     \tabularnewline\cline{1-1}\cline{2-2}
      %--------------------------------------------------------------------
        -1 &
        -6% make-rowspan-placeholders
     \tabularnewline\cline{1-1}\cline{2-2}
      %--------------------------------------------------------------------
        -2 &
        -3% make-rowspan-placeholders
     \tabularnewline\cline{1-1}\cline{2-2}
      %--------------------------------------------------------------------
    \end{xtabular}
      \end{center}
%     \begin{center}{\small\bfseries Table 8.6}\end{center}
%     \begin{caption}{Factors of 6}\end{caption}
\end{table}
    \par
      \label{m39394*id276096}Therefore, we have four possibilities:\par 
    % \textbf{m39394*id276099}\par
          \begin{table}[H]
    % \begin{table}[H]
    % \\ '' '0'
        \begin{center}
      \label{m39394*id276099}
    \noindent
    \tabletail{%
%         \hline
%         \multicolumn{4}{|p{\mytableboxwidth}|}{\raggedleft \small \sl continued on next page}\\
%         \hline
      }
      \tablelasttail{}
      \begin{xtabular}[t]{|l|l|l|l|}\hline
        Option 1 &
        Option 2 &
        Option 3 &
        Option 4% make-rowspan-placeholders
     \tabularnewline\cline{1-1}\cline{2-2}\cline{3-3}\cline{4-4}
      %--------------------------------------------------------------------
                $\left(x+1\right)\left(x+6\right)$
               &
                $\left(x-1\right)\left(x-6\right)$
               &
                $\left(x+2\right)\left(x+3\right)$
               &
                $\left(x-2\right)\left(x-3\right)$
              % make-rowspan-placeholders
     \tabularnewline\cline{1-1}\cline{2-2}\cline{3-3}\cline{4-4}
      %--------------------------------------------------------------------
    \end{xtabular}
      \end{center}
%     \begin{center}{\small\bfseries Table 8.7}\end{center}
%     \begin{caption}{Factor Options}\end{caption}
\end{table}
    \par
      \label{m39394*id276261}Next, we expand each set of brackets to see which option gives us the correct middle term.\par 
    % \textbf{m39394*id276265}\par
          \begin{table}[H]
    % \begin{table}[H]
    % \\ '' '0'
        \begin{center}
      \label{m39394*id276265}
    \noindent
    \tabletail{%
%         \hline
%         \multicolumn{4}{|p{\mytableboxwidth}|}{\raggedleft \small \sl continued on next page}\\
%         \hline
      }
      \tablelasttail{}
      \begin{xtabular}[t]{|l|l|l|l|}\hline
        Option 1 &
        Option 2 &
        Option 3 &
        Option 4% make-rowspan-placeholders
     \tabularnewline\cline{1-1}\cline{2-2}\cline{3-3}\cline{4-4}
      %--------------------------------------------------------------------
                $\left(x+1\right)\left(x+6\right)$
               &
                $\left(x-1\right)\left(x-6\right)$
               &
                $\left(x+2\right)\left(x+3\right)$
               &
                $\left(x-2\right)\left(x-3\right)$
              % make-rowspan-placeholders
     \tabularnewline\cline{1-1}\cline{2-2}\cline{3-3}\cline{4-4}
      %--------------------------------------------------------------------
                ${x}^{2}+7x+6$
               &
                ${x}^{2}-7x+6$
               &
                \uline{
                  ${x}^{2}+5x+6$
                }
               &
                ${x}^{2}-5x+6$
              % make-rowspan-placeholders
     \tabularnewline\cline{1-1}\cline{2-2}\cline{3-3}\cline{4-4}
      %--------------------------------------------------------------------
    \end{xtabular}
      \end{center}
%     \begin{center}{\small\bfseries Table 8.8}\end{center}
%     \begin{caption}{Quadratic factors}\end{caption}
\end{table}
    \par
      \label{m39394*id276547}We see that Option 3 $(x+2)(x+3)$ is the correct solution. The process of factorising a quadratic is mostly trial and error but there are some strategies that can be used to simplify the process.\par 
      \label{m39394*uid20}


            \subsection*{Factorising a trinomial}
	    \addcontentsline{toc}{subsection}{Factorising a trinomial}
            \nopagebreak
        \label{m39394*id276561}\begin{enumerate}[itemsep=5pt, label=\textbf{\arabic*}. ] 
            \label{m39394*uid21}\item Divide the entire equation by any common factor of the coefficients so as to obtain an equation of the form $a{x}^{2}+bx+c=0$\hspace{1ex}where $a$, $b$ and $c$ have no common factors and $a$ is positive.
\label{m39394*uid22}\item Write down two brackets with an $x$ in each bracket and space for the remaining terms.
\label{m39394*uid23}\nopagebreak\noindent{}
    \begin{equation}
    \left(x\phantom{\rule{2.em}{0ex}}\right)\left(x\phantom{\rule{2.em}{0ex}}\right)
      \end{equation}
    \label{m39394*uid24}\item Write down a set of factors for $a$ and $c$.
\label{m39394*uid25}\item Write down a set of options for the possible factors for the quadratic using the factors of $a$ and $c$.
\label{m39394*uid26}\item Expand all options to see which one gives you the correct middle term $bx$.
\end{enumerate}
        \label{m39394*id276779}


\Tip{
        \label{m39394*id276789}\begin{itemize}[itemsep=5pt]
            \label{m39394*uid27}\item If $c$ is positive, then the factors of $c$ must be either both positive or both negative. If $c$ is negative, it means only one of the factors of $c$ is negative, the other one being positive.
\label{m39394*uid28}\item Once you get an answer, always multiply out your brackets again just to make sure it really works.
\end{itemize}

}
\par
%             \label{m39394*secfhsst!!!underscore!!!id2510}\vspace{.5cm} 
%       \noindent
%       \hspace*{-30pt}\includegraphics[width=0.5in]{col11306.imgs/pspencil2.png}   \raisebox{25mm}{   
 
\begin{wex}
{ 
Factorising a quadratic trinomial 
}
{
Factorise $3{x}^{2}+2x-1$. 
} 
{
\westep{Check that the quadratic is in the required form ($ax^2+bx+c$)}
\westep{Write down a set of factors for $a$ and $c$.}  
\begin{equation}
\left(x\phantom{\rule{2.em}{0ex}}\right)\left(x\phantom{\rule{2.em}{0ex}}\right)
\end{equation}

The possible factors for $a$ are: (1,3).\par
The possible factors for $c$ are: (-1,1) or (1,-1).\par 
\label{m39394*id277075}Write down a set of options for the possible factors of the quadratic using the factors of $a$ and $c$.
Therefore, there are two possible options.\par 
% \textbf{m39394*id277097}\par
\begin{table}[H]
% \begin{table}[H]
% \\ 'id2892888' '1'
\begin{center}
\label{m39394*id277097}
\noindent
\tabletail{%
\hline
\multicolumn{2}{|p{\mytableboxwidth}|}{\raggedleft \small \sl continued on next page}\\
\hline
}
\tablelasttail{}
\begin{xtabular}[t]{|l|l|}\hline
Option 1 &
Option 2% make-rowspan-placeholders
\tabularnewline\cline{1-1}\cline{2-2}
%--------------------------------------------------------------------
$\left(x-1\right)\left(3x+1\right)$
&
$\left(x+1\right)\left(3x-1\right)$
% make-rowspan-placeholders
\tabularnewline\cline{1-1}\cline{2-2}
%--------------------------------------------------------------------
$3{x}^{2}-2x-1$
&
\uline{
$3{x}^{2}+2x-1$
}
% make-rowspan-placeholders
\tabularnewline\cline{1-1}\cline{2-2}
%--------------------------------------------------------------------
\end{xtabular}
\end{center}
%     \begin{center}{\small\bfseries Table 8.9}\end{center}
%     \begin{caption}{Possible factors}\end{caption}
\end{table}

\westep{Test whether your solution is correct by multiplying the factors} 
\label{m39394*id277257}\nopagebreak\noindent{}

\begin{equation}
\begin{array}{ccl}  
\left(x+1\right)\left(3x-1\right)& =& 3{x}^{2}-x+3x-1\hfill \\ & =& {x}^{2}+2x-1.\end{array}
\end{equation}
\westep{}
\label{m39394*id277481}The factors of $3{x}^{2}+2x-1$\hspace{1ex}are $\left(x+1\right)$ and $\left(3x-1\right)$.

}
\end{wex}

    
    \noindent
\label{m39394*secfhsst!!!underscore!!!id2756}
\begin{exercises}{}
 {
Factorise the following:
\begin{multicols}{2}
\begin{enumerate}[itemsep=5pt, label=\textbf{\arabic*}. ] 
\item ${x}^{2}+8x+15$
\item ${x}^{2}+10x+24$
\item ${x}^{2}+9x+8$
\item ${x}^{2}+9x+14$
\item ${x}^{2}+15x+36$
\item ${x}^{2}+12x+36$
\end{enumerate}
\end{multicols}


Write the following expressions in factorised form:
\begin{multicols}{2}
\begin{enumerate}[itemsep=5pt, label=\textbf{\arabic*}. ] 
\setcounter{enumi}{6}
\item ${x}^{2}-2x-15$
\item ${x}^{2}+2x-3$
\item ${x}^{2}+2x-8$
\item ${x}^{2}+x-20$
\item ${x}^{2}-x-20$
\end{enumerate}
\end{multicols}


Find the factors for the following trinomial expressions:
\begin{multicols}{2}
\begin{enumerate}[itemsep=5pt, label=\textbf{\arabic*}. ] 
\setcounter{enumi}{11}
\item $4{x}^{2}+22x+10$
\item $3{x}^{2}+19x+6$
\item $6{x}^{2}+7x+2$
\item $12{x}^{2}+8x+1$
\item $8{x}^{2}+6x+1$
\end{enumerate}
\end{multicols}

Factorise completely:
\begin{multicols}{2}
\begin{enumerate}[itemsep=5pt, label=\textbf{\arabic*}. ] 
\setcounter{enumi}{16}
\item $3{x}^{2}+17x-6$
\item $7{x}^{2}-6x-1$
\item $8{x}^{2}-6x+1$
\item $6{x}^{2}-15x-9$
\end{enumerate}
\end{multicols}


% 
%     \label{m39394*cid6}
% \par \raisebox{-5 pt}{\includegraphics[width=0.5cm]{col11306.imgs/summary_www.png}} Find the answers with the shortcodes:
%  \par \begin{tabular}[h]{cccccc}
%  (1.) liY  &  (2.) lir  &  (3.) li1  &  (4.) liC  & \end{tabular}

\end{exercises}


            \subsection{ Factorisation by Grouping}
            \nopagebreak
      \label{m39394*id278358}One other method of factorisation involves the use of common factors. We know that the factors of $3x+3$\hspace{1ex} are 3 and $\left(x+1\right)$. Similarly, the factors of $2{x}^{2}+2x$\hspace{1ex}are $2x$\hspace{1ex}and $\left(x+1\right)$. Therefore, if we have an expression:\par 
      \label{m39394*id278452}\nopagebreak\noindent{}
        
    \begin{equation}
    2{x}^{2}+2x+3x+3
      \end{equation}
      \label{m39394*id278488}then we can factorise as:\par 
      \label{m39394*id278494}\nopagebreak\noindent{}
        
    \begin{equation}
    2x\left(x+1\right)+3\left(x+1\right).
      \end{equation}
      \label{m39394*id278536}You can see that there is another common factor: $x+1$. Therefore, we can now write:\par 
      \label{m39394*id278556}\nopagebreak\noindent{}
        
    \begin{equation}
    \left(x+1\right)\left(2x+3\right).
      \end{equation}
      \label{m39394*id278591}We get this by taking out the $x+1$ and seeing what is left over. We have a $+2x$\hspace{1ex}from the first term and a $+3$ from the second term. This is called \textsl{factorisation by grouping}.\par 
\label{m39394*secfhsst!!!underscore!!!id2835}\vspace{.5cm} 
      \noindent
%       \hspace*{-30pt}\includegraphics[width=0.5in]{col11306.imgs/pspencil2.png}   \raisebox{25mm}{   
%      linewidth=4, leftmargin=40, rightmargin=40]  
     

 \begin{wex}{Factorisation by Grouping }{Find the factors of $7x+14y+bx+2by$ by grouping}
{
   
     \westep{} \label{m39394*id278713}There are no factors that are common to all terms.\par 
        
\westep{}      \label{m39394*id278721}7 is a common factor of the first two terms and $b$ is a common factor of the second two terms.\par 
   \westep{}    
      \label{m39394*id278739}\nopagebreak\noindent{}
        
    \begin{equation}
    7x+14y+bx+2by=7\left(x+2y\right)+b\left(x+2y\right)      \end{equation}
      \westep{}
      \label{m39394*id278811}$x+2y$\hspace{1ex}is a common factor.\par 
  \westep{}  
      \label{m39394*id278835}\nopagebreak\noindent{}
        
    \begin{equation}
    7\left(x+2y\right)+b\left(x+2y\right)=\left(x+2y\right)\left(7+b\right)
      \end{equation}
      \westep{}  
      \label{m39394*id278906}The factors of $7x+14y+bx+2by$\hspace{1ex}are $\left(7+b\right)$ and $\left(x+2y\right)$.
 \par 
%       \end{enumerate}
}
    \end{wex}

    }
    \noindent
\label{m39394*eip-280}
    \setcounter{subfigure}{0}
	\begin{figure}[H] % horizontal\label{m39394*factorisingtrinomial}
    \textnormal{Khan Academy video on factorising a trinomial by grouping.}\vspace{.1in} \nopagebreak
  \label{m39394*yt-media32}\label{m39394*yt-video32}
            \raisebox{-5 pt}{ \includegraphics[width=0.5cm]{col11306.imgs/summary_www.png}} { (Video:  MG10065 )}
      \vspace{2pt}
    \vspace{.1in}
 \end{figure}       \par \label{m39394*secfhsst!!!underscore!!!id2920}
\begin{exercises}{Factorisation by Grouping }{
            \nopagebreak
      \label{m39394*id279000}\begin{enumerate}[itemsep=5pt, label=\textbf{\arabic*}. ] 
            \label{m39394*uid48}\item Factorise by grouping: $6x+a+2ax+3$
\newline
        \label{m39394*uid49}\item Factorise by grouping: ${x}^{2}-6x+5x-30$\newline
        \label{m39394*uid50}\item Factorise by grouping: $5x+10y-ax-2ay$\newline
        \label{m39394*uid51}\item Factorise by grouping: ${a}^{2}-2a-ax+2x$\newline
        \label{m39394*uid52}\item Factorise by grouping: $5xy-3y+10x-6$\newline
        \end{enumerate}
  \label{m39394**end}
\par \raisebox{-5 pt}{\includegraphics[width=0.5cm]{col11306.imgs/summary_www.png}} Find the answers with the shortcodes:
 \par \begin{tabular}[h]{cccccc}
 (1.) lih  &  (2.) liS  &  (3.) liJ  &  (4.) liu  &  (5.) liz  & \end{tabular}
}
\end{exercises}
    
    \noindent
\nopagebreak 
\label{m39387*secfhsst!!!underscore!!!id1562}\vspace{.5cm} 
\subsection*{Sum and Difference of Two Cubes}      
%       \noindent
%       \hspace*{-30pt}\includegraphics[width=0.5in]{col11306.imgs/pspencil2.png}   \raisebox{25mm}{   

      \begin{wex}{Sum of Cubes }{Find the product of $x+y$\hspace{1ex} and ${x}^{2}-xy+{y}^{2}$.}{
\westep{}
      \label{m39387*id272709}We are given two expressions: a binomial, $x+y$, and a trinomial, ${x}^{2}-xy+{y}^{2}$. \hspace{1ex}We need to multiply them together.\par 
      \westep{}  
      \label{m39387*id272764}Apply the distributive law and then simplify the resulting expression.\par 
      \westep{}
      \label{m39387*id272771}\nopagebreak\noindent{}
    \begin{equation}
    \begin{array}{cccc}& \phantom{\rule{4pt}{0ex}}& \left(x+y\right)\left({x}^{2}-xy+{y}^{2}\right)\hfill & \\ & =& x\left({x}^{2}-xy+{y}^{2}\right)+y\left({x}^{2}-xy+{y}^{2}\right)\hfill & \left(\mathrm{apply\; distributive\; law}\right)\hfill \\ & =& \left[x\left({x}^{2}\right)+x\left(-xy\right)+x\left({y}^{2}\right)\right]+\left[y\left({x}^{2}\right)+y\left(-xy\right)+y\left({y}^{2}\right)\right]\hfill & \\ & =& {x}^{3}-{x}^{2}y+x{y}^{2}+y{x}^{2}-x{y}^{2}+{y}^{3}\hfill & \left(\mathrm{expand\; the\; brackets}\right)\hfill & \\ & =& {x}^{3}+\left(-{x}^{2}y+y{x}^{2}\right)+\left(x{y}^{2}-x{y}^{2}\right)+{y}^{3}\hfill & \left(\mathrm{group\; like\; terms\; to\; simplify}\right)\hfill & \\ & =& {x}^{3}+{y}^{3}\hfill & \left(\mathrm{simplify\; to\; get\; final\; answer}\right)\hfill & \end{array}
      \end{equation}
      \westep{}
      \label{m39387*id273290}The product of $x+y$\hspace{1ex} and ${x}^{2}-xy+{y}^{2}$\hspace{1ex} is ${x}^{3}+{y}^{3}$. \par 
      }
    \end{wex}
   
    
    \noindent
\label{m39387*notfhsst!!!underscore!!!id1885}
% \begin{tabular}{cc}
% 	   \hspace*{-50pt}\raisebox{-8 mm}{ \includegraphics[width=0.5in]{col11306.imgs/pstip2.png}  }& 
% 	\begin{minipage}{0.85\textwidth}
% 	\begin{note}
      \Tip{We have seen that:

        
    \begin{equation}
    \left(x+y\right)\left({x}^{2}-xy+{y}^{2}\right)={x}^{3}+{y}^{3}
      \end{equation}
      \label{m39387*id273451}This is known as a \textsl{sum of cubes}.} 

	\par
\label{m39387*secfhsst!!!underscore!!!id1926}
\begin{Investigation}{Difference of Cubes }
            \nopagebreak
      \label{m39387*id273469}Show that the difference of cubes
(${x}^{3}-{y}^{3}$\hspace{1ex}) is given by the product of $x-y$\hspace{1ex} and ${x}^{2}+xy+{y}^{2}$. \par 
\end{Investigation}


         \section{ Simplification of Fractions}
    \nopagebreak
            \label{m39392} $ \hspace{-5pt}\begin{array}{cccccccccccc}   \includegraphics[width=0.75cm]{col11306.imgs/summary_fullmarks.png} &   \end{array} $ \hspace{2 pt}\raisebox{-5 pt}{} {(subsection shortcode: MG10066 )} \par 
%   
%     \label{m39392*cid7}
%             \subsection{ Simplification of Fractions}
%             \nopagebreak
      \label{m39392*id279238}In some cases of simplifying an algebraic expression, the expression will be a fraction. For example,\par 
      \label{m39392*id279242}\nopagebreak\noindent{}
        
    \begin{equation}
    \dfrac{{x}^{2}+3x}{x+3}
      \end{equation}
      \label{m39392*id279276}has a quadratic in the numerator and a binomial in the denominator. You can apply the different factorisation methods to simplify the expression.\par 
      \label{m39392*id279282}\nopagebreak\noindent{}
    \begin{equation}
    \begin{array}{cccc}& \phantom{\rule{4pt}{0ex}}& \dfrac{{x}^{2}+3x}{x+3}\hfill & \\ & =& \dfrac{x\left(x+3\right)}{x+3}\hfill & \\ & =& x\hfill & \mathrm{provided}\phantom{\rule{2pt}{0ex}}x\ne -3\hfill \end{array}
      \end{equation}
      \label{m39392*id279389}If $x$ were 3 then the denominator, $x-3$, would be 0 and the fraction undefined.\par 
% \label{m39392*secfhsst!!!underscore!!!id3026}\vspace{.5cm} 
%       \noindent
%       \hspace*{-30pt}\includegraphics[width=0.5in]{col11306.imgs/pspencil2.png}   \raisebox{25mm}{   
  
      \begin{wex}{ Simplification of Fractions }{Simplify: $\dfrac{2x-b+x-ab}{a{x}^{2}-abx}$}{
       

\westep{}Use \textsl{grouping} for numerator and \textsl{common factor} for denominator in this example.\par 

        
    \begin{equation}
    \begin{array}{ccc}& =& \dfrac{\left(ax-ab\right)+\left(x-b\right)}{a{x}^{2}-abx}\hfill \\ & =& \dfrac{a\left(x-b\right)+\left(x-b\right)}{ax\left(x-b\right)}\hfill \\ & =& \dfrac{\left(x-b\right)\left(a+1\right)}{ax\left(x-b\right)}\hfill \end{array}
      \end{equation}
    \westep{}  
      \label{m39392*id279695}The simplified answer is:\par 
      \label{m39392*id279699}\nopagebreak\noindent{}
        
    \begin{equation}
    \begin{array}{ccc}& =& \dfrac{a+1}{ax}\hfill \end{array}
      \end{equation}
  }
    \end{wex}


      \begin{wex}{ Simplification of Fractions }
  {Simplify:$\dfrac{{x}^{2}-x-2}{{x}^{2}-4}÷\dfrac{{x}^{2}+x}{{x}^{2}+2x}$} {
     \westep{} 
    \begin{equation}
    \begin{array}{ccc}& =& \dfrac{\left(x+1\right)\left(x-2\right)}{\left(x+2\right)\left(x-2\right)}÷\dfrac{x\left(x+1\right)}{x\left(x+2\right)}\hfill \end{array}
      \end{equation}
    \westep{} 

        
    \begin{equation}
    \begin{array}{ccc}& =& \dfrac{\left(x+1\right)\left(x-2\right)}{\left(x+2\right)\left(x-2\right)}\ensuremath{\times}\dfrac{x\left(x+2\right)}{x\left(x+1\right)}\hfill \end{array}
      \end{equation}
   \westep{}
      \label{m39392*id280081}The simplified answer is
     
    \begin{equation}
    \begin{array}{ccc}& =& 1\hfill \end{array}
      \end{equation}
   }
    \end{wex}
 
    
%     \noindent
%  \noindent
%       \hspace*{-30pt}\includegraphics[width=0.5in]{col11306.imgs/pspencil2.png}   \raisebox{25mm}{   
%      
      \begin{wex}{Simplification of Fractions}{Simplify the following expression: $\dfrac{x-2}{{x}^{2}-4}+\dfrac{{x}^{2}}{x-2}-\dfrac{{x}^{3}+x-4}{{x}^{2}-4}$}
     {
\westep{}
    \begin{equation}
    \frac{x-2}{\left(x+2\right)\left(x-2\right)}+\frac{{x}^{2}}{x-2}-\frac{{x}^{3}+x-4}{\left(x+2\right)\left(x-2\right)}
      \end{equation}
     \westep{} \label{m39392*id79242}We make all the denominators the same so that we can add or subtract the fractions. The lowest common denominator is $\left(x-2\right)\left(x+2\right)$.\par 
      \label{m39392*id27943653977}\nopagebreak\noindent{}
    \begin{equation}
    \frac{x-2}{\left(x+2\right)\left(x-2\right)}+\frac{\left({x}^{2}\right)
\left(x+2\right)}{\left(x+2\right)\left(x-2\right)}-\frac{{x}^{3}+x-4}{\left(x+2\right)\left(x-2\right)}
      \end{equation}
      \westep  \label{m39392*id639247}Since the fractions all have the same denominator we can write them all as one fraction with the appropriate operator\par 
     \label{m39392*id653977}\nopagebreak\noindent{}
    \begin{equation}
    \frac{x-2+\left({x}^{2}\right
\left(x+2\right)-{x}^{3}+x-4}{\left(x+2\right)\left(x-2\right)}
      \end{equation}
     \westep{} \label{m39392*id9657}\nopagebreak\noindent{}
    \begin{equation}
    \frac{2{x}^{2}
+2x-6}{\left(x+2\right)\left(x-2\right)}
      \end{equation}
\westep{}
\label{m39392*id65677}\nopagebreak\noindent{}
    \begin{equation}
    \frac{2\left({x}^{2}
+x-3\right)}{\left(x+2\right)\left(x-2\right)}
      \end{equation}
}
    \end{wex}
   
    
\begin{exercises}{ Simplification of Fractions }
 {           \nopagebreak
      \label{m39392*id280133}\begin{enumerate}[itemsep=5pt, label=\textbf{\arabic*}. ] 
            \label{m39392*uid53}\item Simplify:
    % \textbf{m39392*id280148}\par
          \begin{table}[H]
    % \begin{table}[H]
    % \\ 'id2896760' '1'
        \begin{center}
      \label{m39392*id280148}
    \noindent
    \tabletail{%
        \hline
%         \multicolumn{2}{|p{\mytableboxwidth}|}{\raggedleft \small \sl continued on next page}\\
        \hline
      }
      \tablelasttail{}
      \begin{xtabular}[t]{|l|l|}\hline
        (a) $\dfrac{3a}{15}$\hspace{1ex} &
        (b) $\dfrac{2a+10}{4}$\hspace{1ex}% make-rowspan-placeholders
     \tabularnewline\cline{1-1}\cline{2-2}
      %--------------------------------------------------------------------
        (c) $\dfrac{5a+20}{a+4}$\hspace{1ex} &
        (d) $\dfrac{{a}^{2}-4a}{a-4}$\hspace{1ex}% make-rowspan-placeholders
     \tabularnewline\cline{1-1}\cline{2-2}
      %--------------------------------------------------------------------
        (e) $\dfrac{3{a}^{2}-9a}{2a-6}$\hspace{1ex} &
        (f) $\dfrac{9a+27}{9a+18}$\hspace{1ex}% make-rowspan-placeholders
     \tabularnewline\cline{1-1}\cline{2-2}
      %--------------------------------------------------------------------
        (g) $\dfrac{6ab+2a}{2b}$\hspace{1ex} &
        (h) $\dfrac{16{x}^{2}y-8xy}{12x-6}$\hspace{1ex}% make-rowspan-placeholders
     \tabularnewline\cline{1-1}\cline{2-2}
      %--------------------------------------------------------------------
        (i) $\dfrac{4xyp-8xp}{12xy}$\hspace{1ex} &
        (j) $\dfrac{3a+9}{14}÷\dfrac{7a+21}{a+3}$\hspace{1ex}% make-rowspan-placeholders
     \tabularnewline\cline{1-1}\cline{2-2}
      %--------------------------------------------------------------------
        (k) $\dfrac{{a}^{2}-5a}{2a+10}÷\dfrac{3a+15}{4a}$\hspace{1ex} &
        (l) $\dfrac{3xp+4p}{8p}÷\dfrac{12{p}^{2}}{3x+4}$\hspace{1ex}% make-rowspan-placeholders
     \tabularnewline\cline{1-1}\cline{2-2}
      %--------------------------------------------------------------------
        (m) $\dfrac{16}{2xp+4x}÷\dfrac{6{x}^{2}+8x}{12}$\hspace{1ex} &
        (n) $\dfrac{24a-8}{12}÷\dfrac{9a-3}{6}$\hspace{1ex}% make-rowspan-placeholders
     \tabularnewline\cline{1-1}\cline{2-2}
      %--------------------------------------------------------------------
        (o) $\dfrac{{a}^{2}+2a}{5}÷\dfrac{2a+4}{20}$\hspace{1ex} &
        (p) $\dfrac{{p}^{2}+pq}{7p}÷\dfrac{8p+8q}{21q}$\hspace{1ex}% make-rowspan-placeholders
     \tabularnewline\cline{1-1}\cline{2-2}
      %--------------------------------------------------------------------
        (q) $\dfrac{5ab-15b}{4a-12}÷\dfrac{6{b}^{2}}{a+b}$\hspace{1ex} &
        (r) $\dfrac{{f}^{2}a-f{a}^{2}}{f-a}$% make-rowspan-placeholders
     \tabularnewline\cline{1-1}\cline{2-2}
      %--------------------------------------------------------------------
    \end{xtabular}
      \end{center}
%     \begin{center}{\small\bfseries Table 8.11}\end{center}
%     \begin{caption}{\small\bfseries Table 8.11}\end{caption}
\end{table}
    \par
          \label{m39392*uid54}\item Simplify: $\dfrac{{x}^{2}-1}{3}\ensuremath{\times}\dfrac{1}{x-1}-\dfrac{1}{2}$
\newline
\newline
\end{enumerate}
}
\end{exercises}

    \label{m39392*eip-770}
\par \raisebox{-5 pt}{\includegraphics[width=0.5cm]{col11306.imgs/summary_www.png}} Find the answers with the shortcodes:
 \par \begin{tabular}[h]{cccccc}
 (1.) lit  &  (2.) lie  & \end{tabular}
%             \subsubsection{ Adding and subtracting fractions}
%             \nopagebreak
%             \label{m39392*eip-107}Using the concepts learnt in simplification of fractions, we can now add and subtract simple fractions. To add or subtract fractions we note that we can only add or subtract fractions that have the same denominator. So we must first make all the denominators the same and then perform the addition or subtraction. This is called finding the lowest common denominator or multiple. 
% \par \label{m39392*eip-997}For example, if you wanted to add: $\frac{1}{2}$ and $\frac{3}{5}$ we would note that the lowest common denominator is 10. So we must multiply the first fraction by 5 and the second fraction by 2 to get both of these with the same denominator. Doing so gives: $\frac{5}{10}$ and $\frac{6}{10}$. Now we can add the fractions. Doing so, we get $\frac{11}{10}$.\par \label{m39392*eip-337}\vspace{.5cm} 
%       
%     \noindent
%   \label{m39392*eip-755}
%             \subsubsection{ Two interesting mathematical proofs}
%             \nopagebreak
%             \label{m39392*id76534}We can use the concepts learnt in this chapter to demonstrate two interesting mathematical proofs. The first proof states that ${n}^{2}+n$ is even for all $n\in {Z}$. The second proof states that ${n}^{3}-n$ is divisible by 6 for all $n\in {Z}$. Before we demonstrate that these two laws are true, we first need to note some other mathematical rules.
% \par 
% \label{m39392*id736}If we multiply an even number by an odd number, we get an even number. Similarly if we multiply an odd number by an even number we get an even number. Also, an even number multiplied by an even number is even and an odd number multiplied by an odd number is odd. This result is shown in the following table:
% 
% \par 
% 
%     % \textbf{m39392*eip-556}\par
%     
%           \begin{table}[H]
%         
%     % \begin{table}[H]
%     % \\ '' '0'
%     
%         \begin{center}
%       
%       \label{m39392*eip-556}
%       
%     \noindent
%     \tabletail{%
%         \hline
%         \multicolumn{3}{|p{\mytableboxwidth}|}{\raggedleft \small \sl continued on next page}\\
%         \hline
%       }
%       \tablelasttail{}
%       \begin{xtabular}[t]{|l|l|l|}\hline
%     
%     
%          &
%     
%     
%         \textbf{Odd number} &
%     
%     
%         \textbf{Even number}% make-rowspan-placeholders
%      \tabularnewline\cline{1-1}\cline{2-2}\cline{3-3}
%       %--------------------------------------------------------------------
%     
%     
%         \textbf{Odd number} &
%     
%     
%         Odd &
%     
%     
%         Even% make-rowspan-placeholders
%      \tabularnewline\cline{1-1}\cline{2-2}\cline{3-3}
%       %--------------------------------------------------------------------
%     
%     
%         \textbf{Even number} &
%     
%     
%         Even &
%     
%     
%         Even% make-rowspan-placeholders
%      \tabularnewline\cline{1-1}\cline{2-2}\cline{3-3}
%       %--------------------------------------------------------------------
%     \end{xtabular}
%       \end{center}
%     \begin{center}{\small\bfseries Table 8.12}\end{center}
%     \begin{caption}{\small\bfseries Table 8.12}\end{caption}
\end{table}
%       
%     \par
%   \label{m39392*id697}If we take three consecutive numbers and multiply them together, the resulting number is always divisible by three. This should be obvious since if we have any three consecutive numbers, one of them will be divisible by 3.
% \par 
% \label{m39392*id67354}Now we are ready to demonstrate that ${n}^{2}+n$ is even for all $n\in {Z}$. If we factorise this expression we get:
% $n\left(n+1\right)$. If $n$ is even, than $n+1$ is odd. If $n$ is odd, than $n+1$ is even. Since we know that if we multiply an even number with an odd number or an odd number with an even number, we get an even number, we have demonstrated that ${n}^{2}+n$ is always even. Try this for a few values of $n$ and you should find that this is true.
% \par 
% \label{m39392*id67744}To demonstrate that ${n}^{3}-n$ is divisible by 6 for all $n\in {Z}$, we first note that the factors of 6 are 3 and 2. So if we show that ${n}^{3}-n$ is divisible by both 3 and 2, then we have shown that it is also divisible by 6! If we factorise this expression we get:
% $n\left(n+1\right)\left(n-1\right)$. Now we note that we are multiplying three consecutive numbers together (we are taking $n$ and then adding 1 or subtracting 1. This gives us the two numbers on either side of $n$.) For example, if $n=4$, then $n+1=5$ and $n-1=3$. But we know that when we multiply three consecutive numbers together, the resulting number is always divisible by 3. So we have demonstrated that ${n}^{3}-n$ is always divisible by 3. To demonstrate that it is also divisible by 2, we can also show that it is even. We have shown that ${n}^{2}+n$ is always even. So now we recall what we said about multiplying even and odd numbers. Since one number is always even and the other can be either even or odd, the result of multiplying these numbers together is always even. And so we have demonstrated that ${n}^{3}-n$ is divisible by 6 for all $n\in {Z}$.
% \par 
% \
% 
% 
\section{Summary}
\subsection*{Rational Numbers}
             \nopagebreak
             \label{m38348*cid8} $ \hspace{-5pt}\begin{array}{cccccccccccc}   \end{array} $ \hspace{2 pt}\raisebox{-5 pt}{\includegraphics[width=0.5cm]{col11306.imgs/summary_www.png}} {(subsection shortcode: MG10040 )} \par \label{m38348*eip-280}\begin{itemize}[itemsep=5pt, label=\textbullet{}]
             \item Real numbers can be either rational or irrational.\item  A rational number is any number which can be written as 
 $\dfrac{a}{b}$
where $a$ and $b$ are integers and $b\ne 0$\item The following are rational numbers:
       \label{m38348*id64890}\begin{enumerate}[itemsep=5pt, label=\textbf{\alph*}. ] 
             \label{m38348*uid37}\item Fractions with both denominator and numerator as integers.
 \label{m38348*uid38}\item Integers.
 \label{m38348*uid39}\item Decimal numbers that end.
 \label{m38348*uid40}\item Decimal numbers that repeat.
 \end{enumerate}
         \end{itemize}
  \subsection* {Irrational Numbers and Rounding Off}  
            \nopagebreak
            \label{m38349*eip-361} $ \hspace{-5pt}\begin{array}{cccccccccccc}   \end{array} $ \hspace{2 pt}\raisebox{-5 pt}{\includegraphics[width=0.5cm]{col11306.imgs/summary_www.png}} {(subsection shortcode: MG10058 )} \par \label{m38349*uid0821}\begin{itemize}[itemsep=5pt]
            \item Irrational numbers are numbers that cannot be written as a fraction with the numerator and denominator as integers.\item For convenience irrational numbers are often rounded off to a specified number of decimal places\end{itemize}
  \subsection* {Estimating Surds}
%     \subsection{ Summary}
            \nopagebreak
            \label{m38347*eip-194} $ \hspace{-5pt}\begin{array}{cccccccccccc}   \end{array} $ \hspace{2 pt}\raisebox{-5 pt}{\includegraphics[width=0.5cm]{col11306.imgs/summary_www.png}} {(subsection shortcode: MG10053 )} \par \label{m38347*eip-50}\begin{itemize}[itemsep=5pt]
            \item If the ${n}^{\mathrm{th}}$ root of a number cannot be simplified to a rational number, we call it a $\mathit{surd}$\item If $a$ and $b$ are positive whole numbers, and $a\lessthan{}b$, then $\sqrt[n]{a}\lessthan{}\sqrt[n]{b}$\item Surds can be estimated by finding the largest perfect square (or perfect cube) that is less than the surd and the smallest perfect square (or perfect cube) that is greater than the surd. The surd lies between these two numbers.\end{itemize}
  \subsection* {Products and Factorisation}
\label{m39392*eip-735}
%             \subsubsection{ Summary}
            \nopagebreak
            \label{m39392*uid0812}\begin{itemize}[itemsep=5pt]
            \item A binomial is a mathematical expression with two terms. The product of two identical binomials is known as the square of the binomial. The difference of two squares is when we multiply
                $\left(ax+b\right)\left(ax-b\right)$\item Factorising is the opposite of expanding the brackets. You can use common factors or the difference of two squares to help you factorise expressions.\item The distributive law ($\left(A+B\right)\left(C+D+E\right)=A\left(C+D+E\right)+B\left(C+D+E\right)$) helps us to multiply a binomial and a trinomial.\item The sum of cubes is: $\left(x+y\right)\left({x}^{2}-xy+{y}^{2}\right)={x}^{3}+{y}^{3}$ and the difference of cubes is: ${x}^{3}-{y}^{3}=\left(x-y\right)\left({x}^{2}+xy+{y}^{2}\right)$\item To factorise a quadratic we find the two binomials that were multiplied together to give the quadratic.\item We can also factorise a quadratic by grouping. This is where we find a common factor in the quadratic and take it out and then see what is left over.\item We can simplify fractions by using the methods we have learnt to factorise expressions.\item Fractions can be added or subtracted. To do this the denominators of each fraction must be the same.\end{itemize}
        \label{m39392*cid8}

\begin{eocexercises}{End of Chapter Exercises}
\subsubsection*{Rational Numbers}
% ----------------------------------------------------------------------------------------------
% RATIONAL NUMBERS
%         
%     
%     \subsection{ End of Section Exercises}  

            \nopagebreak
            \label{m38348*cid9} $ \hspace{-5pt}\begin{array}{cccccccccccc}   \end{array} $ \hspace{2 pt}\raisebox{-5 pt}{\includegraphics[width=0.5cm]{col11306.imgs/summary_www.png}} {(subsection shortcode: MG10041 )} \par \label{m38348*id64954}\begin{enumerate}[itemsep=5pt, label=\textbf{\arabic*}. ] 
            \label{m38348*uid41}\item If $a$ is an integer, $b$ is an integer and $c$ is irrational, which of the following are rational numbers?
\label{m38348*id64997}\begin{enumerate}[itemsep=5pt, label=\textbf{\alph*}. ] 
            \label{m38348*uid42}\item $\dfrac{5}{6}$\label{m38348*uid43}\item $\dfrac{a}{3}$\label{m38348*uid44}\item $\dfrac{b}{2}$\label{m38348*uid45}\item $\dfrac{1}{c}$
\end{enumerate}
\label{m38348*uid46}\item Write each decimal as a simple fraction:
\label{m38348*id65104}\begin{enumerate}[itemsep=5pt, label=\textbf{\alph*}. ] 
            \label{m38348*uid47}\item $0,5$\label{m38348*uid48}\item $0,12$\label{m38348*uid49}\item $0,6$\label{m38348*uid50}\item $1,59$\label{m38348*uid51}\item $12,27\dot{7}$
\end{enumerate}
\label{m38348*uid52}\item Show that the decimal $3,21\dot{1}\dot{8}$ is a rational number.
\newline
\label{m38348*uid53}\item Express $0,7\dot{8}$ as a fraction $\dfrac{a}{b}$ where $a,b\in \mathbb{Z}$ (show all working).
\newline
\end{enumerate}
  \label{m38348**end}
\par \raisebox{-5 pt}{\includegraphics[width=0.5cm]{col11306.imgs/summary_www.png}} Find the answers with the shortcodes:
 \par \begin{tabular}[h]{cccccc}
 (1.) l3v  &  (2.) l3f  &  (3.) l3G  &  (4.) lOf  & \end{tabular}
% -----------------------------------------------------------------------------------
\subsubsection*{Irrational Numbers and Rounding Off}
% IRRATIONAL NUMBERS & ROUNDING OFF
%        \subsection{End of Section Exercises}
          \nopagebreak
            \label{m38349*cid5} $ \hspace{-5pt}\begin{array}{cccccccccccc}   \end{array} $ \hspace{2 pt}\raisebox{-5 pt}{\includegraphics[width=0.5cm]{col11306.imgs/summary_www.png}} {(subsection shortcode: MG10059 )} \par \label{m38349*id325742}\begin{enumerate}[itemsep=5pt, label=\textbf{\arabic*}. ] 
            \label{m38349*uid17}\item Write the following rational numbers to 2 decimal places:
\label{m38349*id325757}\begin{enumerate}[itemsep=5pt, label=\textbf{\alph*}. ] 
            \label{m38349*uid18}\item $\dfrac{1}{2}$\label{m38349*uid19}\item $1$
\label{m38349*uid20}\item $0,11111\overline{1}$\label{m38349*uid21}\item $0,99999\overline{1}$\end{enumerate}
        \label{m38349*uid22}\item Write the following irrational numbers to 2 decimal places:
\label{m38349*id325863}\begin{enumerate}[itemsep=5pt, label=\textbf{\alph*}. ] 
            \label{m38349*uid23}\item $3,141592654...$\label{m38349*uid24}\item $1,618\phantom{\rule{0.166667em}{0ex}}033\phantom{\rule{0.166667em}{0ex}}989\phantom{\rule{0.166667em}{0ex}}...$\label{m38349*uid25}\item $1,41421356...$\label{m38349*uid26}\item $2,71828182845904523536...$\end{enumerate}
        \label{m38349*uid27}\item Use your calculator and write the following irrational numbers to 3 decimal places:
\label{m38349*id325991}\begin{enumerate}[itemsep=5pt, label=\textbf{\alph*}. ] 
            \label{m38349*uid28}\item $\sqrt{2}$\label{m38349*uid29}\item $\sqrt{3}$\label{m38349*uid30}\item $\sqrt{5}$\label{m38349*uid31}\item $\sqrt{6}$\end{enumerate}
        \label{m38349*uid32}\item Use your calculator (where necessary) and write the following numbers to 5 decimal places. State whether the numbers are irrational or rational.
\label{m38349*id326080}\begin{enumerate}[itemsep=5pt, label=\textbf{\alph*}. ] 
            \label{m38349*uid33}\item $\sqrt{8}$\label{m38349*uid34}\item $\sqrt{768}$\label{m38349*uid35}\item $\sqrt{100}$\label{m38349*uid36}\item $\sqrt{0,49}$\label{m38349*uid37}\item $\sqrt{0,0016}$\label{m38349*uid38}\item $\sqrt{0,25}$\label{m38349*uid39}\item $\sqrt{36}$\label{m38349*uid40}\item $\sqrt{1960}$\label{m38349*uid41}\item $\sqrt{0,0036}$\label{m38349*uid42}\item $-8\sqrt{0,04}$\label{m38349*uid43}\item $5\sqrt{80}$\end{enumerate}
        \label{m38349*uid44}\item Write the following irrational numbers to 3 decimal places and then write them as a rational number to get an approximation to the irrational number. For example, $\sqrt{3}=1,73205...$. To 3 decimal places, $\sqrt{3}=1,732$. $1,732=1\dfrac{732}{1000}=1\dfrac{183}{250}$. Therefore, $\sqrt{3}$ is approximately $1\dfrac{183}{250}$.
\label{m38349*id326443}\begin{enumerate}[itemsep=5pt, label=\textbf{\alph*}. ] 
            \label{m38349*uid45}\item $3,141592654...$\label{m38349*uid46}\item $1,618\phantom{\rule{0.166667em}{0ex}}033\phantom{\rule{0.166667em}{0ex}}989\phantom{\rule{0.166667em}{0ex}}...$\label{m38349*uid47}\item $1,41421356...$\label{m38349*uid48}\item $2,71828182845904523536...$\end{enumerate}
        \end{enumerate}
  \label{m38349**end}
\par \raisebox{-5 pt}{\includegraphics[width=0.5cm]{col11306.imgs/summary_www.png}} Find the answers with the shortcodes:
 \par \begin{tabular}[h]{cccccc}
 (1.) llN  &  (2.) llR  &  (3.) lln  &  (4.) llQ  &  (5.) llU  & \end{tabular}
\subsubsection*{Estimating Surds}% -------------------------------------------------------
% ESTIMATING SURDS
%           \subsubsubsection \subsection{ End of Chapter Exercises}
            \nopagebreak
            \label{m38347*cid4} $ \hspace{-5pt}\begin{array}{cccccccccccc}   \end{array} $ \hspace{2 pt}\raisebox{-5 pt}{\includegraphics[width=0.5cm]{col11306.imgs/summary_www.png}} {(subsection shortcode: MG10054 )} \par \label{m38347*id260269}\begin{enumerate}[itemsep=5pt, label=\textbf{\arabic*}. ] 
            \item Answer the following multiple choice questions:
            \label{m38347*id7221}\begin{enumerate}[itemsep=5pt, label=\textbf{\alph*}. ] 
            \item $\sqrt{5}$ lies between:
\label{m38347*id7241}\begin{enumerate}[itemsep=5pt, label=\textbf{\roman*}. ] 
            \item 1 and 2\item 2 and 3\item 3 and 4\item 4 and 5\end{enumerate}
        \item 
              $\sqrt{10}$ lies between:
\label{m38347*id72245}\begin{enumerate}[itemsep=5pt, label=\textbf{\roman*}. ] 
            \item  1 and 2\item  2 and 3\item  3 and 4\item  4 and 5\end{enumerate}
        \item 
              $\sqrt{20}$ lies between:
\label{m38347*id72345}\begin{enumerate}[itemsep=5pt, label=\textbf{\roman*}. ] 
            \item 2 and 3\item 3 and 4\item 4 and 5\item 5 and 6\end{enumerate}
                      \item $\sqrt{30}$ lies between:
\label{m38347*id722643}\begin{enumerate}[itemsep=5pt, label=\textbf{\roman*}. ] 
            \item 3 and 4\item 4 and 5\item 5 and 6\item 6 and 7\end{enumerate}
                      \item $\sqrt[3]{5}$ lies between:
\label{m38347*id7351}\begin{enumerate}[itemsep=5pt, label=\textbf{\roman*}. ] 
            \item  1 and 2\item  2 and 3\item  3 and 4\item  4 and 5\end{enumerate}
                     \item $\sqrt[3]{10}$ lies between:
\label{m38347*id76451}\begin{enumerate}[itemsep=5pt, label=\textbf{\roman*}. ] 
            \item  1 and 2\item  2 and 3\item  3 and 4\item  4 and 5\end{enumerate}
                      \item $\sqrt[3]{20}$ lies between:
\label{m38347*id7334}\begin{enumerate}[itemsep=5pt, label=\textbf{\roman*}. ] 
            \item 2 and 3\item 3 and 4\item 4 and 5\item 5 and 6\end{enumerate}
                      \item $\sqrt[3]{30}$ lies between:
\label{m38347*id73224}\begin{enumerate}[itemsep=5pt, label=\textbf{\roman*}. ] 
            \item 3 and 4\item 4 and 5\item 5 and 6\item 6 and 7\end{enumerate}
                      \end{enumerate}
        \item  Find two consecutive integers such that $\sqrt{7}$ lies between them.          \item  Find two consecutive integers such that $\sqrt{15}$ lies between them.          \end{enumerate}
  \label{m38347**end}
\par \raisebox{-5 pt}{\includegraphics[width=0.5cm]{col11306.imgs/summary_www.png}} Find the answers with the shortcodes:
 \par \begin{tabular}[h]{cccccc}
 (1.a) lqr  &  (1.b) lqY  &  (1.c) lqg  &  (1.d) lq4  &  (1.e) lq2  &  (1.f) lqT  &  (1.g) lqb  &  (1.h) ll5  &  (2.) lqW  &  (3.) lq1  & \end{tabular}
% ---------------------------------------------------------------------------------------
\subsubsection*{Products and Factors}
% PRODUCTS AND FACTORS
%             \subsubsection{ End of Chapter Exercises}
          
 \nopagebreak
      \label{m39392*id281011}\begin{enumerate}[itemsep=5pt, label=\textbf{\arabic*}. ] 
            \label{m39392*uid55}\item Factorise:
\label{m39392*id281026}\begin{enumerate}[itemsep=5pt, label=\textbf{\alph*}. ] 
            \item ${a}^{2}-9$\item ${m}^{2}-36$\item $9{b}^{2}-81$\item $16{b}^{6}-25{a}^{2}$\item ${m}^{2}-\left(1/9\right)$\item $5-5{a}^{2}{b}^{6}$\item $16b{a}^{4}-81b$\item ${a}^{2}-10a+25$\item $16{b}^{2}+56b+49$\item $2{a}^{2}-12ab+18{b}^{2}$\item $-4{b}^{2}-144{b}^{8}+48{b}^{5}$\end{enumerate}
                \label{m39392*id632}\item Factorise completely: \label{m39392*id6423}\begin{enumerate}[itemsep=5pt, label=\textbf{\alph*}. ] 
            \item $\left(16-{x}^{4}\right)$\item ${7x}^{2}-14x+7xy-14y$
\item ${y}^{2}-7y-30$
\item $1-x-{x}^{2}+{x}^{3}$
\item $-3\left(1-{p}^{2}\right)+p+1$\end{enumerate}
\item Simplify the following:
\label{m39392*eip-id1166762435067}\begin{enumerate}[itemsep=5pt, label=\textbf{\alph*}. ] 
            \item ${\left(a-2\right)}^{2}-a\left(a+4\right)$\item $\left(5a-4b\right)\left(25{a}^{2}+20\mathrm{ab}+16{b}^{2}\right)$\item $\left(2m-3\right)\left(4{m}^{2}+9\right)\left(2m+3\right)$\item $\left(a+2b-c\right)\left(a+2b+c\right)$\end{enumerate}
\item Simplify the following:
\label{m39392*eip-id1153}\begin{enumerate}[itemsep=5pt, label=\textbf{\alph*}. ] 
            \item $\dfrac{{p}^{2}-{q}^{2}}{p}÷\dfrac{p+q}{{p}^{2}-\mathrm{pq}}$\item $\dfrac{2}{x}+\dfrac{x}{2}-\dfrac{2x}{3}$\end{enumerate}
\label{m39392*uid56}\item Show that ${\left(2x-1\right)}^{2}-{\left(x-3\right)}^{2}$ can be simplified to $\left(x+2\right)\left(3x-4\right)$
\newline
\newline
\label{m39392*uid57}\item What must be added to ${x}^{2}-x+4$\hspace{1ex}to make it equal to ${\left(x+2\right)}^{2}$
\newline
\newline
\end{enumerate}
  \label{m39392**end}
  \label{d4e6ddcad4e2d9e383c4732da6858c66**end}
\par \raisebox{-5 pt}{\includegraphics[width=0.5cm]{col11306.imgs/summary_www.png}} Find the answers with the shortcodes:
 \par \begin{tabular}[h]{cccccc}
 (1.) liM  &  (2.) lTY  &  (3.) lTg  &  (4.) lT4  &  (5.) lib  &  (6.) liT  & \end{tabular}
\end{eocexercises}
