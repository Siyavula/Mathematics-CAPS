\chapter{Analitiese Meetkunde}
Analitiese meetkunde, ook bekend as koördinaatmeetkunde en vroëer bekend as Cartesiese meetkunde, is die studie van meetkunde op grond van die beginsels van algebra en die Cartesiese koördinaatstelsel. Dit is gemoeid met die definisie van meetkundige figure op ’n numeriese wyse en onttrek numeriese inlligting uit die voorstelling. Sommige beskou die ontwikkeling van analitiese meetkunde as die begin van moderne wiskunde.\par 

\section{Drawing figures on the Cartesian plane}
As ons die koördinate van die hoekpunte van ’n figuur het, dan kan ons die figuur op die Cartesiese vlak teken.
Byvoorbeeld, neem die vierhoek $ABCD$ met koördinate $A(1;1)$, $B(3;1)$, $C(3;3)$ en $D(1;3)$ en stel dit voor op die
Cartesiese vlak. Dit word getoon in Figuur 15.1.\par 

\setcounter{subfigure}{0}
\begin{figure}[H] % horizontal\label{m39107*id63458}
\begin{center}
\scalebox{.8}{
\begin{pspicture}(-5,-5)(5.5,5.5)
% \psaxes{<->}(0,0)(5,5)
% \psgrid[gridcolor=lightgray,linecolor=lightgray,subgriddiv=1](0,0)(0,0)(4,4)
\psaxes[linewidth=2pt,labels=none,ticks=none]{->}(0,0)(0,0)(5,5)
\pspolygon[linewidth=.1cm](1,1)(1,3)(3,3)(3,1)(1,1)
\uput[dl](1,1){\Large{$A$}}
\uput[dr](3,1){\Large{$B$}}
\uput[ur](3,3){\Large{$C$}}
\uput[ul](1,3){\Large{$D$}}
\uput[l](5.7,0){\Large{$x$}}
\uput[d](0,5.7){\Large{$y$}}
\end{pspicture}
}
% \caption{Drawing a figure on the Cartesian plane}
\end{center}
\label{fig:cartesianplane}
\end{figure} 

The order of the letters for naming a figure is important. Dit dui vir ons aan dat ons beweeg van punt: $A$ na punt $B$, $B$ na punt $C$, $C$ na punt $D$ na punt $D$  en dan weer terug na punt  $A$. It would also be correct to write quadrilateral $CBAD$ or $BADC$ but it is better to follow the convention of writing letters in alphabetical order.     

\section{Afstand tussen Twee Punte}
\Definition{Point}{A point is an ordered pair of numbers written as $(x;y)$.}
A point is a simple geometric object having location as its only property. 
\Definition{Distance}{Afstand is a getal wat beskryf hoe ver twee punte van mekaar is.}

\mindsetvid{the distance formula}{VMbls}

\begin{activity}{Afstand tussen twee punte}
Punte $P~(2;1)$, $Q~(-2;2)$ en $R~(2,-2)$ word gegee. 
\begin{itemize}
 \item Can we assume that $\hat{R}=90^{\circ}$? If so, why?
\item Gebruik die Stelling van Pythagoras $\triangle PQR$ om die lente van $PQ$ te vind.
\end{itemize}


\setcounter{subfigure}{0}
\begin{figure}[H] % horizontal\label{m39107*id63458}
\begin{center}
\scalebox{.8}{
\begin{pspicture}(-5,-5)(5.5,5.5)
% \psaxes{<->}(0,0)(5,5)
% \psgrid[gridcolor=lightgray,linecolor=lightgray,subgriddiv=1](0,0)(-3,-3)(3,3)
\psaxes[linewidth=1pt,labels=none,ticks=none]{->}(0,0)(-3,-3)(3,3)
\psline[linewidth=0.04cm](-2,-2)(2,-2)(2,1)
\psline[linestyle=dashed,linewidth=0.04cm](-2,-2)(2,1)
\uput[l](-2,-1.8){\Large{$Q$}}
\uput[dl](-1.5,-2){\Large{$(-2;-2)$}}
\uput[dr](2,-2){\Large{$R$}}
\uput[u](2.3,0.5){\Large{$P$}}
\uput[ur](2,1){\Large{$(2;1)$}}
\uput[l](3.5,0){\Large{$x$}}
\uput[d](0,3.5){\Large{$y$}}
\uput[dr](2.5,-2){\Large{$(2;-2)$}}
\end{pspicture}
}
% \caption{Triangle $PQR$}
\end{center}
\label{fig:trianglePQR}
\end{figure} 
\end{activity}       
% SOLUTION FOR TEACHERS' GUIDE BELOW:
% In figure~\ref{fig:trianglePQR}, it can be seen that the length of the line $PR$ is 3 units and the length of the line $QR$ is four units. However, $\Delta PQR$, has a right angle at $R$. Therefore, the length of the side $PQ$ can be obtained by using the Theorem of Pythagoras:\par 
%       
% \begin{eqnarray*}
% P{Q}^{2} & = & P{R}^{2}+Q{R}^{2} \\ \
% \therefore P{Q}^{2} & = & {3}^{2}+{4}^{2} \\ 
% \therefore PQ & = & \sqrt{{3}^{2}+{4}^{2}}=5  
% \end{eqnarray*}
% The length of $PQ$ is the distance between the points $P$ and $Q$.\par 
Die formule vir die berekening van die afstand tussen twee punte word as volg afgelei. Die afstand tussen twee
punte $A~({x}_{1};{y}_{1})$ en $B~({x}_{2};{y}_{2})$, gebruik ons die Stelling van Pythagoras:\par 

\setcounter{subfigure}{0}
\begin{figure}[H] % horizontal\label{m39107*id63458}
\begin{center}
\scalebox{.8}{
\begin{pspicture}(-5,-5)(5.5,5.5)
% \psaxes{<->}(0,0)(5,5)
% \psgrid[gridcolor=lightgray,linecolor=lightgray,subgriddiv=1,gridlabels=0.0cm](0,0)(-3,-3)(3,3)
\psaxes[linewidth=1pt,labels=none,ticks=none]{<->}(0,0)(-3,-3)(3,3)
\psline[linewidth=0.04cm](-2,-2)(2,-2)(2,1)
\psline[linestyle=dashed,linewidth=0.04cm](-2,-2)(2,1)
\uput[l](-2,-1.8){\Large{$A$}}
\uput[dl](-1,-2){\Large{$(x_{1};y_{1})$}}
\uput[dr](2,-1.8){\Large{$C$}}
\uput[ur](2.5,-2.5){\Large{$(x_{2};y_{1})$}}
\uput[u](2.3,0.5){\Large{$B$}}
\uput[ur](1.8,1){\Large{$(x_{2};y_{2})$}}
\uput[l](4,0){\Large{$x$}}
\uput[d](0,4){\Large{$y$}}
\uput[d](-0.2,-0.05){\Large{$0$}}
\end{pspicture}
}
\end{center}
\end{figure}       
}
\Tip{Always draw a sketch - it helps with your calculation and makes it easier to check if your answer is correct.}
\begin{align*}
AB^2&=AC^{2}+BC^{2}\\
\therefore AB&=\sqrt{AC^{2}+BC^{2}}
\end{align*}
}ons sien


\begin{eqnarray*}
AC & = & {x}_{2}-{x}_{1}\\
BC & = & {y}_{2}-{y}_{1} 
\end{eqnarray*}
Dan is,
\begin{eqnarray*} 
AB & = & \sqrt{A{C}^{2}+B{C}^{2}} \\ 
& =& \sqrt{(x_{2}-x_{1})^{2}+(y_{2}-y_{1})^{2}} 
\end{eqnarray*}
Gevolglik, vir enige twee punte, $({x}_{1};{y}_{1})$ en $({x}_{2};{y}_{2})$, is die formule:\par 
\Identity{Afstand}{$\mbox{Afstand } d=\sqrt{{({x}_{1}-{x}_{2})}^{2}+{({y}_{1}-{y}_{2})}^{2}}$}
\Note{Note that $(x_1 - x_2)^2 = (x_2 - x_1)^2$, but remember to be consistent or you will get the wrong answer.\\
$AB \neq \\ \sqrt{(x_1 - x_2)^2 + (y_2 - 2_1)^2}$.}

 

% Die volgende video oor die afstandformule.

\vspace{2pt}
\vspace{.1in}
\end{figure}       

\begin{wex}{Deur die gebruik van die afstand formule}{Vind die afstand tussen $S(-2;-5)$ en $Q(7;-2)$.}{
\westep{Draw a sketch}
% \begin{figure}
 \begin{center}

\scalebox{1} % Change this value to rescale the drawing.
{
\begin{pspicture}(0,-4.096719)(8.579062,4.1367188)
\rput(3.0,1.9032812){\psaxes[linewidth=0.028222222,arrowsize=0.05291667cm 2.0,arrowlength=1.4,arrowinset=0.4,labels=none,ticks=none,ticksize=0.10583333cm]{<->}(0,0)(-3,-6)(5,2)}
\psline[linewidth=0.028222222cm](1.8,-2.0367188)(6.3,0.9632813)
\psline[linewidth=0.028222222,linestyle=dashed,dash=0.17638889cm 0.10583334cm](1.8,-2.0367188)(6.3,-2.0367188)(6.3,0.9632813)
\psline[linewidth=0.028222222](6.06,-2.0367188)(6.06,-1.7832812)(6.3,-1.7832812)
\usefont{T1}{ptm}{m}{n}
\rput(2.7045312,1.6632812){$0$}
\usefont{T1}{ptm}{m}{n}
\rput(3.3545313,3.9332812){$y$}
\usefont{T1}{ptm}{m}{n}
\rput(8.204532,2.1432812){$x$}
\usefont{T1}{ptm}{m}{n}
\rput(5.744531,1.1932813){$T(7,-2)$}
\usefont{T1}{ptm}{m}{n}
\rput(2,-2.3367188){$S(-2,-5)$}
\psdots[dotsize=0.127](6.3,0.9632813)
\psdots[dotsize=0.127](1.8,-2.0367188)
\end{pspicture} 
}
 \end{center}
% \end{figure}

\westep{Assign values to $(x_1;y_1)$ and $(x_2;y_2)$}
Gestel die koördinate van punt $S$ is $(x_1;y_1)$ en die koördinate van punt $T$ is $(x_2;y_2)$.
\begin{equation*}
x_1 = -2 \hskip2em y_1 = -5 \hskip2em x_2 = -7 \hskip2em y_2 = -2
\end{equation*}
\westep{Write down the distance formula}
\begin{equation*}
d = \sqrt{(x_1 - x_2)^2 + (y_1 - y_2)^2}
\end{equation*}
\westep{Substitute values}
\begin{equation*}
\begin{array}{cl}
d_{\mbox{ST}} &= \sqrt{(2 - (-7))^2 + (-5- (-2))^2}\\
& = \sqrt{(-9)^2 + (-3)^2}\\
&= \sqrt{81 + 9}\\
&= \sqrt{90}\\
&= 9,5
\end{array}
\end{equation*}
\westep{Write the final answer}
Die afstand tussen $S$ en $T$ is $9,5$ units.
\vspace{2pt}
\vspace{.1in}
}
\end{wex}

\begin{wex}{Using the distance formula}{Gegee $RS = 13$, $R(3;9)$ $S(8;y)$, Vind $y$.}{
\westep{Draw a sketch}
% \usepackage{pst-plot} % For axes
 \begin{center}
\scalebox{1} % Change this value to rescale the drawing.
{
\begin{pspicture}(0,-3.6167188)(7.8990626,3.6567187)
\rput(2.0,-1.6167188){\psaxes[linewidth=0.028222222,arrowsize=0.05291667cm 2.0,arrowlength=1.4,arrowinset=0.4,labels=none,ticks=none,ticksize=0.10583333cm]{<->}(0,0)(-2,-2)(5,5)}
\psline[linewidth=0.028222222,linestyle=dashed,dash=0.17638889cm 0.10583334cm](4.68,2.4567187)(3.1393335,-0.2630555)(4.64,-2.5032814)
\usefont{T1}{ptm}{m}{n}
\rput(1.7445313,-1.8767188){$0$}
\usefont{T1}{ptm}{m}{n}
\rput(2.3545313,3.4532812){$y$}
\usefont{T1}{ptm}{m}{n}
\rput(7.3,-1.4567188){$x$}
\usefont{T1}{ptm}{m}{n}
\rput(5.264531,2.7132812){$S_2(8;y_2)$}
\usefont{T1}{ptm}{m}{n}
\rput(5.0845313,-2.7567186){$S_1(8;y_1)$}
\psdots[dotsize=0.12](3.14,-0.22328125)
\psdots[dotsize=0.12](4.68,2.4367187)
\psdots[dotsize=0.12](4.66,-2.4632812)
\usefont{T1}{ptm}{m}{n}
\rput(2.6245313,-0.20671874){$R(3;9)$}
\psline[linewidth=0.04cm,linestyle=dotted,dotsep=0.16cm](4.68,3.4967186)(4.66,-3.5032814)
\end{pspicture} 
}
\end{center}
\westep{Assign values to $(x_1;y_1)$ and $(x_2;y_2)$:}
Gestel die koördinate van $R$ is $(x_1;y_1)$ en die koördinate van punt $S$ is $(x_2;y_2)$.
\begin{equation*}
x_1 = 3 \hskip2em y_1 = 9 \hskip2em x_2 = 8 \hskip2em y_2 = y
\end{equation*}
\westep{Write down the distance formula}
\begin{equation*}
d = \sqrt{(x_1 - x_2)^2 + (y_1 - y_2)^2}
\end{equation*}
\westep{Substitute values and solve for $y$}
\begin{equation*}
\begin{array}{cl}
13 &= \sqrt{(3 - 8)^2 + (9 - y)^2}\\
13^2 & = (-5)^2 + (81 - 18y + y^2)\\
0 &= y^2 - 18y - 63\\
&= (y+3) (y-21)\\
\therefore y &= -3 \mbox{ or } y = 21
\end{array}

\end{equation*}
\westep{Write the final answer}
$S$ is $(8;-3)$ or $(8;21)$
\vspace{2pt}
\vspace{.1in}
}
\end{wex}

\begin{exercises}{}
{
\begin{enumerate}[noitemsep, label=\textbf{\arabic*}. ]
\item Vind die lente $AB$ as:
    \begin{enumerate}[noitemsep, label=\textbf{(\alph*)} ] 
    \item $A(2;7)$ en $B(-3;5)$.
    \item $A(-3;5)$ en $B(-9;1)$.
    \item $A(x;y)$ en $B(x+4;y-1)$.
    \end{enumerate}

\item Die lengte van $CD=5$. Find the missing coordinate if:
    \begin{enumerate}[noitemsep, label=\textbf{(\alph*)} ] 
    \item $C(6;-2)$ en $D(x;2)$.
    \item $C(4;y)$ en $D(1;-1)$.
    \end{enumerate}
\end{enumerate}

\insertpracticeinfo{2}
}
\end{exercises}

%          \section{ Calculation of the gradient line}
%     \nopagebreak
%%%             \label{m39108} $ \hspace{-5pt}\begin{array}{cccccccccccc}   \includegraphics[width=0.75cm]{col11306.imgs/summary_video.png} &   \end{array} $ \hspace{2 pt}\raisebox{-5 pt}{} {(section shortcode: MG10109 )} \par 
%     
%     
%     

\section{Berekening van die Gradiënt van ’n Lyn}
\Definition{Die gradiënt van die lyn }{ die verhouding tussen die vertikale verandering in posisie en die
horisontale verandering in posisie.}
Die gradiënt $m$  van ’n lyn beskryf hoe steil die lyn is, hoe groot die helling van die lyn is.In die figuur hieronder is lyn $OQ$ se helling die grootste, en die lyn $OT$ is die steilste.

\mindsetvid{The gradient line}{VMbmr}

\setcounter{subfigure}{0}
\begin{figure}[H] % horizontal\label{m39107*id63458}
\begin{center}
\scalebox{.8}{
\begin{pspicture}(-5,-5)(5.5,5.5)
% \psaxes{<->}(0,0)(5,5)
% \psgrid[gridcolor=lightgray,linecolor=lightgray,subgriddiv=1,gridlabels=0.0cm](0,0)(-1,-1)(4,4)
\psaxes[linewidth=1pt,labels=none,ticks=none]{<->}(0,0)(-1,-1)(4,4)
\psline[linewidth=.05cm](0,0)(1,3.5)
\psline[linewidth=.05cm](0,0)(3,3.5)
\psline[linewidth=.05cm](0,0)(3,2)
\psline[linewidth=.05cm](0,0)(3,0.5)
\uput[ur](.9,3.5){\Large{$T$}}
% \uput[dl](-1,-2){\Large{$(x_{2},y_{2})$}}
\uput[ur](3,3.5){\Large{$S$}}
\uput[ur](3,2){\Large{$R$}}
\uput[ur](3,0.4){\Large{$Q$}}
\uput[dl](0,0){\Large{$O$}}
% \uput[ur](1.8,1){\Large{$(x_{1},y_{1})$}}
\uput[l](4.8,0){\Large{$x$}}
\uput[d](0,4.8){\Large{$y$}}
\end{pspicture}
}
\end{center}
\end{figure}        
To derive the formula for gradient,Dink aan ’n lyn met koördinate $A~(x_1;y_1)$ en $B~(x_2;y_2)$ met skuinssy $AB$, as shown in the diagram below.  
Die gradiënt is die verhouding van die lengte van die vertikale sy van die driehoek $y$-values of points $A$ and $B$. The length of the horizontal side of the triangle is the difference in $x$-values of points $A$ and $B$. 
\setcounter{subfigure}{0}
\begin{figure}[H] % horizontal\label{m39107*id63458}
\begin{center}
\scalebox{.8}{
\begin{pspicture}(-5,-5)(5.5,5.5)
% \psaxes{<->}(0,0)(5,5)
% \psgrid[gridcolor=lightgray,linecolor=lightgray,subgriddiv=1,gridlabels=0.0cm](0,0)(-3,-3)(3,3)
\psaxes[linewidth=1pt,labels=none,ticks=none]{<->}(0,0)(-3,-3)(3,3)
\psline(-2,-2)(2,-2)(2,1)
\psline[linestyle=solid,linewidth=0.04cm](-2,-2)(2,1)
\uput[l](-2,-1.8){\Large{$A$}}
\uput[dl](-1,-2){\Large{$(x_{1},y_{1})$}}
\uput[dr](2,-2){\Large{$C$}}
\uput[u](2.3,0.5){\Large{$B$}}
\uput[ur](1.8,1){\Large{$(x_{2},y_{2})$}}
\uput[l](4,0){\Large{$x$}}
\uput[d](0,4){\Large{$y$}}
\uput[dl](0,0){\Large{$0$}}
\end{pspicture}
}
\end{center}
\end{figure}  
Therefore, gradient is determined using the following equation:

\Identity{Gradiënt}{Gradiënt $m= \dfrac{y_{2} - y_{1}}{x_{2} - x_{1}}$ or $= \dfrac{y_{1} - y_{2}}{x_{1} - x_{2}}$
 \\ [5pt] Remember to be consistent: $m \neq \dfrac{y_{1} - y_{2}}{x_{2} - x_{1}}$}
\Tip{Remember: we can also use $m$ for the gradient of a line}
\begin{wex}{Gradient between two points}{Find the gradient of the line between points E$(2;5)$ and F$(-3;9)$.}{
\westep{Draw a sketch}
\begin{center}
\scalebox{1} % Change this value to rescale the drawing.
{
\begin{pspicture}(0,-3.6267188)(6.6990623,3.6667187)
\rput(3.0,-1.6267188){\psaxes[linewidth=0.04,arrowsize=0.05291667cm 2.0,arrowlength=1.4,arrowinset=0.4,labels=none,ticks=none,ticksize=0.10583333cm]{<->}(0,0)(-3,-2)(3,5)}
\psdots[dotsize=0.12](4.82,0.69328123)
\psdots[dotsize=0.12](1.06,2.8732812)
\psline[linewidth=0.04cm](4.84,0.67328125)(1.08,2.8532813)
\usefont{T1}{ptm}{m}{n}
\rput(6.324531,-1.3767188){$x$}
\usefont{T1}{ptm}{m}{n}
\rput(3.3045313,3.4632812){$y$}
\usefont{T1}{ptm}{m}{n}
\rput(5.094531,0.5232813){$E~(2;5)$}
\usefont{T1}{ptm}{m}{n}
\rput(1.2945312,3.0832813){$F~(-3;9)$}
\usefont{T1}{ptm}{m}{n}
\rput(2.7445312,-1.8367188){$0$}
\end{pspicture} 
}
\end{center}
\westep{Assign values to $(x_1;y_1)$ and $(x_2;y_2)$}
Let the coordinates of $E$ be $(x_1;y_1)$ and the coordinates of $F$ be $(x_2;y_2)$.
\begin{equation*}
x_1 = 2 \hskip2em y_1 = 5 \hskip2em x_2 = -3 \hskip2em y_2 = 9
\end{equation*}
\westep{Write down the formula for gradient}
\begin{equation*}
m = \dfrac{y_2 - y_1}{x_2 - x_1}
\end{equation*}
\westep{Substitute known values}
\begin{equation*}
\begin{array}{cl}
m_{EF} &= \dfrac{9 - 5}{-3 - 2}\\[5pt]
&= \dfrac{4}{-5}
\end{array}
\end{equation*}
\westep{Write the final answer}
The gradient of $EF = \dfrac{4}{-5}$

}
\end{wex}


\begin{wex}{Gradient between two points}{Given $G(7;-9)$ and $H(x;0)$, with $m_{GH}= 3$, Find $x$.}{
\westep{Draw a sketch}
\begin{center}
\scalebox{1} % Change this value to rescale the drawing.
{
\begin{pspicture}(0,-3.6867187)(7.1790624,3.7267187)
\rput(3.0,1.3132813){\psaxes[linewidth=0.04,arrowsize=0.05291667cm 2.0,arrowlength=1.4,arrowinset=0.4,labels=none,ticks=none,ticksize=0.10583333cm]{<->}(0,0)(-3,-3)(3,2)}
\psdots[dotsize=0.12](4.9,1.2932812)
\psdots[dotsize=0.12](3.9,-0.74671876)
\psline[linewidth=0.04cm](3.88,-0.76671875)(4.9,1.2732812)
\usefont{T1}{ptm}{m}{n}
\rput(6.324531,1.5632813){$x$}
\usefont{T1}{ptm}{m}{n}
\rput(3.1445312,3.5232813){$y$}
\usefont{T1}{ptm}{m}{n}
\rput(5.324531,1.6032813){$H~(x;0)$}
\usefont{T1}{ptm}{m}{n}
\rput(4.134531,-0.99671876){$G~(7;-9)$}
\usefont{T1}{ptm}{m}{n}
\rput(2.7445312,1.0432812){$0$}
\usefont{T1}{ptm}{m}{n}
\rput(5.7745314,0.28984374){$m_{GH} = 3$}
\end{pspicture} 
}
\end{center}
\westep{Assign values to $(x_1;y_1)$ and $(x_2;y_2)$}
Laat die koördinate of $G$ wees $(x_1;y_1)$ en die koördinate van $H$ wees $(x_2;y_2)$.
\begin{equation*}
x_1 = 7 \hskip2em y_1 = -9 \hskip2em x_2 = x \hskip2em y_2 = 0
\end{equation*}
\westep{Write down the formula for gradient}
\begin{equation*}
m = \dfrac{y_2 - y_1}{x_2 - x_1}
\end{equation*}
\westep{Substitute values and solve for $x$}
\begin{equation*}
\begin{array}{cl}
3 &= \dfrac{0 - (-9)}{x - 7}\\[5pt]
3(x-7)&= 9\\
x-7 &= \frac{9}{3}\\
x-7 &= 3\\
x &= 3 + 7\\
&= 10 \\
\end{array}
\end{equation*}
\westep{Write the final answer}
laat die koördinate van $H$ wees $(10;0)$.
\vspace{2pt}
\vspace{.1in}
}
\end{wex}


\begin{exercises}{}
{
\begin{enumerate}[noitemsep, label=\textbf{(\arabic*)} ]
\item Find the gradient of $AB$ if
    \begin{enumerate}[noitemsep, label=\textbf{(\alph*)} ] 
    \item $A(7;10)$ and $B(-4;1)$
    \item $A(-5;-9)$ and $B(3;2)$
    \item $A(x-3;y)$ and $B(x;y+4)$
    \end{enumerate}

\item If the gradient of $CD=\frac{2}{3}$, vind $P$ gegee
    \begin{enumerate}[noitemsep, label=\textbf{(\alph*)} ] 
    \item $C(16;2)$ en $D(8;P)$
    \item $C(3;2P)$ en $D(9;14)$
    \end{enumerate}
\end{enumerate}

\insertpracticeinfo{2}
}
\end{exercises}


\subsection*{Reguitlyn}
\Definition{Reguitlyn}{A straight line is a set of points with a constant gradient between any two of the points.}
Consider the diagram below with points $A(x;y)$, $B(x_2;y_2)$ and $C(x_1;y_1)$. 
\begin{center}
 \scalebox{1} % Change this value to rescale the drawing.
{
\begin{pspicture}(0,-3.0667188)(7.5290623,3.1067188)
\rput(3.97,-0.06671875){\psaxes[linewidth=0.04,arrowsize=0.05291667cm 2.0,arrowlength=1.4,arrowinset=0.4,labels=none,ticks=none,ticksize=0.10583333cm]{<->}(0,0)(-3,-3)(3,3)}
\psdots[dotsize=0.12](3.39,0.47328126)
\psdots[dotsize=0.12](4.57,2.2132812)
\psline[linewidth=0.04cm](2.17,-1.3467188)(4.69,2.3932812)
\usefont{T1}{ptm}{m}{n}
\rput(7.1545315,0.14328125){$x$}
\usefont{T1}{ptm}{m}{n}
\rput(4.2545314,2.9032812){$y$}
\usefont{T1}{ptm}{m}{n}
\rput(5.514531,2.1832812){$A(x;y)$}
\usefont{T1}{ptm}{m}{n}
\rput(3.6745312,-0.33671874){$0$}
\psdots[dotsize=0.12](2.51,-0.86671877)
\usefont{T1}{ptm}{m}{n}
\rput(2.5845313,0.5832813){$B(x_2;y_2)$}
\usefont{T1}{ptm}{m}{n}
\rput(1.7,-0.7967188){$C(x_1;y_1)$}
\end{pspicture} 
}
\end{center}
We have $m_{AB} = m_{BC}=m_{AC}$ \par
and $m = \dfrac{y_2-y_1}{x_2-x_1} = \dfrac{y_1-y_2}{x_1-x_2}$\par

The general formula for a straight line is $\dfrac{y-y_1}{x-x_1} = \dfrac{y_2-y_1}{x_2-x_1}$ \\(where $(x;y)$ is any point on the line)\par

The standard form of the straight line equation is $y=mx+c$ where $m$ is the gradient and $c$ is the $y$-intercept.

\begin{wex}{Finding the equation of a straight line}
 {Find the equation of the straight line through $P(-1;-5)$ and $Q(5;4)$.}
{
\westep{Draw a sketch}
\begin{center}
\scalebox{1} % Change this value to rescale the drawing.
{
\begin{pspicture}(0,-3.0667188)(6.5590625,3.1067188)
\rput(3.0,-0.06671875){\psaxes[linewidth=0.04,arrowsize=0.05291667cm 2.0,arrowlength=1.4,arrowinset=0.4,labels=none,ticks=none,ticksize=0.10583333cm]{<->}(0,0)(-3,-3)(3,3)}
\psdots[dotsize=0.12,dotangle=-5.9493704](4.5090275,2.306018)
\psline[linewidth=0.04cm](2.14,-0.52671874)(4.48,2.3132813)
\usefont{T1}{ptm}{m}{n}
\rput(6.184531,0.14328125){$x$}
\usefont{T1}{ptm}{m}{n}
\rput(3.2845314,2.9032812){$y$}
\usefont{T1}{ptm}{m}{n}
\rput(4.8545313,2.6232812){$Q(5;4)$}
\usefont{T1}{ptm}{m}{n}
\rput(2.7045312,-0.33671874){$0$}
\psdots[dotsize=0.12,dotangle=-5.9493704](2.1408823,-0.5438744)
\usefont{T1}{ptm}{m}{n}
\rput(1.7045312,-0.81671876){$P(-1;-5)$}
\end{pspicture} 
}
\end{center}
\westep{Assign values}
Let $(x;y)$ be any point on the line. \\
$x_1 = 5 \hskip2em y_1 = 4 \hskip2em x_1 = x \hskip2em y_2 = -5$


\westep{Write down the general formula}
\begin{align*}
\dfrac{y-y_1}{x-x_1} &= \dfrac{y_2-y_1}{x_2-x_1}
\end{align*}
\westep{Substitute values and make $y$ the subject of the equation}
\begin{align*}
 \dfrac{y-(-5)}{x-(-1)} &= \dfrac{4-(-5)}{5-(-1)} \\

 \dfrac{y+5}{x+1)} &= \dfrac{9}{6} = \dfrac{3}{2}\\
2(y+5) &=3(x+1)\\
2y +10&=3x+3\\
2y&=3x-7\\
y&=\frac{3}{2}x - \frac{7}{2}
\end{align*}
\westep{Write the final answer}
The equation of the straight line is $y&=\frac{3}{2}x - \frac{7}{2}$.
}


\end{wex}

\subsection*{Parallel and perpendicular lines}    
%         \label{m39108*eip-332}We can use the gradient of a line to determine if two lines are parallel or perpendicular. If the lines are parallel (Figure~\ref{fig:parallelperpendicular}a) then they will have the same gradient, i.e. ${m}_{\mbox{AB}}={m}_{\mbox{CD}}$. If the lines are perpendicular (Figure~\ref{fig:parallelperpendicular}b) than we have: 
%     \setcounter{subfigure}{0}
%  	\begin{figure}[H] % horizontal\label{m39107*id63458}
%     \begin{center}
% \scalebox{.8}{
% \begin{pspicture}(-5,-5)(5.5,5.5)
% % \psaxes{<->}(0,0)(5,5)
% \rput(-2,-2){
% \psline[linewidth=.05cm](0,0)(0,3)
% \psline[linewidth=.05cm](1,0)(1,3)
% \uput[ur](-1,2.8){\Large{$a)$}}
% \uput[d](0,0){\Large{$A$}}
% \uput[u](0,3){\Large{$B$}}
% \uput[d](1,0){\Large{$C$}}
% \uput[u](1,3){\Large{$D$}}}
% 
% \rput(3.4,2){
% \psline[linewidth=.05cm](1,-4)(-2,-1)
% \psline[linewidth=.05cm](-2,-4)(1,-1)
% \uput[ur](-3.2,-1.2){\Large{$b)$}}
% \uput[dr](1,-4){\Large{$A$}}
% \uput[ul](-2,-1){\Large{$B$}}
% \uput[dl](-2,-4){\Large{$C$}}
% \uput[ur](1,-1){\Large{$D$}}}
% \end{pspicture}
% }
%     \end{center}
% \caption{a) Parallel and b) perpendicular lines}
% \label{fig:parallelperpendicular}
%  \end{figure}       
Two lines have that run parallel to each other have equal gradients. \\
In other words: $\mbox{gradient}_{AB}=\mbox{gradient}_{CD}$. \par
If two lines intersect perpendicularly then the product of their gradients is equal to $-1$. \\

If line $WX \perp $ line $ YZ$ then $m_{WX} \times m_{YZ} = -1$.

\mindsetvid{Parallel and perpendicular lines}{VMbon}

\begin{wex}{Parallel lines}{Prove that the line $AB$ with points $A(2;4)$ and $B(8,16)$ is parallel to the line $y = 2x-2$.}{
\westep{Draw a sketch}

\begin{center}
 \scalebox{1} % Change this value to rescale the drawing.
{
\begin{pspicture}(0,-3.0667188)(6.5590625,3.1067188)
\rput(3.0,-0.06671875){\psaxes[linewidth=0.04,arrowsize=0.05291667cm 2.0,arrowlength=1.4,arrowinset=0.4,labels=none,ticks=none,ticksize=0.10583333cm]{<->}(0,0)(-3,-3)(3,3)}
\psdots[dotsize=0.12,dotangle=-5.9493704](4.5090275,2.306018)
\psline[linewidth=0.04cm](1.56,-1.7267188)(4.52,2.3132813)
\usefont{T1}{ptm}{m}{n}
\rput(6.184531,0.14328125){$x$}
\usefont{T1}{ptm}{m}{n}
\rput(3.2845314,2.9032812){$y$}
\usefont{T1}{ptm}{m}{n}
\rput(4.9445314,2.6232812){$B(8;16)$}
\usefont{T1}{ptm}{m}{n}
\rput(2.7845314,-0.31671876){$0$}
\usefont{T1}{ptm}{m}{n}
\rput(2.8345313,0.94328123){$A(2;4)$}
\psline[linewidth=0.04cm](1.9,-2.0667188)(4.86,1.9732813)
\psdots[dotsize=0.12](3.4,0.79328126)
\usefont{T1}{ptm}{m}{n}
\rput(3.1945312,-0.69671875){$-2$}
\rput(5.2145314,1.2232813){$y=2x-2$}
\end{pspicture} 
}

\end{center}
Be careful - some lines may look parallel but are not!

\westep{Write down the formula for gradient}
\begin{equation*}
m = \dfrac{y_2-y_1}{x_2-x_1}
\end{equation*}
\westep{Substitute values to find the gradient for line $AB$}
\begin{equation*}
\begin{array}{cl}
m_{AB} &= \dfrac{16 - 4}{8 - 2}\\[5pt]
&= \dfrac{12}{6}\\
&= 2
\end{array}
\end{equation*}
\westep{Check that $y$ is in the standard form $y=mx+c$}
\begin{equation*}
\begin{array}{cl}
y&=2x-2\\
\therefore m_{y}&= 2
\end{array}
\end{equation*}
\westep{Write the final answer}
\begin{equation*}
\begin{array}{cl}
m_{AB} &= m_{y}\\

\end{array}
\end{equation*}
therefore line $AB$ is parallel to $y=2x-2$.
}
\end{wex}

Perpendicular lines have gradients that are the negative inverse of each other. This means that when you multiply the gradients together, the answer will be -1.: $-\frac{1}{m_{AB}}=m_{CD}$

\begin{wex}{Perpendicular lines}{Line $AB$ is perpendicular to line $CD$. Find $y$ given $A(2;-3)$, $B(-2;6)$, $C(4;3)$ and $D(7;y)$.}{
\westep{Draw a sketch}
\begin{center}
 \scalebox{1} % Change this value to rescale the drawing.
{
\begin{pspicture}(0,-3.0667188)(6.7390623,3.1067188)
\rput(3.0,-0.06671875){\psaxes[linewidth=0.04,arrowsize=0.05291667cm 2.0,arrowlength=1.4,arrowinset=0.4,labels=none,ticks=none,ticksize=0.10583333cm]{<->}(0,0)(-3,-3)(3,3)}
\psdots[dotsize=0.12,dotangle=-5.9493704](4.089028,0.96601814)
\psline[linewidth=0.04cm](1.62,-1.5267187)(5.68,2.5732813)
\usefont{T1}{ptm}{m}{n}
\rput(6.184531,0.14328125){$x$}
\usefont{T1}{ptm}{m}{n}
\rput(3.2845314,2.9032812){$y$}
\usefont{T1}{ptm}{m}{n}
\rput(1.3445313,2.4432812){$B(8;16)$}
\usefont{T1}{ptm}{m}{n}
\rput(2.6845312,-0.25671875){$0$}
\usefont{T1}{ptm}{m}{n}
\rput(4.264531,-0.7967188){$A(2;-3)$}
\psline[linewidth=0.04cm](1.52,2.1732812)(4.2,-0.56671876)
\psdots[dotsize=0.12](1.56,2.1732812)
\psdots[dotsize=0.12](4.18,-0.52671874)
\usefont{T1}{ptm}{m}{n}
\rput(3.9845312,1.2232813){$C(4;3)$}
\psline[linewidth=0.04cm,linestyle=dashed,dash=0.16cm 0.16cm](5.1,2.8132813)(5.12,-0.44671875)
\usefont{T1}{ptm}{m}{n}
\rput(5.914531,1.9032812){$D(7;y)$}
\psdots[dotsize=0.12](5.12,2.0332813)
\end{pspicture} 
}
\end{center}


\westep{Write the formula for the product of perpendicular gradients}
\begin{equation*}
m_{AB} \times m_{CD} = -1
\end{equation*}
\westep{Substitute values and solve for $y$}
\begin{equation*}
\begin{array}{rl}
\dfrac{6 - (-3)}{-2 -2} \times \dfrac{y - 3}{7 - 4} &= -1\\[5pt]
\dfrac{9}{-4} \times \dfrac{y-3}{3} &= -1\\[5pt]
\dfrac{y-3}{3} &= -1 \times \dfrac{-4}{9}\\[5pt]
\dfrac{y-3}{3} &= \dfrac{4}{9}\\[5pt]
y-3 &= \dfrac{4}{9} \times 3\\[5pt]
y-3 &= \dfrac{4}{3}\\[5pt]
y &= \dfrac{4}{3} + 3\\[5pt]
&= \dfrac{4 + 9}{3}\\[5pt]
&= \dfrac{13}{3}\\[5pt]
&= 4 \dfrac{1}{3}
\end{array}
\end{equation*}
\westep{Write the final answer}
Therefore the coordinates of $D$ are $(7; 4\frac{1}{3})$.
}
\end{wex}

\subsection*{Horizontal and vertical lines}

A line that runs parallel to the $x$-axis is called a horizontal line and has a gradient of zero. This is
because there is no vertical change in the gradient of the line:\par
$m = \dfrac{\mbox{change in }y}{\mbox{change in }x} = \dfrac{0}{\mbox{change in }x} =0$\par

A line runs parallel to the $y$-axis,is called a vertical line and its gradient is undefined. This is because there is no horizontal change in the gradient of the line:\par
$m = \dfrac{\mbox{change in }y}{\mbox{change in }x} = \dfrac{\mbox{change in }y}{0}=$ undefined\par

\subsection*{Collinear points}

Points are collinear if they lie on the same line. There are two methods to prove that points are collinear; the gradient method, and a
longer method using the distance formula.

If three points are collinear then the line joining them will have the same gradient at any point along the line.
Therefore to prove collinearity, we must prove that two of the gradients between any of the three points are
equal.

\begin{wex}{Collinear points}{Prove that $A(-3;3)$, $B(0;5)$ and $C(3;7)$ are collinear.}{
\westep{Draw a sketch}
\begin{center}
 \scalebox{1} % Change this value to rescale the drawing.
{
\begin{pspicture}(0,-2.0667188)(6.3990626,2.1067188)
\rput(3.0,-1.0667187){\psaxes[linewidth=0.04,arrowsize=0.05291667cm 2.0,arrowlength=1.4,arrowinset=0.4,labels=none,ticks=none,ticksize=0.10583333cm]{<->}(0,0)(-3,-1)(3,3)}
\psdots[dotsize=0.12,dotangle=-5.9493704](4.4890275,1.3460182)
\usefont{T1}{ppl}{m}{n}
\rput(6.0245314,-0.81671876){$x$}
\usefont{T1}{ppl}{m}{n}
\rput(3.2845314,1.9032812){$y$}
\usefont{T1}{ppl}{m}{n}
\rput(2.8945312,0.94328123){$B(0;5)$}
\usefont{T1}{ppl}{m}{n}
\rput(2.7845314,-1.3167187){$0$}
\usefont{T1}{ppl}{m}{n}
\rput(1.2845312,0.16328125){$A(-3;3)$}
\psdots[dotsize=0.12](1.22,-0.16671875)
\psdots[dotsize=0.12](3.0,0.6532813)
\usefont{T1}{ppl}{m}{n}
\rput(4.684531,1.5432812){$C(3;7)$}
\end{pspicture} 
}

\end{center}

\westep{Calculate two gradients between any of the three points}
\begin{equation*}
m_{AB} = \frac{5-3}{0-(-3)} = \frac{2}{3}
\end{equation*}
and
\begin{equation*}
m_{BC} = \frac{7-5}{3-0} = \frac{2}{3}
\end{equation*}
\fbox{OR:}
\begin{equation*}
m_{AC} = \frac{3-7}{3-3} = \frac{-4}{-6}=\frac{2}{3}
\end{equation*}
and
\begin{equation*}
m_{BC} = \frac{7-5}{3-0} = \frac{2}{3}
\end{equation*}
\westep{Explain your answer}
\begin{equation*}
m_{AB} = m_{BC}(= m_{AC})
\end{equation*}
Therefore the points $A$, $B$ and $C$ are collinear.
}
\end{wex}

To prove that three points are collinear using the distance formula, we must calculate the
distances between each pair of points and then provethat the sum of the two smaller distances
equals the longest distance.


\begin{wex}{Collinear points}{Prove that $A(-3;3)$, $B(0;5)$ and $C(3;7)$ are collinear.}{
\westep{Draw a sketch}

\begin{center}
 \scalebox{1} % Change this value to rescale the drawing.
{
\begin{pspicture}(0,-2.0667188)(6.3990626,2.1067188)
\rput(3.0,-1.0667187){\psaxes[linewidth=0.04,arrowsize=0.05291667cm 2.0,arrowlength=1.4,arrowinset=0.4,labels=none,ticks=none,ticksize=0.10583333cm]{<->}(0,0)(-3,-1)(3,3)}
\psdots[dotsize=0.12,dotangle=-5.9493704](4.4890275,1.3460182)
\psline[linewidth=0.04cm,linestyle=dashed,dash=0.16cm 0.16cm,arrowsize=0.05291667cm 2.0,arrowlength=1.4,arrowinset=0.4]{<->}(1.32,-0.14671876)(2.98,0.63328123)
\usefont{T1}{ppl}{m}{n}
\rput(6.0245314,-0.81671876){$x$}
\usefont{T1}{ppl}{m}{n}
\rput(3.2845314,1.9032812){$y$}
\usefont{T1}{ppl}{m}{n}
\rput(2.4,0.8){$B(0;5)$}
\usefont{T1}{ppl}{m}{n}
\rput(2.7845314,-1.3167187){$0$}
\usefont{T1}{ppl}{m}{n}
\rput(1.0,0.16328125){$A(-3;3)$}
\psline[linewidth=0.04cm,linestyle=dashed,dash=0.16cm 0.16cm,arrowsize=0.05291667cm 2.0,arrowlength=1.4,arrowinset=0.4]{<->}(1.28,-0.48671874)(4.58,1.0332812)
\psdots[dotsize=0.12](1.22,-0.16671875)
\psdots[dotsize=0.15](3.0,0.6532813)
\psline[linewidth=0.04cm,linestyle=dashed,dash=0.16cm 0.16cm,arrowsize=0.05291667cm 2.0,arrowlength=1.4,arrowinset=0.4]{<->}(3.14,0.7132813)(4.44,1.3332813)
\usefont{T1}{ppl}{m}{n}
\rput(4.684531,1.5432812){$C(3;7)$}
\end{pspicture} 
}
\end{center}
\westep{Calculate the three distances $AB$, $BC$ and $AC$}
\begin{equation*}
d_{AB} = \sqrt{(-3 - 0)^2 + (3 - 5)^2} = \sqrt{(-3)^2 + (-2)^2} = \sqrt{9 + 4} = \sqrt{13}
\end{equation*}
\begin{equation*}
d_{BC} = \sqrt{(0 - 3)^2 + (5 - 7)^2} = \sqrt{(-3)^2 + (-2)^2} = \sqrt{9 + 4} = \sqrt{13}
\end{equation*}
\begin{equation*}
d_{AC} = \sqrt{((-3) - 3)^2 + (3 - 7)^2} = \sqrt{(-6)^2 + (-4)^2} = \sqrt{36 + 16} = \sqrt{52}
\end{equation*}
\westep{Find the sum of the two shorter distances}
\begin{equation*}
d_{AB} + d_{BC} = \sqrt{13} + \sqrt{13} = 2\sqrt{13} = \sqrt{4 \times 13} = \sqrt{52}
\end{equation*}
\westep{Explain your answer}
\begin{equation*}
d_{AB} + d_{BC} = d_{AC}
\end{equation*}
therefore points $A$, $B$ and $C$ are collinear.
}
\end{wex}

\begin{exercises}{}
{
\begin{enumerate}[itemsep=5pt, label=\textbf{\arabic*}. ]

\item Determine whether $AB$ is parallel, perpendicular or neither to $CD$ if:
    \begin{enumerate}[noitemsep, label=\textbf{(\alph*)} ]
    \item $A(3;-4)$, $B(5;2)$, $C(-1;-1)$, $D(7;23)$.
    \item $A(3;-4)$, $B(5;2)$, $C(-1;-1)$, $D(0;-4)$.
    \item $A(3;-4)$, $B(5;2)$, $C(-1;-1)$, $D(1;2)$.
    \end{enumerate}

\item Determine whether the following points are collinear:
    \begin{enumerate}[noitemsep, label=\textbf{(\alph*)} ]
    \item $E(0;3)$, $F(-2;5)$, $G(2;1)$.
    \item $H(-3;-5)$, $I(-0;0)$, $J(6;10)$.
    \item $K(-6;2)$, $L(-3;1)$, $M(1;-1)$.
    \end{enumerate}

\item Points $P(-6;2)$, $Q(2;-2)$ and $R(-3;r)$ lie on a straight line. Find the value of $r$.

\item Line $PQ$ with $P(-1;-7)$ and $Q(q;0)$ has a gradient of $1$. Find $q$.
\end{enumerate}
\insertpracticeinfo{4}
}
\end{exercises}

% Die volgende video bied ’n opsomming van die gradiënt van ’n lyn.


\vspace{2pt}
\vspace{.1in}
\end{figure}      
%          \section{ Midpoint of a line}
%     \nopagebreak
%%%             \label{m39119} $ \hspace{-5pt}\begin{array}{cccccccccccc}   \includegraphics[width=0.75cm]{col11306.imgs/summary_video.png} &   \end{array} $ \hspace{2 pt}\raisebox{-5 pt}{} {(section shortcode: MG10111 )} \par 
%     
%     
%     

\section{Middelpunt van ’n Lynstuk}
\begin{activity}{Finding the mid-point of a line}
On graph paper, accurately plot the points $P(2;1)$ and $Q(-2;2)$ and draw the line $PQ$.
\begin{itemize}
 \item Fold the piece of paper so that point $P$ is exactly on top of point $Q$
\item Where the folded line intersects with line $PQ$, label point $S$
\item Count the blocks and find the exact location of $S$
\item Write down the coordinates of $S$
\end{itemize}

\end{activity}

\mindsetvid{the midpoint of a line segment}{VMbpr}

Die koördinate van die middelpunt $M(x;y)$ van ’n lyn tussen enige twee punte A en B met koördinate $A(x_1;y_1)$ en $B(x_2;y_2)$ word as volg bereken:

\setcounter{subfigure}{0}
\begin{figure}[H] % horizontal\label{m39107*id63458}
\begin{center}
\scalebox{.8}{
\begin{pspicture}(-5,-5)(5.5,5.5)
% \psgrid[gridcolor=lightgray,linecolor=lightgray,subgriddiv=1,gridlabels=0.0cm](0,0)(0,0)(5,5)
\psaxes[linewidth=1pt,labels=none,ticks=none]{->}(0,0)(0,0)(5.5,5.5)
\psline[linewidth=.05cm](0,0)(5,5)
\uput[dl](0,0){\Large{$A (x_{1};y_{1})$}}
\uput[u](0,-.2){\qdisk(0,0){3pt}}
\uput[r](2.6,2.4){\Large{$M (X;Y)$}}
\uput[u](2.5,2.3){\qdisk(0,0){3pt}}
\uput[ur](5,5){\Large{$B (x_{2};y_{2})$}}
\uput[u](5,4.8){\qdisk(0,0){3pt}}
\uput[l](6.3,0){\Large{$x$}}
\uput[d](0,6.3){\Large{$y$}}
\end{pspicture}
}
\end{center}
\end{figure}      

\begin{eqnarray*}
X & = & \frac{{x}_{1} + {x}_{2}}{2} \\ 
Y & = & \frac{{y}_{1} + {y}_{2}}{2} \\  
\end{eqnarray*}
From this we obtain the mid-point of a line:
\Identity{Midpoint}{Mid-point $M=(\dfrac{{x}_{1} + {x}_{2}}{2};\dfrac{{y}_{1}+{y}_{2}}{2}) $}

\begin{wex}{Calculating the mid-point}
 {Calculate the coordinates of the mid-point $F(x;y)$ of the line between point $E(2;1)$ and point $G(-2;-2)$.}
{
\westep{Draw a sketch}
\setcounter{subfigure}{0}
\begin{figure}[H] % horizontal\label{m39107*id63458}
\begin{center}
\scalebox{.8}{
\begin{pspicture}(-5,-5)(5.5,5.5)
% \psaxes{<->}(0,0)(5,5)
% \psgrid[gridcolor=lightgray,linecolor=lightgray,subgriddiv=1,gridlabels=0.0cm](0,0)(-3,-3)(3,3)
\psaxes[linewidth=1pt,labels=none,ticks=none]{->}(0,0)(-3,-3)(3,3)
\psline[linewidth=.05cm](-2,-2)(2,1)
% \psline[linewidth=.05cm](0,0)(3,3.5)
% \psline[linewidth=.05cm](0,0)(3,2)
% \psline[linewidth=.05cm](0,0)(3,0.5)
% \uput[ur](.9,3.5){\Large{$T$}}
\uput[dl](-1,-2){\Large{$Q (-2,-2)$}}
\uput[u](-2,-2.2){\qdisk(0,0){3pt}}
\uput[r](0,-.7){\Large{$S$}}
\uput[r](0,3.2){\Large{$y$}}
\uput[r](3.2,0){\Large{$x$}}
\uput[r](-0.4,-0.2){\Large{$0$}}
\uput[r](0,-1.2){\Large{mid-point}}
\uput[u](0,-.7){\qdisk(0,0){3pt}}
\uput[ur](2,1){\Large{$P (2,1)$}}
\uput[u](2,.8){\qdisk(0,0){3pt}}
% \uput[l](4.8,0){\Large{$x$}}
% \uput[d](0,4.8){\Large{$y$}}
\end{pspicture}
}
\end{center}
\end{figure} 
\westep{Substitute values into the mid-point formula}
\begin{eqnarray*}
X & = & \frac{{x}_{1} + {x}_{2}}{2} \\ 
& = & \frac{-2 + 2}{2} \\ 
& = & 0 \\ 
Y & = & \frac{{y}_{1} + {y}_{2}}{2} \\ 
& = & \frac{-2 + 1}{2} \\ 
& = & -\frac{1}{2} 
\end{eqnarray*}
\westep{Write the answer}
The mid-point is at $F(0;-\frac{1}{2})$
\westep{Confirm answer using distance formula}
Dit kan bewys word dat die afstande vanaf die eindpunte na die middelpunt gelyk is: 
\begin{eqnarray*}
PS & = & \sqrt{{({x}_{1} - {x}_{2})}^{2} + {({y}_{1} - {y}_{2})}^{2}} \\ 
& = & \sqrt{{(0 - 2)}^{2} + {(-0,5 - 1)}^{2}} \\ 
& = & \sqrt{{(-2)}^{2} + {(-1,5)}^{2}} \\ 
& = & \sqrt{4 + 2,25} \\ 
& = & \sqrt{6,25}
\end{eqnarray*}
en
\begin{eqnarray*}
QS & = & \sqrt{{({x}_{1} - {x}_{2})}^{2} + {({y}_{1} - {y}_{2})}^{2}} \\ 
& = & \sqrt{{(0 - (-2))}^{2} + {(-0,5 - (-2))}^{2}} \\ 
& = & \sqrt{{(0 + 2)}^{2}{+(-0,5 + 2)}^{2}} \\ 
& = & \sqrt{{(2)}^{2}{+(-1,5)}^{2}} \\ 
& = & \sqrt{4 + 2,25} \\ 
& = & \sqrt{6,25}
\end{eqnarray*}
Daar kan gesien word dat, $PS=QS$ soos verwag is, dus is $F$ die middelpunt. 
}
\end{wex}




% Die volgende video verskaf ’n opsomming oor die berekening van die middelpunt van ’n lyn.


\vspace{2pt}
\vspace{.1in}
\end{figure}      
%          \section{ Summary \& Exercises}
%     \nopagebreak
%%%             \label{m39167} $ \hspace{-5pt}\begin{array}{cccccccccccc}   \end{array} $ \hspace{2 pt}\raisebox{-5 pt}{\includegraphics[width=0.5cm]{col11306.imgs/summary_www.png}} {(section shortcode: MG10113 )} \par 
%     
%     
% 
\begin{wex}{Calculating the mid-point}{Find the mid-point of line $AB$, given $A(6;2)$ and $B(-5;-1)$.}{
\westep{Draw a sketch}
\begin{center}
 \scalebox{1} % Change this value to rescale the drawing.
{
\begin{pspicture}(0,-3.0667188)(6.7990627,3.1067188)
\rput(3.0,-0.06671875){\psaxes[linewidth=0.04,arrowsize=0.05291667cm 2.0,arrowlength=1.4,arrowinset=0.4,labels=none,ticks=none,ticksize=0.10583333cm]{<->}(0,0)(-3,-3)(3,3)}
\psdots[dotsize=0.12](3.6,0.25328124)
\psdots[dotsize=0.12](5.62,1.2532812)
\psline[linewidth=0.04cm](1.6,-0.74671876)(5.6,1.2532812)
\usefont{T1}{ppl}{m}{n}
\rput(6.184531,0.14328125){$x$}
\usefont{T1}{ppl}{m}{n}
\rput(3.2845314,2.9032812){$y$}
\usefont{T1}{ppl}{m}{n}
\rput(5.844531,1.5032812){$A~(6;2)$}
\usefont{T1}{ppl}{m}{n}
\rput(2.7045312,-0.33671874){$0$}
\psdots[dotsize=0.12](1.62,-0.74671876)
\usefont{T1}{ppl}{m}{n}
\rput(3.7945313,0.66328126){$M(x;y)$}
\usefont{T1}{ppl}{m}{n}
\rput(1.3945312,-0.99671876){$B(-5;-1)$}
\end{pspicture} 
}

\end{center}

From the sketch we estimate that $M$ will lie in Quadrant I, with positive $x$- and $y$-coordinates.
\westep{Assign values to $(x_1;y_1)$ and $(x_2;y_2)$}
Let the mid-point be $M(x;y)$
\begin{equation*}
x_1= 6 \hskip2em y_1=2 \hskip2em x_2=-5 \hskip2em y_2=-1
\end{equation*}
\westep{Write down the mid-point formula}
\begin{equation*}
M(x;y) = \Big(\frac{x_1+x_2}{2};\frac{y_1+y_2}{2}\Big)
\end{equation*}
\westep{Substitute values and simplify}
\begin{equation*}
\Big(\frac{6-5}{2};\frac{2-1}{2}\Big) = \Big(\frac{1}{2};\frac{1}{2}\Big)
\end{equation*}
\westep{Write the final answer}
$M(\frac{1}{2};\frac{1}{2})$ is the mid-point of line $AB$.
}
\end{wex}

\begin{wex}{Using the mid-point formula}{The line joining $C(-2;4)$ and $D(x;y)$ has the mid-point $M(1;-3)$. Find point $D$.}{
\westep{Draw a sketch}
\begin{center}
 \scalebox{1} % Change this value to rescale the drawing.
{
\begin{pspicture}(0,-3.0795312)(6.5590625,3.1067188)
\rput(3.0,-0.06671875){\psaxes[linewidth=0.04,arrowsize=0.05291667cm 2.0,arrowlength=1.4,arrowinset=0.4,labels=none,ticks=none,ticksize=0.10583333cm]{<->}(0,0)(-3,-3)(3,3)}
\psdots[dotsize=0.12](3.56,-0.70671874)
\psdots[dotsize=0.12](5.2,-2.7067187)
\psline[linewidth=0.04cm](1.84,1.3532813)(5.2,-2.7067187)
\usefont{T1}{ppl}{m}{n}
\rput(6.184531,0.14328125){$x$}
\usefont{T1}{ppl}{m}{n}
\rput(3.2845314,2.9032812){$y$}
\usefont{T1}{ppl}{m}{n}
\rput(5.5245314,-2.95){$A~(6;2)$}
\usefont{T1}{ppl}{m}{n}
\rput(2.7045312,-0.33671874){$0$}
\psdots[dotsize=0.12](1.86,1.3332813)
\usefont{T1}{ppl}{m}{n}
\rput(4.3,-0.69671875){$M(x;y)$}
\usefont{T1}{ppl}{m}{n}
\rput(1.6945312,1.5){$B(-5;-1)$}
\psline[linewidth=0.04cm](2.683925,0.52424794)(2.5044158,0.37533036)
\psline[linewidth=0.04cm](2.7355843,0.45123214)(2.556075,0.30231455)
\psline[linewidth=0.04cm](4.483925,-1.6357521)(4.3044157,-1.7846696)
\psline[linewidth=0.04cm](4.5355844,-1.7087679)(4.3560753,-1.8576854)
\end{pspicture} 
}
\end{center}
From the sketch we estimate that $D$ will lie in Quadrant IV, with a positive $x$- and negative $y$-coordinate.
\westep{Assign values to $(x_1;y_1)$ and $(x_2;y_2)$}
Let the coordinates of $C$ be $(x_1;y_1)$ and the coordinates of $D$ be $(x_2;y_2)$.
\begin{equation*}
x_1=-2 \hskip2em y_1=4 \hskip2em x_2=x \hskip2em y_2=y
\end{equation*}
\westep{Write down the mid-point formula}
\begin{equation*}
M(x;y) = \Big(\frac{x_1+x_2}{2}; \frac{y_1+y_2}{2}\Big)
\end{equation*}
\westep{Substitue values and solve for $x$ and $y$}
\begin{equation*}
\begin{array}{rllrl}
1&=\frac{-2+x}{2}&\hskip10em&-3&=\frac{4+y}{2}\\
1\times 2&=-2+x&&-3\times 2&=4+y\\
2&=-2+x&&-6&=4+y\\
x&=2+2&&y&=-6-4\\
x&=4&&y&=-10\\
\end{array}
\end{equation*}
\westep{Write the final answer}
The coordinates of point $D$ are $(4;-10)$.
}
\end{wex}

\begin{wex}{Using the mid-point formula}{Points $E(-1;0)$ , $F(0;3)$ , $G(8;11)$ and $H(x;y)$ are points on the Cartesian plane. Find $H(x;y)$ if $EFGH$ is a parallelogram.}{
\westep{Draw a sketch}
\begin{center}
 \scalebox{1} % Change this value to rescale the drawing.
{
\begin{pspicture}(0,-3.1267188)(7.2090626,3.1667187)
\rput(1.69,-2.1267188){\psaxes[linewidth=0.04,arrowsize=0.05291667cm 2.0,arrowlength=1.4,arrowinset=0.4,labels=none,ticks=none,ticksize=0.10583333cm]{<->}(0,0)(-1,-1)(5,5)}
\psdots[dotsize=0.12](5.33,1.8132813)
\psline[linewidth=0.04cm](1.71,-0.86671877)(5.35,1.8332813)
\psdots[dotsize=0.12](1.69,-0.86671877)
\usefont{T1}{ppl}{m}{n}
\rput(6.8345313,-1.8767188){$x$}
\usefont{T1}{ppl}{m}{n}
\rput(1.9145312,2.9632812){$y$}
\usefont{T1}{ppl}{m}{n}
\rput(5.974531,2.0632813){$G(8;11)$}
\usefont{T1}{ppl}{m}{n}
\rput(1.9345312,-2.3967187){$0$}
\usefont{T1}{ppl}{m}{n}
\rput(5.764531,0.38328126){$H(x;y)$}
\usefont{T1}{ppl}{m}{n}
\rput(0.76453125,-2.4367187){$E(-1;0)$}
\psline[linewidth=0.04cm](1.65,-0.88671875)(1.25,-2.1067188)
\psline[linewidth=0.04cm,linestyle=dashed,dash=0.16cm 0.16cm](1.27,-2.1067188)(4.97,0.5732812)
\psline[linewidth=0.04cm,linestyle=dashed,dash=0.16cm 0.16cm](5.33,1.7932812)(4.93,0.53328127)
\psdots[dotsize=0.12](1.25,-2.1067188)
\usefont{T1}{ppl}{m}{n}
\rput(0.96453124,-0.6767188){$F(0;3)$}
\psline[linewidth=0.02cm](1.25,-2.0867188)(5.33,1.8132813)
\psline[linewidth=0.02cm](1.71,-0.8467187)(4.97,0.59328127)
\usefont{T1}{ptm}{m}{n}
\rput(3.1045313,-0.07671875){$M$}
\end{pspicture} 
}
\end{center}

Method: the diagonals of a parallelogram bisect each other, therefore the mid-point of $EG$
will be the same as the mid-point of $FH$. We must first find the mid-point of $EG$. We can then use it to determine to coordinates of point $H$.
\westep{Assign values to $(x_1;y_1)$ and $(x_2;y_2)$}
Let the mid-point of $EG$ be $M(x;y)$
\begin{equation*}
x_1=-1 \hskip2em y_1=0 \hskip2em x_2=8 \hskip2em y_2=11
\end{equation*}
\westep{Write down the mid-point formula}
\begin{equation*}
 M(x;y) =\Big(\frac{x_1+x_2}{2}; \frac{y_1+y_2}{2}\Big)
\end{equation*}
\westep{Substitue values calculate the coordinates of $M$}
\begin{equation*}
M(x;y) =\Big(\frac{-1+8}{2}; \frac{0+11}{2}\Big) = \Big(\frac{7}{2};\frac{11}{2}\Big)
\end{equation*}

\westep{Use the coordinates of $M$ to determine $H$}
$M$ is also the mid-point of $FH$ so we use the mid-point formula
$M(\frac{7}{2};\frac{11}{2}) = \Big(\frac{x_1+x_2}{2}; \frac{y_1+y_2}{2}\Big)$ and $F(0;3)$ to solve for $H(x;y)$
\westep{Substitute values and solve for $x$ and $y$}

\begin{equation*}
\begin{array}{rllrl}
\frac{7}{2}&=\frac{0+x}{2}&\hskip10em&\frac{11}{2}&=\frac{3+y}{2}\\
7&=x+0&&11&=3+y\\
x&=7&&y&=8\\
\end{array}
\end{equation*}
\westep{Write the final answer}

The coordinates of $H$ are $(7;8)$.
}
\end{wex}

\Tip{Remember to draw sketches!}

\begin{exercises}{}
{
\begin{enumerate}[itemsep=5pt, label=\textbf{\arabic*}. ]
\item Find the mid-points of the following lines:
    \begin{enumerate}[noitemsep, label=\textbf{(\alph*)} ]
    \item $A(2;5)$, $B(-4;7)$
    \item $C(5;9)$, $D(23;55)$
    \item $E(x+2;y-1)$, $F(x-5;y-4)$
    \end{enumerate}

\item The mid-point $M$ of $PQ$ is $(3;9)$. Find $P$ if $Q$ is $(-2;5)$

\item $PQRS$ is a parallelogram with the points $P(5;3)$, $Q(2;1)$ and $R(7;-3)$. Find point $S$.

\item $MPNO$ is a parallelogram with the points $M(5;3)$, $N(2;1)$ and $O(7;-3)$. Find point $P$.
\end{enumerate}

\insertpracticeinfo{4}
}
\end{exercises}    

\summary{}
\begin{itemize}[noitemsep]
\item A point is an ordered pair of numbers written as $(x;y)$.
\item Distance is a measure of the length between two points.
\item The formula for finding the distance between  any two points is: 
\begin{equation*}
d=\sqrt{{({x}_{1}-{x}_{2})}^{2}+{({y}_{1}-{y}_{2})}^{2}}
\end{equation*}
\item The gradient between two points is determined by the ratio vertical change to horizontal change.

\item Die formule om die gradiënt van ’n lyn te vind: 
\begin{equation*}
m=\frac{{y}_{2}-{y}_{1}}{{x}_{2}-{x}_{1}}
\end{equation*}
\item A straight line is a set of points with a constant gradient between any two of the
points.
\item The standard form of the straight line equation is $y=mx+c$.
\item The equation of a straight line can also be written as $y-y_1=mx+c$  or 
\begin{equation*}
\dfrac{y-y_1}{x-x_1}=\dfrac{y_2-y_1}{x_2-x_1}\end{equation*}
\item As twee lyne parallel is, sal hulle dieselfde gradiënt hê.
\item As twee lyne loodreg is op mekaar,
dan het ons $-1$.
\item For horizontal lines the gradient is equal to $0$.
\item For vertical lines the gradient is undefined.
\item Collinear points all lie on the same line.
\item Die formule om die middelpunt van die lyn tussen twee punte te vind: 
\begin{equation*}
M(x;y) = \Big(\frac{{x}_{1}+{x}_{2}}{2};\frac{{y}_{1}+{y}_{2}}{2}\Big)
\end{equation*}
\end{itemize}


\begin{eocexercises}{}
\begin{enumerate}[noitemsep, label=\textbf{\arabic*}. ] 
\item 
Stel die volgende vorms op die Cartesiese vlak voor: 
    \begin{enumerate}[noitemsep, label=\textbf{\alph*}. ] 
    \item Driehoek $DEF$ met $D(1;2)$, $E(3;2)$ en $F(2;4)$ 
    \item Vierhoek $GHIJ$ met $G(2;-1)$, $H(0;2)$, $I(-2;-2)$ en $J(1;-3)$
    \item Vierhoek $MNOP$ met $M(1;1)$, $N(-1;3)$, $O(-2;3)$ en $P(-4;1)$ 
    \item Vierhoek $WXYZ$ met $W(1;-2)$, $X(-1;-3)$, $Y(2;-4)$ en $Z(3;-2)$
    \end{enumerate}
\item 
In die gegewe diagram is die hoekpunte van ’n veelhoek $F(2;0)$, $G(1;5)$, $H(3;7)$ en $I(7;2)$.
\setcounter{subfigure}{0}
\begin{figure}[H] % horizontal\label{m39107*id63458}
\begin{center}
\scalebox{.8}{
\begin{pspicture}(-5,-5)(5.5,5.5)
% \psaxes{<->}(0,0)(5,5)
% \psgrid[gridcolor=lightgray,linecolor=lightgray,subgriddiv=1](0,0)(-1,-1)(7,7)
\psaxes[linewidth=1pt,labels=none,ticks=none]{->}(0,0)(-1,-1)(7,7)
\pspolygon[linewidth=.05cm,fillstyle=solid,fillcolor=lightgray](2,0)(1,5)(3,7)(7,2)(2,0)
% \psline[linewidth=.05cm](0,0)(3,3.5)
% \psline[linewidth=.05cm](0,0)(3,2)
% \psline[linewidth=.05cm](0,0)(3,0.5)
% \uput[ur](.9,3.5){\Large{$T$}}
\uput[d](2,-.2){\normalsize{$F (2;0)$}}
\uput[ul](1.2,5){\normalsize{$G (1;5)$}}
\uput[u](3,7){\normalsize{$H (3;7)$}}
\uput[r](7,2){\normalsize{$I (7;2)$}}
\uput[l](8,0){\Large{$x$}}
\uput[d](0,8){\Large{$y$}}
\end{pspicture}
}
\end{center}
\end{figure}  
    \begin{enumerate}[noitemsep, label=\textbf{\alph*}. ] 
    \item 
    Wat is die lengtes van die sye van $FGHI$?
    \item Is die teenoorstaande sye van $FGHI$ parallel?
    \item Halveer die hoeklyne van $FGHI$ mekaar?
    \item Watter tipe veelhoek is  $FGHI$ Gee redes vir jou antwoord.
    \end{enumerate}
\item 
 ’n Veelhoek
$ABCD$ met hoekpunte $A(3;2)$, $B(1;7)$, $C(4;5)$ en $D(1;3)$ word gegee.
    \begin{enumerate}[noitemsep, label=\textbf{\alph*}. ] 
    \item  Teken die veelhoek.
    \item  Bepaal die sylengtes van die veelhoek.
    \end{enumerate}
\item $ABCD$ is ’n veelhoek met hoekpunte $A(0;3)$, $B(4;3)$, $C(5;-1)$ en $D(-1;-1)$.
    \begin{enumerate}[noitemsep, label=\textbf{\alph*}. ] 
    \item Wys dat:
	\begin{enumerate}[noitemsep, label=\textbf{\roman*}. ] 
	\item $AD = BC$
	\item $AB \parallel DC$
	\end{enumerate}
    \item Benoem $ABCD$?
    \item Wys dat die hoeklyne $AC$ en $BD$ mekaar nie halveer nie.
    \end{enumerate}
\item $P$, $Q$, $R$ en $S$ is die punte $(-2;0)$, $(2;3)$, $(5;3)$, $(-3;-3)$ onderskeidelik.
    \begin{enumerate}[noitemsep, label=\textbf{\alph*}. ] 
    \item Wys dat:
	\begin{enumerate}[noitemsep, label=\textbf{\roman*}. ] 
	\item $SR = 2PQ$
	\item $SR \parallel PQ$
	\end{enumerate}
    \item Bereken:
	\begin{enumerate}[noitemsep, label=\textbf{\roman*}. ] 
	\item $PS$
	\item $QR$
	\end{enumerate}
    \item Watter tipe veelhoek is $PQRS$? Gee redes vir jou antwoord.
    \end{enumerate}
\item $EFGH$ is ’n parallelogram met hoekpunte $E(-1;2)$, $F(-2;-1)$ en $G(2;0)$. Vind die koördinate van H deur gebruik te maak van die feit dat die hoeklyne van ’n parallelogram mekaar halveer.\newline
\item  
$PQRS$ is ’n vierhoek met punte $P(0;-3)$ ; $Q(-2;5)$ ; $R(3;2)$ en $S(3;-2)$  in die Cartesiese vlak.
    \begin{enumerate}[noitemsep, label=\textbf{\alph*}. ] 
    \item Vind die lengte van $QR$.
    \item Bepaal die gradiënt van $PS$.
    \item Vind die middelpunt van $PR$.
    \item Is $PQRS$  ’n parallelogram? Verskaf redes vir jou antwoord.
    \end{enumerate}
\item $A(-2;3)$ en $B(2;6)$ is punte op die Cartesiese vlak. $C(a;b)$ is die middelpunt van $AB$.Vind die waardes
van $a$ en $b$.
\item 
Beskou: Driehoek $ABC$ met hoekpunte $A(1; 3)$ , $B(4;1)$ en $C (6; 4)$:
    \begin{enumerate}[noitemsep, label=\textbf{\alph*}. ] 
    \item Skets die driehoek $ABC$ op die Cartesiese vlak. 
    \item Wys dat $ABC$ is ’n gelykbenige driehoek is.
    \item Bepaal die koördinate van $M$, die middelpunt van $AC$.
    \item Bepaal die gradiënt van $AB$.
    \item Toon aan dat die volgende punte saamlynig is: $A$, $B$ en $D(7;-1)$
    \end{enumerate}
\item In die diagram, $A$ is die punt $(-6;1)$ en $B$ is die punt $(0;3)$
\setcounter{subfigure}{0}
\begin{figure}[H] % horizontal\label{m39107*id63458}
\begin{center}
\scalebox{.8}{
\begin{pspicture}(-5,-5)(5.5,5.5)
% \psaxes{<->}(0,0)(5,5)
% \psgrid[gridcolor=lightgray,linecolor=lightgray,subgriddiv=1](0,0)(-10,-1)(1,7)
\psaxes[linewidth=1pt,labels=none,ticks=none]{->}(0,0)(-10,-1)(1,7)
\psline[linewidth=.05cm](-6,1)(0,3)
% \psline[linewidth=.05cm](0,0)(3,3.5)
% \psline[linewidth=.05cm](0,0)(3,2)
% \psline[linewidth=.05cm](0,0)(3,0.5)
% \uput[ur](.9,3.5){\Large{$T$}}
\uput[d](-6,1){\Large{$A (-6;1)$}}
\uput[d](-5.9,1.2){\qdisk(0,0){3pt}}
\uput[ul](-.2,3){\Large{$B (0;3)$}}
\uput[d](0,3.2){\qdisk(0,0){3pt}}
\uput[l](1.8,0){\Large{$x$}}
\uput[d](0,8){\Large{$y$}}
\end{pspicture}
}
\end{center}
\end{figure} 
    \begin{enumerate}[noitemsep, label=\textbf{\alph*}. ] 
    \item Vind die vergelyking van die lyn $AB$ 
    \item Bereken die lengte van $AB$
% \item  $A'$ is the image of $A$ and $B'$ is the image of $B$. Both these images are obtain by applying the transformation: $(x;y)\to(x - 4;y - 1)$. Give the coordinates of both $A'$ and $B'$
% \item Find the equation of $A'B'$
% \item Calculate the length of $A'B'$
% \item Can you state with certainty that $AA'B'B$ is a parallelogram? Justify your answer.
    \end{enumerate}
\end{enumerate}

\insertpracticeinfo{14}
\end{eocexercises}