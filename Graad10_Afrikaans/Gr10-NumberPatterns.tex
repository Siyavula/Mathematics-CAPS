         \chapter{Getalpatrone}\fancyfoot[LO,RE]{Fokus Area: Wiskunde}

In vorige jare het jy patrone gesien in die vorm van prentjies en getalle. In hierde hoofstuk sal ons meer leer
van die wiskunde van patrone. Patrone is herkenbaar as herhalende reekse wat gevind kan word in die natuur,
vorms, gebeure, groepe van getalle en op baie ander plekke in ons daaglikse lewe. Patrone kan byvoorbeeld
gevind word in die sade van sonneblomme, sneeuvlokkies, geometriese patrone op lappieskomberse en teëls en
reekse getalle soos$0;4;8;12;16;\ldots$.\par 
%english
See if you can spot any patterns in the following sequences: 
\begin{enumerate}[noitemsep, label=\textbf{\arabic*}. ] 
    \item $2;~4;~6;~8;~10;\ldots$
    \item $1;~2;~4;~7;~11;\ldots$
    \item $1;~4;~9;~16;~25;\ldots$
    \item $5;~10;~20;~40;~80;\ldots$
\end{enumerate}

\chapterstartvideo{VMcyv}   
   
% \section{Voorbeelde van getalpatrone}
Reekse van getalle kan interessante patrone bevat. Die volgende is ’n lys van die mees algemene tipes getalpatrone en
hoe hulle gevorm word.
Voorbeelde:
\begin{enumerate}[noitemsep, label=\textbf{\arabic*}. ] 
    \item $1;~4;~7;~10;~13;~16;~19;~22;~25;~\ldots$\\
    Hierdie reeks het ’n verskil van $3$ tussen al die getalle.\\Die patroon word gevorm deur elke keer
     $3$ by te tel by die vorige getal.
    \item $3;~8;~13;~18;~23;~28;~33;~38;~\ldots$\\
     Hierdie reeks het ’n verskil van $5$ tussen al die getalle. \\Die patroon word gevorm
    deur elke keer $5$ by te tel by die vorige getal.
    \item $2;~4;~8;~16;~32;~64;~128;~256;\ldots$\\
    Hierdie reeks het ’n faktor van $2$ tussen al die getalle. \\Die volgende getal in
    die reeks word gevorm deur die vorige een met $2$ vermenigvuldig.
    \item $3;~9;~27;~81;~243;~729;~2187;\ldots$\\
    Hierdie reeks het ’n faktor van $3$ tussen al die getalle. \\Die volgende getal in
    die reeks word gevorm deur die vorige een met  $3$te vermenigvuldig.
\end{enumerate}
     

% \subsection*{Spesiale reekse}
% 
% \subsubsection*{Driehoeksgetalle}
% 
% % \psset{xunit=1.0cm,yunit=1.0cm,algebraic=true,dotstyle=o,dotsize=3pt 0,linewidth=0.8pt,arrowsize=3pt 2,arrowinset=0.25}
% \begin{pspicture*}(-5.3,-3.74)(7.32,6.48)
% \rput[tl](-5.1,0.54){$1$}
% \rput[tl](-2.6,0.54){$3$}
% \rput[tl](0.94,0.5){$6$}
% \rput[tl](5.42,0.54){$10$}
% \psdots[dotstyle=*](0,1)
% \psdots[dotstyle=*](1,1)
% \psdots[dotstyle=*](2,1)
% \psdots[dotstyle=*](1.52,2)
% \psdots[dotstyle=*](0.6,2)
% \psdots[dotstyle=*](1.14,2.98)
% \psdots[dotstyle=*](-5,1)
% \psdots[dotstyle=*](-3,1)
% \psdots[dotstyle=*](-2,1)
% \psdots[dotstyle=*](-2.46,2)
% \psdots[dotstyle=*](7,1)
% \psdots[dotstyle=*](4,1)
% \psdots[dotstyle=*](5,1)
% \psdots[dotstyle=*](6,1)
% \psdots[dotstyle=*](5.46,2)
% \psdots[dotstyle=*](4.48,2)
% \psdots[dotstyle=*](6.32,2)
% \psdots[dotstyle=*](6,3)
% \psdots[dotstyle=*](5,3)
% \psdots[dotstyle=*](5.54,3.96)
% \end{pspicture*}
% $\ldots$\\
% \\
% $1;~3;~6;~10;~15;~21;~28;~36;~45;\ldots$\\
% \\
% Hierdie reekse word gevorm deur ’n patroon van kolletjies wat ’n driehoek vorm. Deur nog ’n ry kolletjies by (waar die elke nuwe ry een meer kolletjie bevat as die vorige een) en die kolletjies te tel, is dit moontlik om
% die te voeg volgende getal in die reeks te vind.\par 
% 
% \subsubsection*{Vierkantsgetalle}
% $1;~4;~9;~16;~25;~36;~49;~64;~81;\ldots$\\
% Die waarde van ’n term in die reeks word gevind deur die posisie (pleknommer in die ry) te kwadreer.\\
% Die $2^{\mathrm{de}}$ getal is ${2}^{2} = 4$.
% Die $7^{\mathrm{de}}$ getal is ${7}^{2} = 49$. Wat sal die $10^{\mathrm{de}}$ getal wees?
%             
% \subsubsection*{Derdemagsgetalle}
% $1;~8;~27;~64;~125;~216;~343;~512;~729;\ldots$\\
% Die waarde van ’n term in die reeks word gevind deur die posisie tot die derde mag te verhef.\\
% Die $2^{\mathrm{de}}$ getal is $2^{3}=8$.
% Die $7^{\mathrm{de}}$ getal is $7^{3}=343$. Wat sal die $10^{\mathrm{de}}$ getal wees?
% 
% \subsubsection*{Fibonacci Getalle}
% $0;~1;~1;~2;~3;~5;~8;~13;~21;~34;...$\\
% Die waarde van ’n term in die reeks word gevind deur die vorige twee getalle in die reeks bymekaar te tel.\\
% Die $4^{\mathrm{de}}$ word gevind deur die vorige twee getalle in die reeks bymekaar te tel $1+1=2$.\\
% Die $9^{\mathrm{de}}$  word gevind deur die twee
% getalle voor die 2 in die reeks bymekaar te tel $8+13=21$.\\
% Die volgende getal in die reeks sal $21+34=55$ wees.
% Kan jy die volgende paar getalle vind?
% 
% \mindsetvid{Number Patterns}{MG10023}

\begin{wex}{Studeertafels}{Gestel jy en $3$ vriende besluit om te
studeer vir wiskunde, en dat julle om ’n vierkantige tafel sit. ’n Paar
minute later sluit $2$ ander vriende by julle aan en hulle wil kom sit. Om
sitplek te kry vir hulle, besluit julle om ’n tafel te skuif en dit langs julle
tafel te sit sodat daar genoeg sitplek is vir die $6$ van julle. Daarna besluit
nog $2$ van jou vriende om by julle aan te sluit en julle skuif ’n derde tafel
sodat daar genoeg plek is vir $8$ van julle. \\
\\
Ondersoek hoe die aantal mense om die tafels verband hou met die
aantal tafels. Is daar 'n patroon?
\begin{figure}[H]
\begin{center}
\begin{pspicture}(0,0)(9.6,1.8)
%\psgrid[gridcolor=lightgray]
\psframe(0.4,0.4)(1.4,1.4)
\psframe(0,0.6)(0.2,1.2)
\psframe(0.6,0)(1.2,0.2)
\psframe(0.6,1.6)(1.2,1.8)
\psframe(1.6,0.6)(1.8,1.2)
\rput(2.4,0){\psframe(0.4,0.4)(1.4,1.4)
\psframe(0,0.6)(0.2,1.2)
\psframe(0.6,0)(1.2,0.2)
\psframe(0.6,1.6)(1.2,1.8)}
\rput(3.4,0){\psframe(0.4,0.4)(1.4,1.4)
\psframe(0.6,0)(1.2,0.2)
\psframe(0.6,1.6)(1.2,1.8)
\psframe(1.6,0.6)(1.8,1.2)}
\rput(5.8,0){\psframe(0.4,0.4)(1.4,1.4)
\psframe(0,0.6)(0.2,1.2)
\psframe(0.6,0)(1.2,0.2)
\psframe(0.6,1.6)(1.2,1.8)}
\rput(6.8,0){\psframe(0.4,0.4)(1.4,1.4)
\psframe(0.6,0)(1.2,0.2)
\psframe(0.6,1.6)(1.2,1.8)}
\rput(7.8,0){\psframe(0.4,0.4)(1.4,1.4)
\psframe(0.6,0)(1.2,0.2)
\psframe(0.6,1.6)(1.2,1.8)
\psframe(1.6,0.6)(1.8,1.2)}
\end{pspicture}\\
\begin{caption*}Twee ekstra mense kan sit vir elke tafel wat hulle bysit.\end{caption*}
\label{fig:mp:s:arithmetictables}
\end{center}
\end{figure}
}
{
\westep{Maak 'n tabel en kyk of 'n patroon sigbaar word}
\begin{center}
\begin{tabular}{|c|l|}\hline
\textbf{Aantal tafels}, $n$ & \textbf{Aantal mense wat sitplek het}\\\hline
\hline 1 & $4 = 4$\\
\hline 2 & $4 + 2 = 6$\\
\hline 3 & $4 + 2 + 2 = 8$\\
\hline 4 & $4 + 2 + 2 + 2 = 10$ \\
\hline \vdots & \qquad \qquad \quad \vdots\\
\hline $n$ & $4 + 2 + 2 + 2 + \ldots + 2 $\\
\hline
\end{tabular}
\end{center}
\westep{Beskryf die patroon}
Ons kan sien dat met $3$ tafels is daar plek vir $8$ mense, om $3$ tafels, met $4$  tafels
is daar plek vir $10$  mense ens. Ons begin met $4$ mense en voeg
elke keer $2$ mense by.So, vir elke tafel wat bygevoeg word, is daar
sitplek vir nog $2$ mense.}

\end{wex}

\section{Beskryf reekse}


 
Ons gebruik die volgende notasie om die terme in 'n getalpatroon aan te dui:\\
\\
Die $1^{ste}$ term van 'n reeks is $T_{1}$\\
Die $4^{de}$  term van 'n reeks is $T_{4}$\\
Die $10^{de}$  term van 'n reeks is $T_{10}$\\
\\
%english
The general term is often expressed as the ${n}^{th}$ term and is written as ${T}_{n}$. \\

\\n Reeks hoef nie ’n patroon te volg nie, maar wanneer dit wel ’n patroon het, kan ons
dit gewoonlik met ’n formule beskryf om die ${n}^{de}$ - term,${T}_{n}$ te bereken. 
Byvoorveeld, beskou die volgende lineêre reeks: 
     
\begin{equation*}
1;3;5;7;9;\ldots
\end{equation}
The ${n}^{de}$ word gegee deur die formule
\begin{equation*}
  \begin{array}{cc}\hfill {T}_{n}={2n-1}\end{array}
\end{equation}
Jy kan dit kontroleer deur waardes in die formule in te stel:
\begin{equation*}
    \begin{array}{ccc}\hfill {T}_{1}& =& 2(1)-1 = 1 \hfill \\ 
    \hfill {T}_{2}& =& 2(2)-1 = 3\hfill \\
    \hfill {T}_{3}& =& 2(3)-1 = 5\hfill \\
    \hfill {T}_{4}& =& 2(4)-1 = 7\hfill \\
    \hfill {T}_{5}& =& 2(5)-1 = 9\hfill \\
\end{array}
\end{equation}

As ons die verband tussen die posisie van 'n term (plek in die ry) en die waarde van die term vasgestel het. kan ons die patroon beskryf en enige term in die ry vind
\mindsetvid{Number Patterns}{MG10025}
\subsection*{Konstante verskil}
Ons kan ’n konstante verskil tussen die terme bepaal vir sekere patrone. Dit word genoem die \textit{konstante verskil}.

\Definition{Konstante verskil} {Die konstante verskil is die verskil tussen opeenvolgende terme en word aagedui met die
letter $d$. } 

Byvoorbeeld, beskou die reeks $10;7;4;1;\ldots$
 Om die gemeenskaplike verskil te vind, trek ons die betrokke term af van die volgende term:

\begin{equation*}
    \begin{array}{ccl}d &=& T_{2} - T_{1}\\
      & =& 7-10\\& =&-3\end{array}
\end{equation}

Toets enige ander twee opeenvolgende terme:
\begin{equation*}
    \begin{array}{ccl}d &=& T_{4} - T_{3}\\
    & =& 1-4\\& =&-3\end{array}
\end{equation}
Ons sien dat $d$ konstant is.
\\
\textbf{Let wel:} $d=T_{2}-T_{1}$, nie $T_{1} -T_{2}$.
       
\begin{wex}{Studeertafel voortgesit...}{Soos voorheen, studeer jy
en $3$ vriende wiskunde, en julle sit rondom ’n vierkantige tafel. ’n Paar
minute later besluit $2$ ander vriende om by julle aan te sluit
en julle sit ’n ekstra tafel by sodat al $6$ van julle kan sit. Weereens besluit
nog $2$ van jou vriende om by julle aan te sluit en julle skuif ’n derde tafel
sodat daar genoeg plek is vir $8$ van julle soos in die skets.\\
\\
\begin{minipage}{\textwidth}
\begin{enumerate}[noitemsep, label=\textbf{\arabic*}.]
\item Vind ’n wiskundige uitdrukking vir die aantal mense wat om $n$ tafels kan sit.
\item Gebruik dan die algemene formule om te bepaal hoeveel mense om
$12$ tafels kan sit.
\item Hoeveel tafels is nodig is sodat $20$ mense kan sit?
\end{enumerate}
\end{minipage}
}{

\begin{figure}[H]
\begin{center}
\begin{pspicture}(0,0)(9.6,1.8)
%\psgrid[gridcolor=lightgray]
\psframe(0.4,0.4)(1.4,1.4)
\psframe(0,0.6)(0.2,1.2)
\psframe(0.6,0)(1.2,0.2)
\psframe(0.6,1.6)(1.2,1.8)
\psframe(1.6,0.6)(1.8,1.2)
\rput(2.4,0){\psframe(0.4,0.4)(1.4,1.4)
\psframe(0,0.6)(0.2,1.2)
\psframe(0.6,0)(1.2,0.2)
\psframe(0.6,1.6)(1.2,1.8)}
\rput(3.4,0){\psframe(0.4,0.4)(1.4,1.4)
\psframe(0.6,0)(1.2,0.2)
\psframe(0.6,1.6)(1.2,1.8)
\psframe(1.6,0.6)(1.8,1.2)}
\rput(5.8,0){\psframe(0.4,0.4)(1.4,1.4)
\psframe(0,0.6)(0.2,1.2)
\psframe(0.6,0)(1.2,0.2)
\psframe(0.6,1.6)(1.2,1.8)}
\rput(6.8,0){\psframe(0.4,0.4)(1.4,1.4)
\psframe(0.6,0)(1.2,0.2)
\psframe(0.6,1.6)(1.2,1.8)}
\rput(7.8,0){\psframe(0.4,0.4)(1.4,1.4)
\psframe(0.6,0)(1.2,0.2)
\psframe(0.6,1.6)(1.2,1.8)
\psframe(1.6,0.6)(1.8,1.2)}
\end{pspicture}\\
\begin{caption*}Twee ekstra mense kan sit vir elke tafel wat hulle bysit \end{caption*}
\label{fig:mp:s:arithmetictables2}
\end{center}
\end{figure}

\westep{Stel 'n tabel op om die patroon te sien}
\begin{center}
\begin{tabular}{|c|l|c|}\hline
\textbf{Aantal tafels}, $n$ & \textbf{Aantal mense wat kan sit} & \textbf{Formule}\\
\hline 1 & $4 = 4$ & $= 4 + 2 (0)$ \\
\hline 2 & $4 + 2 = 6$ & $= 4 + 2 t(1)$ \\
\hline 3 & $4 + 2 + 2 = 8$ & $= 4 + 2 (2)$ \\
\hline 4 & $4 + 2 + 2 + 2 = 10$ & $= 4 + 2(3)$ \\
\hline \vdots & \qquad \qquad \quad \vdots & \vdots \\
\hline $n$ & $4 + 2 + 2 + 2 + \ldots + 2 $ & \: \: \: $= 4 + 2 (n-1)$\\
\hline
\end{tabular}
\end{center}

\westep{Beskryf die patroon}
Die aantal mense wat rondom $n$ tafels kan sit, is
  $T_n=4 + 2(n-1)$.

\westep{Bereken die $12^{de}$ term, met ander woorde, vind $T_{12}$, as $n = 12$}

\begin{eqnarray*}
T_{12} &=& 4 + 2  (12 - 1) \\
&=& 4 + 2(11) \\
&=& 4 + 22 \\
&=& 26
\end{eqnarray*}
Daarom kan $26$ mense kan rondom $12$ tafels sit .
\westep{Bereken die aantal tafels wat nodig is vir $20$ mense. Met ander woorde, vind $n$ as $T_n=20$}
\begin{eqnarray*}
T_n &=& 4 + 2  (n - 1) \\
20 &=& 4 + 2  (n - 1) \\
20 - 4 &=& 2  (n - 1) \\
\frac{16}{2} &=& n - 1 \\
8 + 1 &=& n \\
n &=& 9
\end{eqnarray*}
Daarom is $9$ tafels nodig sodat $20$ mense kan sit.

}
\end{wex}

Dit is belangrik om te let op die verskil tussen $n$ en ${T}_{n}$. $n$ kan gesien word as ’n plekhouer, terwyl ${T}_{n}$ die
waarde is by die plek wat "gehou" word deur $n$. Soos in ons "Studeertafel" voorbeeld, kan  $4$ mense rondom die
eerste tafel sit. Dus, by plek $n=1$,  is die waarde van ${T}_{1}=4$ ensovoorts:\par 
    
\begin{center}
\begin{tabular}{|c|c|c|c|c|c|}
\hline $n$ & $1$ & $2$ & $3$ & $4$ & $\ldots$ \\
\hline $T_n$ & $4$ & $6$ & $8$ & $10$ & $\ldots$ \\
\hline
\end{tabular}
\end{center}


\begin{exercises}{Finding the pattern}
{ 

\begin{enumerate}[noitemsep, label=\textbf{\arabic*}. ] 
\item Skryf die volgende drie terme neer in elk van die reekse:

  \begin{enumerate}[noitemsep, label=\textbf{(\alph*)} ] 
  \item $5;~15;~25\ldots$
  \item $-8;-3;~ 2 \ldots$
  \item $30;~27;~24\ldots$
  \end{enumerate}
 \item Hieronder is die algemene formules gegee vir ’n paar reekse. Bereken die terme wat weggelaat is.
  \begin{enumerate}[noitemsep, label=\textbf{(\alph*)}] 
  \item $0;3;\ldots;~15;~24$\hspace{2.2cm}$T_{n}={n}^{2}-1$
  \item $3;2;~1;~0;\ldots;-2$\hspace{2cm}$T_{n}=-n+4$
  \item $-11;\ldots;-7;\ldots;-3$\hspace{1.5cm}$T_{n}=-13+2n$
  \end{enumerate}
     
\item Vind die algemene formule vir elk van die volgende reekse en vind dan ${T}_{10}$, ${T}_{50}$ en ${T}_{100}$
  \begin{enumerate}[noitemsep, label=\textbf{(\alph*)} ] 
  \item $2;~5;~8;~11;~14;\ldots$
  \item $0;~4;~8;~12;~16;\ldots$
  \item $2;-1;-4;-7;-10;\ldots$
  \end{enumerate}
\end{enumerate}


\insertpracticeinfo{3}

}%\End of exercise
\end{exercises}


\section{Patrone en bewerings}



In wiskunde is ’n bewering ’n wiskundige stelling wat lyk of dit waar is, maar wat nog nie formeel as waar bewys
is nie. 
'n Bewering (of vermoede) kan gesien word as 'n wiskundige se manier om te s\^e: ``Ek glo dit is waar, maar ek het nog bewyse nie.''
’n Bewering kan gesien word as ’n intelligente raaiskoot of idee wat moontlik ’n patroon kan wees.\par
Byvoorbeeld, maak ’n bewering oor die getal wat sal volg, gebaseer op die patroon $2;~6;~11;~17;\ldots$. Die getalle vermeerder met $4$, dan $5$, en dan $6$.
Bewering: Die volgende getal sal vermeerder met $7$, So ons verwag dat die volgende getal $17+7=24$ sal wees.\\
\mindsetvid{Number Patterns}{MG10026}

\begin{wex}{Optelling van ewe en onewe getalle }{
\begin{minipage}{\textwidth}
\begin{enumerate}[noitemsep, label=\textbf{\arabic*}. ] 
\item Ondersoek die soort getal wat jy kry as jy 'n ewe en 'n onewe getal bymekaar tel.
\item Druk jou antwoord in woorde uit as 'n bewering.
\item Gebruik algebra om hierdie bewering te bewys.
\end{enumerate}
\end{minipage}
}
{   
\westep{Ondersoek eers 'n paar voorbeelde}  
  \begin{equation*}
    \begin{array}{ccc}\hfill 23 + 12 &=& 35\\ 148 + 31 &=& 179\\ 11 +200 &=& 211 \end{array}
   \end{equation}

\westep{Maak 'n bewering}  
        Die som van enige onewegetal en enige ewegetal is altyd onewe.  
\westep{Beskryf algebra\"ies}
Druk die ewe getal uit as $2x$.\\
Druk die onewe getal uit as $2y-1$.\\

\begin{equation*}
    \begin{array}{ccc}\hfill \mbox{Som} &=& 2x + (2y-1)\\  &=& 2x + 2y - 1\\ &=& (2x + 2y) - 1 \\ &=& 2(x+y) -1 \end{array}
\end{equation}
Dis duidelik dat $2(x+y)$ 'n ewe getal is. Dus sal $2(x+y)-1$ onewe wees.\\
Dus ons bewering is waar.    
}
\end{wex}



\begin{wex}{Vermenigvuldig 'n twee-syfer getal met $11$ }{
\begin{minipage}{\textwidth}
\begin{enumerate}[noitemsep, label=\textbf{\arabic*}. ] 
\item Beskou die volgende voorbeelde van vermenigvuldiging van enige twee-syfer getal met $11$. Watter patrron sien jy?
\begin{equation*}
    \begin{array}{ccc}\hfill 11 \times 42 &=& 462\\ 11 \times 71 &=& 781\\ 11 \times 45 &=& 495\end{array}
\end{equation}

\item Druk jou antwoord uit as 'n bewering.
\item Kan jy 'n teen-voorbeeld kry, een wat jou bewering weerl\^e? Indien wel, heroonweeg jou bewering.
\item Gebruik algebra om jou bewering te bewys.
\end{enumerate}
\end{minipage}
}
{   
\westep{Vind die patroon}
Ons sien in die antwoord, die middelste syfer is die som van die syfers in die oorspronklike twee-syfer getal.  \\

\westep{Maak 'n bewering}
DIe middelste getal van die produk is die som van syfers van die twee-syfer getal wat met $11$ vermenigvuldig word.\\
   
\westep{Heroonweeg die bewering}
Ons sien 
\begin{equation*}
    \begin{array}{ccc}\hfill 11 \times 67 &=& 737\\ 11 \times 56 &=& 638\end{array}
\end{equation}
Ons pas ons bewering aan om te s\^e dat dit net waar is indien die syfersom kleiner is as $10$.\\

\westep{Beskryf die bewering algebra\"ies}
Enige twee-syfer getal kan geskryf word as $10a+b$. Byvoorbeeld $34= 10(3)+4$. \\
Enige drie-syfer getal kan geskryf word as $100a+10b+c$. Byvoorbeeld $582=100(5)+ 10(8) +2$.\\
\begin{equation*}
    \begin{array}{ccl}\hfill 11 \times (10x+y) &=& 110x + 11y\\ &=&(100x + 10x) + 10y + y\\&=& 100x + (10x+ 10y) + y\\&=& 100x + 10(x+y) + y\end{array}
\end{equation} 

Hieruit sien ons die middelste syfer van 'n drie-syfer getal is gelyk aan die som van die twee-syfers $x$ en $y$.
}
\end{wex}



\summary{VMcyz}

\begin{itemize}[noitemsep]
\item Daar is ’n hele paar spesiale reekse van getalle: 
    \begin{itemize}[noitemsep]
    \item Driehoeksgetalle  $1;~3;~6;~10;~15;~21;~28;~36;~45;~...$
    \item Vierkantsgetalle $1;~4;~9;~16;~25;~36;~49;~64;~81;\ldots$
    \item Derdemagsgetalle $1;~8;~27;~64;~125;~216;~343;~512;~729;\ldots$
%     \item Fibonacci Getalle  $0;~1;~1;~2;~3;~5;~8;~13;~21;~34;\ldots$
    \end{itemize}
\item Die algemene ${n}^{th}$ term word geskryf as ${T}_{n}$.
\item Die konstante verskil ($d$) is die verskil tussen twee opeenvolgende terme.
\item Ons kan 'n algemene forule bepaal vir elke getalpatroon en dit gebruik om enige term in die patroon te vind.
\item 'n Bewering is 'n vermoede wat jy aanneem om waar te wees maar nog nie bewys het nie.
\end{itemize}
            

\begin{eocexercises}{Hoofstukoefeninge}
\begin{enumerate}[noitemsep, label=\textbf{\arabic*}. ] 
\item Vind die $6^{de}$ term vir die volgende reekse:
    \begin{enumerate}[noitemsep, label=\textbf{(\alph*)} ] 
    \item $4;~13;~22;~31;\ldots$
    \item $~5;~2;~-1;~-4;\ldots$
    \item $7,4;~ 9,7; ~12; ~14,3; \ldots$
    \end{enumerate}


\item Vind die algemene term vir die volgende reekse:
    \begin{enumerate}[noitemsep, label=\textbf{(\alph*)} ] 
    \item$3;~7;~11;~15;\ldots$
    \item $-2;~1;~4;~7;\ldots$
    \item $11;~15;~19;~23;\ldots$
    \item $\dfrac{1}{3};~ \dfrac{2}{3};~ 1; ~1\dfrac{1}{3};\ldots$
    \end{enumerate}

\item Die sitplekke in ’n gedeelte van ’n sportstadion kan so gerangskik word dat die eerste ry $15$ sitplekke het,
die tweede ry $19$ sitplekke, die derde ry $23$ sitplekke, ens. Bereken hoeveel sitpleke is daar in ry $25$.
\item ’n Enkele vierkant kan gemaak word van $4$ vuurhoutjies. Om twee vierkante langs mekaar te maak het $7$ vuurhoutjies nodig, om drie vierkante langs mekaar in ’n ry te maak het $10$ vuurhoutjies nodig. Bepaal:
    \begin{enumerate}[noitemsep, label=\textbf{(\alph*)} ] 
    \item die eerste term
    \item die konstante verskil
    \item die algemene formule
    \item hoeveel vuurhoutjies benodig word om $25$ vierkante langs mekaar te maak
    \end{enumerate}

\setcounter{subfigure}{0}
\begin{figure}[H] 
\begin{center}
\begin{pspicture}(0,0)(8,2)
%\psgrid[gridcolor=gray]
\def\match{\psline(0,0)(2,0)\psellipse*(1.8,0)(0.2,0.1)}
\rput(0,0){\match}
\rput{90}(2,0){\match}
\rput{180}(2,2){\match}
\rput{270}(0,2){\match}
\rput(2,0){\rput(0,0){\match}
\rput{90}(2,0){\match}
\rput{180}(2,2){\match}}
\rput(4,0){\rput(0,0){\match}
\rput{90}(2,0){\match}
\rput{180}(2,2){\match}}
\rput(6,0){\rput(0,0){\match}
\rput{90}(2,0){\match}
\rput{180}(2,2){\match}}
\end{pspicture}

\vspace{2pt}
\vspace{.1in}
\end{center}
\end{figure}       

\item Jy wil begin om geld te spaar, maar omdat jy dit nog nooit gedoen het nie, besluit jy om stadig te begin.
Aan die einde van die eerste week sit jy R$~5$ in jou bankrekening, aan die einde van die tweede week R$~10$ en aan die einde van die derde week R$~15$. Na hoeveel weke sit jy
 R$~50$ in jou bankrekening?

\item ’n Horisontale lyn kruis ’n tou op vier punte en deel die tou op in $4$ dele, soos hieronder gewys word.
\setcounter{subfigure}{0}
\begin{figure}[H] 
\begin{center}

\begin{pspicture}(-1,-2)(6,2)
%\psgrid[gridcolor=gray]
\psplot[xunit=0.00556]{90}{810}{x sin}
\psline[linestyle=dashed](-1,0)(6,0)
\psdots[dotsize=5pt](1,0)(2,0)(3,0)(4,0)
\rput(0.5,1.5){\psframebox{1}}
\rput(1.5,-1.5){\psframebox{2}}
\rput(2.5,1.5){\psframebox{3}}
\rput(3.5,-1.5){\psframebox{4}}
\rput(4.5,1.5){\psframebox{5}}
\end{pspicture}
\end{center}
\end{figure}  
     
As die tou $19$ keer gekruis word deur ewewydige lyne en elke lyn kruis die tou $4$ keer op verskillende
plekke, bereken in hoeveel dele die tou opgedeel word.

 
\item Beskou die volgende patroon: 
  \begin{equation*}
    \begin{array}{ccl}\hfill 9+16&=& 25\\ 9+28 &=& 37\\9+43&=& 52\end{array}
  \end{equation} 

  \begin{enumerate}[noitemsep, label=\textbf{(\alph*)} ] 
  \item Watter patroon sien jy?
  \item Maak 'n bewering en druk dit uit in woorde.
  \item Veralgemeen jou bewering en druk dit algebra\"ies uit.
  \item Bewys dat jou bewering waar is.
  \end{enumerate}

\end{enumerate}


\insertpracticeinfo{7}

\end{eocexercises}
