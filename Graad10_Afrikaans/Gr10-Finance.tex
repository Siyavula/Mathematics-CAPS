\chapter{Finance and Growth}

\section{ Being Interested in Interest}

Welkom die by die Graad 10 hoofstuk oor Finansiële Wiskunde. Ons gaan wiskundige vaardighede gebruik wat
jy heel waarskynlik nou gaan nodig kry.\par

As jy R~$1 000$, het, kan jy dit in jou beursie hou, of jy kan dit deponeer in ’n bankrekening. As dit in jou beursie bly,
kan jy dit enige tyd uitgee wanneer jy wil. As die bank daarna kyk vir jou, kan hulle dit spandeer met die doel om
wins daaruit te maak. Die bank "betaal" jou gewoonlik om geld te deponeer in ’n rekening om jou aan te moedig
om met hulle sake te doen. Hierdie betaling is soos ’n beloning, wat vir jou die rede is om die geld liewer in die
bank te los vir ’n rukkie as om die geld in jou beursie te hou.\par Ons noem hierdie beloning "rente".
As jy geld in ’n bankrekening deponeer, is jy eintlik besig om jou geld aan die bank te leen - en jy kan verwag om
rente te ontvang van die bank. Net so, as jy geld leen van ’n bank (of van ’n afdelingswinkel, of ’n motorhandelaar,
byvoorbeeld) dan kan jy verwag om rente te betaal op die lening. Dit is die prys vir die leen van geld.
.\par

Die idee is eenvoudig, en tog is dit die kern van die wêreld van finansies. Boekhouers, rekenmeesters en
bankiers, byvoorbeeld, bestee hulle hele professionele loopbaan deur te werk met die gevolge van rente op
finansiële aangeleenthede.\par

\section{Enkelvoudige Rente}
\Definition{Enkelvoudige Rente}{Enkelvoudige rente is wanneer jy rente verdien op die aanvanklike bedrag wat jy belê het,
maar nie rente op rente nie.}
  
As ’n maklike voorbeeld van enkelvoudige rente, dink hoeveel jy sal kry deur R~$1~000$ te belê vir 1 jaar by ’n bank
wat vir jou enkelvoudige rente teen $5\%$ per jaar gee. Aan die einde van die jaar sal jy rente ontvang van:\par
\begin{align*}
    \mathrm{Interest} &= \text{R}~1~000 \times 5\%\\
    &= \text{R}~1~000 \times \frac{5}{100}\\
    &= \text{R}~1~000 \times 0,05\\
    &= \text{R}~50
\end{align*}

Dus, met ’n “aanvangsbedrag" van R~$1~000$ aan die begin van die jaar, sal jou “eindbedrag" aan die einde van die
jaar dan wees:\par 
\begin{align*}
    \mathrm{Closing~Balance} &= \mathrm{Opening~Balance + Interest}\\
    &= \text{R}~1~000 + \text{R}~50\\
    &= \text{R}~1~050
\end{align*}

Ons noem soms die aanvangsbedrag in finansiële wiskunde die \textsl{Hoofsom ("Principal amount")}, wat afgekort word as $P$ (R~$1~000$ in die voorbeeld). Die rentekoers vir die tydsinterval word gewoonlik as ’n persentasie aangedui deur $i$ ($5\%$ in die voorbeeld),  en die bedrag aan rente verdien (in terme van rand) word aangedui deur $I$ (R~$50$ in die voorbeeld).\par 

So, ons sien dat:
        
\begin{align}
    I = P \times i
\end{align}

en,

\begin{align}
    \mathrm{Eindbedrag} &= \mathrm{Aanvangsbedrag + Rente} \nonumber\\
    &= P + I \nonumber\\
    &= P + (P \times i)\nonumber\\
    &= P(1 + i)
\end{align}

\Tip{Remember that percentage is a number written over the denominator of $100$. In other words, $12\%$ can be written as $\frac{12}{100}$ --- and thus $12\%$ can be changed to the decimal $0,12$.}


The above calculations give us a good idea of what the simple interest formula looks like, however, the example talks
about an investment that only lasts one year. If the investment or loan is over a longer period, we need to take this
into account. We use the symbol ‘n’ to indicate time period and this value must always be written in years.\par

The formula used to calculate simple interest is:
\Tip{Jaarlikse rentekoerse beteken die koers word bereken oor ’n periode van ’n jaar, (per annum) = per jaar.}

\Identity{enkelvoudige rente}{
    \begin{eqnarray}
	\label{FG:SI}
	A &=& P (1 + i \cdot n)\\
	\text{Waar:~} \nonumber\\
	A &=& \text{Eindbedrag} \nonumber\\
	P &=& \text{Aanvangsbedrag} \nonumber\\
	i &=& \text{Rentekoers per tydsinterval} \nonumber\\
	n &=& \text{aantal tydsintervalle} \nonumber
    \end{eqnarray}
}




\begin{wex}{Calculating interest on a deposit}{
    As ek R~$1 000$ vir 3 jaar deponeer
in ’n spesiale bankrekening wat $7\%$ per jaar enkelvoudige rente gee,
hoeveel geld sal ek aan die einde van hierdie tydperk hê?}{

    \westep{Ons weet vanuit dat}
    \begin{align*}
	A &= ~?\\
	P &= 1~000\\
	i &= 0,07\\
	n &= 3
    \end{align*}
    
    \westep{Write down the formula}
    \begin{align*}
	A &= P(1 + i \cdot n)
    \end{align*}

    \westep{Substitute the values and solve for the remaining variable}
    \begin{align*}
	A &= 1~000(1 + 0,07 \times 3)\\
	  &= 1~210
    \end{align*}

    \westep{Explain your answer}
    Die eindbedrag nadat R1 000 vir 3 jaar belê is teen ’n rentekoers
van $7\%$ per jaar, is R~$1~210$.
    }
\end{wex}


\begin{wex}{Calculating interest on a loan}{
    Sarah borrows R~$5~000$ from her neighbour at an agreed simple interest rate of $12,5\%$ p.a. She will pay back the loan in one lump sum at the end of 2 years. How much will she have to pay her neighbour?}{

    \westep{Write down the known variables}
    \begin{align*}
	A &= ~?\\
	P &= 5~000\\
	i &= 0,125\\
	n &= 2
    \end{align*}

    \westep{Write down the formula}
    \begin{align*}
	A &= P(1 + i \cdot n)
    \end{align*}

    \westep{Substitute the values and solve for the remaining variable}
    \begin{align*}
	A &= 5~000(1 + 0,125 \times 2)\\
	  &= 6~250
    \end{align*}

    \westep{Explain your answer}
    At the end of 2 years, Sarah will pay her neighbour R~$6~250$.
    }
\end{wex}


We can use the simple interest formula to discover pieces of missing information. For instance, if we have an amount of money that we want to invest for a set amount of time to achieve a certain outcome, we can rearrange the variables to solve for the interest rate we need to invest the money at. The same principles apply to find the length of time we would need to invest the money for, if we knew the principal and accumulated amount and the interest rate.


\begin{wex}{Determining the investment period to achieve a goal amount}{
    As ek  R~$30~000$ deponeer in ’n spesiale
bankrekening wat $7,5\%$ enkelvoudige rente betaal, vir hoeveel jaar moet
ek hierdie bedrag belê om die bedrag van R~$45~000$ te genereer?}{

    \westep{Write down the known variables}
    \begin{align*}
	A &= 45~000\\
	P &= 30~000\\
	i &= 0,075\\
	n &= ~?
    \end{align*}

    \westep{Write down the formula}
    \begin{align*}
	A &= P(1 + i \cdot n)
    \end{align*}

    \westep{Substitute the values and solve for the remaining variable}
    \begin{align*}
	45~000 &= 30~000(1 + 0,075 \times n)\\
	\frac{45~000}{30~000} &= 1 + 0,075 \times n\\
	\frac{45~000}{30~000} -1 &= 0,075 \times n\\
	\frac{(\frac{45~000}{30~000}) -1}{0,075} &= n\\
	n &= 6,6
    \end{align*}

    \westep{Explain your answer}
    Dit sal ’n periode van 6 jaar en 8 maande neem vir R~$45~000$ om te groei tot R~$30~000$ teen ’n enkelvoudige rentekoers van $7,5\%$.
    }
\end{wex}


\Tip{To get a more accurate answer, try to do all your calculations on the calculator in one go. This will prevent rounding off errors from influencing your final answer.}


\begin{wex}{Calculating the interest rate to achieve the desired growth}{
    At what simple interest rate should I invest if I want to grow R~$2~500$ to R~$4~000$ in 5 years?}{

    \westep{Write down the known variables}
    \begin{align*}
	A &= 4~000\\
	P &= 2~500\\
	i &= ~?\\
	n &= 5
    \end{align*}

    \westep{Write down the formula}
    \begin{align*}
	A &= P(1 + i \cdot n)
    \end{align*}

    \westep{Substitute the values and solve for the remaining variable}
    \begin{align*}
	4~000 &= 2~500(1 + i \times 5)\\
	\frac{4~000}{2~500} &= 1 + i \times 5\\
	\frac{4~000}{2~500} - 1&= i \times 5\\
	\frac{(\frac{4~000}{2~500}) - 1}{5} &= i\\
	i &= 0,12
    \end{align*}

    \westep{Explain your answer}
    A simple interest rate of $12\%$ will be needed when investing R~$2~500$ for 5 years to become R~$4~000$.
    }
\end{wex}


\subsection{When the Time Period is not in Years}

Often someone is not able to invest money for a full year, or they want to take out a loan for a shorter period. How do we calculate the accumulated amount if the time period is not in years?\par

You need to remember that in every year there are 12 months. If I wanted to invest money for just three months, $n$ would be equal to $\frac{3}{12}$. Similarly, if I wanted to take out a loan for a period of 7 months, $n = \frac{7}{12}$.\\


\begin{exercises}{Simple Interest Exercises}
    \begin{enumerate}[label=\textbf{\arabic*}.]
	\item ’n Bedrag van R~$3~500$ word belê in ’n spaarrekening wat enkelvoudige rente betaal teen ’n koers van $7,5\%$ p.a. Bereken die eindbedrag na 2 jaar.

	\item Bereken die enkelvoudige rente in die volgende probleme:
	\begin{enumerate}
	    \item ’n Lening van R~$300$  teen ’n koers van $8\%$ vir 1 jaar.

	    \item ’n Belegging van R~$2~250$ teen ’n koers van $12,5\%$ per jaar vir 6 jaar.
	\end{enumerate}

	\item Sally wanted to calculate the number of years she needed to invest R~$1~000$ for in order to accumulate R~$2~500$. She has been offered a simple interest rate of $8,2\%$ p.a. How many years will it take for the money to grow to R~$2~500$?

	\item Ek het ’n deposito van R~$5~000$ in die bank gemaak. Sestien jaar later was die eindbedrag van hierdie
belegging R~$18~000$ teen watter koers is die geld belê indien enkelvoudige rente bereken is?\\
    \end{enumerate}

    Find the answers with the short codes:\\
    \begin{tabularx}{\textwidth}{ XXXX }
	(1)	&	(2)	&	(3)	&	(4)\\
    \end{tabularx}
\end{exercises}


\subsection{Hire Purchase}

As a general rule, it’s a bad idea to buy things on credit. Buying things on credit means that you have had to borrow money to pay for the object, meaning you will have to pay more for it due to the interest on the loan. That being said, occasionally there are appliances, such as a fridge, that are very difficult to live without. Most people don’t have the cash up front to purchase such items, so they buy it on a hire purchase agreement.\par

A hire purchase agreement is a financial agreement between the shop and the customer on how the customer will pay for the desired product. The interest on a hire purchase loan is always charged at a simple interest rate and only charged on the amount owing. Most agreements require that a deposit is paid before you can take home the product. The principal amount of the loan is therefore the cash price minus the deposit. The accumulated loan will be worked out using the number of years the loan is needed for. The total loan amount is then divided into monthly payments over the period of the loan.


\begin{wex}{Hire Purchase}{
    Troy wil graag ’n ekstra hardeskyf vir sy
skootrekenaar koop teen R~$2~500$ soos dit op die internet adverteer word.
Daar is die opsie om ’n deposito van $10\%$ van die koopprys te betaal
en dan in ’n huurkoop-ooreenkoms 24 gelyke maandelikse paaiemente
te betaal waar rente bereken sal word teen $7,5\%$ per jaar enkelvoudige
rente. Bereken wat Troy se maandelikse paaiement sal wees.}{

    \westep{’n Nuwe aanvangsbedrag is nodig, want die $10\%$ deposito is kontant betaal.}
    \begin{itemize}
	\item $10\%$ van R~$2~500$ = R~$250$\\
	\item Nuwe openingsbalans, $P$ = R~$2~500$ − R~$250$ = R~$2~250$\\
	\item Rentekoers, $i = 7,5\%$\\
	\item Periode, $n = 2$ jare
    \end{itemize}

    Ons moet die eindbedrag ($A$) bepaal en dan die maandelikse
paaiemente bereken.

    \westep{Van \ref{FG:SI} weet ons dat:}
    \begin{align*}
	    A &= P(1 + i \cdot n)
    \end{align*}

    \westep{Substitute the values and solve for the remaining variable}
    \begin{align*}
	A &= 2~250(1 + 0,075 \times 2)\\
	  &= 2~587,50\\
    \end{align*}

    \westep{Calculate the monthly repayments on the hire purchase agreement}
    \begin{align*}
	\text{Maandelikse paaiement} &= \frac{2~587,50}{24}\\
			&= \text{R}~107,81
    \end{align*}

    \westep{Explain your answer}
    Troy se maandelikse paaiment = R~$107,81$.
}
\end{wex}


Occasionally the shop will add a monthly insurance premium to the monthly instalments. This insurance premium will be an amount of money paid monthly and will buy you more time between a missed payment and possible repossession of the product.


\begin{wex}{Hire Purchase with extra conditions}{
    Cassidy desperately wants to buy a tv and as a result, decides to buy one on a hire purchase agreement. The chosen TV’s cash price is R~$5~500$. She will pay it off over 54 months at an interest rate of $21\%$p.a. An insurance premium of R~$12,50$ is added to every monthly payment. How much are her monthly payments?}{

    \westep{Write down the known variables}
    \begin{eqnarray*}
	A &=& ?\\
	P &=& 5~500\\
	i &=& 0,21\\
	n &=& \frac{54}{12} = 4,5
    \end{eqnarray*}
    \text{(The question says nothing about a deposit, therefore we assume that Cassidy did not pay one)}

    \westep{Write down the formula}
    \begin{align*}
	    A &= P(1 + i \cdot n)
    \end{align*}

    \westep{Substitute the values and solve for the remaining variable}
    \begin{align*}
	A &= 5~500(1 + 0,21 \times 4,5)\\
	  &= 10~697,50\\
    \end{align*}

    \westep{Calculate the monthly repayments on the hire purchase agreement}
    \begin{align*}
	\text{Monthly Payment} &= \frac{10~697,50}{54}\\
			&= 198,10
    \end{align*}

    \westep{Add the insurance premium}
    \begin{align*}
	198,10 + 12,50 &= 210,60
    \end{align*}

    \westep{Explain your answer}
    Cassidy will pay R~$210,60$ monthly for 54 months until her TV is paid off.
}
\end{wex}


\Tip{You are expected to know that hire purchase is charged at a simple interest rate. When you are asked a hire purchase question in a test, don’t forget to always use the simple interest formula.}


\begin{exercises}{Hire Purchase}
    \begin{enumerate}[label=\textbf{\arabic*}.]
	\item Vanessa wants to buy a fridge on a hire purchase agreement. The cash price of the fridge is R~$4~500$. She is required to put down a deposit of $15\%$ and pay the remaining loan amount off over 24 months at an interest rate of $12\%$ p.a.
	\begin{enumerate}
	    \item What is the principal loan amount?
	    \item What is the accumulated loan amount?
	    \item What are Vanessa’s monthly repayments?
	    \item What is the total amount she has paid for the fridge?
	\end{enumerate}


	\item Bongani koop ’n eetkamertafel van R~$8~500$ op huurkoop. Hy moet enkelvoudige rente van $17,5\%$per jaar
betaal oor 3 jaar.
	\begin{enumerate}
	    \item Hoeveel sal Bongani in totaal betaal?
	    \item  Hoeveel rente betaal hy?
	    \item Wat is sy maandelikse paaiement?
	\end{enumerate}

	\item A lounge suite is advertised on tv, to be paid off over 36 months at R~$150$ a month.
	\begin{enumerate}
	    \item Assuming that no deposit is needed, how much will the buyer pay for the lounge suite once it has been paid off?
	    \item If the interest rate is $9\%$p.a, what is the cash price of the suite?\\
	\end{enumerate}
    \end{enumerate}

    Kry die oplossing met die kortkodes:\\
    \begin{tabularx}{\textwidth}{ XXX }
	(1)	&	(2)	&	(3)\\
    \end{tabularx}
\end{exercises}


\section{Saamgestelde Rente}

Compound interest allows us to earn interest on our interest. In simple interest, you only every earn interest on your original investment, but in compound interest, you are able to earn interest both on your original investment and the interest earned on it.\par

Compound interest is a great concept if you are investing, however, when taking out a loan, you will be paying more if it is calculated on a compound interest rate rather than simple.


\Definition{Saamgestelde Rente}{Saamgestelde rente is die rente wat bereken word op die aanvangsbedrag en op die
opgeloopte rente}


Laat ons ’n formule ontwikkel vir saamgestelde rente op dieselfde manier as wat ons ’n formule ontwikkel het vir
enkelvoudige rente.\par

Indien ons aanvangsbedrag $P$ is en ons het ’n rentekoers van $i$ per jaar, dan sal die eindbedrag aan die einde
van die eerste jaar gelyk wees aan:
\begin{eqnarray*}
    \text{Eindbedrag na 1 jaar} = P(1 + i)
\end{eqnarray*}

Dit is dieselfde as enkelvoudige rente, want dit strek net oor een tydsinterval (’n jaar in hierdie geval). As ons die
geld dan onttrek en weer belê vir nog ’n jaar - soos wat ons in die uitgewerkte voorbeeld hierbo gedoen het - sal
die eindbedrag aan die einde van die tweede jaar as volg wees:
\begin{eqnarray*}
    \text{Eindbedrag na 2 jaar} &=& [P(1 + i)] \times (1 + i)\\
    &=& P(1 + i)^2
\end{eqnarray*}

As ons hierdie geld onttrek en weer vir nog ’n jaar belê, sal die eindbedrag wees:

\begin{eqnarray*}
    \text{Eindbedrag na 3 jaar} &=& [P(1 + i)^2] \times (1 + i)\\
    &=& P(1 + i)^3
\end{eqnarray*}

Ons kan sien dat die eksponent van die term $(1 + i)$ gelyk is aan die aantal tydsintervalle (jare in hierdie voor-
beeld.) Dus:


\Identity{Saamgestelde rente}{
    \begin{eqnarray}
	\label{FG:CI}
	A &=& P(1 + i)^n\\
	\text{Waar:~} \nonumber\\
	A &=& \text{Eindbedrag} \nonumber\\
	P &=& \text{Aanvangsbedrag} \nonumber\\
	i &=& \text{Rentekoers per tydsinterval, skryf as disimaal} \nonumber\\
	n &=& \text{tydsduur van die lening bepaal} \nonumber
    \end{eqnarray}
}


\begin{wex}{Saamgestelde rente}{
    Mpho wants to invest R~$30~000$ into an account that offers a compound interest rate of $6\%$p.a. How much money will be in the account at the end of 4 years?}{
    
    \westep{Write down the known variables}
    \begin{eqnarray*}
	A &=& ?\\
	P &=& 30~000\\
	i &=& 0,06\\
	n &=& 4
    \end{eqnarray*}

    \westep{Write down the formula}
    \begin{align*}
	A &= P(1 + i)^n
    \end{align*}

    \westep{Substitute the values and solve for the remaining variable}
    \begin{align*}
	A &= 30~000(1 + 0,06)^4\\
	  &= 37~874,31
    \end{align*}

    \westep{Explain your answer}
    Mpho will have R~$37~874,31$ in the account at the end of 4 years.
    }
\end{wex}


\begin{wex}{Calculating the compound interest rate to achieve the desired growth}{
    Charlie has been given R~$5~000$ for his sixteenth birthday. Rather than spending it, he has decided to invest it so that on his eighteenth birthday, he can put a deposit of R~$10~000$ on a car. What compound interest rate should he be looking for?}{
    
    \westep{Write down the known variables}
    \begin{eqnarray*}
	A &=& 10~000\\
	P &=& 5~000\\
	i &=& ?\\
	n &=& 2
    \end{eqnarray*}

    \westep{Write down the formula}
    \begin{align*}
	A &= P(1 + i)^n
    \end{align*}

    \westep{Substitute the values and solve for the remaining variable}
    \begin{align*}
	10~000 &= 5~000(1 + i)^2\\
	\frac{10~000}{5~000}&= (1 +i)^2\\
	\sqrt[]{\frac{10~000}{5~000}} &= 1 + i\\
	\sqrt[]{\frac{10~000}{5~000}} - 1 &= i\\
	i &= 0,4142135624
    \end{align*}

    \westep{Explain your answer}
    Charlie needs to find an account that offers a compound interest rate of $41,42\%$.
    }
\end{wex}


\subsection{Die Krag van Saamgestelde Rente}

Om te sien hoe belangrik "rente op rente" is, sal ons kyk na die verskil in die eindbedrae van geld wat teen
enkelvoudige rente belê is en geld wat teen saamgestelde rente belê is. Beskou ’n bedrag van R~$10~000$  wat jy
vir 10 jaar moet belê en aanvaar dat jy rente kan verdien teen $9\%$. per jaar. Wat sal die waarde van die belegging
wees na 10 jaar?\par

Die eindbedrag indien die geld enkelvoudige rente verdien, is:
\begin{align*}
    A &= P(1 + i \cdot n)\\
      &= 10~000(1 + 0,09 \times 10)\\
      &= R~19~000\\
\end{align*}

Die eindbedrag indien die geld saamgestelde rente verdien, is:
\begin{align*}
    A &= P(1 + i)^n\\
      &= 10~000(1 + 0,09)^10\\
      &= R~23~673,64
\end{align*}

If we graph the growth of the money, it’s easier to see how it increases in just a straight line in simple interest but exponentially in compound interest:
\begin{figure}[H]
    \begin{center}
	\begin{pspicture}(-5,-2)(7,8)
	    \psset{yunit=0.75,xunit=1}
	    \psgrid[subgriddiv=1,griddots=10,gridlabels=0](0,0)(10.4,12.3)
	    \psaxes[arrows=-, dx=1, Dx=1, dy=1, Dy=2000](0,0)(0,0)(10.4,12.3)
% 	    \psline[linecolor=red](0,5)(10,9.5)
	    \psplot[linecolor=red, algebraic=true]{0}{10}{5*((1+0.09*x))}
	\end{pspicture}
	\caption{The growth of money in a simple interest account over 10 years.}
	\label{FG:fig:SI10}
    \end{center}
\end{figure}

\begin{figure}[H]
    \begin{center}
	\begin{pspicture}(-5,-2)(7,8)
	    \psset{yunit=0.75,xunit=1}
	    \psgrid[subgriddiv=1,griddots=10,gridlabels=0](0,0)(10.4,12.3)
	    \psaxes[arrows=-, dx=1, Dx=1, dy=1, Dy=2000](0,0)(0,0)(10.4,12.3)
	    \psplot[linecolor=red, algebraic=true]{0}{10}{5*((1+0.09)^x)}
	\end{pspicture}
	\caption{The growth of money in a compound interest account over 10 years.}
	\label{FG:fig:CI10}
    \end{center}
\end{figure}

It’s easier to see the vast difference in growth if we extend the time period to 50 years:
\begin{figure}[H]
    \begin{center}
	\begin{pspicture}(-2,-2)(7,8)
	    \psset{yunit=0.75,xunit=0.65}
	    \psgrid[subgriddiv=1,griddots=10,gridlabels=0](0,0)(20,15)
	    \psaxes[arrows=-, dx=2, Dx=5, dy=1, Dy=50000](0,0)(0,0)(20,15)
% 	    \psline[linecolor=red](0,0.2)(20,1.1)
	    \psplot[linecolor=red, algebraic=true]{0}{20}{0.2*((1+0.09*x*5/2))}
	\end{pspicture}
	\caption{The growth of money in a simple interest account over 50 years.}
	\label{FG:fig:SI10}
    \end{center}
\end{figure}

\begin{figure}[H]
    \begin{center}
	\begin{pspicture}(-2,-2)(7,8)
	    \psset{yunit=0.75,xunit=0.65}
	    \psgrid[subgriddiv=1,griddots=10,gridlabels=0](0,0)(20,15)
	    \psaxes[arrows=-, dx=2, Dx=5, dy=1, Dy=50000](0,0)(0,0)(20,15)
	    \psplot[linecolor=red, algebraic=true]{0}{20}{0.2*((1+0.09)^(x*5/2))}
	\end{pspicture}
	\caption{The growth of money in a compound interest account over 50 years.}
	\label{FG:fig:CI10}
    \end{center}
\end{figure}

Weereens, hou in gedagte dat dit goeie nuus en slegte nuus is. As jy rente verdien op geld wat jy belê het,
sal saamgestelde rente daartoe lei dat die bedrag eksponensieel vermeerder. Maar, as jy geld geleen het, sal
daardie bedrag ook eksponensieel groei.


\begin{exercises}{Compound Interest}
    \begin{enumerate}[label=\textbf{\arabic*}.]
	\item ’n Bedrag van R~$3~500$ word belê in ’n spaarrekening wat saamgestelde rente verdien teen $7,5\%$ per jaar.
Bereken die bedrag wat opgebou is in die rekening na verloop van 2 jaar.

	\item Morgan invests R~$5~000$ into an account which pays out a lump sum at the end of 5 years. If he gets R~$7~500$ at the end of the period, what compound interest rate did the bank offer him?

	\item Shrek wil geld belê teen  $11\%$ per jaar saamgestelde rente. Hoeveel geld (tot die naaste rand) moet hy belê
indien hy oor 5 jaar ’n bedrag van R~$100~000$ wil hê?\\
    \end{enumerate}

    Kry die oplossing met die kortkodes:\\
    \begin{tabularx}{\textwidth}{ XXX }
	(1)	&	(2)	&	(3)\\
    \end{tabularx}
\end{exercises}


\subsection{Inflation}

You may have heard many older people recalling times when things were cheaper. You may remember buying things as a child and be shocked at the increase in price. There are many factors that influence the change in price of an item, one of them being inflation.\par

Inflation is the average percentage that all goods increase by every year. As this increase happens every year, it is calculated using the compound interest formula.


\begin{wex}{Calculating future cost based on inflation}
    {Milk costs R~$14$ for two litres. How much will it cost in 4 years time if the inflation rate is $9\%$p.a.?}{
    
    \westep{Write down the known variables}
    \begin{eqnarray*}
	A &=& ?\\
	P &=& 14\\
	i &=& 0,09\\
	n &=& 4
    \end{eqnarray*}

    \westep{Write down the formula}
    \begin{align*}
	A &= P(1 + i)^n
    \end{align*}

    \westep{Substitute the values and solve for the remaining variable}
    \begin{align*}
	A &= 14(1 + 0,09)^4\\
	  &= 19,76
    \end{align*}

    \westep{Explain your answer}
    In four years time, milk will cost R~$19,76$.
    }
\end{wex}


\begin{wex}{Calculating past cost based on inflation}
    {A box of chocolates costs R~$55$ today. How much did it cost 3 years ago if the average rate of inflation was $11\%$p.a.?}{
    
    \westep{Write down the known variables}
    \begin{eqnarray*}
	A &=& 55\\
	P &=& ?\\
	i &=& 0,11\\
	n &=& 3
    \end{eqnarray*}

    \westep{Write down the formula}
    \begin{align*}
	A &= P(1 + i)^n
    \end{align*}

    \westep{Substitute the values and solve for the remaining variable}
    \begin{align*}
	55 &= P(1 + 0,11)^3\\
	\frac{55}{(1 + 0,11)^3} &= P\\
	P  &= 40,22
    \end{align*}

    \westep{Explain your answer}
    Three years ago, a box of chocolates would have cost R~$40,22$.
    }
\end{wex}


\subsection{Population Growth}

How many people are your grandparents responsible for bringing into this world? Maybe they just had one child (your mom or dad) or maybe they had a dozen. How many of your parent’s siblings then had children of their own? How many of your cousins have already started having babies? A family tree typically starts with two people who have offspring. These people then have more babies which in turn procreate further. The family tree increases exponentially as every person who is born has the ability to be a part of the creation process of another. For this reason we calculate population growth using the compound interest formula.


\begin{wex}{Population growth}
    {If the current population of Johannesburg is $3~888~180$, and the average rate of population growth in South Africa is $2,1\%$p.a., what can city planners expect the population of Johannesburg to be in 10 years?}{
    
    \westep{Write down the known variables}
    \begin{eqnarray*}
	A &=& ?\\
	P &=& 3~888~180\\
	i &=& 0,021\\
	n &=& 10
    \end{eqnarray*}

    \westep{Write down the formula}
    \begin{align*}
	A &= P(1 + i)^n
    \end{align*}

    \westep{Substitute the values and solve for the remaining variable}
    \begin{align*}
	A &= 3~888~180(1 + 0,021)^{10}\\
	  &= 4~786~342,614
    \end{align*}

    \westep{Explain your answer}
    City planners can expect a population of $4~786~343$ in Johannesburg in ten years time.
    }
\end{wex}


\begin{exercises}{Inflation and Population}
    \begin{enumerate}[label=\textbf{\arabic*}.]
	\item Die gemiddelde inflasiekoers vir die afgelope aantal jaar is $7,3\%$ per jaar en jou water- en elektrisiteitsrekening is gemiddeld R~$1~425$ Bereken wat jy kan verwag om te betaal oor 6 jaar?

	\item The price of pop corn and coke at the movies is now R~$60$. If the average rate of inflation is $9,2\%$, what was the price of pop corn and coke 5 years ago?

	\item A small town in Ohio, USA is experience a huge increase in births. If the average growth rate of the population is $16\%$ p.a, how many babies will be born to the 1600 residents in the next 2 years?\\
    \end{enumerate}

    Kry die oplossing met die kortkodes:\\
    \begin{tabularx}{\textwidth}{ XXX }
	(1)	&	(2)	&	(3)\\
    \end{tabularx}
\end{exercises}



\subsection{Buitelandse Wisselkoerse}

Different countries have developed their own currencies through the years. In England, a Big Mac from McDonalds will cost you £~$4$, in South Africa it will cost you R~$20$ and in Norway it will cost you $48$~kr. In all three of those places, you will get the same meal, however, some will be more expensive than others. At the time of writing this paragraph, £~$1$ = R~$12,41$ and $1$~kr = R~$1,37$, this means that the Big Mac would cost a South African who was travelling to England R~$49,64$ and one who was travelling to Norway R~$65,76$.\par

Exchange rates affect a lot more than just the price of a Big Mac. The price of oil will go up when the South African Rand gets weaker. This is because when the Rand is weaker, we can buy less of other currencies with the same amount of money.\par

A currency gets stronger when money is invested into the country. When we buy products that are made in South Africa, we are investing in South African business and keeping the money in the country. When we buy products imported from other countries, we are investing money into those countries and as a result, the Rand will weaken. Remember, the more South African products we buy, the greater demand there will be for them, meaning that more jobs will open for South Africans. Local is lekker!


\begin{wex}{Buitelandse Wisselkoerse}
    {Saba wants to travel to see family in Spain. She has been given R~$10~000$ spending money. How many Euros can she buy if the exchange rate is currently €~$1$ = R~$10,68$?}{
    
    Saba can buy €~$936,33$ with R~$10~000$.
\end{wex}


\begin{exercises}{Buitelandse Wisselkoerse}
    \begin{enumerate}[label=\textbf{\arabic*}.]
	\item Ek wil ’n iPod koop wat £~$100$, kos, en die huidige wisselkoers is tans £~$1$ = R~$14$. Ek glo die wisselkoers gaan verander na R~$12$ binne ’n maand.
	\begin{enumerate}
	    \item Hoeveel sal die Ipod speler in rand kos as ek dit nou koop?
	    \item Hoeveel sal ek spaar as die wisselkoers daal na R~$12$?
	    \item Hoeveel sal ek verloor as die wisselkoers verander na R~$15$?
	\end{enumerate}

	\item Bestudeer die volgende tabel met wisselkoerse:
	\begin{center}
	    \begin{tabular}{ |l|l|l| }
		\hline
		Land	&	Geldeenheid 	&	Wisselkoers\\ \hline
		Verenigde Koninkryk(VK) 	&	Pounds(£)	&	R~$14,13$\\ \hline
		Amerika (VSA) 	&	Dollars(\$}	&	R~7,04\\ \hline
	    \end{tabular}
	\end{center}
	
	\begin{enumerate}
	    \item In Suid-Afrika is die koste van ’n nuwe Honda Civic R~$173~400$. In Engeland kos dieselfde voertuig £~$12~200$ en in die \$~$21~900$. In watter land is die voertuig die goedkoopste as jy die pryse
omskakel na Suid-Afrikaanse rand ?

	    \item Sollie en Arinda is kelners in ’n Suid-Afrikaanse restaurant wat baie oorsese toeriste lok. Sollie kry ’n £~$6$ fooitjie van ’n toeris en Arinda kry \$~$12$.  Hoeveel Suid-Afrikaanse rand het elkeen gekry?
	\end{enumerate}
    \end{enumerate}

    Kry die oplossing met die kortkodes:\\
    \begin{tabularx}{\textwidth}{ XX }
	(1)	&	(2)\\
    \end{tabularx}
\end{exercises}


\subsection{Opsomming}

\begin{itemize}
    \item Daar is twee soorte rente: enkelvoudige rente en saamgetelde rente:\\
    
% \framebox{
    \begin{tabularx}{\textwidth}{ XX }
	enkelvoudige rente	&	saamgetelde rente\\
	$A = P (1 + i \cdot n)$	&	$A = P(1 + i)^n$\\
    \end{tabularx}
    \par
    Waar:
    \begin{eqnarray*}
% 	\text{Where:~}\\
	A &=& \text{Aanvangsbedrag}\\
	P &=& \text{Eindbedrag}\\
	i &=& \text{rentekoers per tydsinterval}\\
	n &=& \text{aantal tydsintervalle}
    \end{eqnarray*}
% }

    \item Hire Purchase loan repayments are calculated using the simple interest formula on the remaining loan amount. Monthly repayments are calculated by dividing the accumulated amount by the number of months the loan needs to be paid off in.

    \item Population growth and inflation are calculated using the compound interest formula.

    \item ’n Buitelandse wisselkoers is die prys van een geldeenheid in terme van ’n ander.
\end{itemize}


Die volgende 3 videos gee ’n opsomming oor hoe om rente te bereken. Let daarop dat hoewel die voorbeelde
met dollars gedoen word, ons die feit kan gebruik dat die dollar ’n desimale geldeenheid is net soos die rand
(ignoreer die wisselkoers). Dit is wat in die onderafdelings gedoen is.\par 

\begin{figure}[H]
    \textnormal{Khan Akademie video oor rente - 1}
    \vspace{.1in}
    \nopagebreak
    \raisebox{-5 pt}{ \includegraphics[width=0.5cm]{col11306.imgs/summary_www.png}} { (Video:  MG10030 )}
 \end{figure}

\begin{figure}[H] % horizontal\label{m39335*interest-2}
    \textnormal{Khan Akademie video oor rente  - 2}
    \vspace{.1in}
    \nopagebreak
    \raisebox{-5 pt}{ \includegraphics[width=0.5cm]{col11306.imgs/summary_www.png}} { (Video:  MG10031 )}
\end{figure}

Let Wel: Aan die einde van hierdie video word die reël van 72 genoem. Jy sal nie hierdie reël gebruik nie, maar
sal liewer die probeer-en-tref metode gebruik om die gevraagde probleem op te los.

\begin{figure}[H] % horizontal\label{m39335*interest-3}
    \textnormal{Khan Akademie video oor rente  - 3}
    \vspace{.1in}
    \nopagebreak
    \raisebox{-5 pt}{ \includegraphics[width=0.5cm]{col11306.imgs/summary_www.png}} { (Video:  MG10032 )}
\end{figure}


\begin{eocexercises}{Finance and Growth}
    \begin{enumerate}[label=\textbf{\arabic*}.]
	\item Jy is met vakansie in Europa. Die hotel vra €~$200$ euro per nag. Hoeveel rand het jy nodig om die hotelreken-
ing te betaal as die wisselkoers  €~$1$ = R~$9,20$?

	\item Bereken hoeveel rente jy sal verdien as jy R~$500$ vir 1 jaar belê teen die volgende rentekoerse:
	\begin{enumerate}
	    \item $6,85\%$ enkelvoudige rente.
	    \item $4,00\%$ saamgetelde rente.
	\end{enumerate}

	\item Bianca het R~$1~450$ om vir 3 jaar te belê. Bank A bied ’n spaarrekening aan wat enkelvoudige rente betaal
teen ’n koers van $11\%$ per annum. Bank B bied ’n spaarrekening aan wat saamgestelde rente betaal
teen ’n koers van $10,5\%$ per annum. Watter bank gaan vir Bianca die spaarrekening gee met die grootste
opgeloopte bedrag aan die einde van die 3 jaar?

	\item Hoeveel enkelvoudige rente is betaalbaar op ’n lening van R~$2~000$ vir ’n jaar indien die rentekoers $10\%$ per jaar is?

	\item Hoeveel saamgestelde rente is betaalbaar op ’n lening van  R~$2~000$ vir ’n jaar indien die rentekoers $10\%$ per jaar is?

	\item Bespreek:
	\begin{enumerate}
	    \item watter soort rente jy sal verkies as jy die lener is?

	    \item watter soort rente jy sal verkies as jy die bankier is?
	\end{enumerate}

	\item Bereken die saamgestelde rente vir die volgende probleme.
	\begin{enumerate}
	    \item ’n lening van R~$2~000$ vir 2 jaar teen  $5\%$ per jaar.
	    \item ’n belegging van  R~$1~500$ vir 3 jaar teen $6\%$ per jaar.
	    \item ’n lening van R~$800$ vir l jaar teen $16\%$per jaar.
	\end{enumerate}

	\item As die wisselkoers vir ¥~$100$ = R~$6,2287$ (¥ = Yen) en 1 Australiese dollar (AUD) = R~$5,1094$, bepaal die wisselkoers
tussen die Australiese dollar en die Japanese jen.
	\item Bonnie het ’n stoof gekoop vir R~$3~750$. Na 3 jaar het sy dit klaar betaal, asook die R~$956,25$ huurkoopkoste.
Bereken die koers waarteen enkelvoudige rente bereken is.
    \end{enumerate}

    Kry die oplossing met die kortkodes:\\
    \begin{tabularx}{\textwidth}{ XXXXXXXXX }
	(1) & (2) & (3) & (4) & (5) & (6) & (7) & (8) & (9)\\
    \end{tabularx}
\end{eocexercises}
