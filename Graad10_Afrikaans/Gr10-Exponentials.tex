\chapter{Eksponensiale}\fancyfoot[LO,RE]{Focus Area: Wiskunde}
\setcounter{figure}{1}
\setcounter{subfigure}{1}



\label{m38359*id62562}Eksponensiaalnotasie is ’n kort manier om te skryf dat ’n getal meermale met homself vermenigvuldig word. Laat ons beter definieer hoe om eksponensiaalnotasie te gebruik

 \begin{Large}
\begin{center}
$ _{\mbox{grondtal}~\leftarrow} $\begin{Large} $ ~a^{n~\rightarrow~$ \end{Large}$\mbox{eksponent / indeks}} $
\end{center}
 \end{Large}

Vir enige re\"ele getal $a$ en natuurlike getal $n$, kan ons $a$ wat $n$ keer vermenigvuldig word met homself skryf as $a^n$.

 
\Identity{Definisie:}{
\begin{flushleft}
\begin{enumerate}[itemsep=5pt, label=\textbf{\arabic*}.]
 \item $a^n = a \times a \times a \times \ldots \times a ~~ (n ~ \mbox{keer}) ~~~~ (a \in \mathbb{R}, n \in \mathbb{N})$
 \item $ a^0 = 1 \hspace{1cm}$  ($a \ne 0 $ want $0^0$ is ongedefinieer)
 \item $a^{-n} &=& \frac{1}{a^n}\hspace{0.5cm}$ ($a \ne 0 $ want $\frac{1}{0}$ is ongedefinieer)
\end{enumerate}



\end{flushleft}

} 

Voorbeelde:
\begin{enumerate}[noitemsep, label=\textbf{\arabic*.}]
\item $3 \times 3 = 3^2$
\item $5 \times 5 \times 5 \times 5 = 5^4 $
\item $p \times p \times p = p^3$
\item $(3^x)^0 = 1$
\item $ 2^{-4} = \dfrac{1}{2^4} = \frac{1}{16}$
\item $ \dfrac{1}{5^{-x}} = 5^x$
\end{enumerate}

% We now expand the definition to include 0 and the negative integers.
% 
% If exponent $n$ is 0, then we have 
% 
% 
% 
% \Tip{The restriction applies because $0^0$ is undefined }
% 
% We also define what it means if we have a negative exponent $a^{-n}$.
% \begin{eqnarray*}
%     a^{-n} &=& \frac{1}{a^n} \\
%            &=& \frac{1}{a \times a \times a \times \ldots \times a ~~ (n ~ \textrm{times})} 
% \end{eqnarray*}
% 
% 
% \Tip{Here $a$ cannot equal 0 because $\frac{1}{0}$ is undefined}      
      
\textbf{Nota:} Ons skryf altyd die finale antwoord met positiewe eksponente.


% If exponent $n$ is an even integer, then ${a}^{n}$ will always be positive for any non-zero real number $a$. 
% 
% For example, although $-3$ is negative, but $(-3)^2=-3 \times -3 = 9$ which is positive and $(-3)^{-2} = \frac{1}{-3 \times -3} = \frac{1}{9} $ is also positive.
% 
% If the exponent $n$ is an odd integer, then for any non-zero real number $a$ 
% 
% \begin{eqnarray*}
% a^n ~~ \mbox{is positive if} ~~ a > 0 \\
% a^n ~~ \mbox{is negative if} ~~ a < 0
% \end{eqnarray*}
% 
% For example, $(-2)^3 = -2 \times -2 \times -2 = -8$ and $(2)^{-5} = \dfrac{1}{2 \times 2 \times 2 \times 2 \times 2} = \dfrac{1}{32}$.

   
    


\section {Eksponentwette}
\nopagebreak

\label{m38359*id63061}Daar is 'n aantal eksponentwette wat ons kan gebruik om getalle met eksponente te vereenvoudig. Sommige
van hierdie wette het ons reeds in vorige grade teëgekom, maar ons sal die volledige lys hier sien en elke wet
verduidelik, sodat jy hulle kan verstaan en nie bloot memoriseer nie


\begin{eqnarray*}
{a}^{m}\ensuremath{\times}{a}^{n}& =& {a}^{m+n}\hfill \\ 
\frac{{a}^{m}}{{a}^{n}}& =& {a}^{m-n}\hfill \\ 
{\left(ab\right)}^{n}& =& {a}^{n}{b}^{n}\hfill \\ 
        (\frac{a}{b})^n & = & \frac{a^n}{b^n} \\
{\left({a}^{m}\right)}^{n}& =& {a}^{mn}\hfill 
\end{eqnarray*}
waar $a > 0, b > 0$ en $m,n \in \mathbb{Z}$

% 
%       \label{m38359*uid4}
%             \  \subsubsubsection { Exponential Law 1: ${a}^{0}=1$}
%             \nopagebreak
%         \label{m38359*id63512}Our definition of exponential notation shows that\par 
%         \label{m38359*uid5}\nopagebreak\noindent{}
%           
%     \begin{center}
%     \begin{array}{ccl}\hfill {a}^{0}& =& 1,\left(a\ne 0\right)\hfill \end{array}
%       \end{center}
%         \label{m38359*eip-662}To convince yourself of why this is true, use the fourth exponential law above (division of exponents) and consider what happens when $m=n$.\par \label{m38359*id63571}For example, ${x}^{0}=1$ and ${\left(1\phantom{\rule{0.277778em}{0ex}}000\phantom{\rule{0.277778em}{0ex}}000\right)}^{0}=1$.\par 
% \label{m38359*secfhsst!!!underscore!!!id339}
%               \subsubsubsection { Exercise: Exponential Law 1: ${a}^{0}=1,\left(a\ne 0\right)$ }
%             \nopagebreak
%         \label{m38359*id63666}\begin{enumerate}[noitemsep, label=\textbf{\arabic*}. ] 
%             \label{m38359*uid6}\item 
%             ${16}^{0}$
%       \label{m38359*uid7}\item 
%         $16{a}^{0}$
%       \label{m38359*uid8}\item 
%         ${\left(16+a\right)}^{0}$
%       \label{m38359*uid9}\item 
%         ${\left(-16\right)}^{0}$
%       \label{m38359*uid10}\item 
%         $-{16}^{0}$ 
% \newline
% \newline
%           \end{enumerate}
%       \label{m38359*uid11}
% \par \raisebox{-5 pt}{\includegraphics[width=0.5cm]{col11306.imgs/summary_www.png}} Find the answers with the shortcodes:
%  \par \begin{tabular}[h]{cccccc}
%  (1.) lOG  & \end{tabular}
%             \subsubsubsection{ Exponential Law 2: 

% \begin{center}
%   \begin{array}[lc]
%       \textrm{\textbf{Exponential Law:}} & c
%   \end{array}
% \end{center}

\Identity{Exponential Law}
{
$${a}^{m}\ensuremath{\times}{a}^{n}={a}^{m+n}$$

Die definisie bepaal dat

\begin{eqnarray*}
	      {a}^{m}\ensuremath{\times}{a}^{n}& =& a\ensuremath{\times}a\ensuremath{\times} \ldots \ensuremath{\times}a\hfill & \left(\mbox{$m$ keer}\right)\hfill \\ 
	      & & \phantom{\rule{-0.166667em}{0ex}}\phantom{\rule{-0.166667em}{0ex}}\phantom{\rule{-0.166667em}{0ex}}\phantom{\rule{-0.166667em}{0ex}}\ensuremath{\times}a\ensuremath{\times}a\ensuremath{\times}\ldots\ensuremath{\times}a\hfill & \left(\mbox{$n$ keer}\right)\hfill \\ 
	      & =& a\ensuremath{\times}a\ensuremath{\times}\ldots\ensuremath{\times}a\hfill & \left(\mathrm{m}+\mbox{$n$ keer}\right)\hfill \\ 
	      & =& {a}^{m+n}\hfill & 
\end{eqnarray*}



}


\label{m38359*id64082}Byvoorbeeld,\par 


\begin{eqnarray*}
{2}^{4}\ensuremath{\times}{2}^{3}& =& \left(2\ensuremath{\times}2\ensuremath{\times}2\ensuremath{\times}2\right)\ensuremath{\times}\left(2\ensuremath{\times}2\ensuremath{\times}2\right)\hfill \\
	      & =& {2}^{4+3} \\
	      & =& {2}^{7}
\end{eqnarray*}

\Note{Hierdie eenvoudige wet is die rede waarom eksponente oorspronklik geskep
is. Voor die dae van rekenaars moes vermenigvuldiging met potlood en papier gedoen
word. Dit vat baie lank om vermenigvuldiging te doen, maar dit is vinnig en eenvoudig
om getalle bymekaar te tel. Hierdie eksponentwet wys dat dit moontlik is om twee
getalle te vermenigvuldig deur hulle eksponente bymekaar te tel (indien hulle dieselfde
grondtal het). Dit het wiskundiges baie tyd gespaar, wat hulle toe kon
gebruik om iets meer produktiefs te doen.}

    
% \par
% \subsubsubsection{  Application using Exponential Law 2: ${a}^{m}\ensuremath{\times}{a}^{n}={a}^{m+n}$ }
%             \nopagebreak
%         \label{m38359*id64269}\begin{enumerate}[noitemsep, label=\textbf{\arabic*}. ] 
%             \label{m38359*uid13}\item 
%             ${x}^{2}\ensuremath{\cdot}{x}^{5}$
%       \label{m38359*uid14}\item 
%         ${2}^{3}\ensuremath{\cdot}{2}^{4}$
%         [Take note that the base (2) stays the same.]
%       \label{m38359*uid15}\item 
%         $3\ensuremath{\times}{3}^{2a}\ensuremath{\times}{3}^{2}$
% \newline
% \newline
%           \end{enumerate}
%       \label{m38359*uid16}
% \par \raisebox{-5 pt}{\includegraphics[width=0.5cm]{col11306.imgs/summary_www.png}} Find the answers with the shortcodes:
%  \par \begin{tabular}[h]{cccccc}
%  (1.) lO7  & \end{tabular}
%             \subsubsubsection{ Exponential Law 3: ${a}^{-n}=\frac{1}{{a}^{n}},a\ne 0$}
%             \nopagebreak
%         \label{m38359*id64482}Our definition of exponential notation for a negative exponent shows that\par 
%         \label{m38359*uid17}\nopagebreak\noindent{}
%     \begin{center}
%     \begin{array}{cccc}\hfill {a}^{-n}& =& 1÷a÷...÷a\hfill & \left(\mathrm{n\; times}\right)\hfill \\ \hfill & =& \frac{1}{1\ensuremath{\times}a\ensuremath{\times}\cdots \ensuremath{\times}a}\hfill & \left(\mathrm{n\; times}\right)\hfill \\ \hfill & =& \frac{1}{{a}^{n}}\hfill & \end{array}\tag{5.8}
%       \end{center}
%         \label{m38359*id64624}This means that a minus sign in the exponent is just another way of showing that the whole exponential number is to be divided instead of multiplied.\par 
%         \label{m38359*id64630}For example,\par 
%         \label{m38359*id64634}\nopagebreak\noindent{}
%           
%     \begin{center}
%     \begin{array}{ccl}\hfill {2}^{-7}& =& \frac{1}{2\ensuremath{\times}2\ensuremath{\times}2\ensuremath{\times}2\ensuremath{\times}2\ensuremath{\times}2\ensuremath{\times}2}\hfill \\ & =& \frac{1}{{2}^{7}}\hfill \end{array}\tag{5.9}
%       \end{center}
% \label{m38359*eip-294}This law is useful in helping us simplify fractions where there are exponents in both the denominator and the numerator. For example:
% \label{m38359*id7902}\nopagebreak\noindent{}
% 
%     \begin{center}
%     \begin{array}{ccl}\frac{{a}^{-3}}{{a}^{4}}& =& \frac{1}{{a}^{3}{a}^{4}}\\ & =& \frac{1}{{a}^{7}}\end{array}\tag{5.10}
%       \end{center}
% \par \label{m38359*secfhsst!!!underscore!!!id835}
%             \subsubsubsection{  Application using Exponential Law 3: ${a}^{-n}=\frac{1}{{a}^{n}},a\ne 0$ }
%             \nopagebreak
%         \label{m38359*id64771}\begin{enumerate}[noitemsep, label=\textbf{\arabic*}. ] 
%             \label{m38359*uid18}\item 
%             ${2}^{-2}=\frac{1}{{2}^{2}}$
%       \label{m38359*uid19}\item 
%         $\frac{{2}^{-2}}{{3}^{2}}$
%       \label{m38359*uid20}\item 
%         ${\left(\frac{2}{3}\right)}^{-3}$
%       \label{m38359*uid21}\item 
%         $\frac{m}{{n}^{-4}}$
%       \label{m38359*uid22}\item 
%         $\frac{{a}^{-3}\ensuremath{\cdot}{x}^{4}}{{a}^{5}\ensuremath{\cdot}{x}^{-2}}$
% \newline
% \newline
%           \end{enumerate}
%       \label{m38359*uid23}
% \par \raisebox{-5 pt}{\includegraphics[width=0.5cm]{col11306.imgs/summary_www.png}} Find the answers with the shortcodes:
%  \par \begin{tabular}[h]{cccccc}
%  (1.) lcx  & \end{tabular}

% \begin{center}
% \begin{array}[lc]
% \textrm{\textbf{Exponential Law:}} & $$ \frac{ {a}^{m} }{ {a}^{n} }={a}^{m-n}$$
% \end{array}
% \end{center}


\Identity{Eksponente wet}
{
$$ \frac{ {a}^{m} }{ {a}^{n} }={a}^{m-n}$$
\begin{eqnarray*}
\frac{a^m}{a^n} &=& \frac{a \times a \times a \ldots \times a ~ (m~\mbox{keer})} {a \times a \times a \ldots \times a ~ (n~\mbox{keer})} \\
&= & a \times a \times a \ldots \times a ~ (m-n~\mbox{keer}) \\
&= & a^{m-n}
\end{eqnarray*}

}




          
\label{m38359*id65293}Byvoorbeeld,

\begin{eqnarray*}
    \dfrac{{2}^{7}}{{2}^{3}}& =& \dfrac{2\ensuremath{\times}2\ensuremath{\times}2\ensuremath{\times}2\ensuremath{\times}2\ensuremath{\times}2\ensuremath{\times}2}{2\ensuremath{\times}2\ensuremath{\times}2}\hfill \\
						     & =& 2\ensuremath{\times}2\ensuremath{\times}2\ensuremath{\times}2\hfill \\
						     & =& {2}^{4}\hfill \hfill 
\end{eqnarray*}


% \par \label{m38359*secfhsst!!!underscore!!!id1192}
%             \subsubsubsection{ Application using Exponential Law 4: ${a}^{m}÷{a}^{n}={a}^{m-n}$}
%             \nopagebreak
%         \label{m38359*id65493}\begin{enumerate}[noitemsep, label=\textbf{\arabic*}. ] 
%             \label{m38359*uid25}\item 
%             $\frac{{a}^{6}}{{a}^{2}}={a}^{6-2}$
%       \label{m38359*uid26}\item 
%         $\frac{{3}^{2}}{{3}^{6}}$
%       \label{m38359*uid27}\item 
%         $\frac{32{a}^{2}}{4{a}^{8}}$
%       \label{m38359*uid28}\item 
%         $\frac{{a}^{3x}}{{a}^{4}}$
% \newline8
% \newline
%           \end{enumerate}
%       \label{m38359*uid29}
% \par \raisebox{-5 pt}{\includegraphics[width=0.5cm]{col11306.imgs/summary_www.png}} Find the answers with the shortcodes:
%  \par \begin{tabular}[h]{cccccc}
%  (1.) lOA  & \end{tabular}


% \begin{center}
%   \begin{array}[lc]
%       \textrm{\textbf{Exponential Law:}} & $$ {\left(ab\right)}^{n}={a}^{n}{b}^{n}$$
%   \end{array}
% \end{center}

\Identity{Eksponentwette}
{
 $$ {\left(ab\right)}^{n}={a}^{n}{b}^{n}$$

\par
Die volgorde waarin twee getalle vermenigvuldig word, is onbelangrik. Dus $a \times b = b \times a$ 
\begin{center}
\begin{array}{cclc}\hfill {\left(ab\right)}^{n}& =& a\ensuremath{\times}b\ensuremath{\times}a\ensuremath{\times}b\ensuremath{\times}\ldots\ensuremath{\times}a\ensuremath{\times}b\hfill & \left(\mbox{$n$ keer}\right)\hfill \\
	\hfill & =& a\ensuremath{\times}a\ensuremath{\times}\ldots\ensuremath{\times}a\hfill & \left(\mbox{$n$ keer}\right)\hfill \\
	\hfill & & \phantom{\rule{-0.166667em}{0ex}}\phantom{\rule{-0.166667em}{0ex}}\phantom{\rule{-0.166667em}{0ex}}\phantom{\rule{-0.166667em}{0ex}}\ensuremath{\times}b\ensuremath{\times}b\ensuremath{\times}\ldots\ensuremath{\times}b\hfill & \left(\mbox{$n$ keer}\right)\hfill \\
	\hfill & =& {a}^{n}{b}^{n}\hfill & 
\end{array}
\end{center}
}


%             \subsubsubsection{ Exponential Law 5: ${\left(ab\right)}^{n}={a}^{n}{b}^{n}$}


\label{m38359*id66030}Byvoorbeeld,

\begin{center}
    \begin{array}{ccl}\hfill {\left(2\ensuremath{\cdot}3\right)}^{4}& =& \left(2\ensuremath{\cdot}3\right)\ensuremath{\times}\left(2\ensuremath{\cdot}3\right)\ensuremath{\times}\left(2\ensuremath{\cdot}3\right)\ensuremath{\times}\left(2\ensuremath{\cdot}3\right)\hfill \\
	    & =& \left(2\ensuremath{\times}2\ensuremath{\times}2\ensuremath{\times}2\right)\ensuremath{\times}\left(3\ensuremath{\times}3\ensuremath{\times}3\ensuremath{\times}3\right)\hfill \\
	    & =& \left({2}^{4}\right)\ensuremath{\times}\left({3}^{4}\right)\hfill \\ & =& {2}^{4}\cdot{3}^{4}\hfill 
    \end{array}
\end{center}



% \begin{center}
% \begin{array}[lc]
%   \textrm{\textbf{Exponential Law:}} & $$ (\frac{a}{b})^n = \frac{a^n}{b^n} $$
% \end{array}
% \end{center}


\Identity{Eksponentwette}
{
$$ \Big(\frac{a}{b}\Big)^n = \frac{a^n}{b^n} $$

\begin{eqnarray*}

 \Big(\frac{a}{b}\Big)^n & = & \Big(\frac{a}{b}\Big) \times \Big(\frac{a}{b}\Big) \times \Big(\frac{a}{b}\Big) \times \ldots \times \Big(\frac{a}{b}\Big) ~~~(n~\mbox{keer}) \\
                         & = & \frac{a \times a \times a \times \ldots \times a ~~~(n~\mbox{keer})}{b \times b \times b \times \ldots \times b ~~~(n~\mbox{keer})}\\
                         & = & \frac{a^n}{b^n}
\end{eqnarray*}



}




Byvoorbeeld,

\begin{eqnarray*}

\Big(\frac{2}{3}\Big)^3 & = & \Big(\frac{2}{3}\Big) \times  \Big(\frac{2}{3}\Big) \times \Big(\frac{2}{3}\Big) \\
                        & = & \frac{2 \times 2 \times 2}{3 \times 3 \times 3} \\
		        & = & \frac{2^3}{3^3}




\end{eqnarray*}







%             \subsubsubsection{ Application using Exponential Law 5: ${\left(ab\right)}^{n}={a}^{n}{b}^{n}$ }
%             \nopagebreak
%         \label{m38359*id66288}\begin{enumerate}[noitemsep, label=\textbf{\arabic*}. ] 
%             \label{m38359*uid31}\item 
%             ${\left(2xy\right)}^{3}={2}^{3}{x}^{3}{y}^{3}$
%       \label{m38359*uid32}\item 
%         ${\left(\frac{7a}{b}\right)}^{2}$
%       \label{m38359*uid33}\item 
%         ${\left(5a\right)}^{3}$
% \newline
% \newline
%           \end{enumerate}
%       \label{m38359*uid34}
% \par \raisebox{-5 pt}{\includegraphics[width=0.5cm]{col11306.imgs/summary_www.png}} Find the answers with the shortcodes:
%  \par \begin{tabular}[h]{cccccc}
%  (1.) lOs  & \end{tabular}
            

% \begin{center}
% \begin{array}[lc]
%   \textrm{\textbf{Exponential Law:}} & $$ {\left({a}^{m}\right)}^{n}={a}^{mn} $$
% \end{array}
% \end{center}


\Identity{Eksponentwette}
{
$$ {\left({a}^{m}\right)}^{n}={a}^{mn} $$

Dit is moontlik om die eksponensiaal van ’n eksponensiaal te bereken. So, selfs al klink die eerste sin ingewikkeld, beteken dit bloot dat ’n mens die eksponensiaal van ’n getal
bereken en dan die eksponensiaal van die resultaat bereken. \par


\begin{center}
    \begin{array}{ccll}\hfill {\left({a}^{m}\right)}^{n}& =& {a}^{m}\ensuremath{\times}{a}^{m}\ensuremath{\times}\ldots\ensuremath{\times}{a}^{m}\hfill & \left(\mbox{$n$ keer}\right)\hfill \\
	\hfill & =& a\ensuremath{\times}a\ensuremath{\times}\ldots\ensuremath{\times}a\hfill & \left({m}\ensuremath{\times}\mbox{$n$ keer}\right)\hfill \\
	\hfill & =& {a}^{mn}\hfill & 
    \end{array}
\end{center}

}

% \subsubsubsection{ Exponential Law 6: ${\left({a}^{m}\right)}^{n}={a}^{mn}$}
%             \nopagebreak

\label{m38359*id66694}Byvoorbeeld,

\begin{center}
    \begin{array}{ccl}\hfill {\left({5}^{2}\right)}^{3}& =& \left({5}^{2}\right)\ensuremath{\times}\left({5}^{2}\right)\ensuremath{\times}\left({5}^{2}\right)\hfill \\ 
	      & =& \left(5\ensuremath{\times}5\right)\ensuremath{\times}\left(5\ensuremath{\times}5\right)\ensuremath{\times}\left(5\ensuremath{\times}5\right)\hfill \\
	      & =& \left({5}^{6}\right)\hfill
    \end{array}
\end{center}

\Note{

\begin{center}
    5^2 \times 5^4 & = & 5^{2+4} & = & 5^6  \\
\end{center}

\begin{center}
    (5^2)^4        & = & 5^{2\times4} & = & 5^8 
\end{center}

}

      
% \label{m38359*secfhsst!!!underscore!!!id1894}
%             \subsubsubsection{ Application using Exponential Law 6: ${\left({a}^{m}\right)}^{n}={a}^{mn}$ }
%             \nopagebreak
%         \label{m38359*id66924}\begin{enumerate}[noitemsep, label=\textbf{\arabic*}. ] 
%             \label{m38359*uid36}\item 
%             ${\left({x}^{3}\right)}^{4}$
%       \label{m38359*uid37}\item 
%         ${\left[{\left({a}^{4}\right)}^{3}\right]}^{2}$
%       \label{m38359*uid38}\item 
%         ${\left({3}^{n+3}\right)}^{2}$
% \newline
% \newline
%           \end{enumerate}
% \label{m38359*eip-323}\par
%             \label{m38359*secfhsst!!!underscore!!!id41}\vspace{.5cm} 

\begin{wex}
{ %title
Toepassing van eksponentwette
}
{%question
Vereenvoudig:

\begin{enumerate}[noitemsep, label=\textbf{\arabic*}.]
\item  $2^{3x} \times 2^{4x}$
\\
 \item $\frac{12p^2t^5}{3pt^3}$
\\
 \item $ (3x)^2 $
\\
 \item $(3^4\cdot5^2)^3$\\
\end{enumerate}

}
{%answer

\begin{enumerate}[noitemsep, label=\textbf{\arabic*}.]
\item  $2^{3x} \times 2^{4x} = 2^{3x+4x} = 2^{7x}$
\\
 \item $\frac{12p^2t^5}{3pt^3} = 4p^{(2-1)}t^{(5-3)} = 4pt^2$
\\
 \item $ (3x)^2 = 3^2x^2 = 9x^2$
\\
 \item $(3^4\cdot5^2)^3 = 3^{(4\times3)}\cdot5^{(2\times3)} = 3^{12}\cdot5^6  $
\end{enumerate}



}


\end{wex}









\begin{wex}
{
Eksponentuitdrukkings
}
{
Vereenvoudig: $$\frac{2^{2n} \cdot 4^n \cdot 2 }{ 16^n} $$
}
{

\westep{Verander grondtalle na priemgetalle}

$$ \frac{2^{2n} \cdot 4^n \cdot 2 }{ 16^n} ~~ = ~~ \frac{2^{2n} \cdot (2^2)^n \cdot 2^1 }{ (2^4)^n} $$


\westep{Vereenvoudig die eksponente}

\begin{eqnarray*}
 &= & \frac{ 2^{2n} \cdot 2^{2n} \cdot 2^1 }{ 2^{4n} } \\
& = & \frac{ 2^{2n + 2n +1}}{2^{4n}} \\
& = & \frac{2^{4n+1}}{2^{4n}} \\
& = & 2^{4n+1-(4n)} \\
& = & 2
\end{eqnarray*}
 
}
\end{wex}


     
\begin{wex}
{%title
Eksponentuitdrukkings
}
{
Vereenvoudig: $\dfrac{{5}^{2x-1}\ensuremath{\cdot}{9}^{x-2}}{{15}^{2x-3}}$
}
{
%     \begin{enumerate}[noitemsep, label=\textbf{Step} \textbf{\arabic*}. ] 
%             \leftskip=20pt\rightskip=\leftskip\item  
%     \label{m38359*id118023}\nopagebreak\noindent{}
\westep{Verander grondgetalle na priemgetalle}
      
\begin{center}
\begin{array}{lcl} \dfrac{{5}^{2x-1}\ensuremath{\cdot}{9}^{x-2}}{{15}^{2x-3}}& =& \dfrac{{5}^{2x-1}\ensuremath{\cdot}{\left({3}^{2}\right)}^{x-2}}{{\left(5.3\right)}^{2x-3}}\hfill \\
		  & =& \dfrac{{5}^{2x-1}\ensuremath{\cdot}{3}^{2x-4}}{{5}^{2x-3}\ensuremath{\cdot}{3}^{2x-3}}\hfill 
\end{array}
  \end{center}

  
\westep{Trek eksponente af (Dieselfde grondgetal)}

\begin{center}
\begin{array}{lcl}
\phantom{\dfrac{{5}^{2x-1}\ensuremath{\cdot}{9}^{x-2}}{{15}^{2x-3}}}& =& {5}^{(2x-1)-(2x-3)}\ensuremath{\cdot}{3}^{(2x-4)-(2x-3)}\hfill \\ 
& =& 5^{2x-1-2x+3} \cdot 3^{2x-4 - 2x+3} \\
& =& {5}^{2}\ensuremath{\cdot}{3}^{-1}\hfill \end{array}
\end{center}


\westep{Skryf die antwoord as 'n breuk}  
\begin{eqnarray*}
=\frac{25}{3} \\\\ = 8\frac{1}{3}
\end{eqnarray*}

}
\end{wex}

\Note{Wanneer ons met eksponentuitdrukkings werk, geld al die re\"els vir algebra\"ese bewerkings nog steed.}

\begin{wex}
{%title
Vereenvoudig deur die uithaal van 'n gemene faktor
}
{%question
Vereenvoudig: $$\frac{2^t-2^{t-2}}{3\cdot2^t-2^t} $$

}
{%answer
\westep{Vereenvoudig tot 'n fakotoriseerbare vorm}
\begin{eqnarray*}
\frac{2^t-2^{t-2}}{3\cdot2^t-2^t} & = & \frac{2^t-(2^t\cdot2^{-2})}{3\cdot2^t-2^t} \\
\end{eqnarray*}

\westep{Haal die gemeenskaplike faktor uit}

\begin{eqnarray*}
\phantom{\frac{2^t-2^{t-2}}{3\cdot2^t-2^t}}  & = & \frac{2^t(1-2^{-2})}{2^t(3-1)} \\
\end{eqnarray*}


\westep{Kanselleer die gemeenskaplike faktor uit en vereenvoudig}

\begin{eqnarray*}
\phantom{\frac{2^t-2^{t-2}}{3\cdot2^t-2^t}}  & = & \frac{1-^1/_4}{2} \\
					     & = & \frac{^3/_4}{2} \\
					     & = & \frac{3}{8}
\end{eqnarray*}


} 
\end{wex}




\begin{wex}
{
Vereenvoudig met die gebruik van die verskil van twee kwadrate
}
{
Simplify: 
$$ \frac{9^x-1}{3^x+1} $$
}
{
\westep{Verander grondtalle na preimgetalle}
\begin{eqnarray*}
 \frac{9^x-1}{3^x+1} & = & \frac{(3^2)^x -1}{3^x+1} \\
		     & = & \frac{(3^x)^2-1}{3^x+1} 
\end{eqnarray*}

\westep{Faktoriseer deur die verskil van twee vierkante te gebruik}

\begin{eqnarray*}
 \phantom{\frac{9^x-1}{3^x+1}} & = & \frac{(3^x-1)(3^x+1)}{3^x+1}\\
\end{eqnarray*}

\westep{Vereenvoudig}

\begin{eqnarray*}
 \phantom{\frac{9^x-1}{3^x+1}} & = & 3^x-1\\
\end{eqnarray*}

}
\end{wex}


\begin{exercises}{Vereenvoudig eksponentuitdrukkings}
{
Vereenvoudig sonder om ’n sakrekenaar te gebruik:
\begin{multicols}{2}
\begin{enumerate}[noitemsep, label=\textbf{\arabic*}., itemsep=5pt]
 \item $16^0$
 \item $16a^0$
 \item $\frac{2^{-2}}{3^2}$
 \item $ \frac{5}{2^{-3}}$
 \item $ \Big(\frac{2}{3}\Big)^{-3} $
 \item $ x^2 \cdot x^{3t+1} $
 \item $ 3 \times 3^{2a} \times 3^2$
 \item $ \frac{a^{3x}}{a^x} $
 \item $ \frac{32p^2}{4p^8}$
 \item $ (2t^4)^3$
 \item $ (3^{n+3})^2$
 \item $ \frac{3^n \cdot 9^{n-3}}{27^{n-1}}$
\end{enumerate}
\end{multicols}

 
\insertpracticeinfo{12}

}
\end{exercises}




% \par 
% \label{m38359*secfhsst!!!underscore!!!id2193}
% \par \raisebox{-5 pt}{\includegraphics[width=0.5cm]{col11306.imgs/summary_www.png}} Find the answers with the shortcodes:
%  \par \begin{tabular}[h]{cccccc}
%  (1.) lO6  & \end{tabular}
\section{Rasionale eksponente}

Ons kan ook die eksponentwette boepas op uitdrukkings met rasionale eksponenete.

\mindsetvid{Fractions with exponents}{VMaln}

\begin{wex}
{%title
Vereenvoudig rasionale eksponente
} 
{%question
Vereenvoudig: 
$$ (0.008)^{^1/_3} $$
}
{% answer
\westep{Skryf as 'n breuk en verander grondtalle na priemgetalle}

\begin{eqnarray*}
 0.008 & = & \frac{8}{1000} \\
       & = & \frac{2^3}{10^3} \\
       & = & \Big(\frac{2}{10}\Big)^3\\
\end{eqnarray*}

\westep{Dus}
\begin{eqnarray*}
 (0.008)^{^1/_3} & = & \Big(\Big(\frac{2}{10}\Big)^3\Big)^{^1/_3} \\
		 & = & \frac{2}{10} \\
		 & = & \frac{1}{5}
\end{eqnarray*}
}
\end{wex}

\begin{wex}
{%Title
Vereenvoudig rasionale eksponente 
}

{
Vereenvoudig:
$$ 2x^{^1/_2} \times 4x^{^{-1}/_{2}} $$
}
{% Answer
\westep{Skakel die negatiewe eksponent om na 'n positiewe eksponent}
\begin{eqnarray*}
 2x^{^1/_2} \times 4x^{^{-1}/_{2}} & = & 2x^{^1/_2} \times \frac{4}{x^{^1/_2}} \\
\end{eqnarray*}
Let daarop dat die $4$ bo die breukstreep bly.\\

\westep{Vereenvoudig}
\begin{eqnarray*}
 \phantom{2x^{^1/_2} \times 4x^{^{-1}/_{2}}} & = & \dfrac{8x^{^1/_2}}{x^{^1/_2}} \\
					     & = & 8
\end{eqnarray*}


}
\end{wex}

\begin{exercises}{Vereenvoudig rasionale eksponenete}
{
Vereenvoudig sonder om ’n sakrekenaar te gebruik.:
\begin{multicols}{2}
\begin{enumerate}[noitemsep, label=\textbf{\arabic*}., itemsep=5pt]
 \item $ t^{^1/_4} \times 3t^{^7/_4} $
 \item $ \dfrac{16x^2}{(4x^2)^{^1/_2}} $
 \item $ (0.25)^{^1/_2} $
 \item $ (27)^{^{-1}/_3} $
 \item $ (3p^2)^{^1/_2} \times (3p^4)^{^1/_2} $
\end{enumerate}
\end{multicols}

\insertpracticeinfo{5}
}
\end{exercises}



% \begin{exercises}{Exponential Numbers }
% {
% \label{m38359*id67549}Match the answers to the questions, by filling in the correct answer into the \textbf{Answer} column.
% Possible answers are: $\frac{3}{2}$, $1$, $-1$, $-\frac{1}{3}$, $8$. Answers may be repeated.\par 
% % \textbf{m38359*id67604}\par
%   \begin{table}[H]
% % \begin{table}[H]
% % \\ '' '0'
% \begin{center}
% \label{m38359*id67604}
% \noindent
% \tabletail{%
% \hline
% \multicolumn{2}{|p{\mytableboxwidth}|}{\raggedleft \small \sl continued on next page}\\
% \hline
% }
% \tablelasttail{}
% \begin{xtabular}[t]{|l|l|}\hline
% 	  \textbf{Question}
% 	  &
% 	  \textbf{Answer}
% 	% make-rowspan-placeholders
% \tabularnewline\cline{1-1}\cline{2-2}
% %--------------------------------------------------------------------
% 	  ${2}^{3}$
% 	  &
% % make-rowspan-placeholders
% \tabularnewline\cline{1-1}\cline{2-2}
% %--------------------------------------------------------------------
% 	  ${7}^{3-3}$
% 	  &
% % make-rowspan-placeholders
% \tabularnewline\cline{1-1}\cline{2-2}
% %--------------------------------------------------------------------
% 	  ${\left(\frac{2}{3}\right)}^{-1}$
% 	  &
% % make-rowspan-placeholders
% \tabularnewline\cline{1-1}\cline{2-2}
% %--------------------------------------------------------------------
% 	  ${8}^{7-6}$
% 	  &
% % make-rowspan-placeholders
% \tabularnewline\cline{1-1}\cline{2-2}
% %--------------------------------------------------------------------
% 	  ${\left(-3\right)}^{-1}$
% 	  &
% % make-rowspan-placeholders
% \tabularnewline\cline{1-1}\cline{2-2}
% %--------------------------------------------------------------------
% 	  ${\left(-1\right)}^{23}$
% 	  &
% % make-rowspan-placeholders
% \tabularnewline\cline{1-1}\cline{2-2}
% %--------------------------------------------------------------------
% \end{xtabular}
% \end{center}
% % \begin{center}{\small\bfseries Table 5.1}\end{center}
% % \begin{caption}{\small\bfseries Table 5.1}\end{caption}
% \end{table}
% }
% 
% 
% 
% 
% We will use all these laws in Equations and Inequalities\footnote{\raggedright{}"Equations and Inequalities - Grade 10 [CAPS]" <http://http://cnx.org/content/m38372/latest/>} to help us solve exponential equations.\par \label{m38359*eip-160}



\section{Eksponent vergelykings}

Eksponent vergelykings het die onbekende veranderlike in die eksponent, Hier is 'n voorbeeld:
\begin{eqnarray*}
 3^{x+1} & = & 9 \\
5^t + 3 \cdot 5^{t-1} & = & 400
\end{eqnarray*}

Om eksponent vergelykings op te los, moet ons die eksponentwette toepas. Dit beteken dat ons weerskante van die gelykaan teken 'n enkele term het met dieselfde grondtal, kan ons ide eksponenete gelykstel.

\Note{Indien $a>0$ en $a \ne 0$ en
$$ a^x = a^y $$
dan
$$ x = y ~~\mbox{(dieselfde grondtal)}$$
(Indien $a=1$, dan kan $x$ en $y$ verskil)
}



\begin{wex}
{%title
Stel eksponente gelyk
}
{%question
Los op vir $x$
$$ 3^{x+1} = 9 $$
}
{%answer

\westep{Verander grondtalle na priemgetalle}
\begin{eqnarray*}
 3^{x+1} & = & 3^2 \\
\end{eqnarray*}

\westep{Grondtalle is dieselfde, dus kan ons eksponente gelyk stel}
\begin{eqnarray*}
 {x+1} & = & 2 \\
\therefore x & = & 1
\end{eqnarray*}
}
\end{wex}

\Note{Om eksponent vergelykings op te los, gebruik ons al die prosedures vir die line\^ere en kwadratiese vergelykings.}



\begin{wex}
{%title
Oplos van vergelykings deur die uithaal van 'n gemene faktor
}
{%question
Los op vir $t$
$$ 5^t + 3 \cdot 5^{t+1} = 400 $$
}
{%answer

\westep{Herskryf die uitdrukking}
\begin{eqnarray*}
 5^t + 3 ( 5^t \cdot 5) & = & 400 \\
\end{eqnarray*}

\westep{Haal 'n gemeenskaplike faktoor uit}
\begin{eqnarray*}
 5^t(1 + 3 \cdot 5) & = & 400 \\
\end{eqnarray*}


\westep{Vereenvoudig}

\begin{eqnarray*}
 5^t(16) & = & 400 \\
  5^t & = & 25 \\
\end{eqnarray*}


\westep{Verander grondtalle na priemgetalle}
\begin{eqnarray*}
  5^t & = & 5^2 \\
\end{eqnarray*}


\westep{Grondtalle is dieselfde, dus kan ons eksponente gelyk stel}
\begin{eqnarray*}
\therefore t & = & 2
\end{eqnarray*}

}
\end{wex}

\begin{wex}
{%title
Los vergelykings op deur die faktoisering van 'n kwadratiese drieterm
}
{%question
Los op
$$ p-13 p^{^1/_2} + 36 =  0$$
}
{ % Answer

\westep{Let po dat $(p^{^1/_2})^2=p$ so ons kan die vergelyking herskryf as}

$$ (p^{^1/_2})^2 -13p^{^1/_2} + 36 = 0 $$

\westep{Faktoriseer as 'n drieterm}

$$ (p^{^1/_2} -9)(p^{^1/_2}-4) = 0 $$

\westep{Los op om beide wortels te vind}

%Left Column				%right column
\begin{align*}
p^{^1/_2} - 9 &= 0			&   p^{^1/_2} - 4 &= 0		\\
p^{^1/_2} &= 9				&   p^{^1/_2} &= 4		\\		
(p^{^1/_2})^2 &= (9)^2			&   (p^{^1/_2})^2 &= (4)^2\\
p &= 9^2				&   p &= 16\\
p &= 81					&\\
\end{align*} 
Therefore $p=81$ or $p=16$


}
\end{wex}

\begin{exercises}{Solving exponential equations}
{
Los op:
\begin{multicols}{2}
\begin{enumerate}[noitemsep, label=\textbf{\arabic*}., itemsep=5pt]
\item $ 2^{x+5} = 32 $
\item $ 5^{2x+2} = \dfrac{1}{125} $
\item $ 64^{y+1} = 16^{2y+5} $
\item $ 3^{9x-2} = 27 $
\item $ 81^{k+2} = 27^{k+4} $
\setcounter{enumi}{6}
\item $ 25^{(1-2x)}-5^4 = 0 $
\item $ 27^x \times 9^{x-2} = 1 $
\item $ 2^t + 2^{t+2} = 40 $
\item $ 2 \cdot 5^{2-x} = 5+ 5^X $
\item $ 9^m + 3^{3-2m} = 28 $
\item $ y - 2y^{^1/_2} + 1 = 0 $
\end{enumerate}
\end{multicols}

\begin{enumerate}[noitemsep, label=\textbf{\arabic*}., itemsep=5pt]
\setcounter{enumi}{5}
 \item Die groei van alge kan gemodelleer word met die funksie $f(t) = 2^t$. Vind die waarde van $t$ as $f(t)=128$.   
\end{enumerate}

\insertpracticeinfo{12}
}
\end{exercises}




\summary{VMdgh}



\begin{itemize}[noitemsep, label=\textbullet{}]
    \item Eksponensiaalnotasie verwys na ’n getal wat geskryf word as ${a}^{n}$ waar $n$ ’n heelgetal is en $a$ enige reële getal is.
    \item $a$ is die \textsl{grondtal} en $n$ is die \textsl{eksponent} of \textsl{indeks}.
    \item Definisie: 
	  \begin{itemize}[noitemsep]
	   \item ${a}^{n}=a\ensuremath{\times}a\ensuremath{\times}\cdots \ensuremath{\times}a\phantom{\rule{2.em}{0ex}}\left(\mbox{$n$ times}\right)$
	   \item ${a}^{0}=1,a\ne 0$
	   \item ${a}^{-n}=\frac{1}{{a}^{n}},a\ne 0$
	  \end{itemize}

    
    \item  Die eksponentewette: 
	\begin{itemize}[itemsep=4pt]
	    \item  ${a}^{m}\ensuremath{\times}{a}^{n}={a}^{m+n}$
	    \item  ${\dfrac{{a}^{m}}{{a}^{n}}={a}^{m-n}$
	    \item  ${\left(ab\right)}^{n}={a}^{n}{b}^{n}$
            \item  $\Big(\dfrac{a}{b}\Big)^n = \dfrac{a^n}{b^n}$
	    \item  ${\left({a}^{m}\right)}^{n}={a}^{mn}$
	\end{itemize}
    \end{itemize}


\begin{eocexercises}{}
%  $ \hspace{-5pt}\begin{array}{cccccccccccc}   \end{array} $ 
% \hspace{2 pt}\raisebox{-5 pt}{\includegraphics[width=0.5cm]{col11306.imgs/summary_www.png}} {(section shortcode: MG10051 )} \par 
% \label{m38359*id67892}


Vereenvoudig:
\begin{multicols}{2}
\begin{enumerate}[noitemsep, label=\textbf{\arabic*}., itemsep=5pt]
\item $ t^3 \times 2t^0 $
\item $ 5^{2x+y} \cdot 5^{3(x+z)} $
\item $ (b^{k+1})^k $
\item $ \dfrac{6^{5p}}{9^p} $
\item $ m^{-2t} \times (3m^t)^3 $
\item $\dfrac{3{x}^{-3}}{{\left(3x\right)}^{2}}$
\item $\dfrac{{5}^{b-3}}{{5}^{b+1}}$
\item $\dfrac{{2}^{a-2}.{3}^{a+3}}{{6}^{a}}$
\item $\dfrac{{3}^{n}\ensuremath{\cdot}{9}^{n-3}}{{27}^{n-1}}$
\item ${\left(\dfrac{2{x}^{2a}}{{y}^{-b}}\right)}^{3}$
\item $\dfrac{{2}^{3x-1}\ensuremath{\cdot}{8}^{x+1}}{{4}^{2x-2}}$
\item $\dfrac{{6}^{2x}\ensuremath{\cdot}{11}^{2x}}{{22}^{2x-1}\ensuremath{\cdot}{3}^{2x}}$
\item $\dfrac{{\left(-3\right)}^{-3}\ensuremath{\cdot}{\left(-3\right)}^{2}}{{\left(-3\right)}^{-4}}$
\item ${\left({3}^{-1}+{2}^{-1}\right)}^{-1}$
\item $\dfrac{{9}^{n-1}\ensuremath{\cdot}{27}^{3-2n}}{{81}^{2-n}}$
\item $\dfrac{{2}^{3n+2}\ensuremath{\cdot}{8}^{n-3}}{{4}^{3n-2}}$
\item $\dfrac{3^{t+3} + 3^t}{2 \cdot 3^t} $
\item $\dfrac{2^{3p} +1}{2^p + 1} $
\end{enumerate}
\end{multicols}



Los op:
\begin{multicols}{2}
\begin{enumerate}[noitemsep, label=\textbf{\arabic*}., itemsep=5pt]
\setcounter{enumi}{18}
\item $ 3^x = \dfrac{1}{27} $
\item $ 5^{t-1} = 1 $
\item $ 2 \cdot 7^{3x} = 98 $
\item $ 2^{m+1} = (0.5)^{m-2}$
\item $ 3^{y+1} = 5^{y+1} $
\item $ z^{^3/_2} = 64 $
\item $ 16x^{^1/_2} - 4 = 0 $
\item $ m^0 + m^{-1} = 0 $
\item $ t^{^1/_2} - 3t^{^1/_4} + 2 = 0 $
\item $ 3^p + 3^p + 3^p = 27 $
\item $ k^{-1} - 7x^{^{-1}/_2} -18 = 0 $
\item $ x^{^1/_2}+3x^{^1/_4}-18 = 0 $
\end{enumerate}
\end{multicols}


\end{eocexercises}

% \begin{enumerate}[noitemsep, label=\textbf{\arabic*}. ] 
% %     \item Simplify as far as possible:
% % 	\begin{enumerate}[noitemsep, label=\textbf{\alph*}. ] 
% % 	    \item ${302}^{0}$
% % 	    \item ${1}^{0}$
% % 	    \item ${\left(xyz\right)}^{0}$
% % 	    \item ${\left[{\left(3{x}^{4}{y}^{7}{z}^{12}\right)}^{5}{\left(-5{x}^{9}{y}^{3}{z}^{4}\right)}^{2}\right]}^{0}$
% % 	    \item ${\left(2x\right)}^{3}$
% % 	    \item ${\left(-2x\right)}^{3}$
% % 	    \item ${\left(2x\right)}^{4}$
% % 	    \item ${\left(-2x\right)}^{4}$
% % 	\end{enumerate}
%     \item Simplify
%     \item Simplify without using a calculator. Leave your answers with positive exponents.
% 	\begin{enumerate}[noitemsep, label=\textbf{\alph*}. ] 
% 	    \item $\dfrac{3{x}^{-3}}{{\left(3x\right)}^{2}}$
% 	    \item $5{x}^{0}+{8}^{-2}-{\left(\frac{1}{2}\right)}^{-2}\ensuremath{\cdot}{1}^{x}$
% 	    \item $\dfrac{{5}^{b-3}}{{5}^{b+1}}$
% 	\end{enumerate}
%     \item Simplify, showing all steps:
% 	\begin{enumerate}[noitemsep, label=\textbf{\alph*}. ] 
% 	    \item $\dfrac{{2}^{a-2}.{3}^{a+3}}{{6}^{a}}$
% 	    \item $\dfrac{{a}^{2m+n+p}}{{a}^{m+n+p}\ensuremath{\cdot}{a}^{m}}$
% 	    \item $\dfrac{{3}^{n}\ensuremath{\cdot}{9}^{n-3}}{{27}^{n-1}}$
% 	    \item ${\left(\dfrac{2{x}^{2a}}{{y}^{-b}}\right)}^{3}$
% 	    \item $\dfrac{{2}^{3x-1}\ensuremath{\cdot}{8}^{x+1}}{{4}^{2x-2}}$
% 	    \item $\dfrac{{6}^{2x}\ensuremath{\cdot}{11}^{2x}}{{22}^{2x-1}\ensuremath{\cdot}{3}^{2x}}$
% 	\end{enumerate}
%     \item Simplify, without using a calculator:
% 	\begin{enumerate}[noitemsep, label=\textbf{\alph*}. ] 
% 	    \item $\dfrac{{\left(-3\right)}^{-3}\ensuremath{\cdot}{\left(-3\right)}^{2}}{{\left(-3\right)}^{-4}}$
% 	    \item ${\left({3}^{-1}+{2}^{-1}\right)}^{-1}$
% 	    \item $\dfrac{{9}^{n-1}\ensuremath{\cdot}{27}^{3-2n}}{{81}^{2-n}}$
% 	    \item $\dfrac{{2}^{3n+2}\ensuremath{\cdot}{8}^{n-3}}{{4}^{3n-2}}$
% 	\end{enumerate}
% \end{enumerate}
% 
% \raisebox{-5 pt}{\includegraphics[width=0.5cm]{col11306.imgs/summary_www.png}} Find the answers with the shortcodes:
% \begin{tabular}[h]{cccccc}
% (1.) lOJ  &  (2.) lOu  &  (3.) lOS  &  (4.) lOh  & 
% \end{tabular}
