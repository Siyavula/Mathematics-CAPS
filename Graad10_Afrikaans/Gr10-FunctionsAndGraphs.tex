\chapter{Functions}
\setcounter{figure}{0}
\setcounter{subfigure}{0}

\section{Functions in the real world}
Functions are mathematical building blocks for designing machines, predicting natural disasters, curing diseases, understanding world economies and for keeping aeroplanes in the air. Functions can take input from many variables, but always give the same output, unique to that function. It is the fact that you always get a predictable answer from a set of inputs that makes functions special.\par 
A major advantage of functions is that they allow us to visualise relationships in terms of a graph. A graph is a two-dimensional representation of a function in the Cartesian plane. Functions are much easier to read and interpret than lists of numbers.\par 
In addition to their use in the problems facing humanity, functions also appear on a day-to-day level, so they are worth learning about. A function is always dependent on one or more variables, like time, distance or a more abstract quantity.\par 

Some examples of functions include:\par 
\begin{itemize}[noitemsep]
\item Money as a function of time. You never have more than one amount of money at any time because you can always add everything to give one number. By understanding how your money changes over time, you can plan to spend your money sensibly. Businesses find it very useful to plot the graph of their money over time so that they can see when they are spending too much. Such observations are not always obvious from looking at the numbers alone.
\item Temperature as a function of various factors. Temperature is a very complicated function because it has so many inputs, including; the time of day, the season, the amount of clouds in the sky, the strength of the wind, where you are and many more. But the important thing is that there is only one temperature when you measure it in a specific place. By understanding how the temperature is affected by these things, you can plan
for the day.
\item Location as a function of time. You can never be in two places at the same time. If you were to plot the graphs of where two people are as a function of time, the place where the lines cross means that the two people meet each other at that time. This idea is used in logistics, an area of mathematics that tries to plan where people and items are for businesses.
\item Your weight as a function of how much you eat and how much exercise you do, so it differs from person to person.
\end{itemize}

\Definition{function}{A function is a mathematical relationship between two variables, where every input variable has one output.}
\Note{One output can have different input variables.}




\subsection*{Dependent and independent variables}
In functions, the $x$-value is known as the input or independent variable, because its value can be chosen freely. The calculated $y$-value is known as the output or dependent variable, because its value depends on the chosen input value.\par 

\subsection*{Domain and range}

The domain of a function is the set of all input values ($x$-values) for which there is at least one output value ($y$-value) value according to that relation. 
\\The range is the set of all output values ($y$-values) which can be obtained using at least one input value ($x$-value).\par 



\subsection*{Set notation}
Examples:
\\
\begin{table}[H]
\begin{tabular}{ |p{5cm} | p{8cm} | }
\hline
  $\{x: x \in \mathBB{R}, x > 0\}$ &  The set of all $x$-values such that $x$ is an element of the set of real numbers and is greater than $0$
\\ \hline
    $\{y: y \in \mathBB{N}, 3 < y \leq 5}$ & The set of all $y$-values such that $y$ is an element of the set of natural numbers, is greater than $3$ and less than or equal to $5$. 
\\ \hline
  $\{z: z \in \mathBB{Z}, z \leq 100}$ & The set of all $z$-values such that $z$ is an element of the set of integers and is less than or equal to $100$.  
\\ \hline
\end{tabular}
\end{table}
\subsection*{Interval notation}
It is important to note that this notation can only be used to represent an interval of real numbers.
Examples:
\\
\begin{table}[H]
\begin{tabular}{ |p{5cm} | p{8cm} | }
\hline
  $(4;12)$ &  Round brackets indicate that the number is not included. This interval indicates all real numbers greater than, but not equal to $4$ and less than, but not equal to $12$.
\\ \hline
 $(- \infty; -1)$ & Round brackets are always used for positive and negative infinity. 
\\ \hline
 $[1; 13)$ & A square bracket indicates that the number is included. This interval indicates all real numbers greater than, and equal to $1$ and less than, but not equal to $13$.
\\ \hline
\end{tabular}
\end{table}

\subsection*{Function notation}
This is a very useful way to express a function. Instead of writing $y=2x+1$ we write $f(x) = 2x+1$. We say ``$f$ of $x$ is equal to $2x+1$''. Any letter can be used, e.g. $g(x)$, $h(x)$, $p(x)$ etc. 
\begin{enumerate}[noitemsep, label=\textbf{\arabic*}. ] 
 \item Determine the output value: ``find the value of the function for $x=-3$'' can be written as`` find $f(-3)$''.
\\Replace $x$ with $-3$: $f(-3)=2(-3)+1=-5 \therefore f(-3)=-5$. \\
This means that when $x=-3$ the value of the function $(y)$ $~=5$.
\item Determine the input value: ``find the value of $x$ that will give a $y$-value of $27$'' can be written as ``find $x$ if $f(x)=2x+1 = 27$''. \\
We write the following equation and solve for $x$: $2x+1 = 27 \therefore x=13$.\\
This means that when $x=13$ the value of the function $(y)$ $~=27$.

\end{enumerate}
\Note{$x$-value = input value\\
$y$-value = output/function value}

\subsection*{Representations of functions}
Functions can be expressed in many different ways for different purposes. 
\begin{enumerate}[noitemsep, label=\textbf{\arabic*}. ] 
 \item Words: the relationship between two variables is such that one is always $5$ less than the other.
\item Mapping diagram: 
\begin{figure}[H]
\begin{center}
\scalebox{1} % Change this value to rescale the drawing.
{
\begin{pspicture}(0,-1.1360937)(6.7240624,1.1360937)
\psframe[linewidth=0.04,dimen=outer](3.7046876,0.32890624)(2.6246874,-0.75109375)
\psline[linewidth=0.04cm,arrowsize=0.05291667cm 2.0,arrowlength=1.4,arrowinset=0.4]{->}(4.1446877,0.14890625)(5.0246873,0.40890625)
\psline[linewidth=0.04cm,arrowsize=0.05291667cm 2.0,arrowlength=1.4,arrowinset=0.4]{->}(4.1446877,-0.57109374)(5.0046873,-0.85109377)
\psline[linewidth=0.04cm,arrowsize=0.05291667cm 2.0,arrowlength=1.4,arrowinset=0.4]{->}(4.1046877,-0.21109375)(4.9846873,-0.21109375)
\psline[linewidth=0.04cm,arrowsize=0.05291667cm 2.0,arrowlength=1.4,arrowinset=0.4]{<-}(2.073937,-0.58032197)(1.2008146,-0.86255836)
\psline[linewidth=0.04cm,arrowsize=0.05291667cm 2.0,arrowlength=1.4,arrowinset=0.4]{<-}(2.055675,0.13944638)(1.1888499,0.39754358)
\psline[linewidth=0.04cm,arrowsize=0.05291667cm 2.0,arrowlength=1.4,arrowinset=0.4]{<-}(2.104793,-0.21942326)(1.2250762,-0.24174327)
% \usefont{T1}{ptm}{m}{n}
\rput(0.44671875,0.93890625){Input:}
% \usefont{T1}{ptm}{m}{n}
\rput(6.0242186,0.93890625){Output:}
\usefont{T1}{ptm}{m}{n}
\rput(0.82609373,0.41890624){$-3$}
\usefont{T1}{ptm}{m}{n}
\rput(0.7753125,-0.22109374){$0$}
\usefont{T1}{ptm}{m}{n}
\rput(0.82953125,-0.92109376){$5$}
\usefont{T1}{ptm}{m}{n}
\rput(3.1746874,-0.19609375){\large $-5$}
\usefont{T1}{ptm}{m}{n}
\rput(5.368125,0.45890626){$-8$}
\usefont{T1}{ptm}{m}{n}
\rput(5.3646874,-0.24109375){$-5$}
\usefont{T1}{ptm}{m}{n}
\rput(5.3553123,-0.98109376){$0$}
\end{pspicture} 
}
\end{center}
\end{figure}
\item Table: 

 \begin{table}[H]
\begin{center}
  \begin{tabular}{|c|c|c|c|}
   \hline
Input variable & $-3$&$0$&$5$
\\ \hline
Output variable &$-8$&$-5$&$0$
\\ \hline
  \end{tabular}
\end{center}
 \end{table}



\item Set of ordered number pairs: $(-3;-8), (0;-5), (5;0)$
\item Algebraic formula: $y = x-5$, $x \in \{-3; 0; 5\}$
\item Graph:
\begin{figure}[H]
\begin{center}
\scalebox{1} % Change this value to rescale the drawing.
{
\begin{pspicture}(0,-3.0284376)(6.52125,3.0684376)
\rput(3.0,-0.0284375){\psaxes[linewidth=0.04,arrowsize=0.05291667cm 2.0,arrowlength=1.4,arrowinset=0.4,labels=none,ticks=none,ticksize=0.10583333cm]{<->}(0,0)(-3,-3)(3,3)}
\usefont{T1}{ptm}{m}{n}
\rput(6.3754687,0.1615625){$x$}
\usefont{T1}{ptm}{m}{n}
\rput(3.3557813,2.9415624){$y$}
\psline[linewidth=0.04cm,arrowsize=0.05291667cm 2.0,arrowlength=1.4,arrowinset=0.4]{<->}(2.0,-2.0884376)(5.18,0.8915625)
\psdots[dotsize=0.16](2.98,-1.1684375)
\psdots[dotsize=0.16](4.16,-0.0284375)
\usefont{T1}{ptm}{m}{n}
\rput(4.1448436,0.2415625){$5$}
\usefont{T1}{ptm}{m}{n}
\rput(3.3,-1.2384375){$-5$}
\usefont{T1}{ptm}{m}{n}
\rput(2.9814062,-0.0584375){$O$}
\end{pspicture} 
}
\end{center}
\end{figure}
\end{enumerate}

\begin{exercises}{}
{
Write the following in set notation:
\begin{enumerate}[noitemsep, label=\textbf{\arabic*}. ] 
 \item $(- \infty; 7]$
\item $[13;4)$
\item $(35; \infty)$
\item $[\frac{3}{4}; 21)$
\item $[-\frac{1}{2}; \frac{1}{2}$
\item $(-3; \infty)$
\end{enumerate}

Write the following in interval notation:
\begin{enumerate}[noitemsep, label=\textbf{\arabic*}. ] 
\setcounter{enumi}{6}
 \item $\{p: p \in \mathBB{R}, p \leq 6\}$
 \item $\{k: k \in \mathBB{R}, -5 < k < 5}$
 \item $\{x: x \in \mathBB{R}, x > \frac{1}{5}\}$
 \item $\{z: z \in \mathBB{R}, 21 \leq z < 41\}$
\end{enumerate}

} 
\end{exercises}

\section{Linear functions of the form $y=mx+c$}


\subsection*{Plotting the graph}       
Functions with a general form of $y=mx+c$ are called straight line functions. In the equation, $y=mx+c$, $m$ and $c$ are constants and have different effects on the graph of the function. 

\begin {wex}{Plotting a straight line}
{
\begin{figure}[H]
\begin{center}
\begin{pspicture}(-6,-2)(6,4.6)
\psset{yunit=0.3}
\psaxes[Dy=3]{<->}(0,0)(-6,-8)(6,14)
\psplot[plotstyle=curve,arrows=<->]{-5}{5}{x 2 mul 3 add}
\psplot[plotstyle=dots,arrows=<->,plotpoints=9]{-4}{4}{x 2 mul 3 add}
\rput(0, 0){$0$}
\rput(0.3, 13.6){$y$}
\rput(6.3, 0.3){$x$}
\rput(4.5, 9){$f(x)=2x+3$}
%\psplot[showpoints=true](-5,-7)(-5,-7)(-4,-5)(-3,-3)(-2,-1)(-1,1)(-1.5,0)(0,3)(1,5)(2,7)(3,9)(4,11)(5,13)
\rput(-1.6,1.5){$-\frac{3}{2}$}
\end{pspicture}
% \caption{Graph of $f(x) = 2x + 3$}
% \label{fig:mf:g:straightline}
\end{center}
\end{figure}  
}
{
\westep{Step 1}
Steps needed here
}
\end{wex}

  

\subsection*{Investigating the effects of $m$ and $c$}

\begin{Investigation}{The effects of $m$ and $c$ on a straight line graph}

On the same set of axes, plot the following graphs:
\begin{enumerate}[noitemsep, label=\textbf{\arabic*}. ] 

    \item $a(x)=x-2$
    \item $b(x)=x-1$
    \item $c(x)=x$
    \item $d(x)=x+1$
    \item $e(x)=x+2$
    \end{enumerate}
Use your results to deduce the effect of different values of $c$ on the resulting graph.\\
\\

On the same set of axes, plot the following graphs:

    \begin{enumerate}[noitemsep, label=\textbf{\arabic*}. ] 
\setcounter{enumi}{5}
    \item $f(x)=-2x$
    \item $g(x)=-x$
    \item $j(x)=x$
\item $k(x)=2x$
    \end{enumerate}
Use your results to deduce the effect of different values of $m$ on the resulting graph.
\end{Investigation}


\begin{table}[htb]
\begin{center}
% \caption{Table summarising general shapes and positions of graphs of functions of the form $y=mx+q=c$.}
\label{tab:mf:graphs:summarystr10}
\begin{tabular}{|c|c|c|c|}\hline
& $m<0$&$m=0$ & $m>0$\\ \hline
$c>0$&
\begin{pspicture}(-1.2,-1.2)(1.2,1.2)
\psset{yunit=0.25,xunit=0.25}
\psaxes[arrows=<->,dx=0,Dx=10,dy=0,Dy=10](0,0)(-4,-4)(4,4)
\psplot[plotstyle=curve,arrows=<->]{-2.5}{2.5}{x neg 1 add}
\end{pspicture}

&
\begin{pspicture}(-1.2,-1.2)(1.2,1.2)
\psset{yunit=0.25,xunit=0.25}
\psaxes[arrows=<->,dx=0,Dx=10,dy=0,Dy=10](0,0)(-4,-4)(4,4)
\psplot[plotstyle=curve,arrows=<->]{-2.5}{2.5}{1.5}
\rput(-0.5, 2.1){\footnotesize$q$}
\end{pspicture}
&
\begin{pspicture}(-1.2,-1.2)(1.2,1.2)
\psset{yunit=0.25,xunit=0.25}
\psaxes[arrows=<->,dx=0,Dx=10,dy=0,Dy=10](0,0)(-4,-4)(4,4)
\psplot[plotstyle=curve,arrows=<->]{-2.5}{2.5}{x 1 add}
\end{pspicture}
\\\hline
$c=0$&
\begin{pspicture}(-1.2,-1.2)(1.2,1.2)
\psset{yunit=0.25,xunit=0.25}
\psaxes[arrows=<->,dx=0,Dx=10,dy=0,Dy=10](0,0)(-4,-4)(4,4)
\psplot[plotstyle=curve,arrows=<->]{-2.5}{2.5}{x neg}
\end{pspicture}

&
---
&

\begin{pspicture}(-1.2,-1.2)(1.2,1.2)
\psset{yunit=0.25,xunit=0.25}
\psaxes[arrows=<->,dx=0,Dx=10,dy=0,Dy=10](0,0)(-4,-4)(4,4)
\psplot[plotstyle=curve,arrows=<->]{-2.5}{2.5}{x}
\end{pspicture}
\\ \hline
$c<0$
&

\begin{pspicture}(-1.2,-1.2)(1.2,1.2)
\psset{yunit=0.25,xunit=0.25}
\psaxes[arrows=<->,dx=0,Dx=10,dy=0,Dy=10](0,0)(-4,-4)(4,4)
\psplot[plotstyle=curve,arrows=<->]{-2.5}{2.5}{x neg 1 sub}
\end{pspicture}
&
\begin{pspicture}(-1.2,-1.2)(1.2,1.2)
\psset{yunit=0.25,xunit=0.25}
\psaxes[arrows=<->,dx=0,Dx=10,dy=0,Dy=10](0,0)(-4,-4)(4,4)
\psplot[plotstyle=curve,arrows=<->]{-2.5}{2.5}{1.5 neg}
\rput(-0.5, -2.1){\footnotesize$q$}
\end{pspicture}
&
\begin{pspicture}(-1.2,-1.2)(1.2,1.2)
\psset{yunit=0.25,xunit=0.25}
\psaxes[arrows=<->,dx=0,Dx=10,dy=0,Dy=10](0,0)(-4,-4)(4,4)
\psplot[plotstyle=curve,arrows=<->]{-2.5}{2.5}{x 1 sub}

\end{pspicture}
\\\hline
\end{tabular}
\end{center}
\end{table}

\textbf{The effect of $m$}\\
We notice that the value of $m$ affects the slope of the graph. As $m$ increases, the slope of the graph increases. If $m>0$ then the graph increases from left to right (slopes upwards). If $m<0$ then the graph increases from right to left (slopes downwards). For this reason, $m$ is referred to as the slope or gradient of a straight-line function.\par 

\textbf{The effect of $c$}\\
We also notice that the value of $c$ affects where the graph cuts the $y$-axis. For this reason, $c$ is known as the y-intercept. If $c$ increases the graph shifts vertically upwards. If $c$ decreases, the graph shifts vertically downwards.\par 

% These different properties are summarised in Table 1.7.\par 

\subsection*{Discovering the characteristics} 
The standard form of a straight line graph is the equation $y=mx + c$. 
\subsection*{Domain and range}
\nopagebreak
For $f(x)=mx+c$, the domain is $\{x:x\in \mathbb{R}\}$ because there is no value of $x\in \mathbb{R}$ for which $f(x)$ is undefined.\par 
The range of $f(x)=mx+c$ is also $\{f(x):f(x)\in \mathbb{R}\}$ because there is no value of $f(x)\in \mathbb{R}$ for which $f(x)$ is undefined.\par 
\par 

\subsection*{Intercepts}
\textbf{The $y$-intercept}\\
Every point on the $y$-axis has an $x$-coordinate of $0$. Therefore, to calculate the $y$-intercept, left $x=0$.\par
For example, the $y$-intercept of $g(x)=x-1$ is given by setting $x=0$ to get:\par 

\begin{equation*}
\begin{array}{ccl}\hfill g(x)& =& x-1\hfill \\
\hfill g(0) &=& 0-1\hfill \\
& =& -1\hfill 
\end{array}
\end{equation*}
This gives the point $(0;-1)$.\par

\textbf{The $x$-intercept}\\
Every point on the $x$-axis has a $y$-coordinate of $0$. Therefore to calculate the $x$-intercept, let $y=0$. \par
For example, the $x$-intercepts of $g(x)=x-1$ is given by setting $y=0$ to get:\par 

\begin{equation*}
\begin{array}{ccl}\hfill g(x)& =& x-1\hfill \\
\hfill 0& =& x-1\hfill \\
\hfill x& =& 1\hfill 
\end{array}
\end{equation*}
This gives the point $(1;0)$.


\subsection*{Sketching graphs of the form $f(x)=mx+c$}


In order to sketch graphs of the form, $f(x)=mx+c$, we need to determine three characteristics:\par 
\begin{enumerate}[noitemsep, label=\textbf{\arabic*}. ] 
\item sign of $m$
\item $y$-intercept
 and axis of symmetry\item $x$-intercept
\end{enumerate}
Only two points are needed to plot a straight line graph. The easiest points to use are the $x$-intercept and the $y$-intercept.\par 

\subsection*{Dual intercept method}
\begin{wex}{Sketching a straigh line graph using the dual intercept method}
{Sketch the graph of $g(x)=x-1$ using the dual intercept method}
{
\westep{Examine the standard form of the equation}
$m>0$. This means that the graph increases as $x$ increases.

\westep{Calculate intercepts}
For $y$-intercept, let $x=0$; therefore $g(0)=-1$.This gives the point $(0;-1)$. \\

For $x$-intercept, let $y=0$; therefore $x=1$.This gives the point $(1;0)$. 

\westep{Plot points and draw graph}

\begin{center}
\begin{pspicture}(-4,-4)(4,3)
%\psgrid
\psset{yunit=0.75,xunit=0.75}
\psaxes[arrows=<->](0,0)(-5,-5)(5,4)
\psplot[plotstyle=curve,arrows=<->]{-4}{4}{x 1 sub}
\psdots(0,-1)(1,0)
\uput[r](0,-1.3){$(0;-1)$}
\uput[ul](1.3,0.3){$(1;0)$}
\rput(5.2,0.3){$x$}
\rput(0.3, 4.2){$y$}
\end{pspicture}
% \caption{Graph of the function $g(x)=x-1$}
% \label{fig:mf:g:sketchexamplestr}
\end{center}

Note: this graph is continuous and extends in both directions.       
}
\end{wex}

\subsection*{Gradient and $y$-intercept method}
We can draw a straight line graph of the form $y=mx+c$ using the gradient $m$ and the $y$-intercept $c$. \par We calculate the $y$-intercept by letting $x=0$. This gives us one point for drawing the graph and to calculate the second we use the gradient ($m$).\par

The gradient of a line is the measure of steepness. Steepness is determined by the ratio of vertical change to horizontal change:
\begin{equation*}
m = \dfrac{\mbox{\footnotesize change in $y$}}{\mbox{\footnotesize change in $x$}} = \dfrac{\mbox{\footnotesize vertical change}}{\mbox{\footnotesize horizontal change}}
\end{equation*}
For example: $y=\frac{3}{2}x-1$, therfore $m > 0$ and $y$ increases as $x$ increases
\begin{equation*}
 m = \dfrac{\mbox{\footnotesize change in $y$}}{\mbox{\footnotesize change in $x$}} = \dfrac{3\uparrow}{2\rightarrow} = \dfrac{-3\downarrow}{-2\leftarrow}
\end{equation*}
\begin{center}
\scalebox{1} % Change this value to rescale the drawing.
{
\begin{pspicture}(0,-4.7584376)(9.34125,4.7584376)
\rput(4.0,0.3215625){\psaxes[linewidth=0.04,labels=none,ticksize=0.2cm, arrows=<->](0,0)(-4,-5)(5,4)}
\psline[linewidth=0.04cm,dotsize=0.07055555cm 2.0]{*-*}(1.92,-3.6984375)(5.96,2.3815625)
\psarc[linewidth=0.04](4.49,2.3515625){0.49}{0.0}{180.0}
\psarc[linewidth=0.04](5.51,2.3315625){0.51}{0.0}{180.0}
\rput{178.47418}(6.9182076,-7.536676){\psarc[linewidth=0.04](3.509283,-3.7222767){0.49}{0.0}{180.0}}
\rput{178.47418}(4.8816133,-7.4152513){\psarc[linewidth=0.04](2.4901774,-3.675124){0.51}{0.0}{180.0}}
\rput{88.38591}(3.753576,-4.18539){\psarc[linewidth=0.04](4.029283,-0.1622766){0.49}{0.0}{180.0}}
\rput{273.13806}(6.9809837,0.9839488){\psarc[linewidth=0.04](4.010177,-3.195124){0.51}{0.0}{180.0}}
\rput{273.13806}(5.9635777,1.9092364){\psarc[linewidth=0.04](3.9901774,-2.195124){0.51}{0.0}{180.0}}
\rput{273.13806}(4.9839826,2.874464){\psarc[linewidth=0.04](4.010177,-1.1951239){0.51}{0.0}{180.0}}
\rput{88.38591}(4.694869,-3.1535814){\psarc[linewidth=0.04](3.969283,0.8377234){0.49}{0.0}{180.0}}
\rput{88.38591}(5.6944723,-2.1817489){\psarc[linewidth=0.04](3.969283,1.8377234){0.49}{0.0}{180.0}}
\usefont{T1}{ptm}{m}{n}
\rput(9.195469,0.6515625){$x$}
\usefont{T1}{ptm}{m}{n}
\rput(4.355781,4.6315627){$y$}
\usefont{T1}{ptm}{m}{n}
\rput(5.1275,3.1515625){$+2\rightarrow$}
\usefont{T1}{ptm}{m}{n}
\rput(2.9310937,0.9715625){$+3\uparrow$}
\usefont{T1}{ptm}{m}{n}
\rput(5.161406,-2.1884375){$-3\downarrow$}
\usefont{T1}{ptm}{m}{n}
\rput(2.9578125,-4.6084375){$-2\leftarrow$}
\end{pspicture} 
}
\end{center}

\begin{wex}{Sketching a straigh line graph using the gradient-intercept method}
{Sketch the graph of $y=\frac{1}{2}x-3$ using the gradient-intercept method}
{
\westep{Use the intercept}
The value of $c$ tells us the $y$-intercept, where the graph cuts the $y$-axis. \\
$c=-3$ which gives us the point $(0;-3)$.

\westep{Use the gradient}

\begin{equation*}
 m = \dfrac{\mbox{\footnotesize change in $y$}}{\mbox{\footnotesize change in $x$}} = \dfrac{1\uparrow}{2\rightarrow} = \dfrac{-1\downarrow}{-2\leftarrow}
\end{equation*}
We start at  $(0;-3)$. Move $1$ unit up and $2$ units to the right. This gives us the second point $(2;-2)$. \\
Or move $1$ unit down and $2$ units to the lift. This gives us the second point $(-1,-4)$.

\westep{Plot points and draw graph}

\begin{center}
\begin{pspicture}(-5,-5)(5,5)
%\psgrid
\psset{yunit=0.75,xunit=0.75}
\psaxes[arrows=<->](0,0)(-6,-6)(7,4)
\psplot[plotstyle=curve,arrows=<->]{-4}{7}{x .5 mul 3 sub}
\psline[linewidth=.7pt,arrows=->](0,-2)(2,-2)
\psline[linewidth=.7pt,arrows=->](0,-3)(0,-2)
\psdots(0,-3)(2,-2)
\uput[r](0,-3.3){$(0;-3)$}
\uput[ul](4,-2.5){$(2;-2)$}
\rput(7.4,0.3){$x$}
\rput(0.3, 4.2){$y$}
\rput(-0.2, -2.5){\footnotesize$1$}
\rput(1, -1.7){\footnotesize$2$}
\end{pspicture}
% \caption{Graph of the function $g(x)=x-1$}
% \label{fig:mf:g:sketchexamplestr}
\end{center}

Note: this graph is continuous and extends in both directions.       
}
\end{wex}

\Tip{Always write thefunction in the form $y=mx+c$ and take note of $m$. After plotting the graph, make sure that if $m>0$, the graph increases and that if $m<0$, the graph decreases.}


\begin{exercises}{}
{
\nopagebreak

 List the $x$- and $y$-intercepts for the following straight-line graphs. Indicate whether the graph is increasing or decreasing:
    \begin{enumerate}[noitemsep, label=\textbf{\arabic*}. ] 
    \item $y=x+1$
    \item $y=x-1$
    \item $y=2x-1$
    \item $y+1=2x$
\item $3y-2x=6$
\item$y=-3$
\item $x=3y$
\item $\frac{x}{2} - \frac{y}{3} = 1$
    \end{enumerate}


For the functions in the diagram below, give the equation and domain and range:
    \begin{enumerate}[noitemsep, label=\textbf{\arabic*}. ] 
\setcounter{enumi}{8}
    \item $a(x)$
%etc - other functions on graph go here
    \end{enumerate} 
\setcounter{subfigure}{0}
\begin{figure}[H]
\begin{center}
\scalebox{1} % Change this value to rescale the drawing.
{
\begin{pspicture}(0,-4.1467185)(9.519062,4.1867185)
\rput(4.0,-0.14671862){\psaxes[linewidth=0.03,tickstyle=bottom,labels=none,ticks=none,ticksize=0.08cm, arrows=<->](0,0)(-4,-4)(4,4)}
\psline[linewidth=0.04cm](2.78,1.7732813)(7.76,-0.9067186)
\usefont{T1}{ptm}{m}{n}
\rput(4.2023435,3.9832811){$y$}
\usefont{T1}{ptm}{m}{n}
\rput(8.26,-0.036718626){$x$}
\usefont{T1}{ptm}{m}{n}
\rput(4.4696875,1.3432813){$(0;3)$}
\usefont{T1}{ptm}{m}{n}
\rput(6.9896874,0.043281376){$(4;0)$}
\usefont{T1}{ptm}{m}{n}
\rput(3.2445312,1.9167186){$a(x)$}
\psline[linewidth=0.04cm](3.2,-3.6332815)(8.42,2.3067186)
\usefont{T1}{ptm}{m}{n}
\rput(4.6996875,-2.8167186){$(0;-6)$}
\usefont{T1}{ptm}{m}{n}
\rput(8.894531,2.4967186){$b(x)$}
\psline[linewidth=0.04cm](0.98,1.1267186)(7.88,1.1067187)
\usefont{T1}{ptm}{m}{n}
\rput(8.3,1.1567186){$c(x)$}
\psline[linewidth=0.04cm](7.4,-2.0532813)(1.08,1.4467186)
\usefont{T1}{ptm}{m}{n}
\rput(7.9,-2.1032813){$d(x)$}
\usefont{T1}{ptm}{m}{n}
\rput(4.2,0.1){$0$}
\psline[linewidth=0.04](4.7,-0.87328136)(4.58,-0.45328137)(5.0,-0.45328137)
\psline[linewidth=0.04](4.94,0.30671862)(4.82,0.7267186)(5.24,0.7267186)
\end{pspicture} 
}
\end{center}

\end{figure}  
            
\begin{enumerate}[noitemsep, label=\textbf{\arabic*}. ] 
\setcounter{enumi}{12}
\item Sketch the following functions on the same set of axes, using the dual intercept method. Clearly indicating the intercepts with the axes as well as the co-ordinates of the point of interception of the graphs: $x+2y-5=0$ and $3x-y-1=0$
\item On the same system of axes, draw the graphs of $f(x)=3-3x$ and $g(x)=\frac{1}{3}x+1$ using the gradient-intercept method.
\end{enumerate}

\par \raisebox{-5 pt}{\includegraphics[width=0.5cm]{col11306.imgs/summary_www.png}} Find the answers with the shortcodes:
\par \begin{tabular}[h]{cccccc}
(1.) lxT  &  (2.) lxb  &  (3.) lxj  & \end{tabular}
}
\end{exercises}
   

\section{Quadratic functions of the form $y=a{x}^{2}+q$}
\subsection*{Plotting the graph}         
Functions with a general form of $y=a{x}^{2}+q$ are called parabolic functions. In the equation $y=a{x}^{2}+q$, $a$ and $q$ are constant and have different effects on the parabola. 

\begin{wex}{Plotting a quadratic function}
{
\begin{equation*}
 y = f(x) = x^{2}
\end{equation*}

Complete the following table for $f(x)=x^{2}$ and plot the points on a system of axes.
\\
\begin{center}
\begin{tabular}{|c|c|c|c|c|c|c|c|}
\hline
  $x$ &  $-3$ & $-2$ & $-1$ & $0$ & $1$ & $2$ & $3$
\\ \hline
 $f(x)$& $9$ &&&&&&
\\ \hline
\end{tabular}
\end{center}
\vspace{10pt}
\begin{enumerate}[noitemsep, label=\textbf{\arabic*}. ] 
 \item Join the points with a smooth curve.
\item The domain of $f$ is $x \in \mathBB{R}$. Determine the range.
\item About which line is $f$ symmetrical?
\item From your graph, determine the value of $f(x) = \frac{25}{4}$. \\Confirm your answer graphically.
\item Write down the point where the graph cuts the axes.
\end{enumerate}
}
{
\westep{Substitute values into the equation}
\begin{equation*}
 \begin{array}{cclcc}
  f(x) &=& x^{2} & &\\
 f(-3) &=& (-3)^{2} &=& 9 \\ 
 f(-2) &=& (-2)^{2} &=& 4 \\
 f(-1) &=& (-1)^{2} &=& 1 \\
f(0) &=& 0^{2} &= &0 \\
f(1) &=& (1)^{2} &= &1 \\ 
f(2) &=& (2)^{2} &= &4 \\
f(3) &=& (3)^{2} &= &9
 \end{array}
\end{equation*}
\\
\\
\begin{center}
\begin{tabular}{|c|c|c|c|c|c|c|c|}
\hline
  $x$ &  $-3$ & $-2$ & $-1$ & $0$ & $1$ & $2$ & $3$
\\ \hline
 $f(x)$& $9$ &$4$&$1$&$0$&$1$&$4$&$9$
\\ \hline
\end{tabular}
\end{center}

\westep{Plot the points and join with a smooth curve}
From the table, we get the following points: \\
$(-3;9)$, $(-2;4)$, $(-1;1)$, $(0;0)$, $(1;1)$, $(2;4)$, $(3;9)$ \\
and the following graph: 
\begin{figure}[H]
\begin{center}
\begin{pspicture}(-5,-1)(5,5)
%\psgrid-\infty ;q
\psaxes[arrows=<->,dy=0.5](0,0)(-5,-1)(5,5)
\psset{yunit=0.5}
\psplot[plotstyle=curve,arrows=<->]{-3}{3}{x 2 exp}
\psdots(-2.5, 6.25)(2.5,6.25)
\rput(-2.2, 6.3){$B$}
\rput(2.2, 6.3){$A$}
\rput(0.1, 0.3){$0$}
\end{pspicture}
% \caption{Graph of $f(x)=x^2-1$.}
\label{fig:mf:g:parabola10}
\end{center}
\end{figure}    

\westep{Determine domain and range}
Domain: $x \in \mathBB{R}$\\
From the graph we see that for all values of $x$, $y$ is greater than or equal to $0$.\\
Range: $y \in [0; \infty)$

\westep{Find the axis of symmetry}
$f$ is symmetrical about the $y$-axis. Therefore the axis of symmetry of $f$ is the line $x=0$. 

\westep{Determine the $x$-value}
\begin{equation*}
 \begin{array}{ccl}
f(x) &=& \frac{25}{4} \\
\therfore \frac{25}{4} &=& x^{2} \\
x &=& \pm \frac{5}{2} 
\end{array}
\end{equation*}
See points $A$ and $B$ on the graph.

\westep{Determine the intercept}
The function $f(x)$ intercepts the axes at the origin $(0;0)$. \\
We notice that as the value of $x$ increases from $-\infty$ to $0$, $f(x)$ decreases. At the turning point $(0;0)$, $f(x) = 0$. As the value of $x$ increases from $0$ to $\infty$, $f(x)$ increases.
}
\end{wex}




  

\subsection*{Investigating the effects of $a$ and $q$ }
\begin{Investigation}{The effects of $a$ and $q$ on a parabolic graph}
Complete the table and plot the following graphs on the same system of axes:
    \begin{enumerate}[noitemsep, label=\textbf{\arabic*}. ] 
  \item $a(x)={x}^{2}-2$
    \item $b(x)={x}^{2}-1$
    \item $c(x)={x}^{2}$
    \item $d(x)={x}^{2}+1$
    \item $e(x)={x}^{2}+2$
        \end{enumerate}

\begin{table}[H]
% \begin{table}[H]
% \\ '' '0'
\begin{center}

\noindent

\begin{tabular}{|l|l|l|l|l|l|}\hline
 $x$&
$-2$&
$-1$&
$0$&
$1$&
$2$
\\ \hline


$a(x)$
&
&
&
&
&
\\ \hline

$b(x)$
&
&
&
&
&
\\ \hline

$c(x)$
&
&
&
&
&
\\ \hline

$d(x)$
&
&
&
&
&
\\ \hline

$e(x)$
&
&
&
&
&
\\ \hline

\end{tabular}
\end{center}
% \begin{center}{\small\bfseries Table 1.8}\end{center}
% \begin{caption}{\small\bfseries Table 1.8}\end{caption}
\end{table}
Use your results to deduce the effect of $q$.
\\
Complete the table and plot the following graphs on the same system of axes:
    \begin{enumerate}[noitemsep, label=\textbf{\arabic*}. ] 
\setcounter{enumi}{5}
  \item $f(x)=-2{x}^{2}$
    \item $g(x)=-{x}^{2}$
    \item $h(x)={x}^{2}$
    \item $i(x)=2{x}^{2}$
    \end{enumerate}
\begin{table}[H]
\begin{center}
\begin{tabular}{|l|l|l|l|l|l|}\hline
$x$&
$-2$&
$-1$&
$0$&
$1$&
$2$
\\ \hline

$f(x)$&
&
&
&
&
\\ \hline

$g(x)$&
&
&
&
&
\\ \hline


$h(x)$
&
&
&
&
&
\\ \hline

$i(x)$
&
&
&
&
&
\\ \hline

\end{tabular}
\end{center}
% \begin{center}{\small\bfseries Table 1.8}\end{center}
% \begin{caption}{\small\bfseries Table 1.8}\end{caption}
\end{table}

Use your results to deduce the effect of $a$.

\end{Investigation}

\begin{table}[H]
\begin{center}
% \caption{Table summarising general shapes and positions of graphs of functions of the form $y=mx+q=c$.}
\label{tab:mf:graphs:summarystr10}
\begin{tabular}{|c|c|c|}
\hline
 & $a<0$ & $a>0$
\\ \hline
$q>0$&
\begin{pspicture}(-1.2,-1.2)(1.2,1.2)
\psset{yunit=0.25,xunit=0.25}
\psaxes[arrows=<->,dx=0,Dx=10,dy=0,Dy=10](0,0)(-4,-4)(4,4)
\psplot[plotstyle=curve,arrows=<->]{-1.6}{1.6}{x 2 exp neg 1 add}
\end{pspicture}

&

\begin{pspicture}(-1.2,-1.2)(1.2,1.2)
\psset{yunit=0.25,xunit=0.25}
\psaxes[arrows=<->,dx=0,Dx=10,dy=0,Dy=10](0,0)(-4,-4)(4,4)
\psplot[plotstyle=curve,arrows=<->]{-1.6}{1.6}{x 2 exp 1 add}
\end{pspicture}
\\\hline
$q=0$&
\begin{pspicture}(-1.2,-1.2)(1.2,1.2)
\psset{yunit=0.25,xunit=0.25}
\psaxes[arrows=<->,dx=0,Dx=10,dy=0,Dy=10](0,0)(-4,-4)(4,4)
\psplot[plotstyle=curve,arrows=<->]{-1.6}{1.6}{x 2 exp neg}
\end{pspicture}
&
\begin{pspicture}(-1.2,-1.2)(1.2,1.2)
\psset{yunit=0.25,xunit=0.25}
\psaxes[arrows=<->,dx=0,Dx=10,dy=0,Dy=10](0,0)(-4,-4)(4,4)
\psplot[plotstyle=curve,arrows=<->]{-1.6}{1.6}{x 2 exp }
\end{pspicture}

\\ \hline
$q<0$
&

\begin{pspicture}(-1.2,-1.2)(1.2,1.2)
\psset{yunit=0.25,xunit=0.25}
\psaxes[arrows=<->,dx=0,Dx=10,dy=0,Dy=10](0,0)(-4,-4)(4,4)
\psplot[plotstyle=curve,arrows=<->]{-1.6}{1.6}{x 2 exp neg 1 sub}
\end{pspicture}
&

\begin{pspicture}(-1.2,-1.2)(1.2,1.2)
\psset{yunit=0.25,xunit=0.25}
\psaxes[arrows=<->,dx=0,Dx=10,dy=0,Dy=10](0,0)(-4,-4)(4,4)
\psplot[plotstyle=curve,arrows=<->]{-1.6}{1.6}{x 2 exp 1 sub}
\end{pspicture}
\\\hline
\end{tabular}
\end{center}
\end{table}

\textbf{The effect of $q$}
\\
The effect of $q$ is called a vertical shift because all points are moved the same distance in the same direction (because it slides the entire graph up or down). 
\begin{itemize}
\item For $q>0$, then the graph of $f(x)$ is shifted vertically upwards by $q$ units. The turning point of $f(x)$ is above the $y$-axis.
\item For $q<0$, then the graph of $f(x)$ is shifted vertically downwards by $q$ units. The turning point of $f(x)$ is below the $y$-axis.
\end{itemize}
\textbf{The effect of $a$}
\\
The sign of $a$ determines the shape of the graph. 
\begin{itemize}
 \item If $a>0$, the graph of $f(x)$ is a ``smile'' and has a minimum turning point at $(0;q)$.\\
The graph of $f(x)$ is stretched vertically upwards; as $a$ gets larger, the graph gets narrower.
\\For $0<a<1$, $a$ is a fraction, which makes the graph of $f(x)$ get wider.
\item If $a<0$, the graph of $f(x)$ is a ``frown'' and has a maximum turning point at $(0;q)$. 
\\The graph of $f(x)$ is stretched vertically downwards; as $a$ gets smaller, the graph gets narrower. \\
For $-1<a<0$, $a$ is a fraction, which makes the graph of $f(x)$ get wider.
\end{itemize}

\setcounter{subfigure}{0}
\begin{figure}[!ht]
\begin{center}
\begin{pspicture}(-4,-0.5)(6,2)
%\psgrid
\psset{yunit=0.5}
\psplot[plotstyle=curve,arrows=<->]{-2}{0}{x 1 add 2 exp}
\psplot[plotstyle=curve,arrows=<->]{3}{5}{x 4 sub 2 exp neg 1 add}
\psdots(-1.5,3)(-0.5,3)(3.5,3)(4.5,3)
\uput[d](-1,-0.5){$a>0$ ($a$ positive smile)}
\uput[d](4,-0.5){$a<0$ ($a$ negative frown)}
\end{pspicture}
% \caption{Distinctive shape of graphs of a parabola if $a>0$ and $a<0$.}
\label{fig:mf:g:parabola10a}
\end{center}
\end{figure}   

This simulation allows you to visualise the effect of changing $a$ and $q$. Note that in this simulation $q = c$. Also an extra term $bx$ has been added in. You can leave $bx$ as $0$, or you can also see what effect this has on the graph.
\par 
\setcounter{subfigure}{0}
\begin{figure}[H] % horizontal\label{m39345*Ohms-Law}
\textnormal{Phet simulation for graphing}\vspace{.1in} \nopagebreak
\label{m39345*phet!!!underscore!!!sim}\label{m39345*phet-simulation}
\raisebox{-5 pt}{ \includegraphics[width=0.5cm]{col11306.imgs/summary_www.png}} { (Simulation:  MG10018 )}
\vspace{2pt}
\vspace{.1in}
\end{figure}       


\subsection*{Discovering the characterstics}
The general form of a parabola is the equation $y=ax^{2} + q$.
\subsection*{Domain and range}

For $f(x)=a{x}^{2}+q$, the domain is $\{x:x\in \mathbb{R}\}$ because there is no value of $x\in \mathbb{R}$ for which $f(x)$ is undefined.\par 
\par 
If $a>0$ then we have:
\begin{equation*}
\begin{array}{cccl}\hfill {x}^{2}& \ge & 0\hfill & (\mbox{Perfect square is always positive})\hfill \\
 \hfill a{x}^{2}& \ge & 0\hfill & (\mbox{since } a>0)\hfill \\
 \hfill a{x}^{2}+q& \ge & q\hfill & (\mbox{add $q$ to both sides}) \\
 \hfill \therefore f(x)& \ge & q\hfill & 
\end{array}
\end{equation*}
Therefore if $a>0$, the range is $[q,\infty )$.\par 
Similarly, if $a<0$ that the range of is $ (-\infty ,q]$. 

\begin{wex}{Domain and range of a parabola}
{If $g(x)={x}^{2}+2$, determine the domain and range of the function.}
{
\westep{Determine the domain}
The domain of  is $\{x:x\in \mathbb{R}\}$ because there is no value of $x\in \mathbb{R}$ for which $g(x)$ is undefined.
\westep{Determine the range}

The range of $g(x)$ can be calculated as follows:

\begin{equation*}
\begin{array}{ccc}\hfill {x}^{2}& \ge & 0\hfill \\
 \hfill {x}^{2}+2& \ge & 2\hfill \\
 \hfill g(x)& \ge & 2\hfill 
\end{array}
\end{equation*}
Therefore the range is $\{g(x):g(x)\geq 2\}$.
}
\end{wex}



\subsection*{ Intercepts}
\textbf{The $y$-intercept}\\
Every point on the $y$-axis has an $x$-coordinate of $0$, therefore to calculate the $y$-intercept let $x=0$.\\

For example, the $y$-intercept of $g(x)={x}^{2}+2$ is given by setting $x=0$:\par 

\begin{equation*}
\begin{array}{ccl}\hfill g(x)& =& {x}^{2}+2\hfill \\ 
\hfill {y}& =& {0}^{2}+2\hfill \\
 & =& 2\hfill 
\end{array}
\end{equation*}
This gives us the point $(0;2)$.\par

\textbf{The $x$-intercept}\\
Every point on the $x$-axis has an $y$-coordinate of $0$, therefore to calculate the $x$-intercept let $y=0$.\\

For example, the $x$-intercepts of $g(x)={x}^{2}+2$ is given by setting $y=0$:\par

\begin{equation*}
\begin{array}{ccl}\hfill g(x)& =& {x}^{2}+2\hfill \\
 \hfill 0& =& x^{2}+2\hfill \\
 \hfill -2& =& x^{2}\hfill 
\end{array}
\end{equation*}
There is no real solution, therefore the graph of $g(x)={x}^{2}+2$ does not have any $x$-intercepts. 

\subsection*{Turning points}

The turning point of the function of the form $f(x)=ax^{2}+q$ is given by examining the range of the function. 
\begin{itemize}
 \item If $a>0$, the graph of $f(x)$ is a ``smile'' and has a minimum turning point at $(0;q)$.
\item If $a<0$, the graph of $f(x)$ is a ``frown'' and has a maximum turning point at $(0;q)$.
\end{itemize}


\subsection*{Axes of symmetry}

The axis of symmetry for functions of the form $f(x)=ax^{2}+q$ is the $y$-axis, which is the line $x=0$. 

\subsection*{Sketching graphs of the form $f(x)=ax^{2}+q$}

In order to sketch graphs of the form $f(x)=a^{2}+q$ we need to determine four characteristics:
\begin{enumerate}[noitemsep, label=\textbf{\arabic*}. ] 
\item sign of $a$
\item $y$-intercept
\item $x$-intercept
\item turning point

\end{enumerate}

\begin{wex}
 {Sketching a parabola}
{Sketch the graph of $g(x)=-\frac{1}{2}x^{2}-3$. Mark the intercepts and the turning point.}
{
\westep{Examine the standard form of the equation}
We notice that $a<0$. Therefore the graph of the function is a ``frown'' and has a maximum turning point.
\westep{Calculate the intercepts}
For the $y$-intercept, let $x=0$:
\begin{equation*}
\begin{array}{ccl}\hfill {y}& =& -\frac{1}{2}(0)^{2}-3\hfill \\
 h(0) =& \frac{1}{0} &= \mbox{undefined}\vspace{5pt} \\ 
 & =& -\frac{1}{2}(0)-3\hfill \\
 & =& -3\hfill 
\end{array}
\end{equation*}
This gives the point $(0; -3)$.\\

For $x$-intercept let $y=0$:
\begin{equation*}
\begin{array}{ccl}\hfill 0& =& -\frac{1}2x^{2}-3\hfill \\ 
\hfill 3& =& -\frac{1}2x^{2}\hfill \\
 \hfill -3(2)& =& x^{2}\hfill \\
\hfill -6& =& x^{2}\hfill \\
 h(0) &= \frac{1}{0} =& \mbox{undefined}\vspace{5pt} \\ 
\end{array}
\end{equation*}
There is no real solution, therefore there are no $x$-intercepts.
\westep{Determine the turning point}
From the standard form of the equation we see that the turning point is $(0;-3)$.
\westep{Plot the points and sketch the graph}
% \begin{figure}[!ht]
\begin{center}
\begin{pspicture}(-5,-5)(5,1)
%\psgrid
\psset{yunit=0.75,xunit=0.75}
\psaxes[arrows=<->](0,0)(-5,-7)(5,1)
\psplot[plotstyle=curve,arrows=<->]{-2.5}{2.5}{x 2 exp 0.5 mul neg 3 sub}
\psdots(0,-3)
\uput[r](0,-2.7){$(0;-3)$}
\rput(0.3, 0.8){$y$}
\rput (5.2, 0.2){$x$}
\end{pspicture}
% \caption{Graph of the function $f(x)=-\frac{1}{2}x^2-3$}
% % \label{fig:mf:g:sketchexamplepar10}
\end{center}
% \end{figure}
\\
The domain is the set of all real numbers and the range is $(- \infty; -3]$. 
The axis of symmetry is the line $x=0$.
}

\end{wex}


\begin{wex}
{Sketching a parabola}
{Sketch the graph of $y={3x}^{2}+5$. Mark the intercepts and the turning point.}
{
\westep{Examine the standard form of the equation}
We notice that $a>0$. Therefore the graph of the function is a ``smile'' and has a minimum turning point.
\westep{Calculate the intercepts}
For the $y$-intercept, let $x=0$:
\begin{equation*}
\begin{array}{ccl}\hfill y& =& 3x^{2}+5\hfill \\
 \hfill y& =& 3(0)^{2}+5\hfill \\
 \hfill & =& 5\hfill 
\end{array}
\end{equation*}
This gives us the point $(0;5)$.
For $x$-intercept let $y=0$:
\begin{equation*}
\begin{array}{ccc}\hfill y& =& 3x^{2}+5\hfill \\
 \hfill 0& =& 3x^{2}+5\hfill \\
 \hfill x^{2}& =& -\frac{3}{5}\hfill 
\end{array}
\end{equation*}
There is no real solution, therefore there are no $x$-intercepts.

\westep{Determine the turning point}
From the standard form of the equation we see that the turning point is $(0;5)$.

\westep{Plot the points and sketch the graph}
\begin{center}
\scalebox{1}
{
\psset{xunit=1.0cm,yunit=0.5cm,algebraic=true,dotstyle=o,dotsize=3pt 0,linewidth=0.8pt,arrowsize=3pt 2,arrowinset=0.25}
\begin{pspicture*}(-2,-2)(4,8)
\psaxes[xAxis=true,yAxis=true,Dx=1,Dy=2,ticksize=-2pt 0,subticks=2]{->}(0,0)(-2,0)(2,8)[,140] [,-40]
\rput{0}(0,5){\psplot{-2}{2}{x^2/2/0.17}}
% \rput[bl](0.56,5.34){$y = 3x^{2} + 5$}
\usefont{T1}{ptm}{m}{n}
\rput(0.3,7.6){$y$}
\usefont{T1}{ptm}{m}{n}
\rput(2.3,0.3){$x$}
\end{pspicture*}
}
\end{center}\\
The domain is $\{x:x \in \mathBB{R}\}$\\
The range is $\{y: y \geq 5, y \in \mathBB{R}\}$\\
The axis of symmetry is the line $x=0$.
}

\end{wex}


\noindent
The following video shows one method of graphing parabolas. Note that in this video the term vertex is used in place of turning point. The vertex and the turning point are the same thing.
\setcounter{subfigure}{0}
\begin{figure}[H] % horizontal\label{m39345*parabola-1}
\textnormal{Khan academy video on graphing parabolas - 1}\vspace{.1in} \nopagebreak
\label{m39345*yt-media1}\label{m39345*yt-video1}
\raisebox{-5 pt}{ \includegraphics[width=0.5cm]{col11306.imgs/summary_www.png}} { (Video:  MG10019 )}
\vspace{2pt}
\vspace{.1in}
\end{figure} 

   
\begin{exercises}{}
{
\begin{enumerate}[noitemsep, label=\textbf{\arabic*}. ] 
\item Show that if $a<0$ that the range of $f(x)=ax^{2}+q$ is $\{f(x):f(x) \leq q \}$.
\item Draw the graph of the function $y=-x^{2}+4$ showing all intercepts with the axes.
\item Two parabolas are drawn: $g:y=ax^{2}+p$ and $h:y=bx^{2}+q$.
\setcounter{subfigure}{0}
\begin{center}
\scalebox{1} % Change this value to rescale the drawing.
{
\begin{pspicture}(0,-2.61)(9.072187,2.61)
\psline[linewidth=0.04cm,arrowsize=0.05291667cm 2.0,arrowlength=1.4,arrowinset=0.4]{->}(0.0,-0.57)(9.04,-0.57 )
\psline[linewidth=0.04cm,arrowsize=0.05291667cm 2.0,arrowlength=1.4,arrowinset=0.4]{->}(4.54,-2.59)(4.56,2.59)
\psbezier[linewidth=0.04](1.8801522,2.249799)(2.0476475,1.0067352)(2.6697726,-0.42756912)(2.9090517 ,-0.81005025)(3.1483305,-1.1925315)(3.6508162,-1.83)(4.584004,-1.83)
\psbezier[linewidth=0.04](7.24,2.249799)(7.0725045,1.0067352)(6.4503794,-0.42756912)(6.2111006,-0.81005025)(5.971822,-1.1925315)(5.469336,-1.83)(4.536148 ,-1.83)
\psbezier[linewidth=0.04](1.2402416,-2.409998)(1.4477341,-1.215622)(2.2184203,0.16250414)(2.5148382,0.53000444)(2.811256,0.8975047)(3.4337332,1.5100052)(4.5897627,1.5100052)
\psbezier[linewidth=0.04](7.88,-2.409998 )(7.672508,-1.215622)(6.901821,0.16250414)(6.6054034,0.53000444)(6.3089857,0.8975047)(5.686508,1.5100052)(4.530479,1.5100052)
\usefont{T1}{ptm}{m}{it}
\rput(4.8129687,2.48){$y$}
\usefont{T1}{ptm}{m}{it}
\rput(8.908281 ,-0.3){$x$}
\usefont{T1}{ptm}{m}{n}
\rput(4.2504687,1.73){$23$}
\usefont{T1}{ptm}{m}{n}
\rput(4.269219,-2.07){$ -9$}
\usefont{T1}{ptm}{m}{n}
\rput(1.730625,0.39){ $(-4;7)$} 
\usefont{T1}{ptm}{m}{n}
\rput(7.240625,0.39){$(4;7)$}
\usefont{T1}{ptm}{m}{n}
\rput(6.4275,-0.81){ $3$}
\usefont{T1}{ptm}{m}{it}
\rput(7.369219,1.87){$g$}
\usefont{T1}{ptm}{m}{it} 
\rput{-0.1}(0.0029207817,0.01384337){\rput(7.9021873,-1.69){$h$}}
\end{pspicture} 
}       
\end{center}

    \begin{enumerate}[noitemsep, label=\textbf{\alph*}. ] 
    \item Find the values of $a$ and $p$.
    \item Find the values of $b$ and $q$.
    \item Find the values of $x$ for which $g{x}\ge h{x}$.
    \item For what values of $x$ is $g$ increasing ?
    \end{enumerate}
\end{enumerate}

\par \raisebox{-5 pt}{\includegraphics[width=0.5cm]{col11306.imgs/summary_www.png}} Find the answers with the shortcodes:
\par \begin{tabular}[h]{cccccc}
(1.) lxD  &  (2.) lxW  &  (3.) lxZ  & \end{tabular}
}
\end{exercises}   

\section{Hyperbolic functions of the form $y=\frac{a}{x}+q$}


\subsection*{Plotting the graph}  
Functions of the general form $y=\frac{a}{x}+q$ are known as hyperbolic functions. 

\begin{wex}
{Plotting a hyperbolic function}
{
\begin{equation*}
 y = h(x) = \frac{1}{x}
\end{equation*}

Complete the following table for $h(x) = \frac{1}{x}$ and plot the points on a system of axes.

\begin{table}[H]
\begin{center}
\begin{tabular}{|c|c|c|c|c|c|c|c|c|c|c|c|}
\hline
  $x$ &  $-3$ & $-2$ & $-1$ & $-\frac{1}{2}$ & $-\frac{1}{4}$ &$0$&$\frac{1}{4}$&$\frac{1}{2}$&$1$&$2$&$3$
\\ \hline
 $h(x)$& $-\frac{1}{3}$ &&&&&&&&&&
\\ \hline
\end{tabular}
\end{center}
\end{table}


\begin{enumerate}[noitemsep, label=\textbf{\arabic*}. ] 
 \item Join the points with smooth curves.
\item What happens if $x=0$?
\item Explain why the graph of $h(x)=\frac{1}{x}$ consists of two separate curves.
\item What happens to $h(x)$ as the value of $x$ becomes very small or very large?
\item The domain of $h(x)$ is $x \in \mathBB{R} - \{0\}$. Determine the range.
\item About which two lines is $h$ symmetrical?
\end{enumerate}
}
{
\westep{Substitute values into the equation}{
\begin{equation*}
 \begin{array}{cllll}
  h(x) &=& \frac{1}{x} & &\\
 h(-3) &=& \frac{1}{-3} &=& -\frac{1}{3} \vspace{5pt}\\ 
 h(-2) &=& \frac{1}{-2} &=& -\frac{1}{2}\vspace{5pt} \\ 
 h(-1) &=& \frac{1}{-1} &=& -1 \vspace{5pt}\\ 
 h(-\frac{1}{2}) &=& \dfrac{1}{-\frac{1}{2}} &=& -2 \vspace{5pt}\\ 
 h(-\frac{1}{4}) &=& \dfrac{1}{-\frac{1}{4}} &=& -4 \vspace{5pt}\\ 
 h(0) &=& \frac{1}{0} &=& \mbox{undefined}\vspace{5pt} \\ 
 h(\frac{1}{4}) &=& \dfrac{1}{\frac{1}{4}} &=& 4 \vspace{5pt}\\ 
 h(\frac{1}{2}) &=& \dfrac{1}{\frac{1}{2}} &=& 2 \vspace{5pt}\\ 
 h(1) &=& \frac{1}{1} &=& 1 \vspace{5pt}\\ 
 h(2) &=& \frac{1}{2} &=& \frac{1}{2} \vspace{5pt}\\ 
 h(3) &=& \frac{1}{3} &=& \frac{1}{3} \vspace{5pt}\\ 
 \end{array}
\end{equation*}

\begin{table}[H]
\begin{center}
\begin{tabular}{|c|c|c|c|c|c|c|c|c|c|c|c|}
\hline
  $x$ &  $-3$ & $-2$ & $-1$ & $-\frac{1}{2}$ & $-\frac{1}{4}$ &$0$&$\frac{1}{4}$&$\frac{1}{2}$&$1$&$2$&$3$
\\ \hline
 $h(x)$& $-\frac{1}{3}$ &$-\frac{1}{2}$&$-1$&$-2$&$-4$&undefined&$4$&$2$&$1$&$\frac{1}{2}$&$\frac{1}{3}$
\\ \hline
\end{tabular}
\end{center}
\end{table}
}
\westep{Plot the points and join with two smooth curves}
From the table we get the following points: $(-3; -\frac{1}{3})$, $(-2; -\frac{1}{2})$, $(-1;-1)$, $(-\frac{1}{2}; -2)$, $(-\frac{1}{4}; -4)$, $(\frac{1}{4}; -4)$, $(\frac{1}{2}; 2)$, $(1; 1)$, $(2; \frac{1}{2})$ and $(3; \frac{1}{3})$. \vspace{8pt} \\


% \setcounter{subfigure}{0}
% % \begin{figure}[tbp]
% \begin{center}
% \begin{pspicture}(-5,-2)(5,5)
% %\psgrid
% \psset{yunit=0.75,xunit=0.75}
% \psaxes[arrows=<->](0,0)(-5,-2.667)(5,6)
% \psplot[plotstyle=curve,arrows=<->]{-5}{-0.25}{x -1 exp }
% \psplot[plotstyle=curve,arrows=<->]{0.25}{5}{x -1 exp }
% % \psplot[linestyle=dotted,plotstyle=curve]{-5}{2}{x 2 add}
% \end{pspicture}
% % \caption{General shape and position of the graph of a function of the form $f(x)=\frac{a}{x} + q$.}
% \label{fig:mf:g:hyperbola10}
% \end{center}
% % \end{figure}      


For $x=0$ the function $h$ is undefined. This is called a discontinuity at $x=0$. \vspace{8pt} \\
$y=h(x) = \frac{1}{x}$ therefore we can write that $x \times y = 1$. Since the product of two positive numbers \textbf{and} the product of two negative numbers can be equal to $1$, the graph lies in the first and third quadrants.

\westep{Determine the asymptotes}
As the value of $x$ gets bigger, the value of $h(x)$ gets closer to, but does not equal $0$. This is called a horizontal asymptote, the line $y=0$. The same happens in the third quadrant; as $x$ gets smaller, $h(x)$ also approaches the $x$-axis asymptotically.\vspace{8pt} \\

We also notice that there is a vertical asymptote, the line $x=0$; as $x$ gets closer to $0$, $h(x)$ approaches the $y$-axis asymptotically.
\westep{Determine the range}
Domain: $x \in \mathBB{R} - \{0\}$\\
From the graph, we see that $y$ is defined for all values except $0$.\\
Range: $x \in \mathBB{R} - \{0\}$ 
\westep{Determine the lines of symmetry}
The graph of $h(x)$ has two axes of symmetry: the lines $y=x$ and $y=-x$. About these two lines, one half of the hyperbola is a mirror image of the other half. 
}
\end{wex}




\subsection*{Investigating the effects of $a$ and $q$ }
\begin{Investigation}{Investigating the effects of $a$ and $q$ on a hyperbolic graph}
 On the same set of axes, plot the following graphs:
    \begin{enumerate}[itemsep=3pt, label=\textbf{\arabic*}. ] 
    \item $a(x)=\dfrac{1}{x}-2$
    \item $b(x)=\dfrac{1}{x}-1$
    \item $c(x)=\dfrac{1}{x}$
    \item $d(x)=\dfrac{1}{x}+1$
    \item $e(x)=\dfrac{1}{x}+2$
\end{enumerate}
Use your results to deduce the effect of $q$.\par

On the same set of axes, plot the following graphs:
    \begin{enumerate}[itemsep=3pt, label=\textbf{\arabic*}. ] 
\setcounter{enumi}{5}
    \item $f(x)=\dfrac{-2}{x}$
    \item $g(x)=\dfrac{-1}{x}$
    \item $h(x)=\dfrac{1}{x}$
    \item $i(x)=\dfrac{2}{x}$
    \end{enumerate}
Use your results to deduce the effect of $a$.
\end{Investigation}

\begin{table}[H]
\begin{center}
% \caption{Table summarising general shapes and positions of functions of the form $y=\frac{a}{x} + q$. The axes of symmetry are shown as dashed lines.}
\label{tab:mf:graphs:summaryhyp10}
\begin{tabular}{|c|c|c|}\hline
& $a<0$&$a>0$\\\hline
$q>0$&
\begin{pspicture}(-1.2,-1.2)(1.2,1.2)
%\psgrid
\psset{xunit=0.2,yunit=0.2}
\psaxes[arrows=<->,dx=0,Dx=10,dy=0,Dy=10](0,0)(-5,-5)(5,5)
\psplot[plotstyle=curve,arrows=<->]{-5}{-0.25}{x -1 exp neg 2 add}
\psplot[plotstyle=curve,arrows=<->]{0.25}{5}{x -1 exp neg 2 add}
\psplot[linestyle=dotted,plotstyle=curve]{-4}{4}{x neg 2 add}
\end{pspicture}
&

\begin{pspicture}(-1.2,-1.2)(1.2,1.2)
%\psgrid
\psset{xunit=0.2,yunit=0.2}
\psaxes[arrows=<->,dx=0,Dx=10,dy=0,Dy=10](0,0)(-5,-5)(5,5)
\psplot[plotstyle=curve,arrows=<->]{-5}{-0.25}{x -1 exp 2 add}
\psplot[plotstyle=curve,arrows=<->]{0.25}{5}{x -1 exp 2 add}
\psplot[linestyle=dotted,plotstyle=curve]{-4}{4}{x 2 add}
\end{pspicture}
\\\hline
$q=0$ & &
\\ \hline
$q<0$&
\begin{pspicture}(-1.2,-1.2)(1.2,1.2)
%\psgrid
\psset{xunit=0.2,yunit=0.2}
\psaxes[arrows=<->,dx=0,Dx=10,dy=0,Dy=10](0,0)(-5,-5)(5,5)
\psplot[plotstyle=curve,arrows=<->]{-5}{-0.25}{x -1 exp neg 2 sub}
\psplot[plotstyle=curve,arrows=<->]{0.25}{5}{x -1 exp neg 2 sub}
\psplot[linestyle=dotted,plotstyle=curve]{-2}{4}{x neg 2 sub}
\end{pspicture}
&

\begin{pspicture}(-1.2,-1.2)(1.2,1.2)
%\psgrid
\psset{xunit=0.2,yunit=0.2}
\psaxes[arrows=<->,dx=0,Dx=10,dy=0,Dy=10](0,0)(-5,-5)(5,5)
\psplot[plotstyle=curve,arrows=<->]{-5}{-0.25}{x -1 exp 2 sub}
\psplot[plotstyle=curve,arrows=<->]{0.25}{5}{x -1 exp 2 sub}
\psplot[linestyle=dotted,plotstyle=curve]{-4}{4}{x 2 sub}
\end{pspicture}
\\\hline
\end{tabular}
\end{center}
\end{table}

\textbf{The effect of $q$}\newline

The effect of $q$ is called a vertical shift because all points are moved the same distance in the same direction (because it slides the entire graph up or down). 
\begin{itemize}
\item For $q>0$, then the graph of $h(x)$ is shifted vertically upwards by $q$ units. 
\item For $q<0$, then the graph of $h(x)$ is shifted vertically downwards by $q$ units.
\end{itemize}
The horizontal asymptote is the line $y=q$ and the vertical asymptote is always the $y$-axis, the line $x=0$.\par
\vspace{8pt}
\textbf{The effect of $a$}\newline
The sign of $a$ determines the shape of the graph. 
\begin{itemize}
 \item If $a>0$, the graph of $h(x)$ lies in the first and third quadrants. \\
For $a>1$, the graph of $h(x)$ will be further away from the axes than $y=\frac{1}{x}$.
\\For $0<a<1$, $a$ is a fraction and the graph will be closer to the axes than $y=\frac{1}{x}$. 
\item If $a<0$, the graph of $h(x)$ lies in the second and fourth quadrants.\\
For $a<-1$, the graph of $h(x)$ will be further away from the axes than $y=-\frac{1}{x}$.
\\For $-1<a<1$, $a$ is a fraction and the graph will be closer to the axes than $y=-\frac{1}{x}$. 
\end{itemize}



\subsection*{Discovering the characteristics}  
The general form of a hyperbola is the equation $y=\frac{a}{x}+q$.

\subsection*{Domain and range}

For $y=\frac{a}{x}+q$, the function is undefined for $x=0$. \\
The domain is therefore $\{x:x\in \mathbb{R},x\ne 0\}$.\par 
We see that $y=\frac{a}{x}+q$ can be re-written as:
\begin{equation*}
\begin{array}{ccl}\hfill y& =& \dfrac{a}{x}+q\hfill \vspace{4pt} \\
 \hfill y-q& =& \dfrac{a}{x}\hfill \\
 \hfill \mbox{If }x \ne  0 \mbox{ then}:(y-q)x& =& a\hfill \\
 \hfill x& =& \dfrac{a}{y-q}\hfill 
\end{array}
\end{equation*}
This shows that the function is undefined at $y=q$. \\
Therefore the range of $f(x)=\frac{a}{x}+q$ is $\{f(x):f(x)\in (-\infty ;q)\cup (q;\infty )\}$.\par 

\begin{wex}{Domain and range of a hyperbola}
{If $g(x)=\frac{2}{x}+2$, determine the domain and range of the function.}
{
\westep{Determine the domain}
The domain is $\{x:x\in \mathbb{R},x\ne 0\}$ because $g(x)$ is undefined at $x=0$.
\westep{Determine the range}
We see that $g(x)$ is also undefined at $y=2$. Therefore the range is $\{g(x):g(x)\in (-\infty ;2)\cup (2;\infty )\}$.
}
\end{wex}


\subsection*{Intercepts}

\textbf{The $y$-intercept} \\
Every point on the $y$-axis has an $x$-coordinate of $0$, therefore to calculate the $y$-intercept, let $x=0$.\\
For example, the $y$-intercept of $g(x)=\frac{2}{x}+2$ is given by setting $x=0$:
\begin{equation*}
\begin{array}{ccc}\hfill y& =& \dfrac{2}{x}+2\hfill \vspace{4pt}\\
 \hfill y& =& \dfrac{2}{0}+2\hfill 
\end{array}
\end{equation*}
which is undefined, therefore there is no $y$-intercept.\\
\\

\textbf{The $x$-intercept} \\
Every point on the $x$-axis has an $y$-coordinate of $0$, therefore to calculate the $x$-intercept, let $y=0$.\\
For example, the $x$-intercept of $g(x)=\frac{2}{x}+2$ is given by setting $y=0$:
\begin{equation*}
\begin{array}{ccl}
\hfill y& =& \dfrac{a}{x}+q\hfill \vspace{4pt}\\
 \hfill 0& =& \dfrac{a}{x}+q\hfill \vspace{4pt} \\
 \hfill \dfrac{a}{x}& =& -q\hfill \\
 \hfill a& =& -q(x)\hfill \vspace{4pt}\\
 \hfill x& =& \dfrac{a}{-q}\hfill 
\end{array}
\end{equation*}
This gives us the point $(-1; 0)$.


\subsection*{Asymptotes}

There are two asymptotes for functions of the form $y=\frac{a}{x}+q$. \par 
The horizontal asymptote is the line $y=q$ and the vertical asymptote is always the $y$-axis, the line $x=0$. 

\subsection*{Axes of symmetry}
There are two lines about which a hyperbola is symmetrical: $y=ax+q$ and $y = -x +q$.


\subsection*{Sketching graphs of the form $f(x)=\frac{a}{x}+q$}

In order to sketch graphs of functions of the form, $f(x)=\frac{a}{x}+q$, we need to determine four characteristics:
\begin{enumerate}[noitemsep, label=\textbf{\arabic*}. ] 
\item sign of $a$
\item $y$-intercept
\item $x$-intercept
\item asymptotes
\end{enumerate}

\begin{wex}{Sketching a hyperbola}
{Sketch the graph of $g(x)=\frac{2}{x}+2$. Mark the intercepts and the asymptotes.}
{
\westep{Examine the standard form of the equation}
We notice that $a>0$ therefore the graph of $g(x)$ lies in the first and third quadrant. 
\westep{Calculate the intercepts}
For the $y$-intercept, let $x=0$:
\begin{equation*}
\begin{array}{ccl}
  g(x) = & \dfrac{2}{x} + 2  \vspace{4pt} \\
  g(0) = & \dfrac{2}{0} +2  
 \end{array}
\end{equation*}
This is undefined, therefore there is no $y$-intercept. 
\\
For the $x$-intercept, let $y=0$:
\begin{equation*}
 \begin{array}{ccl}
  g(x) = &  \dfrac{2}{x} + 2 \vspace{4pt}\\
 0 = & \dfrac{2}{x} +2 \\
\dfrac{2}{x} = & -2 \\
\therefore x = &-1
 \end{array}
\end{equation*}

This gives the point $(-1;0)$


\westep{Determine the asymptotes}
The horizontal asymptote is the line $y=2$. The vertical asymptote is the line $x=0$.

\westep{Sketch the graph}
\setcounter{subfigure}{0}
\begin{figure}[H]
\begin{center}
\begin{pspicture}(-5,-3)(5,6)
%\psgrid
\psset{yunit=0.75,xunit=0.75}
\psaxes[arrows=<->](0,0)(-5,-4)(5,7)
\psplot[plotstyle=curve,arrows=<->]{-5}{-0.4}{x -1 exp 2 mul 2 add}
\psplot[plotstyle=curve,arrows=<->]{0.4}{5}{x -1 exp 2 mul 2 add}
\psline[linestyle=dashed](-5,2)(5,2)
\end{pspicture}
% \caption{Graph of $g(x)=\frac{2}{x} + 2$.}
% \label{fig:mf:g:hyperbolasketchexample}
\end{center}
\end{figure} 
Domain: $x \in \mathBB{R} - \{0\}$\\
Range: $y \in \mathBB{R} - \{2\}$
}
\end{wex}




\begin{wex}
{Drawing a hyperbola}
{
Sketch the graph of $y=\frac{-4}{x}+7$.}
{

\westep{Examine the standard form of the equation}
We see that $a<0$ therefore the graph lies in the second and fourth quadrants.
\westep{Calculate the intercepts}
For the $y$-intercept, let $x=0$:
\begin{equation*}
 \begin{array}{ccc}
 \hfill  y &= & \dfrac{-4}{x}+7 \vspace{4pt}\hfill \\
 \hfill &= & \dfrac {-4}{0} +7  \hfill \\

 \end{array}
\end{equation*}
This is undefined, therefore there is no $y$-intercept. \\
For the $x$-intercept, let $y=0$:
\begin{equation*}
 \begin{array}{ccl}
 y &=&  \dfrac{-4}{x}+7\vspace{4pt}\\
 0 &=&  \dfrac{-4}{x}+7\vspace{4pt}\\ 
 \dfrac{-4}{x} &=& -7\vspace{4pt} \\
\therefore x &= &\dfrac{4}{7}
 \end{array}
\end{equation*}

This gives the point $\Big(\dfrac{4}{7};0\Big)$


\westep{Determine the asymptotes}
The horizontal asymptote is the line $y=7$. The vertical asymptote is the line $x=0$.

\westep{Sketch the graph}
\setcounter{subfigure}{0}
\begin{figure}[H]
\psset{xunit=0.5cm,yunit=0.5cm,algebraic=true,dotstyle=o,dotsize=3pt 0,linewidth=0.8pt,arrowsize=3pt 2,arrowinset=0.25}
\begin{pspicture*}(-10.85,-7.73)(10.36,13.74)
\psaxes[Axis=true,yAxis=true,Dx=2,Dy=2,ticksize=-2pt 0,subticks=2]{->}(0,0)(-10.85,-7.73)(10.36,13.74)[x,140] [y,-40]
\psplot[plotpoints=200]{-10.849663672784777}{10.361464563594}{-4/x+7}
\psplot[linestyle=dashed,dash=5pt 5pt]{-10.85}{10.36}{(--7-0*x)/1}
\rput[bl](-8.13,7.85){$y=-\frac{4}{x} + 7$}
\end{pspicture*}   
\end{figure}

Domain: $x \in \mathBB{R} - \{0\}$\\
Range: $y \in \mathBB{R} - \{7\}$\\
Axis of symmetry: $y=x+7$ and $y=-x+7$


}
\end{wex}

\begin{exercises}{}
{
Using graph (grid) paper, draw the graph of $xy=-6$.
    \begin{enumerate}[noitemsep, label=\textbf{\arabic*}. ] 
    \item Does the point $(-2; 3)$ lie on the graph ? Give a reason for your answer.

    \item If the $x$-value of a point ohtbn the drawn graph is $0,25$ what is the corresponding $y$-value ?
    \item What happens to the $y$-values as the $x$-values become very large ?
\item Calculate the coordinates of the points $P$origin to the graph.
\item Give the equations of the asymptotes.
    \item With the line $y=-x$ as line of symmetry, what is the point symmetrical to $(-2; 3)$ ?
    \end{enumerate}
Draw the graph of $h(x)=\frac{8}{x}$.
    \begin{enumerate}[noitemsep, label=\textbf{\arabic*}. ] 
\setcounter{enumi}{7}
    \item How would the graph $g(x)=\frac{8}{x}+3$ compare with that of $h(x)=\frac{8}{x}$? Explain your answer fully.
    \item Draw the graph of $y=\frac{8}{x}+3$ on the same set of axes, showing asymptotes, axes of symmetry and the coordinates of one point on the graph.
    \end{enumerate}


\par \raisebox{-5 pt}{\includegraphics[width=0.5cm]{col11306.imgs/summary_www.png}} Find the answers with the shortcodes:
\par \begin{tabular}[h]{cccccc}
(1.) lxB  &  (2.) lxK  & \end{tabular}
}
\end{exercises}

\section{Exponential functions of the form $y=ab^{x}+q$}

\subsection*{Plotting the graph}         
Functions with the general form $y=ab^{x}+q$ are called exponential functions. In the equation $a$ aned $q$ are constants and have different effects on the function.

\begin{wex}{Plotting an exponential function}
 {
\begin{equation*} y=f(x) =b^{x} \mbox{ for } b>0 \mbox{ and } b \neq 1 \end{equation*}

Complete the following table for each of the functions and draw the graphs on the same system of axes:
$f(x)=2^{x}$, $g(x)=3^{x}$, $h(x)=5^{x}$.


\begin{table}[H]
\begin{center}
\begin{tabular}{|c|c|c|c|c|c|}
\hline
   &  $-2$ & $-1$ & $0$ & $1$ & $2$ 
\\ \hline
 $y=2^{x}$&  &&&&
\\ \hline
 $y=3^{x}$&  &&&&
\\ \hline
 $y=5^{x}$&  &&&&
\\ \hline

\end{tabular}
\end{center}
\end{table}

\begin{enumerate}[noitemsep, label=\textbf{\arabic*}. ] 
 \item At what point do these graphs intersect?
\item Explain why they do not cut the $x$-axis.
\item Give the domain and range of $h(x)$.
 \item As $x$ increases, does $h(x)$ increase or decrease?
\item Which of these graphs increases at the slowest rate?
\item For $y=k^{x}$ and $k>1$, the greater the value of $k$ the steeper the curve of the graph. True or false?
\end{enumerate}

Complete the following table for each of the functions and draw the graphs on the same system of axes:
$F(x) =(\frac{1}{2})^{x}$, $G(x) =(3)^{-x}$, $H(x) =(\frac{1}{5})^{x}$, 
\begin{table}[H]
\begin{center}
\begin{tabular}{|c|c|c|c|c|c|}
\hline
   &  $-2$ & $-1$ & $0$ & $1$ & $2$ 
\\ \hline
 $y=(\frac{1}{2})^{x}$&  &&&&
\\ \hline
$y=(\frac{1}{3})^{x}$&  &&&&
\\ \hline
$y=(\frac{1}{5})^{x}$&  &&&&
\\ \hline

\end{tabular}
\end{center}
\end{table}

\begin{enumerate}[noitemsep, label=\textbf{\arabic*}. ] 
 \item Give the $y$-intercepts for each function.
\item Describe the relationship between the graphs $f(x)$ and $F(x)$.
\item Describe the relationship between the graphs $g(x)$ and $G(x)$.
\item Give the domain and range of $H(x)$.
\item For $y=(\frac{1}{k})^{x}$ and $k>1$, the greater the value of $k$ the steeper the curve of the graph. True or false?
\item Give the equations of the asymptotes for each of the functions.
\end{enumerate}

}
{
\westep{Substitute values into the equations}
\begin{table}[H]
\begin{center}
\begin{tabular}{|c|c|c|c|c|c|}
\hline
   &  $-2$ & $-1$ & $0$ & $1$ & $2$ 
\\ \hline
 $f(x)=2^{x}$& $\frac{1}{4}$ &$\frac{1}{2}$&$1$&$2$&$4$
\\ \hline
 $g(x)=3^{x}$& $\frac{1}{9}$ &$\frac{1}{3}$&$1$&$3$&$9$
\\ \hline
 $g(x)=5^{x}$& $\frac{1}{25}$ &$\frac{1}{5}$&$1$&$5$&$25$
\\ \hline

\end{tabular}
\end{center}
\end{table}

\begin{table}[H]
\begin{center}
\begin{tabular}{|c|c|c|c|c|c|}
\hline
   &  $-2$ & $-1$ & $0$ & $1$ & $2$ 
\\ \hline
 $F(x)=(\frac{1}{2})^{x}$& $4$ &$2$&$1$&$\frac{1}{2}$&$\frac{1}{4}$
\\ \hline
$G(x)=(\frac{1}{3})^{x}$&  $9$&$3$&$1$&$\frac{1}{3}$&$\frac{1}{9}$
\\ \hline
$H(x)=(\frac{1}{5})^{x}$& $25$& $5$&$1$&$\frac{1}{5}$&$\frac{1}{25}$
\\ \hline

\end{tabular}
\end{center}
\end{table}
\westep{Plot the points and join with a smooth curve}
% Diagram needed here
\begin{enumerate}[noitemsep, label=\textbf{\arabic*}. ] 
\item We notice that all graphs pass through the point $(0;1)$. Any number with exponent $0$ is equal to $1$.
\item The graphs do not cut the $x$-axis because $0^{0}$ is undefined.
\item Domain: $x \in \mathBB{R}$\\
Range: $(0; \infty)$
\item As $x$ increases, $h(x)$ increases.
\item $f(x)=2^{x}$ increases at the slowest rate because it has the smallest base.
\item True: the greater the value of $k ~(k>1)$, the steeper the graph of $y=k^{x}$.
\end{enumerate}
% Another diagram needed here
\begin{enumerate}[noitemsep, label=\textbf{\arabic*}. ] 
\item The $y$-intercept is the point $(0; 1)$ for all the graphs. For any real number $z$, $z^{0}=1$.
\item $F(x)$ is the reflection of $f(x)$ about the $y$-axis. 
\item $G(x)$ is the reflection of $g(x)$ about the $y$-axis. 
\item  Domain: $x \in \mathBB{R}$\\
Range: $(0; \infty)$
\item True: the smaller the value of $k ~(k>1)$, the steeper the graph of $y=(\frac{1}{k})^{x}$.
\item The equation of the horizontal asymptote is $y=0$, the $x$-axis.
\end{enumerate}

}
\end{wex}

% \par 
% \setcounter{subfigure}{0}
% \begin{figure}[H]
% \begin{center}
% \begin{pspicture}(-5,-1)(5,4)
% %\psgrid
% \psset{yunit=0.75,xunit=0.75}
% \psaxes[arrows=<->](0,0)(-5,-1)(5,5)
% \psplot[plotstyle=curve,arrows=<->]{-5}{1.2}{2 x exp 2 add}
% \end{pspicture}
% % \caption{General shape and position of the graph of a function of the form $f(x)=ab^{x} + q$.}
% % \label{fig:mf:g:exponential10}
% \end{center}
% \end{figure}     

\subsection*{Investigating the effects of $a$ and $q$}
\begin{Investigation}{Investigating the effects of $a$ and $q$ on an exponential graph}
On the same set of axes, plot the following graphs ($b=2$, $a=1$ and $q$ changes):
\begin{enumerate}[noitemsep, label=\textbf{\alph*}. ] 
\item $a(x)=2^{x}-2$
\item $b(x)=2^{x}-1$
\item $c(x)=2^{x}$
\item $d(x)=2^{x}+1$
\item $e(x)=2^{x}+2$
\end{enumerate}

\begin{table}[H]
\begin{center}
\begin{tabular}{|l|c|c|c|c|c|}
\hline
   &  $-2$ & $-1$ & $0$ & $1$ & $2$ 
\\ \hline
$a(x)=2^{x}-2$&  &&&&
\\ \hline
 $b(x)=2^{x}-1$&  &&&&
\\ \hline
$c(x)=2^{x}$&  &&&&
\\ \hline
$d(x)=2^{x}+1$&  &&&&
\\ \hline
$e(x)=2^{x}+2$&  &&&&
\\ \hline
\end{tabular}
\end{center}
\end{table}
Use your results to deduce the effect of $q$.
\\

On the same set of axes, plot the following graphs ($b=2$, $q=0$ and $a$ changes):
\begin{enumerate}[noitemsep, label=\textbf{\arabic*}. ] 
\item $f(x)=2^{x}$
\item $g(x)=2.2^{x}$
\item $h(x)=-2^{x}$
\item $i(x)=-2.2^{x}$
\end{enumerate}

\begin{table}[H]
\begin{center}
\begin{tabular}{|l|c|c|c|c|c|}
\hline
   &  $-2$ & $-1$ & $0$ & $1$ & $2$ 
\\ \hline
$f(x)=2^{x}$&  &&&&
\\ \hline
$g(x)=2.2^{x}$&  &&&&
\\ \hline
 $h(x)=-2^{x}$&  &&&&
\\ \hline
$i(x)=-2.2^{x}$&  &&&&
\\ \hline

\end{tabular}
\end{center}
\end{table}
Use your results to deduce the effect of $a$.
\end{Investigation}


\begin{table}[H]
\begin{center}
% \caption{Table summarising general shapes and positions of functions of the form $y=ab^{x} + q$.}
% \label{tab:mf:graphs:summaryexp10}
\begin{tabular}{|c|c|c|}\hline
& $a<0$&$a>0$\\\hline
$q>0$&


\begin{pspicture}(-1.2,-1.2)(1.2,1.2)
%\psgrid
\psset{xunit=0.2,yunit=0.2}
\psaxes[arrows=<->,dx=0,Dx=10,dy=0,Dy=10](0,0)(-5,-5)(5,5)
\psplot[plotstyle=curve,arrows=<->]{-5}{2}{2 x exp -1 mul 2 add}
\end{pspicture}
&
\begin{pspicture}(-1.2,-1.2)(1.2,1.2)
\psset{xunit=0.2,yunit=0.2}
\psaxes[arrows=<->,dx=0,Dx=10,dy=0,Dy=10](0,0)(-5,-5)(5,5)
\psplot[plotstyle=curve,arrows=<->]{-5}{2}{2 x exp 2 add}
\end{pspicture}
\\\hline
$q<0$&


\begin{pspicture}(-1.2,-1.2)(1.2,1.2)
%\psgrid
\psset{xunit=0.2,yunit=0.2}
\psaxes[arrows=<->,dx=0,Dx=10,dy=0,Dy=10](0,0)(-5,-5)(5,5)
\psplot[plotstyle=curve,arrows=<->]{-5}{2}{2 x exp -1 mul 2 sub}
\end{pspicture}
&
\begin{pspicture}(-1.2,-1.2)(1.2,1.2)
%\psgrid
\psset{xunit=0.2,yunit=0.2}
\psaxes[arrows=<->,dx=0,Dx=10,dy=0,Dy=10](0,0)(-5,-5)(5,5)
\psplot[plotstyle=curve,arrows=<->]{-5}{2}{2 x exp 2 sub}
\end{pspicture}
\\\hline
\end{tabular}cos
\end{center}
\end{table}

\textbf{The effect of $q$}\newline

The effect of $q$ is called a vertical shift because all points are moved the same distance in the same direction (because it slides the entire graph up or down). 
\begin{itemize}
\item For $q>0$, then the graph is shifted vertically upwards by $q$ units. 
\item For $q<0$, then the graph is shifted vertically downwards by $q$ units.
\end{itemize}
The horizontal asymptote is shifted by $q$ units and is the line $y=q$. \vspace{8pt}\\


\textbf{The effect of $a$}\newline
The sign of $a$ determines whether the graph curves upwards or downwards. 
\begin{itemize}
 \item If $a>0$, the graph curves upwards.
\item If $a<0$, the graph curves downwards. It reflects the graph about the horizontal asymptote.
\end{itemize}

\subsection*{Discovering the characteristics}
The general form of an exponaential function is $y=ab^{x} + q$.
\subsection*{Domain and range}

For $y=ab^{x}+q$, the function is defined for all real values of $x$. Therefore, the domain is $\{x:x\in \mathbb{R}\}$.\par 
The range of $y=ab^{x}+q$ is dependent on the sign of $a$.\par 
If $a>0$ then:\par
\begin{equation*}
\begin{array}{ccc}\hfill b^{x}& > & 0\hfill \\
 \hfill ab^{x}& > & 0\hfill \\ 
\hfill ab^{x}+q& > & q\hfill \\ 
\hfill f(x)& > & q\hfill 
\end{array}
\end{equation*}
Therefore, if $a>0$, then the range is $\{f(x):f(x) > q\}$.\par 
If $a<0$ then:\par 

\begin{equation*}
\begin{array}{ccc}\hfill b^{x}&< & 0\hfill \\
 \hfill ab^{x}& < & 0\hfill \\
\hfill ab^{x}+q& < & q\hfill \\
 \hfill f(x)& < & q\hfill 
\end{array}
\end{equation*}
Therefore, if $a<0$, then the range is $\{f(x):f(x) < q\}$.\par 

\begin{wex}{Domain and range of an exponential function}
{Find the domain and range of $g(x)=3.2^{x}+2$}
{
\westep{Find the domain}
The domain of $g(x)=3.2^{x}+2$ is $\{x:x\in \mathbb{R}\}$.
\westep{Find the range}
\begin{equation*}
\begin{array}{ccc}\hfill 2^{x}& > & 0\hfill \\
 \hfill 3.2^{x}& > & 0\hfill \\
 \hfill 3.2^{x}+2& > & 2\hfill 
\end{array}
\end{equation*}
Therefore the range is $\{g(x):g(x) > 2)\}$.\par 
}
 
\end{wex}




\subsection*{Intercepts}
\textbf{The $y$-intercept}\\
For the $y$-intercept, let $x=0$:
\begin{equation*}
\begin{array}{ccl}\hfill y& =& ab^{x}+q\hfill \\
 \hfill y}& =& ab^{0}+q\hfill \\
 & =& a(1)+q\hfill \\
 & =& a+q\hfill 
\end{array}
\end{equation*}

For example, the $y$-intercept of $g(x)=3.2^{x}+2$ is given by setting $x=0$ to get:\par 

\begin{equation*}
\begin{array}{ccl}\hfill y& =& 3.2^{x}+2\hfill \\
 \hfill y& =& 3.2^{0}+2\hfill \\
 & =& 3+2\hfill \\ & =& 5\hfill 
\end{array}
\end{equation*}
This gives the point $(0;5)$.\vspace{10pt}
\\
\textbf{The $x$-intercept}\\
For the $x$-intercept, let $y=0$. \\
For example, the $x$-intercept of $g(x)=3.2^{x}+2$ is given by setting $y=0$:\par 
\begin{equation*}
\begin{array}{ccl}\hfill y& =& 3.2^{x}+2\hfill \\
 \hfill 0& =& 3.2^{x}+2\hfill \\
 \hfill -2& =& 3.2^{x}\hfill \\
 \hfill {2}^{x}& =& \dfrac{-2}{3}\hfill 
\end{array}
\end{equation*}
There is no real solution. Therefore, the graph of $g(x)$ does not have any $x$-intercepts.\par 

\subsection*{Asymptotes}

Exponential functions of the form $y=ab^{x}+q$ have a single horizontal asymptote, the line $x=q$. 


\subsection*{Sketching graphs of the form $f(x)=ab^{x}+q$}

In order to sketch graphs of functions of the form, $f(x)=ab^{x}+q$, we need to determine four characteristics:\par 
\begin{enumerate}[noitemsep, label=\textbf{\arabic*}. ] 
\item shape
\item $y$-intercept
\item $x$-intercept
\item asymptote
\end{enumerate}

\begin{wex}{Sketching an exponential function}
{Sketch the graph of $g(x)=3(2^{x})+2$. Mark the intercept and the asymptote.}
{
\westep{Examine the standard form of the equation}
From the equation we see that $a>1$, therefore the graph curves upwards. $q>0$ therefore the graph is shifted vertically upwards by $2$ units.

\westep{Calculate the intercepts}
For the $y$-intercept, let $x=0$:
\begin{equation*}
\begin{array}{ccl}\hfill y& =& 3.2^{x}+2\hfill \\
 \hfill y& =& 3.2^{0}+2\hfill \\
 & =& 3+2\hfill \\ & =& 5\hfill 
\end{array}
\end{equation*}
This gives the point $(0;5)$.\\

For the $x$-intercept, let $y=0$:
\begin{equation*}
\begin{array}{ccl}\hfill y& =& 3.2^{x}+2\hfill \\
 \hfill 0& =& 3.2^{x}+2\hfill \\
 \hfill -2& =& 3.2^{x}\hfill \\
 \hfill {2}^{x}& =& \dfrac{-2}{3}\hfill 
\end{array}
\end{equation*}
There is no real solution, therefore there is no $x$-intercept.

\westep{Determine the asymptote}
The horizontal asymptote is the line $y=2$.

\westep{Plot the points and sketch the graph}
\setcounter{subfigure}{0}
\begin{figure}[htbp]
\begin{center}
\begin{pspicture}(-5,-1)(5,6)
%\psgrid
\psset{yunit=0.75,xunit=0.75}
\psaxes[arrows=<->](0,0)(-5,-1)(5,7)
\psplot[plotstyle=curve,arrows=<->]{-5}{0.5}{2 x exp 3 mul 2 add}
\end{pspicture}
% \caption{Graph of $g(x)=3( 2^{x}) + 2$.}
% \label{fig:mf:g:exponentialsketchexample10}
\end{center}
\end{figure}    
\\
Domain: $\{x \in \mathBB{R}\}$\\
Range: $\{g(x): g(x) >2\}$\\

Note that there is no axis of symmetry for exponential functions.
} 
\end{wex}


 
\begin{wex}{Drawing an exponential graph}
{
Sketch the graph of $y=-2.3^{x}+6$.}
{
\westep{Examine the standard form of the equation} 
From the equation we see that $a<0$ therefore the graph curves downwards. $q>0$ therefore the graph is shifted vertically upwards by $6$ units.
\westep{Calculate the intercepts}

For the $y$-intercept, let $x=0$:
\begin{equation*}
\begin{array}{ccl}\hfill y& =& -2.3^{x}+6\hfill \\
 \hfill y& =& -2.3^{0}+6\hfill \\
 & =& 4\hfill \\

\end{array}
\end{equation*}
This gives the point $(0;4)$.\\

For the $x$-intercept, let $y=0$:
\begin{equation*}
\begin{array}{ccl}\hfill y& =& -2.3^{x}+6\hfill \\
 \hfill 0& =& -2.3^{x}+6\hfill \\
 \hfill -6& =& -2.3^{x}\hfill \\
 \hfill {3}^{1}& =& {3}^{x}\hfill \\
\hfill \therefore x & =& 1 
\end{array}
\end{equation*}
This gives the point $(1; 0)$
\westep{Determine the asymptote} The horizontal asymptote is the line $y=5$.



\westep{Plot the points and sketch the graph} 
% Diagram needed here
\\
Domain: $\{x \in \mathBB{R}\}$\\
Range: $\{g(x): g(x) <6\}$\\

}
\end{wex}




\begin{exercises}{ }
 {
 
Draw the graphs of $y=2^{x}$ and $y=(\frac{1}{2})^{x}$ on the same set of axes.
\begin{enumerate}[noitemsep, label=\textbf{\arabic*}. ] 
\item Is the $x$-axis an asymptote or an axis of symmetry to both graphs? Explain your answer.
\item Which graph is represented by the equation $y=2^{-x}$ ? Explain your answer.
\item Solve the equation $2^{x}=(\frac{1}{2})^{x}$ graphically and check that your answer is correct by using substitution.
\end{enumerate}
The curve of the exponential function $f$ in the accompanying diagram cuts the y-axis at the point $A(0; 1)$ and $B(3; 9)$ is on $f$.
% THIS DIAGRAM NEEDS FIXING!
% \setcounter{subfigure}{0}
% \begin{center}
% \scalebox{1} % Change this value to rescale the drawing.
% {
% \begin{pspicture}(0,-3.19)(6.3184376,3.21)
% \psdots[dotsize=0.12](5.0,1.49)
% \psdots[dotsize= 0.12](1.0,-1.51)
% \psbezier[linewidth=0.04](0.0,-1.81)(5.0,-0.41)(5.5,2.59)(5.6,3.19)
% \rput(5.7725,1.7){$B(2;4)$}
% \rput(1.7076563,-1.8){$A(0;1)$}
% \rput(1.0,-2.51){\psaxes[linewidth=0.04,arrowsize=0.05291667cm 2.0,arrowlength= 1.4,arrowinset=0.4,ticksize=0.08cm,dx=2.0cm,dy=1.0cm]{<->}(0,0)(-1,0)(5,5)}
% \rput(5.994375,-2.8){$x$}
% \rput(0.564375,2.5){$f$}
% \end{pspicture} 
% }     
% \end{center}
\begin{enumerate}[noitemsep, label=\textbf{\arabic*}. ] 
\setcounter{enumi}{3}
\item Determine the equation of the function $f$.
\item Determine the equation of $h$, the function of which the curve is the reflection of the curve of $f$ in the $x$-axis.
\item Determine the range of $h$.
\item Determine the equation of $g$, the reflection of the curve $f$ on the $y$-axis.
\item Determine the equation of $j$ if $j$ is a vertical stretch of $f$ by $+2$ units.
\item Determine the equation of $k$ if $k$ is a vertical shift of $f$ by $-3$ units.
\end{enumerate}


\par \raisebox{-5 pt}{\includegraphics[width=0.5cm]{col11306.imgs/summary_www.png}} Find the answers with the shortcodes:
\par \begin{tabular}[h]{cccccc}
(1.) lxk  &  (2.) lx0  & \end{tabular}
}
\end{exercises}

\section{Trigonometric functions}
\nopagebreak
This section describes the graphs of trigonometric functions.\par 

\subsection{Graph of $y=sin~\theta $}
\subsection*{Plotting the graph}
\nopagebreak
Complete the following table, using your calculator to calculate the values. Then plot the values with $sin~\theta $ on the $y$-axis and $\theta $ on the $x$-axis. Round answers to $1$ decimal place.\par 

\setlength\mytablespace{16\tabcolsep}
\addtolength\mytablespace{9\arrayrulewidth}
\setlength\mytablewidth{\linewidth}
\setlength\mytableroom{\mytablewidth}
\addtolength\mytableroom{-\mytablespace}
\setlength\myfixedwidth{0pt}
\setlength\mystarwidth{\mytableroom}
\addtolength\mystarwidth{-\myfixedwidth}
\divide\mystarwidth 80
% ----- Table with code
% \begin{table}[H]
% \\ '' '0'
\begin{center}
\label{m39414*id83562}
\noindent

\begin{tabular*}{\mytablewidth}{|p{10\mystarwidth}|p{10\mystarwidth}|p{10\mystarwidth}|p{10\mystarwidth}|p{10\mystarwidth}|p{10\mystarwidth}|p{10\mystarwidth}|p{10\mystarwidth}|}\hline

$\theta $  &
$0^{\circ }$ &
$30^{\circ }$ &
$60^{\circ }$ &
$90^{\circ }$ &
$120^{\circ }$ &
$150^{\circ }$ &
\\ \hline

$sin~\theta $ 
&
&
&
&
&
&
&
\\ \hline

$\theta $&
$180^{\circ }$ &
$210^{\circ }$ &
$240^{\circ }$ &
$270^{\circ }$ &
$300^{\circ }$ &
$330^{\circ }$ &
$360^{\circ }$% make-rowspan-placeholders
\\ \hline

$sin~\theta $
&
&
&
&
&
&
&
\\ \hline

\multicolumn{8}{|p{\dimexpr10\mystarwidth+10\mystarwidth+10\mystarwidth+10\mystarwidth+10\mystarwidth+10\mystarwidth+10\mystarwidth+10\mystarwidth+14\tabcolsep+7\arrayrulewidth\relax}|}{}

\\ \hline

\multicolumn{8}{|p{\dimexpr10\mystarwidth+10\mystarwidth+10\mystarwidth+10\mystarwidth+10\mystarwidth+10\mystarwidth+10\mystarwidth+10\mystarwidth+14\tabcolsep+7\arrayrulewidth\relax}|}{
\setcounter{subfigure}{0}
\label{m39414*id84030}
\begin{center}
\begin{pspicture}(0,-1)(4,1)
\psset{xunit=2}
%\psgrid[gridcolor=gray]
\psset{xunit=0.01111}
\psaxes[dx=30,Dx=30]{<->}(0,0)(0,-1.2)(370,1.2)
\end{pspicture}
\end{center}    
  }
\\ \hline
%--------------------------------------------------------------------
\end{tabular*}
\end{center}
\begin{center}{\small\bfseries Table 14.4}\end{center}
%\end{table}
\par
\label{m39414*id84056}Let us look back at our values for $sin~\theta $\par 
\begin{table}[H]
% \begin{table}[H]
% \\ '' '0'
\begin{center}
\label{m39414*id84073}

\begin{tabular}{|l|l|l|l|l|l|l|}\hline
$\theta $&
${0}^{\circ }$&
${30}^{\circ }$&
${45}^{\circ }$&
${60}^{\circ }$&
${90}^{\circ }$&
${180}^{\circ }$
  % make-rowspan-placeholders
\\ \hline
%--------------------------------------------------------------------
$sin~\theta $&
$0$ &
$\frac{1}{2}$&
$\frac{1}{\sqrt{2}}$&
$\frac{\sqrt{3}}{2}$&
$1$ &
$0$% make-rowspan-placeholders
\\ \hline
%--------------------------------------------------------------------
\end{tabular}
\end{center}
\begin{center}{\small\bfseries Table 14.5}\end{center}
% \begin{caption}{\small\bfseries Table 14.5}\end{caption}
\end{table}
\par
As you can see, the function $sin~\theta $ has a value of $0$ at $\theta ={0}^{\circ }$. Its value then smoothly increases until $\theta ={90}^{\circ }$ when its value is $1$. We also know that it later decreases to $0$ when $\theta ={180}^{\circ }$. Putting all this together we can start to picture the full extent of the sine graph. The sine graph is shown in Figure~14.23. Notice the wave shape, with each wave having a length of ${360}^{\circ }$. We say the graph has a period of ${360}^{\circ }$. The height of the wave above (or below) the $x$-axis is called the wave's amplitude. Thus the maximum amplitude of the sine-wave is $1$, and its minimum amplitude is $-1$.\par 
\setcounter{subfigure}{0}
\begin{figure}[h]
\begin{center}
\begin{pspicture}(-6,-2)(6,2)
\psaxes[Ox=0, Dx=180, dx=2]{<->}(0,0)(-4.5,-1.5)(4.5,1.5)
\psplot[xunit=0.0111,yunit=1.0, plotpoints=1000]{-360}{360}{x sin}
% \rput(5.3,-.3){$degrees$}
\psline[linestyle=dashed](0, 1)(4, 1)
\psline[linestyle=dashed](0, -1)(4, -1)
\end{pspicture}
\caption{The graph of $sin~ \theta$.}
\label{trig:sin}
\end{center}
\end{figure}
      

\subsection*{Graphs of the form $y=a~sin~x+q$}
\nopagebreak
In the equation, $y=a~sin~x+q$, $a$ and $q$ are constants and have different effects on the graph of the function. The general shape of the graph of functions of this form is shown in Figure~14.24 for the function $f(\theta )=2sin~\theta +3$.\par 
\setcounter{subfigure}{0}
\begin{figure}[!ht]
\begin{center}
\begin{pspicture}(-4,-2)(4,6)
%\psgrid[gridcolor=gray]
\psset{yunit=1,xunit=0.01111}
\psaxes[dx=90,Dx=90]{<->}(0,0)(-360,-2)(360,6)
\psplot[plotstyle=curve,arrows=<->]{-360}{360}{x sin 2 mul 3 add}
% \rput(420,-0.3){$degrees$}
\end{pspicture}
\caption{Graph of $f(\theta)=2 \sin \theta +3$}
\label{fig:mt:g:sin}
\end{center}
\end{figure}   

\subsection*{Investigating the effects of a and q}
\nopagebreak
\begin{enumerate}[noitemsep, label=\textbf{\arabic*}. ] 
\item On the same set of axes, plot the following graphs:
\begin{enumerate}[noitemsep, label=\textbf{\alph*}. ] 
\item $a(\theta )=sin~\theta -2$
\item $b(\theta )=sin~\theta -1$
\item $c(\theta )=sin~\theta $
\item $d(\theta )=sin~\theta +1$
\item $e(\theta )=sin~\theta +2$
\end{enumerate}
Use your results to deduce the effect of $q$.
\item On the same set of axes, plot the following graphs:
\begin{enumerate}[noitemsep, label=\textbf{\alph*}. ] 
\item $f(\theta )=-2~sin~\theta $
\item $g(\theta )=-1~sin~\theta $
\item $h(\theta )=0~sin~\theta $
\item $j(\theta )=1~sin~\theta $
\item $k(\theta )=2~sin~\theta $\end{enumerate}
Use your results to deduce the effect of $a$.
\end{enumerate}
You should have found that the value of $a$ affects the height of the peaks of the graph. As the magnitude of $a$ increases, the peaks get higher. As it decreases, the peaks get lower.\par 
$q$ is called the vertical shift. If $q=2$, then the whole sine graph shifts up $2$ units. If $q=-1$, the whole sine graph shifts down $1$ unit.\par 
These different properties are summarised in Table 14.6.\par 
\begin{table}[htb]
\begin{center}
\caption{Table summarising general shapes and positions of graphs of functions of the form $y=a \sin(x) + q$.}
\label{tab:mt:g:summarysin10}
\begin{tabular}{|c|c|c|}\hline
& $a>0$&$a<0$\\\hline
$q>0$&
\begin{pspicture}(-1.2,-0.6)(1.2,1.2)
\psset{yunit=0.5,xunit=0.0111}
\psaxes[arrows=<->,dx=0,Dx=720,dy=0,Dy=10,xunit=0.25](0,0)(-450,-1)(450,2)
\psplot[plotstyle=curve,arrows=<->,xunit=0.25]{-360}{360}{x sin 0.5 add}
% \rput(5.1,-.3){Degrees}
\end{pspicture}
&
\begin{pspicture}(-1.2,-0.6)(1.2,1.2)
\psset{yunit=0.5,xunit=0.0111}
\psaxes[arrows=<->,dx=0,Dx=720,dy=0,Dy=10,xunit=0.25](0,0)(-450,-1)(450,2)
\psplot[plotstyle=curve,arrows=<->,xunit=0.25]{-360}{360}{x sin neg 0.5 add}
% \rput(5.1,-.3){Degrees}
\end{pspicture}\\\hline
$q<0$&
\begin{pspicture}(-1.2,-1.2)(1.2,0.6)
%\psgrid
\psset{yunit=0.5,xunit=0.0111}
\psaxes[arrows=<->,dx=0,Dx=720,dy=0,Dy=10,xunit=0.25](0,0)(-450,-2)(450,1)
\psplot[plotstyle=curve,arrows=<->,xunit=0.25]{-360}{360}{x sin 0.5 sub}
% \rput(5.1,-.3){Degrees}
\end{pspicture}
&
\begin{pspicture}(-1.2,-1.2)(1.2,0.6)
%\psgrid
\psset{yunit=0.5,xunit=0.0111}
\psaxes[arrows=<->,dx=0,Dx=720,dy=0,Dy=10,xunit=0.25](0,0)(-450,-2)(450,1)
\psplot[plotstyle=curve,arrows=<->,xunit=0.25]{-360}{360}{x sin neg 0.5 sub}
% \rput(5.1,-.3){Degrees}
\end{pspicture}\\\hline
\end{tabular}
\end{center}
\end{table}
\par

\subsection*{Discovering the characteristics}
\subsection*{Domain and range}
\nopagebreak
For $f(\theta )=a~sin~\theta +q$, the domain is $\{\theta :\theta \in \mathbb{R}\}$ because there is no value of $\theta \in \mathbb{R}$ for which $f(\theta )$ is undefined.\par 
The range of $f(\theta )=a~sin~\theta +q$ depends on whether the value for $a$ is positive or negative. We will consider these two cases separately.\par 
If $a>0$ we have:\par 
\nopagebreak\noindent{}
\begin{equation*}
\begin{array}{ccc}\hfill -1\le sin~\theta & \le & 1\hfill \\ \hfill -a\le a~sin~\theta & \le & a\hfill \\ \hfill -a+q\le a~sin~\theta +q& \le & a+q\hfill \\ \hfill -a+q\le f(\theta )& \le & a+q\hfill \end{array}
\end{equation*}
This tells us that for all values of $\theta $, $f(\theta )$ is always between $-a+q$ and $a+q$. Therefore if $a>0$, the range of $f(\theta )=a~sin~\theta +q$ is $\{f(\theta ):f(\theta )\in [-a+q,a+q]\}$.\par 
Similarly, it can be shown that if $a<0$, the range of $f(\theta )=a~sin~\theta +q$ is $\{f(\theta ):f(\theta )\in [a+q,-a+q]\}$. This is left as an exercise.\par 

\Tip{The easiest way to find the range is simply to look for the "bottom" and the "top" of the graph.}


\subsection*{Intercepts}
\nopagebreak
The $y$-intercept, ${y}_{int}$, of $f(\theta )=a~sin~x+q$ is simply the value of $f(\theta )$ at $\theta ={0}^{\circ }$.\
\begin{equation*}
\begin{array}{ccc}\hfill {y}_{int}& =& f({0}^{\circ })\hfill \
\ & =& a~sin({0}^{\circ })+q\hfill \\
 & =& a(0)+q\hfill \\
 & =& q\hfill 
\end{array}
\end{equation*}



\subsection{Graph of $y=cos~\theta $}
\subsection*{Plotting the graph}
\nopagebreak
 Complete the following table, using your calculator to calculate the values correct to $1$ decimal place. Then plot the values with $cos~\theta $ on the $y$-axis and $\theta $ on the $x$-axis.\par 

% -------------------------
\setlength\mytablespace{16\tabcolsep}
\addtolength\mytablespace{9\arrayrulewidth}
\setlength\mytablewidth{\linewidth}
\setlength\mytableroom{\mytablewidth}
\addtolength\mytableroom{-\mytablespace}
\setlength\myfixedwidth{0pt}
\setlength\mystarwidth{\mytableroom}
\addtolength\mystarwidth{-\myfixedwidth}
\divide\mystarwidth 80subsub
% ----- Table with code
% \begin{table}[H]
% \\ '' '0'
\begin{center}
\label{m39414*id86399}
\noindent

\begin{tabular*}{\mytablewidth}{|p{10\mystarwidth}|p{10\mystarwidth}|p{10\mystarwidth}|p{10\mystarwidth}|p{10\mystarwidth}|p{10\mystarwidth}|p{10\mystarwidth}|p{10\mystarwidth}|}\hline
$\theta $     &
$0^{\circ }$ &
$30^{\circ }$ &
$60^{\circ }$ &
$90^{\circ }$ &
$120^{\circ }$ &
$150^{\circ }$ &
\\ \hline

$cos~\theta $  &
&
&
&
&
&
&
\\ \hline

$\theta $    &
$180^{\circ }$ &
$210^{\circ }$ &
$240^{\circ }$ &
$270^{\circ }$ &
$300^{\circ }$ &
$330^{\circ }$ &
$360^{\circ }$

\\ \hline

$cos~\theta $&
&
&
&
&
&
&

\\ \hline

\multicolumn{8}{|p{\dimexpr10\mystarwidth+10\mystarwidth+10\mystarwidth+10\mystarwidth+10\mystarwidth+10\mystarwidth+10\mystarwidth+10\mystarwidth+14\tabcolsep+7\arrayrulewidth\relax}|}{}

\\ \hline

\multicolumn{8}{|p{\dimexpr10\mystarwidth+10\mystarwidth+10\mystarwidth+10\mystarwidth+10\mystarwidth+10\mystarwidth+10\mystarwidth+10\mystarwidth+14\tabcolsep+7\arrayrulewidth\relax}|}{
\setcounter{subfigure}{0}
\begin{pspicture}(0,-1)(4,1)
\psset{xunit=2}
%\psgrid[gridcolor=gray]
\psset{xunit=0.01111}
\psaxes[dx=30,Dx=30]{<->}(0,0)(0,-1.2)(370,1.2)
\end{pspicture} 
  }

\\ \hline

\end{tabular*}
\end{center}
\begin{center}{\small\bfseries Table 14.7}\end{center}
%\end{table}
\par
\label{m39414*id86892}Let us look back at our values for $cos~\theta $\par 
% \textbf{m39414*uid46}\par
\begin{table}[H]
% \begin{table}[H]
% \\ '' '0'
\begin{center}
\label{m39414*id86909}
\noindent

\begin{tabular}{|l|l|l|l|l|l|l|}\hline
$\theta $
${0}^{\circ }$&
${30}^{\circ }$&
${45}^{\circ }$&
${60}^{\circ }$&
${90}^{\circ }$&
${180}^{\circ }$&
  % make-rowspan-placeholders
\\ \hline
%--------------------------------------------------------------------
$cos~\theta $&
$1$ &
$\frac{\sqrt{3}}{2}$&
$\frac{1}{\sqrt{2}}$&
$\frac{1}{2}$&
$0$ &
$-1$
  % make-rowspan-placeholders
\\ \hline
%--------------------------------------------------------------------
\end{tabular}
\end{center}
\begin{center}{\small\bfseries Table 14.8}\end{center}
% \begin{caption}{\small\bfseries Table 14.8}\end{caption}
\end{table}
\par
If you look carefully, you will notice that the cosine of an angle $\theta $ is the same as the sine of the angle ${90}^{\circ }-\theta $. Take for example,\par 
\nopagebreak\noindent{}
\begin{equation*}
cos~{60}^{\circ }=\frac{1}{2}=sin~{30}^{\circ }=sin~({90}^{\circ }-{60}^{\circ })
\end{equation*}
This tells us that in order to create the cosine graph, all we need to do is to shift the sine graph ${90}^{\circ }$ to the left. The graph of $cos~\theta $ is shown in Figure~14.30. As the cosine graph is simply a shifted sine graph, it will have the same period and amplitude as the sine graph.\par 
\setcounter{subfigure}{0}
\begin{figure}[h]
\begin{center}
\begin{pspicture}(-6,-2)(6,2)
\psaxes[Ox=0, Dx=180, dx=2]{<->}(0,0)(-4.5,-1.5)(4.5,1.5)
%\psline[]{<-}(0,1.2)(1,1.2)
\psplot[xunit=0.0111,yunit=1.0, plotpoints=1000]{-360}{360}{x cos}
% \rput(5.1,-.3){Degrees}
\end{pspicture}
\caption{The graph of $\cos \theta$.}
\label{trig:cos}
\end{center}
\end{figure}      

\subsection*{Graphs of the form $y=a~cos~x+q$}
\nopagebreak
In the equation, $y=a~cos~x+q$, $a$ and $q$ are constants and have different effects on the graph of the function. The general shape of the graph of functions of this form is shown in Figure~14.31 for the function $f(\theta )=2~cos~\theta +3$.\par 
\setcounter{subfigure}{0}
\begin{figure}[!ht]
\begin{center}
\begin{pspicture}(-4,-2)(4,6)
%\psgrid[gridcolor=gray]
\psset{yunit=1,xunit=0.01111}
\psaxes[dx=90,Dx=90]{<->}(0,0)(-360,-2)(360,6)
\psplot[plotstyle=curve,arrows=<->]{-360}{360}{x cos 2 mul 3 add}
% \rput(5.1,-.3){Degrees}
\end{pspicture}
\caption{Graph of $f(\theta)=2 ~\cos ~\theta +3$}
\label{fig:mt:g:cos}
\end{center}
\end{figure}      

\subsection*{Investigating the effects of a and q}
\nopagebreak
\begin{enumerate}[noitemsep, label=\textbf{\arabic*}. ] 
\item On the same set of axes, plot the following graphs:
\begin{enumerate}[noitemsep, label=\textbf{\alph*}. ] 
\item $a(\theta )=cos~\theta -2$
\item $b(\theta )=cos~\theta -1$
\item $c(\theta )=cos~\theta $
\item $d(\theta )=cos~\theta +1$
\item $e(\theta )=cos~\theta +2$\end{enumerate}
Use your results to deduce the effect of $q$.
\item On the same set of axes, plot the following graphs:
\begin{enumerate}[noitemsep, label=\textbf{\alph*}. ] 
\item $f(\theta )=-2~cos~\theta $
\item $g(\theta )=-1~cos~\theta $
\item $h(\theta )=0~cos~\theta $
\item $j(\theta )=1~cos~\theta $
\item $k(\theta )=2~cos~\theta $\end{enumerate}
Use your results to deduce the effect of $a$.
\end{enumerate}
You should have found that the value of $a$ affects the amplitude of the cosine graph in the same way it did for the sine graph.\par 
You should have also found that the value of $q$ shifts the cosine graph in the same way as it did the sine graph.\par 
These different properties are summarised in Table 14.9.\par 
\begin{table}[htb]
\begin{center}
\caption{Table summarising general shapes and positions of graphs of functions of the form $y=a~ \cos~x + q$.}
\label{tab:mt:g:summarycos10}
\begin{tabular}{|c||c|c|}\hline
& $a>0$&$a<0$\\\hline\hline
$q>0$&
\begin{pspicture}(-1.2,-0.6)(1.2,1.2)
\psset{yunit=0.5,xunit=0.0111}
\psaxes[arrows=<->,dx=0,Dx=720,dy=0,Dy=10,xunit=0.25](0,0)(-450,-1)(450,2)
\psplot[plotstyle=curve,arrows=<->,xunit=0.25]{-360}{360}{x cos 0.5 add}
\end{pspicture}
&
\begin{pspicture}(-1.2,-0.6)(1.2,1.2)
\psset{yunit=0.5,xunit=0.0111}
\psaxes[arrows=<->,dx=0,Dx=720,dy=0,Dy=10,xunit=0.25](0,0)(-450,-1)(450,2)
\psplot[plotstyle=curve,arrows=<->,xunit=0.25]{-360}{360}{x cos neg 0.5 add}
\end{pspicture}\\\hline
$q<0$&
\begin{pspicture}(-1.2,-1.2)(1.2,0.6)
%\psgrid
\psset{yunit=0.5,xunit=0.0111}
\psaxes[arrows=<->,dx=0,Dx=720,dy=0,Dy=10,xunit=0.25](0,0)(-450,-2)(450,1)
\psplot[plotstyle=curve,arrows=<->,xunit=0.25]{-360}{360}{x cos 0.5 sub}
\end{pspicture}
&
\begin{pspicture}(-1.2,-1.2)(1.2,0.6)
%\psgrid
\psset{yunit=0.5,xunit=0.0111}
\psaxes[arrows=<->,dx=0,Dx=720,dy=0,Dy=10,xunit=0.25](0,0)(-450,-2)(450,1)
\psplot[plotstyle=curve,arrows=<->,xunit=0.25]{-360}{360}{x cos neg 0.5 sub}
\end{pspicture}\\\hline
\end{tabular}
\end{center}
\end{table}
\par

\subsection*{Discovering the characteristics}
\subsection*{Domain and range}
\nopagebreak
For $f(\theta )=a~cos(\theta )+q$, the domain is $\{\theta :\theta \in \mathbb{R}\}$ because there is no value of $\theta \in \mathbb{R}$ for which $f(\theta )$ is undefined.\par 
It is easy to see that the range of $f(\theta )$ will be the same as the range of $a~sin(\theta )+q$. This is because the maximum and minimum values of $a~cos(\theta )+q$ will be the same as the maximum and minimum values of $a~sin(\theta )+q$.\par 

\subsection*{Intercepts}
\nopagebreak
The $y$-intercept of $f(\theta )=a~cos~x+q$ is calculated in the same way as for sine.\par 
\nopagebreak\noindent{}
\begin{equation*}
\begin{array}{ccc}\hfill {y}_{int}& =& f({0}^{\circ })\hfill \\
 & =& a~cos({0}^{\circ })+q\hfill \\
 & =& a(1)+q\hfill \\
 & =& a+q\hfill 
\end{array}
\end{equation*}

\subsection*{Comparison of graphs of $y=sin~\theta $ and $y=cos~\theta $}
\nopagebreak
\setcounter{subfigure}{0}
\begin{figure}[h]
\begin{center}
\begin{pspicture}(-6,-2)(6,2)
\psaxes[Ox=0, Dx=180, dx=2]{<->}(0,0)(-4.5,-1.5)(4.5,1.5)
\psline[]{<-}(0,1.2)(1,1.2)
\psplot[xunit=0.0111, , yunit=1.0, plotpoints=1000, linestyle=dashed]{-360}{360}{x sin}
\psplot[xunit=0.0111,yunit=1.0, plotpoints=1000]{-360}{360}{x cos}
% \rput(5.1,-.3){Degrees}
\rput(1.8,1.2){$90^\circ$ shift}
\end{pspicture}
\caption{The graph of $\cos \theta$ (solid line) and the graph of $\sin \theta$ (dashed line).}
\end{center}
\end{figure}    
Notice that the two graphs look very similar. Both oscillate up and down around the $x$-axis as you move along the axis. The distances between the peaks of the two graphs is the same and is constant along each graph. The height of the peaks and the depths of the troughs are the same.\par 
The only difference is that the $sin$ graph is shifted a little to the right of the $cos$ graph by $90^{\circ }$. That means that if you shift the whole $cos$ graph to the right by $90 ^{\circ }$ it will overlap perfectly with the $sin$ graph. You could also move the $sin$ graph by $90 ^{\circ }$to the left and it would overlap perfectly with the $cos$ graph. This means that:\par 
\nopagebreak\noindent{}
\begin{equation*}
\begin{array}{ccc}\hfill sin~\theta & =& cos~(\theta -90)(\mbox{shift the } cos{ graph to the right})\hfill \\
 & \mathbf{a}\mbox{nd}& \\
 \hfill cos~\theta & =& sin(\theta +90)(\mbox{shift the }sin\mbox{ graph to the left})\hfill 
\end{array}
\end{equation*}

\subsection{Graph of $y=tan~\theta $ }
\subsection*{Plotting the graph}
\nopagebreak
Complete the following table, using your calculator to calculate the values correct to $1$ decimal place. Then plot the values with $tan~\theta $ on the $y$-axis and $\theta $ on the $x$-axis.\par 

% -------------------------
\setlength\mytablespace{16\tabcolsep}
\addtolength\mytablespace{9\arrayrulewidth}
\setlength\mytablewidth{\linewidth}
\setlength\mytableroom{\mytablewidth}
\addtolength\mytableroom{-\mytablespace}
\setlength\myfixedwidth{0pt}
\setlength\mystarwidth{\mytableroom}
\addtolength\mystarwidth{-\myfixedwidth}
\divide\mystarwidth 80

\begin{center}
\label{m39414*id89083}
\noindent

\begin{tabular*}{\mytablewidth}{|p{10\mystarwidth}|p{10\mystarwidth}|p{10\mystarwidth}|p{10\mystarwidth}|p{10\mystarwidth}|p{10\mystarwidth}|p{10\mystarwidth}|p{10\mystarwidth}|}\hline

$\theta $ &
$0^{\circ }$ &
$30^{\circ }$ &
$60^{\circ }$ &
$90^{\circ }$ &
$120^{\circ }$ &
$150^{\circ }$ &

\\ \hline

$tan~\theta $ &
&
&
&
&
&
&

\\ \hline

$\theta $ &
$180^{\circ }$ &
$210^{\circ }$ &
$240^{\circ }$ &
$270^{\circ }$ &
$300^{\circ }$ &
$330^{\circ }$ &
$360^{\circ }$
\\ \hline

$tan~\theta $ &
&
&
&
&
&
&

\\ \hline
%--------------------------------------------------------------------

\multicolumn{8}{|p{\dimexpr10\mystarwidth+10\mystarwidth+10\mystarwidth+10\mystarwidth+10\mystarwidth+10\mystarwidth+10\mystarwidth+10\mystarwidth+14\tabcolsep+7\arrayrulewidth\relax}|}{}

\\ \hline

\multicolumn{8}{|p{\dimexpr10\mystarwidth+10\mystarwidth+10\mystarwidth+10\mystarwidth+10\mystarwidth+10\mystarwidth+10\mystarwidth+10\mystarwidth+14\tabcolsep+7\arrayrulewidth\relax}|}{
\setcounter{subfigure}{0}
\begin{pspicture}(0,-1)(4,1)
\psset{xunit=2}
%\psgrid[gridcolor=gray]
\psset{xunit=0.01111}
\psaxes[dx=30,Dx=30]{<->}(0,0)(0,-1.2)(370,1.2)
\end{pspicture}   
  }

\\ \hline
%--------------------------------------------------------------------
\end{tabular*}
\end{center}
\begin{center}{\small\bfseries Table 14.10}\end{center}
%\end{table}
\par
\label{m39414*id89576}Let us look back at our values for $tan~\theta $\par 
% \textbf{m39414*uid65}\par
\begin{table}[H]
% \begin{table}[H]
% \\ '' '0'
\begin{center}
\label{m39414*id89593}
\noindent

\begin{tabular}[t]{|l|l|l|l|l|l|l|}\hline
$\theta $&
${0}^{\circ }$&
${30}^{\circ }$&
${45}^{\circ }$&
${60}^{\circ }$&
${90}^{\circ }$&
${180}^{\circ }$
\\ \hline
%--------------------------------------------------------------------
$tan~\theta $&
$0$ &
$\frac{1}{\sqrt{3}}$&
$1$ &
$\sqrt{3}$&
$\infty $&
$0$% make-rowspan-placeholders
\\ \hline
%--------------------------------------------------------------------
\end{tabular}
\end{center}
\begin{center}{\small\bfseries Table 14.11}\end{center}
% \begin{caption}{\small\bfseries Table 14.11}\end{caption}
\end{table}
\par
Now that we have graphs for $sin~\theta $ and $cos~\theta $, there is an easy way to visualise the tangent graph. Let us look back at our definitions of $sin~\theta $ and $cos~\theta $ for a right-angled triangle.\par 
\nopagebreak\noindent{}
\begin{equation*}
\frac{sin~\theta }{cos~\theta }=\frac{\frac{\mbox{opposite}}{\mbox{hypotenuse}}}{\frac{\mbox{adjacent}}{\mbox{hypotenuse}}}=\frac{\mbox{opposite}}{\mbox{adjacent}}=tan~\theta 
\end{equation*}
This is the first of an important set of equations called trigonometric identities. An identity is an equation, which holds true for any value which is put into it. In this case we have shown that\par 
\nopagebreak\noindent{}
\begin{equation*}
tan~\theta =\frac{sin~\theta }{cos~\theta }
\end{equation*}
for any value of $\theta $.\par 
So we know that for values of $\theta $ for which $sin~\theta =0$, we must also have $tan~\theta =0$. Also, if $cos~\theta =0$ our value of $tan~\theta $ is undefined as we cannot divide by $0$. The graph is shown in Figure~14.38. The dashed vertical lines are at the values of $\theta $ where $tan~\theta $ is not defined.\par 
\setcounter{subfigure}{0}
\\begin{figure}[h]
\begin{center}
\begin{pspicture}(-6,-3)(6,3)
\psaxes[Dx=180, dx=2, Dy=2, dy=1]{<->}(0,0)(-4.5,-3)(4.5,3)
\psline[linestyle=dashed](-1,-2.5)(-1,2.5)
\psline[linestyle=dashed](1,-2.5)(1,2.5)
\psline[linestyle=dashed](-3,-2.5)(-3,2.5)
\psline[linestyle=dashed](3,-2.5)(3,2.5)
\psplot[xunit=0.0111,yunit=0.5, plotpoints=500, arrows=<->]{-80}{80}{x sin x cos div}
\psplot[xunit=0.0111,yunit=0.5,plotpoints=500, arrows=<->]{-260}{-100}{x sin x cos div}
\psplot[xunit=0.0111,yunit=0.5,plotpoints=500, arrows=<->]{260}{100}{x sin x cos div}
\psplot[xunit=0.0111,yunit=0.5,plotpoints=500, arrows=<->]{-100}{-260}{x sin x cos div}
\psplot[xunit=0.0111,yunit=0.5,plotpoints=500, arrows=<-]{-280}{-360}{x sin x cos div}
\psplot[xunit=0.0111,yunit=0.5,plotpoints=500, arrows=<-]{280}{360}{x sin x cos div}
% \rput(5.1,-.3){$\theta$\ Degrees}
\rput(.4,3.3){$f(\theta$)}
\end{pspicture}
% \caption{The graph of $\tan \theta$.}
\label{trig:tan}
\end{center}
\end{figure}       

\subsection*{Graphs of the form $y=a ~tan~x+q$}
\nopagebreak
In the figure below is an example of a function of the form $y=a ~tan(x)+q$.\par 
\setcounter{subfigure}{0}
\begin{figure}[!ht]
\begin{center}
\begin{pspicture}(-5,-3)(5,3.2)
%\psgrid
\psset{yunit=0.25}
\psaxes[Dx=180, dx=2, Dy=5, dy=5]{<->}(0,0)(-4.5,-12)(4.5,12)
\psline[linestyle=dashed](-1,-12.5)(-1,12.5)
\psline[linestyle=dashed](1,-12.5)(1,12.5)
\psline[linestyle=dashed](-3,-12.5)(-3,12.5)
\psline[linestyle=dashed](3,-12.5)(3,12.5)
\psplot[xunit=0.0111, plotpoints=500, arrows=<->]{-80}{80}{x sin x cos div 2 mul 1 add}
\psplot[xunit=0.0111,plotpoints=500, arrows=<->]{-260}{-100}{x sin x cos div 2 mul 1 add}
\psplot[xunit=0.0111,plotpoints=500, arrows=<->]{260}{100}{x sin x cos div 2 mul 1 add}
\psplot[xunit=0.0111,plotpoints=500, arrows=<->]{-100}{-260}{x sin x cos div 2 mul 1 add}
\psplot[xunit=0.0111,plotpoints=500, arrows=<-]{-280}{-360}{x sin x cos div 2 mul 1 add}
\psplot[xunit=0.0111,plotpoints=500, arrows=<-]{280}{360}{x sin x cos div 2 mul 1 add}
% \rput(5.3,-.6){$\theta$\ Degrees}
\rput(.4,9.3){$f(\theta$)}
\end{pspicture}
\caption{The graph of $2\tan \theta + 1$.}
\label{trig:tan2}
\end{center}
\end{figure}
       

\subsection*{Investigating the effects of a and q}
\nopagebreak
\begin{enumerate}[noitemsep, label=\textbf{\arabic*}. ] 
\item On the same set of axes, plot the following graphs:
\begin{enumerate}[noitemsep, label=\textbf{\alph*}. ] 
\item $a(\theta )=tan~\theta -2$
\item $b(\theta )=tan~\theta -1$
\item $c(\theta )=tan~\theta $
\item $d(\theta )=tan~\theta +1$
\item $e(\theta )=tan~\theta +2$\end{enumerate}
Use your results to deduce the effect of $q$.
\item On the same set of axes, plot the following graphs:
\begin{enumerate}[noitemsep, label=\textbf{\alph*}. ] 
\item $f(\theta )=-2~tan~\theta $
\item $g(\theta )=-1~tan~\theta $
\item $h(\theta )=0~tan~\theta $
\item $j(\theta )=1~tan~\theta $
\item $k(\theta )=2~tan~\theta $\end{enumerate}
Use your results to deduce the effect of $a$.
\end{enumerate}
You should have found that the value of $a$ affects the steepness of each of the branches. The larger the absolute magnitude of $a$, the quicker the branches approach their asymptotes, the values where they are not defined. Negative $a$ values switch the direction of the branches.
You should have also found that the value of $q$ affects the vertical shift as for $sin~\theta $ and $cos~\theta $.
These different properties are summarised in Table 14.12.\par 
\begin{table}[htb]
\begin{center}
\caption{Table summarising general shapes and positions of graphs of functions of the form $y=a~ \tan~x + q$.}
\label{tab:mt:g:summarytan10}
\begin{tabular}{|c||c|c|}\hline
& $a>0$&$a<0$\\\hline\hline
$q>0$&
\begin{pspicture}(-1.2,-0.6)(1.2,0.8)
%\psgrid[gridcolor=gray]
\psset{yunit=0.1,xunit=0.0111}
\psaxes[arrows=<->,dx=0,Dx=720,dy=0,Dy=10,xunit=0.25](0,0)(-450,-6)(450,7)
\psplot[plotstyle=curve,arrows=<->,xunit=0.25]{-81.5}{78}{x sin x cos div 1.5 add}cos
\end{pspicture}
&
\begin{pspicture}(-1.2,-0.6)(1.2,0.8)
%\psgrid[gridcolor=gray]
\psset{yunit=0.1,xunit=0.0111}
\psaxes[arrows=<->,dx=0,Dx=720,dy=0,Dy=10,xunit=0.25](0,0)(-450,-6)(450,7)
\psplot[plotstyle=curve,arrows=<->,xunit=0.25]{-78}{82.5}{x sin x cos div neg 1.5 add}
\end{pspicture}\\\hline
$q<0$&
\begin{pspicture}(-1.2,-0.8)(1.2,0.8)
%\psgrid[gridcolor=gray]
\psset{yunit=0.1,xunit=0.0111}
\psaxes[arrows=<->,dx=0,Dx=720,dy=0,Dy=10,xunit=0.25](0,0)(-450,-7)(450,6)
\psplot[plotstyle=curve,arrows=<->,xunit=0.25]{-80}{80}{x sin x cos div 1.5 sub}
\end{pspicture}
&
\begin{pspicture}(-1.2,-0.8)(1.2,0.8)
%\psgrid
\psset{yunit=0.1,xunit=0.0111}
\psaxes[arrows=<->,dx=0,Dx=720,dy=0,Dy=10,xunit=0.25](0,0)(-450,-7)(450,6)
\psplot[plotstyle=curve,arrows=<->,xunit=0.25]{-80}{80}{x sin x cos div neg 1.5 sub}
\end{pspicture}\\\hline
\end{tabular}
\end{center}
\end{table}
\par

\subsection*{Discovering the characteristics}
\subsection*{Domain and range}
\nopagebreak
The domain of $f(\theta )=a~tan(\theta )+q$ is all the values of $\theta $ such that $cos~\theta $ is not equal to $0$. We have already seen that when $cos~\theta =0$, $tan~\theta =\frac{sin~\theta }{cos~\theta }$ is undefined, as we have division by zero. We know that $cos~\theta =0$ for all $\theta ={90}^{\circ }+{180}^{\circ }n$, where $n$ is an integer. So the domain of $f(\theta )=a~tan(\theta )+q$ is all values of $\theta $, except the values $\theta ={90}^{\circ }+{180}^{\circ }n$.\par 
The range of $f(\theta )=a~tan~\theta +q$ is $\{f(\theta ):f(\theta )\in (-\infty ,\infty )\}$.\par 

\subsection*{Intercepts}
\nopagebreak
The $y$-intercept, ${y}_{int}$, of $f(\theta )=a~tan~x+q$ is again simply the value of $f(\theta )$ at $\theta ={0}^{\circ }$.\par 
\nopagebreak\noindent{}
\begin{equation*}
\begin{array}{ccc}\hfill {y}_{int}& =& f({0}^{\circ })\hfill \\
 & =& a~tan({0}^{\circ })+q\hfill \\
 & =& a(0)+q\hfill \\
 & =& q\hfill 
\end{array}
\end{equation*}

\subsection*{Asymptotes}
\nopagebreak
As $\theta $ approaches ${90}^{\circ }$, $tan~\theta $ approaches infinity. But as $\theta $ is undefined at ${90}^{\circ }$, $\theta $ can only approach ${90}^{\circ }$, but never equal it. Thus the $tan~\theta $ curve gets closer and closer to the line $\theta ={90}^{\circ }$, without ever touching it. Thus the line $\theta ={90}^{\circ }$ is an asymptote of $tan~\theta $. $tan~\theta $ also has asymptotes at $\theta ={90}^{\circ }+{180}^{\circ }n$, where $n$ is an integer.\par 

\begin{exercises}{Trigonometric Functions }
 {

\begin{enumerate}[noitemsep, label=\textbf{\arabic*}. ] 
\item Using your knowldge of the effects of $a$ and $q$, sketch each of the following graphs, without using a table of values, for $\theta \in [{0}^{\circ };{360}^{\circ }]$
\begin{enumerate}[noitemsep, label=\textbf{\alph*}. ] 
\item $y=2~sin~\theta $
\item $y=-4~cos~\theta $
\item $y=-2~cos~\theta +1$
\item $y=sin~\theta -3$
\item $y=tan~\theta -2$\item $y=2~cos~\theta -1$
\end{enumerate}
 \item Give the equations of each of the following graphs:
\setcounter{subfigure}{0}
% \begin{minipage}{0.6\textwidth}
\begin{center}
\begin{pspicture}(-2.5,-2)(5,2)
\psset{yunit=0.25}
\psaxes[Dx=180, dx=2, Dy=2, dy=4]{<->}(0,0)(-2,-5.1)(4.5,5.1)
\psplot[xunit=0.0111, plotpoints=500, arrows=<->]{-90}{360}{x cos -4 mul }
\uput[d](4.7,0){$x$}
\uput[r](0,5.1){$y$}
% \rput(-2.5,2){a)}
\end{pspicture}

\begin{pspicture}(-0.3,-2)(5,2)
%\psgrid
\psset{yunit=0.25}
\psaxes[Dx=90, dx=1, Dy=2, dy=4]{<->}(0,0)(0,-5.1)(4.5,5.1)
\psplot[xunit=0.0111, plotpoints=500, arrows=->]{0}{360}{x sin 1 add 2 mul}
\uput[d](4.7,0){$x$}
\uput[r](0,5.1){$y$}
% \rput(-1.2,3){b)}
%\psplot[xunit=0.0111,plotpoints=500, arrows=<->]{-260}{-100}{x sin x cos div 2 mul 1 add}
%\psplot[xunit=0.0111,plotpoints=500, arrows=<->]{260}{100}{x sin x cos div 2 mul 
\end{pspicture}
\end{center}
% \end{minipage}
% \begin{minipage}{0.35\textwidth}
\begin{pspicture}(-2.2,-3)(2.2,3.2)
%\psgrid
\psset{yunit=0.2}
\psaxes[Dx=90, dx=1, Dy=5, dy=5]{<->}(0,0)(-2,-12)(2,12)
\psline[linestyle=dashed](-1,0)(-1,12.5)
\psline[linestyle=dashed](1,-12.5)(1,-3)
\psline[linestyle=dashed](-1,-12.5)(-1,-3)
\psline[linestyle=dashed](1,0)(1,12.5)
% \rput(-2,8){c)}
\psplot[xunit=0.0111, plotpoints=500, arrows=<->]{-75}{83}{x sin x cos div -2 mul 5 add}
\end{pspicture}
% \end{minipage}      \par 

\par \raisebox{-5 pt}{\includegraphics[width=0.5cm]{col11306.imgs/summary_www.png}} Find the answers with the shortcodes:
\par \begin{tabular}[h]{cccccc}
(1.) la8  &  (2.) la0  & \end{tabular}
}
\end{exercises}

\section{Interpretation of graphs}


\summary

\begin{itemize}[noitemsep]
\item You should know the following charecteristics of functions: 
\begin{itemize}[noitemsep]
\item The given or chosen $x$-value is known as the independent variable, because its value can be chosen freely. The calculated $y$-value is known as the dependent variable, because its value depends on the chosen $x$-value.
\item The domain of a relation is the set of all the $x$ values for which there exists at least one $y$ value according to that relation. The range is the set of all the $y$ values, which can be obtained using at least one $x$ value.
\item The intercept is the point at which a graph intersects an axis. The $x$-intercepts are the points at which the graph cuts the $x$-axis and the $y$-intercepts are the points at which the graph cuts the $y$-axis. 
\item Only for graphs of functions whose highest power is more than $1$. There are two types of turning points: a minimal turning point and a maximal turning point. A minimal turning point is a point on the graph where the graph stops decreasing in value and starts increasing in value and a maximal turning point is a point on the graph where the graph stops increasing in value and starts decreasing. 
\item An asymptote is a straight or curved line, which the graph of a function will approach, but never touch.
\item A line about which the graph is symmetric
\item  The interval on which a graph increases or decreases
\item A graph is said to be continuous if there are no breaks in the graph. 
\end{itemize}
\item 
A set of certain x values has the following form: \{$x$ : conditions, more conditions\}
\item 

Here we write an interval in the form 'lower bracket, lower number, comma, upper number, upper bracket'
\item  You should know the following functions and their properties:
    \begin{itemize}[noitemsep]
    \item Functions of the form $y=ax+q$. These are straight lines.
    \item Functions of the Form $y=a{x}^{2}+q$ These are known as parabolic functions or parabolas.
    \item Functions of the Form $y=\frac{a}{x}+q$. These are known as hyperbolic functions.
    \item Functions of the Form $y=a{b}^{(x)}+q$. These are known as exponential functions.
    \end{itemize}
\end{itemize}

\begin{eocexercises}{}
\nopagebreak
\begin{enumerate}[noitemsep, label=\textbf{\arabic*}. ] 
\item Sketch the following straight lines: 
    \begin{enumerate}[noitemsep, label=\textbf{\alph*}. ] 
    \item $y=2x+4$ 
    \item $y-x=0$ 
    \item $y=-\frac{1}{2}x+2$
    \end{enumerate}
\item Sketch the following functions: 
    \begin{enumerate}[noitemsep, label=\textbf{\alph*}. ] 
    \item $y={x}^{2}+3$ 
    \item $y=\frac{1}{2}{x}^{2}+4$
    \item $y=2{x}^{2}-4$
    \end{enumerate}
\item Sketch the following functions and identify the asymptotes: 
    \begin{enumerate}[noitemsep, label=\textbf{\alph*}. ] 
    \item $y={3}^{x}+2$ 
    \item $y=-4.{2}^{x}+1$ 
    \item $y=2.{3}^{x}-2$ 
    \end{enumerate}
\item Sketch the following functions and identify the asymptotes: 
    \begin{enumerate}[noitemsep, label=\textbf{\alph*}. ] 
    \item $y=\frac{3}{x}+4$ 
    \item $y=\frac{1}{x}$ 
    \item $y=\frac{2}{x}-2$ 
    \end{enumerate}
\item Determine whether the following statements are true or false. If the statement is false, give reasons why:
    \begin{enumerate}[noitemsep, label=\textbf{\alph*}. ] 
    \item The given or chosen y-value is known as the independent variable.
    \item An intercept is the point at which a graph intersects itself.
    \item There are two types of turning points -- minimal and maximal.
    \item A graph is said to be congruent if there are no breaks in the graph.
    \item Functions of the form $y=ax+q$ are straight lines.
    \item Functions of the form $y=\frac{a}{x}+q$ are exponential functions.
    \item  An asymptote is a straight or curved line which a graph will intersect once.
    \item Given a function of the form $y=ax+q$ , to find the y-intersect put $x=0$ and solve for $y$.
    \item The graph of a straight line always has a turning point.
    \end{enumerate}
\item Given the functions $f(x)=-2{x}^{2}-18$ and $g(x)=-2x+6$
    \begin{enumerate}[noitemsep, label=\textbf{\alph*}. ] 
    \item Draw $f$ and $g$ on the same set of axes.
    \item Calculate the points of intersection of $f$ and $g$.
    \item Hence use your graphs and the points of intersection to solve for $x$ when:
	\begin{enumerate}[noitemsep, label=\textbf{\roman*}. ] 
	\item $f(x)>0$
	\item $\frac{f(x)}{g(x)}\le 0$
	\end{enumerate}
    \item Give the equation of the reflection of $f$ in the $x$-axis.
    \end{enumerate}
\item After a ball is dropped, the rebound height of each bounce decreases. The equation $y=5{(0,8)}^{x}$ shows the relationship between $x$, the number of bounces, and $y$, the height of the bounce, for a certain ball. What is the approximate height of the fifth bounce of this ball to the nearest tenth of a unit ?\newline
\item Mark had $15$ coins in R$~5$ and R$~2$ pieces. He had $3$ more R$~2$ coins than R$~5$ coins. He wrote a system of equations to represent this situation, letting $x$ represent the number of R$~5$ coins and $y$ represent the number of R$~2$ coins. Then he solved the system by graphing.
    \begin{enumerate}[noitemsep, label=\textbf{\alph*}. ] 
    \item Write down the system of equations.
    \item Draw their graphs on the same set of axes.
    \item What is the solution?
    \end{enumerate}

\end{enumerate}

\par \raisebox{-5 pt}{\includegraphics[width=0.5cm]{col11306.imgs/summary_www.png}} Find the answers with the shortcodes:
\par \begin{tabular}[h]{cccccc}
(1.) lTN  &  (2.) lTR  &  (3.) lTp  &  (4.) lTn  &  (5.) lTy  &  (6.) lx8  &  (7.) lxX  &  (8.) lx9  & \end{tabular}
\end{eocexercises}