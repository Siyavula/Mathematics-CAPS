\chapter{Funksies}
\setcounter{figure}{0}
\setcounter{subfigure}{0}

\section{Funksies in die regte w\^ereld}
Funksies is wiskundige verwantskappe tussen veranderlikes en hulle vorm boustene wat toepassings het in masjienontwerp, die voorspelling van natuurrampe, die mediese veld,
ekonomiese analise en vliegtuigontwerp. ’n Funksie het vir elke invoerwaarde net ’n enkele uitvoerwaarde. Dit is moontlik dat
’n funksie meer as een inset van verskillende veranderlikes kan hê, maar dan sal dit steeds net ’n enkele uitset hê. Ons gaan
egter nie in hierdie hoofstuk na sulke tipe funksies kyk nie.\par 
Een van die groot voordele van funksies is dat hulle visueel voorgestel kan word op die cartesiese vlak deur middel van ’n grafiek.
Grafieke is bloot ’n tekening van ’n funksie en dit word gebruik as ’n ander voorstellingswyse in wat makliker is omte interpreteer as by voorbeeld ’n tabel met getalle.
.\par 
Funksies se toepassing sluit van groot wetenskap- en ingenieursprobleme in, tot alledaagse probleme. So, dit is nuttig om meer
te leer van funksies. ’n Funksie is altyd afhanklik van een of meer veranderlikes, soos tyd, afstand of ’n meer abstrakte entiteit.\par 

n Paar tipiese voorbeelde van funksies waarmee jy moontlik bekend is:\par 
\begin{itemize}[noitemsep]
\item Die hoeveelheid geld wat jy het as ’n funksie van tyd. Hier is tyd die inset vir die funksie en die uitset is die bedrag geld. Jy sal op
enige oomblik net een bedrag geld hê. As jy verstaan hoe jou bedrag geld verander oor tyd, kan jy beplan hoe om jou
geld beter te spandeer. Besighede teken die grafiek van hulle geldsake oor tyd, sodat hulle kan sien wanneer hulle te
veel geld spandeer. Sulke waarnemings is nie altyd duidelik deur slegs na die getalle te kyk nie
\item Die temperatuur is ’n voorbeeld van ’n funksie met veelvuldige insette, insluitend die tyd van die dag, die seisoen, die
wolkbedekking, die wind, die plek en vele ander. Die belangrike ding om in te sien, is dat daar net een waarde vir
temperatuur is op ’n spesifieke plek, op ’n spesifieke tyd. As ons verstaan hoe die insette die temperatuur beïnvloed,
kan ons ons dag beter beplan.
\item Jou posisie is ’n funksie van tyd omdat jy nie op twee plekke op dieselfde tyd kan wees nie. Indien jy twee mense se
posisie as ’n funksie van tyd sou teken of stip (’plot’), sal die plek waar die lyne kruis, aandui waar die mense mekaar
ontmoet. Hierdie idee word gebruik in logistiek – ’n veld van Wiskunde wat probeer voorspel waar mense en items is,
hoofsaaklik vir besigheid.
\item Jou massa is ’n funksie van hoeveel jy eet en hoe baie oefening jy doen, maar elke persoon se liggaam hanteer die
insette anders en mens kan dan verskillende liggame voorstel as verskillende funksies.
\end{itemize}

\Definition{Funksie}{'n Funksie is 'n wiskundige verband tussen twee veranderlikes, waar elke invoerwaarde slegs een invoenvaarde het.}
\Note{Een uitvoerwaarde kan verskillende invoerwaardes h\^e.}




\subsection*{Afhanklike en Onafhanklike veranderlikes}
Die $x$-waarde van 'n funksie staan bekend as die invoerwaardes of onafhanklike veranderlikes omdat die waarde vrylik gekies. Die $y$-waarde
word bepaal deur die verband gebaseer op ’n gegewe of gekose $x$-waarde. Die berekende $y$ -waarde is bekend as die afhanklike veranderlikes, omdat die waardesafhanklik is van die gekose $x$-waardes.\par 

\subsection*{Definisieversameling en waardeversameling}

Die definisieversameling (ook bekend as die gebied) van ’n funskie is die stel onafhanklike $x$-waardes waarvoor daar $y$-waardes bestaan. 
\\Die waardeversameling (ook bekend as die terrein) is die stel afhanklike $y$ waardes wat
bepaal kan word deur die ($x$-waarde).\par 



\subsection*{Versamelkeurdernotasie}
Voorbeelde:
\\
\begin{table}[H]
\begin{tabular}{ |p{5cm} | p{8cm} | }
\hline
  $\{x: x \in \mathbb{R}, x > 0\}$ &  Die stel van alle $x$ waardes, waar $x$  ’n reële getal groter as $0$ is.
\\ \hline
    $\{y: y \in \mathbb{N}, 3 < y \leq 5}$ & Die stel van alle $y$-waardes wat so is dat $y$ 'n element is van die natuurlike getalle groter as $3$ en kleiner of gelyk aan $5$. 
\\ \hline
  $\{z: z \in \mathbb{Z}, z \leq 100}$ & Die stel van alle $z$-waardes wat so is dat $z$ 'n element is van die versameling heelgetalle kleiner of gelyk aan $100$.  
\\ \hline
\end{tabular}
\end{table}
\subsection*{Intervalnotasie}
Hierdie notasie kan nie gebruik word om heelgetalle in ’n interval te beskryf nie.
Voorbeelde:
\\
\begin{table}[H]
\begin{tabular}{ |p{5cm} | p{8cm} | }
\hline
  $(4;12)$ &  ’n Ronde hakie beteken die getal word uitgesluit uit die interval. Hierdie interval dui al die re\"ele getalle groter as $4$(maar $4$ uitgesluit) en kleiner as $12$(maar $12$ uitgesluit).
\\ \hline
 $(- \infty; -1)$ & Ronde hakies word altyd gebruik vir positief en negatief oneindig. 
\\ \hline
 $[1; 13)$ & ’n Reghoekige hakie beteken die getal word ingesluit by die interval. Hierdie interval dui al die re\"ele getalle groter of gelyk aan $1$ en kleiner maar nie gelyk aan $13$.
\\ \hline
\end{tabular}
\end{table}

\subsection*{Funksie notasie}
Dit is 'n baie handige manier om 'n funksie voor te stel. In stede van $y=2x+1$ skryf ons $f(x) = 2x+1$. ons s\^e  ``$f$ of $x$ is gelyk aan $2x+1$''. Enige letter kan gebruik word, b.v. $g(x)$, $h(x)$, $p(x)$ ens. 
\begin{enumerate}[noitemsep, label=\textbf{\arabic*}. ] 
 \item Bepaal die uitvoerwaarde: ``Vind die waarde van die funksie vir $x=-3$'' kan gesryf word as`` Vind $f(-3)$''.
\\Vervang $x$ met $-3$: $f(-3)=2(-3)+1=-5 \therefore f(-3)=-5$. \\
Dit beteken dat wanneer $x=-3$ is die waarde van die funksie $(y)$ $~=5$.
\item Bepaal die invoerwaarde: ``Vind die waarde van $x$ wat 'n $y$-waarde van $27$'' kan geskryf word as ``Vind $x$ if $f(x)=2x+1 = 27$''. \\
Ons skryf die volgende vergelyking en los op vir $x$: $2x+1 = 27 \therefore x=13$.\\
Dit beteken dat die funksie waarde van $x=13$, $(y)$ $~=27$ salwees.

\end{enumerate}
\Note{$x$-waarde = invoer waarde\\
$y$-waarde = uitvoer waarde/of funksie waarde}

\subsection*{Voorstellings van funksies}
Funksies kan op verskillende maniere voorgestel word vir verskillende doeleindes. 
\begin{enumerate}[noitemsep, label=\textbf{\arabic*}. ] 
 \item Woorde: die verband tussen tweegetalle is so dat  die een altyd $5$ minder is as die ander een.
\item Vloeidiagram: 
\begin{figure}[H]
\begin{center}
\scalebox{1} % Change this value to rescale the drawing.
{
\begin{pspicture}(0,-1.1360937)(6.7240624,1.1360937)
\psframe[linewidth=0.04,dimen=outer](3.7046876,0.32890624)(2.6246874,-0.75109375)
\psline[linewidth=0.04cm,arrowsize=0.05291667cm 2.0,arrowlength=1.4,arrowinset=0.4]{->}(4.1446877,0.14890625)(5.0246873,0.40890625)
\psline[linewidth=0.04cm,arrowsize=0.05291667cm 2.0,arrowlength=1.4,arrowinset=0.4]{->}(4.1446877,-0.57109374)(5.0046873,-0.85109377)
\psline[linewidth=0.04cm,arrowsize=0.05291667cm 2.0,arrowlength=1.4,arrowinset=0.4]{->}(4.1046877,-0.21109375)(4.9846873,-0.21109375)
\psline[linewidth=0.04cm,arrowsize=0.05291667cm 2.0,arrowlength=1.4,arrowinset=0.4]{<-}(2.073937,-0.58032197)(1.2008146,-0.86255836)
\psline[linewidth=0.04cm,arrowsize=0.05291667cm 2.0,arrowlength=1.4,arrowinset=0.4]{<-}(2.055675,0.13944638)(1.1888499,0.39754358)
\psline[linewidth=0.04cm,arrowsize=0.05291667cm 2.0,arrowlength=1.4,arrowinset=0.4]{<-}(2.104793,-0.21942326)(1.2250762,-0.24174327)
% \usefont{T1}{ptm}{m}{n}
\rput(0.44671875,0.93890625){Invoer:}
% \usefont{T1}{ptm}{m}{n}
\rput(6.0242186,0.93890625){Uitvoer:}
\usefont{T1}{ptm}{m}{n}
\rput(0.82609373,0.41890624){$-3$}
\usefont{T1}{ptm}{m}{n}
\rput(0.7753125,-0.22109374){$0$}
\usefont{T1}{ptm}{m}{n}
\rput(0.82953125,-0.92109376){$5$}
\usefont{T1}{ptm}{m}{n}
\rput(3.1746874,-0.19609375){\large $-5$}
\usefont{T1}{ptm}{m}{n}
\rput(5.368125,0.45890626){$-8$}
\usefont{T1}{ptm}{m}{n}
\rput(5.3646874,-0.24109375){$-5$}
\usefont{T1}{ptm}{m}{n}
\rput(5.3553123,-0.98109376){$0$}
\end{pspicture} 
}
\end{center}
\end{figure}
\item Tabel: 

 \begin{table}[H]
\begin{center}
  \begin{tabular}{|c|c|c|c|}
   \hline
Invoer($x$) & $-3$&$0$&$5$
\\ \hline
Uitvoer ($y$) &$-8$&$-5$&$0$
\\ \hline
  \end{tabular}
\end{center}
 \end{table}



\item Versameling geordende getallepare: $(-3;-8), (0;-5), (5;0)$
\item Algebra\"ese formule: $y = x-5$, $x \in \{-3; 0; 5\}$
\item Grafiek:
\begin{figure}[H]
\begin{center}
\scalebox{1} % Change this value to rescale the drawing.
{
\begin{pspicture}(0,-3.0284376)(6.52125,3.0684376)
\rput(3.0,-0.0284375){\psaxes[linewidth=0.04,arrowsize=0.05291667cm 2.0,arrowlength=1.4,arrowinset=0.4,labels=none,ticks=none,ticksize=0.10583333cm]{<->}(0,0)(-3,-3)(3,3)}
\usefont{T1}{ptm}{m}{n}
\rput(6.3754687,0.1615625){$x$}
\usefont{T1}{ptm}{m}{n}
\rput(3.3557813,2.9415624){$y$}
\psline[linewidth=0.04cm,arrowsize=0.05291667cm 2.0,arrowlength=1.4,arrowinset=0.4]{<->}(2.0,-2.0884376)(5.18,0.8915625)
\psdots[dotsize=0.16](2.98,-1.1684375)
\psdots[dotsize=0.16](4.16,-0.0284375)
\usefont{T1}{ptm}{m}{n}
\rput(4.1448436,0.2415625){$5$}
\usefont{T1}{ptm}{m}{n}
\rput(3.3,-1.2384375){$-5$}
\usefont{T1}{ptm}{m}{n}
\rput(2.9814062,-0.0584375){$O$}
\end{pspicture} 
}
\end{center}
\end{figure}
\end{enumerate}

\begin{exercises}{}
{
Skryf die volgende in keurdernotasie:
\begin{enumerate}[noitemsep, label=\textbf{\arabic*}. ] 
 \item $(- \infty; 7]$
\item $[13;4)$
\item $(35; \infty)$
\item $[\frac{3}{4}; 21)$
\item $[-\frac{1}{2}; \frac{1}{2}$
\item $(-3; \infty)$
\end{enumerate}

Skryf die volgende in interval notasie:
\begin{enumerate}[noitemsep, label=\textbf{\arabic*}. ] 
\setcounter{enumi}{6}
 \item $\{p: p \in \mathbb{R}, p \leq 6\}$
 \item $\{k: k \in \mathbb{R}, -5 < k < 5}$
 \item $\{x: x \in \mathbb{R}, x > \frac{1}{5}\}$
 \item $\{z: z \in \mathbb{R}, 21 \leq z < 41\}$
\end{enumerate}

\insertpracticeinfo{10}
} 
\end{exercises}

\section{Reguitlynfunksies in die vorm $y=mx+c$}


\subsection*{Teken die grafiek}       
Funksies met die algemene vorm $y=mx+c$ word reguitlynfunksies genoem. In die vergelyking, $y=mx+c$, is $m$ en $c$ konstantes en het verskillende invloede op die grafiek van die funksie. 

\begin {wex}{Trek reguitlyngrafiek}
{
\begin{figure}[H]
\begin{center}
\begin{pspicture}(-6,-2)(6,4.6)
\psset{yunit=0.3}
\psaxes[Dy=3]{<->}(0,0)(-6,-8)(6,14)
\psplot[plotstyle=curve,arrows=<->]{-5}{5}{x 2 mul 3 add}
\psplot[plotstyle=dots,arrows=<->,plotpoints=9]{-4}{4}{x 2 mul 3 add}
\rput(0, 0){$0$}
\rput(0.3, 13.6){$y$}
\rput(6.3, 0.3){$x$}
\rput(4.5, 9){$f(x)=2x+3$}
%\psplot[showpoints=true](-5,-7)(-5,-7)(-4,-5)(-3,-3)(-2,-1)(-1,1)(-1.5,0)(0,3)(1,5)(2,7)(3,9)(4,11)(5,13)
\rput(-1.6,1.5){$-\frac{3}{2}$}
\end{pspicture}
% \caption{Graph of $f(x) = 2x + 3$}
% \label{fig:mf:g:straightline}
\end{center}
\end{figure}  
}
{
\westep{Step 1}
Steps needed here
}
\end{wex}

  

\subsection*{Ondersoek die invloed van $m$ en $c$}

\mindsetvid{Analysing the gradient}{VMaxf}

\begin{Ondersoek}{Die invloed van $m$ en $c$ op 'n reguitlyn grafiek}

Op dieselfde assestelsel, trek die volgende grafieke:
\begin{enumerate}[noitemsep, label=\textbf{\arabic*}. ] 

    \item $a(x)=x-2$
    \item $b(x)=x-1$
    \item $c(x)=x$
    \item $d(x)=x+1$
    \item $e(x)=x+2$
    \end{enumerate}
Gebruik jou resultate om die invloed van verskillende waardes van $c$ op die resulterende grafieke af te lei.\\
\\

Op dieselfde assestelsel, trek die volgende grafieke:

    \begin{enumerate}[noitemsep, label=\textbf{\arabic*}. ] 
\setcounter{enumi}{5}
    \item $f(x)=-2x$
    \item $g(x)=-x$
    \item $j(x)=x$
\item $k(x)=2x$
    \end{enumerate}
Gebruik jou resultate om die invloed van verskillende waardes van $m$ op die resulterende grafiek af te lei.
\end{Investigation}


\begin{table}[htb]
\begin{center}
% \caption{Table summarising general shapes and positions of graphs of functions of the form $y=mx+q=c$.}
\label{tab:mf:graphs:summarystr10}
\begin{tabular}{|c|c|c|c|}\hline
& $m<0$&$m=0$ & $m>0$\\ \hline
$c>0$&
\begin{pspicture}(-1.2,-1.2)(1.2,1.2)
\psset{yunit=0.25,xunit=0.25}
\psaxes[arrows=<->,dx=0,Dx=10,dy=0,Dy=10](0,0)(-4,-4)(4,4)
\psplot[plotstyle=curve,arrows=<->]{-2.5}{2.5}{x neg 1 add}
\end{pspicture}

&
\begin{pspicture}(-1.2,-1.2)(1.2,1.2)
\psset{yunit=0.25,xunit=0.25}
\psaxes[arrows=<->,dx=0,Dx=10,dy=0,Dy=10](0,0)(-4,-4)(4,4)
\psplot[plotstyle=curve,arrows=<->]{-2.5}{2.5}{1.5}
\rput(-0.5, 2.1){\footnotesize$q$}
\end{pspicture}
&
\begin{pspicture}(-1.2,-1.2)(1.2,1.2)
\psset{yunit=0.25,xunit=0.25}
\psaxes[arrows=<->,dx=0,Dx=10,dy=0,Dy=10](0,0)(-4,-4)(4,4)
\psplot[plotstyle=curve,arrows=<->]{-2.5}{2.5}{x 1 add}
\end{pspicture}
\\\hline
$c=0$&
\begin{pspicture}(-1.2,-1.2)(1.2,1.2)
\psset{yunit=0.25,xunit=0.25}
\psaxes[arrows=<->,dx=0,Dx=10,dy=0,Dy=10](0,0)(-4,-4)(4,4)
\psplot[plotstyle=curve,arrows=<->]{-2.5}{2.5}{x neg}
\end{pspicture}

&
---
&

\begin{pspicture}(-1.2,-1.2)(1.2,1.2)
\psset{yunit=0.25,xunit=0.25}
\psaxes[arrows=<->,dx=0,Dx=10,dy=0,Dy=10](0,0)(-4,-4)(4,4)
\psplot[plotstyle=curve,arrows=<->]{-2.5}{2.5}{x}
\end{pspicture}
\\ \hline
$c<0$
&

\begin{pspicture}(-1.2,-1.2)(1.2,1.2)
\psset{yunit=0.25,xunit=0.25}
\psaxes[arrows=<->,dx=0,Dx=10,dy=0,Dy=10](0,0)(-4,-4)(4,4)
\psplot[plotstyle=curve,arrows=<->]{-2.5}{2.5}{x neg 1 sub}
\end{pspicture}
&
\begin{pspicture}(-1.2,-1.2)(1.2,1.2)
\psset{yunit=0.25,xunit=0.25}
\psaxes[arrows=<->,dx=0,Dx=10,dy=0,Dy=10](0,0)(-4,-4)(4,4)
\psplot[plotstyle=curve,arrows=<->]{-2.5}{2.5}{1.5 neg}
\rput(-0.5, -2.1){\footnotesize$q$}
\end{pspicture}
&
\begin{pspicture}(-1.2,-1.2)(1.2,1.2)
\psset{yunit=0.25,xunit=0.25}
\psaxes[arrows=<->,dx=0,Dx=10,dy=0,Dy=10](0,0)(-4,-4)(4,4)
\psplot[plotstyle=curve,arrows=<->]{-2.5}{2.5}{x 1 sub}

\end{pspicture}
\\\hline
\end{tabular}
\end{center}
\end{table}

\textbf{Die invloed van $m$}\\
Jy behoort te vind dat die waarde van $m$ die helling van die grafiek beïnvloed. Soos $m$ vermeerder, vermeerder die
helling van die grafiek ook. Indien $m>0$ sal die grafiek vermeerder van links na regs (opwaartse helling). Indien $m<0$ sal die grafiek verminder van links na regs (afwaartse helling). Dit is hoekom daar na, $m$  verwys word as die helling of die gradiënt van ’n reguitlynfunksie.\par 

\textbf{Die invloed van $c$}\\
Jy behoort ook te vind dat die waarde van $c$ die punt bepaal waar die grafiek die $y$-as sny. Om hierdie rede, staan $c$ bekend as die y-afsnit. As $c$ toeneem, skryf die grafieke vertikaal opwaarts. As $c$ afneem, skryf die grafiek vertikaal afwaarts.\par 

% These different properties are summarised in Table 1.7.\par 

\subsection*{Ontdek die kenmerke van $y=mx +c$} 
Die standaardvorm van a reguitlynfunksie is:  $y=mx + c$. 
\subsection*{Definisieversameling en waardeversameling}
\nopagebreak
Vir $f(x)=mx+c$, is die definisieversameling $\{x:x\in \mathbb{R}\}$ omdat daar geen waarde is van  $x\in \mathbb{R}$ waarvoor $f(x)$ ongedefinieërd is nie.\par 
Die waardeversameling van $f(x)=mx+c$ is ook $\{f(x):f(x)\in \mathbb{R}\}$ omdat $f(x)$ enige re\"ele waarde kan h\^e.\par 
\par 

\subsection*{Afsnitte}
\textbf{Die $y$-afsnit:}\\
Elke punt op die $y$-as het 'n $x$-Ko\"ordinaat van $0$. Daarom, bereken die $y$-afsnit deur $x=0$ te stel.\par
Byvoorbeeld, die $y$-afsnit van $g(x)=x-1$ word bepaal deur $x=0$ te stel en dan op te los:\par 

\begin{equation*}
\begin{array}{ccl}\hfill g(x)& =& x-1\hfill \\
\hfill g(0) &=& 0-1\hfill \\
& =& -1\hfill 
\end{array}
\end{equation*}
This gives the point $(0;-1)$.\par

\textbf{Die $x$-afsnitte word as volg bereken:}\\
Elke punt op die $x$-as het 'n $y$-Ko\"ordinaat van $0$. Daarom, bereken die $x$-afsnit deur $y=0$ te stel. \par
Byvoorbeeld, die $x$-afsnit van $g(x)=x-1$ word gegee deur $y=0$ in te stel en dan op te los:\par 

\begin{equation*}
\begin{array}{ccl}\hfill g(x)& =& x-1\hfill \\
\hfill 0& =& x-1\hfill \\
\hfill x& =& 1\hfill 
\end{array}
\end{equation*}
Dit gee die punt $(1;0)$.


\subsection*{Skets grafieke van die vorm $f(x)=mx+c$}


Om die grafieke van die vorm, $f(x)=mx+c$, te skets, het ons die volgende drie kenmerkende einskappe nodig:\par 
\begin{enumerate}[noitemsep, label=\textbf{\arabic*}. ] 
\item die teken van $m$
\item $y$-afsnit en waarde van
\item $x$-afsnit
\end{enumerate}
Slegs twee punte word benodig om ’n reguitlyn te trek. Die maklikste punte is die $x$-afsnit (waar die lyn die x-as
sny) en die $y$-afsnit.\par 

\subsection*{Afsnitte metode}
\begin{wex}{Skets 'n reguitlyngrafiek deur die afsnitte metode}
{Skets die grafiek van $g(x)=x-1$ met die afsnitte metode}
{
\westep{Ondersoek die standaardvorm van die funksie}
$m>0$. Due neem die grafieke toe soos $x$ toeneem.

\westep{Bereken afsnitte}
Die $y$-afsnit word bepaal deur $x=0$ te stel; Daarom $g(0)=-1$. $\therefore$ $y$-afsnit is  $(0;-1)$. \\

Die $x$-afsnit word bepaal deur $y=0$ te stel; daarom $x=1$.  $\therefore$ $x$-afsnit is $(1;0)$. 

\westep{Stip punte en trek die grafiek}

\begin{center}
\begin{pspicture}(-4,-4)(4,3)
%\psgrid
\psset{yunit=0.75,xunit=0.75}
\psaxes[arrows=<->](0,0)(-5,-5)(5,4)
\psplot[plotstyle=curve,arrows=<->]{-4}{4}{x 1 sub}
\psdots(0,-1)(1,0)
\uput[r](0,-1.3){$(0;-1)$}
\uput[ul](1.3,0.3){$(1;0)$}
\rput(5.2,0.3){$x$}
\rput(0.3, 4.2){$y$}
\end{pspicture}
% \caption{Graph of the function $g(x)=x-1$}
% \label{fig:mf:g:sketchexamplestr}
\end{center}

Let op: hierdie grafiek is kontinu en gaan voort in beide rigtings.       
}
\end{wex}

\subsection*{Gradient en $y$-afsnit metode}
Ons kan 'n reguitlyn grafiek trek van $y=mx+c$ deur die gradi\"ent $m$ en die $y$-afsnit $c$ te gebruik. \par Ons bereken die $y$-afsnit deur $x=0$. Die gee een punt waardeur die grafiek gaan. Bereken 'n ander punt seur die gradi\"ent ($m$) te gebruik.\par

Die gradi\"ent van 'n lyn is die maatstat van helling. Helling word gegee deur die verhouding vertikale verandering tot horisontale verandering te bepaal:
\begin{equation*}
m = \dfrac{\mbox{\footnotesize verandering in $y$}}{\mbox{\footnotesize verandering in $x$}} = \dfrac{\mbox{\footnotesize vertikale change}}{\mbox{\footnotesize horisontale change}}
\end{equation*}
Byvoorbeeld: $y=\frac{3}{2}x-1$, Daarom $m > 0$ en $y$ neem toe as $x$ toeneem
\begin{equation*}
 m = \dfrac{\mbox{\footnotesize change in $y$}}{\mbox{\footnotesize change in $x$}} = \dfrac{3\uparrow}{2\rightarrow} = \dfrac{-3\downarrow}{-2\leftarrow}
\end{equation*}
\begin{center}
\scalebox{1} % Change this value to rescale the drawing.
{
\begin{pspicture}(0,-4.7584376)(9.34125,4.7584376)
\rput(4.0,0.3215625){\psaxes[linewidth=0.04,labels=none,ticksize=0.2cm, arrows=<->](0,0)(-4,-5)(5,4)}
\psline[linewidth=0.04cm,dotsize=0.07055555cm 2.0]{*-*}(1.92,-3.6984375)(5.96,2.3815625)
\psarc[linewidth=0.04](4.49,2.3515625){0.49}{0.0}{180.0}
\psarc[linewidth=0.04](5.51,2.3315625){0.51}{0.0}{180.0}
\rput{178.47418}(6.9182076,-7.536676){\psarc[linewidth=0.04](3.509283,-3.7222767){0.49}{0.0}{180.0}}
\rput{178.47418}(4.8816133,-7.4152513){\psarc[linewidth=0.04](2.4901774,-3.675124){0.51}{0.0}{180.0}}
\rput{88.38591}(3.753576,-4.18539){\psarc[linewidth=0.04](4.029283,-0.1622766){0.49}{0.0}{180.0}}
\rput{273.13806}(6.9809837,0.9839488){\psarc[linewidth=0.04](4.010177,-3.195124){0.51}{0.0}{180.0}}
\rput{273.13806}(5.9635777,1.9092364){\psarc[linewidth=0.04](3.9901774,-2.195124){0.51}{0.0}{180.0}}
\rput{273.13806}(4.9839826,2.874464){\psarc[linewidth=0.04](4.010177,-1.1951239){0.51}{0.0}{180.0}}
\rput{88.38591}(4.694869,-3.1535814){\psarc[linewidth=0.04](3.969283,0.8377234){0.49}{0.0}{180.0}}
\rput{88.38591}(5.6944723,-2.1817489){\psarc[linewidth=0.04](3.969283,1.8377234){0.49}{0.0}{180.0}}
\usefont{T1}{ptm}{m}{n}
\rput(9.195469,0.6515625){$x$}
\usefont{T1}{ptm}{m}{n}
\rput(4.355781,4.6315627){$y$}
\usefont{T1}{ptm}{m}{n}
\rput(5.1275,3.1515625){$+2\rightarrow$}
\usefont{T1}{ptm}{m}{n}
\rput(2.9310937,0.9715625){$+3\uparrow$}
\usefont{T1}{ptm}{m}{n}
\rput(5.161406,-2.1884375){$-3\downarrow$}
\usefont{T1}{ptm}{m}{n}
\rput(2.9578125,-4.6084375){$-2\leftarrow$}
\end{pspicture} 
}
\end{center}

\begin{wex}{Skets reguitlyngrafieke deur gradi\"ent en $y$-afsnit metode te gebruik}
{Skets reguitlyngrafiek $y=\frac{1}{2}x-3$ deur gradi\"ent en $y$-afsnit metode te gebruik}
{
\westep{Gebruik $y$-afsnit}
 \\
$c=-3$ wat die punt $(0;-3)$ gee.

\westep{Gebruik die gradi\"ent}

\begin{equation*}
 m = \dfrac{\mbox{\footnotesize verandering in $y$}}{\mbox{\footnotesize verandering in $x$}} = \dfrac{1\uparrow}{2\rightarrow} = \dfrac{-1\downarrow}{-2\leftarrow}
\end{equation*}
Begin by $(0;-3)$. Beweeg $1$ eenheid op en $2$ eenhede regs. Dit gee die tweede punt $(2;-2)$. \\
Of, beweeg $1$ eenheid af en $2$ eenhede links. Dit gee die tweede punt $(-2;-4)$.

\westep{Stip punte en trek die grafiek}

\begin{center}
\begin{pspicture}(-5,-5)(5,5)
%\psgrid
\psset{yunit=0.75,xunit=0.75}
\psaxes[arrows=<->](0,0)(-6,-6)(7,4)
\psplot[plotstyle=curve,arrows=<->]{-4}{7}{x .5 mul 3 sub}
\psline[linewidth=.7pt,arrows=->](0,-2)(2,-2)
\psline[linewidth=.7pt,arrows=->](0,-3)(0,-2)
\psdots(0,-3)(2,-2)
\uput[r](0,-3.3){$(0;-3)$}
\uput[ul](4,-2.5){$(2;-2)$}
\rput(7.4,0.3){$x$}
\rput(0.3, 4.2){$y$}
\rput(-0.2, -2.5){\footnotesize$1$}
\rput(1, -1.7){\footnotesize$2$}
\end{pspicture}
% \caption{Graph of the function $g(x)=x-1$}
% \label{fig:mf:g:sketchexamplestr}
\end{center}

Let op: Die grafiek is kontinu en strek in beide rigtings.       
}
\end{wex}

\Tip{Skryf altyd die funksie vorskrif in die vorm $y=mx+c$ en let op die waarde van $m$. Na die grafiek getrek is, maak seker dat as $m>0$, neem die grafiek toe en as $m<0$, neem die grafiek af as $x$ toeneem.}


\begin{exercises}{}
{
\nopagebreak

 Gee die $x$- en $y$-afsnitte van die volgende reguitlyn grafieke. Dui aan of die grafiek toeneem of afneem as $x$ toeneem:
    \begin{enumerate}[noitemsep, label=\textbf{\arabic*}. ] 
    \item $y=x+1$
    \item $y=x-1$
    \item $y=2x-1$
    \item $y+1=2x$
\item $3y-2x=6$
\item$y=-3$
\item $x=3y$
\item $\frac{x}{2} - \frac{y}{3} = 1$
    \end{enumerate}


Gee die vergelyking, gebied en terrien van die funksies in die skets:
    \begin{enumerate}[noitemsep, label=\textbf{\arabic*}. ] 
\setcounter{enumi}{8}
    \item $a(x)$
%etc - other functions on graph go here
    \end{enumerate} 
\setcounter{subfigure}{0}
\begin{figure}[H]
\begin{center}
\scalebox{1} % Change this value to rescale the drawing.
{
\begin{pspicture}(0,-4.1467185)(9.519062,4.1867185)
\rput(4.0,-0.14671862){\psaxes[linewidth=0.03,tickstyle=bottom,labels=none,ticks=none,ticksize=0.08cm, arrows=<->](0,0)(-4,-4)(4,4)}
\psline[linewidth=0.04cm](2.78,1.7732813)(7.76,-0.9067186)
\usefont{T1}{ptm}{m}{n}
\rput(4.2023435,3.9832811){$y$}
\usefont{T1}{ptm}{m}{n}
\rput(8.26,-0.036718626){$x$}
\usefont{T1}{ptm}{m}{n}
\rput(4.4696875,1.3432813){$(0;3)$}
\usefont{T1}{ptm}{m}{n}
\rput(6.9896874,0.043281376){$(4;0)$}
\usefont{T1}{ptm}{m}{n}
\rput(3.2445312,1.9167186){$a(x)$}
\psline[linewidth=0.04cm](3.2,-3.6332815)(8.42,2.3067186)
\usefont{T1}{ptm}{m}{n}
\rput(4.6996875,-2.8167186){$(0;-6)$}
\usefont{T1}{ptm}{m}{n}
\rput(8.894531,2.4967186){$b(x)$}
\psline[linewidth=0.04cm](0.98,1.1267186)(7.88,1.1067187)
\usefont{T1}{ptm}{m}{n}
\rput(8.3,1.1567186){$c(x)$}
\psline[linewidth=0.04cm](7.4,-2.0532813)(1.08,1.4467186)
\usefont{T1}{ptm}{m}{n}
\rput(7.9,-2.1032813){$d(x)$}
\usefont{T1}{ptm}{m}{n}
\rput(4.2,0.1){$0$}
\psline[linewidth=0.04](4.7,-0.87328136)(4.58,-0.45328137)(5.0,-0.45328137)
\psline[linewidth=0.04](4.94,0.30671862)(4.82,0.7267186)(5.24,0.7267186)
\end{pspicture} 
}
\end{center}

\end{figure}  
            
\begin{enumerate}[noitemsep, label=\textbf{\arabic*}. ] 
\setcounter{enumi}{12}
\item Skets die volgende funksies op dieselfde stel asse deur die afsnitte metode te gebruik. Dui duidelik die as-afsnitte aan as ook die ko\"ordinate van die snypunt van die grafiek van: $x+2y-5=0$ en $3x-y-1=0$
\item Trek die grafieke van $f(x)=3-3x$ en $g(x)=\frac{1}{3}x+1$ met die gradi\"ent-afsnit metode.
\end{enumerate}

\insertpracticeinfo{14}
}
\end{exercises}
   

\section{Paraboliese funksies in die vorm $y=a{x}^{2}+q$}
\subsection*{Skets die grafiek}         
Funksies van die algemene vorm  $y=a{x}^{2}+q$  word paraboliese funksies genoem. In die vergelyking $y=a{x}^{2}+q$, is $a$ en $q$ konstantes wat die parabool op bepaalde maniere be\"invloed. 

\mindsetvid{The quadratic function}{VMaxl}

\begin{wex}{Teken die grafiek van 'n kwadratiese funksie}
{
\begin{equation*}
 y = f(x) = x^{2}
\end{equation*}

Voltooi die volgende tabel van funksiewaardes vir $f(x)=x^{2}$ en stel die waardes op dieselfde assestelsel voor:
\\
\begin{center}
\begin{tabular}{|c|c|c|c|c|c|c|c|}
\hline
  $x$ &  $-3$ & $-2$ & $-1$ & $0$ & $1$ & $2$ & $3$
\\ \hline
 $f(x)$& $9$ &&&&&&
\\ \hline
\end{tabular}
\end{center}
\vspace{10pt}
\begin{enumerate}[noitemsep, label=\textbf{\arabic*}. ] 
 \item Verbind die punte met 'n gladde kurve.
\item Die gebied van $f$ is $x \in \mathbb{R}$. Bepaal die terrein.
\item Bepaal die as van simmetrie $f$
\item Bepaal die waarde van $x$ as $f(x) = \frac{25}{4}$. \\Bevestig jou antwoord grafies.
\item Skryf neer waar die grafiek die asse sny.
\end{enumerate}
}
{
\westep{Vervang die waardes in die vergelyking}
\begin{equation*}
 \begin{array}{cclcc}
  f(x) &=& x^{2} & &\\
 f(-3) &=& (-3)^{2} &=& 9 \\ 
 f(-2) &=& (-2)^{2} &=& 4 \\
 f(-1) &=& (-1)^{2} &=& 1 \\
f(0) &=& 0^{2} &= &0 \\
f(1) &=& (1)^{2} &= &1 \\ 
f(2) &=& (2)^{2} &= &4 \\
f(3) &=& (3)^{2} &= &9
 \end{array}
\end{equation*}
\\
\\
\begin{center}
\begin{tabular}{|c|c|c|c|c|c|c|c|}
\hline
  $x$ &  $-3$ & $-2$ & $-1$ & $0$ & $1$ & $2$ & $3$
\\ \hline
 $f(x)$& $9$ &$4$&$1$&$0$&$1$&$4$&$9$
\\ \hline
\end{tabular}
\end{center}

\westep{Stip die punt en verbind hulle met 'n dladde kromme.}
Ons kry die volgende punte uit die tabel: \\
$(-3;9)$, $(-2;4)$, $(-1;1)$, $(0;0)$, $(1;1)$, $(2;4)$, $(3;9)$ \\
En die volgende grafiek: 
\begin{figure}[H]
\begin{center}
\begin{pspicture}(-5,-1)(5,5)
%\psgrid-\infty ;q
\psaxes[arrows=<->,dy=0.5](0,0)(-5,-1)(5,5)
\psset{yunit=0.5}
\psplot[plotstyle=curve,arrows=<->]{-3}{3}{x 2 exp}
\psdots(-2.5, 6.25)(2.5,6.25)
\rput(-2.2, 6.3){$B$}
\rput(2.2, 6.3){$A$}
\rput(0.1, 0.3){$0$}
\end{pspicture}
% \caption{Graph of $f(x)=x^2-1$.}
\label{fig:mf:g:parabola10}
\end{center}
\end{figure}    

\westep{Bepaal die terrein}
Domain: $x \in \mathbb{R}$\\
Van die grafiek sien ons dat vir alle $x$ waardes sal $y$ groter of gelyk aan $0$ wees.\\
Range: $y \in [0; \infty)$

\westep{Vind die simmetrie}
$f$ is simmertries rondom die $y$-as. Dus is die lyn $x=0$ die as van simmertrie. 

\westep{Bepaal die $x$-waarde}
\begin{equation*}
 \begin{array}{ccl}
f(x) &=& \frac{25}{4} \\
\therefore \frac{25}{4} &=& x^{2} \\
x &=& \pm \frac{5}{2} 
\end{array}
\end{equation*}
Sien punte $A$ en $B$ on die grafiek.

\westep{Bepaal die as-afsnitte}
Funskie $f(x)$ sny die asse by die oorspring $(0;0)$. \\
Let op as die waarde van $x$ toeneem van $-\infty$ tot $0$, verminder $f(x)$. By die draaipunt $(0;0)$,is $f(x) = 0$. $x$ die minimumwaarde van die funksie, soos $x$ toeneem van $0$ tot $\infty$, neem $f(x)$ toe.
}
\end{wex}




  

\subsection*{Ondersoek die invloed van $a$ en $q$ }
\begin{Ondersoek}{Die invloed van $a$ en $q$ op 'n paraboliese grafiek}
Voltooi die tabel en teken die volgende grafiek op dieselfde assesstelsel:
    \begin{enumerate}[noitemsep, label=\textbf{\arabic*}. ] 
  \item $a(x)={x}^{2}-2$
    \item $b(x)={x}^{2}-1$
    \item $c(x)={x}^{2}$
    \item $d(x)={x}^{2}+1$
    \item $e(x)={x}^{2}+2$
        \end{enumerate}

\begin{table}[H]
% \begin{table}[H]
% \\ '' '0'
\begin{center}

\noindent

\begin{tabular}{|l|l|l|l|l|l|}\hline
 $x$&
$-2$&
$-1$&
$0$&
$1$&
$2$
\\ \hline


$a(x)$
&
&
&
&
&
\\ \hline

$b(x)$
&
&
&
&
&
\\ \hline

$c(x)$
&
&
&
&
&
\\ \hline

$d(x)$
&
&
&
&
&
\\ \hline

$e(x)$
&
&
&
&
&
\\ \hline

\end{tabular}
\end{center}
% \begin{center}{\small\bfseries Table 1.8}\end{center}
% \begin{caption}{\small\bfseries Table 1.8}\end{caption}
\end{table}
Gebruik jou resultate en maak 'n afleiding oor die invloed van $q$.
\\
Voltooi die tabel en teken die volgende grafiek op dieselfde assesstelsel:
    \begin{enumerate}[noitemsep, label=\textbf{\arabic*}. ] 
\setcounter{enumi}{5}
  \item $f(x)=-2{x}^{2}$
    \item $g(x)=-{x}^{2}$
    \item $h(x)={x}^{2}$
    \item $i(x)=2{x}^{2}$
    \end{enumerate}
\begin{table}[H]
\begin{center}
\begin{tabular}{|l|l|l|l|l|l|}\hline
$x$&
$-2$&
$-1$&
$0$&
$1$&
$2$
\\ \hline

$f(x)$&
&
&
&
&
\\ \hline

$g(x)$&
&
&
&
&
\\ \hline


$h(x)$
&
&
&
&
&
\\ \hline

$i(x)$
&
&
&
&
&
\\ \hline

\end{tabular}
\end{center}
% \begin{center}{\small\bfseries Table 1.8}\end{center}
% \begin{caption}{\small\bfseries Table 1.8}\end{caption}
\end{table}

Gebruik jou resultate en maak 'n afleiding oor die invloed van $a$.

\end{Investigation}

\begin{table}[H]
\begin{center}
% \caption{Table summarising general shapes and positions of graphs of functions of the form $y=mx+q=c$.}
\label{tab:mf:graphs:summarystr10}
\begin{tabular}{|c|c|c|}
\hline
 & $a<0$ & $a>0$
\\ \hline
$q>0$&
\begin{pspicture}(-1.2,-1.2)(1.2,1.2)
\psset{yunit=0.25,xunit=0.25}
\psaxes[arrows=<->,dx=0,Dx=10,dy=0,Dy=10](0,0)(-4,-4)(4,4)
\psplot[plotstyle=curve,arrows=<->]{-1.6}{1.6}{x 2 exp neg 1 add}
\end{pspicture}

&

\begin{pspicture}(-1.2,-1.2)(1.2,1.2)
\psset{yunit=0.25,xunit=0.25}
\psaxes[arrows=<->,dx=0,Dx=10,dy=0,Dy=10](0,0)(-4,-4)(4,4)
\psplot[plotstyle=curve,arrows=<->]{-1.6}{1.6}{x 2 exp 1 add}
\end{pspicture}
\\\hline
$q=0$&
\begin{pspicture}(-1.2,-1.2)(1.2,1.2)
\psset{yunit=0.25,xunit=0.25}
\psaxes[arrows=<->,dx=0,Dx=10,dy=0,Dy=10](0,0)(-4,-4)(4,4)
\psplot[plotstyle=curve,arrows=<->]{-1.6}{1.6}{x 2 exp neg}
\end{pspicture}
&
\begin{pspicture}(-1.2,-1.2)(1.2,1.2)
\psset{yunit=0.25,xunit=0.25}
\psaxes[arrows=<->,dx=0,Dx=10,dy=0,Dy=10](0,0)(-4,-4)(4,4)
\psplot[plotstyle=curve,arrows=<->]{-1.6}{1.6}{x 2 exp }
\end{pspicture}

\\ \hline
$q<0$
&

\begin{pspicture}(-1.2,-1.2)(1.2,1.2)
\psset{yunit=0.25,xunit=0.25}
\psaxes[arrows=<->,dx=0,Dx=10,dy=0,Dy=10](0,0)(-4,-4)(4,4)
\psplot[plotstyle=curve,arrows=<->]{-1.6}{1.6}{x 2 exp neg 1 sub}
\end{pspicture}
&

\begin{pspicture}(-1.2,-1.2)(1.2,1.2)
\psset{yunit=0.25,xunit=0.25}
\psaxes[arrows=<->,dx=0,Dx=10,dy=0,Dy=10](0,0)(-4,-4)(4,4)
\psplot[plotstyle=curve,arrows=<->]{-1.6}{1.6}{x 2 exp 1 sub}
\end{pspicture}
\\\hline
\end{tabular}
\end{center}
\end{table}

\textbf{Die invloed $q$}
\\
Die waarde van $q$ word 'n vertikale skuif omdat alle punte dieselfde afstand in die selfde rigting beweeg (dit skyf die hele grafiek op of af). 
\begin{itemize}
\item Indien $q>0$, sal die grafiek van $f(x)$ $q$ eenhede vertikaal opwaarts skyf. Dir draaipunt van $f(x)$ is bokant die $y$-as.
\item Indien $q<0$, sal die grafiek van $f(x)$ $q$ eenhede vertikaal afwaarts skyf. Die draaipunt van $f(x)$ is onderkant die $y$-as.
\end{itemize}
\textbf{Die invloed van $a$}
\\
Die waarde van $a$ bepaal die vorm van die grafiek. 
\begin{itemize}
 \item Indien $a>0$, sal die grafiek $f(x)$  ``glimlag'' en het 'n minimum draaipunt by $(0;q)$.\\
Die grafiek van $f(x)$ word opwaarts gestek. Soos $a$ groter word, word die grafiek nouer. 
\\Vir $0<a<1$, Dus as $a$ 'n positiewe breuk is, sal die grafiek van $f(x)$ wyder word.
\item Indien $a<0$, sal die grafiek van $f(x)$ 'frons' en het 'n maksimum draaipunt $(0;q)$. 
\\Die grafiek van $f(x)$ word vertikaal afwaarts gerek; soos $a$ kleiner word, word die grafiek nouer. \\
Vir $-1<a<0$, dus $a$ 'n negatiewe breuk, sal die grafiek $f(x)$ wyder word.
\end{itemize}

\setcounter{subfigure}{0}
\begin{figure}[!ht]
\begin{center}
\begin{pspicture}(-4,-0.5)(6,2)
%\psgrid
\psset{yunit=0.5}
\psplot[plotstyle=curve,arrows=<->]{-2}{0}{x 1 add 2 exp}
\psplot[plotstyle=curve,arrows=<->]{3}{5}{x 4 sub 2 exp neg 1 add}
\psdots(-1.5,3)(-0.5,3)(3.5,3)(4.5,3)
\uput[d](-1,-0.5){$a>0$ ($a$ 'n positiewe glimlag)}
\uput[d](4,-0.5){$a<0$ ($a$ 'n negatiewe frons)}
\end{pspicture}
% \caption{Distinctive shape of graphs of a parabola if $a>0$ and $a<0$.}
\label{fig:mf:g:parabola10a}
\end{center}
\end{figure}   

This simulasie stel die invloed van veranderinge aan $a$ en $q$. Let daarop dat $q = c$ in hierdie simulasie. 'n Ekstra term, $bx$, is bygevoeg. Jy kan voorlopig $bx$ as $0$ los, of jy kan ondersoek watter invloed dit het op die grafiek.


\subsection*{Ondek die kenmerke van $y=ax^2{2} + q$}
Die standardvorm die paraboliese funksie is:  $y=ax^{2} + q$.
\subsection*{Definisieversameling en Waardeversameling}

Vir $f(x)=a{x}^{2}+q$, is die gebied $\{x:x\in \mathbb{R}\}$ omdat daar nie ’n waarde is van $x\in \mathbb{R}$ waarvoor $f(x)$ ongedefinieerd is nie.\par 
\par 
Indien $a>0$ dan het ons:
\begin{equation*}
\begin{array}{cccl}\hfill {x}^{2}& \geq & 0\hfill & (\mbox{die kwadraat van ’n uitdrukking is altyd positief})\hfill \\
 \hfill a{x}^{2}& \geq & 0\hfill & (\mbox{aangesien } a>0)\hfill \\
 \hfill a{x}^{2}+q& \geq & q\hfill & (\mbox{tel $q$ weerskante by.}) \\
 \hfill \therefore f(x)& \geq & q\hfill & 
\end{array}
\end{equation*}
Dus indien $a>0$, is die terrein gelyk aan $[q,\infty )$.\par 
Soortgelyk, kan ons aantoon dat indien  $a<0$ is die waardeversameling van $ (-\infty ,q]$. 

\begin{wex}{Gebeid en terrein van 'n parabool}
{As $g(x)={x}^{2}+2$, bepaal die gebied en terrein van die funksie.}
{
\westep{ Bepaal die gebied }
Die gebied is $\{x:x\in \mathbb{R}\}$ want daar is geen waarde van $x\in \mathbb{R}$ waarvoor $g(x)$ ongedefinieerd is nie.
\westep{Bepaal die terrein}

Die terrein van $g(x)$ kan as volg bereken word:

\begin{equation*}
\begin{array}{ccc}\hfill {x}^{2}& \geq & 0\hfill \\
 \hfill {x}^{2}+2& \geq & 2\hfill \\
 \hfill g(x)& \geq & 2\hfill 
\end{array}
\end{equation*}
Dus die waardeversameling is gelyk aan $\{g(x):g(x)\geq 2\}$.
}
\end{wex}



\subsection*{Afsnitte}
\textbf{Die $y$-afsnit}\\
Elke punt op die $y$-as het 'n $x$ -ko\"ordinaat van $0$. Dus bereken ons die $y$-afsnit deur $x = 0$ te stel.\\

Byvoorbeeld, die $y$-afsnit van $g(x)={x}^{2}+2$ word verkry deur $x=0$ te stel, en dan:\par 

\begin{equation*}
\begin{array}{ccl}\hfill g(x)& =& {x}^{2}+2\hfill \\ 
\hfill {y}& =& {0}^{2}+2\hfill \\
 & =& 2\hfill 
\end{array}
\end{equation*}
Dit gee die punt $(0;2)$.\par

\textbf{Die $x$-afsnit}\\
Elke punt op die $x$-as het 'n $y$--ko\"ordinaat van $0$, Dus bereken ons die $x$-afsnit deur $y=0$ te stel.\\

Byvoorbeeld, die $x$-afsnit van $g(x)={x}^{2}+2$ word verkry deur $y=0$ stel, en dan:\par

\begin{equation*}
\begin{array}{ccl}\hfill g(x)& =& {x}^{2}+2\hfill \\
 \hfill 0& =& x^{2}+2\hfill \\
 \hfill -2& =& x^{2}\hfill 
\end{array}
\end{equation*}
Hierdie antwoord is nie reël nie. Daarom het die grafiek van $g(x)={x}^{2}+2$ geen $x$-afsnitte nie. 

\subsection*{Draaipunte}

Die draaipunte van funksies van die vorm $f(x)=ax^{2}+q$ word gegee deur na die waardeversameling van
die funksie te kyk. 
\begin{itemize}
 \item Indien $a>0$, is die grafiek van $f(x)$ is 'n ``glimlag'' en het 'n minimum draaipunt by $(0;q)$.
\item Indien $a<0$, is die grafiek van $f(x)$ is 'n frons en het 'n maksimum
draaipunt by $(0;q)$.
\end{itemize}


\subsection*{Asse van Simmetrie}

Daar is een simmetrie-as vir die funksie met die vorm $f(x)=ax^{2}+q$ en dit gaan deur die draaipunt. Omdat die draaipunt op die $y$-as lê, is die simmertrie-as die lyn $x=0$. 

\subsection*{Skets grafieke van die vorm $f(x)=ax^{2}+q$}

Om ’n grafiek te skets van die vorm, $f(x)=a^{2}+q$ het ons vyf kenmenerkende eienskappe nodig:
\begin{enumerate}[noitemsep, label=\textbf{\arabic*}. ] 
\item die teken en grotte van $a$
\item $y$-afsnit
\item $x$-afsnit
\item draaipunte

\end{enumerate}

\begin{wex}
 {Sketse van parabole}
{Trek 'n grafiek van $g(x)=-\frac{1}{2}x^{2}-3$.  Merk die afsnitte, draaipunt en die simmetrie-as.}
{
\westep{Ondersoek die standaard vorm van die parabool}
Ons sien dat $a<0$.  Dit beteken dat die grafiek ’n maksimum draaipunt het. $-1 < a < 0$, dus die grafiek word 'uitwaarts' getrek
\westep{Bereken die afsnitte}
Die $y$-afsnit word bepaal deur $x=0$ te stel:
\begin{equation*}
\begin{array}{ccl}\hfill {y}& =& -\frac{1}{2}(0)^{2}-3\hfill \\
 h(0) =& \frac{1}{0} &= \mbox{ongedefinieërd}\vspace{5pt} \\ 
 & =& -\frac{1}{2}(0)-3\hfill \\
 & =& -3\hfill 
\end{array}
\end{equation*}
Die koördinate van die $y$ -afsnit is dan $(0; -3)$.\\

Die $x$-afsnit word bepaal deur $y=0$ te stel:
\begin{equation*}
\begin{array}{ccl}\hfill 0& =& -\frac{1}2x^{2}-3\hfill \\ 
\hfill 3& =& -\frac{1}2x^{2}\hfill \\
 \hfill -3(2)& =& x^{2}\hfill \\
\hfill -6& =& x^{2}\hfill \\
 h(0) &= \frac{1}{0} =& \mbox{ongedefinieërd}\vspace{5pt} \\ 
\end{array}
\end{equation*}
Die oplossing van die vergelyking is nie reëel nie. Daarom is daar geen $x$-afsnitte nie.
\westep{Bepaal die  die draaipunt}
Die draaipunt is gelyk aan die y-afsnit wat ons bereken het as $(0;-3)$.
\westep{Stip die punte en skets die grafiek}
% \begin{figure}[!ht]
\begin{center}
\begin{pspicture}(-5,-5)(5,1)
%\psgrid
\psset{yunit=0.75,xunit=0.75}
\psaxes[arrows=<->](0,0)(-5,-7)(5,1)
\psplot[plotstyle=curve,arrows=<->]{-2.5}{2.5}{x 2 exp 0.5 mul neg 3 sub}
\psdots(0,-3)
\uput[r](0,-2.7){$(0;-3)$}
\rput(0.3, 0.8){$y$}
\rput (5.2, 0.2){$x$}
\end{pspicture}
% \caption{Graph of the function $f(x)=-\frac{1}{2}x^2-3$}
% % \label{fig:mf:g:sketchexamplepar10}
\end{center}
% \end{figure}
\\
Definisievesameling is alle reële getalle en die waardeversameling $(- \infty; -3]$. 
Simmetrie-as $x=0$.
}

\end{wex}


\begin{wex}
{Sketse van Parabole}
{Skets die grafiek van $y={3x}^{2}+5$. Merk die afsnitte, draaipunt en die simmetrie-as.}
{
\westep{Examine the standard form of the equation}
Die teken van $a>0$. s positief. Die parabool sal dus ’n minimum-
draaipunt hê.
\westep{Bereken die afsnit punte}
Die $y$-afsnit word bepaal deur $x=0$te stel:
\begin{equation*}
\begin{array}{ccl}\hfill y& =& 3x^{2}+5\hfill \\
 \hfill y& =& 3(0)^{2}+5\hfill \\
 \hfill & =& 5\hfill 
\end{array}
\end{equation*}
Die koördinate van die $y$-as is dan $(0;5)$.
Die $x$-afsnit word bepaal deur $y=0$ te stel:
\begin{equation*}
\begin{array}{ccc}\hfill y& =& 3x^{2}+5\hfill \\
 \hfill 0& =& 3x^{2}+5\hfill \\
 \hfill x^{2}& =& -\frac{3}{5}\hfill 
\end{array}
\end{equation*}
Die oplossing van die vergelyking is nie reëel nie. Daarom is daar geen $x$ -afsnitte nie.
 $x$-intercepts.

\westep{bereken die draaipunte}
Die draaipunt is by (0, q). Vir hierdie funksie is q = 5, dus die
draaipunt is by $(0;5)$.

\westep{Plot the points and sketch the graph}
\begin{center}
\scalebox{1}
{
\psset{xunit=1.0cm,yunit=0.5cm,algebraic=true,dotstyle=o,dotsize=3pt 0,linewidth=0.8pt,arrowsize=3pt 2,arrowinset=0.25}
\begin{pspicture*}(-2,-2)(4,8)
\psaxes[xAxis=true,yAxis=true,Dx=1,Dy=2,ticksize=-2pt 0,subticks=2]{->}(0,0)(-2,0)(2,8)[,140] [,-40]
\rput{0}(0,5){\psplot{-2}{2}{x^2/2/0.17}}
% \rput[bl](0.56,5.34){$y = 3x^{2} + 5$}
\usefont{T1}{ptm}{m}{n}
\rput(0.3,7.6){$y$}
\usefont{T1}{ptm}{m}{n}
\rput(2.3,0.3){$x$}
\end{pspicture*}
}
\end{center}\\
Die definisievesameling is alle reële getalle $\{x:x \in \mathbb{R}\}$\\
die waardeversameling $\{y: y \geq 5, y \in \mathbb{R}\}$\\
Ons weet dat die y -as die simmetrie-as is $x=0$.
}

\end{wex}


   
\begin{exercises}{}
{
\begin{enumerate}[noitemsep, label=\textbf{\arabic*}. ] 
\item  Wys dat indien $a<0$ is die waardeversameling van  $f(x)=ax^{2}+q$ is $\{f(x):f(x) \leq q \}$.
\item Skets die grafiek van die funksie $y=-x^{2}+4$ en toon al die afsnitte met die asse.
\item Twee parabole is geteken: $g:y=ax^{2}+p$ en $h:y=bx^{2}+q$.
\setcounter{subfigure}{0}
\begin{center}
\scalebox{1} % Change this value to rescale the drawing.
{
\begin{pspicture}(0,-2.61)(9.072187,2.61)
\psline[linewidth=0.04cm,arrowsize=0.05291667cm 2.0,arrowlength=1.4,arrowinset=0.4]{->}(0.0,-0.57)(9.04,-0.57 )
\psline[linewidth=0.04cm,arrowsize=0.05291667cm 2.0,arrowlength=1.4,arrowinset=0.4]{->}(4.54,-2.59)(4.56,2.59)
\psbezier[linewidth=0.04](1.8801522,2.249799)(2.0476475,1.0067352)(2.6697726,-0.42756912)(2.9090517 ,-0.81005025)(3.1483305,-1.1925315)(3.6508162,-1.83)(4.584004,-1.83)
\psbezier[linewidth=0.04](7.24,2.249799)(7.0725045,1.0067352)(6.4503794,-0.42756912)(6.2111006,-0.81005025)(5.971822,-1.1925315)(5.469336,-1.83)(4.536148 ,-1.83)
\psbezier[linewidth=0.04](1.2402416,-2.409998)(1.4477341,-1.215622)(2.2184203,0.16250414)(2.5148382,0.53000444)(2.811256,0.8975047)(3.4337332,1.5100052)(4.5897627,1.5100052)
\psbezier[linewidth=0.04](7.88,-2.409998 )(7.672508,-1.215622)(6.901821,0.16250414)(6.6054034,0.53000444)(6.3089857,0.8975047)(5.686508,1.5100052)(4.530479,1.5100052)
\usefont{T1}{ptm}{m}{it}
\rput(4.8129687,2.48){$y$}
\usefont{T1}{ptm}{m}{it}
\rput(8.908281 ,-0.3){$x$}
\usefont{T1}{ptm}{m}{n}
\rput(4.2504687,1.73){$23$}
\usefont{T1}{ptm}{m}{n}
\rput(4.269219,-2.07){$ -9$}
\usefont{T1}{ptm}{m}{n}
\rput(1.730625,0.39){ $(-4;7)$} 
\usefont{T1}{ptm}{m}{n}
\rput(7.240625,0.39){$(4;7)$}
\usefont{T1}{ptm}{m}{n}
\rput(6.4275,-0.81){ $3$}
\usefont{T1}{ptm}{m}{it}
\rput(7.369219,1.87){$g$}
\usefont{T1}{ptm}{m}{it} 
\rput{-0.1}(0.0029207817,0.01384337){\rput(7.9021873,-1.69){$h$}}
\end{pspicture} 
}       
\end{center}

    \begin{enumerate}[noitemsep, label=\textbf{\alph*}. ] 
    \item Vind die waardes van $a$ en $p$.
    \item Vind die waardes van $b$ en $q$.
    \item Vind die waardes van $x$ for which $g{x}\geq h{x}$.
    \item For what values of $x$ is $g$ toenemend?
    \end{enumerate}
\end{enumerate}

\insertpracticeinfo{3}
}
\end{exercises}   

\section{Hiperboliese funksies van die vorm $y=\frac{a}{x}+q$}

\mindsetvid{The hyperbolic function}{VMaxw}

\subsection*{Skets die gafiek}  
Funksies van die vorm $y=\frac{a}{x}+q$ staan bekend as hiperboliese funksies. 

\begin{wex}
{Trek die grafiek van 'n hiperbool}
{
\begin{equation*}
 y = h(x) = \frac{1}{x}
\end{equation*}

Voltoi die volgende tabel vir $h(x) = \frac{1}{x}$ en stip die punte op 'n assesstelsel.

\begin{table}[H]
\begin{center}
\begin{tabular}{|c|c|c|c|c|c|c|c|c|c|c|c|}
\hline
  $x$ &  $-3$ & $-2$ & $-1$ & $-\frac{1}{2}$ & $-\frac{1}{4}$ &$0$&$\frac{1}{4}$&$\frac{1}{2}$&$1$&$2$&$3$
\\ \hline
 $h(x)$& $-\frac{1}{3}$ &&&&&&&&&&
\\ \hline
\end{tabular}
\end{center}
\end{table}


\begin{enumerate}[noitemsep, label=\textbf{\arabic*}. ] 
 \item Verbind die punte met 'n gladde kromme.
\item Wat gebeur as $x=0$?
\item Verduidelik waarom die grafiek uit twee aparte krommes bestaan $h(x)=\frac{1}{x}$.
\item Wat gebuer met $h(x)$ as die waarde van $x$ baie klein of baie groot word?
\item Dit gebied van $h(x)$ is $x \in \mathbb{R} - \{0\}$. Bepaal die terrein.
\item Rondom watter twee lyne is die grafiek $h$ simmetries?
\end{enumerate}
}
{
\westep{Stel waardes in die vergelyking in}{
\begin{equation*}
 \begin{array}{cllll}
  h(x) &=& \frac{1}{x} & &\\
 h(-3) &=& \frac{1}{-3} &=& -\frac{1}{3} \vspace{5pt}\\ 
 h(-2) &=& \frac{1}{-2} &=& -\frac{1}{2}\vspace{5pt} \\ 
 h(-1) &=& \frac{1}{-1} &=& -1 \vspace{5pt}\\ 
 h(-\frac{1}{2}) &=& \dfrac{1}{-\frac{1}{2}} &=& -2 \vspace{5pt}\\ 
 h(-\frac{1}{4}) &=& \dfrac{1}{-\frac{1}{4}} &=& -4 \vspace{5pt}\\ 
 h(0) &=& \frac{1}{0} &=& \mbox{ongedefineer}\vspace{5pt} \\ 
 h(\frac{1}{4}) &=& \dfrac{1}{\frac{1}{4}} &=& 4 \vspace{5pt}\\ 
 h(\frac{1}{2}) &=& \dfrac{1}{\frac{1}{2}} &=& 2 \vspace{5pt}\\ 
 h(1) &=& \frac{1}{1} &=& 1 \vspace{5pt}\\ 
 h(2) &=& \frac{1}{2} &=& \frac{1}{2} \vspace{5pt}\\ 
 h(3) &=& \frac{1}{3} &=& \frac{1}{3} \vspace{5pt}\\ 
 \end{array}
\end{equation*}

\begin{table}[H]
\begin{center}
\begin{tabular}{|c|c|c|c|c|c|c|c|c|c|c|c|}
\hline
  $x$ &  $-3$ & $-2$ & $-1$ & $-\frac{1}{2}$ & $-\frac{1}{4}$ &$0$&$\frac{1}{4}$&$\frac{1}{2}$&$1$&$2$&$3$
\\ \hline
 $h(x)$& $-\frac{1}{3}$ &$-\frac{1}{2}$&$-1$&$-2$&$-4$&ongedefineer&$4$&$2$&$1$&$\frac{1}{2}$&$\frac{1}{3}$
\\ \hline
\end{tabular}
\end{center}
\end{table}
}
\westep{Stip die punte en verbind hulle apart met gladde krommes}
Vanaf die tabel kry ons die volgende punte: $(-3; -\frac{1}{3})$, $(-2; -\frac{1}{2})$, $(-1;-1)$, $(-\frac{1}{2}; -2)$, $(-\frac{1}{4}; -4)$, $(\frac{1}{4}; -4)$, $(\frac{1}{2}; 2)$, $(1; 1)$, $(2; \frac{1}{2})$ en $(3; \frac{1}{3})$. \vspace{8pt} \\


% \setcounter{subfigure}{0}
% % \begin{figure}[tbp]
% \begin{center}
% \begin{pspicture}(-5,-2)(5,5)
% %\psgrid
% \psset{yunit=0.75,xunit=0.75}
% \psaxes[arrows=<->](0,0)(-5,-2.667)(5,6)
% \psplot[plotstyle=curve,arrows=<->]{-5}{-0.25}{x -1 exp }
% \psplot[plotstyle=curve,arrows=<->]{0.25}{5}{x -1 exp }
% % \psplot[linestyle=dotted,plotstyle=curve]{-5}{2}{x 2 add}
% \end{pspicture}
% % \caption{General shape and position of the graph of a function of the form $f(x)=\frac{a}{x} + q$.}
% \label{fig:mf:g:hyperbola10}
% \end{center}
% % \end{figure}      


Funksie $h$ is ongedefineered vir $x=0$. Daar is dus 'n breuk(diskontinuiteit) by $x=0$. \vspace{8pt} \\
$y=h(x) = \frac{1}{x}$ dus kan ons skryf $x \times y = 1$. Aangesien die produk van twee positiewe getalle \textbf{asook} die produk van twee negatiewe getalle gelyk aan $1$ kan wees, l\^e die grafiek in die eerste en derde kwadrante.

\westep{Bepaal die asimptote}
As die waarde van $x$ toeneem, kom die waarde van $h(x)$ al nader aan $0$ maar word nooit $0$ nie. Dus noem ons die $x$-as, die lyn $y=0$, 'n horisontalle asimptoot van die grafiek. Dieselfde gebeur in die derde kwadrant; as $x$ kleiner, sal $h(x)$ die $x$-as asimptoties nader.\vspace{8pt} \\

Let ook op dat daar 'n vertikale asimptoot is: die $y$-as met vergelyking $x=0$; soos $x$ nader kom aan $0$, nader $h(x)$ die $y$-as asimptoties.
\westep{Bepaal die waardeversameling}
Definisieversameling: $x \in \mathbb{R} - \{0\}$\\
Van die grafiek sien ons $y$ is gedefinieer vir alle waardes behalwe $0$.\\
waardeversameling: $x \in \mathbb{R} - \{0\}$ 
\westep{Bepaal die asse van simmetrie}
Die grafiek van $h(x)$ het twee asse van simmertrie die lyne $y=x$ en $y=-x$. Die twee halftes van die hiperbool is spie\"elbeelde van mekaar mat betrekking tot hierdie lyne. 
}
\end{wex}




\subsection*{Ondersoek die invloed van $a$ en $q$ }
\begin{Ondersoek}{Ondersoek die invloed van $a$ en $q$ op 'n hiperbool}
 Op dieselfde assestelsel, trek die volgende grafieke:
    \begin{enumerate}[itemsep=3pt, label=\textbf{\arabic*}. ] 
    \item $a(x)=\dfrac{1}{x}-2$
    \item $b(x)=\dfrac{1}{x}-1$
    \item $c(x)=\dfrac{1}{x}$
    \item $d(x)=\dfrac{1}{x}+1$
    \item $e(x)=\dfrac{1}{x}+2$
\end{enumerate}
Gebruik jou resultate om die invloed van  $q$ af te lei.\par

Op dieselfde assestelsel, trek die volgende grafieke::
    \begin{enumerate}[itemsep=3pt, label=\textbf{\arabic*}. ] 
\setcounter{enumi}{5}
    \item $f(x)=\dfrac{-2}{x}$
    \item $g(x)=\dfrac{-1}{x}$
    \item $h(x)=\dfrac{1}{x}$
    \item $i(x)=\dfrac{2}{x}$
    \end{enumerate}
Gebruik jou resultate om die invloed van $a$ af te lei.
\end{Investigation}

\begin{table}[H]
\begin{center}
% \caption{Table summarising general shapes and positions of functions of the form $y=\frac{a}{x} + q$. The axes of symmetry are shown as dashed lines.}
\label{tab:mf:graphs:summaryhyp10}
\begin{tabular}{|c|c|c|}\hline
& $a<0$&$a>0$\\\hline
$q>0$&
\begin{pspicture}(-1.2,-1.2)(1.2,1.2)
%\psgrid
\psset{xunit=0.2,yunit=0.2}
\psaxes[arrows=<->,dx=0,Dx=10,dy=0,Dy=10](0,0)(-5,-5)(5,5)
\psplot[plotstyle=curve,arrows=<->]{-5}{-0.25}{x -1 exp neg 2 add}
\psplot[plotstyle=curve,arrows=<->]{0.25}{5}{x -1 exp neg 2 add}
\psplot[linestyle=dotted,plotstyle=curve]{-4}{4}{x neg 2 add}
\end{pspicture}
&

\begin{pspicture}(-1.2,-1.2)(1.2,1.2)
%\psgrid
\psset{xunit=0.2,yunit=0.2}
\psaxes[arrows=<->,dx=0,Dx=10,dy=0,Dy=10](0,0)(-5,-5)(5,5)
\psplot[plotstyle=curve,arrows=<->]{-5}{-0.25}{x -1 exp 2 add}
\psplot[plotstyle=curve,arrows=<->]{0.25}{5}{x -1 exp 2 add}
\psplot[linestyle=dotted,plotstyle=curve]{-4}{4}{x 2 add}
\end{pspicture}
\\\hline
$q=0$ & &
\\ \hline
$q<0$&
\begin{pspicture}(-1.2,-1.2)(1.2,1.2)
%\psgrid
\psset{xunit=0.2,yunit=0.2}
\psaxes[arrows=<->,dx=0,Dx=10,dy=0,Dy=10](0,0)(-5,-5)(5,5)
\psplot[plotstyle=curve,arrows=<->]{-5}{-0.25}{x -1 exp neg 2 sub}
\psplot[plotstyle=curve,arrows=<->]{0.25}{5}{x -1 exp neg 2 sub}
\psplot[linestyle=dotted,plotstyle=curve]{-2}{4}{x neg 2 sub}
\end{pspicture}
&

\begin{pspicture}(-1.2,-1.2)(1.2,1.2)
%\psgrid
\psset{xunit=0.2,yunit=0.2}
\psaxes[arrows=<->,dx=0,Dx=10,dy=0,Dy=10](0,0)(-5,-5)(5,5)
\psplot[plotstyle=curve,arrows=<->]{-5}{-0.25}{x -1 exp 2 sub}
\psplot[plotstyle=curve,arrows=<->]{0.25}{5}{x -1 exp 2 sub}
\psplot[linestyle=dotted,plotstyle=curve]{-4}{4}{x 2 sub}
\end{pspicture}
\\\hline
\end{tabular}
\end{center}
\end{table}

\textbf{Die invloed van $q$}\newline

Jy behoort te vind dat die waarde van $q$ bepaal of die grafiek bo of onder die $x$-as is dit word 'n vertikale skuif genoem omdat al die punte dieselfde afstand in dieslefde rigting skuif. Die hele grafiek skyf op of af.  
\begin{itemize}
\item Vir $q>0$, skuif die grafiek $h(x)$ eenhede vertikale op $q$. 
\item Vir $q<0$, skuif die grafiek $h(x)$ eenhede vertikale af $q$.
\end{itemize}
Die horisontale asimptoot is die lyn $y=q$ en die vertikale asimptoot is die $y$-axis, die lyn $x=0$.\par
\vspace{8pt}
\textbf{Die invloed van $a$}\newline
Jy behoort te vind dat die waarde van $a$ bepaal die vorm van die grafiek  en as die grafiek in die eeste en derde kwardrante of in die tweede
en vierde kwadrante van die Cartesiese vlak lê. 
\begin{itemize}
 \item Indien $a>0$, sal die grafiek $h(x)$ in die eeste en derde kwardrante lê. \\
Vir $a>1$, sal die grafiek $h(x)$ verder weg l\^e van die asse as $y=\frac{1}{x}$.
\\Vir $0<a<1$, is $a$ 'n breuk en die grafiek sal nader aan die asse l\^e as $y=\frac{1}{x}$. 
\item Indien $a<0$, sal die grafiek $h(x)$ in die tweede
en vierde kwadrante van die Cartesiese vlak lê.\\
Vir $a<-1$, sal die grafiek $h(x)$ verder weg l\^e van die asse as $y=-\frac{1}{x}$.
\\Vir $-1<a<1$, $a$ 'n breuk en die grafiek sal nader aan die asse l\^e as $y=-\frac{1}{x}$. 
\end{itemize}



\subsection*{Ondek die kenmerke van $y=\frac{a}{x}+q$}  
Die standaardsvorm van die funksie is $y=\frac{a}{x}+q$.

\subsection*{Definisieversameling en waardeversameling}

Die funksie $y=\frac{a}{x}+q$, is ongedefinieerd vir $x=0$. \\
Die definisieversameling is dus $\{x:x\in \mathbb{R},x\ne 0\}$.\par 
Ons kan sien dat $y=\frac{a}{x}+q$ herskryf kan word as:
\begin{equation*}
\begin{array}{ccl}\hfill y& =& \dfrac{a}{x}+q\hfill \vspace{4pt} \\
 \hfill y-q& =& \dfrac{a}{x}\hfill \\
 \hfill \mbox{Indien }x \ne  0 \mbox{dan}:(y-q)x& =& a\hfill \\
 \hfill x& =& \dfrac{a}{y-q}\hfill 
\end{array}
\end{equation*}
Dit wys dat die funksie ongedefinieerd is by $y=q$. \\
Die waardeversameling van $f(x)=\frac{a}{x}+q$ is $\{f(x):f(x)\in (-\infty ;q)\cup (q;\infty )\}$.\par 

\begin{wex}{Definisieversameling en waardeversameling van 'n hiperbool}
{If $g(x)=\frac{2}{x}+2$, determine the domain and range of the function.}
{
\westep{Bereken die definisieversameling}
Die definisieversameling is $\{x:x\in \mathbb{R},x\ne 0\}$ omdat $g(x)$ ongedefinieerd is by $x=0$.
\westep{Bereken die waardeversameling}
Ons sien dat  $g(x)$ ongedefinieerd is by $y=2$.  Die waardeversamling is dus $\{g(x):g(x)\in (-\infty ;2)\cup (2;\infty )\}$.
}
\end{wex}


\subsection*{Afsnitte}

\textbf{Die $y$-afsnit} \\
Elke punt op die $y$-as het $x$-Ko\"ordinaat van $0$, Dus bereken die $y$-afsnit, deur $x=0$ te stel.\\
As $g(x)=\frac{2}{x}+2$ stel $x=0$:
\begin{equation*}
\begin{array}{ccc}\hfill y& =& \dfrac{2}{x}+2\hfill \vspace{4pt}\\
 \hfill y& =& \dfrac{2}{0}+2\hfill 
\end{array}
\end{equation*}
Dit is ongedefiniëerd omdat ons deur nul deel. Daar is dus geen $y$-afsnit nie.\\
\\

\textbf{Die $x$-afsnit} \\
Elke punt op die $x$-as het $y$-Ko\"ordinaat van $0$, Dus bereken die $x$-afsnit, deur $y=0$ te stel.\\
As $g(x)=\frac{2}{x}+2$ stel $y=0$:
\begin{equation*}
\begin{array}{ccl}
\hfill y& =& \dfrac{a}{x}+q\hfill \vspace{4pt}\\
 \hfill 0& =& \dfrac{a}{x}+q\hfill \vspace{4pt} \\
 \hfill \dfrac{a}{x}& =& -q\hfill \\
 \hfill a& =& -q(x)\hfill \vspace{4pt}\\
 \hfill x& =& \dfrac{a}{-q}\hfill 
\end{array}
\end{equation*}
This gives us the point $(-1; 0)$.


\subsection*{Asimptote}

Daar is twee asimptote vir die funksies van die vorm $y=\frac{a}{x}+q$. \par 
Ons het gesien dat die funksie ongedefenieer was by $x=0$ en $y=q$ Dus is die asimtote $y=q$, en $x=0$ die $y$ -as. 

\subsection*{Asse van simmetrie}
Daar is twee lyne ten opsigte waarvan die hiperbool simmertrie is $y=ax+q$ en $y = -x +q$.As $y = \frac{2}{x} + 2$ is die simmertrie-asse $y = x + 2$ en $y = -x + 2$


\subsection*{Skets grafieke van die vorm  $f(x)=\frac{a}{x}+q$}

Om grafieke van funksies van die vorm, $f(x)=\frac{a}{x}+q$, te skets, het ons vier eienskappe nodig.
\begin{enumerate}[noitemsep, label=\textbf{\arabic*}. ] 
\item teken van $a$
\item $y$-afsnitte
\item $x$-afsnitte
\item asimptote
\end{enumerate}

\begin{wex}{Skets ’n hiperbool}
{Skets die grafiek $g(x)=\frac{2}{x}+2$. Merk die afsnitte en asimptote.}
{
\westep{Ondersoek die standaardvorm van die hiperbool}
Ons sien dat $a>0$. Daarom lê die grafiek $g(x)$ in die eeste en derde kwardrante. 
\westep{Bereken die afsnitte}
$y$-afsnit, stel $x=0$:
\begin{equation*}
\begin{array}{ccl}
  g(x) = & \dfrac{2}{x} + 2  \vspace{4pt} \\
  g(0) = & \dfrac{2}{0} +2  
 \end{array}
\end{equation*}
Dit is ongedefinieerd, Daar is geen $y$-afsnit nie. 
\\
$x$-afsnit, stel $y=0$:
\begin{equation*}
 \begin{array}{ccl}
  g(x) = &  \dfrac{2}{x} + 2 \vspace{4pt}\\
 0 = & \dfrac{2}{x} +2 \\
\dfrac{2}{x} = & -2 \\
\therefore x = &-1
 \end{array}
\end{equation*}

Dit gee die punt $(-1;0)$


\westep{Bepaal die asimptote}
Die horisontale asimptoot is die lyn $y=2$ en die vertikale asimptoot is die lyn $x=0$.

\westep{Skets die Grafieke}
\setcounter{subfigure}{0}
\begin{figure}[H]
\begin{center}
\begin{pspicture}(-5,-3)(5,6)
%\psgrid
\psset{yunit=0.75,xunit=0.75}
\psaxes[arrows=<->](0,0)(-5,-4)(5,7)
\psplot[plotstyle=curve,arrows=<->]{-5}{-0.4}{x -1 exp 2 mul 2 add}
\psplot[plotstyle=curve,arrows=<->]{0.4}{5}{x -1 exp 2 mul 2 add}
\psline[linestyle=dashed](-5,2)(5,2)
\end{pspicture}
% \caption{Graph of $g(x)=\frac{2}{x} + 2$.}
% \label{fig:mf:g:hyperbolasketchexample}
\end{center}
\end{figure} 
Gebied: $x \in \mathbb{R} - \{0\}$\\
Terrein: $y \in \mathbb{R} - \{2\}$
}
\end{wex}




\begin{wex}
{Skets ’n hiperbool}
{
Skets die grafiek van $y=\frac{-4}{x}+7$.}
{

\westep{Ondersoek die standaardvorm van die hiperbool}
Ons sien dat $a<0$, dus die grafiek l\^e in kwadrant $2$ en $4$.
\westep{Bereken die afsnitte}
Die $y$-afsnit is waar $x=0$:
\begin{equation*}
 \begin{array}{ccc}
 \hfill  y &= & \dfrac{-4}{x}+7 \vspace{4pt}\hfill \\
 \hfill &= & \dfrac {-4}{0} +7  \hfill \\

 \end{array}
\end{equation*}
Die funksie is ongedefinieerd by $y$, $x=0$. Daar is geen $y$-afsnit vir grafieke van hierdie vorm nie. \\
Die $x$-afsnit is waar $y=0$:
\begin{equation*}
 \begin{array}{ccl}
 y &=&  \dfrac{-4}{x}+7\vspace{4pt}\\
 0 &=&  \dfrac{-4}{x}+7\vspace{4pt}\\ 
 \dfrac{-4}{x} &=& -7\vspace{4pt} \\
\therefore x &= &\dfrac{4}{7}
 \end{array}
\end{equation*}

Daar is dus een x-afsnit by $\left(\dfrac{4}{7};0\right)$


\westep{Bepaal die asimptote}
Ons kyk na die gebied en die terrein om te bepaal waar die asimp-
tote lê. Van die gebied kan ons sien dat die funksie ongedefiniëerd
is wanneer $x=0$, Dus daar is een asimptoot by $x=0$. Die funksie is
ongedefiniëerd by y = q. Dus die tweede asimptoot is by $y=7$. 

\westep{Skets die grafiek}
\setcounter{subfigure}{0}
\begin{figure}[H]
\psset{xunit=0.5cm,yunit=0.5cm,algebraic=true,dotstyle=o,dotsize=3pt 0,linewidth=0.8pt,arrowsize=3pt 2,arrowinset=0.25}
\begin{pspicture*}(-10.85,-7.73)(10.36,13.74)
\psaxes[Axis=true,yAxis=true,Dx=2,Dy=2,ticksize=-2pt 0,subticks=2]{->}(0,0)(-10.85,-7.73)(10.36,13.74)[x,140] [y,-40]
\psplot[plotpoints=200]{-10.849663672784777}{10.361464563594}{-4/x+7}
\psplot[linestyle=dashed,dash=5pt 5pt]{-10.85}{10.36}{(--7-0*x)/1}
\rput[bl](-8.13,7.85){$y=-\frac{4}{x} + 7$}
\end{pspicture*}   
\end{figure}

Gebied: $x \in \mathbb{R} - \{0\}$\\
Terrein: $y \in \mathbb{R} - \{7\}$\\
Asse van simmetrie: $y=x+7$ and $y=-x+7$


}
\end{wex}

\begin{exercises}{}
{
Gebruik grafiekpapier en teken die grafiek van  $xy=-6$.
    \begin{enumerate}[noitemsep, label=\textbf{\arabic*}. ] 
    \item Lê die punt $(-2; 3)$  op die grafiek? Gee ’n rede vir jou antwoord.

    \item As die $x$-waarde van ‘n punt op die grafiek $0,25$ wat is die ooreenstemmende $y$-waarde?
    \item Wat gebeur met die $y$-waardes as die $x$-waardes baie groot word?
\item Bereken die kortste afstand van die oorsprong na die grafiek $P$.
\item Gee die vergelykings van die asimptote.
    \item Met die lyn $y=-x$ as ’n lyn van simmetrie, watter punt is simmetries ten opsigte van $(-2; 3)$ ?
    \end{enumerate}
Skets die grafiek van  $h(x)=\frac{8}{x}$.
    \begin{enumerate}[noitemsep, label=\textbf{\arabic*}. ] 
\setcounter{enumi}{7}
    \item Hoe sal die grafiek $g(x)=\frac{8}{x}+3$ vergelyk met die grafiek van $h(x)=\frac{8}{x}$? Verduidelik jou antwoord.
    \item Skets die grafiek van $y=\frac{8}{x}+3$ op dieselfde assestelsel. Toon die asimptote asse van simmetrie en die ko\"ordinate van eenpunt op die grafiek.
    \end{enumerate}


\insertpracticeinfo{9}
}
\end{exercises}

\section{Eksponensiële funksies van die vorm $y=ab^{x}+q$}

\subsection*{Trek die grafiek}         
Funksies van die vorm $y=ab^{x}+q$ staan bekend as eksponensiële funksies. Konstantes $a$ en $q$ be\"invloed die vergelyking op verskillende maniere.

\mindsetvid{The exponential function}{VMaxx}

\begin{wex}{Trek die grafiek van 'n eksponensiële funksies}
 {
\begin{equation*} y=f(x) =b^{x} \mbox{ vir } b>0 \mbox{ en } b \neq 1 \end{equation*}

Voltooi die tabel vir elk van die funksies en trek die grafieke op dieselfde assessisteem:
$f(x)=2^{x}$, $g(x)=3^{x}$, $h(x)=5^{x}$.


\begin{table}[H]
\begin{center}
\begin{tabular}{|c|c|c|c|c|c|}
\hline
   &  $-2$ & $-1$ & $0$ & $1$ & $2$ 
\\ \hline
 $y=2^{x}$&  &&&&
\\ \hline
 $y=3^{x}$&  &&&&
\\ \hline
 $y=5^{x}$&  &&&&
\\ \hline

\end{tabular}
\end{center}
\end{table}

\begin{enumerate}[noitemsep, label=\textbf{\arabic*}. ] 
 \item By watter punt sny al die grafieke?
\item Verduidelik waarom hulle nie die $x$-as sny nie.
\item Gee die gebied en terrein van $h(x)$.
 \item Neem die waarde van $x$ af of toe soos $h(x)$ toeneem?
\item Watter een van hierdie grafieke neem toe teen die stadigste tempo?
\item Is die vlgende bewering ten opsigte van $y=k^{x}$ ; $k>1$, waar of vals: Hoe groter $k$ hoe steiler die grafiek van $y=k^{x}$?
\end{enumerate}

Voltooi die volgende tabel vir elk van die funksies en trek die grafieke op dieselfde asse stelsel:
$F(x) =(\frac{1}{2})^{x}$, $G(x) =(3)^{-x}$, $H(x) =(\frac{1}{5})^{x}$, 
\begin{table}[H]
\begin{center}
\begin{tabular}{|c|c|c|c|c|c|}
\hline
   &  $-2$ & $-1$ & $0$ & $1$ & $2$ 
\\ \hline
 $y=(\frac{1}{2})^{x}$&  &&&&
\\ \hline
$y=(\frac{1}{3})^{x}$&  &&&&
\\ \hline
$y=(\frac{1}{5})^{x}$&  &&&&
\\ \hline

\end{tabular}
\end{center}
\end{table}

\begin{enumerate}[noitemsep, label=\textbf{\arabic*}. ] 
 \item Give the $y$-intercepts for each function.
\item Bespreek die verband tussen die grafieke van $f(x)$ en $F(x)$.
\item Bespreek die verband tussen die grafieke $g(x)$ en $G(x)$.
\item Gee die gebied en terrein van $H(x)$.
\item Is die bewering waar of vals ten opsigte van $y=(\frac{1}{k})^{x}$ en $k>1$, ``hoe groter die waarde van $k$, hoe steiler is die grafiek''
\item Gee die vergelykings van die asimptote van elke funksie.
\end{enumerate}

}
{
\westep{Vervang waardes in die vergelykings}
\begin{table}[H]
\begin{center}
\begin{tabular}{|c|c|c|c|c|c|}
\hline
   &  $-2$ & $-1$ & $0$ & $1$ & $2$ 
\\ \hline
 $f(x)=2^{x}$& $\frac{1}{4}$ &$\frac{1}{2}$&$1$&$2$&$4$
\\ \hline
 $g(x)=3^{x}$& $\frac{1}{9}$ &$\frac{1}{3}$&$1$&$3$&$9$
\\ \hline
 $g(x)=5^{x}$& $\frac{1}{25}$ &$\frac{1}{5}$&$1$&$5$&$25$
\\ \hline

\end{tabular}
\end{center}
\end{table}

\begin{table}[H]
\begin{center}
\begin{tabular}{|c|c|c|c|c|c|}
\hline
   &  $-2$ & $-1$ & $0$ & $1$ & $2$ 
\\ \hline
 $F(x)=(\frac{1}{2})^{x}$& $4$ &$2$&$1$&$\frac{1}{2}$&$\frac{1}{4}$
\\ \hline
$G(x)=(\frac{1}{3})^{x}$&  $9$&$3$&$1$&$\frac{1}{3}$&$\frac{1}{9}$
\\ \hline
$H(x)=(\frac{1}{5})^{x}$& $25$& $5$&$1$&$\frac{1}{5}$&$\frac{1}{25}$
\\ \hline

\end{tabular}
\end{center}
\end{table}
\westep{Stip die punte en verbind hulle met 'n gladdekromme}
% Diagram needed here
\begin{enumerate}[noitemsep, label=\textbf{\arabic*}. ] 
\item Ons sien al die grafieke gaan deur die punt $(0;1)$. Enige getal tot die mag $0$ is gelyk aan $1$.
\item Die grafieke sny nie die $x$-as nit omdat $0^{0}$ ongedefenieerd is.
\item Gebeid: $x \in \mathbb{R}$\\
Terrein: $(0; \infty)$
\item As $x$ toeneem, neem $h(x)$ toe.
\item $f(x)=2^{x}$ neem teen die stadigste tempo toe omdat dit die kleinste grondtal het.
\item Waar: hoe groter $k ~(k>1)$, hoe steiler die grafiek $y=k^{x}$.
\end{enumerate}
% Another diagram needed here
\begin{enumerate}[noitemsep, label=\textbf{\arabic*}. ] 
\item Die $y$-afsnit is die punt $(0; 1)$ vir al grafieke. Vir enige re\"ele getal $z$, $z^{0}=1$.
\item $F(x)$ is die spie\"elbeeld(refleksie) van $f(x)$ in die $y$-as. 
\item $G(x)$ is die spie\"elbeeld(refleksie) van $g(x)$ in die $y$-as. 
\item  Gebied: $x \in \mathbb{R}$\\
Terrein: $(0; \infty)$
\item Waar: hoe groter $k ~(k>1)$, hoe steiler die grafiek van $y=(\frac{1}{k})^{x}$.
\item Die vergelyking van die horisontale asimptoot is $y=0$, die $x$-as.
\end{enumerate}

}
\end{wex}

% \par 
% \setcounter{subfigure}{0}
% \begin{figure}[H]
% \begin{center}
% \begin{pspicture}(-5,-1)(5,4)
% %\psgrid
% \psset{yunit=0.75,xunit=0.75}
% \psaxes[arrows=<->](0,0)(-5,-1)(5,5)
% \psplot[plotstyle=curve,arrows=<->]{-5}{1.2}{2 x exp 2 add}
% \end{pspicture}
% % \caption{General shape and position of the graph of a function of the form $f(x)=ab^{x} + q$.}
% % \label{fig:mf:g:exponential10}
% \end{center}
% \end{figure}     

\subsection*{Ondersoek die invloed van $a$ en $q$}
\begin{Ondersoek}{Ondersoek die invloed van $a$ en $q$ op die exponent grafiek}
Op dieselfde assestelsel, skets die volgende grafieke ($k=2$, $a=1$ en $q$ verander):
\begin{enumerate}[noitemsep, label=\textbf{\alph*}. ] 
\item $a(x)=2^{x}-2$
\item $b(x)=2^{x}-1$
\item $c(x)=2^{x}$
\item $d(x)=2^{x}+1$
\item $e(x)=2^{x}+2$
\end{enumerate}

\begin{table}[H]
\begin{center}
\begin{tabular}{|l|c|c|c|c|c|}
\hline
   &  $-2$ & $-1$ & $0$ & $1$ & $2$ 
\\ \hline
$a(x)=2^{x}-2$&  &&&&
\\ \hline
 $b(x)=2^{x}-1$&  &&&&
\\ \hline
$c(x)=2^{x}$&  &&&&
\\ \hline
$d(x)=2^{x}+1$&  &&&&
\\ \hline
$e(x)=2^{x}+2$&  &&&&
\\ \hline
\end{tabular}
\end{center}
\end{table}
Gebruik jou antwoorde om ’n gevolgtrekking te maak ten opsigte van die invloed van $q$.
\\

Op dieselfde assestelsel, skets die volgende grafieke ($k=2$, $q=0$ en $a$ verander):
\begin{enumerate}[noitemsep, label=\textbf{\arabic*}. ] 
\item $f(x)=2^{x}$
\item $g(x)=2.2^{x}$
\item $h(x)=-2^{x}$
\item $i(x)=-2.2^{x}$
\end{enumerate}

\begin{table}[H]
\begin{center}
\begin{tabular}{|l|c|c|c|c|c|}
\hline
   &  $-2$ & $-1$ & $0$ & $1$ & $2$ 
\\ \hline
$f(x)=2^{x}$&  &&&&
\\ \hline
$g(x)=2.2^{x}$&  &&&&
\\ \hline
 $h(x)=-2^{x}$&  &&&&
\\ \hline
$i(x)=-2.2^{x}$&  &&&&
\\ \hline

\end{tabular}
\end{center}
\end{table}
Gebruik jou antwoorde om ’n gevolgtrekking te maak ten opsigte van die invloed van $a$.
\end{Investigation}


\begin{table}[H]
\begin{center}
% \caption{Table summarising general shapes and positions of functions of the form $y=ab^{x} + q$.}
% \label{tab:mf:graphs:summaryexp10}
\begin{tabular}{|c|c|c|}\hline
& $a<0$&$a>0$\\\hline
$q>0$&


\begin{pspicture}(-1.2,-1.2)(1.2,1.2)
%\psgrid
\psset{xunit=0.2,yunit=0.2}
\psaxes[arrows=<->,dx=0,Dx=10,dy=0,Dy=10](0,0)(-5,-5)(5,5)
\psplot[plotstyle=curve,arrows=<->]{-5}{2}{2 x exp -1 mul 2 add}
\end{pspicture}
&
\begin{pspicture}(-1.2,-1.2)(1.2,1.2)
\psset{xunit=0.2,yunit=0.2}
\psaxes[arrows=<->,dx=0,Dx=10,dy=0,Dy=10](0,0)(-5,-5)(5,5)
\psplot[plotstyle=curve,arrows=<->]{-5}{2}{2 x exp 2 add}
\end{pspicture}
\\\hline
$q<0$&


\begin{pspicture}(-1.2,-1.2)(1.2,1.2)
%\psgrid
\psset{xunit=0.2,yunit=0.2}
\psaxes[arrows=<->,dx=0,Dx=10,dy=0,Dy=10](0,0)(-5,-5)(5,5)
\psplot[plotstyle=curve,arrows=<->]{-5}{2}{2 x exp -1 mul 2 sub}
\end{pspicture}
&
\begin{pspicture}(-1.2,-1.2)(1.2,1.2)
%\psgrid
\psset{xunit=0.2,yunit=0.2}
\psaxes[arrows=<->,dx=0,Dx=10,dy=0,Dy=10](0,0)(-5,-5)(5,5)
\psplot[plotstyle=curve,arrows=<->]{-5}{2}{2 x exp 2 sub}
\end{pspicture}
\\\hline
\end{tabular}cos
\end{center}
\end{table}

\textbf{Die invloed van $q$}\newline

Jy sou gevind het dat die waarde van $q$ 'n vertikale skuif veroorsaak omdat alle punte dieselfde afstand in dieselfde rigting beweeg, die hele grafiek skuif op of af. 
\begin{itemize}
\item As $q>0$, skuif die grafiek $q$ eenhede vertikaal op. 
\item As $q<0$, skuif die grafiek $q$ eenhede vertikaal op.
\end{itemize}
Die horisontale asimptote skuif $q$ eenhede en is die lyn $y=q$. \vspace{8pt}\\


\textbf{Die invloed van $a$}\newline
Jy sou gevind het dat die waarde van $a$ bepaal die vorm van die grafiek. c
\begin{itemize}
 \item Indien $a>0$, dan sal die grafiek opwaarts buig.
\item If $a<0$, dan sal die grafiek afwaarts buig . Die twee grafieke is spie\"elbeelde van mekaar in die horisontale asimptoot
\end{itemize}

\subsection*{ondek die kenmerke van $y=ab^{x} + q$}
Die standaardvorm van die eksponentgrafiek is $y=ab^{x} + q$.
\subsection*{Definisieversameling en waardeversameling}

Die funksie $y=ab^{x}+q$,is gedefinieer vir alle reële waardes van $x$. Dus, die definisieversameling is $\{x:x\in \mathbb{R}\}$.\par 
Die waardeversameling van $y=ab^{x}+q$ word bepaal deur die teken van $a$.\par 
Indien $a>0$ dan:\par
\begin{equation*}
\begin{array}{ccc}\hfill b^{x}& > & 0\hfill \\
 \hfill ab^{x}& > & 0\hfill \\ 
\hfill ab^{x}+q& > & q\hfill \\ 
\hfill f(x)& > & q\hfill 
\end{array}
\end{equation*}
Dus, as  $a>0$, dan is die waardeversameling  $\{f(x):f(x) > q\}$.\par 
Indien $a<0$ dan:\par 

\begin{equation*}
\begin{array}{ccc}\hfill b^{x}&< & 0\hfill \\
 \hfill ab^{x}& < & 0\hfill \\
\hfill ab^{x}+q& < & q\hfill \\
 \hfill f(x)& < & q\hfill 
\end{array}
\end{equation*}
Dus, as $a<0$, dan is die waardeversameling  $\{f(x):f(x) < q\}$.\par 

\begin{wex}{Gebied en terrein van 'n eksponent funksie}
{Vind die gebied en terrein van $g(x)=3.2^{x}+2$}
{
\westep{Vind die definisieversameling}
Die definisieversameling van $g(x)=3.2^{x}+2$ is $\{x:x\in \mathbb{R}\}$.
\westep{Vind die waardeversameling}
\begin{equation*}
\begin{array}{ccc}\hfill 2^{x}& > & 0\hfill \\
 \hfill 3.2^{x}& > & 0\hfill \\
 \hfill 3.2^{x}+2& > & 2\hfill 
\end{array}
\end{equation*}
Dus die waardeversameling is $\{g(x):g(x) > 1)\}$.\par 
}
 
\end{wex}




\subsection*{Afsnitte}
\textbf{Die $y$-afsnit}\\
Die $y$-afsnit word gegee deur $x=0$:
\begin{equation*}
\begin{array}{ccl}\hfill y& =& ab^{x}+q\hfill \\
 \hfill y}& =& ab^{0}+q\hfill \\
 & =& a(1)+q\hfill \\
 & =& a+q\hfill 
\end{array}
\end{equation*}

Byvoorbeeld, die $y$-afsnit van $g(x)=3.2^{x}+2$ word gegee deur 
 $x=0$ te stel, om dan te kry:\par 

\begin{equation*}
\begin{array}{ccl}\hfill y& =& 3.2^{x}+2\hfill \\
 \hfill y& =& 3.2^{0}+2\hfill \\
 & =& 3+2\hfill \\ & =& 5\hfill 
\end{array}
\end{equation*}
Dit gee die punt $(0;5)$.\vspace{10pt}
\\
\textbf{Die $x$-afsnit}\\
Die $x$-afsnitte word bereken deur $y=0$ te stel. \\
Byvoorbeeld, die $x$-afsnit van $g(x)=3.2^{x}+2$ word gegee deur $y=0$ te stel:\par 
\begin{equation*}
\begin{array}{ccl}\hfill y& =& 3.2^{x}+2\hfill \\
 \hfill 0& =& 3.2^{x}+2\hfill \\
 \hfill -2& =& 3.2^{x}\hfill \\
 \hfill {2}^{x}& =& \dfrac{-2}{3}\hfill 
\end{array}
\end{equation*}
Dit het nie ‘n rëele oplossing nie. Dus, het die grafiek van die vorm $g(x)$ geen $x$-afsnitte.\par 

\subsection*{Asimptote}

Daar is een asimptoot vir funksies van die vorm $y=ab^{x}+q$  Die horisontale asimptoot lê by $x=q$. 


\subsection*{Sketse van grafieke van die vorm $f(x)=ab^{x}+q$}

Om grafieke te skets van funksies van die vorm, $f(x)=ab^{x}+q$, moet ons die volgende eienskappe in ag neem:\par 
\begin{enumerate}[noitemsep, label=\textbf{\arabic*}. ] 
\item waardes van $a$ en $q$
\item $y$-afsnit
\item $x$-afsnit
\item asimptote
\end{enumerate}

\begin{wex}{Skets die grafiek van 'n eksponent funksie}
{ Skets die grafiek van $g(x)=3(2^{x})+2$. Merk die afsnitte en asimptote.}
{
\westep{ondersoek die standaardvorm van die vergelyking}
Ons sien $a>1$, due die kromme buig opwaarts. $q>0$ dus die grafiek word $2$ eenhede vertikale op geskuif.

\westep{Bereken die afsnitte}
Ons kry die $y$-afsnit, waar $x=0$:
\begin{equation*}
\begin{array}{ccl}\hfill y& =& 3.2^{x}+2\hfill \\
 \hfill y& =& 3.2^{0}+2\hfill \\
 & =& 3+2\hfill \\ & =& 5\hfill 
\end{array}
\end{equation*}
Dit gee die punt $(0;5)$.\\

Ons kry die $x$-afsnit, waar $y=0$:
\begin{equation*}
\begin{array}{ccl}\hfill y& =& 3.2^{x}+2\hfill \\
 \hfill 0& =& 3.2^{x}+2\hfill \\
 \hfill -2& =& 3.2^{x}\hfill \\
 \hfill {2}^{x}& =& \dfrac{-2}{3}\hfill 
\end{array}
\end{equation*}
Daar is geen re\"ele oplossing, dus daar is geen $x$-afsnitte nie.

\westep{Bepaal die asimptoot}
Die lyn $y=2$ is die horisontale asimptoot.

\westep{Stip die punte en skets die grafiek}
\setcounter{subfigure}{0}
\begin{figure}[htbp]
\begin{center}
\begin{pspicture}(-5,-1)(5,6)
%\psgrid
\psset{yunit=0.75,xunit=0.75}
\psaxes[arrows=<->](0,0)(-5,-1)(5,7)
\psplot[plotstyle=curve,arrows=<->]{-5}{0.5}{2 x exp 3 mul 2 add}
\end{pspicture}
% \caption{Graph of $g(x)=3( 2^{x}) + 2$.}
% \label{fig:mf:g:exponentialsketchexample10}
\end{center}
\end{figure}    
\\
Gebeid: $\{x \in \mathbb{R}\}$\\
Terrein: $\{g(x): g(x) >2\}$\\

Eksponent grafieke het geen simmertrie asse.
} 
\end{wex}


 
\begin{wex}{Skets ’n eksponsiële grafiek}
{
Skets die grafiek van $y=-2.3^{x}+6$.}
{
\westep{Ondersoek die standaardvorm van die vergelyking} 
 $a<0$ dus die grafiek buig af. $q>0$ dus skuif die grafiek $6$ eenhede vertikaal opwaarts.
\westep{Calculate the intercepts}

Ons kry die $y$-afsnit waar $x=0$:
\begin{equation*}
\begin{array}{ccl}\hfill y& =& -2.3^{x}+6\hfill \\
 \hfill y& =& -2.3^{0}+6\hfill \\
 & =& 4\hfill \\

\end{array}
\end{equation*}
Dit gee die punt $(0;4)$.\\

Ons kry die $x$-afsnit waar $y=0$:
\begin{equation*}
\begin{array}{ccl}\hfill y& =& -2.3^{x}+6\hfill \\
 \hfill 0& =& -2.3^{x}+6\hfill \\
 \hfill -6& =& -2.3^{x}\hfill \\
 \hfill {3}^{1}& =& {3}^{x}\hfill \\
\hfill \therefore x & =& 1 
\end{array}
\end{equation*}
Dit gee die punt $(1; 0)$
\westep{Bepaal die asimptoot} 
Funksies van hierdie vorm het een asimptoot. Dit lê by $y=q$, van die grafiek is by $y=5$.



\westep{Stip die punte en skets die grafiek} 
% Diagram needed here
\\
Gebied is: $\{x \in \mathbb{R}\}$\\
Terrein is: $\{g(x): g(x) <6\}$\\

}
\end{wex}




\begin{exercises}{ }
 {
 
Skets die grafieke van $y=2^{x}$ en $y=(\frac{1}{2})^{x}$ op dieselfde assestelsel.
\begin{enumerate}[noitemsep, label=\textbf{\arabic*}. ] 
\item Is die $x$-as die asimptoot en/of simmetrie-as in albei grafieke? Verduidelik jou antwoord.
\item Watter grafiek word aangedui met die volgende vergelyking $y=2^{-x}$ ? Verduidelik jou antwoord.
\item Los die vergelyking $2^{x}=(\frac{1}{2})^{x}$ met behulp van ’n skets op en kontroleer jou antwoord deur middel van substitusie.
\end{enumerate}
Die kurwe van die eksponensiele funksie $f$ in die meegaande diagram sny die y-as by die punt $A(0; 1)$ en $B(3; 9)$ is op $f$.
% THIS DIAGRAM NEEDS FIXING!
% \setcounter{subfigure}{0}
% \begin{center}
% \scalebox{1} % Change this value to rescale the drawing.
% {
% \begin{pspicture}(0,-3.19)(6.3184376,3.21)
% \psdots[dotsize=0.12](5.0,1.49)
% \psdots[dotsize= 0.12](1.0,-1.51)
% \psbezier[linewidth=0.04](0.0,-1.81)(5.0,-0.41)(5.5,2.59)(5.6,3.19)
% \rput(5.7725,1.7){$B(2;4)$}
% \rput(1.7076563,-1.8){$A(0;1)$}
% \rput(1.0,-2.51){\psaxes[linewidth=0.04,arrowsize=0.05291667cm 2.0,arrowlength= 1.4,arrowinset=0.4,ticksize=0.08cm,dx=2.0cm,dy=1.0cm]{<->}(0,0)(-1,0)(5,5)}
% \rput(5.994375,-2.8){$x$}
% \rput(0.564375,2.5){$f$}
% \end{pspicture} 
% }     
% \end{center}
\begin{enumerate}[noitemsep, label=\textbf{\arabic*}. ] 
\setcounter{enumi}{3}
\item  Bepaal die vergelyking van funksie $f$.
\item  Bepaal die vergelyking van $h$, die refleksie van die kurwe van $f$ in die $x$-as.
\item  Bepaal die waardeversameling van $h$.
\item Bepaal die vergelyking van  $g$, die refleksie van die kurwe van $f$ in die $y$-as.
\item Bepaal die vergelyking van $j$ indien $j$ 'n vertikale strekking is van $f$ met $+2$ eenhede.
\item Bepaal die vergelyking van $k$ indien $k$ 'n verikale skuif is van $f$ met $-3$ eenhede.
\end{enumerate}


\insertpracticeinfo{9}
}
\end{exercises}

\section{Trigonometriese funksies}
\nopagebreak
Aangesien die trigonometriese berhoudings verbande tussen, twee veranderlikes is, b.v, $y=sinx$, kan hulle oook grafies voorgestel word.
Hierdie afdeling beskryf die grafieke van trigonometriese funksies.\par 

\mindsetvid{Sine and cosine graphs}{VMazc}

\subsection{Grafiek van $y=sin~\theta $}
\subsection*{Teken die grafiek}
\nopagebreak
Volgooi die volgende tabel en gebruik jou sakrekenaar om die waardes te bereken. Stip dan die waardes met $sin~\theta $ op die $y$-as en $\theta $op die $x$-as.Rond die antwoorde af tot $1$ desimale plek.\par 

\setlength\mytablespace{16\tabcolsep}
\addtolength\mytablespace{9\arrayrulewidth}
\setlength\mytablewidth{\linewidth}
\setlength\mytableroom{\mytablewidth}
\addtolength\mytableroom{-\mytablespace}
\setlength\myfixedwidth{0pt}
\setlength\mystarwidth{\mytableroom}
\addtolength\mystarwidth{-\myfixedwidth}
\divide\mystarwidth 80
% ----- Table with code
% \begin{table}[H]
% \\ '' '0'
\begin{center}
\label{m39414*id83562}
\noindent

\begin{tabular*}{\mytablewidth}{|p{10\mystarwidth}|p{10\mystarwidth}|p{10\mystarwidth}|p{10\mystarwidth}|p{10\mystarwidth}|p{10\mystarwidth}|p{10\mystarwidth}|p{10\mystarwidth}|}\hline

$\theta $  &
$0^{\circ }$ &
$30^{\circ }$ &
$60^{\circ }$ &
$90^{\circ }$ &
$120^{\circ }$ &
$150^{\circ }$ &
\\ \hline

$sin~\theta $ 
&
&
&
&
&
&
&
\\ \hline

$\theta $&
$180^{\circ }$ &
$210^{\circ }$ &
$240^{\circ }$ &
$270^{\circ }$ &
$300^{\circ }$ &
$330^{\circ }$ &
$360^{\circ }$% make-rowspan-placeholders
\\ \hline

$sin~\theta $
&
&
&
&
&
&
&
\\ \hline

\multicolumn{8}{|p{\dimexpr10\mystarwidth+10\mystarwidth+10\mystarwidth+10\mystarwidth+10\mystarwidth+10\mystarwidth+10\mystarwidth+10\mystarwidth+14\tabcolsep+7\arrayrulewidth\relax}|}{}

\\ \hline

\multicolumn{8}{|p{\dimexpr10\mystarwidth+10\mystarwidth+10\mystarwidth+10\mystarwidth+10\mystarwidth+10\mystarwidth+10\mystarwidth+10\mystarwidth+14\tabcolsep+7\arrayrulewidth\relax}|}{
\setcounter{subfigure}{0}
\label{m39414*id84030}
\begin{center}
\begin{pspicture}(0,-1)(4,1)
\psset{xunit=2}
%\psgrid[gridcolor=gray]
\psset{xunit=0.01111}
\psaxes[dx=30,Dx=30]{<->}(0,0)(0,-1.2)(370,1.2)
\end{pspicture}
\end{center}    
  }
\\ \hline
%--------------------------------------------------------------------
\end{tabular*}
\end{center}
\begin{center}{\small\bfseries Table 14.4}\end{center}
%\end{table}
\par
\label{m39414*id84056}Laat ons terugkyk na ons waardes vir $sin~\theta $\par 
\begin{table}[H]
% \begin{table}[H]
% \\ '' '0'
\begin{center}
\label{m39414*id84073}

\begin{tabular}{|l|l|l|l|l|l|l|}\hline
$\theta $&
${0}^{\circ }$&
${30}^{\circ }$&
${45}^{\circ }$&
${60}^{\circ }$&
${90}^{\circ }$&
${180}^{\circ }$
  % make-rowspan-placeholders
\\ \hline
%--------------------------------------------------------------------
$sin~\theta $&
$0$ &
$\frac{1}{2}$&
$\frac{1}{\sqrt{2}}$&
$\frac{\sqrt{3}}{2}$&
$1$ &
$0$% make-rowspan-placeholders
\\ \hline
%--------------------------------------------------------------------
\end{tabular}
\end{center}
\begin{center}{\small\bfseries Table 14.5}\end{center}
% \begin{caption}{\small\bfseries Table 14.5}\end{caption}
\end{table}
\par
Soos jy kan sien, die funksie $sin~\theta $ het ’n waarde van $0$ by $\theta ={0}^{\circ }$. Sy waarde neem egalig toe tot by $\theta ={90}^{\circ }$ wanneer sy waarde $1$.is. Ons weet ook dat dit later afneem na $0$ as $\theta ={180}^{\circ }$.Deur dit alles bymekaar te sit,
kan ons ’n idee kry van die volle omvang van die sinuskurwe. Die sinuskurwe word gewys in Figuur 14.23. Let
op die kurwe se vorm, waar elke kurwe die lengte het van ${360}^{\circ }$. Ons sê die grafiek het ’n periode van ${360}^{\circ }$.  Die
hoogte van die kurwe bo (of onder) die $x$-as word die kurwe se amplitude genoem. Dus is die amplitude van die
sinuskurwe is  $1$, and its minimum amplitude is $-1$.\par 
\setcounter{subfigure}{0}
\begin{figure}[h]
\begin{center}
\begin{pspicture}(-6,-2)(6,2)
\psaxes[Ox=0, Dx=180, dx=2]{<->}(0,0)(-4.5,-1.5)(4.5,1.5)
\psplot[xunit=0.0111,yunit=1.0, plotpoints=1000]{-360}{360}{x sin}
% \rput(5.3,-.3){$degrees$}
\psline[linestyle=dashed](0, 1)(4, 1)
\psline[linestyle=dashed](0, -1)(4, -1)
\end{pspicture}
\caption{Die grafiek van $sin~ \theta$.}
\label{trig:sin}
\end{center}
\end{figure}
      

\subsection*{Funksies in die vorm $y=a~sin~x+q$}
\nopagebreak
In die vergelyking, $y=a~sin~x+q$, $a$ en $q$  is konstantes en het verskillende invloede op die grafiek van die
funksie. Die algemene vorm van hierdie grafiek word gewys in Figuur 14.24 vir die funksie $f(\theta )=2sin~\theta +3$.\par 
\setcounter{subfigure}{0}
\begin{figure}[!ht]
\begin{center}
\begin{pspicture}(-4,-2)(4,6)
%\psgrid[gridcolor=gray]
\psset{yunit=1,xunit=0.01111}
\psaxes[dx=90,Dx=90]{<->}(0,0)(-360,-2)(360,6)
\psplot[plotstyle=curve,arrows=<->]{-360}{360}{x sin 2 mul 3 add}
% \rput(420,-0.3){$degrees$}
\end{pspicture}
\caption{Die grafiek van $f(\theta)=2 \sin \theta +3$}
\label{fig:mt:g:sin}
\end{center}
\end{figure}   

\subsection*{Ondersoek van die invloed van a en q}
\nopagebreak
\begin{enumerate}[noitemsep, label=\textbf{\arabic*}. ] 
\item Op dieselfde stel asse, trek die volgende grafieke:
\begin{enumerate}[noitemsep, label=\textbf{\alph*}. ] 
\item $a(\theta )=sin~\theta -2$
\item $b(\theta )=sin~\theta -1$
\item $c(\theta )=sin~\theta $
\item $d(\theta )=sin~\theta +1$
\item $e(\theta )=sin~\theta +2$
\end{enumerate}
Gebruik jou resultate om afleidings te maak oor die invloed van $q$.
\item Trek grafieke van die volgende op dieselfde stel asse vir $0^{\circ}\leq. \theta\leq 360^ {\circ}$
\begin{enumerate}[noitemsep, label=\textbf{\alph*}. ] 
\item $f(\theta )=-2~sin~\theta $
\item $g(\theta )=-1~sin~\theta $
\item $h(\theta )=0~sin~\theta $
\item $j(\theta )=1~sin~\theta $
\item $k(\theta )=2~sin~\theta $\end{enumerate}
Gebruik jou resultate om die invloed van $a$ af te lei.
\end{enumerate}
Jy behoort te vind dat die waarde van $a$ die hoogte van die pieke van die grafiek beïnvloed. As die grootte van $a$ toeneem, word die pieke hoër. As dit afneem, word die pieke laer.\par 
$q$ veroorsaak 'n vertikale skuif. As $q=2$, skuif die hele sinus-grafiek $2$ eenhede op. As $q=-1$, skuif die hele grafiek $1$ eenheid af.\par 
Hierdie eienskappe word opgesom in Tabel 14.6.\par 
\begin{table}[htb]
\begin{center}
\caption{Tabel wat die funksies in die vorm $y=a \sin(x) + q$ opsom.}
\label{tab:mt:g:summarysin10}
\begin{tabular}{|c|c|c|}\hline
& $a>0$&$a<0$\\\hline
$q>0$&
\begin{pspicture}(-1.2,-0.6)(1.2,1.2)
\psset{yunit=0.5,xunit=0.0111}
\psaxes[arrows=<->,dx=0,Dx=720,dy=0,Dy=10,xunit=0.25](0,0)(-450,-1)(450,2)
\psplot[plotstyle=curve,arrows=<->,xunit=0.25]{-360}{360}{x sin 0.5 add}
% \rput(5.1,-.3){Degrees}
\end{pspicture}
&
\begin{pspicture}(-1.2,-0.6)(1.2,1.2)
\psset{yunit=0.5,xunit=0.0111}
\psaxes[arrows=<->,dx=0,Dx=720,dy=0,Dy=10,xunit=0.25](0,0)(-450,-1)(450,2)
\psplot[plotstyle=curve,arrows=<->,xunit=0.25]{-360}{360}{x sin neg 0.5 add}
% \rput(5.1,-.3){Degrees}
\end{pspicture}\\\hline
$q<0$&
\begin{pspicture}(-1.2,-1.2)(1.2,0.6)
%\psgrid
\psset{yunit=0.5,xunit=0.0111}
\psaxes[arrows=<->,dx=0,Dx=720,dy=0,Dy=10,xunit=0.25](0,0)(-450,-2)(450,1)
\psplot[plotstyle=curve,arrows=<->,xunit=0.25]{-360}{360}{x sin 0.5 sub}
% \rput(5.1,-.3){Degrees}
\end{pspicture}
&
\begin{pspicture}(-1.2,-1.2)(1.2,0.6)
%\psgrid
\psset{yunit=0.5,xunit=0.0111}
\psaxes[arrows=<->,dx=0,Dx=720,dy=0,Dy=10,xunit=0.25](0,0)(-450,-2)(450,1)
\psplot[plotstyle=curve,arrows=<->,xunit=0.25]{-360}{360}{x sin neg 0.5 sub}
% \rput(5.1,-.3){Degrees}
\end{pspicture}\\\hline
\end{tabular}
\end{center}
\end{table}
\par

\subsection*{Ondek die kenmerke}
\subsection*{Gebied en terrein}
\nopagebreak
Die $f(\theta )=a~sin~\theta +q$, is die gebied $\{\theta :\theta \in \mathbb{R}\}$  omdat daar geen waarde is van $\theta \in \mathbb{R}$ waarvoor $f(\theta )$ ongedefinieerd is nie..\par 
Die terrein van $f(\theta )=a~sin~\theta +q$ hang daarvan af of die waarde vir $a$ positief of negatief is. Ons sal die twee
gevalle afsonderlik oorweeg.\par 
As $a>0$ het ons:\par 
\nopagebreak\noindent{}
\begin{equation*}
\begin{array}{ccc}\hfill -1\leq sin~\theta & \leq & 1\hfill \\ \hfill -a\leq a~sin~\theta & \leq & a\hfill \\ \hfill -a+q\leq a~sin~\theta +q& \leq & a+q\hfill \\ \hfill -a+q\leq f(\theta )& \leq & a+q\hfill \end{array}
\end{equation*}
Dit vertel ons dat vir alle waardes van $\theta $, $f(\theta )$ altyd tussen $-a+q$ en $a+q$ is.\par
Daarom as $a>0$ is die terrein van  $f(\theta )=a~sin~\theta +q$ is $\{f(\theta ):f(\theta )\in [a+q,-a+q]\}$.
Insgelyks, kan daar getoon word dat as $a<0$, dan is die terrein van $f(\theta )=a~sin~\theta +q$ $\{f(\theta ):f(\theta )\in [a+q,-a+q]\}$. Dit word as ’n oefening gelaat.\par 

\Tip{Die maklikste manier om die terrein te bepaal is om bloot vir die "bokant" en die "onderkant" van die grafiek te soek.}


\subsection*{As-afsnitte}
\nopagebreak
Die $y$-afsnit, ${y}_{int}$, van $f(\theta )=a~sin~x+q$ is eenvoudig die waarde van$f(\theta )$ by $\theta ={0}^{\circ }$.
\begin{equation*}
\begin{array}{ccc}
  \hfill {y}_{int}& =& f({0}^{\circ })\hfill \\
 & =& a~sin({0}^{\circ })+q\hfill \\
 & =& a(0)+q\hfill \\
 & =& q\hfill 
\end{array}
\end{equation*}



\subsection{Grafiek van $y=cos~\theta $}
\subsection*{Teken die grafiek}
\nopagebreak
 Voltooi die volgende tabel, gebruik jou sakrekenaar om die waardes korrek tot $1$ desimale plek te bereken. Stip
dan die waardes met $cos~\theta $ op die $y$-as en $\theta $ op die $x$-as.\par 

% -------------------------
\setlength\mytablespace{16\tabcolsep}
\addtolength\mytablespace{9\arrayrulewidth}
\setlength\mytablewidth{\linewidth}
\setlength\mytableroom{\mytablewidth}
\addtolength\mytableroom{-\mytablespace}
\setlength\myfixedwidth{0pt}
\setlength\mystarwidth{\mytableroom}
\addtolength\mystarwidth{-\myfixedwidth}
\divide\mystarwidth 80subsub
% ----- Table with code
% \begin{table}[H]
% \\ '' '0'
\begin{center}
\label{m39414*id86399}
\noindent

\begin{tabular*}{\mytablewidth}{|p{10\mystarwidth}|p{10\mystarwidth}|p{10\mystarwidth}|p{10\mystarwidth}|p{10\mystarwidth}|p{10\mystarwidth}|p{10\mystarwidth}|p{10\mystarwidth}|}\hline
$\theta $     &
$0^{\circ }$ &
$30^{\circ }$ &
$60^{\circ }$ &
$90^{\circ }$ &
$120^{\circ }$ &
$150^{\circ }$ &
\\ \hline

$cos~\theta $  &
&
&
&
&
&
&
\\ \hline

$\theta $    &
$180^{\circ }$ &
$210^{\circ }$ &
$240^{\circ }$ &
$270^{\circ }$ &
$300^{\circ }$ &
$330^{\circ }$ &
$360^{\circ }$

\\ \hline

$cos~\theta $&
&
&
&
&
&
&

\\ \hline

\multicolumn{8}{|p{\dimexpr10\mystarwidth+10\mystarwidth+10\mystarwidth+10\mystarwidth+10\mystarwidth+10\mystarwidth+10\mystarwidth+10\mystarwidth+14\tabcolsep+7\arrayrulewidth\relax}|}{}

\\ \hline

\multicolumn{8}{|p{\dimexpr10\mystarwidth+10\mystarwidth+10\mystarwidth+10\mystarwidth+10\mystarwidth+10\mystarwidth+10\mystarwidth+10\mystarwidth+14\tabcolsep+7\arrayrulewidth\relax}|}{
\setcounter{subfigure}{0}
\begin{pspicture}(0,-1)(4,1)
\psset{xunit=2}
%\psgrid[gridcolor=gray]
\psset{xunit=0.01111}
\psaxes[dx=30,Dx=30]{<->}(0,0)(0,-1.2)(370,1.2)
\end{pspicture} 
  }

\\ \hline

\end{tabular*}
\end{center}
\begin{center}{\small\bfseries Table 14.7}\end{center}
%\end{table}
\par
\label{m39414*id86892}Laat ons terugkyk na ons waardes vir $cos~\theta $\par 
% \textbf{m39414*uid46}\par
\begin{table}[H]
% \begin{table}[H]
% \\ '' '0'
\begin{center}
\label{m39414*id86909}
\noindent

\begin{tabular}{|l|l|l|l|l|l|l|}\hline
$\theta $
${0}^{\circ }$&
${30}^{\circ }$&
${45}^{\circ }$&
${60}^{\circ }$&
${90}^{\circ }$&
${180}^{\circ }$&
  % make-rowspan-placeholders
\\ \hline
%--------------------------------------------------------------------
$cos~\theta $&
$1$ &
$\frac{\sqrt{3}}{2}$&
$\frac{1}{\sqrt{2}}$&
$\frac{1}{2}$&
$0$ &
$-1$
  % make-rowspan-placeholders
\\ \hline
%--------------------------------------------------------------------
\end{tabular}
\end{center}
\begin{center}{\small\bfseries Table 14.8}\end{center}
% \begin{caption}{\small\bfseries Table 14.8}\end{caption}
\end{table}
\par
As jy noukeurig kyk, sal jy oplet dat die cosinus van ’n hoek $\theta $  dieselfde is as die sinus van die hoek ${90}^{\circ }-\theta $. Neem byvoorbeeld,\par 
\nopagebreak\noindent{}
\begin{equation*}
cos~{60}^{\circ }=\frac{1}{2}=sin~{30}^{\circ }=sin~({90}^{\circ }-{60}^{\circ })
\end{equation*}
Dit wys ons dat ten einde ’n cosinusgrafiek te skep, al wat ons hoef te doen is om die sinusgrafiek ${90}^{\circ }$ na links te
skuif. die grafiek van $cos~\theta $ word gewys in Figuur 14.30. As die cosinusgrafiek eenvoudig ’n geskuifde sinusgrafiek
is, sal dit dieselfde periode en amplitude as die sinuskurwe hê.\par 
\setcounter{subfigure}{0}
\begin{figure}[h]
\begin{center}
\begin{pspicture}(-6,-2)(6,2)
\psaxes[Ox=0, Dx=180, dx=2]{<->}(0,0)(-4.5,-1.5)(4.5,1.5)
%\psline[]{<-}(0,1.2)(1,1.2)
\psplot[xunit=0.0111,yunit=1.0, plotpoints=1000]{-360}{360}{x cos}
% \rput(5.1,-.3){Degrees}
\end{pspicture}
\caption{The graph of $\cos \theta$.}
\label{trig:cos}
\end{center}
\end{figure}      

\subsection*{Funksies in die vorm  $y=a~cos~x+q$}
\nopagebreak
In die vergelyking, $y=a~cos~x+q$, $a$ en $q$ is konstantes en het verskillende invloede op die grafiek van die
funksie. Die algemene vorm van die grafieke van hierdie soort funksies word getoon in Figuur 14.31 vir die
funksie $f(\theta )=2~cos~\theta +3$.\par 
\setcounter{subfigure}{0}
\begin{figure}[!ht]
\begin{center}
\begin{pspicture}(-4,-2)(4,6)
%\psgrid[gridcolor=gray]
\psset{yunit=1,xunit=0.01111}
\psaxes[dx=90,Dx=90]{<->}(0,0)(-360,-2)(360,6)
\psplot[plotstyle=curve,arrows=<->]{-360}{360}{x cos 2 mul 3 add}
% \rput(5.1,-.3){Degrees}
\end{pspicture}
\caption{Graph of $f(\theta)=2 ~\cos ~\theta +3$}
\label{fig:mt:g:cos}
\end{center}
\end{figure}      

\subsection*{Die invloed van a en q}
\nopagebreak
\begin{enumerate}[noitemsep, label=\textbf{\arabic*}. ] 
\item Trek grafieke van die volgende op dieselfde stel asse (vir $0^{\circ} \leq \theta \leq 360^{circ}$:
\begin{enumerate}[noitemsep, label=\textbf{\alph*}. ] 
\item $a(\theta )=cos~\theta -2$
\item $b(\theta )=cos~\theta -1$
\item $c(\theta )=cos~\theta $
\item $d(\theta )=cos~\theta +1$
\item $e(\theta )=cos~\theta +2$\end{enumerate}
Gebruik jou resultate om die invloed van $q$ af te lei.
\item Trek grafieke van die volgende op dieselfde stel asse (vir $0^{\circ} \leq \theta \leq 360^{circ}$:
\begin{enumerate}[noitemsep, label=\textbf{\alph*}. ] 
\item $f(\theta )=-2~cos~\theta $
\item $g(\theta )=-1~cos~\theta $
\item $h(\theta )=0~cos~\theta $
\item $j(\theta )=1~cos~\theta $
\item $k(\theta )=2~cos~\theta $\end{enumerate}
Gebruik jou resultate om die invloed van $a$ af te lei.
\end{enumerate}
Ons vind dat die waarde van $a$ die amplitude van die cosinusgrafiek op dieselfde manier beïnvloed as wat dit vir
die sinusgrafiek gedoen het.\par 
Verandering in die waarde van $q$ sal die die cosinusgrafiek op dieselfde manier skuif as wat dit vir die sinusgrafiek
gedoen het.\par 
Die verskillende eienskappe word opgesom in Tabel 14.9.\par 
\begin{table}[htb]
\begin{center}
\caption{Tabel wat die algemene vorms en posisies van grafieke en funksies in die vorm $y=a~ \cos~x + q$ opsom.}
\label{tab:mt:g:summarycos10}
\begin{tabular}{|c||c|c|}\hline
& $a>0$&$a<0$\\\hline\hline
$q>0$&
\begin{pspicture}(-1.2,-0.6)(1.2,1.2)
\psset{yunit=0.5,xunit=0.0111}
\psaxes[arrows=<->,dx=0,Dx=720,dy=0,Dy=10,xunit=0.25](0,0)(-450,-1)(450,2)
\psplot[plotstyle=curve,arrows=<->,xunit=0.25]{-360}{360}{x cos 0.5 add}
\end{pspicture}
&
\begin{pspicture}(-1.2,-0.6)(1.2,1.2)
\psset{yunit=0.5,xunit=0.0111}
\psaxes[arrows=<->,dx=0,Dx=720,dy=0,Dy=10,xunit=0.25](0,0)(-450,-1)(450,2)
\psplot[plotstyle=curve,arrows=<->,xunit=0.25]{-360}{360}{x cos neg 0.5 add}
\end{pspicture}\\\hline
$q<0$&
\begin{pspicture}(-1.2,-1.2)(1.2,0.6)
%\psgrid
\psset{yunit=0.5,xunit=0.0111}
\psaxes[arrows=<->,dx=0,Dx=720,dy=0,Dy=10,xunit=0.25](0,0)(-450,-2)(450,1)
\psplot[plotstyle=curve,arrows=<->,xunit=0.25]{-360}{360}{x cos 0.5 sub}
\end{pspicture}
&
\begin{pspicture}(-1.2,-1.2)(1.2,0.6)
%\psgrid
\psset{yunit=0.5,xunit=0.0111}
\psaxes[arrows=<->,dx=0,Dx=720,dy=0,Dy=10,xunit=0.25](0,0)(-450,-2)(450,1)
\psplot[plotstyle=curve,arrows=<->,xunit=0.25]{-360}{360}{x cos neg 0.5 sub}
\end{pspicture}\\\hline
\end{tabular}
\end{center}
\end{table}
\par

\subsection*{Ontdek die kenmerke}
\subsection*{Gebied en Terrein}
\nopagebreak
Vir $f(\theta )=a~cos(\theta )+q$, is die gebied $\{\theta :\theta \in \mathbb{R}\}$ want daar is geen waarde van $\theta \in \mathbb{R}$ waarvoor $f(\theta )$ ongedefinieërd is nie.\par 
Dit is maklik om te sien dat die terrein van $f(\theta )$ dieselfde sal wees as die terrein van $a~sin(\theta )+q$. Dit is omdat
die maksimum en minimumwaardes van $a~cos(\theta )+q$ dieselfde is as die maksimum en minimumwaardes van $a~sin(\theta )+q$.\par 

\subsection*{As-afsnitte}
\nopagebreak
Die $y$-afsnit van $f(\theta )=a~cos~x+q$ word bereken op dieselfde wyse as vir sinus.\par 
\nopagebreak\noindent{}
\begin{equation*}
\begin{array}{ccc}\hfill {y}_{int}& =& f({0}^{\circ })\hfill \\
 & =& a~cos({0}^{\circ })+q\hfill \\
 & =& a(1)+q\hfill \\
 & =& a+q\hfill 
\end{array}
\end{equation*}

\subsection*{Vergelyking die Grafieke van $y=sin~\theta $ en $y=cos~\theta $}
\nopagebreak
\setcounter{subfigure}{0}
\begin{figure}[h]
\begin{center}
\begin{pspicture}(-6,-2)(6,2)
\psaxes[Ox=0, Dx=180, dx=2]{<->}(0,0)(-4.5,-1.5)(4.5,1.5)
\psline[]{<-}(0,1.2)(1,1.2)
\psplot[xunit=0.0111, , yunit=1.0, plotpoints=1000, linestyle=dashed]{-360}{360}{x sin}
\psplot[xunit=0.0111,yunit=1.0, plotpoints=1000]{-360}{360}{x cos}
% \rput(5.1,-.3){Degrees}
\rput(1.8,1.2){$90^\circ$ shift}
\end{pspicture}
\caption{Die grafiek van $\cos \theta$ (soliede lyn) en die grafiek van $\sin \theta$ (stippellyn).}
\end{center}
\end{figure}    
Let daarop dat die twee grafieke baie eenders lyk. Beide ossilleer op en af rondom die $x$-as soos wat jy beweeg
langs die as. Die afstande tussen die pieke van die twee grafieke is dieselfde en is konstant vir elke grafiek. Die
hoogte van elke piek en die diepte van elke trog is dieselfde.\par 
Die enigste verskil is dat die $sin$  grafiek skuif $90 ^{\circ }$ na regs ten opsigte van die $cos$ grafiek, met $90^{\circ }$. Dit beteken
dat as ons die hele $cos$ grafiek $90 ^{\circ }$ na regs skuif, sal dit perfek oorvleuel met die $sin$ grafiek. Jy kan ook die $sin$ grafiek $90 ^{\circ }$na links skuif en dan sal dit perfek oorvleuel met die $cos$  grafiek. Dit beteken dat:\par 
\nopagebreak\noindent{}
\begin{equation*}
  \begin{array}{ccc}
    \hfill sin~\theta &=& cos~(\theta -90)(\mbox{skuif die } cos\mbox{ grafiek na die regterkant})\hfill \\
    & \mbox{en} & \\
    \hfill cos~\theta &=& sin(\theta +90)(\mbox{skuif die}sin\mbox{ grafiek na die linkerkant})\hfill 
  \end{array}
\end{equation*}

\subsection{Grafiek van $y=tan~\theta $ }
\subsection*{Plotting the graph}
\nopagebreak
Voltooi die volgende tabel, gebruik jou sakrekenaar en bereken die waardes korrek tot $1$ desimale plek. Stip dan
die waardes met $tan~\theta $ op die $y$-as en $\theta $ op die $x$-as.\par 

% -------------------------
\setlength\mytablespace{16\tabcolsep}
\addtolength\mytablespace{9\arrayrulewidth}
\setlength\mytablewidth{\linewidth}
\setlength\mytableroom{\mytablewidth}
\addtolength\mytableroom{-\mytablespace}
\setlength\myfixedwidth{0pt}
\setlength\mystarwidth{\mytableroom}
\addtolength\mystarwidth{-\myfixedwidth}
\divide\mystarwidth 80

\begin{center}
\label{m39414*id89083}
\noindent

\begin{tabular*}{\mytablewidth}{|p{10\mystarwidth}|p{10\mystarwidth}|p{10\mystarwidth}|p{10\mystarwidth}|p{10\mystarwidth}|p{10\mystarwidth}|p{10\mystarwidth}|p{10\mystarwidth}|}\hline

$\theta $ &
$0^{\circ }$ &
$30^{\circ }$ &
$60^{\circ }$ &
$90^{\circ }$ &
$120^{\circ }$ &
$150^{\circ }$ &

\\ \hline

$tan~\theta $ &
&
&
&
&
&
&

\\ \hline

$\theta $ &
$180^{\circ }$ &
$210^{\circ }$ &
$240^{\circ }$ &
$270^{\circ }$ &
$300^{\circ }$ &
$330^{\circ }$ &
$360^{\circ }$
\\ \hline

$tan~\theta $ &
&
&
&
&
&
&

\\ \hline
%--------------------------------------------------------------------

\multicolumn{8}{|p{\dimexpr10\mystarwidth+10\mystarwidth+10\mystarwidth+10\mystarwidth+10\mystarwidth+10\mystarwidth+10\mystarwidth+10\mystarwidth+14\tabcolsep+7\arrayrulewidth\relax}|}{}

\\ \hline

\multicolumn{8}{|p{\dimexpr10\mystarwidth+10\mystarwidth+10\mystarwidth+10\mystarwidth+10\mystarwidth+10\mystarwidth+10\mystarwidth+10\mystarwidth+14\tabcolsep+7\arrayrulewidth\relax}|}{
\setcounter{subfigure}{0}
\begin{pspicture}(0,-1)(4,1)
\psset{xunit=2}
%\psgrid[gridcolor=gray]
\psset{xunit=0.01111}
\psaxes[dx=30,Dx=30]{<->}(0,0)(0,-1.2)(370,1.2)
\end{pspicture}   
  }

\\ \hline
%--------------------------------------------------------------------
\end{tabular*}
\end{center}
\begin{center}{\small\bfseries Table 14.10}\end{center}
%\end{table}
\par
\label{m39414*id89576}Kom ons kyk weer na ons waardes vir $tan~\theta $\par 
% \textbf{m39414*uid65}\par
\begin{table}[H]
% \begin{table}[H]
% \\ '' '0'
\begin{center}
\label{m39414*id89593}
\noindent

\begin{tabular}[t]{|l|l|l|l|l|l|l|}\hline
$\theta $&
${0}^{\circ }$&
${30}^{\circ }$&
${45}^{\circ }$&
${60}^{\circ }$&
${90}^{\circ }$&
${180}^{\circ }$
\\ \hline
%--------------------------------------------------------------------
$tan~\theta $&
$0$ &
$\frac{1}{\sqrt{3}}$&
$1$ &
$\sqrt{3}$&
$\infty $&
$0$% make-rowspan-placeholders
\\ \hline
%--------------------------------------------------------------------
\end{tabular}
\end{center}
\begin{center}{\small\bfseries Table 14.11}\end{center}
% \begin{caption}{\small\bfseries Table 14.11}\end{caption}
\end{table}
\par
Nou dat ons die grafieke het vir $sin~\theta $ en $cos~\theta $, is daar ’n maklike manier om die tan-grafiek te visualiseer. Kom ons
kyk weer na ons definisies van $sin~\theta $ en $cos~\theta $ vir ’n reghoekige driehoek.\par 
\nopagebreak\noindent{}
\begin{equation*}
\frac{sin~\theta }{cos~\theta }=\frac{\frac{\mbox{teenoorstaande}}{\mbox{skuinssy}}}{\frac{\mbox{aangrensend}}{\mbox{skuinssy}}}=\frac{\mbox{teenoorstaande}}{\mbox{aangrensend}}=tan~\theta 
\end{equation*}
Dit is die eerste van ’n stel belangrike verbande wat ons trigonometriese identiteite noem. ’n Identiteit is waar vir
enige waarde van die onbekende(s) wat daarin ingestel word. In hierdie geval het ons aangetoon dat\par 
\nopagebreak\noindent{}
\begin{equation*}
tan~\theta =\frac{sin~\theta }{cos~\theta }
\end{equation*}
vir enige waarde van $\theta $.\par 
Dus weet ons dat vir die waardes van $\theta $  waarvoor $sin~\theta =0$, moet ook  $tan~\theta =0$. Soortgelyk, as $cos~\theta =0$ s die
waarde van $tan~\theta $  ongedefinieerd omdat ons nie mag deel met $0$ nie. Die grafiek word getoon in Figuur 14.38. Die
vertikale stippellyne is die waardes van $\theta $  waarvoor $tan~\theta $ nie gedefinieerd is nie.\par 
\setcounter{subfigure}{0}
\\begin{figure}[h]
\begin{center}
\begin{pspicture}(-6,-3)(6,3)
\psaxes[Dx=180, dx=2, Dy=2, dy=1]{<->}(0,0)(-4.5,-3)(4.5,3)
\psline[linestyle=dashed](-1,-2.5)(-1,2.5)
\psline[linestyle=dashed](1,-2.5)(1,2.5)
\psline[linestyle=dashed](-3,-2.5)(-3,2.5)
\psline[linestyle=dashed](3,-2.5)(3,2.5)
\psplot[xunit=0.0111,yunit=0.5, plotpoints=500, arrows=<->]{-80}{80}{x sin x cos div}
\psplot[xunit=0.0111,yunit=0.5,plotpoints=500, arrows=<->]{-260}{-100}{x sin x cos div}
\psplot[xunit=0.0111,yunit=0.5,plotpoints=500, arrows=<->]{260}{100}{x sin x cos div}
\psplot[xunit=0.0111,yunit=0.5,plotpoints=500, arrows=<->]{-100}{-260}{x sin x cos div}
\psplot[xunit=0.0111,yunit=0.5,plotpoints=500, arrows=<-]{-280}{-360}{x sin x cos div}
\psplot[xunit=0.0111,yunit=0.5,plotpoints=500, arrows=<-]{280}{360}{x sin x cos div}
% \rput(5.1,-.3){$\theta$\ Degrees}
\rput(.4,3.3){$f(\theta$)}
\end{pspicture}
% \caption{Die grafiek van $\tan \theta$.}
\label{trig:tan}
\end{center}
\end{figure}       

\subsection*{Funksies van die vorm $y=a ~tan~x+q$}
\nopagebreak
Die figuur hieronder is ’n voorbeeld van ’n funksie van die vorm $y=a ~tan(x)+q$.\par 
\setcounter{subfigure}{0}
\begin{figure}[!ht]
\begin{center}
\begin{pspicture}(-5,-3)(5,3.2)
%\psgrid
\psset{yunit=0.25}
\psaxes[Dx=180, dx=2, Dy=5, dy=5]{<->}(0,0)(-4.5,-12)(4.5,12)
\psline[linestyle=dashed](-1,-12.5)(-1,12.5)
\psline[linestyle=dashed](1,-12.5)(1,12.5)
\psline[linestyle=dashed](-3,-12.5)(-3,12.5)
\psline[linestyle=dashed](3,-12.5)(3,12.5)
\psplot[xunit=0.0111, plotpoints=500, arrows=<->]{-80}{80}{x sin x cos div 2 mul 1 add}
\psplot[xunit=0.0111,plotpoints=500, arrows=<->]{-260}{-100}{x sin x cos div 2 mul 1 add}
\psplot[xunit=0.0111,plotpoints=500, arrows=<->]{260}{100}{x sin x cos div 2 mul 1 add}
\psplot[xunit=0.0111,plotpoints=500, arrows=<->]{-100}{-260}{x sin x cos div 2 mul 1 add}
\psplot[xunit=0.0111,plotpoints=500, arrows=<-]{-280}{-360}{x sin x cos div 2 mul 1 add}
\psplot[xunit=0.0111,plotpoints=500, arrows=<-]{280}{360}{x sin x cos div 2 mul 1 add}
% \rput(5.3,-.6){$\theta$\ Degrees}
\rput(.4,9.3){$f(\theta$)}
\end{pspicture}
\caption{The graph of $2\tan \theta + 1$.}
\label{trig:tan2}
\end{center}
\end{figure}
       

\subsection*{Ondersoek die invloed van a en q}
\nopagebreak
\begin{enumerate}[noitemsep, label=\textbf{\arabic*}. ] 
\item Op dieselfde assestelsel, trek die volgende grafieke::
\begin{enumerate}[noitemsep, label=\textbf{\alph*}. ] 
\item $a(\theta )=tan~\theta -2$
\item $b(\theta )=tan~\theta -1$
\item $c(\theta )=tan~\theta $
\item $d(\theta )=tan~\theta +1$
\item $e(\theta )=tan~\theta +2$\end{enumerate}
Gebruik jou resultate om die invloed van $q$ af te lei.
\item Op dieselfde assestelsel, trek die volgende grafieke:
\begin{enumerate}[noitemsep, label=\textbf{\alph*}. ] 
\item $f(\theta )=-2~tan~\theta $
\item $g(\theta )=-1~tan~\theta $
\item $h(\theta )=0~tan~\theta $
\item $j(\theta )=1~tan~\theta $
\item $k(\theta )=2~tan~\theta $\end{enumerate}
Gebruik jou resultate om die invloed van $a$ af te lei.
\end{enumerate}
Ons vind dat die waarde van $a$ die steilheid van die bene van die grafiek beinvloed. Hoe groter die absolute waarde vanf $a$, hoe vinniger nader die bene die waardes van hulle asimptote, die waardes waar hulle nie gedefinieërd is nie. Negatiewe $a$ waardes keer die rigting waarin die bene van die grafiek loop, om. Ons vind
verder dat die waarde van $q$ beïnvloed die vertikale verskuiwing net soos by $sin~\theta $ en $cos~\theta $.
Hierdie verskillende eienskappe word opgesom in Tabel 14.12.\par 
\begin{table}[htb]
\begin{center}
\caption{Tabel van die algemene vorms en posisies van grafieke en funksies van die vorm $y=a~ \tan~x + q$.}
\label{tab:mt:g:summarytan10}
\begin{tabular}{|c||c|c|}\hline
& $a>0$&$a<0$\\\hline\hline
$q>0$&
\begin{pspicture}(-1.2,-0.6)(1.2,0.8)
%\psgrid[gridcolor=gray]
\psset{yunit=0.1,xunit=0.0111}
\psaxes[arrows=<->,dx=0,Dx=720,dy=0,Dy=10,xunit=0.25](0,0)(-450,-6)(450,7)
\psplot[plotstyle=curve,arrows=<->,xunit=0.25]{-81.5}{78}{x sin x cos div 1.5 add}cos
\end{pspicture}
&
\begin{pspicture}(-1.2,-0.6)(1.2,0.8)
%\psgrid[gridcolor=gray]
\psset{yunit=0.1,xunit=0.0111}
\psaxes[arrows=<->,dx=0,Dx=720,dy=0,Dy=10,xunit=0.25](0,0)(-450,-6)(450,7)
\psplot[plotstyle=curve,arrows=<->,xunit=0.25]{-78}{82.5}{x sin x cos div neg 1.5 add}
\end{pspicture}\\\hline
$q<0$&
\begin{pspicture}(-1.2,-0.8)(1.2,0.8)
%\psgrid[gridcolor=gray]
\psset{yunit=0.1,xunit=0.0111}
\psaxes[arrows=<->,dx=0,Dx=720,dy=0,Dy=10,xunit=0.25](0,0)(-450,-7)(450,6)
\psplot[plotstyle=curve,arrows=<->,xunit=0.25]{-80}{80}{x sin x cos div 1.5 sub}
\end{pspicture}
&
\begin{pspicture}(-1.2,-0.8)(1.2,0.8)
%\psgrid
\psset{yunit=0.1,xunit=0.0111}
\psaxes[arrows=<->,dx=0,Dx=720,dy=0,Dy=10,xunit=0.25](0,0)(-450,-7)(450,6)
\psplot[plotstyle=curve,arrows=<->,xunit=0.25]{-80}{80}{x sin x cos div neg 1.5 sub}
\end{pspicture}\\\hline
\end{tabular}
\end{center}
\end{table}
\par

\subsection*{Discovering the characteristics}
\subsection*{Gebied en Terrein}
\nopagebreak
Die gebied van $f(\theta )=a~tan(\theta )+q$ is al die waardes van $\theta $ sodat $cos~\theta $ nie gelyk is aan $0$nie. Ons het reeds gesien dat as $cos~\theta =0$, $tan~\theta =\frac{sin~\theta }{cos~\theta }$ ongedefinieerd is, want ons het deling deur nul. Ons weet dat $cos~\theta =0$ for all $\theta ={90}^{\circ }+{180}^{\circ }n$, waar $n$ waar n ’n heelgetal is. Dus die gebied van $f(\theta )=a~tan(\theta )+q$ is alle waardes van $\theta $, behalwe die waardes $\theta ={90}^{\circ }+{180}^{\circ }n$.\par 
Die terrein van $f(\theta )=a~tan~\theta +q$ is $\{f(\theta ):f(\theta )\in (-\infty ,\infty )\}$.\par 

\subsection*{As-afsnitte}
\nopagebreak
Die $y$-afsnit, ${y}_{int}$, van $f(\theta )=a~tan~x+q$ is slegs die waarde van $f(\theta )$ by $\theta ={0}^{\circ }$.\par 
\nopagebreak\noindent{}
\begin{equation*}
\begin{array}{ccc}\hfill {y}_{int}& =& f({0}^{\circ })\hfill \\
 & =& a~tan({0}^{\circ })+q\hfill \\
 & =& a(0)+q\hfill \\
 & =& q\hfill 
\end{array}
\end{equation*}

\subsection*{Asimptote}
\nopagebreak
Soos $\theta $ geleidelik naderkom aan ${90}^{\circ }$,sal $tan~\theta $ nader kom aan oneindig. Maar omdat $\theta $ ongedefinieërd is by ${90}^{\circ }$,kan $\theta $ slegs al nader kom aan ${90}^{\circ }$, maar nooit daarby uitkom nie. So, die $tan~\theta $ grafiek kom nader en nader aan
die lyn  $\theta ={90}^{\circ }$,sonder om dit ooit te ontmoet. Dus die lyn $\theta ={90}^{\circ }$ is ’n asimptoot van $tan~\theta $. $tan~\theta $ het ook
asimptote by $\theta ={90}^{\circ }+{180}^{\circ }n$, waar $n$ ’n heelgetal is.\par 

\begin{exercises}{ Trigonometriese Funksies}
 {

\begin{enumerate}[noitemsep, label=\textbf{\arabic*}. ] 
\item Deur jou kennis van die invloed van $a$ en $q$, te gebruik, skets elk van die volgende grafieke, sonder om ’n
tabel van waardes te gebruik, vir $\theta \in [{0}^{\circ };{360}^{\circ }]$
\begin{enumerate}[noitemsep, label=\textbf{\alph*}. ] 
\item $y=2~sin~\theta $
\item $y=-4~cos~\theta $
\item $y=-2~cos~\theta +1$
\item $y=sin~\theta -3$
\item $y=tan~\theta -2$\item $y=2~cos~\theta -1$
\end{enumerate}
 \item Gee die vergelykings van elk van die volgende grafieke:
\setcounter{subfigure}{0}
% \begin{minipage}{0.6\textwidth}
\begin{center}
\begin{pspicture}(-2.5,-2)(5,2)
\psset{yunit=0.25}
\psaxes[Dx=180, dx=2, Dy=2, dy=4]{<->}(0,0)(-2,-5.1)(4.5,5.1)
\psplot[xunit=0.0111, plotpoints=500, arrows=<->]{-90}{360}{x cos -4 mul }
\uput[d](4.7,0){$x$}
\uput[r](0,5.1){$y$}
% \rput(-2.5,2){a)}
\end{pspicture}

\begin{pspicture}(-0.3,-2)(5,2)
%\psgrid
\psset{yunit=0.25}
\psaxes[Dx=90, dx=1, Dy=2, dy=4]{<->}(0,0)(0,-5.1)(4.5,5.1)
\psplot[xunit=0.0111, plotpoints=500, arrows=->]{0}{360}{x sin 1 add 2 mul}
\uput[d](4.7,0){$x$}
\uput[r](0,5.1){$y$}
% \rput(-1.2,3){b)}
%\psplot[xunit=0.0111,plotpoints=500, arrows=<->]{-260}{-100}{x sin x cos div 2 mul 1 add}
%\psplot[xunit=0.0111,plotpoints=500, arrows=<->]{260}{100}{x sin x cos div 2 mul 
\end{pspicture}
\end{center}
% \end{minipage}
% \begin{minipage}{0.35\textwidth}
\begin{pspicture}(-2.2,-3)(2.2,3.2)
%\psgrid
\psset{yunit=0.2}
\psaxes[Dx=90, dx=1, Dy=5, dy=5]{<->}(0,0)(-2,-12)(2,12)
\psline[linestyle=dashed](-1,0)(-1,12.5)
\psline[linestyle=dashed](1,-12.5)(1,-3)
\psline[linestyle=dashed](-1,-12.5)(-1,-3)
\psline[linestyle=dashed](1,0)(1,12.5)
% \rput(-2,8){c)}
\psplot[xunit=0.0111, plotpoints=500, arrows=<->]{-75}{83}{x sin x cos div -2 mul 5 add}
\end{pspicture}
% \end{minipage}      \par 

\insertpracticeinfo{2}
}
\end{exercises}


\summary{VMdkf}

\begin{itemize}[noitemsep]
\item Kenmerke van funksies: 
\begin{itemize}[noitemsep]
\item Die gegewe of gekose $x$-waarde is bekend as die onafhanklike veranderlike want die waarde van x kan
vrylik gekies word. Die berekende $y$-waarde staan bekend as die afhanklike veranderlike aangesien
die waarde van y afhang van die gekose waarde van $x$
\item Die gebied van ’n relasie verband is die versameling van al die $x$ waardes waarvoor daar ten minste een $y$ waarde bestaan volgens die funksievoorskrif. Die terrein is die versameling van al die $y$-waardes, wat verkry kan word deur ten minste een van die $x$ waardes te gebruik.
\item Die afsnit is die punt waar die grafiek ’n as sny. Die $x$-afsnit(te) is die punt(e) waar die grafiek die $x$-as
sny en die $y$-afsnit(te) is die punt(e) waar die grafiek die $y$-as sny. 
\item Slegs vir grafieke van funksies met ’n hoogste mag van groter as 1. Daar is twee tipes draaipunte: ’n
minimum draaipunt en ’n maksimum draaipunt. ’n Minimum draaipunt is ’n punt op die grafiek waar
die grafiek ophou afneem in waarde en begin toeneem in waarde. ’n Maksimum draaipunt is ’n punt
op die grafiek waar die grafiek ophou toeneem in waarde en begin afneem in waarde. 
\item ’n Asimptoot is ’n reguitlyn of kurwe wat die grafiek van ’n funksie sal nader, maar nooit raak nie.
\item n Grafiek is kontinu as daar geen onderbreking in die grafiek is nie. 
\end{itemize}
\item 
Versamelingnotasie: ’n versameling van sekere x-waardes het die volgende notasie: \{$x$ : voorwaardes,
meer voorwaardes\}
\item 

Interval notasie: hier skryf ons ’n interval in die vorm ’laer hakie, laer getal, kommapunt, hoër getal, hoër
hakie’
\item  Jy moet die volgende funksies en hulle eienskappe ken:
    \begin{itemize}[noitemsep]
    \item Line\^re funksies van die vorm $y=ax+q$. 
    \item Paraboliese funksies van die vorm $y=a{x}^{2}+q$.
    \item Hiperboliese funksies van die vorm $y=\frac{a}{x}+q$. 
    \item Ekosponensi\"ele funksies van die vorm $y=a{b}^{(x)}+q$. 
    \item Trigonometriese funksies van die vorm 
    \begin{align*} 
    y&=a\sin\theta + q \\
    y&=a\cos\theta +q \\
    y&=a\tan\theta +q \\
    \end{align*}
    \end{itemize}
\end{itemize}

\begin{eocexercises}{}
\nopagebreak
\begin{enumerate}[noitemsep, label=\textbf{\arabic*}. ] 
\item Skets die grafieke van die volgende: 
    \begin{enumerate}[noitemsep, label=\textbf{\alph*}. ] 
    \item $y=2x+4$ 
    \item $y-x=0$ 
    \item $y=-\frac{1}{2}x+2$
    \end{enumerate}
\item Skets die volgende funksies: 
    \begin{enumerate}[noitemsep, label=\textbf{\alph*}. ] 
    \item $y={x}^{2}+3$ 
    \item $y=\frac{1}{2}{x}^{2}+4$
    \item $y=2{x}^{2}-4$
    \end{enumerate}
\item Skets die volgende funksies en identifeseer die asimptote: 
    \begin{enumerate}[noitemsep, label=\textbf{\alph*}. ] 
    \item $y={3}^{x}+2$ 
    \item $y=-4.{2}^{x}+1$ 
    \item $y=(\frac{1}{3})^x-2$ 
    \end{enumerate}
\item Skets die volgende funksies en identifeseer die asimptote: 
    \begin{enumerate}[noitemsep, label=\textbf{\alph*}. ] 
    \item $y=\frac{3}{x}+4$ 
    \item $y=\frac{1}{x}$ 
    \item $y=\frac{2}{x}-2$ 
    \end{enumerate}
\item Bepaal of die volgende bewerings waar of vals is. As 'n bewering vals is, gee redes hoekom
    \begin{enumerate}[noitemsep, label=\textbf{\alph*}. ] 
    \item  Die gegewe of gekose y-waarde staan bekend as die afhanklike veranderlike.
    \item 'n Grafiek wat geen onderbrekings het nie, word kongruent genoem
    \item Funksies van die vorm $y=ax+q$ is reguitlyne.
    \item Funksies van die vorm $y=\frac{a}{x}+q$ is eksponensiële funksies.
    \item 'n Asimptoot is 'n reguit of gekromdelyn wat'n grafiek ten minste een keer sny. 
    \item Gegee die funksie in die vorm $y=ax+q$ , Vind die $y$-afsnit deur $x=0$ te stel en los op vir $y$.
    \end{enumerate}
\item Gegee die funksies $f(x)=-2{x}^{2}-18$ en $g(x)=-2x+6$
    \begin{enumerate}[noitemsep, label=\textbf{\alph*}. ] 
    \item Skets $f$ en $g$ op dieselfde assestelsel.
    \item Bereken die snypunte van $f$ en $g$.
    \item Gebruik dan nou grafieke van nommer $6$ en hulle snypunte om vir $x$ op te los wanneer:
	\begin{enumerate}[noitemsep, label=\textbf{\roman*}. ] 
	\item $f(x)>0$
	\item $\frac{f(x)}{g(x)}\leq 0$
	\end{enumerate}
    \item Gee die vergelyking van die refleksie van $f$ in die $x$-as.
    \end{enumerate}
\item Nadat ’n bal laat val word, is die hoogte wat die bal terugbons elke keer minder. Die vergelyking $y=5{(0,8)}^{x}$ toon die verwantskap tussen
 $x$,die nommer van die bons, en $y$, die hoogte van die bons vir ’n
spesifieke bal. Wat is die benaderde hoogte van die vyfde bons tot die naaste tiende van ’n eenheid?\newline
\item Mark he $15$ muntstukke in R$~5$ en R$~2$ stukke. Hy het $3$ meer R$~2$ stukke as R$~5$ stukke. Hy het ‘n stelsel
van vergelykings opgestel om die situasie te toon, waar $x$  die aantal R$~5$ stukke voorstel en $y$ die
aantal R$~2$ stukke. Hy het vervolgens die probleem grafies opgelos.
    \begin{enumerate}[noitemsep, label=\textbf{\alph*}. ] 
    \item Skryf die sisteem van vergelykings neer.
    \item Skets die grafieke op dieselfde assestelsel.
    \item Wat is die oplossing?
    \end{enumerate}

\end{enumerate}

\insertpracticeinfo{8}
\end{eocexercises}