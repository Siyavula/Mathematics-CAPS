\chapter{Inleiding}

\section{Ondersteuning vir opvoeders}
Die onderrig van wetenskap behels meer as fisika, chemie en wiskunde...Dit leer jou om te dink en probleme op te los; al twee waardevolle vaardighede wat jy in alle fasette van die lewe kan toepas. Dit is belangrik dat die nuwe generasie hierdie vaardighede aanleer, veral in die huidige globale omgewing waar metodologie, tegnologie en hulpmiddels vinnig ontwikkel. Die onderwys behoort by hierdie ontwikkeling te baat. In ons eenvoudige model is daar drie maniere hoe tegnologie jou onderrig en onderrigmillieu kan beïnvloed.


\subsubsection{Eerste Manier: Samewerking/Medewerking van Onderwysers}
Daar is baie hulpmiddels wat onderwysers kan help om meer effektief saam te werk. Ons weet dat leergemeenskappe waar daar saam geoefen word, die ideale geleentheid bied vir die verfyning van metodologie, inhoud en kennis asook goeie ondersteuning bied aan onderwysers. Een van die grootste uitdagings wat sulke leergemeenskappe in die gesig staar, is die tyd en spasie om voldoende byeenkomste te hê om optimale gemeenskappe te bou en oefeninge, inhoud en leermetodes effektief uit te ruil. Tegnologie is die oplossing – dit transendeer ruimte en tyd. Ons kan nou oor groot afstande (ruimte transendeer) en wanneer dit vir alle partye geleë is (tyd transendeer) virtueel saamwerk (deur e-pos, selfoon, aanlyn ens.)\par


Die doel van ons handboek is daarom heroorweeg met inagneming van inligting wat op die Connexions webblad beskikbaar is \unerline{(http://cnx.org/lenses/fhsst)}. Die inhoud op hierdie webblad is maklik toeganklik en aanpasbaar aangesien dit onder oop lisensie is, in ‘n oop formaat gestoor kan word, gebaseer is op ‘n oop standaard en ‘n oop bronne beleid, en ook gratis is, wat almal toelaat om hul eie boeke te kan vervaardig. Die inhoud op Connexions is beskikbaar onder oop kopiereg lisensie – CC-BU. Hierdie Creative Commons By Attribution License stel ander in staat om inhoud regmatig te versprei, te verander, op te skerp, en om jou eie werk te skep deur dié inhoud as basis te gebruik. Die inoud kan, selfs kommersiële gebruik word, solank die oorspronklike skrywer krediet kry vir hul oorspronklike skepping. Dit beteken dat leerders en opvoeders wettig die inligting kan aflaai, kopieer, deel en versprei sonder vergoeding. Dit bied ook aan opvoeders die vryheid om die inligting te proeflees, te verander, te vertaal en te kontekstualiseer vir hul eie behoeftes.
 \par

Connexion is ‘n hulpmiddel waar individuele inligting kan deel, maar belangriker help dit gemeenskappe om te vorm rondom die samewerkende, ontwikkeling van bronne aanlyn. Jou gemeenskap opvoeders kan daarom:
\begin{itemize}[noitemsep]
\item werkswinkels aanlyn skep oor die inhoud;
\item eie kopieë van die inhoud maak;
\item sekere dele aanpas om hul behoeftes te bevredig;
\item dele van hul eie kopie aanpas
\item hul eie inhoud bysit of bestaande inhoud met hul eie werk vervang;
\item ander inhoud wat gedeel is, in hul werk gebruik;
\item as ‘n gemeenskap hul eie notas/handboek/kursusmateriaal skep.
\end{itemize}
Opvoeders wil gereeld assesseringstake deel aangesien dit hul werkslading verminder, verskeidenheid vergroot en kwaliteit bevorder. Alle oplossings vir die oefeninge in hierdie boek is op ons gratis en oop databasis vir assessering, Monassis \underline{(www.monassis.com)}, opgelaai. Elke oefening het ‘n shortcode wat gekoppel is aan sy oplossing op Monassis. Om by hierdie oplossing uit te kom, gaan eenvoudig na \underline{www.everythingmaths.com}, voer die shortcode in, en jy sal by die oplossing op Monassis uitkom.

Monassis is soortgelyk aan Connexion, maar is meer gefokus op die deel van assesseringstake. Monassis bevat ‘n versameling toetse en eksamens (met oplossings), wat onderwysers vrylik deel met mekaar. Onderwysers kan ook inligting op die databasis soek vir spesifieke vakke of grade en kan ook relevante items by ‘n toets sit. Die webblad genereer outomaties ‘n toets- of eksamenvraestel, met ‘n memo, wat beskikbaar is om af te laai. \par

Al die oefeninge aan die einde van die verskillende hoofstukke (met hul antwoorde), word op die web gelaai as deel van hierdie oop databasis. Sodoende kan die groter gemeenskap van opvoeders in Suid-Afrika toegang kry tot ‘n wye verskeidenheid inhoud wat hulle in toetse en eksamens kan gebruik. Meer inligting oor die gebruik van Monassis as hulpmiddel vir samewerking is vervat in die afdeling oor Monassis.\par

\subsubsection{Tweede Manier: Klaskamerbetrokkenheid}
Ten spyte van ‘n indrukwekkende verskeidenheid media wat as oop opvoedkundige hulpbronne gratis aanlyn beskikbaar is (soos videos, simulasies, oefeninge en aanbiedinge), gebruik slegs ‘n handjie vol opvoeders hierdie hulpbronne. Ons ondersoek het bewys dat die oorweldigende hoeveelheid, oorwegend internasionale konteks, en die feit dat dit moeilik is om dit met ons plaaslike kurrikulum se aanslag te vereenselwig, onderwysers afskrik. Hierdie hulpbronne kan, as hulle reg gebruik word, die leeromgewing soveel meer innemend/interessant maak.\par1b


Daar is soveel beter maniere om inligting lewe te gee deur verskillende aanbiedinge as om net die swartbord te gebruik. Dit bied jou die geleentheid om:
\begin{itemize}[noitemsep]
\item meer grafiese uitbeeldings van die inhoud te skep;
\item die tydsberekening van aanbiedinge meer effektief te beheer;
\item te verseker dat leerders die les onthou as dit goed uiteengesit word;
\item later notas by die aanbieding te voeg;
\item kern assesseringskonsepte vooraf vas te lê om bespreking aan te moedig; en 
\item ander media soos videos vas te lê
\end{itemize}

Die gebruik van videos is bevind om potensiaal innemend en effektief te wees. Dit bied die geleentheid om:
\begin{itemize}[noitemsep]
\item ‘n alternatiewe verduideliking te bied;
\item wanbegrippe aan te spreek sonder om ‘n individueel uit te sonder; en
\item ‘n omgewing of eksperiment te wys wat nie in die klas nageboots kan word nie, omdat dit te duur of te gevaarlik is.
\end{itemize}


Simulasies is ook handig en kan leerders toelaat om:
\begin{itemize}[noitemsep]
\item meer vryheid te hê om te ondersoek, eerder as net vaste en beperkte eksperimente en prosesse te dupliseer;
\item duur of gevaarlike omgewings meer effektief te ondersoek; en 
\item oimplisiete wanbegrippe te oorkom.
\end{itemize}
Ons het reeds ‘n versameling mediabronne vasgelê by die relevante afdeling, soos videos, aanbiedinge, simulasies en skakels, in die aanlyn weergawe van Everything Maths en Everything Science. Dit bied onderwysers nie alleenlik ‘n verskeidenheid van plaaslike, relevante en kurrikulumbaseerde bronne nie, maar hierdie bronne word ook sorteer volgens graad en afdeling. Skakels na aanlynbronne is opgeteken in die gedrukte- en PDF-weergawes van hierdie boek. Die bronne word dus toergidse of betroubare rigtingwysers na die wêreld van aanlyn mediabronne.\par


\subsubsection{Derde Manier: Buite die Klaskamer}
Die internet bied baie geleentheid vir selfstudie en deelneming wat nooit van tevore moontlik was nie. Daar is reuse onafhanklike video-argiewe soos die Khan Academy wat die meeste Wiskundebeginsels van Graad 1 – 12 asook daardie Wetenskaponderwerpe wat nodig is vir die FET bevat. Hierdie videos, as hulle nie in die klas gebruik word nie, bied ook geleenthede vir die leerder om:
\begin{itemize}[noitemsep]
\item inhoud self op te soek;
\item vooruit te werk;
\item onafhanklik te hersien en hul basis te versterk; en 
\item ‘n onderwerp te ondersoek om te kyk of hulle dit interessant vind.
\end{itemize}
Daar is ook baie geleenthede vir leerders om deel te neem aan aanlyn wetenskapprojekte (kyk na die afdeling oor citizen cyberscience “Op die web, kan enige een ‘n wetenskaplike wees”). Daar is nie net simulasies of tutoriale nie, maar regte wetenskap ook, sodat leerders:
\begin{itemize}[noitemsep]
\item kan waardeer hoe wetenskap verander;
\item maklik en veilig vakke kan ondersoek wat hulle nooit voor universiteit mee sou kennis gemaak het nie;
\item bydra tot regte wetenskap (regte internasionale baanbreker wetenskaplike programme);
\item die geleentheid het om regte ontdekkings te maak, selfs van hul rekenaarlaboratoirum by die skool; en
\item aktiewe rolmodelle in die wetenskap kan vind.
\end{itemize}
In ons boek het ons geleenthede vasgelê wat onderwysers sal help en bronne beskikbaar gestel waarby die leerders sal baat sonder om oorweldig te word deur al die inhoud wat aanlyn beskikbaar is.

\subsubsection{Verklarings (annotasies) van die boek}
Indien jy enige kommentaar, opinies of voorstelle oor die boeke het, kan jy \underline{www.everythingmaths.co.za} besoek en dit as teks vasvang wanneer jy as opvoeder ingeteken is. Hierdie kommentaar kan enige iets wees van raad en idees wat jy met ander onderwyser deel tot besprekings oor hoe om konsepte beter in die klas te verduidelik wees. Jy kan ook hier ‘n nota maak oor enige foute wat jy in die boek opgetel het, en ons sal dit betyds vir die volgende druk regstel.\par

\section{Op die web, kan enige een ‘n wetenskaplike wees}
Het jy geweet dat jy proteïen molekules, jag kan vou vir 'n nuwe planete rondom die verre sonne of simuleer hoe' n mens malaria in Afrika versprei, almal van 'n gewone PC of laptop met die internet verbind? En jy hoef nie aan 'n gesertifiseerde wetenskaplike om dit te doen. Om die waarheid te sê sommige van die mees talentvolle bydraers tieners. Die rede hoekom dit moontlik is, is dat wetenskaplikes leer hoe om eenvoudige wetenskaplike take te omskep in 'n mededingende aanlyn speletjies. \par

Dit is die verhaal van hoe 'n eenvoudige idee van die deel van wetenskaplike uitdagings op die web het in' n wêreldwye tendens, genaamd burger cyberscience. En hoe kan jy 'n wetenskaplike op die web ook. 

\subsubsection{Op soek na Little Green Men }
% A long time ago, in 1999, when the World Wide Web was barely ten years old and no one had heard of Google, Facebook or Twitter, a researcher at the University of California at Berkeley, David Anderson, launched an online project called SETI@home. SETI stands for Search for Extraterrestrial Intelligence. Looking for life in outer space.\par
% 
% Although this sounds like science fiction, it is a real and quite reasonable scientific project. The idea is simple enough. If there are aliens out there on other planets, and they are as smart or even smarter than us, then they almost certainly have invented the radio already. So if we listen very carefully for radio signals from outer space, we may pick up the faint signals of intelligent life.\par
% 
% Exactly what radio broadcasts aliens would produce is a matter of some debate. But the idea is that if they do, it would sound quite different from the normal hiss of background radio noise produced by stars and galaxies. So if you search long enough and hard enough, maybe you’ll find a sign of life. 
% \par
% It was clear to David and his colleagues that the search was going to require a lot of computers. More than scientists could afford. So he wrote a simple computer program which broke the problem down into smaller parts, sending bits of radio data collected by a giant radio-telescope to volunteers around the world. The volunteers agreed to download a programme onto their home computers that would sift through the bit of data they received, looking for signals of life, and send back a short summary of the result to a central server in California. \par
% 
% The biggest surprise of this project was not that they discovered a message from outer space. In fact, after over a decade of searching, no sign of extraterrestrial life has been found, although there are still vast regions of space that have not been looked at. The biggest surprise was the number of people willing to help such an endeavour. Over a million people have downloaded the software, making the total computing power of SETI@home rival that of even the biggest supercomputers in the world.
% 
% David was deeply impressed by the enthusiasm of people to help this project. And he realized that searching for aliens was probably not the only task that people would be willing to help with by using the spare time on their computers. So he set about building a software platform that would allow many other scientists to set up similar projects. You can read more about this platform, called BOINC, and the many different kinds of volunteer computing projects it supports today, at \underline{http://boinc.berkeley.edu/}. \par
% 
% There’s something for everyone, from searching for new prime numbers (PrimeGrid) to simulating the future of the Earth’s climate (ClimatePrediction.net). One of the projects, MalariaControl.net, involved researchers from the University of Cape Town as well as from universities in Mali and Senegal. \par
% 
% The other neat feature of BOINC is that it lets people who share a common interest in a scientific topic share their passion, and learn from each other. BOINC even supports teams – groups of people who put their computer power together, in a virtual way on the Web, to get a higher score than their rivals. So BOINC is a bit like Facebook and World of Warcraft combined – part social network, part online multiplayer game.\par
% 
% Here’s a thought: spend some time searching around BOINC for a project you’d like to participate in, or tell your class about. 
N lang tyd gelede, in 1999, toe die World Wide Web skaars tien jaar oud was en niemand het gehoor van Google, Facebook of Twitter,' n navorser by die Universiteit van Kalifornië in Berkeley, David Anderson, van stapel gestuur om 'n aanlyn projek SETI@home. SETI staan vir “Search for Extraterrestrial Intelligence”. Op soek na die lewe in die buitenste ruimte. 
Alhoewel dit klink soos die wetenskap fiksie, is dit 'n werklike en baie redelike wetenskaplike projek. Die idee is eenvoudig genoeg. Indien daar vreemdelinge wat daar is op ander planete, en hulle is so slim of selfs slimmer as ons, dan is hulle byna seker uitgevind het die radio. So as ons luister baie versigtig vir die radio seine uit die buitenste ruimte, kan ons haal die flou seine van intelligente lewe. \par
Presies wat radio-uitsendings vreemdelinge sal produseer, is 'n kwessie van' n debat. Maar die idee is dat as hulle dit doen, sou dit klink heel anders as die normale spot van die agtergrond radio geraas geproduseer deur die sterre en sterrestelsels. Dus as jy lank genoeg soek en hard genoeg nie, miskien sal jy vind 'n teken van die lewe. \par
Dit was duidelik aan Dawid en sy kollegas dat die soektog gaan 'n klomp van rekenaars te vereis. Meer as wetenskaplikes kon bekostig. So het hy geskryf het 'n eenvoudige program op die rekenaar wat die probleem gebreek af in kleiner dele, die stuur van' n stukkies van die radio data wat ingesamel is deur 'n reuse-radio-teleskoop vrywilligers oor die hele wêreld. Die vrywilligers het ingestem om 'n program af te laai op hul eie rekenaars nie wat jou sal sif deur die bietjie van die data wat hulle ontvang het het, op soek na die seine van die lewe, en stuur' n kort opsomming van die resultate terug na 'n sentrale bediener in Kalifornië. \par
Die grootste verrassing van hierdie projek is nie dat hulle 'n boodskap uit die buitenste ruimte ontdek. Om die waarheid te sê, na meer as 'n dekade van soek, het geen teken van buitenaards lewe is gevind, maar daar is steeds groot gebiede van die ruimte wat nog nie gekyk is. Die grootste verrassing was die aantal mense wat bereid is om so 'n poging om te help. Meer as 'n miljoen mense het die sagteware afgelaai, die maak van die totale rekenaar krag van SETI@home mededinger van selfs die grootste super in die wêreld. \par
Dawid was diep beïndruk deur die entoesiasme van die mense om hierdie projek te help. En hy het besef dat die soek vir vreemdelinge was waarskynlik nie die enigste taak wat mense bereid sal wees om te help met die deur die gebruik van die vrye tyd op hul rekenaars. En hy het oor die bou van 'n sagteware platform wat baie ander wetenskaplikes sou toelaat dat soortgelyke projekte op te rig. Jy kan meer lees oor hierdie platform, genoem BOINC, en die baie verskillende soorte van vrywilliger-computing projekte ondersteun vandag by \underline{http://boinc.berkeley.edu/}. \par 
Daar is iets vir almal, van soek na nuwe priemgetalle (PrimeGrid) aan die simuleer van die toekoms van die aarde se klimaat (ClimatePrediction.net). Een van die projekte, MalariaControl.net betrokke navorsers van die Universiteit van Kaapstad, sowel as van universiteite in Mali en Senegal. \par

Die ander netjiese kenmerk van BOINC is dat dit kan mense wat 'n gemeenskaplike belang in' n wetenskaplike onderwerp deel deel hul passie, en leer van mekaar. BOINC ondersteun selfs spanne - groepe van mense wat hul rekenaar krag saam te stel, in 'n virtuele wyse op die web,' n hoër telling as hul teenstanders te kry. So BOINC is 'n bietjie soos Facebook en World of Warcraft gekombineer - deel sosiale netwerk, wat deel uitmaak online multiplayer game. \par
Hier is 'n gedagte: spandeer tyd soek om BOINC vir' n projek wat jy wil deel te neem aan, of vertel vir die klas. 


\subsubsection{Jy is ook 'n rekenaar }
% Before computers were machines, they were people. Vast rooms full of hundreds of government employees used to calculate the sort of mathematical tables that a laptop can produce nowadays in a fraction of a second. They used to do those calculations laboriously, by hand. And because it was easy to make mistakes, a lot of the effort was involved in double-checking the work done by others. \par
% 
% Well, that was a long time ago. Since electronic computers emerged over 50 years ago, there has been no need to assemble large groups of humans to do boring, repetitive mathematical tasks. Silicon chips can solve those problems today far faster and more accurately. But there are still some mathematical problems where the human brain excels.\par
% 
% Volunteer computing is a good name for what BOINC does: it enables volunteers to contribute computing power of their PCs and laptops. But in recent years, a new trend has emerged in citizen cyberscience that is best described as volunteer thinking. Here the computers are replaced by brains, connected via the Web through an interface called eyes. Because for some complex problems – especially those that involve recognizing complex patterns or three-dimensional objects – the human brain is still a lot quicker and more accurate than a computer.\par
% 
% Volunteer thinking projects come in many shapes and sizes. For example, you can help to classify millions of images of distant galaxies (GalaxyZoo), or digitize hand-written information associated with museum archive data of various plant species (Herbaria@home).  This is laborious work, which if left to experts would take years or decades to complete. But thanks to the Web, it’s possible to distribute images so that hundreds of thousands of people can contribute to the search. \par
% 
% Not only is there strength in numbers, there is accuracy, too. Because by using a technique called validation – which does the same sort of double-checking that used to be done by humans making mathematical tables – it is possible to practically eliminate the effects of human error. This is true even though each volunteer may make quite a few mistakes. So projects like Planet Hunters have already helped astronomers pinpoint new planets circling distant stars. The game FoldIt invites people to compete in folding protein molecules via a simple mouse-driven interface. By finding the most likely way a protein will fold, volunteers can help understand illnesses like Alzheimer’s disease, that depend on how proteins fold. \par
% 
% Volunteer thinking is exciting. But perhaps even more ambitious is the emerging idea of volunteer sensing: using  your laptop or even your mobile phone to collect data – sounds, images, text you type in – from any point on the planet, helping scientists to create global networks of sensors that can pick up the first signs of an outbreak of a new disease (EpiCollect), or the initial tremors associated with an earthquake (QuakeCatcher.net), or the noise levels around a new airport (NoiseTube).\par
% 
% There are about a billion PCs and laptops on the planet, but already 5 billion mobile phones. The rapid advance of computing technology, where the power of a ten-year old PC can easily be packed into a smart phone today, means that citizen cyberscience has a bright future in mobile phones. And this means that more and more of the world’s population can be part of citizen cyberscience projects. Today there are probably a few million participants in a few hundred citizen cyberscience initiatives. But there will soon be seven billion brains on the planet. That is a lot of potential citizen cyberscientists. 
Voordat rekenaars masjiene was, was hulle mense. 'n Groot kamers vol van honderde van staatsamptenare wat gebruik word om die soort van wiskundige tabelle dat' n laptop deesdae in 'n breukdeel van' n sekonde kan produseer te bereken. Hulle gebruik om daardie berekeninge te moeisaam te doen, met die hand. En omdat dit maklik om foute te maak, was 'n groot deel van die moeite wat betrokke is in die dubbele kontrole van die werk wat gedoen is deur ander is. \par
Wel, dit was 'n lang tyd gelede. Omdat elektroniese rekenaars oor 50 jaar gelede na vore getree het, is daar nie nodig om groot groepe mense te vergader vervelig, herhalend wiskundige take te doen. Silicon skyfies kan hierdie probleme op te los vandag baie vinniger en meer akkuraat. Maar daar is nog 'n paar wiskundige probleme waar die menslike brein blink. \par
Volunteer-rekenaars is 'n goeie naam vir wat BOINC nie: dit maak dit moontlik vrywilligers rekenkracht van hul rekenaars en skootrekenaars te dra. Maar in die afgelope jaar, het 'n nuwe tendens na vore gekom in die burger cyberscience wat is die beste beskryf word as vrywilliger denke. Hier is die rekenaars is vervang deur die brein, wat verbind is via die web deur middel van 'n koppelvlak genoem oë. Want vir 'n paar ingewikkelde probleme - veral dié wat die volgende behels die erkenning van komplekse patrone of drie-dimensionele voorwerpe - die menslike brein is nog steeds' n baie vinniger en meer akkuraat as 'n rekenaar. \par
Volunteer denke projekte kom in baie vorms en groottes. Byvoorbeeld, kan jy help om miljoene van beelde van verre sterrestelsels (GalaxyZoo), of fotobewerking hand-geskrewe inligting wat verband hou met die museum argief data van verskillende plantspesies (Herbaria@home) te klassifiseer. Dit is 'n moeisame werk, wat as links kenners sou neem jare of dekades te voltooi. Maar danksy die web, dit is moontlik om beelde te versprei, sodat die honderde duisende mense kan bydra tot die soektog. \par
Nie alleen is daar krag in getalle, is daar akkuraatheid, ook. Want deur die gebruik van 'n tegniek genaamd validation - wat nie dieselfde soort dubbele kontrole wat gebruik word gedoen deur mense wat die maak van wiskundige tabelle - is dit moontlik om prakties die gevolge van menslike foute uit te skakel. Dit is waar, selfs al is elke vrywilliger kan nogal 'n paar foute maak. So projekte soos Planet Jagters het reeds gehelp om sterrekundiges uitwys nuwe planete omkring verre sterre. Die spel FoldIt nooi mense om mee te ding in die vou proteïen molekules deur 'n eenvoudige muis-gedrewe koppelvlak. Deur die bevinding van die mees waarskynlike manier om 'n proteïen sal vou, kan vrywilligers help met die verstaan van siektes soos Alzheimer se siekte, wat afhanklik is van hoe proteïen vou. \par
Volunteer denke is opwindend. Maar miskien selfs meer ambisieus is, is die nuwe idee van 'n vrywilliger sensing: jou laptop of selfs jou selfoon gebruik om data in te samel - klanke, beelde, teks wat jy intik - vanaf enige punt op die planeet, help wetenskaplikes om globale netwerke van sensors wat te skep kan optel die eerste tekens van 'n uitbraak van' n nuwe siekte (EpiCollect), of die aanvanklike bewing wat verband hou met 'n aardbewing (QuakeCatcher.net), of die geraas vlakke in' n nuwe lughawe (NoiseTube). \par

Daar is omtrent 'n miljard PC's en skootrekenaars op die planeet nie, maar reeds 5 miljard selfone. Die vinnige ontwikkeling van die rekenaar tegnologie, waar die krag van 'n tien-jaar-ou rekenaar kan maklik in' n slimfoon verpak vandag, beteken dat die burger cyberscience het 'n blink toekoms in selfone. En dit beteken dat meer en meer van die wêreld se bevolking kan deel wees van die burger cyberscience projekte. Vandag is daar waarskynlik 'n paar miljoen deelnemers in' n paar honderd burger cyberscience inisiatiewe. Maar daar sal binnekort sewe miljard brein op die planeet. Dit is 'n baie potensiaal burger cyberscientists. \par

Jy kan verken veel meer oor burger cyberscience op die web. Daar is 'n groot lys van alle soorte projekte, met kort opsommings van hul doelwitte te bereik, by \underline{http://distributedcomputing.info/}. BBC Radio 4 'n kort reeks oor die burger wetenskap \underline{http://www.bbc.co.uk/radio4/science/citizenscience.shtml}  en jy kan inteken op' n nuusbrief oor die nuutste tendense in hierdie gebied op \underline{http://scienceforcitizens.net/} . Die Burger Cyberscience Centre,\underline{www.citizencyberscience.net}  wat geborg word deur die Suid-Afrikaanse Shuttleworth Foundation, is die bevordering van burger cyberscience in Afrika en ander ontwikkelende streke. 


\section{Blog posts}
\subsubsection{Algemene blogs}
\begin{itemize}
\item Teachers Monthly - Education News and Resources
    \begin{itemize}[noitemsep]
      \item “We eat, breathe and live education! “
      \item “Perhaps the most remarkable yet overlooked aspect of the South African teaching community is its enthusiastic, passionate spirit. Every day, thousands of talented, hard-working teachers gain new insight from their work and come up with brilliant, inventive and exciting ideas. Teacher’s Monthly aims to bring teachers closer and help them share knowledge and resources.
      \item Our aim is twofold...
	    \begin{itemize}[noitemsep]
	      \item To keep South African teachers updated and informed.
	    \item To give teachers the opportunity to express their views and cultivate their interests.”
	    \end{itemize}
      \item \underline{http://www.teachersmonthly.com }
    \end{itemize}

\item Head Thoughts – Personal Reflections of a School Headmaster
    \begin{itemize}[noitemsep]
	\item blog by Arthur Preston
	\item “Arthur is currently the headmaster of a growing independent school in Worcester, in the Western Cape province of South Africa. His approach to primary education is progressive and is leading the school through an era of new development and change.”
\item \underline{http://headthoughts.co.za/ }
    \end{itemize}
\end{itemize}

\subsubsection{Wiskunde blogs}
\begin{itemize}

\item CEO: Circumspect Education Officer - Educating The Future
\begin{itemize} [noitemsep]
 \item blog by Robyn Clark
\item “Mathematics teacher and inspirer.”
\item \underline{http://clarkformaths.tumblr.com/ }
\end{itemize}

\item dy/dan - Be less helpful
\begin{itemize} [noitemsep]
\item blog by Dan Meyer
\item “I'm Dan Meyer. I taught high school math between 2004 and 2010 and I am currently studying at Stanford University on a doctoral fellowship. My specific interests include curriculum design (answering the question, "how we design the ideal learning experience for students?") and teacher education (answering the questions, "how do teachers learn?" and "how do we retain more teachers?" and "how do we teach teachers to teach?").”
\item \underline{http://blog.mrmeyer.com }
\end{itemize}

\item Without Geometry, Life is Pointless - Musings on Math, Education, Teaching, and Research

\begin{itemize}[noitemsep]
 \item blog by Avery
\item “I've been teaching some permutation (or is that combination?) of math and science to third through twelfth graders in private and public schools for 11 years. I'm also pursuing my EdD in education and will be both teaching and conducting research in my classroom this year.”
\item \underline{ http://mathteacherorstudent.blogspot.com/ }
\end{itemize}

\item Overthinking my teaching - The Mathematics I Encounter in Classrooms
\begin{itemize}[noitemsep]
\item blog by Christopher Danielson
\item “I think a lot about my math teaching. Perhaps too much. This is my outlet. I hope you find it interesting and that you’ll let me know how it’s going.”
\item \underline{http://christopherdanielson.wordpress.com}
\end{itemize}

\item A Recursive Process - Math Teacher Seeking Patterns
\begin{itemize}[noitemsep]
\item blog by Dan
\item “I am a High School math teacher in upstate NY. I currently teach Geometry, Computer Programming (Alice and Java), and two half year courses: Applied and Consumer Math. This year brings a new 21st century classroom (still not entirely sure what that entails) and a change over to standards based grades (#sbg).”
\item \underline{http://dandersod.wordpress.com }
\end{itemize}

\item Think Thank Thunk – Dealing with the Fear of Being a Boring Teacher 
\begin{itemize} [noitemsep]
\item blog by Shawn Cornally
\item “I am Mr. Cornally. I desperately want to be a good teacher. I teach Physics, Calculus, Programming, Geology, and Bioethics. Warning: I have problem with using colons. I proof read, albeit poorly.”
\item \underline{http://101studiostreet.com/wordpress/}
\end{itemize}
\end{itemize}

\section{Oorsig}
\subsubsection{Kurrikulum Oorsig}
% Before 1994 there existed a number of education departments and
% subsequent curriculum according to the segregation that was so
% evident during the apartheid years.
Voor 1994 het daar 'n aantal verskillende onderwysdepartemente en
kurrikula bestaan volgens die skeiding wat so duidelik was tydens die
apartheid era.
% As a result, the curriculum itself became one of the political icons
% of freedom or suppression.
As 'n gevolg het die kurrikulum self een van die politiese ikone van
vryheid of onderdrukking geword.
% Since then the government and political leaders have sought to try
% and develop one curriculum that is aligned with our national agenda
% of democratic freedom and equality for all, in fore-grounding the
% knowledge, skills and values our country believes our learners need
% to acquire and apply, in order to participate meaningfully in
% society as citizens of a free country.
Sedertdien het die regering en politieke leiers probeer om een
kurrikulum te ontwikkel, wat die nasionale agenda van demokratiese
vryheid en gelykheid ondesteun, deur die kennis, vaardighede en
waardes wat ons leerders moet leer and toepas op die voorgrond te
stel, sodat hulle op 'n betekenisvolle manier kan deelneem in die
samelewing as burgers van 'n vry land.
% The National Curriculum Statement (NCS) of Grades R – 12 (DBE, 2012)
% therefore serves the purposes of: 
Die Nasionale Kurrikulumverklaring (NKV) Graad R--12 (DBE, 2012) dien
dus volgende doelwitte:
\begin{itemize}
\item
% equipping learners, irrespective of their socio-economic background,
% race, gender, physical ability or intellectual ability, with the
% knowledge, skills and values necessary for self-fulfilment, and
% meaningful participation in society as citizens of a free country;
  om leerders toe te rus met die kennis, vaardighede end waardes
  benodig vir selfverwesenliking en betekenisvolle deelname in die
  samelewing as burgers van 'n vry land, ongeag hulle sosio-ekonomiese
  agtergrond, ras, geslag, fisiese of intellektuele vermo\"{e};
\item
% providing access to higher education; 
  om toegang to ho\"{e}r onderrig te verskaf;
\item
% facilitating the transition of learners from education institutions
% to the workplace; and
  om die oorgang van leerders vanaf onderwysinstellings na die
  werkplek te fasiliteer; en
\item
% providing employers with a sufficient profile of a learner’s
% competencies.
  om werkgewers met 'n voldoende profiel van leerdersbevoegdhede te
  verskaf.
\end{itemize}
% Although elevated to the status of political icon, the curriculum
% remains a tool that requires the skill of an educator in
% interpreting and operationalising this tool within the classroom.
Alhoewel dit verhef is tot die status van 'n politiese ikoon, bly die
kurrikulum 'n instrument. Die vaardighede van 'n onderwyser word
benodig om hierdie istrument te interpreteer en operasionaliseer in
die klaskamer.
% The curriculum itself cannot accomplish the purposes outlined above
% without the community of curriculum specialists, material
% developers, educators and assessors contributing to and supporting
% the process, of the intended curriculum becoming the implemented
% curriculum.
Die kurrikulum self kan nie die doelwitte hierbo gelys bereik sonder
dat die gemeenskap van kurrikulumspesialiste, ontwikkelaars van
onderwysmateriaal, onderwysers en assessore die prosess, om die
voorgenome kurrikulum die ge\"{i}mplementeerde kurrikulum te maak,
ondersteun en daartoe bydra nie.
% A curriculum can succeed or fail, depending on its implementation,
% despite its intended principles or potential on paper.
'n Kurrikulum kan slaag of misluk, afhangende van die implementering
en ongeag die voorgenome beginsels of potensiaal op papier daarvan.
% It is therefore important that stakeholders of the curriculum are
% familiar with and aligned to the following principles that the NCS
% is based on:
Daarom is dit belangrik dat belanghebbendes van die kurrikulum
vertroud is en ooreenstem met die volgende beginsels waarop die NKV
gebaseer is:

\begin{table}[H]
\begin{center}
\begin{tabular}{|p{6.5cm}|p{6.5cm}|} \hline
\textbf{Beginsel} &
\textbf{Implementering} \\ \hline
Sosiale Transformasie &
Die regstelling van wanbalanse van die verlede.\par
Die verskaffing van gelyke geleenthede vir almal.\\ \hline
Aktiewe en Kritiese Leer &
Aanmoediging van 'n aktiewe en kritiese benadering tot leer.\par
Vermyding van oormatige onkritiese memorisering van gegewe waarhede.\\ \hline
Diepgaande Kennis en Vaardighede &
Leerders behaal minimum standaarde van kennis en vaardighede, soos
bepaal vir elke graad in elke vak. \\ \hline
Vordering &
Inhoud en konteks toon progressie van eenvoudig na kompleks. \\ \hline
Sosiale en Omgewingsgeregtigheid en Menseregte &
Praktyke soos in die Grondwet omskryf is, verweef in die onderrig en
leer van elk van die vakke. \\ \hline
Waardering vir Inheemse Kennissisteme &
Erken die ryk geskiedenis en erfenis van hierdie land. \\ \hline
Geloofwaardigheid, Gehalte en Doeltreffendheid &
Verskaffing van onderrig wat w\^{e}reldwyd vergelykbaar is i.t.v.\@ kwaliteit. \\ \hline
\end{tabular}
\end{center}
\end{table}

% This guide is intended to add value and insight to the existing
% National Curriculum for Grade 10 Mathematics, in line with its
% purposes and principles.
Hierdie gids is bedoel om waarde en insig toe te voeg tot die
bestaande Nasionale Kurrikulum vir Graad 10 Wiskunde, in lyn met die
doelwitte en beginsels daarvan.
% It is hoped that this will assist you as the educator in optimising
% the implementation of the intended curriculum.
Daar word gehoop dat dit u as die opvoeder sal help om die voorgenome
kurrikulum te optimeer en implementeer.

\subsubsection{Kurrikulumvereistes en doelwitte}
% The main objectives of the curriculum relate to the learners that
% emerge from our educational system.
Die belangrikste doelwitte van die kurrikulum hou verband met die
leerders wat uit ons opvoedkundestelsel kom.
% While educators are the most important stakeholders in the
% implementation of the intended curriculum, the quality of learner
% coming through this curriculum will be evidence of the actual
% attained curriculum from what was intended and then implemented.
Opvoeders is die belangrikste rolspelers in die uitvoering van die
voorgenome kurrikulum. Die kwaliteit van die leerder wat deur hierdie
stelsel beweeg, sal egter die bewys wees dat die kurrikulum soos wat
dit bedoel en ge\"{i}mplementeer is, ook sy doelwitte bereik het.

%These purposes and principles aim to produce learners that are able to: 
Hierdie doelwitte en beginsels beoog om leerders te produseer wat in
staat is:
\begin{itemize}[noitemsep]
\item
% identify and solve problems and make decisions using critical and
% creative thinking;
  om probleme te identifiseer en op te los en om besluite te neem deur
  kritiese en kreatiewe denke;
\item
% work effectively as individuals and with others as members of a team; 
  om doeltreffend te werk as individue en met ander as lede van 'n
  span;
\item
% organise and manage themselves and their activities responsibly and
% effectively; 
  om hulself en hul aktiwiteite verantwoordelik en doeltreffend te
  organiseer en bestuur;
\item
% collect, analyse, organise and critically evaluate information; 
  om inligting te versamel, te analiseer, te organiseer en krities te
  evalueer;
\item
% communicate effectively using visual, symbolic and/or language
% skills in various modes;
  om effektief te kommunikeer deur gebruik te maak van visuele,
  simboliese en/of taalvaardighede in verskillende vorme;
\item
% use science and technology effectively and critically showing
% responsibility towards the environment and the health of others; and
  om wetenskap en tegnologie doeltreffend en krities te gebruik met
  verantwoordelikheid teenoor die omgewing en die gesondheid van
  ander;
\item
% demonstrate an understanding of the world as a set of related
% systems by recognising that problem solving contexts do not exist in
% isolation.
  om begrip van die w\^{e}reld as 'n stel verwante stelsels te toon deur
  te herken dat die kontekste van probleme nie in isolasie bestaan
  nie.
\end{itemize}
% The above points can be summarised as an independent learner who can
% think critically and analytically, while also being able to work
% effectively with members of a team and identify and solve problems
% through effective decision making.
Die bogenoemde punte kan opgesom word as 'n onafhanklike leerder wat
krities en analities kan dink, in staat is om effektief met lede van
'n span te werk, en probleme kan identifiseer en oplos deur middel van
effektiewe besluitneming.
% This is also the outcome of what educational research terms the
% “reformed” approach rather than the “traditional” approach many
% educators are more accustomed to.
Dit is ook die uitkoms waarna binne opvoedkundige navorsing verwys
word as die ``hervormde'' benadering eerder as die
``tradisionele'' benadering waaraan baie opvoeders meer gewoond is.
% Traditional practices have their role and cannot be totally
% abandoned in favour of only reform practices.
Tradisionele praktyke het hul rol en kan nie heeltemal ten gunste van
hervormde praktyke daargelaat word nie.
% However, in order to produce more independent and mathematical
% thinkers, the reform ideology needs to be more embraced by educators
% within their instructional behaviour.
Maar, ten einde meer onafhanklike en wiskundige denkers te produseer,
moet die hervorming ideologie deur opvoeders ingeneem word in hul
optrede as onderwysers.
% Here is a table that can guide you to identify your dominant
% instructional practice and try to assist you in adjusting it (if
% necessary) to be more balanced and in line with the reform approach
% being suggested by the NCS.
Hier is 'n tabel wat kan help om u dominante instruksionele praktyk te
identifiseer en u probeer help om dit aan te pas (indien nodig), om
meer gebalanseerd en in lyn met die hervorming benadering, soos
voorgestel deur die NKV, te wees.

\begin{table}[H]
  \begin{center}
    \begin{tabular}{|p{3.5cm}|p{8.5cm}|} \hline 
&
\textbf{Tradisionele Versus Hervormde Praktyke} \\ \hline
Waardes &
\textbf{Tradisioneel} --- fokus op onderrigmateriaal, korrektheid van leerders se antwoorde en wiskundige geldigheid van metodes.\par
\textbf{Hervorm} --- patrone vind, konsepte verbind, wiskundig kommunikeer en probleemoplossing. \\ \hline
Onderrigmetodes &
\textbf{Tradisioneel} --- verklarend, oordrag van inligting, baie oefen en herhaling, stap vir stap bemeestering van algoritmes.\par
\textbf{Hervorm} --- Geleide ontdekkingsmetodes, eksplorasie, modellering. Ho\"{e} vlak van redenasie is sentraal. \\ \hline
Groepering van Leerders &
\textbf{Tradisioneel} --- oorheersend gelyksoortige groepering. \par
\textbf{Hervorm} --- oorheersend gemengde vermo\"{e}ns. \\ \hline
    \end{tabular}
  \end{center}
\end{table}

% The subject of mathematics, by the nature of the discipline, provides
% ample opportunities to meet the reformed objectives.
Die vak wiskunde verskaf uiter aard ruim geleentheid om te voldoen aan
die hervormde doelwitte.
% In doing so, the definition of mathematics needs to be understood
% and embraced by educators involved in the teaching and the learning
% of the subject.
Die definisie van wiskunde moet verstaan ​​en omhels moet word deur die
opvoeders betrokke by die onderrig en die leer van die vak.
% In research it has been well documented that, as educators, our
% conceptions of what mathematics is, has an influence on our approach
% to the teaching and learning of the subject.
In die navorsing is dit goed gedokumenteer dat ons opvattings oor wat
wiskunde is, 'n invloed het op ons benadering tot die onderrig en leer
van die vak.

% Three possible views of mathematics can be presented.
Drie sienings van wiskunde word hier aangebied.
% The instrumentalist view of mathematics assumes the stance that
% mathematics is an accumulation of facts, rules and skills that need
% to be used as a means to an end, without there necessarily being any
% relation between these components.
Die instrumentalistiese siening van wiskunde aanvaar die standpunt dat
wiskunde 'n versameling feite, re\"{e}ls en vaardighede is wat gebruik
word as 'n middel vir 'n doelwit, sonder dat daar noodwendig 'n
verband is tussen hierdie komponente.
% The Platonist view of mathematics sees the subject as a static but
% unified body of certain knowledge, in which mathematics is
% discovered rather than created.
Die Platonistiese siening van wiskunde is dat die vak 'n statiese,
maar verenigde liggaam van sekere kennis is, waarbinne wiskunde ontdek
word eerder as om geskep te word.
% The problem solving view of mathematics is a dynamic, continually
% expanding and evolving field of human creation and invention that is
% in itself a cultural product.
Die probleemoplossing siening van wiskunde is dat dit 'n dinamiese,
voortdurend ontwikkelende veld van menslike skepping en uitvinding is
wat op sigself 'n kulturele produk is.
% Thus mathematics is viewed as a process of enquiry, not a finished
% product.
Wiskunde word dus beskou as 'n proses van ondersoek, eerder as 'n
voltooide produk.
% The results remain constantly open to revision.
Die resultate bly voortdurend oop vir hersiening.
% It is suggested that a hierarchical order exists within these three
% views, placing the instrumentalist view at the lowest level and the
% problem solving view at the highest.
Een voorgestel is dat 'n hi\"{e}rargiese orde bestaan ​​binne hierdie
drie aansigte, met die instrumentalistiese siening op die laagste vlak
en die probleemoplossing siening op die hoogste vlak.
%% WHAT COMPLETE AND UTTER BULLSHIT! If you're in doubt about this,
%% just count the weasel words.

\subsubsection{Volgens die NKV:}
% Mathematics is the study of quantity, structure, space and change.
Wiskunde is die studie van hoeveelheid, struktuur, ruimte en
verandering.
% Mathematicians seek out patterns, formulate new conjectures, and
% establish axiomatic systems by rigorous deduction from appropriately
% chosen axioms and definitions.
Wiskundiges soek patrone, formuleer nuwe veronderstellings en vestig
aksiomatiese stelsels deur die streng deduktiewe afleiding vanaf
toepaslik gekose aksiomas en definisies.
% Mathematics is a distinctly human activity practised by all
% cultures, for thousands of years.
Wiskunde is 'n menslike aktiwiteit wat deur alle kulture beoefen is,
vir duisende jare reeds.
% Mathematical problem solving enables us to understand the world
% (physical, social and economic) around us, and, most of all, to
% teach us to think creatively.
Wiskundige probleemoplossing stel ons in staat om die w\^{e}reld
rondom ons (fisies, sosiaal en ekonomies) te verstaan en, belangrikste
van alles, om te leer om kreatief te dink.

% This corresponds well to the problem solving view of mathematics and
% may challenge some of our instrumentalist or Platonistic views of
% mathematics as a static body of knowledge of accumulated facts,
% rules and skills to be learnt and applied.
Dit stem goed ooreen met die probleemoplossing siening van wiskunde en
mag dalk sommige van ons instrumentalistiese of Platonistiese
sienings, as 'n statiese versameling van kennis, feite, re\"{e}ls en
vaardighede wat geleer en toegepas moet word, uitdaag.

% The NCS is trying to discourage such an approach and encourage
% mathematics educators to dynamically and creatively involve their
% learners as mathematicians engaged in a process of study,
% understanding, reasoning, problem solving and communicating
% mathematically.
Die NKV probeer om so 'n benadering te ontmoedig en moedig
wiskunde-onderwysers aan om op 'n dinamiese en kreatiewe manier hulle
leerders as wiskundiges te betrek by 'n proses van studie, begrip,
redenering, probleemoplossing en kommunikasie.

% Below is a check list that can guide you in actively designing your
% lessons in an attempt to embrace the definition of mathematics from
% the NCS and move towards a problem solving conception of the
% subject.
Hieronder is 'n lys wat u kan help om u lesse aktief te ontwerp in 'n
poging om die NKV definisie van wiskunde te omhels en om nader te
beweeg aan 'n probleemoplossing konsepsie van die onderwerp.
% Adopting such an approach to the teaching and learning of
% mathematics will in turn contribute to the intended curriculum being
% properly implemented and attained through the quality of learners
% coming out of the education system.
Die aanvaarding van so 'n benadering tot die onderrig en leer van
wiskunde sal op sy beurt bydra tot die implementering en realisering
van die voorgenome kurrikulum, in terme van die kwaliteit van die
leerders wat uit die onderwysstelsel kom.

\begin{table}[H]
  \begin{center}
    \begin{tabular}{|p{6.5cm}|p{6.5cm}|} \hline 
\textbf{Aktiwiteit} & \textbf{Voorbeeld} \\ \hline
Leerders neem deel aan die oplossing van kontekstuele probleme wat
verband hou met hul lewens en wat vereis dat hulle 'n probleem
interpreteer en dan 'n geskikte wiskundige oplossing te vind.
&
Leerders word gevra om uit te werk watter busdiens die goedkoopste is,
gegee die tariewe en die afstand wat hulle wil reis.
\\ \hline
Leerders raak betrokke by die oplos van probleme van 'n suiwer
wiskundige aard, wat ho\"{e}r-orde denke en die toepassing van kennis
(nie-standaard probleme) benodig.
&
Leerders word gevra om 'n grafiek te teken. Hulle het nog nie 'n
spesifieke tekentegniek (byvoorbeeld vir 'n parabool) geleer nie, maar
het geleer om die tabelmetode te gebruik om reguit lyne te teken.
\\ \hline
Leerders kry geleentheid om oor betekenis te redeneer.
&
Leerders bespreek hul begrip van konsepte en strategie\"{e} vir die
oplossing van probleme met mekaar en met die onderwyser.
\\ \hline
Leerders word geleer end gevra om situasies op verskeie ekwivalente
maniere te verteenwoordig (wiskundige modellering).
&
Leerders verteenwoordig dieselfde data met behulp van 'n grafiek, 'n
tabel en 'n formule om die data voor te stel.
\\ \hline
Leerders doen individueel wiskundige ondersoeke in die klas, gelei
deur die onderwyser waar nodig.
&
Elke leerder kry 'n wiskundige probleem (byvoorbeeld om die aantal
priemgetalle minder as 50 te vind) wat ondersoek moet word
en die oplossing neergeskryf moet word. Leerders werk onafhanklik.
\\ \hline
Leerders werk saam as 'n groep/span om ondersoek in te stel of 'n
wiskundige probleem op te los.
&
'n Groep word opdrag gegee om saam te werk aan 'n probleem wat vereis
dat hulle patrone in data ondersoek, om veronderstellings te maak en
'n ​​formule vir die patroon te vind.
\\ \hline
Leerders doen oefeninge om hulle kennis van konsepte
te konsolideer en verskeie vaardighede te bemeester.
&
Voltooiing van 'n oefening wat roetine prosedures benodig.
\\ \hline
Leerders kry geleenthede om die wisselwerking tussen verskillende
aspekte van wiskunde te sien en om te sien hoe die verskillende
uitkomste verwant is.
&
Terwyl leerders deur meetkunde probleme werk, word hulle aangemoedig
om gebruik te maak van algebra.
\\ \hline
Leerders word gevra om probleme vir hulle onderwyser en
klasmaats op te stel.
&
Leerders word gevra 'n algebra\"{i}ese woordsom op te stel
(waarvan hulle ook die oplossing ken), vir die persoon wat langs
hulle sit om op te los.
\\ \hline
    \end{tabular}
  \end{center}
\end{table}

\subsection{Oorsig van die onderwerpe}
Oorsig van onderwerpe en hulle relevansie.

\begin{table}[H]
  \begin{center} 
    \begin{tabular}{|p{8.5cm}|p{3.5cm}|} \hline
\textbf{1. Funksies --- line\^{e}r, kwadraties, eksponensi\"{e}el, rasioneel} &
\textbf{Relevansie}  \\ \hline  
Verwantskappe tussen veranderlikes in terme van die grafiese, verbale
en simboliese voorstellings van funksies (tabelle, grafieke, woorde en
formules).\par
Grafieke en veralgemeningsgevolge van parameters: vertikale
verskuiwings en skalering en refleksies om die x-as.\par
Probleemoplossing en grafiekwerk met betrekking tot voorgeskrewe funksies.
&
Funksies vorm 'n sentrale deel van leerders se wiskundige begrip en
redenasie-prosesse in algebra.\par
Dit is ook 'n uitstekende geleentheid vir kontekstuele wiskundige
modelleringvrae. \\ \hline
    \end{tabular}
  \end{center}
\end{table}

\begin{table}[H]
\begin{center} 
\begin{tabular}{|p{8.5cm}|p{3.5cm}|} \hline
\textbf{2. Getalpatrone, rye en reekse}&\textbf{Relevansie} \\ \hline  
Getalpatrone met konstante verskil.
&
Baie wiskunde wentel rondom die identifisering van patrone.
\\ \hline
 \end{tabular}
\end{center}
\end{table}

\begin{table}[H]
\begin{center} 
\begin{tabular}{|p{8.5cm}|p{3.5cm}|} \hline
\textbf{3. Finansies, groei en krimping}& \textbf{Relevansie} \\ \hline  
Gebruik eenvoudige en saamgestelde groeiformules.\par
Implikasies van die veranderende wisselkoerse.
&
Die wiskunde van finansies is baie relevant vir daaglikse en
langtermyn finansi\"{e}le besluite wat leerders sal moet maak vir
beleggings, lenings, spaar, begrip van wisselkoerse en die invloed
daarvan w\^{e}reldwyd.
\\ \hline

 \end{tabular}
\end{center}
\end{table}

\begin{table}[H]
\begin{center} 
\begin{tabular}{|p{8.5cm}|p{3.5cm}|} \hline
\textbf{4. Algebra}&\textbf{Relevansie}  \\ \hline  
Verstaan ​​dat re\"{e}le getalle kan word irrasionele of rasionele.\par
Vereenvoudig uitdrukkings deur gebruik te maak van die wette van eksponente vir rasionale eksponente.\par
Identifiseer en die omskep die verskillende vorme van rasionale getalle.\par
Werk met eenvoudige wortels wat nie rasioneel is nie.\par
Werk met die wette van heeltallige eksponente.\par
Bepall tussen watter twee heelgetalle 'n eenvoudige wortelvorm l\^{e}.\par
Rond re\"{e}le getal gepas af.\par
Manipuleer en vereenvoudig algebra\"{i}ese uitdrukkings (insluitend vermenigvuldiging en faktorisering).\par
Los line\^{e}re, kwadratiese, letterlike en eksponensi\"{e}le vergelykings op.\par
Los line\^{e}re ongelykhede in een en twee veranderlikes algebra\"{i}es en grafies op.\par
&
Algebra verskaf die grondslag vir wiskunde leerders om te beweeg van
numeriese berekeninge na veralgemeende operasies, vereenvoudiging
van uitdrukkings, oplos van vergelykings en gebruik van
grafieke en ongelykhede vir die oplossing van kontekstuele probleme.
\\ \hline

 \end{tabular}
\end{center}
\end{table}

\begin{table}[H]
 \begin{center} 
\begin{tabular}{|p{8.5cm}|p{3.5cm}|} \hline
\textbf{5. Differensiaalrekening}& \textbf{Relevansie}\\ \hline  
Ondersoek gemiddelde koers van verandering tussen twee onafhanklike
waardes van 'n funksie.
&
Die sentrale aspek van die tempo van verandering vir
differensiaalrekening is as 'n basis vir verdere begrip van grense,
gradi\"{e}nte en berekeninge en formules wat nodig is vir werk in die
ingenieurswese-velde, bv.. die ontwerp van paaie, br\^{u}e ens.
\\ \hline

 \end{tabular}
\end{center}
\end{table}

\begin{table}[H]
 \begin{center} 
\begin{tabular}{|p{8.5cm}|p{3.5cm}|} \hline
\textbf{6. Waarskynlikheid}& \textbf{Relevansie} \\ \hline  
Vergelyk relatiewe frekwensie en teoretiese waarskynlikheid\par
Gebruik Venn diagramme om waarskynlikheid probleme op te los.\par
Uitsluitlike en komplement\^{e}re gebeure.\par
Identiteite vir enige twee gebeure A en B.
&
Hierdie onderwerp is nuttig vir die ontwikkeling van goeie logiese
redenasievermo\"{e} en vir die opvoeding van leerders in terme van
werklike lewenskwessies soos dobbelary en die slaggate daarvan.
\\ \hline

 \end{tabular}
\end{center}
\end{table}


\begin{table}[H]
 \begin{center} 
\begin{tabular}{|p{8.5cm}|p{3.5cm}|} \hline
\textbf{7. Euklidiese Meetkunde en Meting}& \textbf{Relevansie}\\ \hline  
Ondersoek, vorm en probeer om veronderstellings oor die eienskappe van driehoeke, vierhoeke en ander veelhoeke te bewys.\par
Weerl\^{e} valse veronderstellings deur die gebruik van teen-voorbeelde.\par
Ondersoek alternatiewe definisies van verskillende veelhoeke.\par
Los probleme op met betrekking tot die oppervlakte en volume van soliede voorwerpe en kombinasies daarvan.
&
Die denkprosesse en wiskundige vaardighede met betrekking tot die
bewys van veronderstellings en ​​die identifisering van valse
veronderstellings is meer relevant as om die inhoud te studeer.
Die oppervlakte en volume in praktiese kontekste soos die ontwerp van
kombuise, die te\"{e}l en verf van kamers, die ontwerp van verpakking,
ens.\@ is relevant tot die huidige en toekomstige lewens var leerdeers.
\\ \hline

 \end{tabular}
\end{center}
\end{table}

\begin{table}[H]
 \begin{center} 
\begin{tabular}{|p{8.5cm}|p{3.5cm}|} \hline
\textbf{8. Trigonometrie}& \textbf{Relevansie}\\ \hline  

Definisies van trigonometriese funksies.\par
Lei waardes af vir spesiale hoeke.\par
Neem kennis van die name vir inverse funksies.\par
Los probleme op in 2 dimensies.\par
Brei definisies van basiese trigonometriese funksies uit na al vier kwadrante en ken grafieke van hierdie funksies.\par
Ondersoek en weet wat die gevolge van $a$ en $q$ op die grafieke van basiese trigonometriese funksies is.\par
&
Trigonometrie het verskeie gebruike in die samelewing, bv.\@ in
navigasie, musiek, geografie en die ontwerp en konstruksie van
geboue.
\\ \hline

 \end{tabular}
\end{center}
\end{table}

\begin{table}[H]
 \begin{center} 
\begin{tabular}{|p{8.5cm}|p{3.5cm}|} \hline
\textbf{9. Analitiese meetkunde}&  \textbf{Relevansie} \\ \hline  
Stel meetkundige figure op 'n Cartesiese ko\"{o}rdinaatstelsel voor.\par
Vir enige twee punte, lei af en pas toe die formule vir die berekening
van afstand en helling van 'n lynsegment en die ko\"{o}rdinate van die
middelpunt.
&
Hierdie afdeling verskaf 'n verdere toepassing vir leerders se
algebra\"{i}ese en trigonometriese interaksie met die Cartesiese
vlak. Kunstenaars en die ontwerp en uitleg industrie\"{e} maak dikwels
gebruik van die inhoud en denkprosesse van hierdie wiskundige onderwerp.
\\ \hline

 \end{tabular}
\end{center}
\end{table}

\begin{table}[H]
 \begin{center} 
\begin{tabular}{|p{8.5cm}|p{3.5cm}|} \hline
\textbf{10. Statistiek}& \textbf{Relevansie}\\ \hline  
Versamel, organiseer en interpreteer enkelveranderlike numeriese data
om gemiddeld, mediaan, modus, persentiele, kwartiele, desiele,
interkwartiel- en semi-interkwartielvariasiewydte te bepaal.\par
Identifiseer moontlike bronne van vooroordeel en foute in metings.
&
Mense word daagliks gekonfronteer met die interpretasie van data wat
deur die media verskaf word. Dikwels word hierdie data bevooroordeeld of
wanvoorgestel binne 'n sekere konteks. In enige soort navorsing is die
insameling en hantering van data kernprosesse. Hierdie onderwerp help
ook leerders op om meer sosiaal en polities opgevoed te wees ten
opsigte van die media.
\\ \hline

 \end{tabular}
\end{center}
\end{table}

% Mathematics educators also need to ensure that the following important specific aims and general principles are applied in mathematics activities across all grades:
Wiskunde onderwysers moet ook verseker dat die volgende belangrike
spesifieke doelwitte en algemene beginsels toegepas word in
wiskunde-aktiwiteite in alle grade:
\begin{itemize}[noitemsep]
\item
% Calculators should only be used to perform standard numerical computations and verify calculations done by hand.
  Sakrekenaars mag slegs gebruik word om die standaard numeriese
  berekeninge uit te voer en berekeninge wat met die hand gedoen is,
  te kontroleer.
\item
% Real-life problems should be incorporated into all sections to keep mathematical modelling as an important focal point of the curriculum.
  Werklike probleme ge\"{i}ntegreer word in alle afdelings, om
  wiskundige modellering te behou as 'n belangrike fokuspunt van die
  kurrikulum.
\item
% Investigations give learners the opportunity to develop their ability to be more methodical, to generalise and to make and justify and/or prove conjectures.
  Ondersoeke gee leerders die geleentheid om hul vermo\"{e} om meer
  metodies te wees, om te kan veralgemeen, en om veronderstellings te
  kan ontwikkel en regverdig en/of bewys.
\item
% Appropriate approximation and rounding skills should be taught and continuously included and encouraged in activities.
  Gepaste benaderings- en afrondingsvaardighede moet geleer word en
  voortdurend aangemoedig word in aktiwiteite.
\item
% The history of mathematics should be incorporated into projects and tasks where possible, to illustrate the human aspect and developing nature of mathematics. 
  Die geskiedenis van wiskunde moet ingewerk word in projekte en take,
  waar moontlik, om die menslike aspek en ontwikkelende aard van
  wiskunde te illustreer.
\item
% Contextual problems should include issues relating to health, social, economic, cultural, scientific, political and environmental issues where possible. 
  Kontekstuele probleme moet kwessies met betrekking tot gesondheid,
  maatskaplike, ekonomiese, kulturele, wetenskaplike, politieke en
  omgewingskwessies insluit, waar moontlik.
\item
% Conceptual understanding of when and why should also feature in problem types.
  Konseptuele begrip van ``wanneer'' en ``hoekom'' moet ook deel vorm
  van die tipes probleme.
\item
% Mixed ability teaching requires educators to challenge able learners and provide remedial support where necessary. 
  Onderrig vir gemengde vermo\"{e}ns vereis dat opvoeders in staat is om
  leerders uit te daag en remedi\"{e}rende ondersteuning aan te bied, waar
  nodig.
\item
% Misconceptions exposed by assessment need to be dealt with and rectified by questions designed by educators. 
  Wanpersepsies wat deur assessering blootgestel word, moet hanteer
  en reggestel word met behulp van vrae ontwerp deur opvoeders.
\item
% Problem solving and cognitive development should be central to all mathematics teaching and learning so that learners can apply the knowledge effectively. 
  Probleemoplossing en kognitiewe ontwikkeling moet sentraal wees tot
  alle wiskunde onderrig en leerwerk, sodat leerders hulle kennis
  doeltreffend kan toepas.
\end{itemize}

\subsubsection{Toekenning van onderrigtyd}
%Time allocation for Mathematics per week: 4 hours and 30 minutes e.g. six forty-five minute periods per week.
Tydstoekenning vir Wiskunde per week: 4 uur en 30 minute, bv.\@ ses 45-minuut periodes per week.
\begin{table}[H]
 \begin{center} 
\begin{tabular}{|p{2cm}|p{6cm}|p{2cm}|} \hline
\textbf{Kwartaal}& \textbf{Onderwerp} & \textbf{Aantal weeks} \\ \hline  
\textbf{Kwartaal 1} & Algebra\"{i}ese uitdrukkings \par
Eksponente\par
Getalpatrone \par
Vergelykings en ongelykhede\par
Trigonometrie

&
3\par
2\par
1\par
2\par
3 \\ \hline
\textbf{Kwartaal 2} & Funksies \par
Trigonometriese funksies  \par
Euklidiese meetkunde  \par
Half-jaar eksamen &
4  \par
1 \par
3  \par
3  \\ \hline

\textbf{Kwartaal 3} & Analitiese meetkunde \par
Finansies en groei \par
Statistiek \par
Trigonometrie \par
Euklidiese meetkunde\par
Meting &
2 \par
2\par
2\par
2\par
1\par
1 \\ \hline
\textbf{Kwartaal 4} & Waarskynlikheid \par Hersiening \par Eksamen &
2 \par 
4 \par 
3 \\ \hline


 \end{tabular}
\end{center}
\end{table}

%Please see page 18 of the Curriculum and Assessment Policy Statement for the sequencing and pacing of topics.
Sien bladsy 18 van die Kurrikulum- en Assesseringsbeleidsraamwerk vir
die volgordebepaling en tempo van onderwerpe.

\section{Assessering}
%“Educator assessment is part of everyday teaching and learning in the classroom. Educators discuss with learners, guide their work, ask and answer questions, observe, help, encourage and challenge. In addition, they mark and review written and other kinds of work. Through these activities they are continually finding out about their learners’ capabilities and achievements. This knowledge then informs plans for future work. It is this continuous process that makes up educator assessment. It should not be seen as a separate activity necessarily requiring the use of extra tasks or tests.”  \par 
``Opvoeder assessering is deel van die alledaagse onderrig en leerwerk
in die klaskamer. Opvoeders hou besprekings met leerders, lei hulle in
hul werk, vra en beantwoord vrae, neem waar, help, moedig aan en daag
uit. Daarbenewens merk en hersien hulle geskrewe en ander vorme van
werk. Deur middel van hierdie aktiwiteite is hulle voortdurend besig
om meer uit te vind oor hulle leerders se vermo\"{e}ns en
prestasies. Hierdie kennis lig dan planne in vir toekomstige
werk. Opvoeder assessering behels hierdie deurlopende proses. Dit moet
nie gesien word as 'n afsonderlike aktiwiteit wat noodwendig die
gebruik van ekstra take of toetse vereis nie.''\par

%As the quote above suggests, assessment should be incorporated as part of the classroom practice, rather than as a separate activity. Research during the past ten years indicates that learners get a sense of what they do and do not know, what they might do about this and how they feel about it, from frequent and regular classroom assessment and educator feedback. The educator’s perceptions of and approach to assessment (both formal and informal assessment) can have an influence on the classroom culture that is created with regard to the learners’ expectations of and performance in assessment tasks. Literature on classroom assessment distinguishes between two different purposes of assessment; assessment of learning and assessment for learning. \par 
Soos die aanhaling hierbo suggereer, behoort assessering opgeneem te
word as deel van die klaskamerpraktyk, eerder as 'n afsonderlike
aktiwiteit. Navorsing gedurende die afgelope tien jaar dui aan dat
leerders 'n gevoel kry van wat hulle weet en nie weet nie, van wat
hulle hieromtrent kan doen en hoe hulle hieroor voel, vanaf gereelde
klaskamerassessering en onderwyser terugvoer. Die onderwyser se
persepsies van en benadering tot assessering (beide formele en
informele assessering) kan 'n invloed h\^{e} op die klaskamerkultuur,
wat geskep word met betrekking tot die leerders se verwagtinge van en
prestasie in assesseringstake. Literatuur oor klaskamerassessering
onderskei tussen twee verskillende doelwitte van assessering:
assessering van leer en assessering vir leer.\par

%Assessment of learning tends to be a more formal assessment and assesses how much learners have learnt or understood at a particular point in the annual teaching plan. The NCS provides comprehensive guidelines on the types of and amount of formal assessment that needs to take place within the teaching year to make up the school-based assessment mark. The school-based assessment mark contributes 25\% of the final percentage of a learner’s promotion mark, while the end-of-year examination constitutes the other 75\% of the annual promotion mark. Learners are expected to have 7 formal assessment tasks for their school-based assessment mark. The number of tasks and their weighting in the Grade 10 Mathematics curriculum is summarised below: 
Assessering van leer is geneig om 'n meer formele assessering te wees
en assesseer hoeveel leerders geleer het, of op 'n bepaalde punt in
die jaarlikse onderrigplan verstaan. Die NKV bied omvattende riglyne
oor die soorte en hoeveelheid van formele assessering wat moet
plaasvind binne die onderrigjaar, om die skoolgebaseerde
assesseringspunt saam te stel. Die skoolgebaseerde assesseringspunt
dra 25\% by tot die finale persentasie van 'n leerder se promosiepunt;
die einde van die jaar eksamen bepaal die ander 75\% van die jaarlikse
promosiepunt. Daar word van leerders verwag om 7 formele
assesseringstake vir hul skoolgebaseerde assesseringspunt te
h\^{e}. Die aantal take en hul gewig in die graad 10 wiskunde
kurrikulum is hieronder opgesom:

\begin{table}[H]
\begin{center}
\begin{tabular} {|p{5cm}|p{1.5cm}|p{2.5cm}|p{2.8cm}|} \hline
	  & 			& \textbf{Take} 			& \textbf{Gewigstoekenning (\%)} \\ \hline
Skool-gebaseerde Assessering &
Kwartaal 1 &
Toets \par Projek/Ondersoek &
10 \par 20 \\ \hline
&
Kwartaal 2 &
Opdrag/Toets \par Eksamen &
10 \par 30 \\ \hline
&
Kwartaal 3 &
Toets \par Toets &
10 \par 10 \\ \hline
&
Kwartaal 4 & Toets &
10 \\ \hline
Skool-gebaseerde Assesseringspunt &
&
&
100 \\ \hline
Skool-gebaseerde Assesseringspunt  \par
(as 'n \% van Vorderingspunt)
			&	 & 					&  25 \% \\ \hline

Eindeksamen & 	& 					&75 \% \\ \hline
Vorderingspunt 		&       & 					& 100 \% \\ \hline


\end{tabular}
 \end{center}
\end{table}

%The following provides a brief explanation of each of the assessment tasks included in the assessment programme above.
Die volgende is 'n kort verduideliking van elk van die
assesseringstake ingesluit in die assesseringsprogram hierbo.

\subsubsection{Toetse}
%All mathematics educators are familiar with this form of formal assessment. Tests include a variety of items/questions covering the topics that have been taught prior to the test. The new NCS also stipulates that mathematics tests should include questions that cover the following four types of cognitive levels in the stipulated weightings: 
Alle wiskunde-opvoeders is vertroud met hierdie vorm van formele
assessering. Die toetse sluit 'n verskeidenheid van items/vrae in wat
die onderwerpe dek wat reeds voor die toets aangebied is. Die nuwe NKV
bepaal ook dat wiskundetoetse vrae wat betrekking het op die volgende
vier tipes kognitiewe vlakke insluit:

\begin{table}[H]
\begin{center}
\begin{tabular} {|p{3cm}|p{6cm}|p{2.5cm}|} \hline
\textbf{Kognitiewe Vlakke} & \textbf{Beskrywing} & \textbf{Gewigstoekenning (\%)} \\ \hline
Kennis & 
%Estimation and appropriate rounding of numbers. \par 
Beramings- en toepaslike afronding van getalle. \par 
%Proofs of prescribed theorems.\par 
Bewyse van voorgeskrewe stellings. \par 
%Derivation of formulae.\par 
Die aflei van formules. \par 
%Straight recall.\par 
Direkte herroeping. \par 
%Identification and direct use of formula on information sheet (no changing of the subject). Use of mathematical facts.\par 
Identifisering en die direkte gebruik van formules op die
inligtingsblad (geen verandering van die onderwerp). Gebruik van
wiskundige feite. \par
%Appropriate use of mathematical vocabulary.
Toepaslike gebruik van wiskundige woordeskat.
& 
20 \\ \hline

Roetine Prosedures & 
%Perform well known procedures.\par 
Voer bekende prosedures uit. \par 
%Simple applications and calculations.\par 
Eenvoudige toepassings en berekeninge. \par 
%Derivation from given information.\par 
Afleiding vanaf die gegewe inligting. \par 
%Identification and use (including changing the subject) of correct formula.\par 
Identifikasie en die gebruik (insluitend die verandering van die onderwerp) van korrekte formules. \par 
%Questions generally similar to those done in class.
Vrae oor die algemeen soortgelyk aan di\'{e} wat in die klas behandel is.
&
45 \\ \hline

Komplekse Prosedures &
%Problems involve complex calculations and/or higher reasoning. \par 
Probleme behels komplekse berekenings en/of ho\"{e}r redenasie. \par 
%There is often not an obvious route to the solution.\par 
Daar is dikwels nie 'n duidelike pad na die oplossing. \par 
%Problems need not be based on real world context.\par 
Probleme hoef nie gebaseer te wees op 'n werklike konteks nie. \par 
%Could involve making significant connections between different representations.\par 
Kan die vorming van beduidende verbindings tussen verskillende
voorstellings behels.\par
%Require conceptual understanding.
Vereis konseptuele begrip.
&
25 \\ \hline
Probleemoplossing & 
%Unseen, non-routine problems (which are not necessarily difficult). \par 
Voorheen ongesiende, nie-roetine probleme (wat nie noodwendig moeilik
is nie). \par 
%Higher order understanding and processes are often involved.\par 
Ho\"{e}r orde begrip en prosesse is dikwels betrokke. \par 
%Might require the ability to break the problem down into its constituent parts.
Kan die vermo\"{e} om die probleem in sy samestellende dele op te
breek, vereis.
&
10 \\ \hline

\end{tabular}
 \end{center}
\end{table}

%The breakdown of the tests over the four terms is summarised from the NCS assessment programme as follows: \\
Die uiteensetting van die toetse oor die vier kwartale van die NKV
assesseringsprogram word as volg opgesom:\\
%\textbf{Term 1}:	One test of at least 50 marks, and one hour or two/three tests of at least 40 minutes each.\\
\textbf{Kwartaal 1}: Een toets van ten minste 50 punte en een uur, of twee/drie toetse van ten minste 40 minute elk.\\
%\textbf{Term 2}:	Either one test (of at least 50 marks) or an assignment.\\
\textbf{Kwartaal 2}: \'{O}f een toets (van ten minste 50 punte) \'{o}f 'n werkstuk.\\
%\textbf{Term 3}:	Two tests, each of at least 50 marks and one hour.\\
\textbf{Kwartaal 3}: Twee toetse, elk van ten minste 50 punte en een uur.\\
%\textbf{Term 4}:	One test of at least 50 marks.\\
\textbf{Kwartaal 4}: Een toets van ten minste 50 punte.\\

\subsubsection{Projekte/Ondersoeke}

%Investigations and projects consist of open-ended questions that initiate and expand thought processes. Acquiring and developing problem-solving skills are an essential part of doing investigations and projects. These tasks provide learners with the opportunity to investigate, gather information, tabulate results, make conjectures and justify or prove these conjectures.  Examples of investigations and projects and possible marking rubrics are provided in the next section on assessment support. The NCS assessment programme indicates that only one project or investigation (of at least 50 marks) should be included per year. Although the project/investigation is scheduled in the assessment programme for the first term, it could also be done in the second term. 
Ondersoeke en projekte bestaan ​​uit oop vrae wat denkprosesse
inisi\"{e}er en ontwikkel. Die aanleer en ontwikkeling van
probleem-vaardighede is 'n noodsaaklike deel van ondersoeke en
projekte. Hierdie take bied leerders die geleentheid om ondersoek in
te stel, inligting te versamel, resultate te tabuleer,
veronderstellings te maak, en hierdie veronderstellings te regverdig
of bewys. Voorbeelde van ondersoeke en projekte en moontlike
assesseringskale word aangegee in die volgende afdeling oor
assessering ondersteuning. Die NKV assesseringsprogram dui aan dat
slegs een projek of ondersoek (van ten minste 50 punte) per jaar
ingesluit moet word. Alhoewel die projek/ondersoek in die eerste
kwartaal aangegee word van die assesseringsskedule, kan dit ook in die
tweede kwartaal gedoen word.


\subsubsection{Opdragte}
%The NCS includes the following tasks as good examples of assignments: 
Die NKV sluit die volgende take in as goeie voorbeelde van opdragte:
\begin{itemize}[noitemsep]
\item
% Open book test
  Oopboe toets
\item
% Translation task
  Vertalingopdrag
\item
% Error spotting and correction
  Foute identifiseer en verbeter
\item
% Shorter investigation
  Korter ondersoek
\item
% Journal entry
  Joernaalinskrywing
\item
% Mind-map (also known as a metacog)
  Breinkaart (ook bekend as 'n metacog)
\item
% Olympiad (first round)
  Olimpiade (eerste rondte)
\item
% Mathematics tutorial on an entire topic
  Wiskunde-handleiding oor 'n hele onderwerp
\item
% Mathematics tutorial on more complex/problem solving questions
  Wiskunde tutoriaal op meer komplekse/probleemoplossings vrae
\end{itemize}
%The NCS assessment programme requires one assignment in term 2 (of at least 50 marks) which could also be a combination of some of the suggested examples above. More information on these suggested examples of assignments and possible rubrics are provided in the following section on assessment support. 
Die NKV assesseringsprogram vereis dat 'n opdrag(van ten minste 50 punte) in die tweede kwartaal gegee word. Dit kan 'n kombinasie wees van 'n paar van die voorbeelde hierbo. Meer inligting oor hierdie voorgestelde voorbeelde van opdragte en moontlike assesseringsrubrieke word in die volgende afdeling oor assesseringsondersteuning gegee.

\subsubsection{Eksamens}
%Educators are also all familiar with this summative form of assessment that is usually completed twice a year: mid-year examinations and end-of-year examinations. These are similar to the tests but cover a wider range of topics completed prior to each examination. The NCS stipulates that each examination should also cover the four cognitive levels according to their recommended weightings as summarised in the section above on tests. The following table summarises the requirements and information from the NCS for the two examinations.
Opvoeders is goed vertroud met hierdie summatiewe vorm van assessering wat gewoonlik tweekeer per jaar gedoen word: die middel-van-die-jaar eksamens en einde-van-jaar eksamens. Dit is soortgelyk aan toetse, maar dek 'n groter verskeidenheid van onderwerpe wat voltooi is voor elke eksamen. Die NKV bepaal dat elke eksamen die vier kognitiewe vlakke sal dek volgens hul aanbevole gewigte soos saamgevat in die afdeling oor toetse. Die volgende tabel gee 'n opsomming van die vereistes en inligting van die NKV vir die twee eksamens.

\begin{table}[H]
\begin{center}
\begin{tabular} {|p{2cm}|p{1.5cm}|p{3cm}|p{4.5cm}|} \hline
\textbf{Eksamen} &
\textbf{Punte} &
\textbf{Uiteensetting} &
\textbf{Inhoud en puntverspreiding} \\ \hline
Middel-van-die-jaar eksamen &
100 \par 50 + 50 &
Een vraestel: 2 uur\par \textbf{of} \par
Twee vraestelle: 1 uur elk &
Onderwerpe voltooi \\ \hline
Einde-van-die-jaar eksamen	&
100 + &
Vraestel 1: 2 ure &
Getalpatrone ($\pm 10$) \par 
Algebra\"{i}ese uitdrukkings, vergelykings en ongelykhede ($\pm 25$)\par 
Funksies ($\pm 35$)\par 
Eksponente ($\pm 10$)\par 
Finansies ($\pm 10$)\par 
Waarskynlikheid ($\pm 10$)
\\ \hline
&
100 &
Vraestel 2: 2 ure &
Trigonometrie ($\pm 45$) \par 
Analitiese meetkunde ($\pm 15$)\par 
Euklidiese meetkunde en meting ($\pm 25$)\par 
Statistiek ($\pm 15$)
\\ \hline
\end{tabular}
 \end{center}
\end{table}

%In the annual teaching plan summary of the NCS in Mathematics for Grade 10, the pace setter section provides a detailed model of the suggested topics to be covered each week of each term and the accompanying formal assessment.
In die jaarlikse opsommende onderrigplan van die NKV in Wiskunde vir Graad 10, verskaf die pasaanduider afdeling 'n gedetailleerde model van die voorgestelde onderwerpe wat gedek moet word in elke week van elke kwartaal en die gepaardgaande formele assessering. \par

%Assessment \textbf{for} learning tends to be more informal and focuses on using assessment in and of daily classroom activities that can include: 
Assessering \textbf{vir} leer is geneig om meer informeel te wees en fokus op die toepassing van assessering in die loop van die daaglikse klaskameraktiwiteite. Dit sluit in:
\begin{itemize}[noitemsep]
\item
% Marking homework
Nasien van huiswerk
\item
% Baseline assessments
Basislynassesserings
\item
% Diagnostic assessments
Diagnostiese assessering
\item
% Group work
Groepwerk
\item
% Class discussions
Klasbesprekings
\item
% Oral presentations
Mondelinge aanbiedings
\item
% Self-assessment
Self-assessering
\item
% Peer-assessment
Eweknie-assessering
\end{itemize}

%These activities are expanded on in the next section on assessment support and suggested marking rubrics are provided. Where formal assessment tends to restrict the learner to written assessment tasks, the informal assessment is necessary to evaluate and encourage the progress of the learners in their verbal mathematical reasoning and communication skills. It also provides a less formal assessment environment that allows learners to openly and honestly assess themselves and each other, taking responsibility for their own learning, without the heavy weighting of the performance (or mark) component. The assessment for learning tasks should be included in the classroom activities at least once a week (as part of a lesson) to ensure that the educator is able to continuously evaluate the learners’ understanding of the topics covered as well as the effectiveness, and identify any possible deficiencies in his or her own teaching of the topics. 
Hierdie aktiwiteite word uitgebrei in die volgende afdeling oor assesseringsondersteuning en voorgestelde assesseringskale word voorsien. Waar formele assessering geneig is om die leerder te beperk tot skriftelike assesseringstake, is die informele assessering nodig is om leerders se vordering in verbale wiskundige redenasie- en kommunikasievaardighede te evalueer en aan te moedig. Dit bied ook 'n minder formele assesseringsomgewing wat leerders toelaat om hulleself openlik en eerlik te assesseer, om verantwoordelikheid te neem vir hul eie leer, sonder die swaar las van die prestasie (of punte) komponent. Die assessering-vir-leer aktiwiteite moet ten minste een keer 'n week ingesluit word in die klaskamer-aktiwiteite (as deel van' n les) om te verseker dat die opvoeder in staat is om voortdurend die leerders se begrip van die onderwerpe wat gedek is en hulle doeltreffendheid te evalueer.  Dit bemagtig ook die opvoeder om enige moontlike tekortkominge in sy of haar eie onderrig van die onderwerpe te identifiseer.

\subsection{Assessering ondersteuning}
%A selection of explanations, examples and suggested marking rubrics for the assessment of learning (formal) and the assessment for learning (informal) forms of assessment discussed in the preceding section are provided in this section. 
'n Verskeidenheid van verduidelikings, voorbeelde en voorgestelde assesseringskale vir die assessering van leer (formele vorme van assessering) en die assessering vir leer (informele vorme van assessering), wat in die vorige afdeling genoem is, word in hierdie afdeling uiteengesit.

\subsubsection{Basislynassessering}
%Baseline assessment is a means of establishing:
Basislyn- of grondlynassessering is 'n metode vir die vasstelling van:

\begin{itemize}[noitemsep]
\item
% What prior knowledge a learner possesses 
  die voorkennis waaroor ‘n leerder beskik
\item
% What the extent of knowledge is that they have regarding a specific learning area
  ’n leerder se vlak van kennis oor 'n spesifieke leerarea
\item
% The level they demonstrate regarding various skills and applications
  die vaardigheids- en toepassingsvlak wat ‘n leerder toon
\item
% The learner’s level of understanding of various learning areas
  ‘n leerder se vlak van begrip van die verskillende leerareas
\end{itemize}
%It is helpful to educators in order to assist them in taking learners from their individual point of departure to a more advanced level and to thus make progress. This also helps avoid large "gaps” developing in the learners’ knowledge as the learner moves through the education system. Outcomes-based education is a more learner-centered approach than we are used to in South Africa, and therefore the emphasis should now be on the level of each individual learner rather than that of the whole class. \par 
Dit is nuttig vir ‘n opvoeder om te weet wat ‘n leerder se individuele vertrekpunt is, ten einde hom/haar te help na 'n meer gevorderde vlak en om sodoende vordering maak. Dit help ook voorkom dat groot "gapings" bestaan in leerders se kennis soos wat hulle beweeg deur die onderwysstelsel. Uitkomsgebaseerde onderwys is 'n meer leerder-gesentreerde benadering as waaraan ons in Suid-Afrika gewoond is; dus moet die klem moet nou verskuif na die vlak van elke individuele leerder, eerder as dié van die hele klas.\par
%The baseline assessments also act as a gauge to enable learners to take more responsibility for their own learning and to view their own progress. In the traditional assessment system, the weaker learners often drop from a 40\% average in the first term to a 30\% average in the fourth term due to an increase in workload, thus demonstrating no obvious progress. Baseline assessment, however, allows for an initial assigning of levels which can be improved upon as the learner progresses through a section of work and shows greater knowledge, understanding and skill in that area.
Die basislynassessering dien ook as ‘n maatstaf om leerders in staat te stel om meer verantwoordelikheid vir hulle eie leer te neem en hulle eie vordering te monitor. In die tradisionele assesseringstelsel daal die swakker leerders dikwels van 'n gemiddeld van 40\% in die eerste kwartaal tot 'n gemiddeld van 30\% in die vierde kwartaal as gevolg van 'n toename in die werkslading. Hulle toon dus geen duidelike vordering nie. Basislynassessering gee inligting oor die vlakke wat verbeter kan word en soos die leerder vorder deur 'n afdeling van die werk, kan aangetoon word of die leerder meer kennis, begrip en vaardigheid in daardie gebied verwerf.

\subsubsection{Diagnostiese assessering}
%These are used to specifically find out if any learning difficulties or problems exist within a section of work in order to provide the learner with appropriate additional help and guidance. The assessment helps the educator and the learner identify problem areas, misunderstandings, misconceptions and incorrect use and interpretation of notation. \par
Dit word gebruik om uit te vind of enige spesifieke probleme bestaan ten opsigte van 'n afdeling van die werk ten einde die leerder te voorsien van toepaslike addisionele hulp en leiding. Die assessering help die opvoeder en die leerder om probleemareas, misverstande, wanopvattings en foutiewe gebruik en interpretasie van notasie te identifiseer.\par

%Some points to keep in mind:
'n Paar punte om in gedagte te hou:
\begin{itemize}[noitemsep]
\item
% Try not to test too many concepts within one diagnostic assessment.
Probeer om nie te veel konsepte te toets binne 'n enkele diagnostiese assessering nie.
\item
% Be selective in the type of questions you choose. 
  Wees selektief in die tipe vrae wat jy kies.
\item
% Diagnostic assessments need to be designed with a certain structure in mind. As an educator, you should decide exactly what outcomes you will be assessing and structure the content of the assessment accordingly. 
  Diagnostiese assesserings moet met 'n sekere struktuur in gedagte ontwerp word. As 'n opvoeder moet jy besluit presies watter uitkomste jy wil assesseer en die inhoud van die assessering dienooreenkomstig struktureer.
\item
% The assessment is marked differently to other tests in that the mark is not the focus but rather the type of mistakes the learner has made.
  Die beoordeling is anders gemerk ander toetse wat die punt is nie die fokus nie, maar eerder die tipe foute wat die leerder gemaak het.
\end{itemize}
%An example of an understanding rubric for educators to record results is provided below:\\
Die beoordeling word anders gemerk as ander toetse want die punt is nie die fokus nie, maar eerder die tipe foute wat die leerder gemaak het.

%0: indicates that the learner has not grasped the concept at all and that there appears to be a fundamental mathematical problem.\\
0: dui aan dat die leerder nie die konsep onder die knie het nie en dat daar 'n fundamentele wiskundige probleem bestaan.\\
%1: indicates that the learner has gained some idea of the content, but is not demonstrating an understanding of the notation and concept.\\
1: dui aan dat die leerder 'n idee het van die inhoud, maar nie ware begrip van die inhoud of die notasie toon nie.\\
%2: indicates evidence of some understanding by the learner but further consolidation is still required.\\
2: dui op getuienis van 'n mate van begrip deur die leerder, maar verdere konsolidasie is steeds 'n vereiste.\\
%3:indicates clear evidence that the learner has understood the concept and is using the notation correctly.\par
3: dui op 'n duidelike bewys dat die leerder die konsep verstaan en die notasie korrek kan gebruik.

\subsubsection{Sakrekenaar werkblad: assessering van diagnostiese vaardighede}
\begin{enumerate}[itemsep=7pt, label=\textbf{\arabic*}. ] 
 \item Bereken:
\begin{enumerate}[itemsep=6pt,label=\textbf{(\alph*)}]
\item $ 242 + 63=$   ~~~\underline{~~~~~~~~~~~~~}
\item $2-36 \times (114 + 25)=$~~~\underline{~~~~~~~~~~~~~}
\item $\sqrt{144+25}=$~~~\underline{~~~~~~~~~~~~~}
\item $\sqrt[4]{729}=$~~~\underline{~~~~~~~~~~~~~}
\item $-312 + 6 + 879 -321 + 18~ 901=$ ~~~\underline{~~~~~~~~~~~~~}
\end{enumerate}

\item Bereken:
\begin{enumerate}[itemsep=6pt,label=\textbf{(\alph*)}]
\item $\frac{2}{7} + \frac{1}{3}=$  ~~~\underline{~~~~~~~~~~~~~}
\item $2\frac{1}{5} - \frac{2}{9}=$ ~~~\underline{~~~~~~~~~~~~~}
\item $-2\frac{5}{6} + \frac{3}{8}=$ ~~~\underline{~~~~~~~~~~~~~}
\item $ 4 - \frac{3}{4} \times \frac{5}{7}=$ ~~~\underline{~~~~~~~~~~~~~}
\item $\left(\frac{9}{10} - \frac{8}{9}\right) \div \frac{3}{5}=$ ~~~\underline{~~~~~~~~~~~~~}
\item $2\times \left(\frac{4}{5}\right)^2 - \left(\frac{19}{25}\right)=$ ~~~\underline{~~~~~~~~~~~~~}
\item $\sqrt{\frac{9}{4} - \frac{4}{16}} =$ ~~~\underline{~~~~~~~~~~~~~}
\end{enumerate}
\end{enumerate}
\clearpage
Self-assessering rubriek: \\
Naam: \underline{~~~~~~~~~~~~~~~~~~~~~~~~~~~~~~}
\begin{table}[H]
 \begin{center}
  \begin{tabular}{|p{1.5cm}|p{1.5cm}|p{1cm}|p{1cm}|p{6cm}|} \hline

\textbf{Vraag} & \textbf{Antwoord} & \textbf{√} & \textbf{X} & \textbf{Indien X, volgorde neer waarin die sleutels gedruk is} \\ \hline
1a) &&&&\\ \hline
1b)&&&&\\ \hline
1c)&&&&\\ \hline
1d)&&&&\\ \hline
1e)&&&&\\ \hline
\textbf{Subtotaal}&&&&\\ \hline
2a)&&&&\\ \hline
2b)&&&&\\ \hline
2c)&&&&\\ \hline
2d)&&&&\\ \hline
2e)&&&&\\ \hline
2f)&&&&\\ \hline
2g)&&&&\\ \hline
\textbf{Subtotaal}&&&&\\ \hline
\textbf{Totaal}&&&& \\ \hline


   
  \end{tabular}

 \end{center}

\end{table}

Opvoeder-assessering rubriek:
\begin{table}[H]
 \begin{center}
  \begin{tabular}{|p{4.5cm}|p{1.5cm}|p{3cm}|p{1.5cm}|} \hline

\textbf{Tipe Vaardigheid} & \textbf{Bemeester} & \textbf{Benodig oefening} & \textbf{Probleem}   \\ \hline
Verhoging na 'n mag &&&\\ \hline
Vind 'n wortel &&&\\ \hline
Berekeninge met breuke &&&\\ \hline
Hakies en volgorde van berekeninge &&&\\ \hline
Skattings en hoofreken-kontrole &&&\\ \hline
   
  \end{tabular}

 \end{center}

\end{table}
%Guidelines for Calculator Skills Assessment:
Riglyne vir assessering van sakrekenaarvaardighede:
\begin{table}[H]
 \begin{center}
  \begin{tabular}{|p{5cm}|p{4cm}|p{3cm}|} \hline

\textbf{Tipe Vaardigheid} & \textbf{Onderafdeling} & \textbf{Vrae}   \\ \hline
Verhoging tot 'n mag  & Vierkante en derdemagte\par Ho\"{e}r orde magte&1a, 2f \par1b \\ \hline
Vind 'n wortel& Vierkants en derdemagswortels \par Ho\"{e}r order wortels & 1c, 2g \par 1d\\ \hline
Berekeninge met breuke & Basiese erekeninge \par Gemengde getalle \par Negatiewe getalle \par Kwadreer breuke \par Vierkantswortel van breuke &2a,  2d\par
2b, 2c\par
1e, 2c\par
2f\par
2g
\\ \hline
Hakies en volgeorde van berekininge&Korrekte gebruik van hakies of volgorde van berekeninge&1b, 1c, 2e, 2f, 2g\\ \hline
Skattings en hoofrekenkontrole &Algeheel&Almal\\ \hline
   
  \end{tabular}

 \end{center}

\end{table}

\subsubsection{Voorgestelde riglyn vir die toekenning van algehele vlakke}\\
\textbf{Vlak 1}
\begin{itemize}[noitemsep]
\item
% Learner is able to do basic operations on calculator.
Leerder is in staat om basiese bewerkings te doen met die sakrekenaar.
\item
% Learner is able to do simple calculations involving fractions.
Leerder is in staat om basiese bewerkings te doen met die sakrekenaar.
\item
% Learner does not display sufficient mental estimation and control techniques.
Die leerder toon nie voldoende hoofrekenskatting en -kontrole vaardighede nie.
\end{itemize}
\textbf{Vlak 2}\begin{itemize}[noitemsep]
\item
% Learner is able to do basic operations on calculator.
Leerder is in staat om basiese bewerkings te doen met die sakrekenaar.
\item
% Learner is able to square and cube whole numbers as well as find square and cube roots of numbers.
Leerder is in staat om vierkante en derdemagte van heelgetalle asook vierkants-en derdemagswortels van getalle te bereken.
\item
% Learner is able to do simple calculations involving fractions as well as correctly execute calculations involving mixed numbers.
Leerder is in staat om eenvoudige berekeninge met betrekking tot breuke te doen, sowel as om bewerkings met gemengde breuke korrek uit te voer.
\item
% Learner displays some degree of mental estimation awareness.
Leerder toon 'n mate van hoofrekenskatting en –kontrole tegnieke.
\end{itemize}
\textbf{Vlak 3}\begin{itemize}[noitemsep]
\item
% Learner is able to do basic operations on calculator.
Leerder is in staat om basiese bewerkings te doen op die sakrekenaar.
\item
% Learner is able to square and cube rational numbers as well as find square and cube roots of numbers.
Leerder is in staat om vierkante en derdemagte van heelgetalle asook vierkants-en derdemagswortels van getalle te bereken.
\item
% Learner is also able to calculate higher order powers and roots.
Leerder is ook in staat om hoër-orde magte en wortels te bereken.
\item
% Learner is able to do simple calculations involving fractions as well as correctly execute calculations involving mixed numbers.
Leerder is in staat om eenvoudige berekeninge met betrekking tot breuke te doen, sowel as om bewerkings met gemengde breuke korrek uit te voer.
\item
% Learner works correctly with negative numbers.
Leerder werk korrek met negatiewe getalle.
\item
% Learner is able to use brackets in certain calculations but has still not fully understood the order of operations that the calculator has been programmed to execute, hence the need for brackets.
Leerder is in staat om hakies te gebruik in sekere berekeninge, maar verstaan nog nie ten volle die volgorde van bewerkings wat die sakrekenaar geprogrammeer is om uit te voer nie, vandaar die behoefte aan hakies.
\item
% Learner is able to identify possible errors and problems in their calculations but needs assistance solving the problem.
Leerder is in staat om moontlike foute en probleme in hul berekeninge
te identifiseer, maar het hulp nodig om die probleem op te los.
\end{itemize}
\textbf{Vlak 4}\begin{itemize}[noitemsep]
\item
% Learner is able to do basic operations on calculator.
Leerder is in staat om basiese bewerkings te doen op die sakrekenaar.
\item
% Learner is able to square and cube rational numbers as well as find square and cube roots.
eerder is in staat om vierkante en derdemagte van heelgetalle asook vierkants-en derdemagswortels van getalle te bereken.
\item
% Learner is also able to calculate higher order powers and roots.
Leerder is ook in staat om ho\"{e}r-orde-magte en wortels te bereken.
\item
% Learner is able to do simple calculations involving fractions as well as correctly execute calculations involving mixed numbers.
Leerder is in staat om eenvoudige berekeninge met betrekking tot breuke te doen, sowel as om bewerkings met gemengde breuke korrek uit te voer.
\item
% Learner works correctly with negative numbers.
Leerder werk korrek met negatiewe getalle.
\item
% Learner is able to work with brackets correctly and understands the need and use of brackets and the “= key” in certain calculations due to the nature of a scientific calculator.
Leerder is in staat om korrek te werk met hakies en om die noodsaaklikheid en die gebruik van hakies en die "= sleutel" in berekeninge reg te hanteer volgens die aard van 'n wetenskaplike sakrekenaar.
\item
% Learner is able to identify possible errors and problems in their calculations and to find solutions to these in order to arrive at a “more viable” answer.
Leerder is in staat om moontlike foute en probleme in hul berekeninge te identifiseer en om oplossings hiervoor te vind ten einde by 'n "meer geloofwaardige" antwoord uit te kom.
\end{itemize}

\subsubsection{Ander kort diagnostiese toetse}
%These are short tests that assess small quantities of recall knowledge and application ability on a day-to-day basis. Such tests could include questions on one or a combination of the following:
Dit is kort toetse wat klein hoeveelhede herroeping en toepassing van kennis op 'n dag-tot-dag basis assesseer. Sulke toetse kan vrae oor een of 'n kombinasie van die volgende insluit:
\begin{itemize}[noitemsep]
\item
% Definitions
  Definisies
\item
% Theorems
  Stellings
\item
% Riders (geometry)
  Meetkundeprobleme
\item
% Formulae
  Formules
\item
% Applications
  Toepassings
\item
% Combination questions
  Gekombineerde vrae
\end{itemize}
%Here is a selection of model questions that can be used at Grade 10 level to make up short diagnostic tests. They can be marked according to a memorandum drawn up by the educator. \par
Hier is 'n seleksie van modelvrae wat op Graad 10-vlak gebruik kan  word as kort diagnostiese toetse. Hulle kan gemerk word volgens 'n memorandum opgestel deur die opvoeder.
\textbf{Meetkunde}
\begin{enumerate}[itemsep=0pt, label=\textbf{\arabic*}. ] 
\item Punte $A(-5; -3)$, $B(-1; 2)$ en $C(9; -6)$ is die hoekpunte van $\triangle ABC$. 
\begin{enumerate}[itemsep=0pt,label=\textbf{(\alph*)}]
\item Bereken die gradi\"{e}nt van $AB$ en $BC$ en toon vervolgens dat hoek $ABC$ gelyk is aan $90^{\circ}$.				
\begin{flushright}(5)\end{flushright}
\item Gee die afstandformule en gebruik dit om die lengtes van sye
$AB$, $BC$ en $AC$ van $\triangle ABC$ te bereken. (Laat jou antwoorde in wortelvorm.)
\begin{flushright}(5)\end{flushright}
\end{enumerate}
\end{enumerate}

\textbf{Algebra}
\begin{enumerate}[itemsep=0pt, label=\textbf{\arabic*}. ] 
\item Skryf die formele definisie van 'n eksponent neer sowel as die eksponentwette vir heelgetal eksponente.\begin{flushright}(6)\end{flushright}
\item Vereenvoudig: $\dfrac{2x^4y^8z^3}{4xy} \times \dfrac{x^7}{y^3z^0}$	\begin{flushright}(4)\end{flushright}
\end{enumerate}

\textbf{Trigonometrie}
\begin{enumerate}[itemsep=0pt, label=\textbf{\arabic*}. ] 
\item
% A jet leaves an airport and travels $578$ km in a direction of $50^{\circ}$ E of N. The pilot then changes direction and travels $321$ km $10^{\circ}$ W of N.
  'n Vliegtuig styg op van 'n lughawe en vlieg  $578$ km in 'n
  rigting van $50^{\circ}$ O van N. Die vlieënier verander dan van rigting en vlieg $321$ km $10^{\circ}$ W van N.
\begin{enumerate}[itemsep=0pt,label=\textbf{(\alph*)}]
\item Hoe ver weg van die lughawe is die vliegtuig? (Tot die naaste kilometer.) \begin{flushright}(5)\end{flushright}
\item Bepaal die rigting en afstand van die vliegtuig vanaf die lughawe.\begin{flushright}(5)\end{flushright}
\end{enumerate}
\end{enumerate}

\subsubsection{Oefeninge}
%This entails any work from the textbook or other source that is given to the learner, by the educator, to complete either in class or at home. Educators should encourage learners not to copy each other’s work and be vigilant when controlling this work. It is suggested that such work be marked/controlled by a check list (below) to speed up the process for the educator. \par
Dit behels enige werk uit die handboek of 'n ander bron wat aan die leerder gegee word deur die opvoeder om in die klas óf tuis te voltooi. Opvoeders moet leerders aanmoedig om nie mekaar se werk te kopieer nie en moet oplettend en noukeurig wees in die nagaan van hierdie werk. Dit word aanbeveel dat hierdie tipe werk gemerk/gekontroleer sal word met behulp van ‘n kontrolelys (hieronder) om die proses vir die opvoeder te bespoedig. \par

%The marks obtained by the learner for a specific piece of work need not be based on correct and/or incorrect answers but preferably on the following:
Die punte wat behaal word deur die leerder vir 'n spesifieke stuk werk moet nie gebaseer wees op korrekte of verkeerde antwoorde nie, maar verkieslik op die volgende:

\begin{itemize}[noitemsep]
\item
% the effort of the learner to produce answers.
  die poging van die leerder om antwoorde te produseer.
\item
% the quality of the corrections of work that was previously incorrect.
  die kwaliteit van die regstellings aan werk wat voorheen verkeerd was.
\item
% the ability of the learner to explain the content of some selected examples (whether in writing or orally).
  die vermoë van die leerder om die inhoud van 'n paar geselekteerde voorbeelde (hetsy skriftelik of mondeling) te verduidelik.
\end{itemize}
%The following rubric can be used to assess exercises done in class or as homework: 
Die volgende rubriek kan gebruik word om klas- of huiswerkoefeningete assesseer:

\begin{table}[H]
 \begin{center}
  \begin{tabular}{|p{3cm}|p{3cm}|p{3cm}|p{3cm}|} \hline
   \textbf{Kriteria} & \textbf{Prestasie-aanduiders} &&\\ \hline
Werk gedoen & 2 \par Al die werk gedoen  & 1 \par Gedeeltelik voltooi & 0 \par Geen werk gedoen \\ \hline
Werk netjies gedoen & 2 \par Werk netjies gedoen & 1 \par Sekere werk nie netjies gedoen nie & 0 \par Slordig en deurmekaar\\ \hline
Regstellings gedoen & 2 \par Deurgaans alle regstellings gedoen & 1 \par Ten minste helfte van die regstellings gedoen & 0 \par Geen regstellings gedoen \\ \hline
Korrekte wiskundige metode & 2 \par Deurgaans & 1 \par Soms & 0 \par Glad nie \\ \hline
Begrip van wiskundige tegnieke en prosesse & 2 \par Kan konsepte en prosesse akkuraat verduidelik & 1 \par Verduidelikings is dubbelsinnig en nie gefokus nie & 0 \par Verduidelikings is verwarrend of irrelevant \\ \hline
  \end{tabular}

 \end{center}

\end{table}

\subsubsection{Joernaalinskrywings}
%A journal entry is an attempt by a learner to express in the written word what is happening in Mathematics. It is important to be able to articulate a mathematical problem, and its solution in the written word. \par
'n Joernaalinskrywing is 'n poging deur 'n leerder om in geskrewe woorde uit te druk wat in Wiskunde gebeur. Dit is belangrik om in staat wees om 'n wiskundige probleem en die oplossing daarvan in die geskrewe taal te verwoord.

%This can be done in a number of different ways:
Dit kan op verskillende maniere gedoen word:
\begin{itemize}[noitemsep]
\item
% Today in Maths we learnt \underline{~~~~~~~~~~~~~~~~~~~~~~~~~~~~~~~} 
  Vandag in Wiskunde het ons geleer \underline{~~~~~~~~~~~~~~~~~~~~~~~~~~~~~~~} 
\item
% Write a letter to a friend, who has been sick, explaining what was done in class today.
Skryf 'n brief aan 'n vriend wat siek was om te verduidelik wat vandag gebeur het in die klas.
\item
% Explain the thought process behind trying to solve a particular maths problem, e.g. sketch the graph of $ y = x^2 - 2x^2 + 1$ and explain how to sketch such a graph.
  Verduidelik die denkproses agter die poging om 'n spesifieke wiskundeprobleem op te los, bv. skets die grafiek van $y = x^2 - 2x^2 + 1$ en verduidelik hoe om so 'n grafiek te teken.
\item
% Give a solution to a problem, decide whether it is correct and if not, explain the possible difficulties experienced by the person who wrote the incorrect solution. 
   Gee 'n oplossing vir 'n probleem, besluit of dit korrek is, en indien nie, verduidelik die moontlike probleme wat ervaar word deur die persoon wat die verkeerde oplossing geskryf het.
\end{itemize}
%A journal is an invaluable tool that enables the educator to identify any mathematical misconceptions of the learners. The marking of this kind of exercise can be seen as subjective but a marking rubric can simplify the task. 
'n Joernaal is 'n waardevolle hulpmiddel wat die opvoeder in staat stel om wiskundige wanopvattings van die leerders te identifiseer. Die nasien van hierdie soort oefening kan gesien word as subjektief, maar 'n rubriek kan die taak vereenvoudig. \par

Die volgende rubriek kan gebruik word om joernaalinskrywings te merk. Die assesseringsrubriek moet aan leerders gegee voor die taak gedoen word.


%The following rubric can be used to mark journal entries. The learners must be given the marking rubric before the task is done. 

\begin{table}[H]
 \begin{center}
  \begin{tabular}{|p{4cm}|p{2cm}|p{2.5cm}|p{3cm}|} \hline
  \textbf{Taak} & \textbf{Bevoegd \newline(2 Punte)} & \textbf{Ontwikkel nog \newline(1 Punt)}& \textbf{Nog nie ontwikkel \newline (1 Punt)}\\ \hline
Voltooi binne tydsbeperking? &&&\\ \hline
Korrektheid van die verduideliking? &&&\\ \hline
Korrekte en toepaslike gebruik van wiskundige taal? &&&\\ \hline
Is die wiskunde korrek? &&&\\ \hline
Is die konsep korrek ge\"{i}nterpreteer?&&&\\ \hline

  \end{tabular}

 \end{center}

\end{table}

\subsubsection{Vertalings}
%Translations assess the learner’s ability to translate from words into mathematical notation or to give an explanation of mathematical concepts in words. Often when learners can use mathematical language and notation correctly, they demonstrate a greater understanding of the concepts. \par 
Vertaling assesseer die leerder se vermoë om woorde te vertaal in wiskundige notasie of om 'n verduideliking van wiskundige konsepte in woorde te gee. Dikwels wanneer leerders wiskundige taal en notasie korrek kan gebruik, toon hulle 'n groter begrip van die konsepte.\par

%For example: \\
Byvoorbeeld: \\
%Write the letter of the correct expression next to the matching number:\\
Skryf die letter van die korrekte uitdrukking langs die ooreenstemmende nommer:
\begin{table}[H]
 \begin{center}
  \begin{tabular}{lrl} 
$x$ word met $10$ vermeerder&					a)	&$xy$ \\
Die produk van $x$ en $y$		 &		 b)	&$x-2$\\
Die som van 'n sekere getal en	dubbel daardie getal&		c)&	$x^2$\\
Helfte van 'n sekere getal vermenigvuldig met homself	&	d)&	$\frac{1}{2} \times 2$\\
Twee minder as $x$&					e)&	$x + 2x  $\\
'n Sekere getal vermenigvuldig met homself	&		f)&	$x+10$\\
  \end{tabular}
 \end{center}
\end{table}

\subsubsection{Groepwerk}
%One of the principles in the NCS is to produce learners who are able to work effectively within a group. Learners generally find this difficult to do. Learners need to be encouraged to work within small groups. Very often it is while learning under peer assistance that a better understanding of concepts and processes is reached. Clever learners usually battle with this sort of task, and yet it is important that they learn how to assist and communicate effectively with other learners. \par
Een van die beginsels in die NKV is om leerders te produseer wat in staat is om effektief te werk in groepsverband. Leerders vind dit oor die algemeen moeilik om te doen, daarom moet hulle aangemoedig word om in klein groepies te werk. Dikwels ontwikkel leerders ‘n beter begrip van konsepte en prosesse wanneer hulle met bystand van hulle maats werk. Slim leerders vind hierdie tipe taak gewoonlik moeilik, en tog is dit belangrik dat hulle leer hoe om te help en effektief te kommunikeer met ander leerders.\par

\subsubsection{Mind Maps of Metacogs}
%A metacog or “mind map” is a useful tool. It helps to associate ideas and make connections that would otherwise be too unrelated to be linked. A metacog can be used at the beginning or end of a section of work in order to give learners an overall perspective of the work covered, or as a way of recalling a section already completed. It must be emphasised that it is not a summary. Whichever way you use it, it is a way in which a learner is given the opportunity of doing research in a particular field and can show that he/she has an understanding of the required section. \par 
n Metacog of "breinkaart" is 'n nuttige hulpmiddel. Dit help om idees te koppel en verbindings te vorm van sake wat andersins onverwant voorgekom het. 'n Breinkaart kan gebruik word aan die begin of einde van 'n afdeling ten einde leerders 'n oorsigtelike perspektief van die werk te gee, of as 'n herinnering aan 'n gedeelte van die werk wat reeds afgehandel is. Dit moet beklemtoon word dat dit nie 'n opsomming is nie. Ongeag hoe jy dit gebruik, dit is 'n geleentheid wat 'n leerder gegee word om navorsing te doen in 'n bepaalde veld en te kan toon dat hy/sy ‘n begrip het van die betrokke afdeling.\par

%This is an open book form of assessment and learners may use any material they feel will assist them. It is suggested that this activity be practised, using other topics, before a test metacog is submitted for portfolio assessment purposes. \par
Dit is 'n oopboek vorm van assessering en leerders kan enige materiaal gebruik wat hulle voel sal kan help. Daar word voorgestel dat hierdie aktiwiteit geoefen word, met behulp van ander onderwerpe, voor 'n breinkaart-toets voorgelê word vir portefeulje-assessering.\par

%On completion of the metacog, learners must be able to answer insightful questions on the metacog. This is what sets it apart from being just a summary of a section of work. Learners must refer to their metacog when answering the questions, but may not refer to any reference material. Below are some guidelines to give to learners to adhere to when constructing a metacog as well as two examples to help you get learners started. A marking rubric is also provided. This should be made available to learners before they start constructing their metacogs. On the next page is a model question for a metacog, accompanied by some sample questions that can be asked within the context of doing a metacog about analytical geometry. \par
Na voltooiing van die beinkaart, moet die leerders in staat wees om insiggewende vrae daaroor te beantwoord. Dit is wat ‘n breinkaart onderskei van 'n gewone opsomming van 'n afdeling van die werk. Leerders moet verwys na hul breinkaart wanneer die vrae beantwoord word, maar mag nie verwys na enige verwysingsmateriaal nie. Hier is 'n paar riglyne om aan leerders te gee waaraan voldoen moet word wanneer ‘n breinkaart saamgestel word, sowel as twee voorbeelde om leerders te help om aan die gang te kom . 'n Assesseringsrubriek word ook voorsien. Dit moet beskikbaar gestel word aan die leerders voor hulle begin om hulle breinkaarte saam te stel. Op die volgende bladsy is 'n modelvraag vir 'n breinkaart, asook 'n paar voorbeelde van vrae wat gevra kan word binne die konteks van die samestelling van ‘n breinkaart oor analitiese meetkunde.\par

%A basic metacog is drawn in the following way:
'n Basiese breinkaart word op die volgende wyse saamgestel:
\begin{itemize}[noitemsep]
\item
% Write the title/topic of the subject in the centre of the page and draw a circle around it.
Skryf die titel/onderwerp van die vak in die middel van die bladsy en trek 'n sirkel daar rondom.
\item
% For the first main heading of the subject, draw a line out from the circle in any direction, and write the heading above or below the line.
  Vir die eerste hoofopskrif van die onderwerp, trek 'n streep uit die sirkel in enige rigting, en skryf die opskrif bo of onder die lyn.
\item
% For sub-headings of the main heading, draw lines out from the first line for each subheading and label each one. 
 Vir die sub-opskrifte van die hoofopskrif, trek lyne vir elke onderafdeling uit die eerste lyn en benoem elkeen.
\item
% For individual facts, draw lines out from the appropriate heading line. 
 Vir individuele feite, trek lyne uit die toepaslike opskrif.
\end{itemize}


%Metacogs are one’s own property. Once a person understands how to assemble the basic structure they can develop their own coding and conventions to take things further, for example to show linkages between facts. The following suggestions may assist educators and learners to enhance the effectiveness of their metacogs:
’n Metacog (breinkaart) is 'n mens se persoonlike eiendom. Sodra iemand verstaan hoe om die basiese struktuur saam te stel, kan hulle hulle eie kodering en konvensies ontwikkel om dinge verder te neem, byvoorbeeld hoe om verbande en skakels tussen feite aan te toon. Die volgende wenke kan opvoeders en leerders help om die doeltreffendheid van hul breinkaarte  te verbeter:

\begin{itemize}[noitemsep]
\item
% Use single words or simple phrases for information. Excess words just clutter the metacog and take extra time to write down.
Gebruik die enkele woorde of eenvoudige frases vir meer inligting. Oortollige woorde kompliseer die breinkaart en neem onnodige tyd om neer te skryf.
\item
% Print words – joined up or indistinct writing can be more difficult to read and less attractive to look at. 
Gebruik drukskrif: aanmekaar- of onduidelike skrif lees moeiliker en is minder aantreklik om na te kyk.
\item
% Use colour to separate different ideas – this will help your mind separate ideas where it is necessary, and helps visualisation of the metacog for easy recall. Colour also helps to show organisation.
 Gebruik kleur om verskillende idees te skei - dit sal jou brein help om verskillende idees van mekaar te onderskei waar nodig, en help met visualisering van die breinkaart vir maklike herroepping. Kleur help ook om organisasie te wys.
\item
% Use symbols and images where applicable. If a symbol means something to you, and conveys more information than words, use it. Pictures also help you to remember information.
Gebruik simbole en diagramme/sketse waar van toepassing. As 'n simbool iets vir jou beteken en meer inligting oordra as woorde, gebruik dit. Prente/foto's help ook om inligting te onthou.
\item
% Use shapes, circles and boundaries to connect information – these are additional tools to help show the grouping of information.
 Gebruik vorms, sirkels en raampies om brokkies inligting met mekaar te verbind - dit is 'n bykomende hulpmiddels om die groepering van inligting voor te stel.
\end{itemize}
%Use the concept of analytical geometry as your topic and construct a
%mind map (or metacog) containing all the information (including
%terminology, definitions, formulae and examples) that you know about
%the topic of analytical geometry. \par
Gebruik die konsep van analitiese meetkunde as jou onderwerp en konstrueer 'n breinkaart (of metacog) met al die inligting (insluitend terminologie, definisies, formules en voorbeelde) wat jy ken oor die onderwerp van analitiese meetkunde.

%Possible questions to ask the learner on completion of their metacog: 
Moontlike vrae om die leerder te vra na voltooiing van die breinkaart:
\begin{itemize}
\item
% Briefly explain to me what the mathematics topic of analytical geometry entails.
Verduidelik kortliks wat die wiskunde onderwerp van analitiese meetkunde behels.
\item
% Identify and explain the distance formula, the derivation and use thereof for me on your metacog.
Identifiseer en verduidelik die afstandformule, die afleiding daarvan en die gebruik daarvan op jou breinkaart.
\item
% How does the calculation of gradient in analytical geometry differ (or not) from the approach used to calculate gradient in working with functions? 
Hoe verskil of stem die berekening van gradiënt in analitiese meetkunde ooreen met die benadering wat gebruik word om die gradiënt te bereken wanneer daar met funksies gewerk word?
\end{itemize}
%A suggested simple rubric for marking a metacog:
'n Voorgestelde eenvoudige rubriek vir nasien van 'n breinkaart:
\begin{table}[H]
 \begin{center}
  \begin{tabular}{|p{3cm}|p{2.5cm}|p{2.5cm}|p{3cm}|} \hline
  \textbf{Taak} & \textbf{Bevoegd \newline(2 Punte)} & \textbf{In ontwikkeling \newline(1 Punt)}& \textbf{Nog nie ontwikkel \newline 1 Punt)}\\ \hline
Betyds voltooi &&&\\ \hline
Hoofopskrifte &&&\\ \hline
Korrekte teorie (Formules, definisies, terminologie, ens.) &&&\\ \hline
Verduideliking &&&\\ \hline
``Leesbaarheid''&&&\\ \hline

  \end{tabular}

 \end{center}

\end{table}

%10 marks for the questions, which are marked using the following scale:\\
10 punte vir die vrae wat gemerk is met behulp van die volgende skaal:\\
%0	-	no attempt or a totally incorrect attempt has been made \\
0: geen poging of 'n totaal verkeerde poging is gemaak\\
%1	-	a correct attempt was made, but the learner did not get the correct answer \\
1: 'n korrekte poging is aangewend, maar die leerder het nie die korrekte antwoord gekry nie\\
%2	-	a correct attempt was made and the answer is correct\\
2: 'n korrekte poging is aangewend en die antwoord is korrek

\subsubsection{Ondersoeke}
%Investigations consist of open-ended questions that initiate and expand thought processes. Acquiring and developing problem-solving skills are an essential part of doing investigations. \par 
Ondersoeke bestaan uit oop vrae wat denkprosesse inisieer en uitbrei. Die aanleer en ontwikkeling van probleemoplossingsvaardighede is 'n noodsaaklike deel van die uitvoer van ondersoeke.\par
%It is suggested that 2 – 3 hours be allowed for this task.  During the first 30 – 45 minutes learners could be encouraged to talk about the problem, clarify points of confusion, and discuss initial conjectures with others. The final written-up version should be done individually though and should be approximately four pages. \par 
Dit word voorgestel dat 2 - 3 uur toegelaat word vir hierdie taak. Gedurende die eerste 30 - 45 minute behoort leerders aangemoedig te word om te praat oor die probleem, punte van verwarring uit te klaar, en die aanvanklike hipoteses met ander te bespreek. Die finale skriftelike weergawe moet individueel gedoen word en moet ongeveer vier bladsye beslaan.\par

%Assessing investigations may include feedback/ presentations from groups or individuals on the results keeping the following in mind:
Assessering van ondersoeke kan terugvoer of voordragte deur groepe of individue oor die resultate insluit, terwyl die volgende in gedagte gehou word:
\begin{itemize}[noitemsep]
\item
% following of a logical sequence in solving the problems
   gebruik ‘n logiese volgorde in die oplossing van probleme
\item
% pre-knowledge required to solve the problem
   die pre-kennis wat nodig is om die probleem op te los
\item
% correct usage of mathematical language and notation
  die korrekte gebruik van wiskundige taal en notasie
\item
% purposefulness of solution
  doelgerigtheid van die oplossing
\item
% quality of the written and oral presentation 
  die kwaliteit van die geskrewe en mondelinge aanbieding
\end{itemize}
%Some examples of suggested marking rubrics are included on the next few pages, followed by a selection of topics for possible investigations. \par
Enkele voorbeelde van voorgestelde assesseringskale ingesluit is op die volgende paar bladsye, gevolg deur 'n seleksie van onderwerpe vir moontlike ondersoeke.

%The following guidelines should be provided to learners before they begin an investigation: \par
Die volgende riglyne moet aan leerders voorsien word voordat hulle 'n ondersoek begin:

\subsubsection{Algemene instruksies aan leerders}\\
\begin{itemize}[noitemsep]
\item
% You may choose any one of the projects/investigations given (see model question on investigations)
Jy kan enige van die gegewe projekte of ondersoeke kies (kyk modelvraag oor ondersoeke).
\item Volg die instruksies wat saam met elke taak gegee word noukeurig aangesien dit beskryf hoe die finale produk aangebied moet word.
\item Julle mag die probleem in groepe bespreek om kwessies uit te klaar, maar elke individu moet sy/haar eie weergawe op skrif stel.
\item Kopiëring van mede-leerders sal meebring dat die taak gediskwalifiseer word.
\item Jou opvoeders is 'n bron van hulp vir jou, en al sal hulle nie antwoorde of oplossings verskaf nie, kan  hulle genader word vir wenke.\end{itemize}	

\subsubsection{Die aanbieding}\\
%The investigation is to be handed in on the due date, indicated to you by your educator. It should have as a minimum:
Die ondersoek moet ingehandig word op die datum deur jou opvoeder bepaal. Dit moet ten minste die volgende bevat:
\begin{itemize}[noitemsep]
\item
% A description of the problem.
  'n beskrywing van die probleem
\item
% A discussion of the way you set about dealing with the problem.
  'n bespreking van die manier waarop jy te werk gaan om die probleem te hanteer
\item
% A description of the final result with an appropriate justification of its validity.
  'n beskrywing van die finale resultaat met 'n toepaslike motivering oor die geldigheid van die oplossing
\item
% Some personal reflections that include mathematical or other lessons learnt, as well as the feelings experienced whilst engaging in the problem.
   persoonlike refleksies wat wiskundige of ander lesse wat geleer is, insluit, sowel as die gevoelens wat ervaar is in die ondersoek van die probleem.
\item
% The written-up version should be attractively and neatly presented on about four A4 pages.
  die geskrewe weergawe moet aantreklik en netjies aangebied word op sowat vier A4-bladsye.
\item
% Whilst the use of technology is encouraged in the presentation, the mathematical content and processes must remain the major focus.
  hoewel die gebruik van tegnologie in die aanbieding aangemoedig word, moet die wiskundige inhoud en prosesse die hooffokus bly.
\end{itemize}	
%Below are some examples of possible rubrics to use when marking investigations:\par
Hier is 'n paar voorbeelde van ‘n moontlike rubriek wat gebruik kan word vir die nasien van die ondersoek:


\begin{table}[H]
 \begin{center}
  \begin{tabular}{|p{3cm}|p{8.5cm}|} \hline
\textbf{Vlak van prestasie}& \textbf{Kriteria} \\ \hline
4 &
\begin{itemize}[noitemsep]
\item
% Contains a complete response.
  Bevat 'n volledige antwoord.
\item
% Clear, coherent, unambiguous and elegant explanation.
  Duidelike, samehangende, ondubbelsinnige en elegante verduideliking.
\item
% Includes clear and simple diagrams where appropriate.
  Sluit duidelike en eenvoudige diagramme in waar van toepassing.
\item
% Shows understanding of the question’s mathematical ideas and processes.
  Toon begrip van die vraag se wiskundige idees en prosesse.
\item
% Identifies all the important elements of the question.
   Identifiseer die belangrikste elemente van die vraag.
\item
% Includes examples and counter examples.
  Sluit voorbeelde en teenvoorbeelde in.
\item
% Gives strong supporting arguments.
   Gee sterk ondersteunende argumente.
\item
% Goes beyond the requirements of the problem. 
  Gaan verder as die vereistes van die probleem.
   \end{itemize} \\ \hline
3 & 
\begin{itemize}[noitemsep]
\item
% Contains a complete response.
  Bevat 'n volledige antwoord.
\item
% Explanation less elegant, less complete.
  Verduideliking minder elegant, minder volledig.
\item
% Shows understanding of the question’s mathematical ideas and processes.
  Toon begrip van die vraag se wiskundige idees en prosesse.
\item
% Identifies all the important elements of the question.
  Identifiseer die belangrikste elemente van die vraag.
\item
% Does not go beyond the requirements of the problem.
  Gaan nie verder as die vereistes van die probleem nie.
\end{itemize} \\ \hline
2 &
\begin{itemize}[noitemsep]
\item
% Contains an incomplete response.
  Bevat 'n onvolledige antwoord.
\item
% Explanation is not logical and clear.
  Verduideliking is nie logies en duidelik nie.
\item
% Shows some understanding of the question’s mathematical ideas and processes.
  Toon 'n mate van begrip van die vraag se wiskundige idees en prosesse.
\item
% Identifies some of the important elements of the question.
  Identifiseer sommige van die belangrikste elemente van die vraag.
\item
% Presents arguments, but incomplete.
   Bied argumente aan, maar onvolledig.
\item
% Includes diagrams, but inappropriate or unclear.
  Sluit diagramme in, maar onvanpas of onduidelik.
\end{itemize} \\ \hline
1 &
\begin{itemize}[noitemsep]
\item
% Contains an incomplete response.
  Bevat 'n onvolledige antwoord.
\item
% Omits significant parts or all of the question and response.
  Laat belangrike dele van die vraag of die hele vraag en antwoord weg.
\item
% Contains major errors.
  Bevat groot foute.
\item
% Uses inappropriate strategies.
  Gebruik onvanpaste strategie\"{e}.
\end{itemize} \\ \hline
0 &
\begin{itemize}[noitemsep]
\item
% No visible response or attempt
Geen sigbare antwoord of poging.
\end{itemize} \\ \hline
  \end{tabular}

 \end{center}

\end{table}

\subsubsection{Mondelinge Assessering}
%An oral assessment involves the learner explaining to the class as a whole, a group or the educator his or her understanding of a concept, a problem or answering specific questions. The focus here is on the correct use of mathematical language by the learner and the conciseness and logical progression of their explanation as well as their communication skills.\par
'n Mondelinge assessering behels dat die leerder aan die hele klas of ‘n groep of die opvoeder verduidelik wat hy/sy verstaan van 'n konsep of ‘n probleem of spesifieke vrae beantwoord. Die fokus hier is op die korrekte gebruik van wiskundige taal deur die leerders; die bondigheid en logiese volgorde van die verduidelikings, sowel as hul kommunikasievaardighede.\par
%Orals can be done in a number of ways:
Mondelinge gedoen kan op 'n aantal maniere gedoen word:
\begin{itemize}[noitemsep]
\item
% A learner explains the solution of a homework problem chosen by the educator.
  'n Leerder verduidelik die oplossing van 'n ​​huiswerkprobleem wat deur die opvoeder gekies is.
\item
% The educator asks the learner a specific question or set of questions to ascertain that the learner understands, and assesses the learner on their explanation.
  Die opvoeder vra die leerder 'n spesifieke vraag of 'n stel vrae om seker te maak dat die leerder verstaan en evalueer die leerder op sy/haar verduideliking.
\item
% The educator observes a group of learners interacting and assesses the learners on their contributions and explanations within the group.
  Die opvoeder neem 'n groep leerders waar wat interaksie het met mekaar en assesseer die leerders op hul bydraes en verduidelikings binne die groep.
\item
% A group is given a mark as a whole, according to the answer given to a question by any member of a group.
  'n Punt word toegeken aan die groep as geheel, op grond van die antwoord wat enige lid van die groep gee op 'n vraag.
\end{itemize}
%An example of a marking rubric for an oral:\\
'n Voorbeeld van 'n merkrubriek vir 'n mondeling:
1 - die leerder het die vraag verstaan en probeer om dit te beantwoord\\
2 - die leerder gebruik die korrekte wiskundige taal\\
2 - die verklaring van die leerder volg 'n logiese progressie\\
2 - die leerder se verduideliking is bondig en akkuraat\\
2 - die leerder toon 'n begrip van die konsep wat verduidelik is\\
1 - die leerder demonstreer goeie kommunikasievaardighede\\
Maksimum punt = 10\\
\\

%An example of a peer-assessment rubric for an oral: \\
'n Voorbeeld van 'n eweknie-assesseringsrubriek vir 'n mondeling::\\
My naam: \underline{~~~~~~~~~~~~~~~~~~~~~~~~~~~~~~~~~~}\\
Naam van persoon wat ek assesseer: \underline{~~~~~~~~~~~~~~~~~~~~~~~~~~~~~~~~~~}\\

\begin{table}[H]
 \begin{center}
  \begin{tabular}{|p{5cm}|p{2.5cm}|p{2.5cm}|} \hline
  \textbf{Kriteria} & \textbf{Punt toegeken} & \textbf{Maksimum punt}\\ \hline
Korrekte antwoord &&2\\ \hline
Helderheid van verduideliking&&3\\ \hline
Korrektheid van verduideliking  &&3\\ \hline
Bewyse van begrip &&2\\ \hline
\textbf{Totaal} &&10\\ \hline

  \end{tabular}

 \end{center}

\end{table}

\section{Hoofstuk Kontekste}
\subsubsection{Algebra\"{i}ese uitdrukkings}
%Algebra provides the basis for mathematics learners to move from numerical calculations to generalising operations, simplifying expressions, solving equations and using graphs and inequalities in solving contextual problems. Being able to multiply out and factorise are core skills in the process of simplifying expressions and solving equations in mathematics. Identifying irrational numbers and knowing their estimated position on a number line or graph is an important part of any mathematical calculations and processes that move beyond the basic number system of whole numbers and integers. Rounding off irrational numbers (such as the value of $\pi$) when needed, allows mathematics learners to work more efficiently with numbers that would otherwise be difficult to “pin down”, use and comprehend.  \par
Algebra verskaf die grondslag vir wiskunde leerders om te beweeg van numeriese berekeninge na die veralgemening van bewerkings, vereenvoudiging van uitdrukkings, oplos van vergelykings en ongelykhede en die gebruik van grafieke in die oplos van kontekstuele probleme. Uitbreiding van hakies en faktorisering is die kernvaardighede in die proses van die vereenvoudiging van uitdrukkings en die oplos van vergelykings in wiskunde. Identifisering van irrasionale getalle en die kennis van hulle benaderde posisies op 'n getallelyn of grafiek is 'n belangrike deel van enige wiskundige berekeninge en prosesse wat verder beweeg as die basiese stelsel van heelgetalle. Afronding van irrasionale getalle (soos die waarde van π) wanneer dit nodig is, help wiskunde leerders om doeltreffend te werk met getalle wat andersins moeilik sou wees om akkuraat te bepaal, te gebruik en te verstaan.

\par
%Once learners have grasped the basic number system of whole numbers and integers, it is vital that their understanding of the numbers between integers is also expanded. This incorporates their dealing with fractions, decimals and surds which form a central part of most mathematical calculations in real-life contextual issues. Estimation is an extremely important component within mathematics. It allows learners to work with a calculator or present possible solutions while still being able to gauge how accurate and realistic their answers may be, which is relevant for other subjects too. 
Wanneer die leerders die basiese getallestelsel van heelgetalle onder die knie het, is dit noodsaaklik dat hulle begrip van die getalle tussen die heelgetalle ook uitgebrei word. Dit sluit bewerkings in met breuke, desimale getalle en wortels, wat 'n integrale deel uitmaak van die meeste wiskundige berekeninge in werklike lewe kontekste\par

Skatting is 'n uiters belangrike komponent van wiskunde. Dit stel leerders in staat om te werk met 'n sakrekenaar of om moontlike oplossings voor te stel, terwyl hy/sy ook in staat is om te oordeel hoe akkuraat en realisties die antwoorde is – ‘n vaardigheid wat relevant is vir ander vakke ook.

\subsubsection{Vergelykings en ongelykhede}
%If learners are to later work competently with functions and the graphing and interpretation thereof, their understanding and skills in solving equations and inequalities will need to be developed. 
As leerders later suksesvol wil werk met funksies en grafieke en die interpretasie daarvan, moet hulle begrip van en vaardighede in die oplos van vergelykings en ongelykhede goed ontwikkel word.
\subsubsection{Eksponente}
%Exponential notation is a central part of mathematics in numerical calculations as well as algebraic reasoning and simplification. It is also a necessary component for learners to understand and appreciate certain financial concepts such as compound interest and growth and decay. 
Eksponensiële notasie is 'n sentrale deel van wiskunde in numeriese berekenings asook in algebraïese beredenerings en vereenvoudigings. Dit is ook noodsaaklike vir leerders om sekere finansiële begrippe soos saamgestelde rente, vermeerdering en vermindering goed te verstaan.

\subsubsection{Getalpatrone}
%Much of mathematics revolves around the identification of patterns. In earlier grades learners saw patterns in the form of pictures and numbers. In this chapter  we look at the mathematics of patterns. Patterns are repetitive sequences and can be found in nature, shapes, events, sets of numbers and almost everywhere you care to look. For example, seeds in a sunflower, snowflakes, geometric designs on quilts or tiles, the number sequence $0; ~4;~ 8;~ 12; ~16; \ldots$
Baie van wiskunde wentel rondom die identifisering van patrone. Soos in vorige grade leerders sien patrone in die vorm van foto's en nommers. In hierdie hoofstuk gaan ons kyk na die wiskunde van patrone. Patrone is herhalende volgordes, en kan gevind word in die natuur, vorms, gebeure, stel van getalle en amper oral waar jy omgee om te kyk. Byvoorbeeld, sade in 'n sonneblom, sneeuvlokkies, geometriese ontwerpe op quilts of te\"{e}ls, die nommer volgorde $0; ~4;~ 8;~ 12; ~16; \ldots$

\subsubsection{Funksies}
%Functions form a core part of learners’ mathematical understanding and reasoning processes in algebra. This is also an excellent opportunity for contextual mathematical modelling questions. Functions are mathematical building blocks for designing machines, predicting natural disasters, curing diseases, understanding world economies and for keeping aeroplanes in the air. A useful advantage of functions is that they allow us to visualise relationships in terms of a graph. Functions are much easier to read and interpret than lists of numbers. In addition to their use in the problems facing humanity, functions also appear on a day-to-day level, so they are worth learning about. A function is always dependent on one or more variables, like time, distance or a more abstract  quantity.
Funksies vorm 'n sentrale deel van die leerders se wiskundige begrip van en redenasieprosesse in algebra. Dit bied ook 'n uitstekende geleentheid vir kontekstuele wiskundige modelleringsvrae. Funksies is wiskundige boustene vir die ontwerp van masjiene, die voorspelling van natuurrampe, genesing van siektes, begrip van die ekonomieë in die wêreld en om vliegtuie in die lug te hou. 'n Nuttige voordeel van funksies is dat hulle ons toelaat om verhoudings deur middel van 'n grafiek te visualiseer. Die grafiese voorstelling van funksies is baie makliker om te lees en te interpreteer as ‘n lys van getalle. Benewens die gebruik daarvan in die probleme van die mensdom, verskyn funksies ook op 'n dag-tot-dag vlak, sodat dit die moeite werd is om van hulle te leer. Die waarde van 'n funksie is altyd afhanklik van een of meer veranderlikes, soos tyd, afstand of ‘n meer abstrakte hoeveelheid.

\subsubsection{Finansies en groei}
%The mathematics of finance is very relevant to daily and long-term financial decisions learners will need to take in terms of investing, taking loans, saving and understanding exchange rates and their influence more globally.
Die wiskunde van finansies is baie relevant vir die daaglikse en langtermyn finansiële besluite wat leerders sal moet neem ten opsigte van beleggings, die uitneem van lenings, spaar en die begrip van wisselkoerse en hulle globale invloed.

\subsubsection{Trigonometrie}
%There are many applications of trigonometry. Of particular value is the technique of triangulation, which is used in astronomy to measure the distances to nearby stars, in geography to measure distances between landmarks, and in satellite navigation systems. GPS (the global positioning system) would not be possible without trigonometry. Other fields which make use of trigonometry include acoustics, optics, analysis of financial markets, electronics, probability theory, statistics, biology, medical imaging (CAT scans and ultrasound), chemistry, cryptology, meteorology, oceanography, land surveying, architecture, phonetics, engineering, computer graphics and game development.
Daar is baie toepassings van trigonometrie. Van besondere waarde is die tegniek van triangulering, wat in sterrekunde gebruik is om die afstande te meet na die nabygeleë sterre, in geografie om afstande tussen landmerke te meet, en satellietnavigasie, GPS (Global Positioning System), sou nie moontlik wees sonder trigonometrie nie. Ander velde wat gebruik maak van trigonometrie sluit in akoestiek,
optika, ontleding van finansiële markte, elektronika, waarskynlikheidsleer, statistiek, biologie, mediese beelding (CAT-skanderings en ultraklank), chemie, kriptologie, weerkunde, oseanografie, landmeting, argitektuur, fonetiek, ingenieurswese, rekenaargrafika en spelontwikkeling.

\subsubsection{Analitiese meetkunde}
%This section provides a further application point for learners’ algebraic and trigonometric interaction with the Cartesian plane. Artists and design and layout industries often draw on the content and thought processes of this mathematical topic.
Hierdie afdeling verskaf 'n verdere toepassing vir leerders se algebraïese en trigonometriese interaksie met die Cartesiese vlak. Kunstenaars, en die ontwerp- en uitlegindustrie maak dikwels gebruik van die inhoud en denkprosesse van hierdie wiskundige onderwerp.

\subsubsection{Statistiek}
%Citizens are daily confronted with interpreting data presented from the media. Often this data may be biased or misrepresented within a certain context. In any type of research, data collection and handling is a core feature. This topic also educates learners to become more socially and politically educated with regards to the media.
Burgers word daagliks gekonfronteer met die interpretasie van data wat aangebied word deur die media. Hierdie data mag soms bevooroordeeld wees, of verkeerd geïnterpreteer word binne 'n sekere konteks. In enige soort navorsing is die insameling en hantering van data 'n kernfunksie. Hierdie onderwerp bied ook onderrig aan leerders ten einde hulle sosiaal en polities geletterd te maak ten opsigte van die media.

\subsubsection{Waarskynlikheid}
%This topic is helpful in developing good logical reasoning in learners and for educating them in terms of real-life issues such as gambling and the possible pitfalls thereof. We use probability to describe uncertain events: when you accidentally drop a slice of bread, you don’t know if it’s going to fall with the buttered side facing upwards or downwards. When your favourite sports team plays a game, you don’t know whether they will win or not. When the weatherman says that there is a $40\%$ chance of rain tomorrow, you may or may not end up getting wet. Uncertainty presents itself to some degree in every event that occurs around us and in every decision that we make.
Hierdie onderwerp is nuttig in die ontwikkeling van goeie logiese beredenering in die leerders en vir die opvoeding van hulle in terme van die werklike lewe kwessies soos dobbelary en die moontlike slaggate daarvan. Ons maak gebruik van waarskynlikheid onseker gebeure te beskryf: as jy per ongeluk laat val 'n sny brood, jy weet nie of dit gaan om te val met die gesmeerde kant opwaarts of afwaarts. Wanneer jou gunsteling sport span speel 'n speletjie, jy weet nie of hulle wen of nie. Wanneer die weerman s\^{e} daar is 'n 40\%-kans op re\"{e}n m\^{o}re, jy mag of nie mag nie die einde om nat te word. Onsekerheid bied self tot 'n mate in elke gebeurtenis wat rondom ons en in elke besluit wat maak ons.

\subsubsection{Euklidiese meetkunde}
%The thinking processes and mathematical skills of proving conjectures and identifying false conjectures is more the relevance here than the actual content studied. The surface area and volume content studied in real-life contexts of designing kitchens, tiling and painting rooms, designing packages, etc. is relevant to the current and future lives of learners. Euclidean geometry deals with space and shape using a system of logical deductions.
Die denkprosesse en wiskundige vaardighede in die bewys van veronderstellings en die identifisering van valse veronderstellings is meer relevant hier as die werklike inhoud wat bestudeer word. Die oppervlakte en volume inhoude wat bestudeer word in die regte-lewe kontekste van die ontwerp van kombuise, teëlwerk, die verf van kamers, die ontwerp van pakkette ens., is uiters relevant vir die huidige en toekomstige lewens van leerders. Euklidiese meetkunde hanteer ruimte en vorm met behulp van 'n stelsel van logiese afleidings.

\subsubsection{Meting}
%This chapter revises the volume and surface areas of three-dimensional objects, otherwise known as solids. The chapter covers the volume and surface area of prisms and cylinders, Many exercises cover finding the surface area and volume of polygons, prisms, pyramids, cones and spheres, as well as a complex object. The effect on volume and surface area when multiplying a dimension of a factor of $k$ is also explored.
Hierdie hoofstuk hersien die volume en buite-oppervlakte van drie-dimensionele voorwerpe, ook bekend as vaste liggame. Die hoofstuk dek die volume en oppervlakte van prismas en silinders. Oefeninge dek die berekening van die oppervlakte en volume van veelhoeke, prismas, piramides, keëls en sfere, sowel as saamgestelde voorwerpe. Die invloed van die vermenigvuldiging van een of meer afmetings van ‘n voorwerp met 'n faktor ($k$) word ook ondersoek.

\subsubsection{Oplossings van oefeninge}
%This chapter includes the solutions to the exercises covered in each chapter of the book.
Hierdie hoofstuk sluit die oplossings in van die oefeninge wat in elke hoofstuk van die boek gegee word.