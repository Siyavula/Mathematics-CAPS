\chapter{Front matter}

\section{Support for educators}
Science education is about more than physics, chemistry and mathematics... It's about learning to think and to solve problems, which are valuable skills that can be applied through all spheres of life. Teaching these skills to our next generation is crucial in the current global environment where methodologies, technology and tools are rapidly evolving. Education should benefit from these fast moving developments. In our simplified model there are three layers to how technology can significantly influence your teaching and teaching environment. 

\subsubsection{First Layer: educator collaboration}
There are many tools that help educators collaborate more effectively. We know that communities of practice are powerful tools for the refinement of methodology, content and knowledge and are also superb for providing support to educators. One of the challenges facing community formation is the time and space to have sufficient meetings to build real communities and exchange practices, content and learnings effectively. Technology allows us to streamline this very effectively by transcending space and time. It is now possible to collaborate over large distances (transcending space) and when it is most appropriate for each individual (transcending time) by working virtually (email, mobile, online etc.).\par

Our textbooks have been re-purposed from content available on the Connexions website\\ (\underline{http://cnx.org/lenses/fhsst}). The content on this website is easily accessible and adaptable  as it is under an open licence, stored in an open format, based on an open standard, on an open-source platform, for free, where everyone can produce their own books. The content on Connexions is available under an open copyright license - CC-BY. This Creative Commons By Attribution Licence allows others to legally distribute, remix, tweak, and build upon the work available, even commercially, as long as the original author is credited for the original creation. This means that learners and educators are able to download, copy, share and distribute this content legally at no cost. It also gives educators the freedom to edit, adapt, translate and contextualise it, to better suit their teaching needs. \par

Connexions is a tool where individuals can share, but more importantly communities can form around the collaborative, online development of resources. Your community of educators can therefore:
\begin{itemize}[noitemsep]
\item form an online workgroup around the content;
\item make your own copy of the content;
\item edit sections of your own copy;
\item add your own content or replace existing content with your own content;
\item use other content that has been shared on the platform in your own work;
\item create your own notes / textbook / course material as a community.
\end{itemize}
Educators often want to share assessment items as this helps reduce workload, increase variety and improve quality. Currently all the solutions to the exercises contained in the textbooks have been uploaded onto our free and open online assessment bank called Monassis \underline{(www.monassis.com)}, with each exercise having a shortcode link to its solution on Monassis. To access the solution simply go to \underline{www.everythingmaths.co.za},  enter the shortcode, and you will be redirected to the solution on Monassis.\par

Monassis is similar to Connexions but is focused on the sharing of assessment items. Monassis contains a selection of test and exam questions with solutions, openly shared by educators. Educators can further search and browse the database by subject and grade and add relevant items to a test. The website automatically generates a test or exam paper with the corresponding memorandum for download.\par

By uploading all the end-of-chapter exercises and solutions to this open assessment bank, the larger community of educators in South Africa are provided with a wide selection of items to use in setting their tests and exams. More details about the use of Monassis as a collaboration tool are included in the Monassis section.\par

\subsubsection{Second Layer: classroom engagement}
In spite of the impressive array of rich media open educational resources available freely online (such as videos, simulations, exercises and presentations), only a small number of educators actively make use of them. Our investigations revealed that the overwhelming quantity, the predominant international context, and difficulty in correctly aligning them with the local curriculum level acts as deterrents. The opportunity here is that, if used correctly, they can make the classroom environment more engaging.\par

Presentations can be a first step to bringing material to life in ways that are more compelling than are possible with just a blackboard and chalk. There are opportunities to:
\begin{itemize}[noitemsep]
\item create more graphical representations of the content;
\item control timing of presented content more effectively;
\item allow learners to relive the lesson later if constructed well;
 \item supplement the slides with notes for later use;
\item embed key assessment items in advance to promote discussion; and
\item embed other rich media like videos.
\end{itemize}
Videos have been shown to be potentially both engaging and effective. They provide opportunities to:
\begin{itemize}[noitemsep]
\item present an alternative explanation;
\item challenge misconceptions without challenging an individual in the class; and
\item show an environment or experiment that cannot be replicated in the class which could be far away, too expensive or too dangerous.
\end{itemize}
Simulations are also very useful and can allow learners to:
\begin{itemize}[noitemsep]
\item have increased freedom to explore, rather than reproduce a fixed experiment or process;
\item explore expensive or dangerous environments more effectively; and
\item overcome implicit misconceptions.
\end{itemize}
We realised the opportunity for embedding a selection of rich media resources such as presentations, simulations, videos and links into the online version of Everything Maths and Everything Science at the relevant sections. This will not only present them with a selection of locally relevant and curriculum aligned resources, but also position these resources within the appropriate grade and section. Links to these online resources are recorded in the print or PDF versions of the  books, making them a tour-guide or credible pointer to the world of online rich media available. \par
\subsubsection{Third Layer: beyond the classroom}
The internet has provided many opportunities for self-learning and participation which were never before possible. There are huge stand-alone archives of videos like the Khan Academy which covers most Mathematics for Grades 1 - 12 and Science topics required in FET. These videos, if not used in class, provide opportunities for the learners to:
\begin{itemize}[noitemsep]
\item look up content themselves;
\item get ahead of class;
\item independently revise and consolidate their foundation; and
\item explore a subject to see if they find it interesting.
\end{itemize}
There are also many opportunities for learners to participate in science projects online as real participants (see the section on citizen cyberscience “On the web, everyone can be a scientist”). Not just simulations or tutorials but real science so that:
\begin{itemize}[noitemsep]
\item learners gain an appreciation of how science is changing;
\item safely and easily explore subjects that they would never have encountered before university;
\item contribute to real science (real international cutting edge science programmes);
\item have the possibility of making real discoveries even from their school computer laboratory; and
\item find active role models in the world of science.
\end{itemize}
In our book we've embedded opportunities to help educators and learners take advantage of all these resources, without becoming overwhelmed at all the content that is available online. 

\subsubsection{Annotation of the book}
If you have any comments, thoughts or suggestions on the books, you can visit  \underline{www.everythingmaths.co.za} and in educator mode, capture these in the text. These can range from sharing tips and ideas with your fellow educators, to discussing how to better explain concepts in class. Also, if you have picked up any errors in the book you can make a note of them here, and we will correct them in time for the next print run.\par

\section{On the Web, everyone can be a scientist}
Did you know that you can fold protein molecules, hunt for new planets around distant suns or simulate how malaria spreads in Africa, all from an ordinary PC or laptop connected to the Internet? And you don’t need to be a certified scientist to do this. In fact some of the most talented contributors are teenagers. The reason this is possible is that scientists are learning how to turn simple scientific tasks into competitive online games. \par

This is the story of how a simple idea of sharing scientific challenges on the Web turned into a global trend, called citizen cyberscience. And how you can be a scientist on the Web, too.

\subsubsection{Looking for Little Green Men}
A long time ago, in 1999, when the World Wide Web was barely ten years old and no one had heard of Google, Facebook or Twitter, a researcher at the University of California at Berkeley, David Anderson, launched an online project called SETI@home. SETI stands for Search for Extraterrestrial Intelligence. Looking for life in outer space.\par

Although this sounds like science fiction, it is a real and quite reasonable scientific project. The idea is simple enough. If there are aliens out there on other planets, and they are as smart or even smarter than us, then they almost certainly have invented the radio already. So if we listen very carefully for radio signals from outer space, we may pick up the faint signals of intelligent life.\par

Exactly what radio broadcasts aliens would produce is a matter of some debate. But the idea is that if they do, it would sound quite different from the normal hiss of background radio noise produced by stars and galaxies. So if you search long enough and hard enough, maybe you’ll find a sign of life. 
\par
It was clear to David and his colleagues that the search was going to require a lot of computers. More than scientists could afford. So he wrote a simple computer program which broke the problem down into smaller parts, sending bits of radio data collected by a giant radio-telescope to volunteers around the world. The volunteers agreed to download a programme onto their home computers that would sift through the bit of data they received, looking for signals of life, and send back a short summary of the result to a central server in California. \par

The biggest surprise of this project was not that they discovered a message from outer space. In fact, after over a decade of searching, no sign of extraterrestrial life has been found, although there are still vast regions of space that have not been looked at. The biggest surprise was the number of people willing to help such an endeavour. Over a million people have downloaded the software, making the total computing power of SETI@home rival that of even the biggest supercomputers in the world.

David was deeply impressed by the enthusiasm of people to help this project. And he realized that searching for aliens was probably not the only task that people would be willing to help with by using the spare time on their computers. So he set about building a software platform that would allow many other scientists to set up similar projects. You can read more about this platform, called BOINC, and the many different kinds of volunteer computing projects it supports today, at \underline{http://boinc.berkeley.edu/}. \par

There’s something for everyone, from searching for new prime numbers (PrimeGrid) to simulating the future of the Earth’s climate (ClimatePrediction.net). One of the projects, MalariaControl.net, involved researchers from the University of Cape Town as well as from universities in Mali and Senegal. \par

The other neat feature of BOINC is that it lets people who share a common interest in a scientific topic share their passion, and learn from each other. BOINC even supports teams – groups of people who put their computer power together, in a virtual way on the Web, to get a higher score than their rivals. So BOINC is a bit like Facebook and World of Warcraft combined – part social network, part online multiplayer game.\par

Here’s a thought: spend some time searching around BOINC for a project you’d like to participate in, or tell your class about. 

\subsubsection{You are a computer, too}
Before computers were machines, they were people. Vast rooms full of hundreds of government employees used to calculate the sort of mathematical tables that a laptop can produce nowadays in a fraction of a second. They used to do those calculations laboriously, by hand. And because it was easy to make mistakes, a lot of the effort was involved in double-checking the work done by others. \par

Well, that was a long time ago. Since electronic computers emerged over 50 years ago, there has been no need to assemble large groups of humans to do boring, repetitive mathematical tasks. Silicon chips can solve those problems today far faster and more accurately. But there are still some mathematical problems where the human brain excels.\par

Volunteer computing is a good name for what BOINC does: it enables volunteers to contribute computing power of their PCs and laptops. But in recent years, a new trend has emerged in citizen cyberscience that is best described as volunteer thinking. Here the computers are replaced by brains, connected via the Web through an interface called eyes. Because for some complex problems – especially those that involve recognizing complex patterns or three-dimensional objects – the human brain is still a lot quicker and more accurate than a computer.\par

Volunteer thinking projects come in many shapes and sizes. For example, you can help to classify millions of images of distant galaxies (GalaxyZoo), or digitize hand-written information associated with museum archive data of various plant species (Herbaria@home).  This is laborious work, which if left to experts would take years or decades to complete. But thanks to the Web, it’s possible to distribute images so that hundreds of thousands of people can contribute to the search. \par

Not only is there strength in numbers, there is accuracy, too. Because by using a technique called validation – which does the same sort of double-checking that used to be done by humans making mathematical tables – it is possible to practically eliminate the effects of human error. This is true even though each volunteer may make quite a few mistakes. So projects like Planet Hunters have already helped astronomers pinpoint new planets circling distant stars. The game FoldIt invites people to compete in folding protein molecules via a simple mouse-driven interface. By finding the most likely way a protein will fold, volunteers can help understand illnesses like Alzheimer’s disease, that depend on how proteins fold. \par

Volunteer thinking is exciting. But perhaps even more ambitious is the emerging idea of volunteer sensing: using  your laptop or even your mobile phone to collect data – sounds, images, text you type in – from any point on the planet, helping scientists to create global networks of sensors that can pick up the first signs of an outbreak of a new disease (EpiCollect), or the initial tremors associated with an earthquake (QuakeCatcher.net), or the noise levels around a new airport (NoiseTube).\par

There are about a billion PCs and laptops on the planet, but already 5 billion mobile phones. The rapid advance of computing technology, where the power of a ten-year old PC can easily be packed into a smart phone today, means that citizen cyberscience has a bright future in mobile phones. And this means that more and more of the world’s population can be part of citizen cyberscience projects. Today there are probably a few million participants in a few hundred citizen cyberscience initiatives. But there will soon be seven billion brains on the planet. That is a lot of potential citizen cyberscientists. 

You can explore much more about citizen cyberscience on the Web. There’s a great list of all sorts of projects, with brief summaries of their objectives, at \underline{http://distributedcomputing.info/} . BBC Radio 4 produced a short series on citizen science \underline{http://www.bbc.co.uk/radio4/science/citizenscience.shtml}  and you can subscribe to a newsletter about the latest trends in this field at \underline{http://scienceforcitizens.net/}  . The Citizen Cyberscience Centre, \underline{www.citizencyberscience.net}  which is sponsored by the South African Shuttleworth Foundation, is promoting citizen cyberscience in Africa and other developing regions.

\section{Blog posts}
\subsubsection{General blogs}
\begin{itemize}
\item Teachers Monthly - Education News and Resources
    \begin{itemize}[noitemsep]
      \item “We eat, breathe and live education! “
      \item “Perhaps the most remarkable yet overlooked aspect of the South African teaching community is its enthusiastic, passionate spirit. Every day, thousands of talented, hard-working teachers gain new insight from their work and come up with brilliant, inventive and exciting ideas. Teacher’s Monthly aims to bring teachers closer and help them share knowledge and resources.
      \item Our aim is twofold...
	    \begin{itemize}[noitemsep]
	      \item To keep South African teachers updated and informed.
	    \item To give teachers the opportunity to express their views and cultivate their interests.”
	    \end{itemize}
      \item \underline{http://www.teachersmonthly.com }
    \end{itemize}

\item Head Thoughts – Personal Reflections of a School Headmaster
    \begin{itemize}[noitemsep]
	\item blog by Arthur Preston
	\item “Arthur is currently the headmaster of a growing independent school in Worcester, in the Western Cape province of South Africa. His approach to primary education is progressive and is leading the school through an era of new development and change.”
\item \underline{http://headthoughts.co.za/ }
    \end{itemize}
\end{itemize}

\subsubsection{Maths blogs}
\begin{itemize}

\item CEO: Circumspect Education Officer - Educating The Future
\begin{itemize} [noitemsep]
 \item blog by Robyn Clark
\item “Mathematics teacher and inspirer.”
\item \underline{http://clarkformaths.tumblr.com/ }
\end{itemize}

\item dy/dan - Be less helpful
\begin{itemize} [noitemsep]
\item blog by Dan Meyer
\item “I'm Dan Meyer. I taught high school math between 2004 and 2010 and I am currently studying at Stanford University on a doctoral fellowship. My specific interests include curriculum design (answering the question, "how we design the ideal learning experience for students?") and teacher education (answering the questions, "how do teachers learn?" and "how do we retain more teachers?" and "how do we teach teachers to teach?").”
\item \underline{http://blog.mrmeyer.com }
\end{itemize}

\item Without Geometry, Life is Pointless - Musings on Math, Education, Teaching, and Research

\begin{itemize}[noitemsep]
 \item blog by Avery
\item “I've been teaching some permutation (or is that combination?) of math and science to third through twelfth graders in private and public schools for 11 years. I'm also pursuing my EdD in education and will be both teaching and conducting research in my classroom this year.”
\item \underline{ http://mathteacherorstudent.blogspot.com/ }
\end{itemize}

\item Overthinking my teaching - The Mathematics I Encounter in Classrooms
\begin{itemize}[noitemsep]
\item blog by Christopher Danielson
\item “I think a lot about my math teaching. Perhaps too much. This is my outlet. I hope you find it interesting and that you’ll let me know how it’s going.”
\item \underline{http://christopherdanielson.wordpress.com}
\end{itemize}

\item A Recursive Process - Math Teacher Seeking Patterns
\begin{itemize}[noitemsep]
\item blog by Dan
\item “I am a High School math teacher in upstate NY. I currently teach Geometry, Computer Programming (Alice and Java), and two half year courses: Applied and Consumer Math. This year brings a new 21st century classroom (still not entirely sure what that entails) and a change over to standards based grades (#sbg).”
\item \underline{http://dandersod.wordpress.com }
\end{itemize}

\item Think Thank Thunk – Dealing with the Fear of Being a Boring Teacher 
\begin{itemize} [noitemsep]
\item blog by Shawn Cornally
\item “I am Mr. Cornally. I desperately want to be a good teacher. I teach Physics, Calculus, Programming, Geology, and Bioethics. Warning: I have problem with using colons. I proof read, albeit poorly.”
\item \underline{http://101studiostreet.com/wordpress/}
\end{itemize}
\end{itemize}

\section{Oorsig}
\subsubsection{Kurrikulum Oorsig}
% Before 1994 there existed a number of education departments and
% subsequent curriculum according to the segregation that was so
% evident during the apartheid years.
Voor 1994 het daar 'n aantal verskillende onderwysdepartemente en
kurrikula bestaan volgens die skeiding wat so duidelik was tydens die
apartheid era.
% As a result, the curriculum itself became one of the political icons
% of freedom or suppression.
As 'n gevolg het die kurrikulum self een van die politiese ikone van
vryheid of onderdrukking geword.
% Since then the government and political leaders have sought to try
% and develop one curriculum that is aligned with our national agenda
% of democratic freedom and equality for all, in fore-grounding the
% knowledge, skills and values our country believes our learners need
% to acquire and apply, in order to participate meaningfully in
% society as citizens of a free country.
Sedertdien het die regering en politieke leiers probeer om een
kurrikulum te ontwikkel, wat die nasionale agenda van demokratiese
vryheid en gelykheid ondesteun, deur die kennis, vaardighede en
waardes wat ons leerders moet leer and toepas op die voorgrond te
stel, sodat hulle op 'n betekenisvolle manier kan deelneem in die
samelewing as burgers van 'n vry land.
% The National Curriculum Statement (NCS) of Grades R – 12 (DBE, 2012)
% therefore serves the purposes of: 
Die Nasionale Kurrikulumverklaring (NKV) Graad R--12 (DBE, 2012) dien
dus volgende doelwitte:
\begin{itemize}
\item
% equipping learners, irrespective of their socio-economic background,
% race, gender, physical ability or intellectual ability, with the
% knowledge, skills and values necessary for self-fulfilment, and
% meaningful participation in society as citizens of a free country;
  om leerders toe te rus met die kennis, vaardighede end waardes
  benodig vir selfverwesenliking en betekenisvolle deelname in die
  samelewing as burgers van 'n vry land, ongeag hulle sosio-ekonomiese
  agtergrond, ras, geslag, fisiese of intellektuele vermo\"{e};
\item
% providing access to higher education; 
  om toegang to ho\"{e}r onderrig te verskaf;
\item
% facilitating the transition of learners from education institutions
% to the workplace; and
  om die oorgang van leerders vanaf onderwysinstellings na die
  werkplek te fasiliteer; en
\item
% providing employers with a sufficient profile of a learner’s
% competencies.
  om werkgewers met 'n voldoende profiel van leerdersbevoegdhede te
  verskaf.
\end{itemize}
% Although elevated to the status of political icon, the curriculum
% remains a tool that requires the skill of an educator in
% interpreting and operationalising this tool within the classroom.
Alhoewel dit verhef is tot die status van 'n politiese ikoon, bly die
kurrikulum 'n instrument. Die vaardighede van 'n onderwyser word
benodig om hierdie istrument te interpreteer en operasionaliseer in
die klaskamer.
% The curriculum itself cannot accomplish the purposes outlined above
% without the community of curriculum specialists, material
% developers, educators and assessors contributing to and supporting
% the process, of the intended curriculum becoming the implemented
% curriculum.
Die kurrikulum self kan nie die doelwitte hierbo gelys bereik sonder
dat die gemeenskap van kurrikulumspesialiste, ontwikkelaars van
onderwysmateriaal, onderwysers en assessore die prosess, om die
voorgenome kurrikulum die ge\"{i}mplementeerde kurrikulum te maak,
ondersteun en daartoe bydra nie.
% A curriculum can succeed or fail, depending on its implementation,
% despite its intended principles or potential on paper.
'n Kurrikulum kan slaag of misluk, afhangende van die implementering
en ongeag die voorgenome beginsels of potensiaal op papier daarvan.
% It is therefore important that stakeholders of the curriculum are
% familiar with and aligned to the following principles that the NCS
% is based on:
Daarom is dit belangrik dat belanghebbendes van die kurrikulum
vertroud is en ooreenstem met die volgende beginsels waarop die NKV
gebaseer is:

\begin{table}[H]
\begin{center}
\begin{tabular}{|p{6.5cm}|p{6.5cm}|} \hline
\textbf{Beginsel} &
\textbf{Implementering} \\ \hline
Sosiale Transformasie &
Die regstelling van wanbalanse van die verlede.\par
Die verskaffing van gelyke geleenthede vir almal.\\ \hline
Aktiewe en Kritiese Leer &
Aanmoediging van 'n aktiewe en kritiese benadering tot leer.\par
Vermyding van oormatige onkritiese memorisering van gegewe waarhede.\\ \hline
Diepgaande Kennis en Vaardighede &
Leerders behaal minimum standaarde van kennis en vaardighede, soos
bepaal vir elke graad in elke vak. \\ \hline
Vordering &
Inhoud en konteks toon progressie van eenvoudig na kompleks. \\ \hline
Sosiale en Omgewingsgeregtigheid en Menseregte &
Praktyke soos in die Grondwet omskryf is, verweef in die onderrig en
leer van elk van die vakke. \\ \hline
Waardering vir Inheemse Kennissisteme &
Erken die ryk geskiedenis en erfenis van hierdie land. \\ \hline
Geloofwaardigheid, Gehalte en Doeltreffendheid &
Verskaffing van onderrig wat w\^{e}reldwyd vergelykbaar is i.t.v.\@ kwaliteit. \\ \hline
\end{tabular}
\end{center}
\end{table}

% This guide is intended to add value and insight to the existing
% National Curriculum for Grade 10 Mathematics, in line with its
% purposes and principles.
Hierdie gids is bedoel om waarde en insig toe te voeg tot die
bestaande Nasionale Kurrikulum vir Graad 10 Wiskunde, in lyn met die
doelwitte en beginsels daarvan.
% It is hoped that this will assist you as the educator in optimising
% the implementation of the intended curriculum.
Daar word gehoop dat dit u as die opvoeder sal help om die voorgenome
kurrikulum te optimeer en implementeer.

\subsubsection{Kurrikulumvereistes en doelwitte}
% The main objectives of the curriculum relate to the learners that
% emerge from our educational system.
Die belangrikste doelwitte van die kurrikulum hou verband met die
leerders wat uit ons opvoedkundestelsel kom.
% While educators are the most important stakeholders in the
% implementation of the intended curriculum, the quality of learner
% coming through this curriculum will be evidence of the actual
% attained curriculum from what was intended and then implemented.
Opvoeders is die belangrikste rolspelers in die uitvoering van die
voorgenome kurrikulum. Die kwaliteit van die leerder wat deur hierdie
stelsel beweeg, sal egter die bewys wees dat die kurrikulum soos wat
dit bedoel en ge\"{i}mplementeer is, ook sy doelwitte bereik het.

%These purposes and principles aim to produce learners that are able to: 
Hierdie doelwitte en beginsels beoog om leerders te produseer wat in
staat is:
\begin{itemize}[noitemsep]
\item
% identify and solve problems and make decisions using critical and
% creative thinking;
  om probleme te identifiseer en op te los en om besluite te neem deur
  kritiese en kreatiewe denke;
\item
% work effectively as individuals and with others as members of a team; 
  om doeltreffend te werk as individue en met ander as lede van 'n
  span;
\item
% organise and manage themselves and their activities responsibly and
% effectively; 
  om hulself en hul aktiwiteite verantwoordelik en doeltreffend te
  organiseer en bestuur;
\item
% collect, analyse, organise and critically evaluate information; 
  om inligting te versamel, te analiseer, te organiseer en krities te
  evalueer;
\item
% communicate effectively using visual, symbolic and/or language
% skills in various modes;
  om effektief te kommunikeer deur gebruik te maak van visuele,
  simboliese en/of taalvaardighede in verskillende vorme;
\item
% use science and technology effectively and critically showing
% responsibility towards the environment and the health of others; and
  om wetenskap en tegnologie doeltreffend en krities te gebruik met
  verantwoordelikheid teenoor die omgewing en die gesondheid van
  ander;
\item
% demonstrate an understanding of the world as a set of related
% systems by recognising that problem solving contexts do not exist in
% isolation.
  om begrip van die w\^{e}reld as 'n stel verwante stelsels te toon deur
  te herken dat die kontekste van probleme nie in isolasie bestaan
  nie.
\end{itemize}
% The above points can be summarised as an independent learner who can
% think critically and analytically, while also being able to work
% effectively with members of a team and identify and solve problems
% through effective decision making.
Die bogenoemde punte kan opgesom word as 'n onafhanklike leerder wat
krities en analities kan dink, in staat is om effektief met lede van
'n span te werk, en probleme kan identifiseer en oplos deur middel van
effektiewe besluitneming.
% This is also the outcome of what educational research terms the
% “reformed” approach rather than the “traditional” approach many
% educators are more accustomed to.
Dit is ook die uitkoms waarna binne opvoedkundige navorsing verwys
word as die ``hervormde'' benadering eerder as die
``tradisionele'' benadering waaraan baie opvoeders meer gewoond is.
% Traditional practices have their role and cannot be totally
% abandoned in favour of only reform practices.
Tradisionele praktyke het hul rol en kan nie heeltemal ten gunste van
hervormde praktyke daargelaat word nie.
% However, in order to produce more independent and mathematical
% thinkers, the reform ideology needs to be more embraced by educators
% within their instructional behaviour.
Maar, ten einde meer onafhanklike en wiskundige denkers te produseer,
moet die hervorming ideologie deur opvoeders ingeneem word in hul
optrede as onderwysers.
% Here is a table that can guide you to identify your dominant
% instructional practice and try to assist you in adjusting it (if
% necessary) to be more balanced and in line with the reform approach
% being suggested by the NCS.
Hier is 'n tabel wat kan help om u dominante instruksionele praktyk te
identifiseer en u probeer help om dit aan te pas (indien nodig), om
meer gebalanseerd en in lyn met die hervorming benadering, soos
voorgestel deur die NKV, te wees.

\begin{table}[H]
  \begin{center}
    \begin{tabular}{|p{3.5cm}|p{8.5cm}|} \hline 
&
\textbf{Tradisionele Versus Hervormde Praktyke} \\ \hline
Waardes &
\textbf{Tradisioneel} --- fokus op onderrigmateriaal, korrektheid van leerders se antwoorde en wiskundige geldigheid van metodes.\par
\textbf{Hervorm} --- patrone vind, konsepte verbind, wiskundig kommunikeer en probleemoplossing. \\ \hline
Onderrigmetodes &
\textbf{Tradisioneel} --- verklarend, oordrag van inligting, baie oefen en herhaling, stap vir stap bemeestering van algoritmes.\par
\textbf{Hervorm} --- Geleide ontdekkingsmetodes, eksplorasie, modellering. Ho\"{e} vlak van redenasie is sentraal. \\ \hline
Groepering van Leerders &
\textbf{Tradisioneel} --- oorheersend gelyksoortige groepering. \par
\textbf{Hervorm} --- oorheersend gemengde vermo\"{e}ns. \\ \hline
    \end{tabular}
  \end{center}
\end{table}

% The subject of mathematics, by the nature of the discipline, provides
% ample opportunities to meet the reformed objectives.
Die vak wiskunde verskaf uiter aard ruim geleentheid om te voldoen aan
die hervormde doelwitte.
% In doing so, the definition of mathematics needs to be understood
% and embraced by educators involved in the teaching and the learning
% of the subject.
Die definisie van wiskunde moet verstaan ​​en omhels moet word deur die
opvoeders betrokke by die onderrig en die leer van die vak.
% In research it has been well documented that, as educators, our
% conceptions of what mathematics is, has an influence on our approach
% to the teaching and learning of the subject.
In die navorsing is dit goed gedokumenteer dat ons opvattings oor wat
wiskunde is, 'n invloed het op ons benadering tot die onderrig en leer
van die vak.

% Three possible views of mathematics can be presented.
Drie sienings van wiskunde word hier aangebied.
% The instrumentalist view of mathematics assumes the stance that
% mathematics is an accumulation of facts, rules and skills that need
% to be used as a means to an end, without there necessarily being any
% relation between these components.
Die instrumentalistiese siening van wiskunde aanvaar die standpunt dat
wiskunde 'n versameling feite, re\"{e}ls en vaardighede is wat gebruik
word as 'n middel vir 'n doelwit, sonder dat daar noodwendig 'n
verband is tussen hierdie komponente.
% The Platonist view of mathematics sees the subject as a static but
% unified body of certain knowledge, in which mathematics is
% discovered rather than created.
Die Platonistiese siening van wiskunde is dat die vak 'n statiese,
maar verenigde liggaam van sekere kennis is, waarbinne wiskunde ontdek
word eerder as om geskep te word.
% The problem solving view of mathematics is a dynamic, continually
% expanding and evolving field of human creation and invention that is
% in itself a cultural product.
Die probleemoplossing siening van wiskunde is dat dit 'n dinamiese,
voortdurend ontwikkelende veld van menslike skepping en uitvinding is
wat op sigself 'n kulturele produk is.
% Thus mathematics is viewed as a process of enquiry, not a finished
% product.
Wiskunde word dus beskou as 'n proses van ondersoek, eerder as 'n
voltooide produk.
% The results remain constantly open to revision.
Die resultate bly voortdurend oop vir hersiening.
% It is suggested that a hierarchical order exists within these three
% views, placing the instrumentalist view at the lowest level and the
% problem solving view at the highest.
Een voorgestel is dat 'n hi\"{e}rargiese orde bestaan ​​binne hierdie
drie aansigte, met die instrumentalistiese siening op die laagste vlak
en die probleemoplossing siening op die hoogste vlak.
%% WHAT COMPLETE AND UTTER BULLSHIT! If you're in doubt about this,
%% just count the weasel words.

\subsubsection{Volgens die NKV:}
% Mathematics is the study of quantity, structure, space and change.
Wiskunde is die studie van hoeveelheid, struktuur, ruimte en
verandering.
% Mathematicians seek out patterns, formulate new conjectures, and
% establish axiomatic systems by rigorous deduction from appropriately
% chosen axioms and definitions.
Wiskundiges soek patrone, formuleer nuwe veronderstellings en vestig
aksiomatiese stelsels deur die streng deduktiewe afleiding vanaf
toepaslik gekose aksiomas en definisies.
% Mathematics is a distinctly human activity practised by all
% cultures, for thousands of years.
Wiskunde is 'n menslike aktiwiteit wat deur alle kulture beoefen is,
vir duisende jare reeds.
% Mathematical problem solving enables us to understand the world
% (physical, social and economic) around us, and, most of all, to
% teach us to think creatively.
Wiskundige probleemoplossing stel ons in staat om die w\^{e}reld
rondom ons (fisies, sosiaal en ekonomies) te verstaan en, belangrikste
van alles, om te leer om kreatief te dink.

% This corresponds well to the problem solving view of mathematics and
% may challenge some of our instrumentalist or Platonistic views of
% mathematics as a static body of knowledge of accumulated facts,
% rules and skills to be learnt and applied.
Dit stem goed ooreen met die probleemoplossing siening van wiskunde en
mag dalk sommige van ons instrumentalistiese of Platonistiese
sienings, as 'n statiese versameling van kennis, feite, re\"{e}ls en
vaardighede wat geleer en toegepas moet word, uitdaag.

% The NCS is trying to discourage such an approach and encourage
% mathematics educators to dynamically and creatively involve their
% learners as mathematicians engaged in a process of study,
% understanding, reasoning, problem solving and communicating
% mathematically.
Die NKV probeer om so 'n benadering te ontmoedig en moedig
wiskunde-onderwysers aan om op 'n dinamiese en kreatiewe manier hulle
leerders as wiskundiges te betrek by 'n proses van studie, begrip,
redenering, probleemoplossing en kommunikasie.

% Below is a check list that can guide you in actively designing your
% lessons in an attempt to embrace the definition of mathematics from
% the NCS and move towards a problem solving conception of the
% subject.
Hieronder is 'n lys wat u kan help om u lesse aktief te ontwerp in 'n
poging om die NKV definisie van wiskunde te omhels en om nader te
beweeg aan 'n probleemoplossing konsepsie van die onderwerp.
% Adopting such an approach to the teaching and learning of
% mathematics will in turn contribute to the intended curriculum being
% properly implemented and attained through the quality of learners
% coming out of the education system.
Die aanvaarding van so 'n benadering tot die onderrig en leer van
wiskunde sal op sy beurt bydra tot die implementering en realisering
van die voorgenome kurrikulum, in terme van die kwaliteit van die
leerders wat uit die onderwysstelsel kom.

\begin{table}[H]
  \begin{center}
    \begin{tabular}{|p{6.5cm}|p{6.5cm}|} \hline 
\textbf{Aktiwiteit} & \textbf{Voorbeeld} \\ \hline
Leerders neem deel aan die oplossing van kontekstuele probleme wat
verband hou met hul lewens en wat vereis dat hulle 'n probleem
interpreteer en dan 'n geskikte wiskundige oplossing te vind.
&
Leerders word gevra om uit te werk watter busdiens die goedkoopste is,
gegee die tariewe en die afstand wat hulle wil reis.
\\ \hline
Leerders raak betrokke by die oplos van probleme van 'n suiwer
wiskundige aard, wat ho\"{e}r-orde denke en die toepassing van kennis
(nie-standaard probleme) benodig.
&
Leerders word gevra om 'n grafiek te teken. Hulle het nog nie 'n
spesifieke tekentegniek (byvoorbeeld vir 'n parabool) geleer nie, maar
het geleer om die tabelmetode te gebruik om reguit lyne te teken.
\\ \hline
Leerders kry geleentheid om oor betekenis te redeneer.
&
Leerders bespreek hul begrip van konsepte en strategie\"{e} vir die
oplossing van probleme met mekaar en met die onderwyser.
\\ \hline
Leerders word geleer end gevra om situasies op verskeie ekwivalente
maniere te verteenwoordig (wiskundige modellering).
&
Leerders verteenwoordig dieselfde data met behulp van 'n grafiek, 'n
tabel en 'n formule om die data voor te stel.
\\ \hline
Leerders doen individueel wiskundige ondersoeke in die klas, gelei
deur die onderwyser waar nodig.
&
Elke leerder kry 'n wiskundige probleem (byvoorbeeld om die aantal
priemgetalle minder as 50 te vind) wat ondersoek moet word
en die oplossing neergeskryf moet word. Leerders werk onafhanklik.
\\ \hline
Leerders werk saam as 'n groep/span om ondersoek in te stel of 'n
wiskundige probleem op te los.
&
'n Groep word opdrag gegee om saam te werk aan 'n probleem wat vereis
dat hulle patrone in data ondersoek, om veronderstellings te maak en
'n ​​formule vir die patroon te vind.
\\ \hline
Leerders doen oefeninge om hulle kennis van konsepte
te konsolideer en verskeie vaardighede te bemeester.
&
Voltooiing van 'n oefening wat roetine prosedures benodig.
\\ \hline
Leerders kry geleenthede om die wisselwerking tussen verskillende
aspekte van wiskunde te sien en om te sien hoe die verskillende
uitkomste verwant is.
&
Terwyl leerders deur meetkunde probleme werk, word hulle aangemoedig
om gebruik te maak van algebra.
\\ \hline
Leerders word gevra om probleme vir hulle onderwyser en
klasmaats op te stel.
&
Leerders word gevra 'n algebra\"{i}ese woordsom op te stel
(waarvan hulle ook die oplossing ken), vir die persoon wat langs
hulle sit om op te los.
\\ \hline
    \end{tabular}
  \end{center}
\end{table}

\subsection{Oorsig van die onderwerpe}
Oorsig van onderwerpe en hulle relevansie.

\begin{table}[H]
  \begin{center} 
    \begin{tabular}{|p{8.5cm}|p{3.5cm}|} \hline
\textbf{1. Funksies --- line\^{e}r, kwadraties, eksponensi\"{e}el, rasioneel} &
\textbf{Relevansie}  \\ \hline  
Verwantskappe tussen veranderlikes in terme van die grafiese, verbale
en simboliese voorstellings van funksies (tabelle, grafieke, woorde en
formules).\par
Grafieke en veralgemeningsgevolge van parameters: vertikale
verskuiwings en skalering en refleksies om die x-as.\par
Probleemoplossing en grafiekwerk met betrekking tot voorgeskrewe funksies.
&
Funksies vorm 'n sentrale deel van leerders se wiskundige begrip en
redenasie-prosesse in algebra.\par
Dit is ook 'n uitstekende geleentheid vir kontekstuele wiskundige
modelleringvrae. \\ \hline
    \end{tabular}
  \end{center}
\end{table}

\begin{table}[H]
\begin{center} 
\begin{tabular}{|p{8.5cm}|p{3.5cm}|} \hline
\textbf{2. Getalpatrone, rye en reekse}&\textbf{Relevansie} \\ \hline  
Getalpatrone met konstante verskil.
&
Baie wiskunde wentel rondom die identifisering van patrone.
\\ \hline
 \end{tabular}
\end{center}
\end{table}

\begin{table}[H]
\begin{center} 
\begin{tabular}{|p{8.5cm}|p{3.5cm}|} \hline
\textbf{3. Finansies, groei en krimping}& \textbf{Relevansie} \\ \hline  
Gebruik eenvoudige en saamgestelde groeiformules.\par
Implikasies van die veranderende wisselkoerse.
&
Die wiskunde van finansies is baie relevant vir daaglikse en
langtermyn finansi\"{e}le besluite wat leerders sal moet maak vir
beleggings, lenings, spaar, begrip van wisselkoerse en die invloed
daarvan w\^{e}reldwyd.
\\ \hline

 \end{tabular}
\end{center}
\end{table}

\begin{table}[H]
\begin{center} 
\begin{tabular}{|p{8.5cm}|p{3.5cm}|} \hline
\textbf{4. Algebra}&\textbf{Relevansie}  \\ \hline  
Verstaan ​​dat re\"{e}le getalle kan word irrasionele of rasionele.\par
Vereenvoudig uitdrukkings deur gebruik te maak van die wette van eksponente vir rasionale eksponente.\par
Identifiseer en die omskep die verskillende vorme van rasionale getalle.\par
Werk met eenvoudige wortels wat nie rasioneel is nie.\par
Werk met die wette van heeltallige eksponente.\par
Bepall tussen watter twee heelgetalle 'n eenvoudige wortelvorm l\^{e}.\par
Rond re\"{e}le getal gepas af.\par
Manipuleer en vereenvoudig algebra\"{i}ese uitdrukkings (insluitend vermenigvuldiging en faktorisering).\par
Los line\^{e}re, kwadratiese, letterlike en eksponensi\"{e}le vergelykings op.\par
Los line\^{e}re ongelykhede in een en twee veranderlikes algebra\"{i}es en grafies op.\par
&
Algebra verskaf die grondslag vir wiskunde leerders om te beweeg van
numeriese berekeninge na veralgemeende operasies, vereenvoudiging
van uitdrukkings, oplos van vergelykings en gebruik van
grafieke en ongelykhede vir die oplossing van kontekstuele probleme.
\\ \hline

 \end{tabular}
\end{center}
\end{table}

\begin{table}[H]
 \begin{center} 
\begin{tabular}{|p{8.5cm}|p{3.5cm}|} \hline
\textbf{5. Differensiaalrekening}& \textbf{Relevansie}\\ \hline  
Ondersoek gemiddelde koers van verandering tussen twee onafhanklike
waardes van 'n funksie.
&
Die sentrale aspek van die tempo van verandering vir
differensiaalrekening is as 'n basis vir verdere begrip van grense,
gradi\"{e}nte en berekeninge en formules wat nodig is vir werk in die
ingenieurswese-velde, bv.. die ontwerp van paaie, br\^{u}e ens.
\\ \hline

 \end{tabular}
\end{center}
\end{table}

\begin{table}[H]
 \begin{center} 
\begin{tabular}{|p{8.5cm}|p{3.5cm}|} \hline
\textbf{6. Waarskynlikheid}& \textbf{Relevansie} \\ \hline  
Vergelyk relatiewe frekwensie en teoretiese waarskynlikheid\par
Gebruik Venn diagramme om waarskynlikheid probleme op te los.\par
Uitsluitlike en komplement\^{e}re gebeure.\par
Identiteite vir enige twee gebeure A en B.
&
Hierdie onderwerp is nuttig vir die ontwikkeling van goeie logiese
redenasievermo\"{e} en vir die opvoeding van leerders in terme van
werklike lewenskwessies soos dobbelary en die slaggate daarvan.
\\ \hline

 \end{tabular}
\end{center}
\end{table}


\begin{table}[H]
 \begin{center} 
\begin{tabular}{|p{8.5cm}|p{3.5cm}|} \hline
\textbf{7. Euklidiese Meetkunde en Meting}& \textbf{Relevansie}\\ \hline  
Ondersoek, vorm en probeer om veronderstellings oor die eienskappe van driehoeke, vierhoeke en ander veelhoeke te bewys.\par
Weerl\^{e} valse veronderstellings deur die gebruik van teen-voorbeelde.\par
Ondersoek alternatiewe definisies van verskillende veelhoeke.\par
Los probleme op met betrekking tot die oppervlakte en volume van soliede voorwerpe en kombinasies daarvan.
&
Die denkprosesse en wiskundige vaardighede met betrekking tot die
bewys van veronderstellings en ​​die identifisering van valse
veronderstellings is meer relevant as om die inhoud te studeer.
Die oppervlakte en volume in praktiese kontekste soos die ontwerp van
kombuise, die te\"{e}l en verf van kamers, die ontwerp van verpakking,
ens.\@ is relevant tot die huidige en toekomstige lewens var leerdeers.
\\ \hline

 \end{tabular}
\end{center}
\end{table}

\begin{table}[H]
 \begin{center} 
\begin{tabular}{|p{8.5cm}|p{3.5cm}|} \hline
\textbf{8. Trigonometrie}& \textbf{Relevansie}\\ \hline  

Definisies van trigonometriese funksies.\par
Lei waardes af vir spesiale hoeke.\par
Neem kennis van die name vir inverse funksies.\par
Los probleme op in 2 dimensies.\par
Brei definisies van basiese trigonometriese funksies uit na al vier kwadrante en ken grafieke van hierdie funksies.\par
Ondersoek en weet wat die gevolge van $a$ en $q$ op die grafieke van basiese trigonometriese funksies is.\par
&
Trigonometrie het verskeie gebruike in die samelewing, bv.\@ in
navigasie, musiek, geografie en die ontwerp en konstruksie van
geboue.
\\ \hline

 \end{tabular}
\end{center}
\end{table}

\begin{table}[H]
 \begin{center} 
\begin{tabular}{|p{8.5cm}|p{3.5cm}|} \hline
\textbf{9. Analitiese meetkunde}&  \textbf{Relevansie} \\ \hline  
Stel meetkundige figure op 'n Cartesiese ko\"{o}rdinaatstelsel voor.\par
Vir enige twee punte, lei af en pas toe die formule vir die berekening
van afstand en helling van 'n lynsegment en die ko\"{o}rdinate van die
middelpunt.
&
Hierdie afdeling verskaf 'n verdere toepassing vir leerders se
algebra\"{i}ese en trigonometriese interaksie met die Cartesiese
vlak. Kunstenaars en die ontwerp en uitleg industrie\"{e} maak dikwels
gebruik van die inhoud en denkprosesse van hierdie wiskundige onderwerp.
\\ \hline

 \end{tabular}
\end{center}
\end{table}

\begin{table}[H]
 \begin{center} 
\begin{tabular}{|p{8.5cm}|p{3.5cm}|} \hline
\textbf{10. Statistiek}& \textbf{Relevansie}\\ \hline  
Versamel, organiseer en interpreteer enkelveranderlike numeriese data
om gemiddeld, mediaan, modus, persentiele, kwartiele, desiele,
interkwartiel- en semi-interkwartielvariasiewydte te bepaal.\par
Identifiseer moontlike bronne van vooroordeel en foute in metings.
&
Mense word daagliks gekonfronteer met die interpretasie van data wat
deur die media verskaf word. Dikwels word hierdie data bevooroordeeld of
wanvoorgestel binne 'n sekere konteks. In enige soort navorsing is die
insameling en hantering van data kernprosesse. Hierdie onderwerp help
ook leerders op om meer sosiaal en polities opgevoed te wees ten
opsigte van die media.
\\ \hline

 \end{tabular}
\end{center}
\end{table}

% Mathematics educators also need to ensure that the following important specific aims and general principles are applied in mathematics activities across all grades:
Wiskunde onderwysers moet ook verseker dat die volgende belangrike
spesifieke doelwitte en algemene beginsels toegepas word in
wiskunde-aktiwiteite in alle grade:
\begin{itemize}[noitemsep]
\item
% Calculators should only be used to perform standard numerical computations and verify calculations done by hand.
  Sakrekenaars mag slegs gebruik word om die standaard numeriese
  berekeninge uit te voer en berekeninge wat met die hand gedoen is,
  te kontroleer.
\item
% Real-life problems should be incorporated into all sections to keep mathematical modelling as an important focal point of the curriculum.
  Werklike probleme ge\"{i}ntegreer word in alle afdelings, om
  wiskundige modellering te behou as 'n belangrike fokuspunt van die
  kurrikulum.
\item
% Investigations give learners the opportunity to develop their ability to be more methodical, to generalise and to make and justify and/or prove conjectures.
  Ondersoeke gee leerders die geleentheid om hul vermo\"{e} om meer
  metodies te wees, om te kan veralgemeen, en om veronderstellings te
  kan ontwikkel en regverdig en/of bewys.
\item
% Appropriate approximation and rounding skills should be taught and continuously included and encouraged in activities.
  Gepaste benaderings- en afrondingsvaardighede moet geleer word en
  voortdurend aangemoedig word in aktiwiteite.
\item
% The history of mathematics should be incorporated into projects and tasks where possible, to illustrate the human aspect and developing nature of mathematics. 
  Die geskiedenis van wiskunde moet ingewerk word in projekte en take,
  waar moontlik, om die menslike aspek en ontwikkelende aard van
  wiskunde te illustreer.
\item
% Contextual problems should include issues relating to health, social, economic, cultural, scientific, political and environmental issues where possible. 
  Kontekstuele probleme moet kwessies met betrekking tot gesondheid,
  maatskaplike, ekonomiese, kulturele, wetenskaplike, politieke en
  omgewingskwessies insluit, waar moontlik.
\item
% Conceptual understanding of when and why should also feature in problem types.
  Konseptuele begrip van ``wanneer'' en ``hoekom'' moet ook deel vorm
  van die tipes probleme.
\item
% Mixed ability teaching requires educators to challenge able learners and provide remedial support where necessary. 
  Onderrig vir gemengde vermo\"{e}ns vereis dat opvoeders in staat is om
  leerders uit te daag en remedi\"{e}rende ondersteuning aan te bied, waar
  nodig.
\item
% Misconceptions exposed by assessment need to be dealt with and rectified by questions designed by educators. 
  Wanpersepsies wat deur assessering blootgestel word, moet hanteer
  en reggestel word met behulp van vrae ontwerp deur opvoeders.
\item
% Problem solving and cognitive development should be central to all mathematics teaching and learning so that learners can apply the knowledge effectively. 
  Probleemoplossing en kognitiewe ontwikkeling moet sentraal wees tot
  alle wiskunde onderrig en leerwerk, sodat leerders hulle kennis
  doeltreffend kan toepas.
\end{itemize}

\subsubsection{Toekenning van onderrigtyd}
%Time allocation for Mathematics per week: 4 hours and 30 minutes e.g. six forty-five minute periods per week.
Tydstoekenning vir Wiskunde per week: 4 uur en 30 minute, bv.\@ ses 45-minuut periodes per week.
\begin{table}[H]
 \begin{center} 
\begin{tabular}{|p{2cm}|p{6cm}|p{2cm}|} \hline
\textbf{Kwartaal}& \textbf{Onderwerp} & \textbf{Aantal weeks} \\ \hline  
\textbf{Kwartaal 1} & Algebra\"{i}ese uitdrukkings \par
Eksponente\par
Getalpatrone \par
Vergelykings en ongelykhede\par
Trigonometrie

&
3\par
2\par
1\par
2\par
3 \\ \hline
\textbf{Kwartaal 2} & Funksies \par
Trigonometriese funksies  \par
Euklidiese meetkunde  \par
Half-jaar eksamen &
4  \par
1 \par
3  \par
3  \\ \hline

\textbf{Kwartaal 3} & Analitiese meetkunde \par
Finansies en groei \par
Statistiek \par
Trigonometrie \par
Euklidiese meetkunde\par
Meting &
2 \par
2\par
2\par
2\par
1\par
1 \\ \hline
\textbf{Kwartaal 4} & Waarskynlikheid \par Hersiening \par Eksamen &
2 \par 
4 \par 
3 \\ \hline


 \end{tabular}
\end{center}
\end{table}

%Please see page 18 of the Curriculum and Assessment Policy Statement for the sequencing and pacing of topics.
Sien bladsy 18 van die Kurrikulum- en Assesseringsbeleidsraamwerk vir
die volgordebepaling en tempo van onderwerpe.

% --- START HERE ---

\section{Assessering}
%“Educator assessment is part of everyday teaching and learning in the classroom. Educators discuss with learners, guide their work, ask and answer questions, observe, help, encourage and challenge. In addition, they mark and review written and other kinds of work. Through these activities they are continually finding out about their learners’ capabilities and achievements. This knowledge then informs plans for future work. It is this continuous process that makes up educator assessment. It should not be seen as a separate activity necessarily requiring the use of extra tasks or tests.”  \par 
``Opvoeder assessering is deel van die alledaagse onderrig en leer in die klaskamer. Opvoeders met leerders bespreek, hul werk, vra en vrae te beantwoord, neem, help, aan te moedig en uit te daag.
Daarbenewens het hulle merk en hersiening van geskrewe en ander vorme van werk. Deur middel van hierdie aktiwiteite is hulle voortdurend om uit te vind oor hulle leerders se vermo\"{e}ns en prestasies. Hierdie kennis lig dan planne vir die toekoms werk. Dit is hierdie deurlopende proses wat opvoeder assessering. Dit moet nie gesien word as 'n afsonderlike aktiwiteit noodwendig vereis dat die gebruik van ekstra take of toetse.''

%As the quote above suggests, assessment should be incorporated as part of the classroom practice, rather than as a separate activity. Research during the past ten years indicates that learners get a sense of what they do and do not know, what they might do about this and how they feel about it, from frequent and regular classroom assessment and educator feedback. The educator’s perceptions of and approach to assessment (both formal and informal assessment) can have an influence on the classroom culture that is created with regard to the learners’ expectations of and performance in assessment tasks. Literature on classroom assessment distinguishes between two different purposes of assessment; assessment of learning and assessment for learning. \par 
Soos die aanhaling hierbo suggereer, behoort assessering opgeneem word as deel van die klaskamerpraktyk, eerder as 'n afsonderlike aktiwiteit. Navorsing gedurende die afgelope tien jaar dui aan dat leerders 'n gevoel kry van wat hulle doen en weet nie wat hulle kan doen en hoe hulle voel oor dit, van gereelde en gereelde klaskamerassessering en opvoeder terugvoer. Die opvoeder se persepsies van en benadering tot assessering (beide die formele en informele assessering) kan 'n invloed op die klaskamer kultuur wat geskep is met betrekking tot die leerders se verwagtinge van en prestasie in assesseringstake. Literatuur oor klaskamerassessering onderskei tussen twee verskillende doeleindes van assessering, assessering van leer en assessering vir leer.

%Assessment of learning tends to be a more formal assessment and assesses how much learners have learnt or understood at a particular point in the annual teaching plan. The NCS provides comprehensive guidelines on the types of and amount of formal assessment that needs to take place within the teaching year to make up the school-based assessment mark. The school-based assessment mark contributes 25\% of the final percentage of a learner’s promotion mark, while the end-of-year examination constitutes the other 75\% of the annual promotion mark. Learners are expected to have 7 formal assessment tasks for their school-based assessment mark. The number of tasks and their weighting in the Grade 10 Mathematics curriculum is summarised below: 
Assessering van leer is geneig om 'n meer formele assessering en assesseer hoeveel leerders geleer het, of op 'n bepaalde punt in die jaarlikse onderrig plan verstaan. Die NKV bied omvattende riglyne oor die soorte en hoeveelheid van die formele assessering moet plaasvind binne die onderrig jaar te maak van die skoolgebaseerde assesseringspunt. Die skoolgebaseerde assesseringspunt dra 25\% van die finale persentasie van 'n leerder se promosiepunt, terwyl die einde van die jaar eksamen maak die ander 75\% van die jaarlikse promosiepunt. Leerders word verwag om 7 formele assesseringstake vir hul skoolgebaseerde assesseringspunt te h\^{e}. Die aantal take en hul gewig in die graad 10-Wiskunde kurrikulum is hieronder opgesom:

\begin{table}[H]
\begin{center}
\begin{tabular} {|p{4.5cm}|p{1.5cm}|p{3cm}|p{2cm}|} \hline
	  & 			& \textbf{Take} 			& \textbf{Gewigstoekenning (\%)} \\ \hline
Skool-gebaseerde Assessering &
Kwartaal 1 &
Toets \par Projek/Ondersoek &
10 \par 20 \\ \hline
&
Kwartaal 2 &
Opdrag/Toets \par Eksamen &
10 \par 30 \\ \hline
&
Kwartaal 3 &
Toets \par Toets &
10 \par 10 \\ \hline
&
Kwartaal 4 & Toets &
10 \\ \hline
Skool-gebaseerde Assesseringspunt &
&
&
100 \\ \hline
Skool-gebaseerde Assesseringspunt  \par
(as 'n \% van Vorderingspunt)
			&	 & 					&  25 \% \\ \hline

Eindeksamen & 	& 					&75 \% \\ \hline
Vorderingspunt 		&       & 					& 100 \% \\ \hline


\end{tabular}
 \end{center}
\end{table}

%The following provides a brief explanation of each of the assessment tasks included in the assessment programme above.
Die volgende is 'n kort verduideliking van elk van die assesseringstake ingesluit in die beoordeling program hierbo.

\subsubsection{Toetse}
%All mathematics educators are familiar with this form of formal assessment. Tests include a variety of items/questions covering the topics that have been taught prior to the test. The new NCS also stipulates that mathematics tests should include questions that cover the following four types of cognitive levels in the stipulated weightings: 
Alle wiskunde-opvoeders is vertroud met hierdie vorm van formele assessering. Die toetse sluit in 'n verskeidenheid van items / vrae wat die onderwerpe wat reeds voor die toets geleer. Die nuwe NKV bepaal ook dat wiskunde toetse vrae wat betrekking het op die volgende vier tipes van kognitiewe vlakke insluit:

\begin{table}[H]
\begin{center}
\begin{tabular} {|p{3cm}|p{7cm}|p{1.5cm}|} \hline
\textbf{Kognitiewe Vlakke} & \textbf{Beskrywing} & \textbf{Gewigstoekenning (\%)} \\ \hline
Kennis & 
%Estimation and appropriate rounding of numbers. \par 
Beramings-en toepaslike afronding van getalle. \par 
%Proofs of prescribed theorems.\par 
Bewyse van voorgeskrewe stellings. \par 
%Derivation of formulae.\par 
Die aflei van formules. \par 
%Straight recall.\par 
Direk weergee. \par 
%Identification and direct use of formula on information sheet (no changing of the subject). Use of mathematical facts.\par 
Identifisering en die direkte gebruik van formule op die inligtingsblad (geen verandering van die vak). Gebruik van wiskundige feite. \par 
%Appropriate use of mathematical vocabulary.
Toepaslike gebruik van wiskundige woordeskat.
& 
20 \\ \hline

Roetine Prosedures & 
%Perform well known procedures.\par 
Voer bekende prosedures. \par 
%Simple applications and calculations.\par 
Eenvoudige toepassings en berekeninge. \par 
%Derivation from given information.\par 
Afleiding van die gegewe inligting. \par 
%Identification and use (including changing the subject) of correct formula.\par 
Identifikasie en die gebruik (insluitend die verandering van die onderwerp) van korrekte formule. \par 
%Questions generally similar to those done in class.
Vrae oor die algemeen soortgelyk aan di\'{e} wat in die klas gedoen is.
&
45 \\ \hline

Komplekse Prosedures &
%Problems involve complex calculations and/or higher reasoning. \par 
Probleme behels komplekse berekenings en / of ho\"{e}r redenasie. \par 
%There is often not an obvious route to the solution.\par 
Daar is dikwels nie 'n duidelike pad na die oplossing. \par 
%Problems need not be based on real world context.\par 
Probleme hoef nie gebaseer op 'n werklike w\^{e}reld konteks. \par 
%Could involve making significant connections between different representations.\par 
Kan behels die maak van beduidende verband tussen verskillende voorstellings. \par 
%Require conceptual understanding.
Vereis konseptuele begrip.
&
25 \\ \hline
Probleemoplossing & 
%Unseen, non-routine problems (which are not necessarily difficult). \par 
Ongesiende, nie-roetine probleme (wat nie noodwendig moeilik). \par 
%Higher order understanding and processes are often involved.\par 
Ho\"{e}r orde begrip en prosesse is dikwels betrokke is. \par 
%Might require the ability to break the problem down into its constituent parts.
Kan vereis dat die vermo\"{e} om die probleem in sy samestellende dele af te breek.
&
10 \\ \hline

\end{tabular}
 \end{center}
\end{table}

%The breakdown of the tests over the four terms is summarised from the NCS assessment programme as follows: \\
Die uiteensetting van die toetse oor die vier kwartale van die NKV assessering program opgesom as volg:\\
%\textbf{Term 1}:	One test of at least 50 marks, and one hour or two/three tests of at least 40 minutes each.\\
\textbf{Kwartaal 1}: Een toets van ten minste 50 punte en een uur of twee/drie toetse van ten minste 40 minute elk.\\
%\textbf{Term 2}:	Either one test (of at least 50 marks) or an assignment.\\
\textbf{Kwartaal 2}: \'{O}f een toets (van ten minste 50 punte) \'{o}f 'n werkstuk.\\
%\textbf{Term 3}:	Two tests, each of at least 50 marks and one hour.\\
\textbf{Kwartaal 3}: Twee toetse, wat elk van ten minste 50 punte en een uur.\\
%\textbf{Term 4}:	One test of at least 50 marks.\\
\textbf{Kwartaal 4}: Een toets van ten minste 50 punte.\\

\subsubsection{Projekte/Ondersoeke}

%Investigations and projects consist of open-ended questions that initiate and expand thought processes. Acquiring and developing problem-solving skills are an essential part of doing investigations and projects. These tasks provide learners with the opportunity to investigate, gather information, tabulate results, make conjectures and justify or prove these conjectures.  Examples of investigations and projects and possible marking rubrics are provided in the next section on assessment support. The NCS assessment programme indicates that only one project or investigation (of at least 50 marks) should be included per year. Although the project/investigation is scheduled in the assessment programme for the first term, it could also be done in the second term. 
Ondersoeke en projekte bestaan ​​uit oop vrae wat denkprosesse te inisieer en uit te brei. Aanleer en ontwikkeling van probleem-vaardighede is 'n noodsaaklike deel van die doen van ondersoeke en projekte. Hierdie take bied leerders die geleentheid om ondersoek in te stel, inligting te versamel, tabuleer die resultate, maak veronderstellings en regverdig of bewys hierdie veronderstellings. Voorbeelde van ondersoeke en projekte en moontlike assesseringskale word in die volgende afdeling oor assessering ondersteuning. Die NKV assessering program dui aan dat slegs een projek of ondersoek (van ten minste 50 punte) per jaar moet ingesluit word. Hoewel die projek / ondersoek word in die beoordeling program vir die eerste kwartaal geskeduleer word, kan dit ook gedoen word in die tweede kwartaal.

\subsubsection{Opdragte}
%The NCS includes the following tasks as good examples of assignments: 
Die NKV sluit die volgende take in as goeie voorbeelde van opdragte:
\begin{itemize}[noitemsep]
\item
% Open book test
  Oopboek toets
\item
% Translation task
  Vertaling opdrag
\item
% Error spotting and correction
  Foute raaksien en regstel
\item
% Shorter investigation
  Kort ondersoek
\item
% Journal entry
  Joernaalinskrywing
\item
% Mind-map (also known as a metacog)
  Breinkaart
\item
% Olympiad (first round)
  Olimpiade (eerste rondte)
\item
% Mathematics tutorial on an entire topic
  Wiskunde tutoriaal oor 'n hele onderwerp
\item
% Mathematics tutorial on more complex/problem solving questions
  Wiskunde tutoriaal oor meer komplekse of probleemoplossingsvrae
\end{itemize}
%The NCS assessment programme requires one assignment in term 2 (of at least 50 marks) which could also be a combination of some of the suggested examples above. More information on these suggested examples of assignments and possible rubrics are provided in the following section on assessment support. 
Die NKV assessering program vereis 'n opdrag in kwartaal 2 (van ten minste 50 punte) wat ook 'n kombinasie van 'n paar van die voorbeelde hierbo. Meer inligting oor hierdie voorgestelde voorbeelde van opdragte en moontlike rubrieke word in die volgende afdeling oor assessering ondersteuning.

\subsubsection{Eksamens}
%Educators are also all familiar with this summative form of assessment that is usually completed twice a year: mid-year examinations and end-of-year examinations. These are similar to the tests but cover a wider range of topics completed prior to each examination. The NCS stipulates that each examination should also cover the four cognitive levels according to their recommended weightings as summarised in the section above on tests. The following table summarises the requirements and information from the NCS for the two examinations.
Opvoeders word ook al vertroud met hierdie opsommende vorm van assessering wat voltooi is gewoonlik twee keer per jaar: die middel van die jaar eksamens en einde-van-jaar eksamens. Dit is soortgelyk aan die toetse, maar dek 'n groter verskeidenheid van onderwerpe voltooi voor elke eksamen. Die NKV bepaal dat elke eksamen moet ook die vier kognitiewe vlakke dek volgens hul aanbeveel gewigte soos saamgevat in die artikel hierbo op toetse. Die volgende tabel gee 'n opsomming van die vereistes en inligting van die NKV vir die twee eksamens.

\begin{table}[H]
\begin{center}
\begin{tabular} {|p{2cm}|p{1.5cm}|p{3cm}|p{4.5cm}|} \hline
\textbf{Eksamen} &
\textbf{Punte} &
\textbf{Uiteensetting} &
\textbf{Inhoud en puntverspreiding} \\ \hline
Mid-jaar Eksamen &
100 \par 50 + 50 &
Een vraestel: 2 uur\par \textbf{of} \par
Twee vraestelle: 1 uur elk &
Voltooide onderwerpe \\ \hline
Jaareindeksamen	&
100 + &
Vraestel 1: 2 ure &
Getalpatrone ($\pm 10$) \par 
Algebra\"{i}ese uitdrukkings, vergelykings en ongelykhede ($\pm 25$)\par 
Funksies ($\pm 35$)\par 
Eksponente ($\pm 10$)\par 
Finansies ($\pm 10$)\par 
Waarskynlikheid ($\pm 10$)
\\ \hline
&
100 &
Vraestel 2: 2 ure &
Trigonometrie ($\pm 45$) \par 
Analitiese meetkunde ($\pm 15$)\par 
Euklidiese meetkunde en meting ($\pm 25$)\par 
Statistiek ($\pm 15$)
\\ \hline
\end{tabular}
 \end{center}
\end{table}

%In the annual teaching plan summary of the NCS in Mathematics for Grade 10, the pace setter section provides a detailed model of the suggested topics to be covered each week of each term and the accompanying formal assessment.
In die jaarlikse onderrig plan opsomming van die NKV in Wiskunde vir Graad 10, die pasaanduider afdeling verskaf 'n gedetailleerde model van die voorgestelde onderwerpe wat gedek moet word elke week van elke kwartaal en die gepaardgaande formele assessering.

%Assessment \textbf{for} learning tends to be more informal and focuses on using assessment in and of daily classroom activities that can include: 
Assessering \textbf{vir} leer is geneig om meer informele en fokus op die gebruik van assessering in en van die daaglikse klaskamer-aktiwiteite wat insluit:
\begin{itemize}[noitemsep]
\item
% Marking homework
Huiswerk werk
\item
% Baseline assessments
Basislyn assesserings
\item
% Diagnostic assessments
Diagnostiese assesserings
\item
% Group work
Groepwerk
\item
% Class discussions
Klasbesprekings
\item
% Oral presentations
Mondelinge voorleggings
\item
% Self-assessment
Self-assessering
\item
% Peer-assessment
Assessering van eweknie\"{e}
\end{itemize}

%These activities are expanded on in the next section on assessment support and suggested marking rubrics are provided. Where formal assessment tends to restrict the learner to written assessment tasks, the informal assessment is necessary to evaluate and encourage the progress of the learners in their verbal mathematical reasoning and communication skills. It also provides a less formal assessment environment that allows learners to openly and honestly assess themselves and each other, taking responsibility for their own learning, without the heavy weighting of the performance (or mark) component. The assessment for learning tasks should be included in the classroom activities at least once a week (as part of a lesson) to ensure that the educator is able to continuously evaluate the learners’ understanding of the topics covered as well as the effectiveness, and identify any possible deficiencies in his or her own teaching of the topics. 
Hierdie aktiwiteite word uitgebrei in die volgende afdeling oor assessering ondersteuning en voorgestel assesseringskale word voorsien. Waar formele assessering is geneig om die leerder te beperk tot skriftelike assesseringstake, is die informele assessering nodig is om die vordering van die leerders in hul verbale wiskundige redenasie en kommunikasie vaardighede te evalueer en aan te moedig. Dit bied ook 'n minder formele assessering omgewing wat kan leerders om openlik en eerlik te assesseer hulself en mekaar, om verantwoordelikheid te neem vir hul eie leer, sonder die swaar gewig van die prestasie (of merk) komponent. Die assessering vir leeraktiwiteite moet ingesluit word in die klaskamer-aktiwiteite ten minste een keer 'n week as deel van 'n les om te verseker dat die opvoeder is in staat om voortdurend te evalueer die leerders se begrip van die onderwerpe wat gedek sowel as die doeltreffendheid, en enige moontlike tekortkominge in sy of haar eie onderrig van die onderwerpe te identifiseer.

\subsection{Assessering ondersteuning}
%A selection of explanations, examples and suggested marking rubrics for the assessment of learning (formal) and the assessment for learning (informal) forms of assessment discussed in the preceding section are provided in this section. 
'n Seleksie van verduidelikings, voorbeelde en voorgestelde assesseringskale vir die assessering van leer (formele) en die assessering vir leer (informele) vorme van assessering wat in die vorige afdeling bespreek word in hierdie afdeling.

\subsubsection{Grondlynassessering}
%Baseline assessment is a means of establishing:
Grondlynassessering is 'n middel van die stigting van:

\begin{itemize}[noitemsep]
\item
% What prior knowledge a learner possesses 
  Wat voorkennis, 'n leerder beskik oor
\item
% What the extent of knowledge is that they have regarding a specific learning area
  Wat is die mate van kennis is dat hulle oor 'n spesifieke leerarea
\item
% The level they demonstrate regarding various skills and applications
  Die vlak van hulle demonstreer ten opsigte van verskillende vaardighede en toepassings
\item
% The learner’s level of understanding of various learning areas
  Die leerder se vlak van begrip van die verskillende leerareas
\end{itemize}
%It is helpful to educators in order to assist them in taking learners from their individual point of departure to a more advanced level and to thus make progress. This also helps avoid large "gaps” developing in the learners’ knowledge as the learner moves through the education system. Outcomes-based education is a more learner-centered approach than we are used to in South Africa, and therefore the emphasis should now be on the level of each individual learner rather than that of the whole class. \par 
Dit is nuttig om opvoeders ten einde hulle te help om leerders van hul individuele punt van vertrek na 'n meer gevorderde vlak en om sodoende vordering maak. Dit help ook vermy groot ``gapings'' in die leerders se kennis as die leerder beweeg deur die onderwysstelsel Uitkomsgebaseerde onderwys is 'n meer leerder-gesentreerde benadering as wat ons in Suid-Afrika gebruik word, en dus die klem moet nou op die vlak van elke individuele leerder, eerder as di\'{e} van die hele klas.

%The baseline assessments also act as a gauge to enable learners to take more responsibility for their own learning and to view their own progress. In the traditional assessment system, the weaker learners often drop from a 40\% average in the first term to a 30\% average in the fourth term due to an increase in workload, thus demonstrating no obvious progress. Baseline assessment, however, allows for an initial assigning of levels which can be improved upon as the learner progresses through a section of work and shows greater knowledge, understanding and skill in that area.
Die basislyn aanslae ook optree as 'n meter in staat te stel om leerders meer verantwoordelikheid vir hul eie leer te neem en hul eie vordering te sien. In die tradisionele assessering stelsel daal, die swakker leerders dikwels uit 'n gemiddeld van 40\% in die eerste kwartaal tot 'n gemiddeld van 30\% in die vierde kwartaal as gevolg van 'n toename in die werkslading, toon geen duidelike vooruitgang. Grondlynassessering, egter, kan vir 'n aanvanklike toeken van vlakke wat verbeter kan word op die leerder vorder deur middel van 'n afdeling van die werk en toon 'n groter kennis, begrip en vaardigheid in daardie gebied.

\subsubsection{Diagnostiese assessering}
%These are used to specifically find out if any learning difficulties or problems exist within a section of work in order to provide the learner with appropriate additional help and guidance. The assessment helps the educator and the learner identify problem areas, misunderstandings, misconceptions and incorrect use and interpretation of notation. \par
Dit word gebruik om spesifiek uit te vind of enige leer probleme of probleme bestaan ​​binne 'n afdeling van die werk ten einde die leerder te voorsien met toepaslike addisionele hulp en leiding. Die assessering help die opvoeder en die leerder identifiseer probleemareas, misverstande, wanopvattings en foutiewe gebruik en interpretasie van notasie.

%Some points to keep in mind:
'n Paar punte om in gedagte te hou:
\begin{itemize}[noitemsep]
\item
% Try not to test too many concepts within one diagnostic assessment.
  Probeer om nie te veel konsepte te toets binne 'n diagnostiese assessering nie.
\item
% Be selective in the type of questions you choose. 
  Wees selektief in die tipe vrae wat jy kies.
\item
% Diagnostic assessments need to be designed with a certain structure in mind. As an educator, you should decide exactly what outcomes you will be assessing and structure the content of the assessment accordingly. 
  Diagnostiese assesserings moet met 'n sekere struktuur in gedagte te ontwerp. As 'n opvoeder, moet jy besluit presies watter uitkomste jy sal assesseer en struktuur van die inhoud van die assessering dienooreenkomstig.
\item
% The assessment is marked differently to other tests in that the mark is not the focus but rather the type of mistakes the learner has made.
  Die beoordeling is anders gemerk ander toetse wat die punt is nie die fokus nie, maar eerder die tipe foute wat die leerder gemaak het.
\end{itemize}
%An example of an understanding rubric for educators to record results is provided below:\\
'n Voorbeeld van 'n begripnrubriek vir opvoeders aan te teken resultate word hieronder gegee:

%0: indicates that the learner has not grasped the concept at all and that there appears to be a fundamental mathematical problem.\\
0: dui aan dat die leerder nie die konsep onder die knie het nie en dat daar blyk 'n fundamentele wiskundige probleem.\\
%1: indicates that the learner has gained some idea of the content, but is not demonstrating an understanding of the notation and concept.\\
1: dui dat die leerder 'n idee gekry het van die inhoud, maar is nie 'n begrip van die notasie en begrip te toon.\\
%2: indicates evidence of some understanding by the learner but further consolidation is still required.\\
2: dui op getuienis van 'n paar begrip deur die leerder nie, maar verdere konsolidasie is steeds 'n vereiste.\\
%3:indicates clear evidence that the learner has understood the concept and is using the notation correctly.\par
3: dui op 'n duidelike bewys dat die leerder die konsep verstaan ​​en die gebruik van die notasie korrek.

\subsubsection{Sakrekenaar werkblad: assessering van diagnostiese vaardighede}
\begin{enumerate}[itemsep=7pt, label=\textbf{\arabic*}. ] 
 \item Bereken:
\begin{enumerate}[itemsep=6pt,label=\textbf{(\alph*)}]
\item $ 242 + 63=$   ~~~\underline{~~~~~~~~~~~~~}
\item $2-36 \times (114 + 25)=$~~~\underline{~~~~~~~~~~~~~}
\item $\sqrt{144+25}=$~~~\underline{~~~~~~~~~~~~~}
\item $\sqrt[4]{729}=$~~~\underline{~~~~~~~~~~~~~}
\item $-312 + 6 + 879 -321 + 18~ 901=$ ~~~\underline{~~~~~~~~~~~~~}
\end{enumerate}

\item Bereken:
\begin{enumerate}[itemsep=6pt,label=\textbf{(\alph*)}]
\item $\frac{2}{7} + \frac{1}{3}=$  ~~~\underline{~~~~~~~~~~~~~}
\item $2\frac{1}{5} - \frac{2}{9}=$ ~~~\underline{~~~~~~~~~~~~~}
\item $-2\frac{5}{6} + \frac{3}{8}=$ ~~~\underline{~~~~~~~~~~~~~}
\item $ 4 - \frac{3}{4} \times \frac{5}{7}=$ ~~~\underline{~~~~~~~~~~~~~}
\item $\left(\frac{9}{10} - \frac{8}{9}\right) \div \frac{3}{5}=$ ~~~\underline{~~~~~~~~~~~~~}
\item $2\times \left(\frac{4}{5}\right)^2 - \left(\frac{19}{25}\right)=$ ~~~\underline{~~~~~~~~~~~~~}
\item $\sqrt{\frac{9}{4} - \frac{4}{16}} =$ ~~~\underline{~~~~~~~~~~~~~}
\end{enumerate}
\end{enumerate}

Self-assesseringmatriks: \\
Naam: \underline{~~~~~~~~~~~~~~~~~~~~~~~~~~~~~~}
\begin{table}[H]
 \begin{center}
  \begin{tabular}{|p{1.5cm}|p{1.5cm}|p{1cm}|p{1cm}|p{6cm}|} \hline

\textbf{Vraag} & \textbf{Antwoord} & \textbf{√} & \textbf{X} & \textbf{Indien X, skryf volgorde van sleutels wat gedruk is} \\ \hline
1a) &&&&\\ \hline
1b)&&&&\\ \hline
1c)&&&&\\ \hline
1d)&&&&\\ \hline
1e)&&&&\\ \hline
\textbf{Subtotaal}&&&&\\ \hline
2a)&&&&\\ \hline
2b)&&&&\\ \hline
2c)&&&&\\ \hline
2d)&&&&\\ \hline
2e)&&&&\\ \hline
2f)&&&&\\ \hline
2g)&&&&\\ \hline
\textbf{Subtotaal}&&&&\\ \hline
\textbf{Totaal}&&&& \\ \hline


   
  \end{tabular}

 \end{center}

\end{table}

Opvoeder assesseringmatriks:
\begin{table}[H]
 \begin{center}
  \begin{tabular}{|p{4.5cm}|p{1.5cm}|p{3cm}|p{1.5cm}|} \hline

\textbf{Tipe Vaardigheid} & \textbf{Bemeester} & \textbf{Benodig Oefening} & \textbf{Probleem}   \\ \hline
Verhef tot 'n mag &&&\\ \hline
Bereken 'n wortel &&&\\ \hline
Berekeninge met breuke &&&\\ \hline
Hakies en volgorde van operasies &&&\\ \hline
Beramings en verstandelike beheer &&&\\ \hline
   
  \end{tabular}

 \end{center}

\end{table}
%Guidelines for Calculator Skills Assessment:
Riglyne vir sakrekenaarvaardigheid assessering:
\begin{table}[H]
 \begin{center}
  \begin{tabular}{|p{5cm}|p{4cm}|p{3cm}|} \hline

\textbf{Tipe Vaardigheid} & \textbf{Onderverdeling} & \textbf{Vrae}   \\ \hline
Verhef tot 'n mag & Vierkante en derdemagte\par Ho\"{e}r orde magte&1a, 2f \par1b \\ \hline
Worteltrekking& Vierkants en derdemagswortels \par Ho\"{e}r order wortels & 1c, 2g \par 1d\\ \hline
Berekeninge met breuke & Basiese operasies \par Gemengde getalle \par Negatiewe getalle \par Kwadreer breuke \par Vierkantswortels van breuke &2a,  2d\par
2b, 2c\par
1e, 2c\par
2f\par
2g
\\ \hline
Hakies en volgeorde van berekininge&Korrekte gebruik van hakies of volgorde van berekeninge&1b, 1c, 2e, 2f, 2g\\ \hline
Beramings en verstandelike beheer &Algeheel&Alles\\ \hline
   
  \end{tabular}

 \end{center}

\end{table}

\subsubsection{Voorgestelde riglyn vir die toekenning van algehele vlakke}
\textbf{Vlak 1}
\begin{itemize}[noitemsep]
\item
% Learner is able to do basic operations on calculator.
Leerder is in staat om basiese bewerkings te doen op die sakrekenaar.
\item
% Learner is able to do simple calculations involving fractions.
Leerder is in staat om eenvoudige berekeninge met betrekking tot
breuke te doen.
\item
% Learner does not display sufficient mental estimation and control techniques.
Die leerder word nie vertoon voldoende geestelike beraming en beheer.
\end{itemize}
\textbf{Vlak 2}\begin{itemize}[noitemsep]
\item
% Learner is able to do basic operations on calculator.
Leerder is in staat om basiese bewerkings te doen op die sakrekenaar.
\item
% Learner is able to square and cube whole numbers as well as find square and cube roots of numbers.
Leerder is in staat om vierkante en kubus heelgetalle asook
vierkants-en derdemagswortels van getalle.
\item
% Learner is able to do simple calculations involving fractions as well as correctly execute calculations involving mixed numbers.
Leerder is in staat om eenvoudige berekeninge met betrekking tot
breuke te doen, sowel as om korrek die volgende behels gemengde breuke
voer.
\item
% Learner displays some degree of mental estimation awareness.
Leerder toon 'n mate van geestelike skatting bewustheid.
\end{itemize}
\textbf{Vlak 3}\begin{itemize}[noitemsep]
\item
% Learner is able to do basic operations on calculator.
Leerder is in staat om basiese bewerkings te doen op die sakrekenaar.
\item
% Learner is able to square and cube rational numbers as well as find square and cube roots of numbers.
Leerder is in staat om na die vierkant en kubus rasionale getalle
asook soos vierkants-en derdemagswortels van nommers.
\item
% Learner is also able to calculate higher order powers and roots.
Leerder is ook in staat om ho\"{e}r-orde-magte en wortels te bereken.
\item
% Learner is able to do simple calculations involving fractions as well as correctly execute calculations involving mixed numbers.
Leerder is in staat om eenvoudige berekeninge met betrekking tot
breuke te doen, sowel as om korrek die volgende behels gemengde breuke
voer.
\item
% Learner works correctly with negative numbers.
Leerder werk korrek met negatiewe getalle.
\item
% Learner is able to use brackets in certain calculations but has still not fully understood the order of operations that the calculator has been programmed to execute, hence the need for brackets.
Leerder is in staat om tussen hakies te gebruik in sekere berekeninge,
maar het nog nie ten volle verstaan ​​die volgorde van bewerkings dat
die sakrekenaar geprogrammeer is om uit te voer, vandaar die behoefte
aan hakies.
\item
% Learner is able to identify possible errors and problems in their calculations but needs assistance solving the problem.
Leerder is in staat om moontlike foute en probleme in hul berekeninge
te identifiseer, maar het hulp nodig om die probleem op te los.
\end{itemize}
\textbf{Vlak 4}\begin{itemize}[noitemsep]
\item
% Learner is able to do basic operations on calculator.
Leerder is in staat om basiese bewerkings te doen op die sakrekenaar.
\item
% Learner is able to square and cube rational numbers as well as find square and cube roots.
Leerder is in staat om vierkante en sny in blokkies rasionale getalle,
asook vierkants-en derdemagswortels.
\item
% Learner is also able to calculate higher order powers and roots.
Leerder is ook in staat om ho\"{e}r-orde-magte en wortels te bereken.
\item
% Learner is able to do simple calculations involving fractions as well as correctly execute calculations involving mixed numbers.
Leerder is in staat om eenvoudige berekeninge met betrekking tot
breuke te doen, sowel as om korrek die volgende behels gemengde breuke
voer.
\item
% Learner works correctly with negative numbers.
Leerder werk korrek met negatiewe getalle.
\item
% Learner is able to work with brackets correctly and understands the need and use of brackets and the “= key” in certain calculations due to the nature of a scientific calculator.
Leerder is in staat om te werk met hakies korrek nie en verstaan ​​die
noodsaaklikheid en die gebruik van hakies en die ``= sleutel'' in sekere
berekeninge te danke aan die aard van 'n wetenskaplike sakrekenaar.
\item
% Learner is able to identify possible errors and problems in their calculations and to find solutions to these in order to arrive at a “more viable” answer.
Leerder is in staat om moontlike foute en probleme in hul berekeninge
te identifiseer en om oplossings te vind vir hierdie om te kom op 'n
``meer lewensvatbaar'' antwoord.
\end{itemize}

\subsubsection{Ander kort diagnostiese toetse}
%These are short tests that assess small quantities of recall knowledge and application ability on a day-to-day basis. Such tests could include questions on one or a combination of the following:
Dit is kort toetse wat klein hoeveelhede van die herroeping-kennis en die toepassing vermo\"{e} om op 'n dag-tot-dag basis te assesseer. Sulke toetse kan die volgende insluit: vrae oor een of 'n kombinasie van die volgende:
\begin{itemize}[noitemsep]
\item
% Definitions
  Definisies
\item
% Theorems
  Stellings
\item
% Riders (geometry)
  Riders (geometrie)
\item
% Formulae
  Formules
\item
% Applications
  Aansoeke
\item
% Combination questions
  Kombinasie vrae
\end{itemize}
%Here is a selection of model questions that can be used at Grade 10 level to make up short diagnostic tests. They can be marked according to a memorandum drawn up by the educator. \par
Hier is 'n seleksie van die model vrae wat op Graad 10-vlak gebruik te maak kort diagnostiese toetse. Hulle kan gemerk word volgens 'n memorandum opgestel deur die opvoeder.

\textbf{Meetkunde}
\begin{enumerate}[itemsep=0pt, label=\textbf{\arabic*}. ] 
\item Punte $A(-5; -3)$, $B(-1; 2)$ en $C(9; -6)$ is die hoekpunte van $\triangle ABC$. 
\begin{enumerate}[itemsep=0pt,label=\textbf{(\alph*)}]
\item Bereken die gradi\"{e}nt van $AB$ en $BC$ en wys dus dat hoek $ABC$ gelyk is aan $90^{\circ}$.				
\begin{flushright}(5)\end{flushright}
\item Gee die afstandformule en gebruik dit om die lengtes van sye
$AB$, $BC$ en $AC$ van $\triangle ABC$ te bereken. (Laat jou antwoorde in wortelvorm.)
\begin{flushright}(5)\end{flushright}
\end{enumerate}
\end{enumerate}

\textbf{Algebra}
\begin{enumerate}[itemsep=0pt, label=\textbf{\arabic*}. ] 
\item Skryf die formele definisie van 'n eksponent en ook die eksponentwette vir integrale eksponente neer.\begin{flushright}(6)\end{flushright}
\item Vereenvoudig: $\dfrac{2x^4y^8z^3}{4xy} \times \dfrac{x^7}{y^3z^0}$	\begin{flushright}(4)\end{flushright}
\end{enumerate}

\textbf{Trigonometrie}
\begin{enumerate}[itemsep=0pt, label=\textbf{\arabic*}. ] 
\item
% A jet leaves an airport and travels $578$ km in a direction of $50^{\circ}$ E of N. The pilot then changes direction and travels $321$ km $10^{\circ}$ W of N.
  'n Vliegtuig vertrek vanaf 'n lughawe en reis $578$ km met 'n
  rigting van $50^{\circ}$ O van N. Die loods verander dan van rigting
  en vlieg $321$ km $10^{\circ}$ W van N.
\begin{enumerate}[itemsep=0pt,label=\textbf{(\alph*)}]
\item Hoe ver weg van die lughawe is die vliegtuig? (Tot die naaste kilometer.) \begin{flushright}(5)\end{flushright}
\item Bepaal die peiling van die vliegtuig vanaf die lughawe.\begin{flushright}(5)\end{flushright}
\end{enumerate}
\end{enumerate}

\subsubsection{Oefeninge}
%This entails any work from the textbook or other source that is given to the learner, by the educator, to complete either in class or at home. Educators should encourage learners not to copy each other’s work and be vigilant when controlling this work. It is suggested that such work be marked/controlled by a check list (below) to speed up the process for the educator. \par
Dit behels 'n werk uit die handboek of 'n ander bron wat aan die leerder gegee is, deur die opvoeder, \'{o}f in die klas of tuis te voltooi. Opvoeders moet leerders aan te moedig om nie mekaar se werk te kopieer en waaksaam wees wanneer die beheer van hierdie werk. Daar word voorgestel dat sulke werk nie gemerk word / beheer word deur 'n tjek lys (hieronder) te bespoedig die proses vir die opvoeder.

%The marks obtained by the learner for a specific piece of work need not be based on correct and/or incorrect answers but preferably on the following:
Die punte wat behaal is deur die leerder vir 'n spesifieke stuk werk moet nie gebaseer is op korrekte en / of verkeerde antwoorde maar verkieslik op die volgende:

\begin{itemize}[noitemsep]
\item
% the effort of the learner to produce answers.
  die poging van die leerder om antwoorde te produseer.
\item
% the quality of the corrections of work that was previously incorrect.
  die kwaliteit van die regstellings van die werk wat was voorheen verkeerd is nie.
\item
% the ability of the learner to explain the content of some selected examples (whether in writing or orally).
  die vermo\"{e} van die leerder om die inhoud van 'n paar geselekteerde voorbeelde (hetsy skriftelik of mondeling) om te verduidelik.
\end{itemize}
%The following rubric can be used to assess exercises done in class or as homework: 
Die volgende matriks kan gebruik word om oefeninge in die klas of as huiswerk gedoen te assesseer:

\begin{table}[H]
 \begin{center}
  \begin{tabular}{|p{3cm}|p{3cm}|p{3cm}|p{3cm}|} \hline
   \textbf{Kriteria} & \textbf{Prestasie-aanduiders} &&\\ \hline
Werk gedoen & 2 \par Al die werk & 1 \par Gedeeltelik voltooi & 0 \par Geen werk gedoen \\ \hline
Werk netjies gedoen & 2 \par Werk netjies gedoen & 1 \par Sommige werk nie netjies gedoen & 0 \par Morsig en deurmekaar\\ \hline
Regstellings gedoen & 2 \par Alle regstellings konsekwent gedoen & 1 \par Ten minste helfte van die regstellings gedoen & 0 \par Geen regstellings gedoen \\ \hline
Korrekte wiskundige metode & 2 \par Konsekwent & 1 \par Soms & 0 \par Glad nie \\ \hline
Begrip van wiskundige tegnieke en prosesse & 2 \par Kan konsepte en prosesse akkuraat verduidelik & 1 \par Verduidelikings is dubbelsinnig en nie gefokus & 0 \par Verduidelikings is verwarrend of irrelevant \\ \hline
  \end{tabular}

 \end{center}

\end{table}

\subsubsection{Joernaalinskrywings}
%A journal entry is an attempt by a learner to express in the written word what is happening in Mathematics. It is important to be able to articulate a mathematical problem, and its solution in the written word. \par
'n Joernaalinskrywing is 'n poging om deur 'n leerder uit te druk in
die geskrewe woord wat gebeur in Wiskunde. Dit is belangrik om in
staat wees om 'n wiskundige probleem en die oplossing in die geskrewe
woord te verwoord.

%This can be done in a number of different ways:
Dit kan gedoen word in 'n aantal verskillende maniere:
\begin{itemize}[noitemsep]
\item
% Today in Maths we learnt \underline{~~~~~~~~~~~~~~~~~~~~~~~~~~~~~~~} 
  Vandag in Wiskunde het ons geleer \underline{~~~~~~~~~~~~~~~~~~~~~~~~~~~~~~~} 
\item
% Write a letter to a friend, who has been sick, explaining what was done in class today.
  Skryf 'n brief aan 'n vriend, wat siek was, te verduidelik wat
  gebeur het in die klas vandag.
\item
% Explain the thought process behind trying to solve a particular maths problem, e.g. sketch the graph of $ y = x^2 - 2x^2 + 1$ and explain how to sketch such a graph.
  Verduidelik die denkproses om 'n spesifieke wiskunde probleem te
  probeer oplos, bv.\@ skets die grafiek van $y = x^2 - 2x^2 + 1$ en
  verduidelik hoe om so 'n grafiek te teken.
\item
% Give a solution to a problem, decide whether it is correct and if not, explain the possible difficulties experienced by the person who wrote the incorrect solution. 
  Gee 'n oplossing vir 'n probleem, besluit of dit korrek is, en
  indien nie, verduidelik die moontlike probleme wat ervaar word deur
  die persoon wat die verkeerde oplossing geskryf het.
\end{itemize}
%A journal is an invaluable tool that enables the educator to identify any mathematical misconceptions of the learners. The marking of this kind of exercise can be seen as subjective but a marking rubric can simplify the task. 
'n Joernaal is 'n waardevolle hulpmiddel wat die opvoeder in staat stel om enige wiskundige wanopvattings van die leerders te identifiseer. Die nasien van hierdie soort oefening kan gesien word as subjektief, maar 'n merk rubriek kan die taak te vereenvoudig.

%The following rubric can be used to mark journal entries. The learners must be given the marking rubric before the task is done. 
Die volgende matriks kan gebruik word om joernaalinskrywings te
merk. Die matriks moet aan leerders uitgehandig word voordat die taak
gedoen word.
\begin{table}[H]
 \begin{center}
  \begin{tabular}{|p{4cm}|p{2cm}|p{2.5cm}|p{3cm}|} \hline
  \textbf{Taak} & \textbf{Bevoegd \newline(2 Punte)} & \textbf{Ontwikkel nog \newline(1 Punt)}& \textbf{Nog nie ontwikkel \newline (1 Punt)}\\ \hline
Voltooi binne tydsbeperking? &&&\\ \hline
Korrektheid van die verduideliking? &&&\\ \hline
Korrekte en toepaslike gebruik van wiskundige taal? &&&\\ \hline
Is die wiskunde korrek? &&&\\ \hline
Is die konsep korrek ge\"{i}nterpreteer?&&&\\ \hline

  \end{tabular}

 \end{center}

\end{table}

\subsubsection{Vertalings}
%Translations assess the learner’s ability to translate from words into mathematical notation or to give an explanation of mathematical concepts in words. Often when learners can use mathematical language and notation correctly, they demonstrate a greater understanding of the concepts. \par 
Vertaling assesseer die leerder se vermo\"{e} om woorde te vertaal in
wiskundige notasie of om 'n verduideliking van wiskundige konsepte in
woorde te gee. Dikwels wanneer leerders kan wiskundige taal en notasie
korrek te gebruik, demonstreer hulle 'n groter begrip van die
konsepte.

%For example: \\
Byvoorbeeld: \\
%Write the letter of the correct expression next to the matching number:\\
Skryf die letter van die korrekte term langs die ooreenstemmende nommer:
\begin{table}[H]
 \begin{center}
  \begin{tabular}{lrl} 
$x$ word met $10$ vermeerder&					a)	&$xy$ \\
Die produk van $x$ en $y$		 &		 b)	&$x^2$\\
Die som van 'n sekere getal en	&		c)&	$x^2$\\
twee keer di\'{e} getal&					d)&	$29x$	\\
Die helfte van 'n sekere getal vemenigvuldig met homself	&	e)&	$\frac{1}{2} \times 2$\\
Twee minder as $x$&					f)&	$x + x + 2  $\\
'n Sekere getal vermenigvuldig met homself	&		g)&	$x^ 2$\\
  \end{tabular}
 \end{center}
\end{table}

\subsubsection{Groepwerk}
%One of the principles in the NCS is to produce learners who are able to work effectively within a group. Learners generally find this difficult to do. Learners need to be encouraged to work within small groups. Very often it is while learning under peer assistance that a better understanding of concepts and processes is reached. Clever learners usually battle with this sort of task, and yet it is important that they learn how to assist and communicate effectively with other learners. \par
Een van die beginsels in die NKV is om leerders wat in staat is om om
effektief te werk binne 'n groep te produseer. Leerders moet oor die
algemeen vind dit moeilik om te doen. Leerders moet aangemoedig word
om in klein groepies te werk. Heel dikwels is dit, terwyl die leer
onder eweknie bystand wat 'n beter begrip van konsepte en prosesse
bereik is. Slim leerders gewoonlik stryd met hierdie soort van die
taak, en tog is dit belangrik dat hulle leer hoe om te help en
effektief te kommunikeer met ander leerders.

\subsubsection{Breinkaarte}
%A metacog or “mind map” is a useful tool. It helps to associate ideas and make connections that would otherwise be too unrelated to be linked. A metacog can be used at the beginning or end of a section of work in order to give learners an overall perspective of the work covered, or as a way of recalling a section already completed. It must be emphasised that it is not a summary. Whichever way you use it, it is a way in which a learner is given the opportunity of doing research in a particular field and can show that he/she has an understanding of the required section. \par 
'n Breinkaart is 'n nuttige hulpmiddel. Dit help om idees te
assosi\"{e}er en verbindings wat andersins te onverwante te word
gekoppel maak. 'n Breinkaart kan gebruik word om aan die begin of
einde van 'n afdeling van die werk ten einde te gee leerders 'n
algehele perspektief van die werk wat, of as 'n manier van herinner
aan 'n artikel wat reeds ompleted. Dit moet beklemtoon word dat dit
nie 'n opsomming is nie. Ongeag hoe 'n mens dit gebruik, is dit 'n
manier waarop 'n leerder gegee word van die geleentheid gebruik om
navorsing te doen in 'n bepaalde veld en kan wys dat hy / sy het 'n
begrip van die vereiste afdeling.

%This is an open book form of assessment and learners may use any material they feel will assist them. It is suggested that this activity be practised, using other topics, before a test metacog is submitted for portfolio assessment purposes. \par
Dit is 'n oop boek vorm van assessering en leerders kan gebruik om
enige materiaal wat hulle voel hulle baie sal help. Daar word
voorgestel dat hierdie aktiwiteit beoefen word, met behulp van ander
onderwerpe, voor 'n toets breinkaart voorgel\^{e} word vir doeleindes van
portefeulje-assessering.

%On completion of the metacog, learners must be able to answer insightful questions on the metacog. This is what sets it apart from being just a summary of a section of work. Learners must refer to their metacog when answering the questions, but may not refer to any reference material. Below are some guidelines to give to learners to adhere to when constructing a metacog as well as two examples to help you get learners started. A marking rubric is also provided. This should be made available to learners before they start constructing their metacogs. On the next page is a model question for a metacog, accompanied by some sample questions that can be asked within the context of doing a metacog about analytical geometry. \par
Na voltooiing van die breinkaart, moet die leerders in staat wees om
insiggewende vrae te antwoord op die breinkaart. Dit is wat dit uitsonder
nie net 'n opsomming van 'n afdeling van die werk. Leerders moet
verwys na hul breinkaart wanneer die vrae beantwoord word word, maar mag
nie verwys na enige verwysing materiaal. Hier is 'n paar riglyne aan
leerders te gee om te voldoen aan wanneer die bou van 'n breinkaart
sowel as twee voorbeelde om jou te help leerders begin. 'n Merkmatriks
word ook voorsien. Dit moet beskikbaar gestel word aan leerders voor
hulle begin bou van hulle breinkaarte. Op die volgende bladsy is 'n
model vraag vir 'n breinkaart, vergesel deur 'n paar voorbeelde van
vrae wat gevra word binne die konteks van 'n breinkaart oor analitiese
meetkunde te doen.

%A basic metacog is drawn in the following way:
'n Basiese breinkaart word as volg geteken:
\begin{itemize}[noitemsep]
\item
% Write the title/topic of the subject in the centre of the page and draw a circle around it.
  Skryf die titel / onderwerp van die vak in die middel van die bladsy
  en trek 'n sirkel rondom dit.
\item
% For the first main heading of the subject, draw a line out from the circle in any direction, and write the heading above or below the line.
  Vir die eerste hoof opskrif van die vak, trek 'n streep uit die
  sirkel in enige rigting, en skryf die opskrif bo of onder die lyn.
\item
% For sub-headings of the main heading, draw lines out from the first line for each subheading and label each one. 
  Vir die sub-opskrifte van die hoofopskrif, trek lyne uit die eerste
  re\"{e}l vir elke onderverdeling en benoem elkeen.
\item
% For individual facts, draw lines out from the appropriate heading line. 
  Vir individuele feite, trek lyne uit die toepaslike opskrif.
\end{itemize}
%Metacogs are one’s own property. Once a person understands how to assemble the basic structure they can develop their own coding and conventions to take things further, for example to show linkages between facts. The following suggestions may assist educators and learners to enhance the effectiveness of their metacogs:
Breinkaarte is 'n mens se eie eiendom. Sodra 'n mens verstaan ​​hoe die
basiese struktuur hulle kan ontwikkel hul eie kodering en konvensies
om dinge verder te neem, byvoorbeeld skakeling tussen feite om te wys
om te vergader. Die volgende wenke kan help om opvoeders en leerders
om die doeltreffendheid van hul breinkaarte te verbeter:

\begin{itemize}[noitemsep]
\item
% Use single words or simple phrases for information. Excess words just clutter the metacog and take extra time to write down.
  Gebruik enkele woorde of eenvoudige frases vir inligting. Oortollige
  woorde maak net die breinkaart morsig en dit verg ekstra tyd om
  alles neer te skryf.
\item
% Print words – joined up or indistinct writing can be more difficult to read and less attractive to look at. 
Skryf woorde in drukskrif --- lopende of onduidelik skrif kan
moeiliker wees om te lees en minder aantreklik wees om na te kyk.
\item
% Use colour to separate different ideas – this will help your mind separate ideas where it is necessary, and helps visualisation of the metacog for easy recall. Colour also helps to show organisation.
  Gebruik kleur om verskillende idees te skei --- dit sal help om jou
  gedagtes verskillende idees waar dit nodig is, en help visualisering
  van die breinkaart vir maklike herroep. Kleur help ook om die
  organisasie te wys.
\item
% Use symbols and images where applicable. If a symbol means something to you, and conveys more information than words, use it. Pictures also help you to remember information.
  Gebruik simbole en beelde waar van toepassing. As 'n simbool beteken
  iets vir julle en dra meer inligting as woorde, dit gebruik. Foto's
  ook help om inligting te onthou.
\item
% Use shapes, circles and boundaries to connect information – these are additional tools to help show the grouping of information.
  Gebruik vorms, sirkels en grense om inligting aan te sluit --- dit
  is bykomende gereedskap om te help toon die groepering van
  inligting.
\end{itemize}
%Use the concept of analytical geometry as your topic and construct a
%mind map (or metacog) containing all the information (including
%terminology, definitions, formulae and examples) that you know about
%the topic of analytical geometry. \par
Gebruik die konsep van 'n analitiese meetkunde as jou onderwerp en
bou 'n breinkaart met al die inligting (insluitend terminologie,
definisies, formules en voorbeelde) dat jy weet oor die onderwerp van
analitiese meetkunde.

%Possible questions to ask the learner on completion of their metacog: 
Moontlike vrae om die leerder op die voltooiing van hul breinkaart te vra:
\begin{itemize}
\item
% Briefly explain to me what the mathematics topic of analytical geometry entails.
Verduidelik kortliks aan my wat die wiskunde onderwerp van analitiese
meetkunde behels.
\item
% Identify and explain the distance formula, the derivation and use thereof for me on your metacog.
Identifiseer en verduidelik die afstand formule, die afleiding en
gebruik daarvan vir my op jou breinkaart.
\item
% How does the calculation of gradient in analytical geometry differ (or not) from the approach used to calculate gradient in working with functions? 
Hoe verskil die berekening van gradi\"{e}nt in analitiese meetkunde (of
nie) van die benadering wat gebruik word om die gradi\"{e}nt te bereken in
wat met funksies word?
\end{itemize}
%A suggested simple rubric for marking a metacog:
'n Eenvoudige matriks vir die nasien van 'n breinkaart:
\begin{table}[H]
 \begin{center}
  \begin{tabular}{|p{3cm}|p{2.5cm}|p{2.5cm}|p{3cm}|} \hline
  \textbf{Taak} & \textbf{Bevoegd \newline(2 Punte)} & \textbf{In ontwikkeling \newline(1 Punt)}& \textbf{Nog nie ontwikkel \newline 1 Punt)}\\ \hline
Betyds voltooi &&&\\ \hline
Hoofopskrifte &&&\\ \hline
Korrekte Teorie (Formules, Definisies, Terminologie, ens.) &&&\\ \hline
Verduideliking &&&\\ \hline
Leesbaarheid&&&\\ \hline

  \end{tabular}

 \end{center}

\end{table}

%10 marks for the questions, which are marked using the following scale:\\
10 punte vir die vrae, wat gemerk work met behulp van die volgende skaal:\\
%0	-	no attempt or a totally incorrect attempt has been made \\
0: geen poging of 'n geheel verkeerde poging\\
%1	-	a correct attempt was made, but the learner did not get the correct answer \\
1: 'n korrekte poging, maar die leerder het nie die korrekte antwoord gevind nie\\
%2	-	a correct attempt was made and the answer is correct\\
2: 'n korrekte poging en die antwoord is korrek

\subsubsection{Ondersoeke}
%Investigations consist of open-ended questions that initiate and expand thought processes. Acquiring and developing problem-solving skills are an essential part of doing investigations. \par 
Ondersoeke bestaan ​​uit oop vrae wat denkprosesse te inisieer en uit te brei. Aanleer en ontwikkeling van probleem-vaardighede is 'n noodsaaklike deel van die uitvoer van ondersoeke.

%It is suggested that 2 – 3 hours be allowed for this task.  During the first 30 – 45 minutes learners could be encouraged to talk about the problem, clarify points of confusion, and discuss initial conjectures with others. The final written-up version should be done individually though and should be approximately four pages. \par 
Daar word voorgestel dat 2--3 uur toegelaat word om vir hierdie taak. Gedurende die eerste 30--45 minute Leerders kan aangemoedig word om te praat oor die probleem, verduidelik die punte van verwarring, en die aanvanklike hipoteses met ander bespreek. Die finale skriftelike-up weergawe moet individueel gedoen word al en moet ongeveer vier bladsye.

%Assessing investigations may include feedback/ presentations from groups or individuals on the results keeping the following in mind:
Assessering van die ondersoek kan die volgende insluit terugvoer / voordragte van groepe of individue op die resultate wat die volgende in gedagte hou:
\begin{itemize}[noitemsep]
\item
% following of a logical sequence in solving the problems
  na aanleiding van 'n logiese volgorde in die oplossing van die probleme
\item
% pre-knowledge required to solve the problem
  kennis wat nodig is om die probleem op te los
\item
% correct usage of mathematical language and notation
  korrekte gebruik van wiskundige taal en notasie
\item
% purposefulness of solution
  doelgerigtheid van die oplossing
\item
% quality of the written and oral presentation 
  die kwaliteit van die geskrewe en mondelinge aanbieding
\end{itemize}
%Some examples of suggested marking rubrics are included on the next few pages, followed by a selection of topics for possible investigations. \par
Enkele voorbeelde van voorgestelde assesseringskale ingesluit is op die volgende paar bladsye, gevolg deur 'n seleksie van onderwerpe vir moontlike ondersoeke.

%The following guidelines should be provided to learners before they begin an investigation: \par
Die volgende riglyne moet aan leerders voorsien word voordat hulle 'n ondersoek begin:

\textbf{Algemene instruksies aan leerders}
\begin{itemize}[noitemsep]
\item
% You may choose any one of the projects/investigations given (see model question on investigations)
  Jy kan kies om enige een van die projekte / ondersoeke gegee (kyk model vraag op ondersoeke)
\item
% You should follow the instructions that accompany each task as these describe the way in which the final product must be presented.
  Jy moet volg die instruksies wat saam met elke taak as hierdie beskrywing van die manier waarop die finale produk moet aangebied word.
\item
% You may discuss the problem in groups to clarify issues, but each individual must write-up their own version.
  Jy kan die probleem in groepe kwessies uit te klaar bespreek, maar elke individu moet skryf hul eie weergawe.
\item
% Copying from fellow learners will cause the task to be disqualified.
  Kopi\"{e}ring van mede-leerders sal veroorsaak dat die taak te gediskwalifiseer word.
\item
% Your educator is a resource to you, and though they will not provide you with answers / solutions, they may be approached for hints.
  Jou opvoeder is 'n hulpbron vir jou, en al is hulle nie sal jy met antwoorde / oplossings, hulle kan genader word vir die wenke.\end{itemize}	

\textbf{Die voorlegging}
%The investigation is to be handed in on the due date, indicated to you by your educator. It should have as a minimum:
Die ondersoek moet ingehandig word op die betaaldatum aan jou deur jou opvoeder. Dit moet as 'n minimum te beperk:
\begin{itemize}[noitemsep]
\item
% A description of the problem.
  'n Beskrywing van die probleem.
\item
% A discussion of the way you set about dealing with the problem.
  'n Bespreking van die manier waarop jy oor die hantering van die probleem.
\item
% A description of the final result with an appropriate justification of its validity.
  'n Beskrywing van die finale uitslag met 'n toepaslike motivering van die geldigheid.
\item
% Some personal reflections that include mathematical or other lessons learnt, as well as the feelings experienced whilst engaging in the problem.
  Sommige persoonlike refleksies wat wiskundige of ander lesse wat geleer is, sowel as die gevoelens ervaar, terwyl die beoefening van die probleem.
\item
% The written-up version should be attractively and neatly presented on about four A4 pages.
  Die geskrewe weergawe moet aantreklik en netjies aangebied word op sowat vier A4-bladsye.
\item
% Whilst the use of technology is encouraged in the presentation, the mathematical content and processes must remain the major focus.
  Hoewel die gebruik van tegnologie in die aanbieding word aangemoedig om die wiskundige inhoud en prosesse, moet die hooffokus bly.
\end{itemize}	
%Below are some examples of possible rubrics to use when marking investigations:\par
Hier is 'n paar voorbeelde van moontlike rubrieke te gebruik met die nasien van die ondersoek:

Voorbeeld 1:
\begin{table}[H]
 \begin{center}
  \begin{tabular}{|p{3cm}|p{8.5cm}|} \hline
\textbf{Prestasievlak}& \textbf{Kriteria} \\ \hline
4 &
\begin{itemize}[noitemsep]
\item
% Contains a complete response.
  Bevat 'n volledige antwoord.
\item
% Clear, coherent, unambiguous and elegant explanation.
  Duidelike, samehangende, ondubbelsinnige en elegante verduideliking.
\item
% Includes clear and simple diagrams where appropriate.
  Sluit duidelike en eenvoudige diagramme waar toepaslik.
\item
% Shows understanding of the question’s mathematical ideas and processes.
  Toon begrip van die vraag se wiskundige idees en prosesse.
\item
% Identifies all the important elements of the question.
  Identifiseer die belangrikste elemente van die vraag.
\item
% Includes examples and counter examples.
  Sluit voorbeelde en teenvoorbeelde.
\item
% Gives strong supporting arguments.
  Gee sterk ondersteunende argumente.
\item
% Goes beyond the requirements of the problem. 
  Gaan verder as die vereistes van die probleem.
   \end{itemize} \\ \hline
3 & 
\begin{itemize}[noitemsep]
\item
% Contains a complete response.
  Bevat 'n volledige antwoord.
\item
% Explanation less elegant, less complete.
  Verduideliking minder elegant, minder volledig.
\item
% Shows understanding of the question’s mathematical ideas and processes.
  Toon begrip van die vraag se wiskundige idees en prosesse.
\item
% Identifies all the important elements of the question.
  Identifiseer die belangrikste elemente van die vraag.
\item
% Does not go beyond the requirements of the problem.
  Gaan nie buite die vereistes van die probleem.
\end{itemize} \\ \hline
2 &
\begin{itemize}[noitemsep]
\item
% Contains an incomplete response.
  Bevat 'n onvolledige antwoord.
\item
% Explanation is not logical and clear.
  Verduideliking is nie logies en duidelik.
\item
% Shows some understanding of the question’s mathematical ideas and processes.
  Toon 'n mate van begrip van die vraag se wiskundige idees en prosesse.
\item
% Identifies some of the important elements of the question.
  Identifiseer sommige van die belangrikste elemente van die vraag.
\item
% Presents arguments, but incomplete.
  Bied argumente, maar onvolledig.
\item
% Includes diagrams, but inappropriate or unclear.
  Sluit diagramme, maar onvanpas of onduidelik.
\end{itemize} \\ \hline
1 &
\begin{itemize}[noitemsep]
\item
% Contains an incomplete response.
  Bevat 'n onvolledige antwoord.
\item
% Omits significant parts or all of the question and response.
  Laat belangrike deel van of al van die vraag-en-antwoord.
\item
% Contains major errors.
  Bevat groot foute.
\item
% Uses inappropriate strategies.
  Gebruik onvanpas strategie\"{e}.
\end{itemize} \\ \hline
0 &
\begin{itemize}[noitemsep]
\item
% No visible response or attempt
Geen merkbare reaksie of poging.
\end{itemize} \\ \hline
  \end{tabular}

 \end{center}

\end{table}

\subsubsection{Mondelinge}
%An oral assessment involves the learner explaining to the class as a whole, a group or the educator his or her understanding of a concept, a problem or answering specific questions. The focus here is on the correct use of mathematical language by the learner and the conciseness and logical progression of their explanation as well as their communication skills.\par
'n Mondelinge assessering behels die leerder verduidelik aan die klas as 'n geheel, 'n groep of die opvoeder sy of haar begrip van 'n konsep, 'n probleem of spesifieke vrae te beantwoord. Die fokus hier is op die korrekte gebruik van wiskundige taal deur die leerder en die bondigheid en logiese progressie van hul verduideliking sowel as hul kommunikasievaardighede.

%Orals can be done in a number of ways:
Mondeling gedoen kan word in 'n aantal maniere:
\begin{itemize}[noitemsep]
\item
% A learner explains the solution of a homework problem chosen by the educator.
  'n Leerder verduidelik die oplossing van 'n ​​huiswerk probleem wat deur die opvoeder gekies is.
\item
% The educator asks the learner a specific question or set of questions to ascertain that the learner understands, and assesses the learner on their explanation.
  Die opvoeder vra die leerder 'n spesifieke vraag of 'n stel van vrae om te verseker dat die leerder verstaan ​​en evalueer die leerder op hul verduideliking.
\item
% The educator observes a group of learners interacting and assesses the learners on their contributions and explanations within the group.
  Die opvoeder neem 'n groep leerders wat interaksie en assesseer die leerders op hul bydraes en verduidelikings binne die groep.
\item
% A group is given a mark as a whole, according to the answer given to a question by any member of a group.
  'n Groep gegee word om 'n punt as 'n geheel, volgens die antwoord op 'n vraag deur enige lid van 'n groep.
\end{itemize}
%An example of a marking rubric for an oral:\\
'n Voorbeeld van 'n ​​puntematriks vir 'n mondelinge:\\
%1	-	the learner has understood the question and attempts to answer it\\
1: Die leerder het die vraag verstaan ​​en probeer om dit te beantwoord\\
%2	-	the learner uses correct mathematical language\\
2: Die leerder gebruik die korrekte wiskundige taal\\
%2	-	the explanation of the learner follows a logical progression\\
2: Die verklaring van die leerder volg 'n logiese progressie\\
%2	-	the learner’s explanation is concise and accurate\\
2: Die leerder se verduideliking is bondig en akkuraat\\
%2	-	the learner shows an understanding of the concept being explained\\
2: Die leerder toon 'n begrip van die konsep wat verduidelik\\
%1	-	the learner demonstrates good communication skills\\
1: Die leerder demonstreer goeie kommunikasie vaardighede\\
%Maximum mark = 10 \par
Maksimumpunt = 10

%An example of a peer-assessment rubric for an oral: \\
'n Voorbeeld van 'n eweknie-assessering rubriek vir 'n mondeling:\\
My naam: \underline{~~~~~~~~~~~~~~~~~~~~~~~~~~~~~~~~~~}\\
Naam van persoon wie ek assesseer: \underline{~~~~~~~~~~~~~~~~~~~~~~~~~~~~~~~~~~}\\

\begin{table}[H]
 \begin{center}
  \begin{tabular}{|p{5cm}|p{2.5cm}|p{2.5cm}|} \hline
  \textbf{Kriteria} & \textbf{Punt Toegeken} & \textbf{Maksimum Punt}\\ \hline
Korrekte antwoord &&2\\ \hline
Duidelikheid van verduideliking&&3\\ \hline
Korrektheid van verduideliking  &&3\\ \hline
Bewyse van begrip &&3\\ \hline
\textbf{Totaal} &&10\\ \hline

  \end{tabular}

 \end{center}

\end{table}

\section{Hoofstukoorsig}
\subsubsection{Algebra\"{i}ese uitdrukkings}
%Algebra provides the basis for mathematics learners to move from numerical calculations to generalising operations, simplifying expressions, solving equations and using graphs and inequalities in solving contextual problems. Being able to multiply out and factorise are core skills in the process of simplifying expressions and solving equations in mathematics. Identifying irrational numbers and knowing their estimated position on a number line or graph is an important part of any mathematical calculations and processes that move beyond the basic number system of whole numbers and integers. Rounding off irrational numbers (such as the value of $\pi$) when needed, allows mathematics learners to work more efficiently with numbers that would otherwise be difficult to “pin down”, use and comprehend.  \par
Algebra verskaf die grondslag vir wiskunde leerders om te beweeg van
numeriese berekenings te veralgemeen bedrywighede, vereenvoudig
uitdrukkings, vergelykings op te los en deur gebruik te maak van
grafieke en ongelykhede in die oplossing van kontekstuele probleme. Om
in staat te wees om te vermenigvuldig en te faktoriseer, is die kern
vaardighede in die proses van vereenvoudiging van die uitdrukkings en
vergelykings op te los in wiskunde. Identifisering van irrasionale
getalle en die wete dat hul verwagte posisie op 'n getallelyn of
grafiek is 'n belangrike deel van enige wiskundige berekeninge en
prosesse wat verby die basiese stelsel van heelgetalle en heelgetalle.
Afronding irrasionale getalle (soos die waarde van $\pi$) wanneer dit
nodig is, kan wiskunde leerders om doeltreffend te werk met getalle
wat andersins moeilik om te ``PIN'', gebruik en te verstaan.

%Once learners have grasped the basic number system of whole numbers and integers, it is vital that their understanding of the numbers between integers is also expanded. This incorporates their dealing with fractions, decimals and surds which form a central part of most mathematical calculations in real-life contextual issues. Estimation is an extremely important component within mathematics. It allows learners to work with a calculator or present possible solutions while still being able to gauge how accurate and realistic their answers may be, which is relevant for other subjects too. 
Wanneer die leerders die basiese aantal stelsel van heelgetalle en heelgetalle onder die knie het, is dit noodsaaklik dat hulle begrip van die getalle tussen heelgetalle is ook uitgebrei word. Dit sluit die onderhandelings met breuke, desimale en wortels wat vorm 'n sentrale deel van die meeste wiskundige berekeninge in die werklike lewe kontekstuele kwessies.
Beraming is 'n uiters belangrike komponent van wiskunde. Dit stel leerders in staat om te werk met 'n sakrekenaar of huidige moontlike oplossings, terwyl hy nog in staat is om vas te stel hoe akkuraat en realisties is hulle antwoorde mag wees, wat relevant is vir ander vakke te.

\subsubsection{Vergelykings en ongelykhede}
%If learners are to later work competently with functions and the graphing and interpretation thereof, their understanding and skills in solving equations and inequalities will need to be developed. 
As die leerders later bekwaam werk met funksies en die grafiese en interpretasie daarvan, hul begrip en vaardighede in die oplos van vergelykings en ongelykhede sal ontwikkel moet word.

\subsubsection{Eksponente}
%Exponential notation is a central part of mathematics in numerical calculations as well as algebraic reasoning and simplification. It is also a necessary component for learners to understand and appreciate certain financial concepts such as compound interest and growth and decay. 
Eksponensiaalnotasie is 'n sentrale deel van wiskunde in numeriese berekenings asook algebra\"{i}ese redenasie en vereenvoudiging. Dit is ook 'n noodsaaklike komponent vir leerders om te verstaan ​​en te waardeer sekere finansi\"{e}le begrippe soos saamgestelde rente en groei en verval.

\subsubsection{Getalpatrone}
%Much of mathematics revolves around the identification of patterns. In earlier grades learners saw patterns in the form of pictures and numbers. In this chapter  we look at the mathematics of patterns. Patterns are repetitive sequences and can be found in nature, shapes, events, sets of numbers and almost everywhere you care to look. For example, seeds in a sunflower, snowflakes, geometric designs on quilts or tiles, the number sequence $0; ~4;~ 8;~ 12; ~16; \ldots$
Baie van wiskunde wentel rondom die identifisering van patrone. Soos in vorige grade leerders sien patrone in die vorm van foto's en nommers. In hierdie hoofstuk gaan ons kyk na die wiskunde van patrone. Patrone is herhalende volgordes, en kan gevind word in die natuur, vorms, gebeure, stel van getalle en amper oral waar jy omgee om te kyk. Byvoorbeeld, sade in 'n sonneblom, sneeuvlokkies, geometriese ontwerpe op quilts of te\"{e}ls, die nommer volgorde $0; ~4;~ 8;~ 12; ~16; \ldots$

\subsubsection{Funksies}
%Functions form a core part of learners’ mathematical understanding and reasoning processes in algebra. This is also an excellent opportunity for contextual mathematical modelling questions. Functions are mathematical building blocks for designing machines, predicting natural disasters, curing diseases, understanding world economies and for keeping aeroplanes in the air. A useful advantage of functions is that they allow us to visualise relationships in terms of a graph. Functions are much easier to read and interpret than lists of numbers. In addition to their use in the problems facing humanity, functions also appear on a day-to-day level, so they are worth learning about. A function is always dependent on one or more variables, like time, distance or a more abstract  quantity.
Funksies vorm 'n sentrale deel van die leerders se wiskundige begrip en redenasie-prosesse in algebra. Dit is ook 'n uitstekende geleentheid vir Kontekstuele wiskundige modellering vrae. Funksies wiskundige boustene vir die ontwerp van masjiene, die voorspelling van natuurlike rampe, genesing van siektes, begrip van die ekonomie\"{e} in die w\^{e}reld en vir die hou van vliegtuie in die lug. 'N nuttige voordeel van funksies is dat hulle toelaat dat ons verhoudings in terme van 'n grafiek te visualiseer. Funksies is baie makliker om te lees en te interpreteer as n lys van getalle. Benewens die gebruik daarvan in die probleme van die mensdom, funksies ook op 'n dag-tot-dag-vlak verskyn, sodat hulle die moeite werd is om te leer oor. 'n Funksie is altyd afhanklik is van een of meer veranderlikes, soos tyd, afstand of 'n meer abstrakte hoeveelheid.

\subsubsection{Finansies en groei}
%The mathematics of finance is very relevant to daily and long-term financial decisions learners will need to take in terms of investing, taking loans, saving and understanding exchange rates and their influence more globally.
Die wiskunde van finansies is baie relevant is vir daaglikse en langtermyn-finansi\"{e}le besluite leerders sal moet neem in terme van bel\^{e}, om lenings, spaar en nderstanding wisselkoerse en hul invloed meer globaal.

\subsubsection{Trigonometrie}
%There are many applications of trigonometry. Of particular value is the technique of triangulation, which is used in astronomy to measure the distances to nearby stars, in geography to measure distances between landmarks, and in satellite navigation systems. GPS (the global positioning system) would not be possible without trigonometry. Other fields which make use of trigonometry include acoustics, optics, analysis of financial markets, electronics, probability theory, statistics, biology, medical imaging (CAT scans and ultrasound), chemistry, cryptology, meteorology, oceanography, land surveying, architecture, phonetics, engineering, computer graphics and game development.
Daar is baie toepassings van trigonometrie. Van besondere waarde is die tegniek van triangulering, wat in sterrekunde gebruik is om die afstande te meet na die nabygele\"{e} sterre, in die geografie afstande tussen landmerke te meet, en in satellietnavigatie. GPS (Global Positioning System) sou nie moontlik wees sonder trigonometrie. Ander velde wat gebruik maak van trigonometrie, sluit in akoestiek,
optika, ontleding van finansi\"{e}le markte, elektronika, waarskynlikheidsleer en statistiek, biologie, mediese beelding (CAT-skanderings en ultraklank), chemie, kriptologie, weerkunde, oseanografie, landmeting, argitektuur, fonetiek, ingenieurswese, rekenaargrafika en spel ontwikkeling.

\subsubsection{Analitiese meetkunde}
%This section provides a further application point for learners’ algebraic and trigonometric interaction with the Cartesian plane. Artists and design and layout industries often draw on the content and thought processes of this mathematical topic.
Hierdie afdeling verskaf 'n verdere aansoek vir leerders se algebra\"{i}ese en trigonometriese interaksie met die Cartesiese vlak. Kunstenaars en die ontwerp en uitleg dikwels gebruik maak van die inhoud en denkprosesse van hierdie wiskundige onderwerp.

\subsubsection{Statistiek}
%Citizens are daily confronted with interpreting data presented from the media. Often this data may be biased or misrepresented within a certain context. In any type of research, data collection and handling is a core feature. This topic also educates learners to become more socially and politically educated with regards to the media.
Burgers word daagliks gekonfronteer met die interpretasie van data wat uit die media. Dikwels word hierdie data mag bevooroordeeld wees, of verkeerd binne 'n sekere konteks. In enige soort navorsing, data-insameling en hantering is 'n kern funksie. Hierdie onderwerp bied onderrig aan leerders ook om meer sosiaal en polities opleiding met betrekking tot die media.

\subsubsection{Waarskynlikheid}
%This topic is helpful in developing good logical reasoning in learners and for educating them in terms of real-life issues such as gambling and the possible pitfalls thereof. We use probability to describe uncertain events: when you accidentally drop a slice of bread, you don’t know if it’s going to fall with the buttered side facing upwards or downwards. When your favourite sports team plays a game, you don’t know whether they will win or not. When the weatherman says that there is a $40\%$ chance of rain tomorrow, you may or may not end up getting wet. Uncertainty presents itself to some degree in every event that occurs around us and in every decision that we make.
Hierdie onderwerp is nuttig in die ontwikkeling van goeie logiese beredenering in die leerders en vir die opvoeding van hulle in terme van die werklike lewe kwessies soos dobbelary en die moontlike slaggate daarvan. Ons maak gebruik van waarskynlikheid onseker gebeure te beskryf: as jy per ongeluk laat val 'n sny brood, jy weet nie of dit gaan om te val met die gesmeerde kant opwaarts of afwaarts. Wanneer jou gunsteling sport span speel 'n speletjie, jy weet nie of hulle wen of nie. Wanneer die weerman s\^{e} daar is 'n 40\%-kans op re\"{e}n m\^{o}re, jy mag of nie mag nie die einde om nat te word. Onsekerheid bied self tot 'n mate in elke gebeurtenis wat rondom ons en in elke besluit wat maak ons.

\subsubsection{Euklidiese meetkunde}
%The thinking processes and mathematical skills of proving conjectures and identifying false conjectures is more the relevance here than the actual content studied. The surface area and volume content studied in real-life contexts of designing kitchens, tiling and painting rooms, designing packages, etc. is relevant to the current and future lives of learners. Euclidean geometry deals with space and shape using a system of logical deductions.
Die denkprosesse en wiskundige vaardighede bewys van veronderstellings en die identifisering van valse veronderstellings is meer die relevansie as die werklike inhoud studeer. Die oppervlakte en volume inhoud bestudeer in die werklike lewe konteks van die ontwerp-kombuise, te\"{e}lwerk en verf van kamers, die ontwerp van pakkette, ens.
relevant is vir die huidige en toekomstige lewens van leerders. Euklidiese meetkunde besig met ruimte en vorm met behulp van 'n stelsel van logiese afleidings.

\subsubsection{Meting}
%This chapter revises the volume and surface areas of three-dimensional objects, otherwise known as solids. The chapter covers the volume and surface area of prisms and cylinders, Many exercises cover finding the surface area and volume of polygons, prisms, pyramids, cones and spheres, as well as a complex object. The effect on volume and surface area when multiplying a dimension of a factor of $k$ is also explored.
Hierdie hoofstuk hersien die volume en oppervlak gebiede van drie-dimensionele voorwerpe, andersins bekend as vastestowwe. Die hoofstuk dek die volume en oppervlakte van prismas en silinders, baie oefeninge dek om die oppervlakte en volume van veelhoeke, prismas, piramides, ke\"{e}ls en sfere, sowel as 'n komplekse voorwerp. Die effek op volume en oppervlak toe 'n dimensie van 'n faktor van $k$ vermenigvuldig
word ook ondersoek.

\subsubsection{Oplossings van oefeninge}
%This chapter includes the solutions to the exercises covered in each chapter of the book.
Hierdie hoofstuk sluit die oplossings vir die oefeninge wat in elke hoofstuk van die boek.
