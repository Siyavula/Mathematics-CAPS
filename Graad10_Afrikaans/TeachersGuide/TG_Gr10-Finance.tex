\chapter{Finansies en groei}
% \section{Being interested in interest}
% \section{Simple interest}
% \subsection{When the time period is not in years}
\begin{exercises}{}{
    \begin{enumerate}[itemsep=6pt, label=\textbf{\arabic*}.]
	\item ’n Bedrag van R~$3~500$ word belê in ’n spaarrekening wat enkelvoudige rente betaal teen ’n koers van $7,5\%$ p.a. Bereken die eindbedrag na 2 jaar.

	\item Bereken die enkelvoudige rente in die volgende probleme:
	\begin{enumerate}
	    \item ’n Lening van R~$300$  teen ’n koers van $8\%$ vir 1 jaar.

	    \item ’n Belegging van R~$2~250$ teen ’n koers van $12,5\%$ per jaar vir 6 jaar.
	\end{enumerate}

	\item Sally wil die aantal jare bereken wat sy R~$1~000$ moet bel\^e om te groei tot R~$2~500$. Sy word 'n enkelvoudige rentekoers van $8,2\%$ p.a aangebied. Hoe lank sal dit neem vir die geld om te groei tot R~$2~500$?

	\item Joseph het R~$5~000$ by die bank gedeponeer vir sy $5$ jaar oud seun se $21^{\mathrm{ste}}$ verjaarsdag. Hy het sy seun R~$18~000$ op sy verjaarsday gegee. Teen watter koers was die geld belê, indien enkelvoudige rente bereken was?
    \end{enumerate}
}
\end{exercises}


\begin{solutions}{}{

\begin{enumerate}[itemsep=5pt, label=\textbf{\arabic*}. ] 


\item  $P=3~500\\
i=0,075\\
n=2\\
A=?\\
 A=P(1 + i . n) \\
A = 3~500(1+ (0,075)(2))\\
A = 3~500(1.15)\\
A = \mbox{R }4~025$
\begin{multicols}{2}
\item \begin{enumerate}[noitemsep, label=\textbf{(\alph*)} ]
\item $P=300\\
i=0,08\\
n=1\\
A=?\\
 A=P(1 + i . n) \\
A = 300(1+ (0,08)(1))\\
A = 300(1,08)\\
A = \mbox{R }324$
\item $ P=2~250\\
i=0,125\\
n=6\\
A=?\\
A=P(1 + i . n) \\
A = 2~250(1+ (0,125)(6))\\
A = 2~250(1.75)\\
A =\mbox{R }3~937,50$
\end{enumerate}
\end{multicols}
\item $ A=2~500\\
P=1~000\\
i=0,082\\
n=?\\
A=P(1 + i . n) \\
2~500 = 1~000(1+ (0,082)(n))\\
\dfrac{2~500}{1~000} = 1 + 0,082n\\[4pt]
\dfrac{2~500}{1~000} - 1 = 0,082n \\[4pt]
(\dfrac{2~500}{1~000} - 1) \divide 0,082 = n\\[4pt]
n = 18,3 $\\
Dit sou 19 jare vat vir $\mbox{R }1~000$ tot $\mbox{R }2~500$ te groei by $8,2\%$ p.a.
\item $ A=18~000\\
P=5~000\\
i=?\\
n=21-5=16\\
A=P(1 + i . n) \\
18~000 = 5~000(1+ (i)(16))\\
\dfrac{18~000}{5~000} = 1 + 16i\\[4pt]
\dfrac{18~000}{5~000} - 1 = 16i \\[4pt]
(\dfrac{18~000}{5~000} - 1) \divide 16 = i\\[4pt]
i = 0,0125 $\\
Die rentekoers waarteen die geld belê was, is $1,25\%$.
\end{enumerate}

}
\end{solutions}


% \section{Compound interest}
% \subsection{The power of compound interest}
\begin{exercises}{}{
    \begin{enumerate}[label=\textbf{\arabic*}.]
	\item ’n Bedrag van R~$3~500$ word belê in ’n spaarrekening wat saamgestelde rente verdien teen $7,5\%$ per jaar.
Bereken die bedrag wat opgebou is in die rekening na verloop van 2 jaar.

	\item Morgan bel\^e R~$5~000$ in 'n rekening wat aan die einde van $5$ jaar 'n groot bedrag uitbetaal. As hy R~$7~500$ kry aan die einde van die periode, watter saamgestelde rentekoers het die bank hom aangebeid?

	\item Nicola wil geld belê teen  $11\%$ per jaar saamgestelde rente. Hoeveel geld (tot die naaste Rand) moet hy belê indien hy oor 5 jaar ’n bedrag van R~$100~000$ wil hê?
    \end{enumerate}
}
\end{exercises}


 \begin{solutions}{}{
% \begin{multicols}{2}
\begin{enumerate}[itemsep=5pt, label=\textbf{\arabic*}. ] 


\item $P = 3~500\\
i=0,075\\
n=2\\
A=P(1 + i)^{n}\\
A = 3~500(1 + 0,075)^{2}\\
A = \mbox{R }4~044,69$
\item $A = 7~500
P = 5~000\\
i=?\\
n=5\\
A=P(1 + i)^{n}\\
7~500 = 5~000(1 + i)^{5}\\[4pt]
\dfrac{7~500}{5~500} = (1+i)^5\\[4pt]
\sqrt[5]{\dfrac{7~500}{5~500}} = (1+i)\\[4pt]
\sqrt[5]{\df\mbox{R }rac{7~500}{5~500}} - 1 = i\\[4pt]
i = 0,0844717712$\\
rentekoers is $8,45 \%$ p.a
\item $A = 100~000\\
P=?\\
i=0,11\\
n=5\\
A=P(1 + i)^{n}\\
100~000 = P(1 + 0,11)^{5}\\[4pt]
\dfrac{100~000}{(1,11)^5} = P\\[4pt]
P = \mbox{R }59~345,13$

\end{enumerate}
% \end{multicols}
}
\end{solutions}

% \subsection{Hire purchase}
\begin{exercises}{}{
    \begin{enumerate}[label=\textbf{\arabic*}.]
	\item Vanessa wil 'n yskas op huurkoop koop. Die kontantprys is R~$4~500$. Sy moet 'n deposito van $15\%$ betaal en die oorblywende bedrag oor $24$ maande teen 'n rentekoers van $12\%$ p.a.
	\begin{enumerate}[label=\textbf{(\alph*)}]
	    \item Wat is die aanvangsbedrag (hoofsom)?
	    \item Wat is die eindbedrag wat sy terugbetaal?
	    \item Wat is Vanessa se maandlikse paaiemente?
	    \item Hoeveel het Vanessa in totaal vir haar yskas betaal?
	\end{enumerate}


	\item Bongani koop ’n eetkamertafel van R~$8~500$ op huurkoop. Hy moet enkelvoudige rente van $17,5\%$per jaar betaal oor 3 jaar.
	\begin{enumerate}[label=\textbf{(\alph*)}]
	    \item Hoeveel sal Bongani in totaal betaal?
	    \item  Hoeveel rente betaal hy?
	    \item Wat is sy maandelikse paaiement?
	\end{enumerate}

	\item 'n Sitkamerstel word geadverteer op TV. Dit moet oor $36$ maande terugbetaal word teen R~$150$ 'n maand.
	\begin{enumerate}[label=\textbf{(\alph*)}]
	    \item Aanvaar daar is geen deposito nodig nie, hoeveel sal die koper betaal vir die stel teen die tyd wat dit afbetaal is?
	    \item As die rentekoers $9\%$ p.a is, wat is die konstantprys van die stel?\\
	\end{enumerate}
    \end{enumerate}
}
\end{exercises}


 \begin{solutions}{}{
\begin{enumerate}[itemsep=5pt, label=\textbf{\arabic*}. ] 

\begin{multicols}{2}
\item \begin{enumerate}[itemsep=4pt, label=\textbf{(\alph*)} ]
  \item  $P=4~500 - (4~500 \times 0,15) \\
= 4~500 - 675 \\
= 3~825$
  \item $P = \mbox{R }3~825\\
i=0,12\\
n=\frac{24}{12}=2\\
A=P(1 + i.n)\\
A = 3~825(1 + (0,12)(2))\\
A=\mbox{R }4~743$
  \item $\dfrac{4~743}{24} = \mbox{R }197,63$
  \item $675 + 4743 = \mbox{R }5~418$
  \end{enumerate}
\end{multicols}
\begin{multicols}{2}
\item \begin{enumerate}[itemsep=4pt, label=\textbf{(\alph*)} ]
  \item $A=?\\
P = 8~500\\
i=0,175\\
n=3\\
A=P(1 + i.n)\\
A = 8~500(1 + (0,175)(3))\\
A=\mbox{R }12~962,50$
  \item $12~962,50-8~500=\mbox{R }4~462,50$
  \item $\dfrac{12~962,50}{36}=\mbox{R }360,07$
  \end{multicols}
\begin{multicols}{2}

\item \begin{enumerate}[itemsep=4pt, label=\textbf{(\alph*)} ]
  \item $36 \times 150 = \mbox{R }5~400$
  \item $A=5~400\\
P = ?\\
i=0,09\\
n=3\\
A=P(1 + i.n)\\
5~400 = P(1 + (0,09)(3))\\[4pt]
\dfrac{5~400}{1,27} = P\\[4pt]
P=\mbox{R }4~251,97$
  \end{enumerate}
 \end{multicols}
\end{enumerate}}
\end{solutions}

% \subsection{Inflation}
% \subsection{Population growth}
\begin{exercises}{}{
    \begin{enumerate}[label=\textbf{\arabic*}.]
	\item Die gemiddelde inflasiekoers vir die afgelope aantal jaar is $7,3\%$ per jaar en jou water- en elektrisiteitsrekening is gemiddeld R~$1~425$. Bereken wat jy kan verwag om te betaal oor 6 jaar?

	\item Die prys van springmielies en koeldrank by die fliek is nou R~$60$. As die gemiddelde inflasie $9,2\%$ is, wat was die prys $5$ jaar gelde?

	\item 'n Klien dorpie in Ohio, VSA, ondervind 'n groot toename in geboortes. As die gemiddelde groeikoers van die bevolking $16\%$ p.a, hoeveel babas sal gebore word vir die $1600$ inwoners in die volgende $2$ jaar?
    \end{enumerate}
}
\end{exercises}


 \begin{solutions}{}{
\begin{enumerate}[itemsep=5pt, label=\textbf{\arabic*}. ] 


\item $A = ?\\
P=1~425\\
i=0,073\\
n=6\\
A=P(1 + i)^{n}\\
A = 1~425(1 + 0,073)^{6}\\
A = 2~174,77$
\item $A = \mbox{R }60\\
P=?\\
i=0,092\\
n=5\\
A=P(1 + i)^{n}\\
60 = P(1 + 0,092)^{5}\\
\dfrac{60}{(1,092)^5}=P\\
P = \mbox{R }38,64$
\item $A = ?\\
P=1~600\\
i=0,16\\
n=2\\
A=P(1 + i)^{n}\\
A = 1~600(1 + 0,16)^{2}\\
A = 2~152,96\\
2~153-1~600=553$\\
Daar sal ongeveer $553$ babas in die volgende twee jare gebore wees.
\end{enumerate}}
\end{solutions}


% \subsection{Foreign exchange rates}
\begin{exercises}{}
{
    \begin{enumerate}[noitemsep, label=\textbf{\arabic*}.]
	\item Bridget wil ’n iPod koop wat £~$100$ kos, en die huidige wisselkoers is £~$1$ = R~$14$. Sy glo die wisselkoers gaan verander na R~$12$ binne ’n maand.
	\begin{enumerate}[label=\textbf{(\alph*)}]
	    \item Hoeveel sal die iPod speler in Rand kos as ek dit nou koop?
	    \item Hoeveel sal sy spaar as die wisselkoers daal na R~$12$?
	    \item Hoeveel sal sy verloor as die wisselkoers verander na R~$15$?
	\end{enumerate}

	\item Bestudeer die volgende tabel met wisselkoerse:\\
% 	\begin{center}
	    \begin{tabular}{ |l|l|l| }
		\hline
		Land	&	Geldeenheid 	&	Wisselkoers\\ \hline
		Verenigde Koninkryk(VK) 	&	Ponde(£)	&	R~$14,13$\\ \hline
		Amerika (VSA) 	&	Dollars(\$)	&	R~$7,04$\\ \hline
	    \end{tabular}
% 	\end{center}
\\
	\begin{enumerate}[label=\textbf{(\alph*)}]
	    \item In Suid-Afrika is die koste van ’n nuwe Honda Civic R~$173~400$. In Engeland kos dieselfde voertuig £~$12~200$ en in die \$~$21~900$. In watter land is die voertuig die goedkoopste as jy die pryse
omskakel na Suid-Afrikaanse Rand ?

	    \item Sollie en Arinda is kelners in ’n Suid-Afrikaanse restaurant wat baie oorsese toeriste lok. Sollie kry ’n £~$6$ fooitjie van ’n toeris en Arinda kry \$~$12$.  Hoeveel Suid-Afrikaanse Rand het elkeen gekry?
	\end{enumerate}
    \end{enumerate}
}
\end{exercises}


 \begin{solutions}{}{
\begin{enumerate}[itemsep=5pt, label=\textbf{\arabic*}. ] 


\item \begin{enumerate}[noitemsep, label=\textbf{(\alph*)} ]
\item Rand koste = (Pond koste) $\times$ wisselkoers\\
$\mbox{Rand koste}=100 \times \frac{14}{1} = \mbox{R }1~400$
\item $\mbox{Rand koste}=100 \times \frac{12}{1} = \mbox{R }1~200$\\
Dus sal sy $\mbox{R }200$ spaar ($\mbox{Besparing} = \mbox{R }1~400 - \mbox{R }1~200$)
\item $\mbox{Rand koste}=100 \times \frac{15}{1} = \mbox{R }1~500$ \\
Dus sal sy $\mbox{R }100$ verloor ($\mbox{Verloor} = \mbox{R }1~400 - \mbox{R }1~500$)
\end{enumerate}
\item \begin{enumerate}[noitemsep, label=\textbf{(\alph*)} ]
\item Om hierdie vraag te beantwoord, bereken ons die koste van die motor in Rand vir elke land, en dan die drie antwoorde vergelyk om te sien watter die goedkoopste is. Rand koste  = geldeenheid koste $\times$ wisselkoers .\\
Koste in VK: $12~200 \times \frac{14,13}{1} = \mbox{R }172~386$\\
Koste in VSA: $21~900 \times \frac{7,04}{1} = \mbox{R }154~400$\\
Deur vergelkying van die drie koste sien ons dat die kar die goedkoopste in die VSA is.

\item Sollie: $6 \times \frac{14,31}{1} = \mbox{R }84,78$\\
Arinda: $12 \times \frac{7,04}{1} = \mbox{R }84,48$.
\end{enumerate}

\end{enumerate}}
\end{solutions}


\begin{eocexercises}{}
    \begin{enumerate}[label=\textbf{\arabic*}.]
	\item Alison is met vakansie in Europa. Haar hotel vra €~$200$ euro per nag. Hoeveel Rand het sy nodig om die hotelrekening te betaal as die wisselkoers  €~$1$ = R~$9,20$?

	\item Bereken hoeveel rente jy sal verdien as jy R~$500$ vir 1 jaar belê teen die volgende rentekoerse:
	\begin{enumerate}[noitemsep, label=\textbf{(\alph*)} ]
	    \item $6,85\%$ enkelvoudige rente.
	    \item $4,00\%$ saamgetelde rente.
	\end{enumerate}

	\item Bianca het R~$1~450$ om vir 3 jaar te belê. Bank A bied ’n spaarrekening aan wat enkelvoudige rente betaal
teen ’n koers van $11\%$ per annum. Bank B bied ’n spaarrekening aan wat saamgestelde rente betaal
teen ’n koers van $10,5\%$ per annum. Watter bank gaan vir Bianca die spaarrekening gee met die grootste
opgeloopte bedrag aan die einde van die 3 jaar?

	\item Hoeveel enkelvoudige rente is betaalbaar op ’n lening van R~$2~000$ vir ’n jaar indien die rentekoers $10\%$ per jaar is?

	\item Hoeveel saamgestelde rente is betaalbaar op ’n lening van  R~$2~000$ vir ’n jaar indien die rentekoers $10\%$ per jaar is?

	\item Bespreek:
	\begin{enumerate}[noitemsep, label=\textbf{(\alph*)} ]
	    \item watter soort rente jy sal verkies as jy die lener is?

	    \item watter soort rente jy sal verkies as jy die bankier is?
	\end{enumerate}

	\item Bereken die saamgestelde rente vir die volgende probleme.
	\begin{enumerate}[noitemsep, label=\textbf{(\alph*)} ]
	    \item ’n lening van R~$2~000$ vir 2 jaar teen  $5\%$ per jaar
	    \item ’n belegging van  R~$1~500$ vir 3 jaar teen $6\%$ per jaar
	    \item ’n lening van R~$800$ vir l jaar teen $16\%$per jaar
	\end{enumerate}

	\item As die wisselkoers vir ¥~$100$ = R~$6,2287$ (¥ = Yen) en 1 Australiese dollar (AUD) = R~$5,1094$, bepaal die wisselkoers
tussen die Australiese dollar en die Japanese jen.
	\item Bonnie het ’n stoof gekoop vir R~$3~750$. Na 3 jaar het sy dit klaar betaal, asook die R~$956,25$ huurkoopkoste.
Bereken die koers waarteen enkelvoudige rente bereken is.
	\item Volgens die nuutste sensus, het Suid-Afrika 'n bevolking van $57~000~000$.
	\begin{enumerate}[noitemsep, label=\textbf{(\alph*)} ]
	    \item Indien die jaarlikse groeikoers $0,9\%$ verwag word, bereken hoeveel Suid-Afrikaners daar in $10$ jaar sal wees (korrek tot die naaste honderd duisend).

	    \item As na 10 jaar word dit gevind dat die bevolking eintlik by $10$ miljoen tot by $67$ miljoen gegroei het, wat was die groeikoers?
	\end{enumerate}
    \end{enumerate}
\end{eocexercises}


 \begin{eocsolutions}{}{
\begin{enumerate}[itemsep=5pt, label=\textbf{\arabic*}. ] 


\item $\mbox{Rand koste}=\mbox{Euro koste} \times \mbox{wisselkoers }\\
=200 \times 9,201\\
 = \mbox{R }1~840$
\begin{multicols}{2}
\item \begin{enumerate}[noitemsep, label=\textbf{(\alph*)} ]
	    \item  $P=500\\
i=0,685\\
n=1\\
A=?\\
 A=P(1 + i . n) \\
A = 500(1+ (0,685)(1))\\
A = 500(1,685)\\
A = \mbox{R }534,25$

 \item  $P = 500\\
i=0,04\\
n=1\\
A=?\\
A=P(1 + i)^{n}\\
A = 500(1 + 0,04)^{1}\\
A = \mbox{R }520$
	\end{enumerate}
\end{multicols}
\item Bank A:\\
$P=1~450\\
i=0,11\\
n=3\\
A=?\\
 A=P(1 + i . n) \\
A = 1~450(1+ (0,11)(3))\\
A = 1~450(1,33)\\
A = \mbox{R }1~925,50$\\
Bank B:\\
 $P = 1~450\\
i=0,150\\
n=3\\
A=?\\
A=P(1 + i)^{n}\\
A = 1~450(1 + 0,150)^{3}\\
A = \mbox{R }1956,39$\\
Sy moet Bank B kies as dit sal haar meer geld oor 3 jaar verdien.
\item $P=2~000\\
i=0,10\\
n=1\\
A=?\\
 A=P(1 + i . n) \\
A = 2~000(1+ (0,10)(1))\\
A = 2~000(1,10)\\
A = \mbox{R }2~200$\\
Dus is die bedrag rentekoers:\\
$2~200 - 2~000 = \mbox{R }200$
\item  $P = 2~000\\
i=0,10\\
n=1\\
A=?\\
A=P(1 + i)^{n}\\
A = 2~000(1 + 0,10)^{1}\\
A = \mbox{R }2~200$\\
Dus is die bedrag rentekoers:\\
$2~200 - 2~000 = \mbox{R }200$
\item 	\begin{enumerate}[noitemsep, label=\textbf{(\alph*)} ]
	    \item Enkelvoudige rente. Rente is op die hoofsom en nie op die rente verdien gedurende die vorige tydperke bereken . Dit sal lei tot die lener om minder rente moet betaal.

% Simple interest. Interest is only calculated on the principal amount and not on the interest earned during prior periods. This will lead to the borrower paying less interest.
	    \item Saamgestelde rente. Rente is vanaf die prinsipale bedrag bereken sowel as rente verdien uit vorige tydperke. Dit sal lei tot die bankier om meer geld vir die bank te verdien.
	\end{enumerate}
\begin{multicols}{2}
\item 	\begin{enumerate}[noitemsep, label=\textbf{(\alph*)} ]
	    \item  $P = 2~000\\
i=0,05\\
n=2\\
A=?\\
A=P(1 + i)^{n}\\
A = 2~000(1 + 0,05)^{2}\\
A = \mbox{R }2~205$\\
Dus is die bedrag rentekoers:\\
$2~205 - 2~000 = \mbox{R }205$
	    \item $P = 1~500\\
i=0,06\\
n=3\\
A=?\\
A=P(1 + i)^{n}\\
A = 1~500(1 + 0,06)^{3}\\
A = \mbox{R }1~786,524$\\
Dus is die bedrag rentekoers:\\
$1~786,524 - 1~500 = \mbox{R }286,52$
	    \item $P = 800\\
i=0,16\\
n=1\\
A=?\\
A=P(1 + i)^{n}\\
A = 800(1 + 0,16)^{1}\\
A = \mbox{R }928$\\
Dus is die bedrag rentekoers:\\
$928 - 800 = \mbox{R }128$
	\end{enumerate}
\end{multicols}
\item $\dfrac{\mbox{AUD}}{\mbox{Yen}}=\dfrac{\mbox{ZAR}}{\mbox{Yen}} \times \dfrac{\mbox{AUD}}{\mbox{Yen}} \\[4pt]
= \dfrac{6,2287}{100} \times 15,1094 \\[4pt]
= \dfrac{0,01219}{0,00219} \mbox{ AUD} \\[4pt]
= 1 \mbox{ Yen} \\
\mbox{or } 1 \mbox{ AUD} = 82,03 \mbox{ Yen}$
\item $\mbox{Totale bedrag } = 3~750 + 956,25 = 4~706,25$\\
$P=3~750\\
i=?\\
n=3\\
A=4~706,25\\
 A=P(1 + i . n) \\
4~706,25 = 3~750(1+ i(3))\\
1,255 = (1 + 3i)\\
0,255 = 3i \\
i = 0,085$\\
Dus is die bedrag rentekoers $8,5 \%$

\item
\begin{enumerate}[noitemsep, label=\textbf{(\alph*)} ]
    \item $A = 57~000~000 (1 + \frac{0,9}{100})^{10}\\
	     = 57~000~000 (1,009)^{10}\\
	     = 57~000~000 (1,0937)^{10}\\
	     = 62,3$ miljoen mense
    \item $67 = 57 (1 + \frac{i}{100})^{10}\\
\sqrt[10]{\frac{67}{57}} = 1 + \frac{i}{100}\\
	\frac{i}{100} = 1,01629 - 1\\
	    i = 100 (0,016)\\
	    i = 1,69\\
	    i \approx 1,7\%$
\end{enumerate}
\end{enumerate}}
\end{eocsolutions}


