\chapter{Algebra\"iese uitdrukkings}
%Ex 1-1 Questions
\begin{exercises}{}{
\begin{enumerate}[itemsep=5pt, label=\textbf{\arabic*}. ] 
\item Sê of die volgende getalle rasionaal of irrasionaal is. As die getal rasionaal is, sê of dit 'n natuurlike getal, telgetal of heelgetal is:
\begin{enumerate}[itemsep=5pt, label=\textbf{(\alph*)} ] 
    \item $-\dfrac{1}{3}$
    \item $0,651268962154862\ldots$
    \item $\dfrac{\sqrt{9}}{3}$
    \item $\pi^2$
\end{enumerate}


\item As $a$ 'n heel getal is, $b$ 'n heel getal is en $c$ is irrasionaal, watter van die volgende is rasionale getalle?  
  \begin{enumerate}[itemsep=5pt, label=\textbf{(\alph*)} ] 
    \item $\dfrac{5}{6}$
    \item $\dfrac{a}{3}$
    \item $\dfrac{-2}{b}$
    \item $\dfrac{1}{c}$
    \end{enumerate}
\item Vir watter van die volgende waardes van $a$ is $\frac{a}{14}$ rasionaal of irrasionaal?
    \begin{enumerate}[itemsep=0pt, label=\textbf{(\alph*)} ] 
    \item $1$
    \item $-10$
    \item $\sqrt{2}$
    \item $2,1$
    \end{enumerate}
\item Skryf die volgende as breuke:
    \begin{enumerate}[itemsep=0pt, label=\textbf{(\alph*)} ] 
    \item $0,1$
    \item $0,12$
    \item $0,58$
    \item $0,2589$
    \end{enumerate}
\item Skryf die volgende in repeterende (herhalende) desimale notasie:
    \begin{enumerate}[itemsep=0pt, label=\textbf{(\alph*)} ] 
    \item $0,11111111\ldots$
    \item $0,1212121212\ldots$
    \item $0,123123123123\ldots$
    \item $0,11414541454145\ldots$
    \end{enumerate}
\item Skryf die volgende in repeterende (herhalende) desimale notasie:
    \begin{enumerate}[itemsep=5pt, label=\textbf{(\alph*)} ] 
    \item $\dfrac{2}{3}$
    \item $1\dfrac{3}{11}$
    \item $4\dfrac{5}{6}$
    \item $2\dfrac{1}{9}$
    \end{enumerate}
\item Skryf die volgende in breukvorm:
    \begin{enumerate}[itemsep=2pt, label=\textbf{(\alph*)} ] 
    \item $0,\dot{5}$
    \item $0,6\dot{3}$
    \item $5,\overline{31}$
    \end{enumerate}
\end{enumerate}
}
\end{exercises}


 \begin{solutions}{}{

%Exercise 1-1 Solutions
\begin{enumerate}[itemsep=5pt, label=\textbf{\arabic*}. ] 
 \item \begin{multicols}{2} %State whether the following numbers are $\mathbb{Q}$ or $\mathbb{Q'}$. If the number is $\mathbb{Q}$, state whether it is $\mathbb{N}_0$, $\mathbb{N}$ or $\mathbb{Z}$:
  \begin{enumerate}[noitemsep, label=\textbf{(\alph*)} ] 
    \item $\mathbb{Q}, \mathbb{Z}$ %$-\dfrac{1}{3}$
    \item $\mathbb{Q'}$%$0,651268962154862\ldots$
    \item $\mathbb{Q}, \mathbb{Z}, \mathbb{N}, \mathbb{N}_0$%$\dfrac{\sqrt{9}}{3}$
    \item $\mathbb{Q'}$ %$\pi^2$
  \end{enumerate} 


\item   %Question 1
\begin{enumerate}[itemsep=0pt, label=\textbf{(\alph*)} ] 
   \item Rasionaal
\item Rasionaal
\item Rasionaal
\item Irrasionaal 
  \end{enumerate}
\end{multicols}
\item %Question 2
\begin{multicols}{2}
\begin{enumerate}[itemsep=0pt, label=\textbf{(\alph*)} ] 
 \item Rasionaal
\item Rasionaal
\item Irrasionaal
\item Rasionaal
\end{enumerate}
\end{multicols}
\item %Question 3
\begin{multicols}{2}
\begin{enumerate}[itemsep=6pt, label=\textbf{(\alph*)} ] 
\item $0,1 = \dfrac{1}{10}$
\item $0,12 = \dfrac{12}{100} = \dfrac{3}{25}$
\item $0,58 = \dfrac{58}{100} = \dfrac{29}{50}$
\item $0,2589 = \dfrac{2~589}{10~000}$
\end{enumerate}
\end{multicols}
\item %Question 4
\begin{multicols}{2}
\begin{enumerate}[itemsep=3pt, label=\textbf{(\alph*)} ] 
\item $0,\dot{1}$
\item $0,\overline{12}$
\item $0,\overline{123}$
\item $0,11\overline{4145}$
\end{enumerate}
\end{multicols}
\item %Question 5
\begin{multicols}{2}
\begin{enumerate}[itemsep=5pt, label=\textbf{(\alph*)} ] 
\item 
\begin{array*}
$\frac{2}{3} &= 2\left(\frac{1}{3}\right) \\
&= 2(0,333333\ldots) \\
&= 0,666666\ldots \\
&= 0,\dot{6} \\
$\end{array*}

\item \begin{array*} $1\frac{3}{11} &= 1 + 3\left(\frac{1}{11}\right) \\
        &= 1 + 3(0,090909\ldots)\\
&= 1 + 0,27272727\ldots \\
&= 1,\overline{27}$
       \end{array*}

\item \begin{array*} $
        4\frac{5}{6} &= 4 + 5\left(\frac{1}{6}\right) \\
&= 4+ 5(0,1666666\ldots) \\
&= 4 + 0,833333\ldots \\& = 4,8\dot{3}$
       \end{array*}

\item \begin{array*}$
        2\frac{1}{9} &= 2 + 0,1111111\ldots \\
&= 2,\dot{1}$
       \end{array*}

\end{enumerate}
\end{multicols}
\item %Question 6
\begin{multicols}{2}
\begin{enumerate}[itemsep=5pt, label=\textbf{(\alph*)} ] 
 \item \begin{array*}$        
x=0,55555$ en \\$10x = 5,55555 \\
\therefore 10x-x = 9x\\= 5$\\
 $\therefore x=\frac{5}{9}$
\end{array*}

\item  \begin{array*} $10x = 6,3333$ en\\ $100x= 63,3333 \\
        \therefore 100x-10x = 90x \\= 57$\\
 $\therefore x=\frac{57}{90}$
       \end{array*}

\item \begin{array*}
       $x = 5,313131$ en \\$100x=531,313131\\
 \therefore 100x-x=99x=\\526$ \\
$\therefore x=\frac{526}{99}$
      \end{array*}

\end{enumerate}
\end{multicols}
\end{enumerate}
}
\end{solutions}

%Exercise 1-2 Questions
\begin{exercises}{}
{

Skryf die volgende getalle tot $3$ desimale plekke:
\begin{multicols}{2}
\begin{enumerate}[itemsep=3pt, label=\textbf{\arabic*}. ]
\item $12,56637061\ldots$ %$4\pi$
\item $3,31662479\ldots$ %$\sqrt{11}$
\item $0,26666666\ldots$ %$\dfrac{0,8}{3}$
\item $1,912931183\ldots$ %$\sqrt[3]{7}$
\item $6,32455532\ldots$ %$2\sqrt{10}$
\item $0,05555555\ldots$ %$\dfrac{1}{18}$
\end{enumerate}
\end{multicols}
}
\end{exercises}

% Ex 1-2 Solutions

 \begin{solutions}{}{
\begin{multicols}{2}
\begin{enumerate}[itemsep=0pt, label=\textbf{\arabic*}. ] 
 \item $12,5666$%
\item $3,317$%
\item $0,267$%
\item $1,913$
\item $6,325$%
\item $0,056$%
\end{enumerate}
\end{multicols}}
\end{solutions}


% Ex1-3 Questions
\begin{exercises}{}
 {
Vind twee opeenvolgende heelgetalle wat weerskante van die getalle lê, sonder die gebruik van 'n sakrekenaar:
\begin{multicols}{2}
\begin{enumerate}[noitemsep, label=\textbf{\arabic*}. ]
\item $\sqrt{18}$
\item $\sqrt{29}$
\item $\sqrt[3]{5}$
\item $\sqrt[3]{79}$
\end{enumerate}
\end{multicols}
}
\end{exercises}

% Ex 1-3 solutions
 \begin{solutions}{}{
\begin{multicols}{2}
\begin{enumerate}[noitemsep, label=\textbf{\arabic*}. ]
\item $4$ en $5$ $(4^2 = 16$ en $5^2=25)$%
\item $5$ en $6$ $(5^2 = 25$ en $6^2=36)$%
\item $1$ en $2$ $(1^3 = 1$ en $2^3=8)$%
\item $4$ en $5$ $(4^3 = 64$ en $5^3=125)$%
\end{enumerate}
\end{multicols}
}
\end{solutions}


% Ex 1-4 questions
\begin{exercises}{}
{
Vind die produkte:
\begin{multicols}{2}
\begin{enumerate}[label=\textbf{\arabic*}., itemsep=5pt]
\item $2y(y+4)$ 
\item $(y+5)(y+2) $
\item $(2-t)(1-2t)$
\item $(x-4)(x+4)$
\item $ (2p+9)(3p+1)$
\item $(3k-2)(k+6)$
\item $(s+6)^2$
\item $-(7-x)(7+x)$
\item $(3x-1)(3x+1)$
\item $(7k+2)(3-2k)$
\item $(1-4x)^2$
\item $(-3-y)(5-y)$
\item $(8-x)(8+x)$
\item $(9+x)^2$
\item $(-2{y}^{2}-4y+11)(5y-12)$ 
\item $(7{y}^{2}-6y-8)(-2y+2)$% make-rowspan-placeholders
\item $(10{y}+3)(-2{y}^{2}-11y+2)$ 
\item $(-12y-3)(12{y}^{2}-11y+3)$% make-rowspan-placeholders
\item $(-10)(2{y}^{2}+8y+3)$ 
\item $(2{y}^{6}+3{y}^{5})(-5y-12)$% make-rowspan-placeholders
\item $(-7y+11)(-12y+3)$% make-rowspan-placeholders
\item $(7y+3)(7{y}^{2}+3y+10)$% make-rowspan-placeholders
\item $(9)(8{y}^{2}-2y+3)$ 
\item $(-6{y}^{4}+11{y}^{2}+3y)(y+4)(y-4)$ 
\end{enumerate}
\end{multicols}

}
\end{exercises}

% Ex1-4 solutions
 \begin{solutions}{}{
\begin{multicols}{2}
\begin{enumerate}[itemsep=5pt, label=\textbf{\arabic*}. ]
 \item $2y(y+4)=2y^2 + 8y$%
\item \begin{array*}$(y+5)(y+2)\\&=y^2 + 2y+5y+10 \\&=y^2 + 7y + 10$\end{array*}%$( $
\item \begin{array*}$(2-t)(1-2t)\\&=2+4t-t+2t^2\\&=2 + 3t +2t^2$\end{array*}%$(2-t)(1-2t)$
\item \begin{array*}$(x-4)(x+4)\\&=x^2+4x-4x-16\\&=x^2 - 16$\end{array*}%$(x-4)(x+4)$
\item \begin{array*}$(2p+9)(3p+1)\\&=6p^2+2p+27p+9\\&=6p^2 + 29p + 9$\end{array*}%$ (2p+9)(3p+1)$
\item \begin{array*}$(3k-2)(k+6)\\&=3k^2+18k-2k-12\\&=3k^2 +16k - 12$\end{array*}%$(3k-2)(k+6)$
\item \begin{array*}$(s+6)^2\\&=s^2 + 6s+6s+36\\&=s^2 + 12s +36$\end{array*}%$(s+6)^2$
\item 
\begin{array*}
 $-(7-x)(7+x)\\
&=-(49+7x-7x-x^2)\\
&=-(49-x^2)\\
&=-49 + x^2$%$-(7-x)(7+x)$
\end{array*}


\item \begin{array*}$(3x-1)(3x+1)\\&=9x^2+3x-3x-1\\&=9x^2 - 1$\end{array*}%$(3x-1)(3x+1)$
\item \begin{array*}$(7k+2)(3-2k)\\&=21k-14k^2 + 6-4k\\&=14k^2 + 17k + 6$\end{array*}%$(7k+2)(3-2k)$
\item 
\begin{array*}
$(1-4x)^2\\
&=(1-4x)(1-4x)\\
&=1-4x-4x+16x^2\\
&=1 -8x + 16x^2$%$(1-4x)^2$
\end{array*}

\item \begin{array*}$(-3-y)(5-y)\\&=-15+3y-5y+y^2\\&=y^2 - 2y - 15$\end{array*}%$$
\item \begin{array*}$(8-x)(8+x)\\&=16+8x-8x-x^2\\&=16 - x^2$\end{array*}%$$
\item 
\begin{array*} $(9+x)^2\\ &=(9+x)(9+x)\\&=&81+9x+9x+x^2\\&=&81 + 18x + x^2$\end{array*}%$$
\item\begin{array*} $(-2{y}^{2}-4y+11)(5y-12)\\&=-10y^3+24y^2-20y^2+48y+55y-132\\&=-10y^3 + 4y^2 + 103y - 132$\end{array*}% $$ 
\item \begin{array*}$(7{y}^{2}-6y-8)(-2y+2)\\&=-14y^3+14y^2+12y^2-12y+16y-16\\&=-14y^3 + 26y^2 + 4y -16$\end{array*} % $$% 
\item \begin{array*}$(10{y}^{5}+3)(-2{y}^{2}-11y+2)\\&=-20y^3 - 110y^2 +20y-6y^2-33y+6\\&=-20y^3 -116y^2 -13y +6$\end{array*}%$$ 
\item \begin{array*}$(-12y-3)(12{y}^{2}-11y+3)\\&=-144y^3+132y^2-36y-36y^2+33y-9\\&=-144y^3 + 96y^2 -3y -9$%$$% 
\item \begin{array*}$(-10)(2{y}^{2}+8y+3)==-20y^2 - 80y - 30$\end{array*}%$$ 
\item \begin{array*}$(2{y}^{6}+3{y}^{5})(-5y-12)\\&=-10y^7-24y^6-15y^6-36y^5\\&=-10y^7 - 39y^6 - 36y^5$\end{array*}%$$% 
\item \begin{array*}$(-7y+11)(-12y+3)\\&=84y^2-21y-132y+33\\&=84y^2 - 153y +33$\end{array*}%$$% 
\item\begin{array*} $(7y+3)(7{y}^{2}+3y+10)\\&=49y^3+21y^2+70y+21y^2+9y+10\\&=49y^3 + 42y^2 + 79y + 10$\end{array*}%$$% m
\item \begin{array*}$(9)(8{y}^{2}-2y+3)=72y^2 - 18y + 27$\end{array*}%$$ 
\item \begin{array*}$(-6{y}^{4}+11{y}^{2}+3y)(10y+4)(4y-4)\\&=(-6y^4+11y^2+3y)(y^2+16)&\\&=-6y^6-96y^4+11y^4+176y^2+3y^3+48y\\&=-6y^6 - 85y^4 + 3y^3 + 176y^2 + 48y$\end{array*}%$$ 
\end{enumerate}
\end{multicols}
}
\end{solutions}


% Ex 1-5 questions
% \begin{exercises}{}
% {
% Find the highest common factors of the
% following pairs of terms:
% \begin{multicols}{3}
% \begin{enumerate}[label=\textbf{\arabic*}., itemsep=0pt]
% \item $6y;~18x$
% \item $12mn;~8n$
% \item $3st;~4su$ 
% \item $18kl;~9kp$
% \item $abc;~ac$% 
% \item $2xy;~4xyz$
% \item $3uv;~6u$ 
% \item $9xy;~15xz$
% \item $24xyz;~16yz$
% \item $3m;~45n$
% \end{enumerate}
% \end{multicols}
% 
% }
% \end{exercises}
% 
% % Ex 1-5 solutions
%  \begin{solutions}{}{
% \begin{multicols}{3}
% \begin{enumerate}[label=\textbf{\arabic*}., noitemsep]
% \item $6$%$6y;~18x$
% \item $4n$%$12mn;~8n$
% \item $s$%$3st;~4su$ 
% \item $9k$%$18kl;~9kp$
% \item $ac$%$abc;~ac$% 
% \item $2xy$%$2xy;~4xyz$
% \item $3u$%$3uv;~6u$ 
% \item $3x$%$9xy;~15xz$
% \item $4yz$%$24xyz;~16yz$
% \item $3$%$3m;~45n$
% \end{enumerate}
% \end{multicols}
% }
% \end{solutions}


% Ex 1-5 questions
\begin{exercises}{}{
Faktoriseer:
\begin{multicols}{2}
\begin{enumerate}[itemsep=2pt, label=\textbf{\arabic*}. ] 
\item $2l+2w$
\item $12x+32y$
\item $6{x}^{2}+2x+10{x}^{3}$
\item $2x{y}^{2}+x{y}^{2}z+3xy$
\item $-2a{b}^{2}-4{a}^{2}b$
\item $7a+4$ 
\item $20a-10$ 
\item $18ab-3bc$
\item $12kj+18kq$ 
\item $16{k}^{2}-4$ 
\item $3{a}^{2}+6a-18$
\item $-12a+24a^3$ 
\item $-2ab-8a$ 
\item $24kj-16{k}^{2}j$
\item $-{a}^{2}b-{b}^{2}a$ 
\item $12{k}^{2}j+24{k}^{2}{j}^{2}$ 
\item $72{b}^{2}q-18{b}^{3}{q}^{2}$
\item $4(y-3)+k(3-y)$ 
\item $a^2(a-1)-25(a-1)$ 
\item $bm(b+4)-6m(b+4)$
\item ${a}^{2}(a+7)+9(a+7)$ 
\item $3b(b-4)-7(4-b)$ 
\item ${a}^{2}{b}^{2}{c}^{2}-1$
\end{enumerate}
\end{multicols}

}
\end{exercises}

% Ex 1-5 solutions
 \begin{solutions}{}{
\begin{multicols}{2}
\begin{enumerate}[itemsep=2pt, label=\textbf{\arabic*}. ] 
\item $2l+2w=2(l + w)$%$$
\item $12x+32y=4(3x + 8y)$%$$
\item $6{x}^{2}+2x+10{x}^{3}=2x(3x + 1 +5x^2)$%$$
\item $2x{y}^{2}+x{y}^{2}z+3xy=xy(2y^2 + yz + 3)$%$$
\item $-2a{b}^{2}-4{a}^{2}b=-2ab(b + a)$%$$
\item $7a+4=7a + 4$%$$ 
\item $20a-10=10(2a - 1)$%$$ 
\item $18ab-3bc=3b(6a - c)$%$$
\item $12kj+18kq=6k(2j + 3q)$%$$ 
\item $16{k}^{2}-4=(4k - 2)(4k + 2)$%$$ 
\item $3{a}^{2}+6a-1=3(a^2 + 2a - 9)$%$8$
\item $-12a+24a^3=12a( 2a^2 -1)$%$$ 
\item $-2ab-8a=-2a(b + 4)$%$$ 
\item $24kj-16{k}^{2}j=8kj(3 - 2k)$%$$
\item $-{a}^{2}b-{b}^{2}a=-ab(a + b)$%$$ 
\item $12{k}^{2}j+24{k}^{2}{j}^{2}=12jk^2(2j+1)$%$$ 
\item $72{b}^{2}q-18{b}^{3}{q}^{2}=18b^2q(4 - bq)$%$$
\item $4(y-3)+k(3-y)=(3 - y)(-4 + k)$%$$ 
\item \begin{array*}$a^2(a-1)-25(a-1)\\&=(a-1)(a^2-25)\\&=(a - 1)(a - 5)(a + 5)$\end{array*}%$$ 
\item \begin{array*}$bm(b+4)-6m(b+4)\\&=(b+4)(bm-6m)\\&=(b + 4)(m)(b - 6)$\end{array*}%$$
\item ${a}^{2}(a+7)+9(a+7)=(a + 7)(a^2 + 9)$ %$$ 
\item $3b(b-4)-7(4-b)=(b - 4)(3b + 7)$%$$ 
\item ${a}^{2}{b}^{2}{c}^{2}-1=(abc - 1)(abc + 1)$%$$
\end{enumerate}
\end{multicols}}
\end{solutions}

% ex 1-6 questions


% Ex 1-6 questions
\begin{exercises}{}{

Faktoriseer die volgende:
\begin{multicols}{2}
\begin{enumerate}[itemsep=2pt, label=\textbf{\arabic*}. ] 
\item $6x+a+2ax+3$
\item ${x}^{2}-6x+5x-30$
\item $5x+10y-ax-2ay$
\item ${a}^{2}-2a-ax+2x$
\item $5xy-3y+10x-6$
\item $ab - a^{2} - a + b$
\end{enumerate}
\end{multicols}

}
\end{exercises}

% Ex 1-6 solutions
 \begin{solutions}{}{
\begin{multicols}{2}
\begin{enumerate}[itemsep=5pt, label=\textbf{\arabic*}. ] 
\item \begin{array*}$6x+a+2ax+3\\&=3(2x+1)+a(2x+1)\\&=(3 + a)(2x + 1)$\end{array*}%$$
\item \begin{array*}${x}^{2}-6x+5x-30\\&=x(x-6)+5(x-6)\\&=(x + 5)(x - 6)$\end{array*}%$$
\item \begin{array*}$5x+10y-ax-2ay\\&=5(x+2y)-a(x+2y)\\&=(5 - a)(x + 2y)$\end{array*}%$$
\item \begin{array*}${a}^{2}-2a-ax+2x\\&=a(a-2)-x(a-2)\\&=(a - x)(a - 2)$\end{array*}%$$
\item \begin{array*}$5xy-3y+10x-6\\&=y(5x-3)+2(5x-3)\\&=(y + 2)(5x - 3)$\end{array*}%$$
\item $ab - a^{2} - a + b=(-a + b)(a + 1)$%$$
\end{enumerate}
\end{multicols}}
\end{solutions}

\begin{exercises}{}
{
\begin{enumerate}[itemsep=5pt, label=\textbf{\arabic*}. ] 
\item Faktoriseer die volgende:
\begin{multicols}{2}
\begin{enumerate}[itemsep=2pt, label=\textbf{(\alph*)} ] 
\item ${x}^{2}+8x+15$
\item ${x}^{2}+10x+24$
\item ${x}^{2}+9x+8$
\item ${x}^{2}+9x+14$
\item ${x}^{2}+15x+36$
\item ${x}^{2}+12x+36$
\end{enumerate}
\end{multicols}
\item Ontbind die volgende in faktore:
\begin{multicols}{2}
\begin{enumerate}[itemsep=2pt, label=\textbf{(\alph*)} ]  
\item ${x}^{2}-2x-15$
\item ${x}^{2}+2x-3$
\item ${x}^{2}+2x-8$
\item ${x}^{2}+x-20$
\item ${x}^{2}-x-20$
\item $2{x}^{2}+22x+20$
\end{enumerate}
\end{multicols}
\item Vind die faktore van die volgende drieterme:
\begin{multicols}{2}
\begin{enumerate}[itemsep=2pt, label=\textbf{(\alph*)} ] 
\item $3{x}^{2}+19x+6$
\item $6{x}^{2}+7x+1$
\item $12{x}^{2}+8x+1$
\item $8{x}^{2}+6x+1$
\end{enumerate}
\end{multicols}
\item Faktoriseer:
\begin{multicols}{2}
\begin{enumerate}[itemsep=2pt, label=\textbf{(\alph*)} ] 
\item $3{x}^{2}+17x-6$
\item $7{x}^{2}-6x-1$
\item $8{x}^{2}-6x+1$
\item $6{x}^{2}-15x-9$
\end{enumerate}
\end{multicols}
\end{enumerate}

}
\end{exercises} 

% ex 1-6 solutions
 \begin{solutions}{}{

\begin{enumerate}[itemsep=5pt, label=\textbf{\arabic*}. ] 
\item %%Factorise the following:
\begin{multicols}{2}
\begin{enumerate}[itemsep=2pt, label=\textbf{(\alph*)} ] 
\item ${x}^{2}+8x+15=(x + 5)(x + 3)$%$$
\item ${x}^{2}+10x+24=(x + 6)(x + 4)$%$$
\item ${x}^{2}+9x+8=(x + 8)(x + 1)$%$$
\item ${x}^{2}+9x+14=(x + 7)(x + 2)$%$$
\item ${x}^{2}+15x+36=(x + 12)(x + 3)$%$$
\item ${x}^{2}+12x+36=(x + 6)(x + 6)$%$$

\end{enumerate}

\end{multicols}

\item %%Write the following expressions in factorised form:
\begin{multicols}{2}
\begin{enumerate}[itemsep=2pt, label=\textbf{(\alph*)} ]  

% \setcounter{enumi}{6}
\item ${x}^{2}-2x-15=(x + 5)(x - 3)$%$
\item ${x}^{2}+2x-3=(x + 3)(x - 1)$%$$
\item ${x}^{2}+2x-8=(x + 4)(x - 2)$%$$
\item ${x}^{2}+x-20=(x + 5)(x - 4)$%$$
\item ${x}^{2}-x-20=(x - 5)(x + 4)$%$$
\item \begin{array*}$2{x}^{2}+22x+20\\&=2(x^2+11x+10)\\&=2(x + 10)(x + 1)$\end{array*}%$$

\end{enumerate}
\end{multicols}


\item %%Find the factors of the following trinomial expressions:
\begin{multicols}{2}
\begin{enumerate}[itemsep=2pt, label=\textbf{(\alph*)} ] 


\item $3{x}^{2}+19x+6=(3x + 1)(x + 6)$%$$
\item $6{x}^{2}+7x+2=(6x + 1)(x + 1)$%$$
\item $12{x}^{2}+8x+1=(6x + 1)(2x + 1)$%$$
\item $8{x}^{2}+6x+1=(2x + 1)(4x + 1)$%$$
\end{enumerate}
\end{multicols}

\item %%Factorise completely:
\begin{multicols}{2}
\begin{enumerate}[itemsep=2pt, label=\textbf{(\alph*)} ] 
% \setcounter{enumi}{16}
\item $3{x}^{2}+17x-6=(3x + 1)(x + 6)$%$$
\item $7{x}^{2}-6x-1=(7x + 1)(x - 1)$%$$
\item $8{x}^{2}-6x+1=(4x - 1)(2x - 1)$%$$
\item $6{x}^{2}-15x-9=(6x + 3)(x - 3)$%$$
\end{enumerate}
\end{multicols}
\end{enumerate}

}
\end{solutions}
% Ex 1-9 questions
\begin{exercises}{}
{
Faktoriseer:
\begin{multicols}{2}
\begin{enumerate}[itemsep=2pt, label=\textbf{\arabic*}. ] 
\item ${x}^{3}+8$
\item $27-m^{3}$
\item $2x^{3}-2y^{3}$
\item $3k^{3} + 27q^{3}$
\item $64t^{3}-1$
\item $64x^{2} -1$
\item $125x^{3} +1$
\item $25x^{2} +1$
\item $z-125z^4{}$
\item $8m^{6} + n^{9}$
\item $p^{15} - \frac{1}{8}y^{12}$
\item $1- (x-y)^3$
\end{enumerate}
\end{multicols}

}
\end{exercises}

% Ex 1-9 solutions
 \begin{solutions}{}{
\begin{multicols}{2}
\begin{enumerate}[itemsep=5pt, label=\textbf{\arabic*}. ] 
\item ${x}^{3}+8=(x + 2)(x^2 - 2x + 4)$%$$
\item $27-m^{3}=(3 - m)(9 + 3m + m^2)$%$$
\item \begin{array*}$2x^{3}-2y^{3}\\&=2(x^3-y^3)\\&=2(x - y)(x^2 + xy + y^2)$\end{array*}%$$
\item \begin{array*}$3k^{3} + 27q^{3}\\&=3(k^3+27q^3)\\&=3(k + 3q)(k^2 - 3kq + 9q^2)$\end{array*}%$$
\item $64t^{3}-1=(4t - 1)(16t^2 + 4t + 1)$%$$
\item $64x^{2} -1=(8x - 1)(8x + 1)$%$$
\item $125x^{3} +1=(5x + 1)(25x^2 - 5x + 1)$%$$
\item $25x^{2} +1=(5x + 1)(5x - 1)$%$$
\item \begin{array*}$z-125z^4\\&=z(1-125z^3)\\&=z(1 - 5z)(1 + 5z + 25z^2)$\end{array*}%${}$
\item \begin{array*}$8m^{6} + n^{9}\\&=(2m^2)^3 + (n^3)^3\\&=(2m^2 + n^3)(4m^4 - 2m^2n^3 + n^6)$\end{array*}%$$
\item $p^{15} - \frac{1}{8}y^{12}\\
=\left(p^5\right) - \left(\frac{1}{2} y^4\right)^3\\ = (p^5 - \frac{1}{2} y^4)(p^10 + \frac{1}{2}p^{5}y^{4} + \frac{1}{4}y^8$
Or: $p^{15} - \frac{1}{8}y^{12}\\
=\frac{1}{8}\left(8p^15 - y^12\right) - \frac{1}{8}\left(2p^5 - y^4)(4p^10 + 2p^{5}y^{4} + y^8$
\item $1- (x-y)^3\\=[1-(x-y)][1+(1)(x-y)+(x-y)^2]\\=[1-x+y][1+x-y+x^2-2xy+y^2]$
\end{enumerate}
\end{multicols}}
\end{solutions}


% Ex 1-10 questions
\begin{exercises}{}
{
Vereenvoudig (aanvaar alle noemers is nie nul nie):
\begin{multicols}{2}
\begin{enumerate}[itemsep=5pt, label=\textbf{\arabic*}. ] 
\item$\dfrac{3a}{15}$
\item $\dfrac{2a+10}{4}$
\item $\dfrac{5a+20}{a+4}$
\item $\dfrac{{a}^{2}-4a}{a-4}$
\item $\dfrac{3{a}^{2}-9a}{2a-6}$
\item $\dfrac{9a+27}{9a+18}$
\item $\dfrac{6ab+2a}{2b}$
\item $\dfrac{16{x}^{2}y-8xy}{12x-6}$
\item $\dfrac{4xyp-8xp}{12xy}$
\item $\dfrac{3a+9}{14}÷\dfrac{7a+21}{a+3}$
\item $\dfrac{{a}^{2}-5a}{2a+10} \times \dfrac{4a}{3a+15}$
\item $\dfrac{3xp+4p}{8p}÷\dfrac{12{p}^{2}}{3x+4}$
\item $\dfrac{24a-8}{12}÷\dfrac{9a-3}{6}$
\item $\dfrac{{a}^{2}+2a}{5}÷\dfrac{2a+4}{20}$
\item $\dfrac{{p}^{2}+pq}{7p} \times \dfrac{21q}{8p+8q}$
\item $\dfrac{5ab-15b}{4a-12}÷\dfrac{6{b}^{2}}{a+b}$
\item $\dfrac{{f}^{2}a-f{a}^{2}}{f-a}$
\item $\dfrac{2}{xy} + \dfrac{4}{xz}+\dfrac{3}{yz}$
\item $\dfrac{5}{t-2} - \dfrac{1}{t-3}$
\item $\dfrac{k+2}{k^{2} +2} - \dfrac{1}{k+2}$
\item $\dfrac{t+2}{3q} + \dfrac{t+1}{2q}$
\item $\dfrac{3}{p^{2}-4}+\dfrac{2}{(p-2)^{2}}$
\item $\dfrac{x}{x+y}+\dfrac{x^{2}}{y^{2} - x^{2}}$
\item $\dfrac{1}{m+n} + \dfrac{3mn}{m^{3} + n^{3}}$
\item $\dfrac{h}{h^{3}-f^{3}} - \dfrac{1}{h^{2} + hf + f^{2}}$
\item $\dfrac{{x}^{2}-1}{3}\times\dfrac{1}{x-1}-\dfrac{1}{2}$
\item  $\dfrac{x^2-2x+1}{(x-1)^3} - \dfrac{x^2+x+1}{x^3-1}$
\item $\dfrac{1}{(x-1)^2} - \dfrac{2x}{x^3-1}$
\item $\dfrac{p^3 + q^3}{p^2} \times \dfrac{3p-3q}{p^2-q^2}$
\item $\dfrac{1}{a^2-4ab+4b^2} + \dfrac{a^2+2ab+b^2}{a^3-8b^3} - \dfrac{1}{a^2-4b^2}$
\end{enumerate}
\end{multicols}

}
\end{exercises}

% Ex 1-10 solutions
 \begin{solutions}{}{
\begin{multicols}{2}
\begin{enumerate}[itemsep=5pt, label=\textbf{\arabic*}. ] 
\item $\dfrac{3a}{15}=\dfrac{a}{5}$%$$
\item \begin{array*}$\dfrac{2a+10}{4}\\[4pt]&=\dfrac{2(+5)}{4}\\[4pt]&=\dfrac{a + 5}{2}$\end{array*}%$$
\item $\dfrac{5a+20}{a+4}=5$%$$
\item $\dfrac{{a}^{2}-4a}{a-4}=a$%$\$
\item  \begin{array*}$\dfrac{3{a}^{2}-9a}{2a-6}\\[4pt]&=\dfrac{3a(a-3)}{2(a-3)}\\[4pt]&=\dfrac{3a}{2}$\end{array*}%$$
\item  \begin{array*}$\dfrac{9a+27}{9a+18}\\[4pt]&=\dfrac{9(a+3)}{9(a+2)}\\[4pt]&=\dfrac{a + 3}{a + 2}$\end{array*}%$$
\item  \begin{array*}$\dfrac{6ab+2a}{2b}\\[4pt]&=\dfrac{2a(3b+1)}{2b}\\[4pt]&=\dfrac{a(3b + 1)}{b}$\end{array*}%$$
\item $\dfrac{16{x}^{2}y-8xy}{12x-6}=\dfrac{4xy}{3}$%$$
\item $\dfrac{4xyp-8xp}{12xy}=\dfrac{p(y - 2)}{3y}$%$$
\item $\dfrac{3a+9}{14}÷\dfrac{7a+21}{a+3}=\dfrac{3(a + 3)}{98}$%$$
\item $\dfrac{{a}^{2}-5a}{2a+10} \times \dfrac{4a}{3a+15}=\dfrac{4a^2(a - 5)}{6(a + 5)^2}$%$$
\item $\dfrac{3xp+4p}{8p}÷\dfrac{12{p}^{2}}{3x+4}=\dfrac{(3x + 4)^2}{96p^2}$%$$
\item $\dfrac{24a-8}{12}÷\dfrac{9a-3}{6}=\dfrac{4}{3}$%$$
\item $\dfrac{{a}^{2}+2a}{5}÷\dfrac{2a+4}{20}=2a$%$$
\item $\dfrac{{p}^{2}+pq}{7p} \times \dfrac{21q}{8p+8q}=\dfrac{3q}{8}$%$$
\item $\dfrac{5ab-15b}{4a-12}÷\dfrac{6{b}^{2}}{a+b}=\dfrac{5(a + b)}{24b}$%$$
\item  \begin{array*}$\dfrac{{f}^{2}a-f{a}^{2}}{f-a}\\[4pt]&=\dfrac{af(f-a)}{f-a}\\[4pt]&=af$\end{array*}%$$
\item $\dfrac{2}{xy} + \dfrac{4}{xz}+\dfrac{3}{yz}\dfrac{2z}{xyz} + \dfrac{4y}{xyz} + \dfrac{3x}{xyz}=\dfrac{2z + 4y + 3x}{xyz}$%$$
\item  \begin{array*}$\dfrac{5}{t-2} - \dfrac{1}{t-3}\\[4pt]&=\dfrac{5(t-3)}{(t-2)(t-3)}-\dfrac{t-2}{(t-2)(t-3)}\\[4pt]&= \dfrac{5(t-2)(t-3)}{(t-2)(t-3)}\\[4pt]&=5$\end{array*}%$$
\item \begin{array*} $\dfrac{k+2}{k^{2} +2} - \dfrac{1}{k+2}\\[4pt]&=\dfrac{(k+2)(k+2)-(k^2+2)}{(k^2+2)(k+2)}\\[4pt]& = \dfrac{k^2+4k+4-k^2-2}{k^3+2k^2+2k+4}\\[4pt]&=\dfrac{2(k - 1)}{(k^2 + 2)(k + 2)}$\end{array*}%$\
\item  \begin{array*}$\dfrac{t+2}{3q} + \dfrac{t+1}{2q}\\[4pt]&=\dfrac{2(t+2)+3(t+1)}{6q}\\[4pt]&=\dfrac{5t + 7}{6q}$\end{array*}%$$
\item  \begin{array*}$\dfrac{3}{p^{2}-4}+\dfrac{2}{(p-2)^{2}}\\[4pt]&=\dfrac{3(p-2)+2(p+2)}{(p-2)(p+2)(p-2)} \\[4pt]&=\dfrac{(p-2)+2(p+2)}{(p+2)(p^2 -4)}\\[4pt]&=\dfrac{(5p - 2)}{(p - 2)^2(p+ 2)}$\end{array*}%$$
\item  \begin{array*}$\dfrac{x}{x+y}+\dfrac{x^{2}}{y^{2} - x^{2}}\\[4pt]&=\dfrac{x(x-y)-x^2}{(x+y)(x-y)}\\[4pt]&=\dfrac{-xy}{x^2 - y^2}$\end{array*}%$$
\item  \begin{array*}$\dfrac{1}{m+n} + \dfrac{3mn}{m^{3} + n^{3}}\\[4pt]&=\dfrac{1}{m+n} + \dfrac{3mn}{(m+n)(m^2-mn+n^2)} \\[4pt]&= \dfrac{m^2-mn+n^2+3mn}{m^3-m^3}\\[4pt]&=\dfrac{m^2+2mn+n^2}{m^3-n^3}\\[4pt]&=\dfrac{m+1}{m^2 - mn + n^2}$\end{array*}%$$
\item  \begin{array*}$\dfrac{h}{h^{3}-f^{3}} - \dfrac{1}{h^{2} + hf + f^{2}}\\[4pt]&=\dfrac{h}{(h-f)(h^2+hf+f^2)}-\dfrac{1}{h^2+hf+f^2}\\[4pt]&=\dfrac{h-(h-f)}{h^3-f^3}\\[4pt]&=\dfrac{f}{h^3 - f^3}$%\end{array*}$$
\item $\dfrac{{x}^{2}-1}{3}\times\dfrac{1}{x-1}-\dfrac{1}{2}\\[4pt]&=\dfrac{2x - 1}{6}$%$$
\item  $\dfrac{x^2-2x+1}{(x-1)^3} - \dfrac{x^2+x+1}{x^3-1} \\[4pt]= \dfrac{(x-1)^2}{(x-1)^3} - \dfrac{x^2 + x+1}{x^3-1}\\[4pt]=\dfrac{1}{(x-1)} - \dfrac{x^2 + x + 1}{(x-1)(x^2+x+1)}\\[4pt]= \dfrac{1}{(x-1)} - \dfrac{1}{(x-1)}\\[4pt]=0$
\end{enumerate}
\end{multicols}
\begin{enumerate}[itemsep=5pt, label=\textbf{\arabic*}. ] 
\setcounter{enumi}{27}
\item $\dfrac{1}{(x-1)^2} - \dfrac{2x}{x^3-1}\\[4pt]=\dfrac{1}{(x-1)^2} - \dfrac{2x}{(x-1)(x^2+x+1)}\\[4pt]=\dfrac{x^2+x+1-2x(x-1)}{(x-1)^2(x^2+x+1)}\\[4pt]=\dfrac{x^2+x+1-2x^2-2x}{(x-1)^2(x^2+x+1)}\\[4pt]=\dfrac{-x^2+3x+1}{(x-1)^2(x^2+x+1)}\\[4pt]=-\dfrac{x^2-3x-1}{(x-1)^2(x^2+x+1)}$

\item $\dfrac{p^3 + q^3}{p^2} \times \dfrac{3p-3q}{p^2-q^2}\\[4pt]=\dfrac{(p+q)(p^2-pq+q^2)}{p^2} \times \dfrac{3(p-q)}{(p-q)(p+q)}\\[4pt]=\dfrac{3(p^2-pq+q^2)}{p^2}$
\item $\dfrac{1}{a^2-4ab+4b^2} + \dfrac{a^2+2ab+b^2}{a^3-8b^3} - \dfrac{1}{a^2-4b^2}\\[4pt]=\dfrac{1}{(a-2b)(a-2b)} + \dfrac{a^2+2ab+4b^2}{(a-2b)(a^2+2ab+4b^2} - \dfrac{1}{(a-2b)(a+2b)}\\[4pt]=\dfrac{(a+2b)+(a-2b)(a+2b)-(a-2b)}{(a-2b^2)(a+2b)}\\[4pt]=\dfrac{a+2b+a^2-4b^2 -a+2b}{(a-2b^2)(a+2b)}=\dfrac{a^2+4b-4b^2}{(a-2b^2)(a+2b)}$
\end{enumerate}

}
\end{solutions}

% EOC exercise questions
\begin{eocexercises}{}
\begin{enumerate}[itemsep=5pt, label=\textbf{\arabic*}. ] 
\item Indien $a$ ’n heelgetal is, $b$ ’n heelgetal is en $c$ irrasionaal is, watter van die volgende is rasionaal?
\begin{multicols}{2}
    \begin{enumerate}[itemsep=4pt, label=\textbf{(\alph*)} ] 
    \item $\dfrac{-b}{a}$
    \item $c \div c$
    \item $\dfrac{a}{c}$
    \item $\dfrac{1}{c}$
    \end{enumerate}
\end{multicols}
\item Skryf elkeen van die volgende as ’n onegte breuk:
\begin{multicols}{2}
    \begin{enumerate}[itemsep=0pt, label=\textbf{(\alph*)} ] 
    \item $0,12$
    \item $0,006$
    \item $1,59$
    \item $12,27\dot{7}$
    \end{enumerate}
\end{multicols}
\item Wys dat die desimaal $3,21\dot{1}\dot{8}$ ’n rasionale getal is.
\item Druk  $0,7\dot{8}$ uit as ’n breuk $\dfrac{a}{b}$ waar $a,b\in \mathbb{Z}$ (wys alle stappe).
\item Skryf die volgende rasionalle getalle tot $2$ desimale plekke:
\begin{multicols}{2}
    \begin{enumerate}[itemsep=5pt, label=\textbf{(\alph*)} ]  
    \item $\dfrac{1}{2}$
    \item $1$
    \item $0,11111\overline{1}$
    \item $0,99999\overline{1}$
    \end{enumerate}
\end{multicols}
\item Rond die volgende getalle af tot $3$ desimale plekke:
\begin{multicols}{2}
    \begin{enumerate}[itemsep=2pt, label=\textbf{(\alph*)} ] 
    \item $3,141592654\ldots$
    \item $1,618033989\ldots$
    \item $1,41421356\ldots$
    \item $2,71828182845904523536\ldots$
    \end{enumerate}
\end{multicols}

\item Gebruik jou sakrekenaar om die volgende irrasionale getalle tot $4$ desimale plekke te skryf:
\begin{multicols}{2}
    \begin{enumerate}[itemsep=0pt, label=\textbf{(\alph*)} ] 
    \item $\sqrt{2}$
    \item $\sqrt{3}$
    \item $\sqrt{5}$
    \item $\sqrt{6}$
    \end{enumerate}
\end{multicols}
\item Gebruik jou sakrekenaar (waar nodig) om die volgende getalle tot  $5$ desimale plekke te skryf en dui aan of die getal rasionaal of irrasionaal is:
\begin{multicols}{2}
    \begin{enumerate}[itemsep=0pt, label=\textbf{(\alph*)} ] 
    \item $\sqrt{8}$
    \item $\sqrt{768}$
    \item $\sqrt{0,49}$
    \item $\sqrt{0,0016}$
    \item $\sqrt{0,25}$
    \item $\sqrt{36}$
    \item $\sqrt{1960}$
    \item $\sqrt{0,0036}$
    \item $-8\sqrt{0,04}$
    \item $5\sqrt{80}$
    \end{enumerate}
\end{multicols}
\item Skryf die volgende irrasionale getalle tot $3$ desimale plekke, en skryf die afgeronde getal dan as ’n rasionale getal om ’n benadering van die irrasionale getal te verkry.
\begin{multicols}{2}
\begin{enumerate}[itemsep=0pt, label=\textbf{(\alph*)} ] 
    \item $3,141592654\ldots$
    \item $1,618033989\ldots$
    \item $1,41421356\ldots$
    \item $2,71828182845904523536\ldots$
    \end{enumerate}
\end{multicols}
\item Bepaal sonder 'n sakrekenaar tussen watter twee opeenvolgende heelgetalle die volgende irasionale getalle l\^{e}:
\begin{multicols}{2}
    \begin{enumerate}[itemsep=0pt, label=\textbf{(\alph*)} ] 
    \item $\sqrt{5}$ 
    \item $\sqrt{10}$ 
    \item $\sqrt{20}$ 
    \item $\sqrt{30}$ 
    \item $\sqrt[3]{5}$ 
    \item $\sqrt[3]{10}$ 
    \item $\sqrt[3]{20}$ 
    \item $\sqrt[3]{30}$ 
    \end{enumerate}
\end{multicols}
\item  Vind twee opeenvolgende heelgetalle wat weerskante van $\sqrt{7}$ lê op die getallelyn.         
\item  Vind twee opeenvolgende heelgetalle wat weerskante van $\sqrt{15}$ lê op die getallelyn.          
\item Faktoriseer:
\begin{multicols}{2}
\begin{enumerate}[itemsep=2pt, label=\textbf{(\alph*)} ] 
\item ${a}^{2}-9$
\item ${m}^{2}-36$
\item $9{b}^{2}-81$
\item $16{b}^{6}-25{a}^{2}$
\item ${m}^{2}-\frac{1}{9}$
\item $5-5{a}^{2}{b}^{6}$
\item $16b{a}^{4}-81b$
\item ${a}^{2}-10a+25$
\item $16{b}^{2}+56b+49$
\item $2{a}^{2}-12ab+18{b}^{2}$
\item $-4{b}^{2}-144{b}^{8}+48{b}^{5}$
\item $(16-{x}^{4})$
\item ${7x}^{2}-14x+7xy-14y$
\item ${y}^{2}-7y-30$
\item $1-x-{x}^{2}+{x}^{3}$
\item $-3(1-{p}^{2})+p+1$
\item $x-x^{3} + y - y^{3}$
\item $x^{2} - 2x + 1 - y^{4}$
\item $4b(x^{3} - 1) + x(1-x^{3})$
\item $3p^{3} - \frac{1}{9}$
\item $8x^6-125y^9$
\item $(2+p)^3- 8(p+1)^3$
\end{enumerate}
\end{multicols}
\item Vereenvodig die volgende:
\begin{multicols}{2}
\begin{enumerate}[itemsep=4pt, label=\textbf{(\alph*)} ] 
\item ${(a-2)}^{2}-a(a+4)$
\item $(5a-4b)(25{a}^{2}+20ab+16{b}^{2})$
\item $(2m-3)(4{m}^{2}+9)(2m+3)$
\item $(a+2b-c)(a+2b+c)$
\item $\dfrac{{p}^{2}-{q}^{2}}{p}÷\dfrac{p+q}{{p}^{2}-pq}$
\item $\dfrac{2}{x}+\dfrac{x}{2}-\dfrac{2x}{3}$
\item $\dfrac{1}{a+7}-\dfrac{a+7}{a^{2}-49}$
\item $\dfrac{x+2}{2x^{3}} + 16$
\item $\dfrac{1-2a}{4a^{2} -1} - \dfrac{a-1}{2a^{2}-3a+1} - \dfrac{1}{1-a}$
\item $\dfrac{x^{2} + 2x}{x^{2}+ x + 6} \times \dfrac{x^{2} + 2x + 1}{x^{2} + 3x +2}$
\end{enumerate}
\end{multicols}
\item Wys dat ${(2x-1)}^{2}-{(x-3)}^{2}$ vereenvoudig kan word tot  $(x+2)(3x-4)$.
\item Bepaal wat moet by ${x}^{2}-x+4$ getel word sodat dit gelyk is aan ${(x+2)}^{2}$ ?
\item Evalueer $\dfrac{x^{3}+1}{x^{2}-x+1}$ as $x=7,85$ sonder om 'n sakrekenaar te gebruik. Toon jou bewerkings.
% Still in English
\item With what expression must $(a-2b)$ be multiplied to get a product of $a^3-8b^3$?
\item With what expression must $27x^3+1$ be divided to get a quotient of $3x+1$?
\end{enumerate}

\end{eocexercises}

% EOC solutions
 \begin{eocsolutions}{}{
\begin{enumerate}[itemsep=6pt, label=\textbf{\arabic*}. ] 
\item% If $a$ is an integer, $b$ is an integer and $c$ is irrational, which of the following are rational numbers?
\begin{multicols}{2}
    \begin{enumerate}[noitemsep, label=\textbf{(\alph*)} ] 
    \item Rasionaal%$\dfrac{-b}{a}$
    \item Irrasional%$c \div c$
    \item Irrasional%$\dfrac{a}{c}$
    \item Irrasional%$\dfrac{1}{c}$
    \end{enumerate}
\end{multicols}
%Question 2
\item %% Write each decimal as a simple fraction:
\begin{multicols}{2}
    \begin{enumerate}[itemsep=4pt, label=\textbf{(\alph*)} ] 
    \item $0,12\\&= \frac{1}{10}+\frac{2}{100} \\&= \frac{12}{100} \\&= \frac{6}{10}\\&=\frac{3}{5}$%$$
    \item $0,006\\&=\frac{6}{1000}\\&=\frac{3}{500}$%$$
    \item \begin{array*}$1,59\\&=1+\frac{5}{10} + \frac{9}{100}\\&=1\frac{59}{100}$\end{array*}%$$
    \item$x=12,2\dot{7}$\\ $10x=122,\dot{7}$\\ $100x=1~227,\dot{7}$ \\$\therefore 100x-10x=90x=1105$ \\$\therefore x=\frac{1105}{90}=\frac{221}{18}$%$$
    \end{enumerate}
\end{multicols}
%Question 3
 \item $x=3,21\overline{18}\\10~000x=32~118,\overline{18}$\\ $\therefore 10~000x-x=9~999x=32115$ \\$\therefore x=\frac{32~115}{9~999}$\\ Hierdie is 'n rasionale getal omdat beide die teller en die nommer heeltallige is. %This is a rational number because both the numerator and denominator are integers.

  %%Show that the decimal $3,21\dot{1}\dot{8}$ is a rational number.

% Question 4
% \setcounter{enumi}{3}
\item %%Express $0,7\dot{8}$ as a fraction $\dfrac{a}{b}$ where $a,b\in \mathbb{Z}$ (show all working).
$x=0,\overline{78}$\\
$100x=78,\overline{78}$\\ 
$\therefore 100x-x=99x=\frac{78}{99}$
%Question 5
\item %%Write the following rational numbers to $2$ decimal places:
    \begin{enumerate}[noitemsep, label=\textbf{(\alph*)} ]  
    \item  Om tot by twee desimale plekke af te rond moet ons na desimaal omskep:$\frac{1}{2}=0,5$%$\dfrac{1}{2}$
    \item Om tot by twee desimale plekke af te rond, voeg net 'n komma en twee $0$'s: $1,00$%$1$
    \item  Ons merk waar die afsny punt is, besluit of dit bo of onder toe afrond moet wees ens skryf die antwoord.  In hierdie voorbeeld is daar 'n $1$ na die afsnypunt so ons hoef nie na bo af te rond nie. Die finale antwoord is: $0,11111\overline{1} \sim 0,11$%$0,11111\overline{1}$
    \item Herhaal die stappe in c) maar hierdie keer rond ons na bo af: $0,99999\overline{1} \sim 1,00$%$0,99999\overline{1}$
    \end{enumerate}

%Question 6
\item %%Round off the following irrational numbers to $3$ decimal places:
Ons merk waar die afsny punt is, besluit of dit bo of onder toe afrond moet wees ens skryf die antwoord.
    \begin{enumerate}[noitemsep, label=\textbf{(\alph*)} ] 
\item $3,142$ (rond na boontoe omdat daar 'n $5$ na die afsnypunt is) 
\item $1,618$ (geen afronding want daar 'n $0$ na die afsnypunt is) 
\item $1,414$ (geen afronding want daar 'n $2$ na die afsnypunt is)
\item $2,718$ (rond na boontoe omdat daar 'n $2$ na die afsnypunt is) 
    \end{enumerate}

%Question 7
\item %%Use your calculator and write the following irrational numbers to $4$ decimal places:
\begin{multicols}{2}
    \begin{enumerate}[noitemsep, label=\textbf{(\alph*)} ] 
    \item $1,414$ %$\sqrt{2}$
    \item $1,732$%$\sqrt{3}$
    \item $2,236$%%$\sqrt{5}$
    \item $2,449$%$\sqrt{6}$
    \end{enumerate}
\end{multicols}
% 
\item %Use your calculator (where necessary) and write the following numbers to $5$ decimal places. State whether the numbers are irrational or rational.
\begin{multicols}{2}
    \begin{enumerate}[noitemsep, label=\textbf{(\alph*)} ] 
    \item $2,82843$%$\sqrt{8}$
    \item $27,71281$ - irrasional %$\sqrt{768}$
    \item $10,00000$ - rasional 
    \item $0,70000$ - rasional  %$\sqrt{0,49}$
    \item $0,04000$ - rasional %%$\sqrt{0,0016}$
    \item $0,500000$ - rasional  %$\sqrt{0,25}$
    \item $6,00000$ - rasional  %$\sqrt{36}$
    \item $44,27189$ - irrasional  %$\sqrt{1960}$
    \item $0,06000$ - rasional  %$\sqrt{0,0036}$
    \item $-8(0,2) = -4,00000$ - rasional  %$-8\sqrt{0,04}$
    \item $44,72136$ - irrasional %$5\sqrt{80}$
    \end{enumerate}
\end{multicols}
% \setcounter{enumi}{8}
\item %Write the following irrational numbers to $3$ decimal places and then write each one as a rational number to get an approximation to the irrational number.
\begin{multicols}{2}
\begin{enumerate}[noitemsep, label=\textbf{(\alph*)} ] 
\item $3 \frac{142}{1~000} =\frac{1~571}{500}$
\item $1\frac{618}{1000}=\frac{809}{500}$
\item $1\frac{414}{1000}\frac{707}{500}$
\item $2\frac{718}{1000}=\frac{1~359}{500}$
    \end{enumerate}
\end{multicols}

\item %Determine between which two consecutive integers the following irrational numbers lie, without using a calculator:
\begin{multicols}{2}
    \begin{enumerate}[noitemsep, label=\textbf{(\alph*)} ] 
\item $2$ en $3$
\item $3$ en $4$
\item $4$ en $5$
\item $5$ en $6$
\item $1$ en $2$
\item $2$ en $3$
\item $2$ en $3$
\item $3$ en $4$
    \end{enumerate}
\end{multicols}

\item $2$ en $3$% Find two consecutive integers such that $\sqrt{7}$ lies between them.          
\item $3$ en $4$% Find two consecutive integers such that $\sqrt{15}$ lies between them.          

\item %%Factorise:
\begin{multicols}{2}
\begin{enumerate}[itemsep=5pt, label=\textbf{(\alph*)} ] 

\item ${a}^{2}-9=(a - 3)(a + 3)$%%$$
\item ${m}^{2}-36=(m + 6)(m - 6)$%%$$
\item $9{b}^{2}-81=(3b - 9)(3b + 9)$%%$$
\item $16{b}^{6}-25{a}^{2}=(4b + 5a)(4b - 5a)$%%$$
\item ${m}^{2}-\frac{1}{9}=(m +\frac{1}{3})(m -\frac{1}{3})$%%$$
\item $5-5{a}^{2}{b}^{6}\\=5(1-a^2b^6)\\=5(1 - ab^3)(1 + ab^3)$%%$$
\item $16b{a}^{4}-81b\\=b(4a^2+0)(4a^2+9)\\=b(4a^2 - 9)(2a + 3)(2a - 3)$%%$$
\item ${a}^{2}-10a+25=(a - 5)(a - 5)$%%$$
\item $16{b}^{2}+56b+49=(4b + 7)(4b + 7)$%%$1$
\item $2{a}^{2}-12ab+18{b}^{2}=(2a - 6b)(a - 3b)$%%$$
\item $-4{b}^{2}-144{b}^{8}+48{b}^{5}\\=-4b^2(6b^3 - 1)(6b^3 - 1)$%%$$
\item $(16-{x}^{4})\\=(4-x^2)(4+x^2)\\=(2 - x)(2 + x)(4 + x^2)$%%$$
\item ${7x}^{2}-14x+7xy-14y\\=7x(x-2)+7y(x-2)\\=(x-2)(7x+7y)\\=7(x - 2)(x + y)$%%$$
\item ${y}^{2}-7y-30=(y - 10)(y + 3)$%%$$
\item $1-x-{x}^{2}+{x}^{3}\\=(1-x)-x^2(1-x)\\=(1-x)(1-x^2)=(1 - x)^{2}(1 +x)$%%$$
\item $-3(1-{p}^{2})+p+1\\ =-3(1-p)(1+p)+(p+1)\\=(p+1)[-3(1-p)+1]\\=(p - 1)(-3p-2)$%%$$
\item $x-x^3+y-y^3\\=x(1-x^2)+y(1-y^2)\\=x(1-x)(1+x)+y(1-y)(1+y)\\=(x+y)(1-x^2+xy-y^2)$
\item $x^2-2x+1-y^4\\=x(x-2)+(1-y^2)(1+y^2)\\=x(x-2)+(1+y)(1-y)(1+y^2)\\=(x-1-y^2)(x-1+y^2)$
\item $4b(x^3)+x(1-x^3)\\=(x^3-1)(4b-x)\\=(x-1)(x^2+x+1)(4b-x)\\=4b(x^3-1)(1-x)$
\item $3p^3-\frac{1}{9}=3(p-\frac{1}{3})(p^2+\frac{p}{3}+\frac{1}{9})$
\item $8x^6-125y^9=(2x^2-5y^3)(4x^4+10x^2y^3+25y^6)$
\item $(2+p)^3- 8(p+1)^3\\=[(p+2)-2(p+1)][(p+2)^2 +2(p+2)(p+1) +4(p+1)^2]\\=[p+2-2p-2][p^2+4p+4+2p^2+6p+4+4p^2+8p+4]\\=(-p)(12+18p+7p^2)$
\end{enumerate}
\end{multicols}

\item %Simplify the following:
\begin{multicols}{2}
\begin{enumerate}[itemsep=6pt, label=\textbf{(\alph*)} ] 

\item$(a-2)}^{2}-a(a+4)\\=a^2-4a+4-a^2-4a\\=-8a + 4$%${$
\item $(5a-4b)(25{a}^{2}+20ab+16{b}^{2})\\=125a^3 - 64b^3$ %$$
\item $(2m-3)(4{m}^{2}+9)(2m+3)\\=(4m^2-9)(4m^2+9)\\=16m^4 - 81$%$$
\item $(a+2b-c)(a+2b+c)\\=(a+2b)^2-c^2\\=a^2 + 4ab + 4b^2 - c^2$%$$
\item $\dfrac{{p}^{2}-{q}^{2}}{p}÷\dfrac{p+q}{{p}^{2}-pq}\\[4pt]=\dfrac{(p-q)(p+q)}{p} \times \dfrac{p(p-q)}{p+q}\\[4pt]= (p-q)^2=p^2 - 2pq +q^2$%$$
\item $\dfrac{2}{x}+\dfrac{x}{2}-\dfrac{2x}{3}\\[4pt]=\dfrac{12+3x^2-4x^2}{6x}\\[4pt]=\dfrac{12 - x^2}{6x}$%$$
\item $\dfrac{1}{a+7}-\dfrac{a+7}{a^2-49}\\[4pt]=\dfrac{1}{a+7}-\dfrac{a+7}{(a+7)(a-7)}\\[4pt]= \dfrac{-14}{(a+7)(a-7)}$
\item $\dfrac{x+2}{2x^3}+16\\[4pt]=\dfrac{(x+2)+16(2x^3)}{2x^3}\\[4pt]=\dfrac{32x^3+x+2}{2x^3}$
\item $\dfrac{1-2a}{4a^2-1}-\dfrac{a-1}{2a^2-3a+1}-\dfrac{1}{1-a}\\[4pt]=\dfrac{1-2a}{(2a-1)(2a+1)} - \dfrac{a-1}{(2a-1)(2a+1)} + \dfrac{1}{a-1}\\[4pt]= -\dfrac{(2a-1)}{(2a-1)(2a+1)}-\dfrac{1}{2a-1}+\dfrac{1}{a-1}\\[4pt]=\dfrac{4a-1}{(2a+1)(2a-1)(a-1)}$
\item $\dfrac{x^2+2x}{x^2+x+6} \times \dfrac{x^2 + 2x+1}{x^2+3x+2}\\[4pt]=\dfrac{x(x+2)}{x^2+x+6} \times \dfrac{(x+1)(x+1)}{(x+2)(x+1)}\\[4pt]=\dfrac{x(x+1)}{x^2+x+6}$
\end{enumerate}
\end{multicols}

\item $(2x-1)(2x-1)-(x-3)(x-3)\\= 4x^2-2x-2x+1-(x^2-3x-3x-9)\\=(4-1)x^2+(-4+6)x+(1-9)\\=3x^2 +2x-8\\=(3x - 4)(x + 2)$%Show that ${(2x-1)}^{2}-{(x-3)}^{2}$ can be simplified to $(x+2)(3x-4)$
\item Veronderstal $A$ moet by die uitrdrukking getel word om die verlangde resultaat te kry.\\$\therefore (x^2-x+4) +A = (x+2)^2\\ 
\therefore A = (x+2)(x+2)-(x^2-x+4)\\=x^2+2x+2x+4-x^2+x-4\\=(1-1)x^2+(2+2+1)x +(4-4) \\=5x$\\
Dus $5x$ moet getel wees. %What must be added to ${x}^{2}-x+4$ to makeit equal to ${(x+2)}^{2}$ ?
\item Eerstens,vereenvoudig die uitdrukking: $\dfrac{(x+1)(x^2-x+1)}{x^2-x+1} = x+1$\\[6pt]
Nou vervang die waarde van $x$ in: $7,85+1=8,85$ %Evaluate $\dfrac{x^{3}+1}{x^{2}-x+1}$ if $x=7,85$ without using a calculator. Show your working.
\item $(a-2b)(a^2+2ab+4b^2) = a^3-8b^3$ \\so, die uitdrukking is $a^2+2ab+4b^2$.
\item $27x^3+1=(3x+1)(9x^2-3x+1)\\[4pt]
\dfrac{(3x+1)(9x^2-3x+1)}{9x^2-3x+1} = 3x+1\\[4pt]
$so, die uitdrukking is $9x^2-3x+1$.
\end{enumerate}

\end{enumerate}
}
\end{eocsolutions}


