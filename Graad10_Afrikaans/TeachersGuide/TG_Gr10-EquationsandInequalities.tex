\chapter{Equations and inequalities}

\begin{exercises}{}
{
% English in sentence below
Oplos van lineêre vergelykings (Aanvaar alle noemers is non-zero): \\
\begin{multicols}{2}
\begin{enumerate}[itemsep=5pt, label=\textbf{\arabic*}. ] 
\item   $2y-3=7$
\item   $-3y=0$        
\item   $16y+4=-10$        
\item   $12y+0=144$
\item   $7+5y=62$       
\item  $55=5x+\dfrac{3}{4}$ 
\item   $5x=2x+45$        
\item  $23x-12=6+3x$
\item   $12-6x+34x=2x-24-64$
\item   $6x+3x=4-5(2x-3)$
\item   $18-2p=p+9$   
\item   $\dfrac{4}{p}=\dfrac{16}{24}$
\item   $-(-16-p)=13p-1$
\item   $3f-10=10$
\item   $3f+16=4f-10$
\item   $10f+5=-2f-3f+80$
\item   $8(f-4)=5(f-4)$
\item  $6=6(f+7)+5f$      
\item $(a-1)^{2} - 2a = (a+3)(a-2) - 3$
\item $-7x = x+8(1-x)$ 
\item $5-\dfrac{7}{b} = \dfrac{2(b+4)}{b}$
\item $\dfrac{x+2}{4} - \dfrac{x-6}{3} = \dfrac{1}{2}$
\item $ 3 - \dfrac{y-2}{4} = 4$
\item $ \dfrac{a+1}{a+2} = \dfrac{a-3}{a+1}$
\item $(x-3)(x+2)=x(x-4)$
\item $1,5x+3,125=1,25x$
\item $\frac{1}{3}P + \frac{1}{2}P - 10 = 0$
\item $1 \frac{1}{4} (x-1)-1\frac{1}{2}(3x+2)=0$
\item $\frac{5}{2a}+\frac{1}{6a}-\frac{3}{a}=2$
\item $\frac{3}{2x^2}+\frac{4}{3x}-\frac{5}{6x}=0$  
\end{enumerate}
\end{multicols}

}
\end{exercises}


 \begin{solutions}{}{
\begin{enumerate}[itemsep=3pt, label=\textbf{\arabic*}. ] 
\begin{multicols}{2}
\item   $2y-3=7\\2y=10\\y=5$
\item   $-3y=0\\y=\frac{0}{3}\\y=0$        
\item   $16y+4=-10\\16y=-14\\y=-\frac{14}{16}\\=\-frac{7}{8}$        
\item   $12y+0=144\\y=\frac{144}{12}\\y=12$
\item   $7+5y=62\\5y=62-7\\5y=11\\y=11$       
\item  $55=5x+\frac{3}{4}\\220=20x+3\\20x=117\\x=\frac{117}{20}$ 
\item   $5x=2x+45\\3x=45\\x=15$        
\item  $23x-12=6+3x\\20x=18\\x=\frac{18}{20}\\x=\frac{9}{10}$
\item   $12-6x+34x=2x-24-64\\-6x+34x-2x=-24-64-12\\ 26x=100\\x=\frac{100}{26}$
\item   $6x+3x=4-5(2x-3)\\6x+3x=4-10x+15\\19x=19\\x=1$
\item   $18-2p=p+9\\9=3p\\p=3$   
\item   $\frac{4}{p}=\frac{16}{24}\\96=16p\\p=6$
\item   $-(-16-p)=13p-1\\16+p=13p-1\\17=12p\\p=\frac{17}{12}$
\item   $3f-10=10\\3f=20\\f=\frac{20}{3}$
\item   $3f+16=4f-10\\-f=-10-16\\f=26$
\item   $10f+5=-2f-3f+80\\10f+2f+3f=75\\15f=75\\f=5$
\item   $8(f-4)=5(f-4)\\8f-32=5f-20\\3f=12\\f=4$
\item  $6=6(f+7)+5f\\6=6f+42+5f\\-36=11f\\f=-\frac{36}{11}$      
\item $(a-1)^{2} - 2a = (a+3)(a-2) - 3\\a^2-2a+1-2a=a^2-2a+3a-6-3\\-4a+1=a-9\\5a=10\\a=2$
\item $-7x = x+8(1-x)\\-7x=-7x+8\\0=8$\\ No solution 
\item $5-\dfrac{7}{b} = \dfrac{2(b+4)}{b}\\[4pt]5b-7=2(b+4)\\[4pt]5b-7=2b+8\\3b=15\\b=5$
\item $\dfrac{x+2}{4} - \dfrac{x-6}{3} = \dfrac{1}{2}\\[4pt]\dfrac{3(x=2)}{12}-\dfrac{4(x-6)}{12}=\frac{6}{12}\\[4pt]3(x+2)-4(x-6)=6\\[4pt]3x+6-4x+24=6\\-x=-24\\x=24$
\item $ 3 - \dfrac{y-2}{4} = 4\\[4pt]12-y-2=16\\-4=y+2\\y=-6$
\item $ \dfrac{a+1}{a+2} = \dfrac{a-3}{a+1}\\[4pt]\dfrac{a+1}{a+2} - \dfrac{a-3}{a+1}=0\\[4pt](a+1)^2 -(a-3)(a+2)=0\\a^2+2a+1-(a^2-a-6)=0\\a^2+2a+1-a^2+a+6=0\\3a+7=0\\a=-\frac{7}{3}$
\item $(x-3)(x+2)=x(x-4)\\
x^2 - x - 6 = x^2 -4x\\
-x+4x=6\\
3x=6\\
x=2$
\item $1,5x+3,125=1,25x\\
1,5x-1,25x=-3,125\\
0,25x=-3,125\\
x=-12,5$
\item $\frac{1}{3}P + \frac{1}{2}P - 10 = 0\\
2p+3p=10\\
5p=10\\
p=2$
\item $1 \frac{1}{4} (x-1)-1\frac{1}{2}(3x+2)=0\\
\frac{5}{4}x - \frac{5}{4}-\frac{9}{2}x - 3 = 0\\
\frac{5x-8x}{4}=\frac{5+12}{4}\\
-13x=17\\
x = -\frac{17}{13}$
\item $\frac{5}{2a}+\frac{1}{6a}-\frac{3}{a}=2\\
15+1-18=12a\\
-2=12a\\
a=-\frac{1}{6}$
\item $\frac{3}{2x^2}+\frac{4}{3x}-\frac{5}{6x}=0\\
9+8x-5x=0\\
9+3x\\
3x=-9\\
x=-3$  
\end{multicols}

\end{enumerate}
}
\end{solutions}


%ex 2-2
\begin{exercises}{ }
{
% Still in English
Solve the following equations:
\begin{multicols}{2}
\begin{enumerate}[itemsep=2pt, label=\textbf{\arabic*}. ] 
\item  $9x^{2}-6x-8=0$%$(3x+2)(3x-4)=0$
\item  $5x^{2}-21x-54=0$%$(5x-9)(x+6)=0$
\item  $4y^{2}-9=0$%$(2y+3)(2y-3)=0$ 
\item  $4x^{2}-16x-9=0$%$(2x+1)(2x-9)=0$    
\item  $4x^{2}-12x=-9$%$(4x)(x-3)=-9$    
\item  $20m+25{m}^{2}=0$
\item  $2{x}^{2}-5x-12=0$  
\item  $-75{x}^{2}+290x=240$
\item  $2x=\frac{1}{3}{x}^{2}-3x+14\frac{2}{3}$
\item  ${x}^{2}-4x=-4$      
\item  $-{x}^{2}+4x-6=4{x}^{2}-5x+3$       
\item  ${t}^{2}=3t$  
\item  ${x}^{2}-10x=-25$      
\item  ${x}^{2}=18$
\item  ${p}^{2}-6p=7$
\item  $4{x}^{2}-17x-77=0$
\item  $14{x}^{2}+5x=6$
\item  $2{x}^{2}-2x=12$
\item  $\frac{a+1}{3a-4}+\frac{9}{2a+5}+\frac{2a+3}{2a+5}=0$
\item  $\frac{3}{9a^2-3a+1}-\frac{3a+4}{27a^3+1}=\frac{1}{9a^2-1}$                
\end{enumerate}
\end{multicols}

}
\end{exercises}


 \begin{solutions}{}{
\begin{multicols}{2}
\begin{enumerate}[itemsep=5pt, label=\textbf{\arabic*}. ] 


\item  $(3x + 2)(3x − 4) = 0 \\ \therefore x=-\frac{2}{3}$ or $x=\frac{4}{3}$
\item $(5x-9)(x+6)=0\\ \therefore  x=\frac{9}{5}$ or $x=-6$% $$
\item $(2y+3)(2y-3)=0\\ \therefore  y=\frac{-3}{2}$ or $y=\frac{3}{2}$% $$ 
\item $x(2x+1)(2x-9)\\ \therefore x=-\frac{1}{2}$ or $x=\frac{9}{2}$% $=0$    
\item $(4x)(x-3)=-9\\4x^2-12x+9=0\\(2x-3)(2x-3)=0\\ \therefore x=\frac{3}{2}$% $$       
\item $20m+25{m}^{2}=0\\5m(4+5m)=0$\\ $\therefore  m=0$ or $m=\frac{-4}{5}$% $$
\item $2{x}^{2}-5x-12=0\\(2x+3)(x-4)=0\\ \therefore  x= 4$ or $\frac{-3}{2}$ % $$  
\item $-75{x}^{2}+290x=240\\-15x^2+58x-48=0\\(5x-6)(3x-8)\\ \therefore  x= \frac{6}{5}$ or $x= \frac{8}{3}$% $$
\item $2x=\frac{1}{3}{x}^{2}-3x+14\frac{2}{3}\\2x=\frac{1}{3}x^2-3x+\frac{44}{3}\\6x=x^2-9x+44\\x^2-15x+44=0\\(x-4)(x-11)=0\\ \therefore   x=4$ or $x=11$% $$
\item ${x}^{2}-4x=-4\\(x-2)(x-2)=0\\ \therefore  x=2$% $$      
\item $-{x}^{2}+4x-6=4{x}^{2}-14x+3\\5x^2-18x+9=0\\(5x-3(x-3)=0\\ \therefore x=\frac{3}{5}$ or $x=3$% $$       
\item ${t}^{2}=3t\\t(t-3)=0\\ \therefore t=0$ or $t=3$% $$  
\item ${x}^{2}-10x=-25\\ x^2-10x+25=0\\(x-5)(x-5)=0\\ \therefore  x=5$% ${x}^{2}-10x=-25$      
\item ${x}^{2}=18\\ \therefore x=\sqrt{18}$ or $x=-\sqrt{18}$% $$
\item ${p}^{2}-6p=7\\ p^2-6p-7=0\\(p+1)(p-7)=0\\ \therefore p=-1$ or $p=7$% $$
\item $4{x}^{2}-17x-77=0\\(4x+11)(x-7)=0\\ \therefore x=7$ or $x=-\frac{11}{4}$% $$
\item $14{x}^{2}+5x=6\\14x^2 +5x-6=0\\(7x-3)(2x+2)=0\\ \therefore  x=\frac{3}{7}$ or $x=-\frac{1}{2}$% $$
\item $2{x}^{2}-2x=12\\ x^2-x-6=0\\(x-3)(x+2)=0\\ \therefore x=-2$ or $x=3$% $$   
\end{enumerate}
\end{multicols}
\begin{enumerate}[itemsep=5pt, label=\textbf{\arabic*}. ] 
\setcounter{enumi}{18}
\item  $\dfrac{a+1}{3a-4}+\dfrac{9}{2a+5}+\dfrac{2a+3}{2a+5}=0\\[4pt]
\dfrac{(a+1)(2a+5)+9(3a-4)+(2a+3)(3a-4)}{(3a-4)(2a+5)}=0\\[4pt]
2a^2+7a+5+27a-36+6a^2+a-12=0\\
8a^2+35a-43=0\\
(8a+43)(a-1)=0\\
a=1 \text{ or } a=\frac{-43}{8}=-5\frac{3}{8}$
\item  $\dfrac{3}{9a^2-3a+1}-\dfrac{3a+4}{27a^3+1}=\dfrac{1}{9a^2-1}\\[4pt]
\dfrac{3}{9a^2-3a+1}-\dfrac{3a+4}{(3a+1)(9a^2-3a+1)}=\dfrac{1}{(3a-1)(3a+1)}\\[4pt]
\dfrac{3(9a^2-1)-(3a-1)(3a+4)}{(9a^2-3a+1)(3a+1)(3a-1)}=\dfrac{9a^2-3a+1}{(9a^2-3a+1)(3a+1)(3a-1)}\\[4pt]
27a^2-3-9a^2-9a+4=9a^2-3a+1\\
9a^2-6a=0\\
3a(3a-2)=0\\
a=0 \text{ or } a=\frac{2}{3}$  
\end{enumerate}
}
\end{solutions}



\begin{exercises}{}
{
% Still in English
\begin{enumerate}[noitemsep, label=\textbf{\arabic*}. ] 
\item Solve for $x$ and $y$: 
\begin{enumerate}[noitemsep, label=\textbf{(\alph*)} ] 
\item $3x-14y=0$ and $x-4y+1=0$
\item $x+y=8$ and $3x + 2y = 21$
\item $y=2x+1$ and $x + 2y + 3 = 0$
\item $\frac{a}{2}+b=4$ and $\frac{a}{4} -\frac{b}{4}=1$
\item $\frac{1}{x}+\frac{1}{y}=3$ and $\frac{1}{x}-\frac{1}{y}=11$
\end{enumerate}
\item Solve graphically and check your answer algebraically:
\begin{enumerate}[noitemsep, label=\textbf{(\alph*)} ] 
\item  $x+2y=1$ and $\frac{x}{3} + \frac{y}{2} = 1$
\item $5= x+y$ and $x = y-2$
\item $3x - 2y = 0$ and $x - 4y + 1 = 0$
\item $\frac{x}{4}=\frac{y}{2}-1$  and $\frac{y}{4}+\frac{x}{2}=1$
\item $2x+y=5$ and $3x-2y=4$
\end{enumerate}
\end{enumerate}

}
\end{exercises}


 \begin{solutions}{}{
\begin{enumerate}[itemsep=10pt, label=\textbf{\arabic*}. ] 
\item Solve algebraically: 
\begin{enumerate}[itemsep=5pt, label=\textbf{(\alph*)} ] 
\item $3x-14y=0\\\therefore 3x=14y\\ \therefore x=\frac{14}{3}y$\\
 Substitute value of $x$ into second equation: \\
$x-4y+1=0\\ \therefore \frac{14}{3}y-14y+1=0\\ 14y-12y+3=0\\2y=-3\\y=-\frac{3}{2}$\\ 
Substitute value of $y$ back into first equation:\\
$\therefore x=\frac{14 (-\frac{3}{2})}{3} = -7$\\
$y=-\frac{3}{2}$ and $x=-7$.

\item $x+y=8 \\ \therefore x=8-y$\\
Substitute value of $x$ into second equation:\\
$3x+2y=21 \\ \therefore 3(8-y)+2y = 21\\ 24-3y+2y=21\\ -y=-3\\y=3\\ $
Substitute value of $y$ back into first equation:\\
$\therefore x=8-3=5$\\
$y=3$ and $x=5$

\item $y=2x+1 \\ $
Substitute value of $y$ into second equation:\\
$ x+2y+3=0\\ \therefore x+2(2x+1)+3=0\\ 
x+4x+2+3=0\\
5x=-5\\
\therefore x=-1$\\ 
Substitute value of $x$ back into first equation:\\
$\therefore y=2(-1)+1=-1$\\
$y=-1$ and $x=-1$

\item $\frac{a}{2}+b=4\\
a+2b=8\\
a-b=4\\
3b=4\\
b=\frac{4}{3}$\\
Substitute $b=\frac{4}{3}$ into the first equation:\\
$\frac{a}{2} +\frac{4}{3}=4\\
\frac{a}{2}=4-\frac{4}{3}\\
\frac{a}{2} = \frac{8}{3}\\
a=\frac{16}{3}$\\
$a=5\frac{1}{3}$ and $b=1\frac{1}{3}$ 
\item $\frac{1}{x}+\frac{1}{y}=3\\
y+x=3xy\\
y-x=11xy\\
2y=14xy\\
\frac{2y}{14y}=x\\
x =\frac{1}{7}$\\
Substitute $x=\frac{1}{7}$ into the first equation:\\
$7+\frac{1}{y}=3\\
\frac{1}{y}=-4\\
y=-\frac{1}{4}$\\
$x=\frac{1}{7}$ and $y=-\frac{1}{4}$
\end{enumerate}
\item %Solve graphically and check your answer algebraically:
\begin{enumerate}[itemsep=20pt, label=\textbf{(\alph*)} ] 
\item  %$x+2y=1$ and $\frac{x}{3} + \frac{y}{2} = 1$\\
Solve graphically:\\
\scalebox{0.9}{
\begin{pspicture}(-3,-2)(4,2)
% \psgrid[gridcolor=lightgray,gridlabels=0,gridwidth=0.5pt]
\psset{yunit=0.5,xunit=0.5}
\psaxes[dx=2,Dx=2,arrows=<->, labels=all](0,0)(-3,-5)(6,3)
\psplot[linecolor=gray,xunit=0.5,plotstyle=curve,arrows=<->]{-2}{10.5}{x 0.5 neg mul 0.5 add}
\psplot[xunit=0.5,plotstyle=curve,arrows=<->]{-1}{10.5}{x 0.67 neg mul 2 add}
\uput[l](4.7,1.5){$y=-\frac{2}{3}x+2$}
\uput[l](-0.8,0.8){\color{gray}$y=-\frac{1}{2}x+\frac{1}{2}$}
% \uput[l](0.1,-0.2){$0$}
\uput[r](6,0.1){$x$}
\uput[r](0,3){$y$}
\psdot(4.5,-4)
\psline[linestyle=dotted](4.5,-4)(4.5,0)
\psline[linestyle=dotted](0,-4)(4.5,-4)
% \psline[arrows=<->](-4,1)(4,1)
\end{pspicture}
}\\To check algebraically:\\
$x+2y=1$\\ $\therefore x=-2y+1\\$ Substitute value of $x$ into the second equation:\\$\therefore \frac{-2y+1}{3} +\frac{y}{2} = 1\\
2(-2y+1) +3y=6\\
-4y+2+3y=6\\
\therefore y=-4$\\ 
Substitute value of $y$ back into the first equation:\\
$x+2(-4)=1\\
x-8=1\\
\therefore x=9$


\item %$5= x+y$ and $x = y-2$\\
Solve graphically:\\
\scalebox{0.9}{
\begin{pspicture}(-3,-2)(4,2)
% \psgrid[gridcolor=lightgray,gridlabels=0,gridwidth=0.5pt]
\psset{yunit=0.5,xunit=0.5}
\psaxes[dx=1,Dx=1,dy=1, Dy=1,arrows=<->](0,0)(-3,-5)(8,6)
\psplot[xunit=1,plotstyle=curve,arrows=<->]{-1}{6}{x 1 neg mul 5 add}
\psplot[linecolor=gray,xunit=1,plotstyle=curve,arrows=<->]{-1}{6}{x 2 sub}
\uput[l](4.5,4.4){$y=-x+5$}
\uput[l](3.8, -1.5){\color{gray}$y=x-2$}
\uput[l](0.1,-0.4){$0$}
\uput[r](8,0.1){$x$}
\uput[r](0,6){$y$}
\psdot(3.5,1.5)
\psline[linestyle=dotted](3.5,1.5)(3.5,0)
\psline[linestyle=dotted](0,1.5)(3.5,1.5)
% \psline[arrows=<->](-4,1)(4,1)
\end{pspicture}}\\
To check algebraically:\\
$5=x+y\\ 
\therefore y=5-x$\\ 
Substitute value of $y$ into second equation:\\
$x=5-x-2$\\
$2x=3\\
\therefore x=\frac{3}{2}$\\ 
Substitute value of $x$ back into first equation:\\
$5=\frac{3}{2} +y\\
y=\frac{10}{2} - \frac{3}{2} = \frac{7}{2}$

\item %$3x - 2y = 0$ and $x - 4y + 1 = 0$\\
Solve graphically:\\
\scalebox{1}{
\begin{pspicture}(-3,-2)(4,2)
% \psgrid[gridcolor=lightgray,gridlabels=0,gridwidth=0.5pt]
\psset{yunit=0.5,xunit=0.5}
\psaxes[dx=1,Dx=1,dy=1, Dy=1,arrows=<->](0,0)(-3,-5)(4,5)
\psplot[linecolor=gray,xunit=1,plotstyle=curve,arrows=<->]{-1.5}{3}{x 1.5 mul}
\psplot[xunit=1,plotstyle=curve,arrows=<->]{-2}{4}{x 0.25 mul 0.25 add}
\uput[l](4.5,4.7){\color{gray}$y=\frac{3}{2}x$}
\uput[ul](5.4, 1){$y=\frac{1}{4}x+\frac{1}{4}$}
\uput[l](0.1,-0.4){$0$}
\uput[r](4,0.1){$x$}
\uput[r](0,5){$y$}
\psdot(0.2,0.3)
\psline[linestyle=dotted,dotsep=0.01](0.2,0.3)(0.2,0)
\psline[linestyle=dotted](0,0.3)(0.2,0.3)
% \psline[arrows=<->](-4,1)(4,1)
\end{pspicture}}\\
To check algebraically:\\
$3x-2y=0\\
\therefore y=\frac{3}{2}x$\\
Substitute value of $y$ into second equation:\\
$\therefore x-4\left(\frac{3}{2}x\right) +1 = 0\\
x-6x+1 = 0\\
5x=1\\
x=\frac{1}{5}\\$
Substitute value of $x$ back into first equation:\\
$3\frac{1}{5} - 2y=0\\
3-10y=0\\
10y=3\\
y=\frac{3}{10}$
\item %$\frac{x}{4}=\frac{y}{2}-1$  and $\frac{y}{4}+\frac{x}{2}=1$ 
Solve graphically:\\
\scalebox{0.9}{
\begin{pspicture}(-3,-2)(4,2)
% \psgrid[gridcolor=lightgray,gridlabels=0,gridwidth=0.5pt]
\psset{yunit=0.5,xunit=0.5}
\psaxes[dx=1,Dx=1,arrows=<->, labels=all](0,0)(-5,-5)(5,6)
\psplot[linecolor=gray,xunit=1,plotstyle=curve,arrows=<->]{-1}{3}{x -2 mul 4 add}
\psplot[xunit=1,plotstyle=curve,arrows=<->]{-5}{4}{x 0.5 mul 2 add}
\uput[l](6,4.5){$y=\frac{1}{2}x+2$}
\uput[l](5.4,1.5){\color{gray}$y=-2x+4$}
% \uput[l](0.1,-0.2){$0$}
\uput[r](5,0.1){$x$}
\uput[r](0,6){$y$}
\psdot(0.8,2.4)
\psline[linestyle=dotted](0.8,2.4)(0.8,0)
\psline[linestyle=dotted](0,2.4)(0.8,2.4)
% \psline[arrows=<->](-4,1)(4,1)
\end{pspicture}
}\\To check algebraically:\\
$\frac{x}{4}=\frac{y}{2}-1\\
x=2y-4\\
y=\frac{1}{2}x+2\\
\frac{y}{4}+\frac{x}{2}=1 \\
y+2x=4\\
y=-2x+4$\\
Substitute $y=\frac{1}{2}x+2$ into $y=-2x+4$\\
$\frac{1}{2}x + 2=-2x+4\\
2\frac{1}{2}x = 2\\
x=\frac{4}{5}$\\
Substitute into $y=-2x+4$\\
$y=-2\frac{4}{5}+4\\
=-\frac{8}{5}+4\\
=2\frac{2}{5}$
\item %$2x+y=5$ and $3x-2y=4$ 
Solve graphically:\\
\scalebox{0.9}{
\begin{pspicture}(-3,-2)(4,2)
% \psgrid[gridcolor=lightgray,gridlabels=0,gridwidth=0.5pt]
\psset{yunit=0.5,xunit=0.5}
\psaxes[dx=1,Dx=1,arrows=<->, labels=all](0,0)(-5,-5)(5,7)
\psplot[linecolor=gray,xunit=1,plotstyle=curve,arrows=<->]{-1}{4}{x -2 mul 5 add}
\psplot[xunit=1,plotstyle=curve,arrows=<->]{-1}{4}{x 1.5 mul 2 sub}
\uput[l](6,4.5){$y=\frac{3}{2}-2$}
\uput[l](5,-3.2){\color{gray}$y=-2x-5$}
% \uput[l](0.1,-0.2){$0$}
\uput[r](5,0.1){$x$}
\uput[r](0,6){$y$}
\psdot(2,1)
\psline[linestyle=dotted](2,1)(2,0)
\psline[linestyle=dotted](0,1)(2,1)
% \psline[arrows=<->](-4,1)(4,1)
\end{pspicture}
}\\
$y=-2x+5\\
y=\frac{3}{2}x-2\\
0=-\frac{7}{2}x+7\\
\frac{7}{2}x=7\\
x=2$\\
Substitute $x=2$ into the first equation:\\
$2(2)+y=5\\
y=1$
\end{enumerate}
\end{enumerate}}
\end{solutions}



\begin{exercises}{}
{
\begin{enumerate}[noitemsep, label=\textbf{\arabic*}. ] 
\item Twee vliegtuie vlieg na mekaar toe vanat lughauwens $1~200$ km van mekaar af. Een vlieg teen $250$ km/h en die ander teen $350$ km/h. As hulle dieselfde tyd vertrek, hoe lank sal dit hulle neem om by mekaar verby te vlieg?
\item Kadesh het $20$ hempde gekoop teen 'n totale bedrag van R$~980$. As die groot hempde R$~50$ kos en die klieneres R$~40$ kos, hoeveel van elke groote het hy gekoop?
\item Die diagonaal van ’n reghoek is $25~$cm meer as die wydte. Die lengte van die reghoek is $17~$cm meer as die wydte. Wat is die afmetings van die reghoek?   
\item Die som van $27$ en $12$ is $73$ meer as ’n onbekende getal. Vind die onbekende getal.
\item Die twee kleiner hoeke van ’n reghoekige driehoek is in die verhouding $1~:~2$. Wat is die groottes van die twee hoeke?
\item Die lengte van 'n reghoek is tweemaal die breedte. As die oppervlakke $128$ cm$^{2}$ is, bepaal die lengte en breedte.       
\item As $4$ keer ’n getal met $6$ vermeerder word, is die resultaat $15$ minder as die vierkant (kwadraat) van die getal. Vind die getal wat hierdie stelling bevredig deur ’n vergelyking op te stel en dan op te los.
\item Die lengte van ’n reghoek is $2~$cm cm meer as die wydte van. Die omtrek van die reghoek is  $20~$cm. Vind die lengte en breedte van die reghoek.
\item Stefan het $1~l$ liter van ’n mengsel wat $69\%$ out bevat. Hoeveel water moet Vian bygooi om die mengsel $50\%$ sout te maak? Skryf jou antwoord as ’n breukdeel van ’n liter.
%Still in English
\item The sum of two consecutive odd numbers is $20$ and their difference is $2$. Find the two numbers. 
\item The denominator of a fraction is 1 more than the numerator. The sum of the fraction and its reciprocal is $\frac{5}{2}$. Find the fraction.
\item Thembi is 21 years older than her daughter, Phumzi. The sum of their ages is 37. How old is Phumzi?
\item Mark is now five times as old as his son Sam. In seven years from now, Mark will be three times as old as his son. Find their ages now.  
\end{enumerate}

}
\end{exercises}


 \begin{solutions}{}{
\begin{enumerate}[itemsep=10pt, label=\textbf{\arabic*}. ] 


\item Let distance $d_1=1~200-x$ km and $d_2=x$ km \\
speed $S_1=250$ km/h and $S_2=350$ km/h. \\
$\mbox{Time } t = \dfrac{\mbox{Distance}}{\mbox{Speed}} \\[5pt]$
When the jets pass each other $\dfrac{1~200-x}{250} = \dfrac{x}{350}\\[5pt]
350(1~200-x)=250x\\
420~000 - 350x = 250x\\
600x = 420~000\\
x= 700$ km\\[5pt]

$\therefore t=\dfrac{700\mbox{ km}}{350{\mbox{ km/h}}}\\[5pt]
=2$ hours\\
It will take take the jets $2$ hours to pass each other.

\item  Let $x=$ the number of large shirts and $20-x$ the number of small shirts.\\
Populate the data in a table format:\\
  \begin{tabularx}{8cm}{ |X|X|X|X| }\hline
& Nr shirts & Cost & Total \\ \hline
Large & $x$ & $50$ & $980$\\ \hline
Small & $20-x$ & $40$&\\ \hline
\end{tabularx}\\
\\
$50x+40(20-x)=980\\
50x+800-40x=980\\
10x=180\\
\therefore x=18\\$
Kadesh buys $18$ large shirts and $2$ small shirts.

\item Let length $=l$, width $=w$ and diagonal $=d$.\\
$\therefore d=w+25$ and $l=w+17$.\\
By the theorem of Pythagoras:\\
$d^2 = l^2+w^2\\
\therefore w^2 = d^2-l^2\\
=(w+25)^2 - (w+17)^2\\
=w^2+50w+625-w^2-34w-289\\
\therefore w^2 - 16w - 336 = 0\\
(w+12)(w-28)=0\\
\therefore w= -12$ or $w=28$\\
The width must be positive, therefore \\
width $w=28$ cm\\ 
length $l=(w+17)=45$ cm \\
and diagonal $d=(w+25)=53$ cm.


\item Let the unkown number $=x$\\
$\therefore 27+12=x+73\\
39=x+73\\
x=39-73\\
x=-34\\$
The unknown number is $-34$.
\item Let $x=$ the smallest angle. Therefore the other angle $=2x$. \\
We are given the third angle $=90^{\circ}$. \\
$x+2x+90^{\circ} = 180^{\circ}$ (sum of angles in a triangle)\\
$\therefore 3x = 90^{\circ}\\
\therefore x= 30^{\circ}$\\
The sizes of the angles are $30^{\circ}$ and $60^{\circ}$. 
\item 
We are given length $l=2b$ and $l\times b=128$\\
$\therefore 2b\times b =128\\
2b^2=128\\
b^2=64\\
b=\pm8\\$
But breadth must be positive, \\
therefore $b=8$ cm, and $l=2b=16$ cm.
\item Let the number $=x$. \\
Therefore the equation is $4x+6=x^2-15$.\\
$\therefore x^2-4x-21=0\\
(x-7)(x+3)=0\\
\therefore x=7$ or $x=-3$
\item Let length $l=x$, width $w=x-2$ and perimeter $=p$\\
$\therefore p = 2l+2w\\
=2x+2(x-2)\\
20=2x+2x-4\\
4x=24\\
x=6$\\
$\therefore$ $l=6$ cm and $w=l-2=4$ cm.
\item The new volume ($x$) of mixture must contain $50\%$ salt, therefore\\
$0,69 = 0.5x\\
\therefore x = \frac{0,69 }{0,5}\\
x = 2(0,69) = 1,38\\$
The volume of the new mixture is $1,38~ \ell$\\
The amount of water ($y$) to be added is \\
$y= x-1,00$\\
$= 1,38-1,00\\
=0,38$\\
Therefore $0,38~\ell$ of water must be added.\\
To write this as a fraction of a litre:\\
$0,38 = \frac{38}{100}\\
=\frac{19}{50}~\ell$

\item Let the numbers be $x$ and $y$.\\
$x+y=20\\
x-y=2\\
2x=22\\
x=11\\
11-y=2\\
y=9$\\
So the numbers are $9$ and $11$
\item Let the number be $x$.\\
So the denominator is $x+1$.\\
$\frac{x}{x+1}+\frac{x+1}{x}=\frac{5}{2}$\\
Solve for $x$:\\
$2x^2+2(x+1)^2=5x(x+1)\\
2x^2+2x^2+4x+2=5x^2+5x\\
-x^2-x+2=0\\
x^2+x-2=0\\
(x-1)(x+2)=0$\\
$x=1$ or $x=-1$\\
$x+1=2$ or $x+1=-1$\\
So the fraction is $\frac{1}{2}$ 
\item Let Phumzi be $x$ years old.\\
So Thembi is $x+21$ years old.\\
$x+x+21=37\\
2x=16\\
x=8$\\
Phumzi is $8$ years old.
\item Let Sam be $x$ years old.\\
So Mark is $5x$ years old.\\
In $7$ years time Sams age will be $x+7$\\
Marks age will be $5x+7$\\
$5x+7=3(x+7)\\
5x+7=3x+21\\
2x=14\\
x=7$\\
So Sam is $7$ years old and Mark is $35$ years old.
\end{enumerate}}
\end{solutions}


% Ex 2-5
\begin{exercises}{}
{
% \begin{multicols}{2}
\begin{enumerate}[itemsep=4pt, label=\textbf{\arabic*}. ] 
\item Make $a$ the subject of the formula: $s=ut+\frac{1}{2}at^{2}$
\item Solve for $n$: $pV=nRT$ 
\item Make $x$ the subject of the formula: $\dfrac{1}{b}+\dfrac{2b}{x}=2$
\item Solve for $r$: $V = \pi r^{2} h$
\item Solve for $h$: $E=\dfrac{hc}{\lambda}$
\item Solve for $h$: $A=2\pi rh + 2 \pi r$
\item Solve for $\lambda$: $t=\dfrac{D}{f \lambda}$
\item Solve for $m$: $E=mgh + \frac{1}{2}mv^{2}$
\item Solve for $x$: $x^2+x(a+b)+ab=0$
\item Solve for $b$: $c=\sqrt{a^2+b^2}$
\item Make $u$ the subject of the formula: $\frac{1}{v}=\frac{1}{u}+\frac{1}{u}$
\item Solve for $r$: $A=\pi R^2 -\pi r^2$
\item $F=\frac{9}{5}C + 32^\circ$ is the formula for converting temperature in degrees Celcius to degrees Fahrenheit. Derive a formula for converting degrees Fahrenheit to degrees Celcius.
\item $V=\frac{4}{3}\pi r^3$ is the formula for determining the volume of a soccer ball. Express the radius in terms of the volume.
\end{enumerate}
% \end{multicols}
}
\end{exercises}


 \begin{solutions}{}{
\begin{multicols}{2}
\begin{enumerate}[itemsep=5pt, label=\textbf{\arabic*}. ] 


\item $s=ut+\frac{1}{2}at^2\\
s-ut=\frac{1}{2}at^2\\
2(s-ut)=at^2\\[3pt]
\therefore a=\dfrac{2(s-ut)}{t^2}$

\item $pV=nRT\\[4pt]
\therefore n=\dfrac{pV}{RT}$

\item $\dfrac{1}{b} + \dfrac{2b}{x} = 2\\[4pt]
\dfrac{2b}{x}=2-\dfrac{1}{b}\\[4pt]
\dfrac{2b}{x} = \dfrac{2b-1}{b}\\[4pt]
\therefore x(2b-1)=2b^2\\[4pt]
x=\dfrac{2b^2}{2b-1}$

\item $V \pi r^2 h\\[4pt]
r^2=\dfrac{V}{\pi h}\\[4pt]
\therefore V = \sqrt{\frac{V}{\pi h}}$

\item $E = \dfrac{hc}{\lambda}\\[4pt]
E\lambda = hc\\[4pt]
\therefore h= \dfrac{E\lambda}{c}$

\item $A=P(1+i)^n\\[4pt]
\left(\dfrac{A}{P}\right)^{\frac{1}{h}} = (1+i)^{n \times \frac{1}{n}}\\[4pt]
\left(\dfrac{A}{P}\right) - 1 = i\\[4pt]
\therefore i = \left(\dfrac{A}{P}\right) - 1$

\item $A=2\pi rh + 2\pi r\\
2\pi rh = A - 2\pi r\\[4pt]
\therefore h = \dfrac{A-2\pi r}{2\pi r}$

\item $t=\dfrac{D}{f\lambda}\\[4pt]
ft\lambda=D\\[4pt]
\therefore \lambda = d\frac{D}{ft}$

\item $E = mgh +\frac{1}{2}mV^2\\[4pt]
E = m\left(gh + \frac{1}{2}V^2}\right)\\[4pt]
\therefore m = \dfrac{E}{gh + \frac{1}{2}V^2}$

\item $x^2+x(a+b)+ab=0\\
(x+a)(x-b)=0$\\
$x=-a$ or $x=-b$
\item $c=\sqrt{a^2 +b^2}\\
c^2=a^2+b^2\\
c^2-a^2=b^2\\
b=\pm \sqrt{c^2-a^2}$ 
\item $\frac{1}{v} =\frac{1}{u} + \frac{1}{w}\\
uw-uv=vw\\
u(w-v)=vw\\
u=\frac{vw}{w-v}$
\item $A=\pi R^2 -\pi r^2\\
A = \pi(R^2 - r^2)\\
\frac{A}{\pi}=R^2 -r^2\\
r^2 = R^2 - \frac{A}{\pi}\\
r = \pm \sqrt{R^2 - \frac{A}{\pi}}$
\item $F=\frac{9}{5}C +32\\
5F = 9C +160\\
9C=160 -5F\\
C = \frac{160}{9} -\frac{5}{9}F$
\item $V = \frac{4}{3}\pi r^3 \\
3V = 4 \pi r^3\\
r^3=\frac{3V}{4 \pi}\\
r = \sqrt[3]{\frac{3V}{4 \pi}}$
\end{enumerate}
\end{multicols}
}

\end{solutions}


%Ex 2-6
\begin{exercises}{ }
{
Solve for $x$ and represent the answer on a number line and in interval notation:
\begin{enumerate}[itemsep=5pt, label=\textbf{\arabic*}. ] 
\begin{multicols}{2}
    \item $3x+4>5x+8$
    \item $3(x-1)-2\leq 6x+4$ 
    \item $\dfrac{x-7}{3}>\dfrac{2x-3}{2}$
    \item $-4(x-1)<x+2$
    \item $\dfrac{1}{2}x+\dfrac{1}{3}(x-1)\geq \dfrac{5}{6}x-\dfrac{1}{3}$ 
    \item $-2\leq x-1<3$ 
    \item $-5<2x-3\leq7$ 
\item $7(3x+2)-5(2x-3)>7$
\item $\frac{5x - 1}{-6} \leq \frac{1-2x}{3}$
\item $3 \leq 4 - x \leq 16$
\item $\frac{-7y}{3} - 5 > -7$
\item $1 \leq 1 - 2y < 9$
\item $-2 < \frac{x-1}{-3}<7$
\end{multicols}
    \end{enumerate}

}
\end{exercises}


 \begin{solutions}{}{
\begin{enumerate}[itemsep=7pt, label=\textbf{\arabic*}. ] 


\item 
$3x+4>5x+8\\
3x-5x>8-4\\
-2x>4\\
2x<-4\\
x<-2\\$\\
\scalebox{0.8}{
\begin{pspicture}(-5,0.75)(4,1.75)
%\psgrid
\psline[arrows=<->](-5,1)(1.5,1)
\multido{\n=-5+1}{7}
{\uput[d](\n,1){$\n$}
\psline(\n,1.1)(\n,0.9)}
\uput[u](-2,1){$x<-2$}
\psline[linewidth=3pt]{->}(-2.1,1)(-6.1,1)

\psdot[dotsize=6pt,dotstyle=o](-2,1)
\end{pspicture}}

    \item $3(x-1)-2\leq 6x+4\\
3x-3-2\leq 6x+4\\
3x-6x\leq 4+5\\
-3x\leq 9\\
3x \geq -9\\
x\geq -3\\$\\
\scalebox{0.8}{
\begin{pspicture}(-5,0.75)(4,1.75)
%\psgrid
\psline[arrows=<->](-4,1)(2,1)
\psdot[dotsize=6pt](-3,1)
\multido{\n=-4+1}{6}
{\uput[d](\n,1){$\n$}
\psline(\n,1.1)(\n,0.9)}
\uput[u](-3,1){$x\geq -3$}
\psline[linewidth=3pt]{->}(-3,1)(2.1,1)
\end{pspicture}}

    \item $\dfrac{x-7}{3}>\dfrac{2x-3}{2}
2(x-7)>3(2x-3)\\
2x-14>6x-9\\
-4x>5\\
x<-\frac{5}{4}$\\ \\
\scalebox{0.8}{
\begin{pspicture}(-5,0.75)(4,1.75)
%\psgrid
\psline[arrows=<->](-3,1)(1,1)

\multido{\n=-5+1}{7}
{\uput[d](\n,1){$\n$}
\psline(\n,1.1)(\n,0.9)}
\uput[u](-1.25,1){$x<-\dfrac{5}{4}$}
\psline[linewidth=3pt]{->}(-1.3,1)(-6.1,1)
\psdot[dotsize=6pt,dotstyle=o](-1.25,1)
\end{pspicture}}

    \item $-4(x-1)<x+2\\
-4x+4 < x+2\\
-5x<-2\\
x>\frac{2}{5}\\$\\
\scalebox{0.8}{
\begin{pspicture}(-5,0.75)(4,1.75)
%\psgrid
\psline[arrows=<->](0,1)(4,1)

\multido{\n=0+1}{5}
{\uput[d](\n,1){$\n$}
\psline(\n,1.1)(\n,0.9)}
\uput[u](0.5,1){$x>\dfrac{2}{5}$}
\psline[linewidth=3pt]{->}(0.5,1)(4.4,1)
\psdot[dotsize=6pt,dotstyle=o](0.4,1)
\end{pspicture}}
    \item $\dfrac{1}{2}x+\dfrac{1}{3}(x-1)\geq \dfrac{5}{6}x-\dfrac{1}{3}\\
3x +2(x-1) \geq 5x-2\\
3x+2x-2 \geq 5x-2\\
5x-5x\geq 2-2\\
0x\geq 0\\$
This is true for all real values of $x$.\\ \\
\scalebox{0.8}{
\begin{pspicture}(-5,0.75)(4,1.75)
%\psgrid
\psline[arrows=<->](-2,1)(2,1)
% \psdot[dotsize=6pt](0,1)
\multido{\n=-2+1}{5}
{\uput[d](\n,1){$\n$}
\psline(\n,1.1)(\n,0.9)}
\uput[u](0.5,1){$0x\geq0$}
\psline[linewidth=3pt]{<->}(-2.5,1)(2.5,1)
\end{pspicture}}
   

 \item $-2\leq x-1<3\\
-2+1 \leq x-1+1< 3+1\\
-1 \leq x < 4\\$ \\
\scalebox{0.8}{
\begin{pspicture}(-5,0.75)(4,1.75)
%\psgrid
\psline[arrows=<->](-2,1)(5,1)

\psdot[dotsize=6pt](-1,1)
\multido{\n=-2+1}{8}
{\uput[d](\n,1){$\n$}
\psline(\n,1.1)(\n,0.9)}
\uput[u](1.5,1){$-1 \leq x < 4$}
\psline[linewidth=3pt]{}(-0.9,1)(3.9,1)
\psdot[dotsize=6pt,dotstyle=o](4,1)
\end{pspicture}}
    \item $-5<2x-3\leq7
-5+3<2x-3\leq 7+3\\
-2 < 2x \leq 10\\
-1 < x \leq 5\\$\\
\scalebox{0.8}{
\begin{pspicture}(-5,0.75)(4,1.75)
%\psgrid
\psline[arrows=<->](-2,1)(6,1)
\psdot[dotsize=6pt](5,1)

\multido{\n=-2+1}{9}
{\uput[d](\n,1){$\n$}
\psline(\n,1.1)(\n,0.9)}
\uput[u](2,1){$-1 <x \leq 5$}
\psline[linewidth=3pt]{}(-0.9,1)(4.9,1)
\psdot[dotsize=6pt,dotstyle=o](-1,1)
\end{pspicture}}

\item $7(3x+2)-5(2x-3)>7\\
21x+14 - 10x+15 > 7\\
11x > 7-14-15\\
11x>-22\\
x> -2\\$\\
\scalebox{0.8}{
\begin{pspicture}(-5,0.75)(4,1.75)
%\psgrid
\psline[arrows=<->](-2.8,1)(2,1)

\multido{\n=-2+1}{5}
{\uput[d](\n,1){$\n$}
\psline(\n,1.1)(\n,0.9)}
\uput[u](0,1){$x>-2$}
\psline[linewidth=3pt]{->}(-1.9,1)(2.3,1)
\psdot[dotsize=6pt,dotstyle=o](-2,1)
\end{pspicture}}
\item $\frac{5x - 1}{-6} \leq \frac{1-2x}{3}\\
5x -1 \geq -2(1-2x)\\
5x-1 \geq -2 +4x\\
5x-4x \geq -1\\
x \geq -1$\\
\scalebox{0.8}{
\begin{pspicture}(-5,0.75)(4,1.75)
%\psgrid
\psline[arrows=<->](-3,1)(2,1)
\multido{\n=-3+1}{5}
{\uput[d](\n,1){$\n$}
\psline(\n,1.1)(\n,0.9)}
\uput[u](-1,1){$x \geq -1$}
\psline[linewidth=3pt]{->}(-1.1,1)(2,1)

\psdot[dotsize=6pt](-1,1)
\end{pspicture}}

\item $3 \leq 4 - x \leq 16\\
-1 \leq -x \leq 12\\
1 \geq x \geq -12\\
-12 \leq x \leq 1$\\
\scalebox{0.8}{
\begin{pspicture}(-5,0.75)(4,1.75)
%\psgrid
\psset{xunit=0.5}
\psline[arrows=<->](-12.9,1)(2,1)
\multido{\n=-12+2}{7}
{\uput[d](\n,1){$\n$}
\psline(\n,1.1)(\n,0.9)}
\uput[u](-5,1){$-12\leq x \leq 1$}
\psline[linewidth=3pt](-11.9,1)(1,1)
\uput[d](1,1){$1$}

\psdots[dotsize=6pt](-12,1)(1,1)
\end{pspicture}}
\item $\frac{-7y}{3} - 5 > -7\\
-7y-15>-21\\
-7y>-6\\
y < \frac{6}{7}$\\
\scalebox{0.8}{
\begin{pspicture}(-5,0.75)(4,1.75)
%\psgrid
% \psset{xunit=0.5}
\psline[arrows=<->](-2.8,1)(2,1)
\multido{\n=-3+1}{4}
{\uput[d](\n,1){$\n$}
\psline(\n,1.1)(\n,0.9)}
\uput[u](-1,1){$y < \frac{6}{7}$}
\psline[linewidth=3pt]{<-}(-4,1)(0.8,1)
\uput[d](0.857,1){$\frac{6}{7}$}

\psdots[dotsize=6pt, dotstyle=o](0.857,1)
\end{pspicture}}
\item $1 \leq 1 - 2y < 9\\
0 \leq -2y < 8\\
0 \geq y >-4\\
-4 < y \leq 0$\\
\scalebox{0.8}{
\begin{pspicture}(-5,0.75)(4,1.75)
%\psgrid

\psline[arrows=<->](-4.5,1)(1,1)
\multido{\n=-4+1}{5}
{\uput[d](\n,1){$\n$}
\psline(\n,1.1)(\n,0.9)}
\uput[u](-2,1){$-4 < x \leq 0$}
\psline[linewidth=3pt](-3.9,1)(-0.1,1)


\psdot[dotsize=6pt, dotstyle=o](-4,1)
\psdot[dotsize=6pt](0,1)
\end{pspicture}}
\item $-2 < \frac{x-1}{-3}<7\\
6 > x-1 > -21\\
7 > x > -20\\
-20<x<7$\\
\scalebox{0.8}{
\begin{pspicture}(-5,0.75)(4,1.75)
%\psgrid
\psset{xunit=0.3}
\psline[arrows=<->](-21.8,1)(8.8,1)
\multido{\n=-20+5}{6}
{\uput[d](\n,1){$\n$}
\psline(\n,1.1)(\n,0.9)}
\uput[u](-5,1){$-20<x < 7$}
\psline[linewidth=3pt](-19.9,1)(6.9,1)
\uput[d](7,1){$7$}

\psdots[dotsize=6pt, dotstyle=o](-20,1)(7,1)
\end{pspicture}}
\end{enumerate}}
\end{solutions}


\begin{eocexercises}{}
\begin{enumerate}[itemsep=5pt, label=\textbf{\arabic*}. ] 
 \item 
Solve:
\begin{enumerate}[itemsep=5pt,label=\textbf{(\alph*)}]  
\begin{multicols}{2} 
\item $2(p-1) = 3(p+2)$
\item $3-6k = 2k-1$
\item $m + 6(-m+1) + 5m = 0$
\item $2k + 3 = 2-3(k+3)$
\item $5t-1=t^{2}-(t+2)(t-2)$
\item $3+\dfrac{q}{5} = \dfrac{q}{2}$ 
\item $5-\dfrac{2(m+4)}{m} = \dfrac{7}{m}$
\item $\dfrac{2}{t} - 2 - \dfrac{1}{2} = \dfrac{1}{2}\left(1+\dfrac{2}{t}\right)$
\item $x^{2} - 3x + 2=0$
\item $y^{2} + y = 6$
\item $0=2x^{2} - 5x - 18$
\item $(d+4)(d-3)-d=(3d-2)^{2} - 8d(d-1)$
\item $5x+2\leq4(2x-1)$
\item $\dfrac{4x-2}{6} > 2x+1$
\item $\dfrac{x}{3} - 14 > 14 - \dfrac{x}{7}$
\item $\dfrac{1-a}{2} - \dfrac{2-a}{3} \geq 1$
\item $-5 \leq 2k + 1 < 5$
\item $x-1=\dfrac{42}{x}$  
\end{multicols}
\end{enumerate}

\item Consider the following literal equations:
\begin{multicols}{2}
\begin{enumerate}[itemsep=4pt,label=\textbf{(\alph*)}]
\item Solve for $I$: $P = VI$
\item Solve for $m$: $E=mc^{2}$
\item Solve for $t$: $v = u + at$
\item Solve for $f$: $\dfrac{1}{u} + \dfrac{1}{v} = \dfrac{1}{f}$
\item Solve for $C$: $F=\frac{9}{5}C + 32$
\item Solve for $y$: $m = \dfrac{y-c}{x}$
\end{enumerate}
\end{multicols}
\item Solve the following simultaneous equations:
\begin{multicols}{2}
\begin{enumerate}[itemsep=2pt,label=\textbf{(\alph*)}]
\item $7x+3y=13$ and $2x-3y=-4$  
\item $10=2x+y$ and $y=x-2$
\item $7x-41=3y$ and $17=3x-y$
\item $2y=x+8$ and $4y=2x-44$
\end{enumerate}
\end{multicols}
\item Find the solutions to the following word problems:
\begin{enumerate}[itemsep=2pt,label=\textbf{(\alph*)}]
\item $\frac{7}{8}$ of a certain number is $5$ more than of $\frac{1}{3}$ of the number. Find the number.
\item Three rulers and two pens have a total cost of R $21,00$. One ruler and one pen have a total cost of R $8,00$. How much does a ruler cost and how much does a pen cost? 
\item A man runs to the bus stop and back in $15$ minutes. His speed on the way to the bus stop is $5$ km/h and his speed on the way back is $4$ km/h. Find the distance to the bus stop.
\item Zanele and Piet skate towards each other on a straight path. They set off $20$ km apart. Zanele skates at $15$ km/h and Piet at $10$ km/h. How far will Piet have skated when they reach each other?
\item When the price of chocolates is increased by R $10$, we can buy five fewer chocolates for R $300$. What was the price of each chocolate before the price was increased?
\end{enumerate}
\end{enumerate}

\end{eocexercises}


 \begin{eocsolutions}{}{
\begin{enumerate}[itemsep=7pt, label=\textbf{\arabic*}. ] 

 \item 
\begin{multicols}{2}
\begin{enumerate}[itemsep=6pt,label=\textbf{(\alph*)}]
% \begin{multicols}{2} 
\item $2(p-1) = 3(p+2)\\
2p-2=3p+6\\
-p=8\\
\therefore~p=-8\\$
\item $3-6k = 2k-1\\
-8k=-4\\
\therefore~k=\frac{-4}{-8} =\frac{1}{2}\\$

\item $m + 6(-m+1) + 5m = 0\\
m-6m+6+5m=0\\
0m=-6\\
$No solution
\item $2k + 3 = 2-3(k+3)
2k+3=2-3k-9\\
5k=-10\\
\therefore~k=\frac{-10}{5} = -2\\$
\item $5t-1=t^{2}-(t+2)(t-2)\\ = 5t-1=t^2-t^2+4\\ 5t=5\\ t=1$

\item $3+\frac{q}{5} = \frac{q}{2}\\30q+2q=5q\\3q=30\\q=10$
\item $5-\frac{2(m+4)}{m} = \frac{7}{m}\\ 5m-2m-8=7\\3m=15\\m=5$
\item $\frac{2}{t} - 2 - \frac{1}{2} = \frac{1}{2}\left(1+\frac{2}{t}\right)\\ \frac{2}{t} - 2 - \frac{1}{2} = \frac{1}{2} +\frac{1}{t}\\
\frac{2}{t} - \frac{1}{t} = 1+2\\
\frac{1}{t}=3\\
\therefore t=\frac{1}{3}$
\item $x^{2} - 3x + 2=0\\
(x-2)(x-1)=0\\
\therefore x = 2$ or $x=1$
\item $y^{2} + y = 6\\
y^2+y-6=0\\
(y+3)(y-2)=0\\
\therefore y=-3$ or $y=2$
\item $0=2x^{2} - 5x - 18\\
(2x+9)(x-2)=0\\ \therefore x = -\frac{9}{2}$ or $ x=2$
\item $(d+4)(d-3)-d=(3d-2)^{2} - 8d(d-1)\\
d^2+d-12-d=9d^2-12d+4-8d^2+8d\\
d^2-12 = d^2 -4d +4\\
4d=16\\
\therefore d=4$
\item $5x+2\leq4(2x-1)\\
5x-8x \leq -4-2\\
-3x \leq -6\\
\therefore~ x \geq 2$
\item $\dfrac{4x-2}{6} > 2x+1\\
4x-2 > 12x+6\\
4x-12x >6+2\\
-8x > 8\\
\therefore~x<-1$
\item $\frac{x}{3} - 14 > 14 - \frac{x}{7}\\
7x - 294 > 294-3x\\
10x > 588\\
\therefore x > \frac{588}{10}$
\item $\dfrac{1-a}{2} - \dfrac{2-a}{3} \geq 1\\
\dfrac{1-a}{2} - \dfrac{(2+a)}{3} \geq 1\\
3-3a-4-2a \geq 6\\
-5a \geq 7\\
\therefore~ a \leq -\frac{7}{5}$
\item $-5 \leq 2k + 1 < 5\\
-6 \leq 2k < 4\\
\therefore -3 \leq k < 2$
\item $x-1=\dfrac{42}{x}\\$
Note that $x \neq 0$.\\
$x^2-x=42\\
x^2-x-42=0\\
(x-7)(x+6)=0\\
\therefore x=7$ or $x=-6$
\end{enumerate}
\end{multicols}
\item
\begin{multicols}{2}
\begin{enumerate}[itemsep=6pt,label=\textbf{(\alph*)}]
\item $P = VI\\
\therefore I =\frac{P}{V}$
\item $E=mc^{2}\\[4pt]
\therefore m=\dfrac{e}{c^2}$
\item $v = u + at\\[4pt]
\therefore t = \dfrac{v-u}{a}$
\item $\dfrac{1}{u} + \dfrac{1}{v} = \dfrac{1}{f}\\
\dfrac{v+u}{uv} = \dfrac{1}{f}\\
f(v+u) = uv\\[4pt]
\therefore f= \dfrac{uv}{u+v}$
\item  $F=\frac{9}{5}C + 32\\
\therefore C=\frac{5}{9}(F-32)$
\item $m = \dfrac{y-c}{x}\\[4pt]
\therefore y = mx+c$
\end{enumerate}
\end{multicols}
\item 
\begin{enumerate}[itemsep=5pt,label=\textbf{(\alph*)}]
\item $7x+3y=13$ and $2x-3y=-4$\\
Add the two equations together to solve for $x$:\\

  \begin{array}{cccc}$
    \hfill & 7x+3y& =& 13\hfill \\ 
    \hfill+ & 2x-3y& =& -4\hfill \\
    \hline
    \hfill & 9x +0 &=& 9 $
  \end{array}\\
$\therefore x=1\\$
Substitute value of $x$ into second equation:\\
$2(1)-3y = -4\\
-3y=-6\\
y=2$
\item $10=2x+y$ and $y=x-2$\\
Substitute value of $y$ into first equation:\\
$10=2x+(x-2)\\
3x =12\\
\therefore x=4\\$
Substitute value of $x$ into second equation:\\
$y=4-2\\
=2$


\item $7x-41=3y$ and $17=3x-y$\\
$17=3x-y\\
\therefore y= 3x-17\\$
Substitute the value of $y$ into first equation:\\
$7x-41=3(3x-17)\\
7x-41=9x-51\\
2x=10\\
\therefore x=5\\$
Substitute value of $x$ into second equation:\\
$y=3(5)-17\\
=-2$
\item $2y=x+8$ and $4y=2x-44\\
2y=x+8\\
\therefore x = 2y-8\\$
Substitute value of $x$ into second equation:\\
$4y=2(2y-8) - 44\\
4y=4y-16-44\\
\therefore 0y=-60\\$
No solution\\
$(y_1 = \frac{1}{2}x - 11$ and $y_2 = \frac{1}{2}x + 4$. These lines have the same gradient therefore they are parallel and will never intersect - hence there is no solution.)
\end{enumerate}
\item
\begin{enumerate}[itemsep=5pt,label=\textbf{(\alph*)}]
\item 
Let $x$ be the number.\\
$\therefore~\frac{7}{8}x = \frac{1}{3}x +5\\
21x=8x+120\\
13x=120\\
x=\frac{120}{13}$

\item Let the price of rulers be $x$ and the price of pens be $y$\\
$\therefore 3x+2y=21$ and $x+y=8$\\
From the second equation: $x=8-y$\\
Substitute the value of $x$ into the first equation:\\
$3(8-y) +2y=21\\
24-3y+2y=21\\
-y=-3\\
\therefore y=3\\$
Substitute the value of $y$ into the second equation:\\
$x+3=8\\
\therefore x=5$\\
Therefore each ruler costs R $5$ and each pen costs R $3$.

\item
Let $x$ be the distance to the bus.\\
Speed $s_1 = 5$ km/h and $s_2 = 4$ km/h\\
$\mbox{Distance } = \mbox{ speed} \times \mbox{ time} = s \times t\\

\therefore x = 5 \times t_1 = 4 \times t_1\\
t_1+t_2 = 15 \mbox{ minutes} = 0,25 \mbox {hours}\\
t_1 =  \frac{x}{5}$ and $t_2 = \frac{x}{4}\\
\therefore \frac{x}{5} +\frac{x}{4} = 0,25\\
4x+5x = 0,25 \times 20\\
9x = 5\\
\therefore x = \frac{5}{9}$ km\\
The bus stop is $\frac{5}{9}$ km or $0,56$ km away.
\item 
Let $x$ be the distance Zanele skates and $20-x$ the distance Piet skates.\\
Populate the data in a table:\\
  \begin{tabularx}{8cm}{ |X|X|X|X| }\hline
& Speed & Distance & Time \\ \hline
Zanele & $15$ & $x$ & $\frac{x}{15}$\\ \hline
Piet & $20-x$ & $20-x$& $\frac{20-x}{10}$\\ \hline
\end{tabularx}\\
$\dfrac{x}{15} = \dfrac{20-x}{10}\\[5pt]
10x=15(20-x)\\
10x=300-15x\\
25x=300\\
\therefore x=12\\
\therefore y=20-12=8\\$
Zanele will have skated $12$ km and Piet will have skated $8$ km when they reach each other.
\item Let $x$ be the original price of chocolates.\\
New price $\times$ number of chocolates $=300$\\
$\therefore (x+10)(\frac{300}{x} - 5)=300\\
300 - 5x+\frac{3000}{x} - 50 = 300\\
-5x + \frac{3000}{x} - 50 = 0\\
-5x^2 + 3000 - 50x = 0\\
x^2 + 10x- 600 = 0\\
(x-20)(x+30)=0\\
\therefore x=20$ or $x=-30$\\
Price must be positive $\therefore x = 20$.\\
The price of each chocolate before the price increase was R~$20$.
\end{enumerate}

\end{enumerate}}
\end{eocsolutions}


