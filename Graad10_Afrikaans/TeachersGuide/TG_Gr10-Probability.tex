\chapter{Waarskynlikheid}

\begin{exercises}{}
{
  \begin{enumerate}[itemsep=5pt, label=\textbf{\arabic*}.]
  \item 'n Sak bevat $6$ rooi balle, $3$ blou, $2$ groen en $1$ wit
    bal. 'n Bal word willekeurig gekies. Wat is die waarskynlikheid van die volgende:
    \begin{enumerate}
    \item rooi
    \item blou of wit
    \item nie groen
    \item nie groen of rooi
    \end{enumerate}
  \item 'n Speelkaart word willekeurig gekies uit 'n pak van $52$ kaarte. Wat is die waarskynlikheid van die volgende:
    \begin{enumerate}

    \item die $2$ van harte
    \item 'n rooi kaart
    \item 'n prentkaart
    \item 'n 'A'-getal
    \item 'n getal kleiner as $4$
    \end{enumerate}

  \item Ewe getalle in die interval $2$ tot $100$ word op kaarte geskryf. Wat is
    die waarskynlikheid om 'n veelvoud van $5$ op 'n willekeurig wyse te kies?
  \end{enumerate}
}
\end{exercises}


 \begin{solutions}{}{
\begin{enumerate}[itemsep=5pt, label=\textbf{\arabic*}. ] 
\item %solution 1
    \begin{enumerate}[noitemsep, label=\textbf{(\alph*)} ]
    \item $\frac{6}{12} &=& \frac{1}{2}$
    \item $\frac{(3 + 1)}{12} &=& \frac{1}{3}$
    \item $1 - (\frac{2}{12}) &=& \frac{5}{6}$
    \item $1 - \frac{(2 + 6)}{12} &= \frac{4}{12}\\
				  &= \frac{1}{3}$   
    \end{enumerate}
\item %solution 2
    \begin{enumerate}[noitemsep, label=\textbf{(\alph*)} ]
    \item $\frac{1}{52}$ (net een in die pak)
    \item $\frac{1}{2}$ (helfte van die kaarte is rooi, helfte is swart)
    \item $\frac{3}{13}$ (vir elke troef van 13 karte is daar drie prentkarte: J, Q, K)
    \item $\frac{4}{52} = \frac{1}{13}$ (vier ``Aces'' in die pak)
    \item $\frac{3}{13}$ (vir elke troef van 13 kaarte, is daar drie kaarte minder as $4$: A, $2$ en $3$)
    \end{enumerate}
\item %solution 3
    Daar is $50$ kaarte.  Hulle is almal ewe.\\
    Alle ewe getalle wat ook veelvoude van $5$ is, is veelvoude van $10$ $(10,~ 20,\ldots, 100)$.\\
    Daar is $10$ van hulle.\\
    Dus is die waarskynlikheid $\frac{10}{50} = \frac{1}{5}$.
\end{enumerate}}
\end{solutions}


% \section{Relative frequency}
% \section{Venn diagrams}
\begin{exercises}{}
{
  \begin{enumerate}[itemsep=5pt, label=\textbf{\arabic*}.]
  \item  Laat $S$ die versameling heelgetalle van $1$ tot $16$ aandui, $X$
    die versameling ewe getalle van $1$ tot $16$ en $Y$ die versameling priemgetalle van $1$ tot $16$
    \begin{enumerate}[noitemsep, label=\textbf{(\alph*)} ]
    \item Teken 'n Venn-diagram wat $S$, $X$ and $Y$ akkuraat aandui.
    \item Vind $n\left(S\right)$, $n\left(X\right)$, $n\left(Y\right)$,
      $n\left(X\cup Y\right)$, $n\left(X\cap Y\right)$.
    \end{enumerate}
  \item Daar is $79$ Graad $10$ leerders in die skool. Almal neem Wiskunde, Aardrykskunde of Geskiedenis. $41$ neem Aardrykskunde, $36$ neem Geskiedenis en $30$ neem Wiskunde. $16$ neem Wiskunde en Geskiedenis, $6$ neem Aardrykskunde en Geskiedenis, $8$ neem net Wiskunde en daar is $16$ wat slegs Geskiedenis neem.
    \begin{enumerate}[noitemsep, label=\textbf{(\alph*)} ]
    \item Teken 'n Venn-dagram om hierdie inligting te illustreer.
    \item Hoeveel leerders neem Wiskunde en Aardrykskunde, maar nie Geskiedenis nie?
    \item Hoeveel leerders neem net Aardrykskunde?
    \item Hoeveel leerders neem al drie vakke?
    \end{enumerate}
  \item Papiere word genommer van $1$ tot $12$ en in 'n houer geplaas en die houer geskud. Een papier op 'n slag word getrek en weer teruggeplaas.
    \begin{enumerate}[noitemsep, label=\textbf{(\alph*)} ]
    \item Wat is die monsterruimte, $S$?
    \item Skryf die versameling $A$ neer, wat die gebeurtenis voorstel van die trek van 'n papier met 'n faktor van $12$ daarop.
    \item Skryf die versameling $B$ neer, wat die gebeurtenis voorstel van die trek 'n papier gemerk met 'n priemgetal.
    \item Gebruik 'n Venn-diagram om $A$, $B$ and $S$ voor te stel.
    \item Vind
      \begin{enumerate}[noitemsep, label=\textbf{\roman*.} ]
      \item $n\left(S\right)$
      \item $n\left(A\right)$
      \item $n\left(B\right)$

      \end{enumerate}

    \end{enumerate}
 \end{enumerate}
 \end{enumerate}
% \practiceinfo
%   \begin{tabularx}{\textwidth}{XXXX}
%     (1.) AAA & (2.) AAA & (3.) AAA \\
%   \end{tabularx}
}
\end{exercises}


 \begin{solutions}{}{
\begin{enumerate}[itemsep=5pt, label=\textbf{\arabic*}. ] 
\item %solution 1
	  \scalebox{0.8} % Change this value to rescale the drawing.
	  {
	  \begin{pspicture}(0,-2.2)(4.5590625,2.2)
	  \pscircle[linewidth=0.04,dimen=outer](2.2,0.0){2.2}
	  \pscircle[linewidth=0.04,dimen=outer](1.45,-0.03){1.29}
	  \pscircle[linewidth=0.04,dimen=outer](2.82,0.48){1.18}
	  \usefont{T1}{ppl}{m}{n}
	  \rput(4.184531,1.89){\LARGE$S$}
	  \usefont{T1}{ppl}{m}{n}
	  \rput(2.1845312,0.27){\LARGE$2$}
	  \usefont{T1}{ppl}{m}{n}
	  \rput(1.2445313,0.89){\LARGE$4$}
	  \usefont{T1}{ppl}{m}{n}
	  \rput(0.64453125,0.51){\LARGE$6$}
	  \usefont{T1}{ppl}{m}{n}
	  \rput(1.3245312,0.19){\LARGE$8$}
	  \usefont{T1}{ppl}{m}{n}
	  \rput(0.6745312,-0.15){\LARGE$10$}
	  \usefont{T1}{ppl}{m}{n}
	  \rput(2.6045313,1.19){\LARGE$3$}
	  \usefont{T1}{ppl}{m}{n}
	  \rput(3.3245313,1.13){\LARGE $5$}
	  \usefont{T1}{ppl}{m}{n}
	  \rput(2.9645312,0.63){\LARGE$7$}
	  \usefont{T1}{ppl}{m}{n}
	  \rput(2.9045312,-1.41){\LARGE$9$}
	  \usefont{T1}{ppl}{m}{n}
	  \rput(1.0045313,1.5){\LARGE$X$}
	  \usefont{T1}{ppl}{m}{n}
	  \rput(2.4945312,1.9){\LARGE$Y$}
	  \usefont{T1}{ppl}{m}{n}
	  \rput(0.75453126,-0.59){\LARGE$12$}
	  \usefont{T1}{ppl}{m}{n}
	  \rput(1.7545313,-0.61){\LARGE$14$}
	  \usefont{T1}{ppl}{m}{n}
	  \rput(1.3945312,-1.03){\LARGE$16$}
	  \usefont{T1}{ppl}{m}{n}
	  \rput(3.5145311,0.35){\LARGE$11$}
	  \usefont{T1}{ppl}{m}{n}
	  \rput(3.2545311,-0.17){\LARGE$13$}
	  \usefont{T1}{ppl}{m}{n}
	  \rput(1.8445313,-1.69){\LARGE$1$}
	  \usefont{T1}{ppl}{m}{n}
	  \rput(3.6745312,-0.87){\LARGE$15$}
	  \end{pspicture} 
	  }
\item%solution 2
    \begin{enumerate}[noitemsep, label=\textbf{(\alph*)} ]
    \item
		      \scalebox{0.8} % Change this value to rescale the drawing.
		      {
		      \begin{pspicture}(0,-2.393125)(4.7790623,2.393125)
		      \pscircle[linewidth=0.04,dimen=outer](1.64,0.7196875){1.18}
		      \pscircle[linewidth=0.04,dimen=outer](1.16,-0.8203125){1.16}
		      \pscircle[linewidth=0.04,dimen=outer](2.64,-0.2603125){1.18}
		      \usefont{T1}{ppl}{m}{n}
		      \rput(4.3,0.6096875){\LARGE$G:41$}
		      \usefont{T1}{ppl}{m}{n}
		      \rput(0.77453125,-2.2){\LARGE$H:36$}
		      \usefont{T1}{ppl}{m}{n}
		      \rput(1.9145312,2.1896875){\LARGE$M:30$}
		      \usefont{T1}{ptm}{m}{n}
		      \rput(0.81453127,-1.1503125){\LARGE$16$}
		      \usefont{T1}{ptm}{m}{n}
		      \rput(1.7445313,1.3096875){\LARGE$8$}
		      \usefont{T1}{ptm}{m}{n}
		      \rput(1.9245312,-0.7503125){\LARGE$6$}
		      \usefont{T1}{ptm}{m}{n}
		      \rput(1.1945312,-0.0303125){\LARGE$16$}
		      \end{pspicture} 
		      }

    \item Elke leerder moet presies een van die volgende doen:
    \begin{itemize}
	\item net aardrykskunde neem;
	\item  net wiskunde en/of geskiedenis neem;
    \end{itemize}
    Daar is $30 + 36 -16 =50$ leerders wat die tweede opsie neem dus moet daar $79 - 50 = 29$ leerders wees wat net aardrykskunde neem.
    \newline
    Elke leerder moet presies een van die volgende doen:
    \begin{itemize}
	\item net aardrykskunde neem;
	\item net wiskunde neem;
	\item geskiedenis neem;
	\item aardrykskunde en wiskunde neem maar nie geskiedenis nie;
    \end{itemize}
    Daar is $29, 8$ en $36$ van die eerste drie dus is die antwoord vir b):\\
    $79 - 29 -8 - 36 = 6$ leerders

    \item Alreeds bereken: $29$

    \item Elke leerder moet presies een van die volgende doen:
    \begin{itemize}
	\item Aardrykskunde
	\item Net wiskunde
	\item Net geskiedenis
	\item wiskunde en geskiedenis maar nie aardrykskunde nie
    \end{itemize}
    As ons dieselfde metode as voorheen gebruik, is die getal leerders in die laaste groep:\\
 
    $79 - 41 - 8 - 16 = 14$\\

    Maar $16$ leerders neem wiskunde en geskiedenis dus moet daar $16 - 14=2$ leerders wat al drie neem.
    \end{enumerate}

\item %solution 3
    \begin{enumerate}[noitemsep, label=\textbf{(\alph*)} ]
    \item $S = \{1; 2;\ldots; 12\}$
    
    \item $A = \{1; 2; 3; 4; 6; 12\}$

    \item $B = \{2; 3; 5; 7; 11\}$

    \item

    		  \scalebox{0.8} % Change this value to rescale the drawing.

		  {
		  \begin{pspicture}(0,-2.2)(4.5590625,2.2)
		  \pscircle[linewidth=0.04,dimen=outer](2.2,0.0){2.2}
		  \pscircle[linewidth=0.04,dimen=outer](1.45,-0.03){1.29}
		  \pscircle[linewidth=0.04,dimen=outer](2.82,0.48){1.18}
		  \usefont{T1}{ppl}{m}{n}
		  \rput(4.184531,1.89){\LARGE $S$}
		  \usefont{T1}{ppl}{m}{n}
		  \rput(2.0045311,0.65){\LARGE$2$}
		  \usefont{T1}{ppl}{m}{n}
		  \rput(0.8645313,0.19){\LARGE$4$}
		  \usefont{T1}{ppl}{m}{n}
		  \rput(1.4645313,-0.27){\LARGE$6$}
		  \usefont{T1}{ppl}{m}{n}
		  \rput(1.9245312,-1.67){\LARGE$8$}
		  \usefont{T1}{ppl}{m}{n}
		  \rput(3.5345314,-1.03){\LARGE$10$}
		  \usefont{T1}{ppl}{m}{n}
		  \rput(2.2245312,0.07){\LARGE$3$}
		  \usefont{T1}{ppl}{m}{n}
		  \rput(2.7645311,1.17){\LARGE$5$}
		  \usefont{T1}{ppl}{m}{n}
		  \rput(3.3045313,0.57){\LARGE$7$}
		  \usefont{T1}{ppl}{m}{n}
		  \rput(2.8045313,-1.47){\LARGE$9$}
		  \usefont{T1}{ppl}{m}{n}
		  \rput(0.9945313,1.41){\LARGE$A$}
		  \usefont{T1}{ppl}{m}{n}
		  \rput(2.4945312,1.83){\LARGE$B$}
		  \usefont{T1}{ppl}{m}{n}
		  \rput(1.6145313,-0.95){\LARGE$12$}
		  \usefont{T1}{ppl}{m}{n}
		  \rput(3.1945312,-0.01){\LARGE$11$}
		  \usefont{T1}{ppl}{m}{n}
		  \rput(0.9845312,0.77){\LARGE$1$}
		  \end{pspicture} 
		  }

    \item
	\begin{enumerate}[noitemsep, label=\textbf{\roman*.} ]
	\item $12$
	\item $6$
	\item $5$
	
	\end{enumerate}
    

    \end{enumerate}
\end{enumerate}}
\end{solutions}


% \section{Union and intersection}
% \section{Probability identities}
% \section{Mutually exclusive events}
% \section{Complementary events}
\begin{exercises}{}
{
  \begin{enumerate}[itemsep=3pt, label=\textbf{\arabic*}.]
  \item  'n Kas bevat gekleurde blokkies. Die aantal van elke kleur word in die onderstaande tabel aangedui.

  \begin{center}
      \begin{tabular}{|l|c|c|c|c|}
        \hline
        \textbf{Kleur} & Pers & Oranje & Wit & Pink \\ \hline
       
        \textbf{Aantal blokkies} & $24$ & $32$ & $41$ & $19$ \\ \hline
   
      \end{tabular}
    \end{center}
    'n Blokkie word willekeurig gekies. Wat is die waarskynlikheid dat die blokkie die volgende kleur sal wees:
    \begin{enumerate}[noitemsep, label=\textbf{(\alph*)} ]
    \item pers
    \item pers of wit
    \item pienk en oranje
    \item nie oranje nie?
    \end{enumerate}

  \item 'n Klein skooltjie het 'n klas met kinders van verskillende ouderdomme. Die tabel gee die aantal kinders van elke ouderdom in die klas.

    \begin{center}
      \begin{tabular}{|l|c|c|c|}
        \hline
               & $3$ jaar oud & $4$ jaar oud & $5$ jaar oud \\\hline
   
        \textbf{Manlik}   & $2$ & $7$ & $6$ \\\hline
        \textbf{Vroulik} & $6$ & $5$ & $4$ \\\hline
       
      \end{tabular}
    \end{center}

    As 'n leerder willekeurig gekies word, wat is die waarskynlikheid dat die leerder die volgende sal wees:
    \begin{enumerate}[noitemsep, label=\textbf{(\alph*)} ]
    \item vroulik
    \item $4$ jaar oud en manlik 
    \item $3$ of $4$ jaar oud
    \item $3$ en $4$ jaar oud
    \item nie $5$ nie
    \item $3$ of vroulik?
    \end{enumerate}

  \item Fiona het $85$ skyfies, genommer van $1$ to
    $85$. Indien 'n skyfie willekeurig gekies word, wat is die waarskynlikheid dat die getal: 
    \begin{enumerate}[noitemsep, label=\textbf{(\alph*)} ]
    \item eindig met $5$
    \item 'n veelvoud van $3$ is
    \item 'n veelvoud van $6$ is
    \item $65$ is
    \item nie 'n veelvoud van $5$ is nie
    \item 'n veelvoud van $4$ of $3$ is
    \item 'n veelvoud van $2$ en $6$ is
    \item  $1$ is?
    \end{enumerate}
  \end{enumerate}

% \practiceinfo
%   \begin{tabularx}{\textwidth}{XXXX}
%     (1.) AAA & (2.) AAA & (3.) AAA \\
%   \end{tabularx}
}
\end{exercises}


\begin{solutions}{}{
\begin{enumerate}[itemsep=5pt, label=\textbf{\arabic*}. ] 

\item %solution 1
    Voor ons die vrae antwoord moet ons eerstens bereken hoeveel blokkies daar in totaal is. Die gee die monsterruimte \\
    $n(S) = 24 + 32 + 41 + 19$\\
	 $= 116$
    \begin{enumerate}[noitemsep, label=\textbf{(\alph*)} ]

    \item Die waarskynlikheid dat 'n blokkie pers is, is:\\
    $P(\mbox{pers}) = \frac{n(E)}{n(S)}\\
     P(\mbox{pers}) = \frac{24}{116}\\
     P(\mbox{pers}) = 0,21$

    \item Die waarskylikeheid dat 'n blokkie óf pers of wit is, is:\\
    $P(\mbox{pers} \cup \mbox{wit}) = P(\mbox{pers}) + P(\mbox{wit}) - P(\mbox{pers} \cap \mbox{wit})\\
    = \frac{24}{116} + \frac{41}{116} - \frac{24}{116} \times \frac{41}{116}\\
    = 0,64$

    \item Aangesien een blokkie nie twee kleure kan hê nie, is die waarskynlikheid van hierdie gebeurtenis $0$.

    \item Ons bereken eerstens die waarskynlikheid dat 'n blokkie oranje is:\\
    $P(\mbox{oranje}) = \frac{32}{116}
	       = 0,28$\\
    die waarskynlikheid dat 'n blokkie nie oranje is nie:\\
    $P(\mbox{nie oranje}) = 1 - 0,28\\
		   = 0,72$
    \end{enumerate}
    
\item %solution 2
Ons bereken die totale getal leerlinge by die skool :\\
$6 + 2 + 5 + 7 + 4 + 6 = 30$
    \begin{enumerate}[noitemsep, label=\textbf{(\alph*)} ]
    \item Die totale getal van vroulike kinders is $6 + 5 + 4 = 15$.\\
    Die waarskynlikheid dat 'n  ewekansig gekies kind vroulik sal wees is:\\
    $P(\mbox{vroulik}) = \frac{n(E)}{n(S)}\\
     P(\mbox{vroulik}) = \frac{15}{30}\\
     P(\mbox{vroulik}) = 0,5$

    \item 
Die waarskynlikheid van om 'n kind wat 4 jaar oud en manlike is te kies is:\\
    $P(\mbox{manlik}) = \frac{7}{30}\\
	      = 0,23$

    \item Daar is $6 + 2 + 5 + 7 = 20$ kinders wat van 3 of 4 is.\\
    
Die waarskynlikheid van om 'n kind van 3 of 4 is ewekansig te kies is:\\
    $\frac{20}{30} = 0,67$

    \item 'n kind kan nie 3 \textsl{en} 4 wees nie dus is die waarskylikeheid $0$.

    \item Dit is dieselfde asof 'n ewekansig kinder óf 3 óf 4 is, en dus is  $0,67$.

    \item Die waarskynlikheid dat 'n kind óf 3 óf vroulik is, is:\\
    $P(3 \cup \mbox{vroulik}) = P(3) + P(\mbox{vroulik}) - P(3 \cap \mbox{vroulik})\\
		      = \frac{10}{30} + 0,5 - \frac{10}{30} \times \frac{15}{30}\\
		      = 0,67$
    \end{enumerate}
    
\item %solution 3
    \begin{enumerate}[noitemsep, label=\textbf{(\alph*)} ]
    \item Die versameling skyfies wat met $5$ eindig is: $\{5$; $15$; $25$; $35$; $45$; $55$; $65$; $75$; $85\}$. Hierdie het $9$ elemente.\\
 
    Die waarskynlikheid van om 'n skyfie wat met $5$ eindig te kies is:\\
    $P(5) = \frac{n(E)}{n(S)}\\
     P(5) = \frac{9}{85}\\
     P(5) = 0,11$

    \item Die versameling van alle skyfies wat veelvoude van $3$ is, is: $\{3$; $6$; $ 9$; $ 12$; $ 15$; $ 18$; $ 21$; $ 24$; $ 27$; $ 30$; $ 33$; $ 36$; $ 39$; $ 42$; $ 45$; $ 48$; $ 51$; $ 54$; $ 57$; $ 60$; $ 63$; $ 66$; $ 69$; $ 72$; $ 75$; $ 78$; $ 81$; $ 84\}$. Hierdie het $28$ elemente.\\
 
    Die waarskynlikheid van om 'n skyfie wat 'n veelvoud van $3$ is te kies is:
    $P(3_m) = \frac{28}{85}
	    = 0,33$

    \item Die versameling van alle skyfies wat veelvoude van $6$ is, is: $\{6$; $ 12$; $ 18$; $ 24$; $ 30$; $ 36$; $ 42$; $ 48$; $ 54$; $ 60$; $ 66$; $ 72$; $ 78$; $ 84\}$. This set has $14$ elemente.
   Die waarskylikeheid van om 'n skyfie wat 'n veelvoud van $6$ is te kies is:\\
    $P(6_m) = \frac{14}{85}\\
	    = 0,16$

    \item 
    \Daar is net een element in hierdie versameling dus is die waarskynlikeheid van om $65$ te kies is:\\
    $P(65) = \frac{1}{85}\\
	   = 0,01$

    \item Die versameling van alle skyfies wat veelvoude van $5$ is, is: $\{5$; $ 10$; $ 15$; $ 20$; $ 25$; $ 30$; $ 35$; $ 40$; $ 45$; $ 50$; $ 55$; $ 60$; $ 65$; $ 70$; $ 75$; $ 80$; $ 85\}$. Hierdie versameling het $17$ elemente. Dus is die getal skyfies wat nie veelvoude van $5$: $85 − 17 = 68$.\\
   
    Die waarskynlikheid van om 'n skyfie wat nie 'n veelvoud van $5$ is te kies is:\\
    $P(\mbox{nie }5_m) = \frac{68}{85}\\
	       = 0,80$

    \item In deel b), het ons die waarskynlikheid vir 'n skyfie wat 'n veelvoud van $3$ is bereken. Nou bereken ons die getal elemente in die versameling van alle skyfies wat veelvoude van $4$ is: $\{4$; $ 8$; $ 12$; $ 16$; $ 20$; $ 24$; $ 28$; $ 32$; $ 36$; $ 40$; $ 44$; $ 48$; $ 52$; $ 56$; $ 60$; $ 64$; $ 68$; $ 72$; $ 76$; $ 80$; $ 84\}$. Hierdie het $28$ elemente.\\
    
    Die waarskynlikeheid dat 'n skyfie 'n veelvoud van óf $3$ of $4$ is, is:\\
    $P(3_m \cup 4_m) = P(3_m) + P(4_m) - P(3_m \cap 4_m)\\
    = 0,33 + \frac{28}{85} - 0,33 \times \frac{28}{85}\\
    = 0,55$
    
    \item Die versameling van alle skyfies wat veelvoude van $2$ en $6$ is is dieselfde as die versameling van alle skyfies van 'n veelvoud van $6$ is. Dus is die waarskynlikheid van om 'n skyfie wat beide 'n veelvoud van $2$ en van $6$ is, is:
 
    $0,16$
    
    \item Daar is net $1$ element in hierdie versameling, dus is die waarskynlikheid $0,01$.
    \end{enumerate}

\end{enumerate}}
\end{solutions}


\begin{eocexercises}{}
  \begin{enumerate}[itemsep=5pt, label=\textbf{\arabic*}.]
  \item 'n Groep van $45$ kinders is gevra of hulle  Frosties en/of
    Strawberry Pops eet. $31$ eet beide en $6$ eet net Frosties.  Wat is die waarskynlikheid dat 'n kind wat willekeurig gekies word net Strawberry
    Pops eet?
  \item In 'n groep van $42$ leerders het almal behalwe $3$ 'n pakkie skyfies of 'n Fanta of beide gehad. As $23$ 'n pakkie skyfies gehad het, en $7$ hiervan het ook 'n Fanta gehad, wat is die waarskynlikheid dat 'n leerder wat willekeurig gekies word die volgende het:
    \begin{enumerate}[noitemsep, label=\textbf{(\alph*)} ]
    \item beide skyfies en Fanta
    \item net Fanta
    \end{enumerate}
  \item Gebruik 'n Venn-diagram om die volgende waarskynlikhede van 'n dobbelsteen wat gerol word te bereken:
    \begin{enumerate}[noitemsep, label=\textbf{(\alph*)} ]
    \item 'n veelvoud van $5$ en 'n onewe getal
    \item 'n getal wat nie 'n veelvoud van $5$ is nie en ook nie 'n onewe getal nie
    \item 'n getal wat nie 'n veelvoud van $5$ is nie, maar onewe.
    \end{enumerate}
  \item 'n Pakkie het geel en pienk lekkers. Die waarskynlikheid om 'n pienk lekker te kies is $\frac{7}{12}$.
Wat is die waarskynlikheid om 'n geel lekker uit te haal?

  \item In 'n parkeerterrein met $300$ motors is daar $190$ Opels. Wat is die waarskynlikheid dat die eerste motor wat uitry die volgende is:
    \begin{enumerate}[noitemsep, label=\textbf{(\alph*)} ]
    \item 'n Opel
    \item nie 'n Opel nie
    \end{enumerate}
  \item Tamara het $18$ los sokkies in 'n laai. Agt van hierdie is oranje en twee pienk. Bereken die waarskynlikheid dat die eerste sokkie wat willekeurig gekies word die volgende is:
    \begin{enumerate}[noitemsep, label=\textbf{(\alph*)} ]
    \item oranje
    \item nie oranje
    \item pienk
    \item nie pienk
    \item oranje of pienk
    \item nie oranje of pienk nie
    \end{enumerate}
  \item Daar is $9$ brosbroodkoekies, $4$ gemmerkoekies,
    $11$ sjokoladekoekies en $18$ Jambos op 'n bord. As 'n koekie willekeurig gekies word, wat is die waarskynlikheid dat:
    \begin{enumerate}[noitemsep, label=\textbf{(\alph*)} ]
    \item dit 'n gemmerkoekie of 'n Jambo is
    \item dit nie 'n brosbroodkoekie is nie
    \end{enumerate}
  \item $280$ kaartjies is verkoop in 'n lotery. Ingrid het $15$ gekoop. Wat is die waarskynlikheid dat Ingrid:
    \begin{enumerate}[noitemsep, label=\textbf{(\alph*)} ]
    \item die prys wen
    \item nie die prys wen nie
    \end{enumerate}
  \item Die kinders in 'n kleuterskool is geklassifiseer per haarkleur en oogkleur. $44$ het rooi hare en nie bruin o\"e nie, $14$ het bruin o\"e maar nie rooi hare nie en $40$ het nie bruin o\"e of rooi hare nie.
    \begin{enumerate}[noitemsep, label=\textbf{(\alph*)} ]
    \item Hoeveel kinders is in die skool?
    \item Wat is die waarskynlikheid dat 'n kind wat willekeurig gekies word die volgende het:
      \begin{enumerate}[noitemsep, label=\roman*. ]
      \item bruin o\"e
      \item rooi hare
      \end{enumerate} 
    \item 'n Kind met bruin o\"e word willekeurig gekies. Wat is die waarskynlikheid dat hierdie kind rooi hare het?
    \end{enumerate}
  \item 'n Houer bevat pers, blou en swart lekkers. Die waarskynlikheid dat 'n lekker wat willekeurig gekies word pers is, is $\frac{1}{7}$, en die waarskynlikheid dat dit swart sal wees is $\frac{3}{5}$.
    \begin{enumerate}[noitemsep, label=\textbf{(\alph*)} ]
    \item As ek 'n lekker willekeurig kies, wat is die waarskynlikheid dit die volgende sal wees:
      \begin{enumerate}[noitemsep, label=\roman*. ]
      \item pers of blou
      \item swart
      \item pers
      \end{enumerate}
    \item As daar $70$ lekkers in die houer is, hoeveel perses is daar?
    \item $\frac{2}{5}$ van die pers lekkers in (b) het strepe en die res nie, hoeveel pers lekkers het strepe?
    \end{enumerate}
  \item Vir elk van die volgende, teken 'n Venn-diagram om die situasie voor te stel en vind 'n voorbeeld ter illustrasie:
    \begin{enumerate}[noitemsep, label=\textbf{(\alph*)} ]
    \item 'n monsterruimte waarin daar twee gebeurtenisse is wat nie ondeling uitsluitend is nie
    \item 'n monsterruimte waarin daar twee komplement\^ere gebeurtenisse is
    \end{enumerate}
  \item Gebruik 'n Venn-diagram om te bewys dat die waarskynlikheid van voorkoms van gebeurtenis  $A$ of $B$ ($A$ en $B$ is nie onderling uitsluitend nie) gegee word deur:
    \[P(A \cup B) = P(A) + P(B) - P(A \cap B)\]
  \item Al die klawerkaarte word uit 'n pak kaarte gehaal. Die res word dan geskommel en een kaart gekies. Die kaart word dan teruggeplaas voor die volgende een gekies word.
    \begin{enumerate}[noitemsep, label=\textbf{(\alph*)} ]
    \item Wat is die monsterruimte?
    \item Vind 'n versameling om gebeurtenis $P$, die trek van 'n prentkaart, voor te stel.
    \item Vind 'n versameling om gebeurtenis $N$, die trek van 'n nommerkaart, voor te stel.
    \item Vertoon die bostaande gebeurtenisse in 'n Venn-diagram.
    \item Wat is 'n goeie beskrywing van die versamelings $P$ en $N$?
      (Wenk: Vind enige elemente van $P$ in $N$ en van $N$ in $P$).
    \end{enumerate}

%english questions below
  \item 'n Opname was by Mutende Primêre Skool uitgevoer om vas te stel hoeveel van die $ 650 $ leerders vetkoek, en hoeveel lekkers tydens pouse koop. Die volgende was bevind:
% A survey was conducted at Mutende Secondary school to establish how many of the $650$ learners buy vetkoek and how many buy sweets during break. The following was found:
\begin{itemize}
 \item $50$ leerders het niks gekoop nie
\item $400$ leerders het vetkoek gekoop
\item $300$ leerders het lekkers gekoop
\end{itemize}
\begin{enumerate}[noitemsep, label=\textbf{(\alph*)} ]
 \item Vertoon hierdie inligting met 'n Venn-diagram
\item Indien 'n leerder willekeurig gekies is, bereken die waarskynlikheid dat die leerder die volgende koop:
% If a learner is chose randomly, calculate the probability that this learner buys:
\begin{enumerate}[noitemsep, label=\roman*. ]
 \item net lekkers
\item net vetkoek
\item nie vetkoek of lekkers nie
\item vetkoek en lekkers
\item vetkoek of lekkers
\end{enumerate}
\end{enumerate}
\item In 'n opname by Lwandani se Sekondêre Skool is $80$ mense gevra of hulle die Sowetan of die Daily Sun of albei koerante lees. Die verslag toondat $45$ mense die Daily Sun lees, $30$ die Sowetan lees en $10$ lees nie een van die twee nie. Gebruik 'n Venn-diagram om die persentasie van mense te vind wat:
\begin{enumerate}[noitemsep, label=\textbf{(\alph*)} ]
 \item net die Daily Sun lees
\item net die Sowetan lees
\item die Daily Sun en die Sowetan lees
\end{enumerate}

  \end{enumerate}
% \practiceinfo
%   \begin{tabularx}{\textwidth}{XXXXX}
%     (1.) AAA & (2.) AAA & (3.) AAA& (4.) AAA& (5.) AAA\\
%     (6.) AAA& (7.) AAA& (8.) AAA& (9.) AAA& (10.) AAA\\
%     (11.) AAA& (12.) AAA& (13.) AAA \\
%   \end{tabularx}
\end{eocexercises}


 \begin{eocsolutions}{}{
\begin{enumerate}[itemsep=5pt, label=\textbf{\arabic*}. ] 


\item %solution 1

$45($Alles$) - 6($net Frosties$) - 31 ($albei$) = 8 ($net Strawberry Pops$)$\\
Dus $\frac{8}{45}=0,18$

\item %solution 2
    \begin{enumerate}[noitemsep, label=\textbf{(\alph*)} ]
    \item $\frac{7}{42} = \frac{1}{6}$

    \item  $42 - 3 = 39$ het te minste een gehad het, en $23 + 7$ het 'n pakkie skyfies gehad het, dan $39 - 30 = 9$ het net Fanta gehad het.\\
    $\frac{9}{42} = \frac{3}{14}$
    \end{enumerate}

\item %solution 3
	    \scalebox{0.8} % Change this value to rescale the drawing.
	    {
	    \begin{pspicture}(0,-1.8767188)(3.62,1.9167187)
	    \pscircle[linewidth=0.04,dimen=outer](1.81,-0.06671875){1.81}
	    \pscircle[linewidth=0.04,dimen=outer](1.08,-0.03671875){0.94}
	    \pscircle[linewidth=0.04,dimen=outer](2.01,1.2132813){0.45}
	    \usefont{T1}{ppl}{m}{n}
	    \rput(3.1245313,1.7132813){\LARGE$S$}
	    \usefont{T1}{ppl}{m}{n}
	    \rput(1.0045313,0.5132812){\LARGE$2$}
	    \usefont{T1}{ppl}{m}{n}
	    \rput(0.6045312,0.03328125){\LARGE$4$}
	    \usefont{T1}{ppl}{m}{n}
	    \rput(1.3245312,-0.34671876){\LARGE$6$}
	    \usefont{T1}{ppl}{m}{n}
	    \rput(2.8445313,-0.50671875){\LARGE$3$}
	    \usefont{T1}{ppl}{m}{n}
	    \rput(1.9845313,1.1732812){\LARGE$5$}
	    \usefont{T1}{ppl}{m}{n}
	    \rput(2.7645311,0.03328125){\LARGE$1$}
	    \pscircle[linewidth=0.04,dimen=outer](2.81,-0.20671874){0.69}
	    \end{pspicture}
	    }\\
    Veelvoude van $5 :5$\\
    Onewe getal: $1, 3, 5$\\
    Nie een: $2, 4, 6$\\
    Albei: $5$
    \begin{enumerate}[noitemsep, label=\textbf{(\alph*)} ]
    \item $\frac{1}{6}$
    \item $\frac{3}{6} = \frac{1}{2}$
    \item $\frac{2}{6} = \frac{1}{3}$
    \end{enumerate}
\item %solution 4\\
$1 - \frac{7}{12} = \frac{5}{12}$
\item %solution 5
    \begin{enumerate}[noitemsep, label=\textbf{(\alph*)} ]
    \item $\frac{190}{300} = \frac{19}{30}$
    \item $1 - \frac{19}{30} = \frac{11}{30}$
    \end{enumerate}
\item %solution 6
    \begin{enumerate}[noitemsep, label=\textbf{(\alph*)} ]
    \item $\frac{8}{18} = \frac{4}{9}$
    \item $1 - \frac{4}{9} = \frac{5}{9}$
    \item $\frac{2}{18} = \frac{1}{9}$
    \item $1 - \frac{1}{9} = \frac{8}{9}$
    \item $\frac{1}{9} + \frac{4}{9} = \frac{5}{9}$
    \item $1 - \frac{5}{9} = \frac{4}{9}$
    \end{enumerate}
\item %solution 7
    \begin{enumerate}[noitemsep, label=\textbf{(\alph*)} ]
    \item Totale getal koekies is $9 + 4 + 11 + 18 = 42$\\
    $\frac{4}{42} + \frac{18}{42} = \frac{22}{42}$\\
    $=\frac{11}{21}$
    \item $1 - \frac{9}{42} = 1 - \frac{3}{14}$\\
    $=\frac{11}{14}$
    \end{enumerate}
\item %solution 8
    \begin{enumerate}[noitemsep, label=\textbf{(\alph*)} ]
    \item $\frac{15}{280} = \frac{3}{56}$
    \item $1 - \frac{3}{56} = \frac{53}{56}$
    \end{enumerate}
\item %solution 9
    \begin{enumerate}[noitemsep, label=\textbf{(\alph*)} ]
    \item Elke $4$ groepe is wedersyds uitsluitend, dus die totale getal kinders is $44 + 14 + 5 + 40 = 103$.
    \item
	\begin{enumerate}[itemsep=1pt,  label=\textbf{\roman*}. ]
	\item $\frac{19}{103}$
	\item $\frac{58}{103}$
	\end{enumerate}
    \item $\frac{14}{(14 + 5)} = \frac{14}{19}$
    \end{enumerate}
\item %solution 10
    \begin{enumerate}[noitemsep, label=\textbf{(\alph*)} ]
    \item 
	\begin{enumerate}[itemsep=1pt,  label=\textbf{\roman*}. ]
	\item Dieselfde as nie swart nie: $1 - \frac{3}{5} = \frac{2}{5}$
	\item $\frac{3}{5}$
	\item $\frac{1}{7}$
	\end{enumerate}
    \item $\frac{1}{7} \times 70 = 10$
    \item $10 \times \frac{2}{5} = 4$
    \end{enumerate}
\item %solution 11
    \begin{enumerate}[noitemsep, label=\textbf{(\alph*)} ]
    \item
    			\scalebox{0.8} % Change this value to rescale the drawing.
			{
			\begin{pspicture}(0,-1.8767188)(3.62,1.9167187)
			\pscircle[linewidth=0.04,dimen=outer](1.81,-0.06671875){1.81}
			\pscircle[linewidth=0.04,dimen=outer](1.2,-0.07671875){0.94}
			\usefont{T1}{ppl}{m}{n}
			\rput(3.1245313,1.7132813){\LARGE$S$}
			\pscircle[linewidth=0.04,dimen=outer](2.45,-0.06671875){0.87}
			\end{pspicture}
			}
    \item
			\scalebox{0.8} % Change this value to rescale the drawing.
			{
			\begin{pspicture}(0,-1.8767188)(3.62,1.9167187)
			\pscircle[linewidth=0.04,dimen=outer](1.81,-0.06671875){1.81}
			\usefont{T1}{ppl}{m}{n}
			\rput(3.1245313,1.7132813){\LARGE$S$}
			\psline[linewidth=0.04cm](2.78,1.4032812)(0.88,-1.5767188)
			\end{pspicture}
			}
    \end{enumerate}

\item %solution 12

\scalebox{0.8} % Change this value to rescale the drawing.
{
    \begin{pspicture}(-2,-2)(3.5,2)
    \usefont{T1}{ppl}{m}{n}
    \rput(0.0,0.0){\LARGE$A$ not $B$}
    \rput(1.0,2.3){\LARGE$A$}
    \rput(3.0,2.3){\LARGE$B$}
    \pscircle[linewidth=0.04,dimen=outer](1.0,0.0){2.0}
    \usefont{T1}{ppl}{m}{n}
    \rput(4.0,0.0){\LARGE$B$ not $A$}
    \pscircle[linewidth=0.04,dimen=outer](3,0.0){2.0}
    \usefont{T1}{ppl}{m}{n}
    \rput(2.0,0.0){\LARGE$A$ and $B$}
    \end{pspicture}
}
\item 
    \begin{enumerate}[noitemsep, label=\textbf{(\alph*)} ]
    \item $\{$pak kaarte sonder die klawers$\}$
    \item $P = \{J$; $Q$; $K$ van harte, diamante of skoppens$\}$
    \item $N = \{A$; $2$; $ 3$; $ 4$; $ 5$; $ 6$; $ 7$; $ 8$; $ 9$; $ 10$ van harte, diamante of skoppens$\}$
    \item
	      \scalebox{0.7} % Change this value to rescale the drawing.
			{
			\begin{pspicture}(0,-2.4442186)(8.7890625,2.4442186)
			\pscircle[linewidth=0.04,dimen=outer](2.0654688,0.26078126){1.37}
			\pscircle[linewidth=0.04,dimen=outer](5.905469,-0.17921875){2.21}
			\usefont{T1}{ptm}{m}{n}
			\rput(1.4245312,1.1407813){\LARGE$J\diamondsuit$}
			\usefont{T1}{ptm}{m}{n}
			\rput(1.9445312,0.50078124){\LARGE$Q\diamondsuit$}
			\usefont{T1}{ptm}{m}{n}
			\rput(2.5845313,-0.07921875){\LARGE$K\diamondsuit$}
			\usefont{T1}{ptm}{m}{n}
			\rput(2.1845312,-0.61921877){\LARGE$K\heartsuit$}
			\usefont{T1}{ptm}{m}{n}
			\rput(1.4645312,-0.33921874){\LARGE$K\spadesuit$}
			\usefont{T1}{ptm}{m}{n}
			\rput(2.9845314,0.68078125){\LARGE$Q\heartsuit$}
			\usefont{T1}{ptm}{m}{n}
			\rput(2.4,1.2807813){\LARGE$Q\spadesuit$}
			\usefont{T1}{ptm}{m}{n}
			\rput(1.3845313,0.7){\LARGE$J\heartsuit$}
			\usefont{T1}{ptm}{m}{n}
			\rput(1.1045313,0.22078125){\LARGE$J\spadesuit$}
			\usefont{T1}{ptm}{m}{n}
			\rput(4.924531,1.5207813){\LARGE$A\diamondsuit$}
			\usefont{T1}{ptm}{m}{n}
			\rput(6.374531,-1.2992188){\LARGE$2\diamondsuit$}
			\usefont{T1}{ptm}{m}{n}
			\rput(5.6345315,-1.6592188){\LARGE$3\diamondsuit$}
			\usefont{T1}{ptm}{m}{n}
			\rput(4.174531,-0.95921874){\LARGE$4\diamondsuit$}
			\usefont{T1}{ptm}{m}{n}
			\rput(4.494531,0.48078126){\LARGE$5\diamondsuit$}
			\usefont{T1}{ptm}{m}{n}
			\rput(6.7345314,1.4807812){\LARGE$6\diamondsuit$}
			\usefont{T1}{ptm}{m}{n}
			\rput(6.2145314,1.7607813){\LARGE$7\diamondsuit$}
			\usefont{T1}{ptm}{m}{n}
			\rput(5.334531,-0.33921874){\LARGE$8\diamondsuit$}
			\usefont{T1}{ptm}{m}{n}
			\rput(7.214531,0.06078125){\LARGE$9\diamondsuit$}
			\usefont{T1}{ptm}{m}{n}
			\rput(5.9245315,0.30078125){\LARGE$10\diamondsuit$}
			\usefont{T1}{ptm}{m}{n}
			\rput(4.824531,1.1007812){\LARGE$A\heartsuit$}
			\usefont{T1}{ptm}{m}{n}
			\rput(4.684531,0.82078123){\LARGE$A\spadesuit$}
			\usefont{T1}{ptm}{m}{n}
			\rput(5.774531,1.1607813){\LARGE$7\heartsuit$}
			\usefont{T1}{ptm}{m}{n}
			\rput(6.554531,0.98078126){\LARGE$7\spadesuit$}
			\usefont{T1}{ptm}{m}{n}
			\rput(7.394531,0.48078126){\LARGE$6\heartsuit$}
			\usefont{T1}{ptm}{m}{n}
			\rput(6.854531,0.68078125){\LARGE$6\spadesuit$}
			\usefont{T1}{ptm}{m}{n}
			\rput(7.5745316,-0.35921875){\LARGE$9\heartsuit$}
			\usefont{T1}{ptm}{m}{n}
			\rput(7.2345314,-0.67921877){\LARGE$9\spadesuit$}
			\usefont{T1}{ptm}{m}{n}
			\rput(6.124531,-0.11921875){\LARGE$10\heartsuit$}
			\usefont{T1}{ptm}{m}{n}
			\rput(6.284531,-0.63921875){\LARGE$10\spadesuit$}
			\usefont{T1}{ptm}{m}{n}
			\rput(4.6945314,0.12078125){\LARGE$5\heartsuit$}
			\usefont{T1}{ptm}{m}{n}
			\rput(4.134531,-0.23921876){\LARGE$5\spadesuit$}
			\usefont{T1}{ptm}{m}{n}
			\rput(4.774531,-0.63921875){\LARGE$8\heartsuit$}
			\usefont{T1}{ptm}{m}{n}
			\rput(5.454531,-1.0392188){\LARGE$8\spadesuit$}
			\usefont{T1}{ptm}{m}{n}
			\rput(4.7145314,-1.2592187){\LARGE$4\heartsuit$}
			\usefont{T1}{ptm}{m}{n}
			\rput(4.6945314,-1.5592188){\LARGE$4\spadesuit$}
			\usefont{T1}{ptm}{m}{n}
			\rput(5.2745314,-2.0192187){\LARGE$3\heartsuit$}
			\usefont{T1}{ptm}{m}{n}
			\rput(6.014531,-2.0){\LARGE$3\spadesuit$}
			\usefont{T1}{ptm}{m}{n}
			\rput(7.2745314,-1.0792187){\LARGE$2\heartsuit$}
			\usefont{T1}{ptm}{m}{n}
			\rput(6.7745314,-1.7792188){\LARGE$2\spadesuit$}
			\usefont{T1}{ptm}{m}{n}
			\rput(2.0845313,1.9007813){\LARGE$P$}
			\usefont{T1}{ptm}{m}{n}
			\rput(5.974531,2.2407813){\LARGE$N$}
			\end{pspicture} 
			}
\item Wedersyds uitsluitend en komplementêre.
 
    \end{enumerate}
\item 
\begin{enumerate}[itemsep=5pt, label=\textbf{(\alph*)} ]
 \item 
\scalebox {0.8} % Change this value to rescale the drawing.
{
\begin{pspicture}(0,-1.97)(5.0,1.97)
\psframe[linewidth=0.04,dimen=outer](5.0,1.97)(0.0,-1.97)
\pscircle[linewidth=0.04,dimen=outer](1.78,0.35){1.08}
\pscircle[linewidth=0.04,dimen=outer](3.0,-0.25){1.0}
\usefont{T1}{ptm}{m}{n}
\rput(4.365,1.34){50}
\usefont{T1}{ptm}{m}{n}
\rput(1.59625,0.2){300}
\usefont{T1}{ptm}{m}{n}
\rput(3.4796875,-0.18){200}
\usefont{T1}{ptm}{m}{n}
\rput(2.3825,-0.04){100}
\usefont{T1}{ptm}{m}{n}
\rput(1.84625,1.12){V}
\usefont{T1}{ptm}{m}{n}
\rput(3.4192188,-0.88){S}
\end{pspicture} 
}
\item %If a learner is chose randomly, calculate the probability that this learner buys:
\begin{enumerate}[noitemsep, label=\textbf{(\roman*)} ]
\item $\frac{200}{650} = 30,8\%$
\item $\frac{300}{650} = 46,2\%$
\item $\frac{50}{650} = 7,7\%$
\item $\frac{100}{650} = 15,4\%$
\item $\frac{600}{650} = 92,3\%$
\end{enumerate}
\end{enumerate}
\item 
\scalebox{0.8} % Change this value to rescale the drawing.
{
\begin{pspicture}(0,-1.97)(5.0,1.97)
\psframe[linewidth=0.04,dimen=outer](5.0,1.97)(0.0,-1.97)
\pscircle[linewidth=0.04,dimen=outer](1.78,0.35){1.08}
\pscircle[linewidth=0.04,dimen=outer](3.0,-0.25){1.0}
\usefont{T1}{ptm}{m}{n}
\rput(4.3525,1.34){10}
\usefont{T1}{ptm}{m}{n}
\rput(1.5107813,0.2){40}
\usefont{T1}{ptm}{m}{n}
\rput(3.3842187,-0.18){25}
\usefont{T1}{ptm}{m}{n}
\rput(2.3895311,0.04){5}
\usefont{T1}{ptm}{m}{n}
\rput(1.7873437,1.02){D/S}
\usefont{T1}{ptm}{m}{n}
\rput(3.1070313,-0.84){D/B}
\end{pspicture} 
}
\begin{enumerate}[noitemsep, label=\textbf{(\alph*)} ]
 \item $\frac{40}{80} = 50\%$
\item $\frac{25}{80} = 31,25\%$
\item $\frac{5}{80} = 6,25\%$
\end{enumerate}
\end{enumerate}}
\end{eocsolutions}


