\chapter{Probability}

\begin{exercises}{}
{
  \begin{enumerate}[itemsep=5pt, label=\textbf{\arabic*}.]
  \item 'n Sak bevat $6$ rooi, $3$ blou, $2$ groen en $1$ wit
    balle. 'n Ball word lukraak gekies. Wat is die waarskynlikheid van die volgende:
    \begin{enumerate}
    \item rooi
    \item blou of wit
    \item nie groen
    \item nie groen of rooi
    \end{enumerate}
  \item 'n Speelkaart word lukraak gekies uit 'n pak van $52$ kaarte. Wat is die waarskynlikheid van die volgende:
    \begin{enumerate}

    \item die $2$ van harte
    \item 'n rooi kaart
    \item 'n prentkaart
    \item 'n 'Ace' getal
    \item 'n nommer kleiner as $4$
    \end{enumerate}

  \item Ewe getalle in die interval $2$ tot $100$ word op kaarte geskryf. Wat is
    die waarskynlikheid om 'n veelvoud van $5$ op 'n lukraak wyse te kies?
  \end{enumerate}
}
\end{exercises}


 \begin{solutions}{}{
\begin{enumerate}[itemsep=5pt, label=\textbf{\arabic*}. ] 
\item %solution 1
    \begin{enumerate}[noitemsep, label=\textbf{(\alph*)} ]
    \item $\frac{6}{12} &=& \frac{1}{2}$
    \item $\frac{(3 + 1)}{12} &=& \frac{1}{3}$
    \item $1 - (\frac{2}{12}) &=& \frac{5}{6}$
    \item $1 - \frac{(2 + 6)}{12} &= \frac{4}{12}\\
				  &= \frac{1}{3}$Bcc   
    \end{enumerate}
\item %solution 2
    \begin{enumerate}[noitemsep, label=\textbf{(\alph*)} ]
    \item $\frac{1}{52}$ (only one in the deck)
    \item $\frac{1}{2}$ (half the cards are red, half are black)
    \item $\frac{3}{13}$ (for each suite of 13 cards, there are three picture cards: J, Q, K)
    \item $\frac{4}{52} = \frac{1}{13}$ (four aces in the deck)
    \item $\frac{3}{13}$ (for each suite of 13 cards, there are three cards less than 4: A, 2 and 3)
    \end{enumerate}
\item %solution 3
    There are 50 cards.  They are all even.\\
    All even numbers that are also multiples of 5 are multiples of 10 (10, 20,...., 100).\\
    There are 10 of them.\\
    Therefore, the probability is $\frac{10}{50} = \frac{1}{5}$.
\end{enumerate}}
\end{solutions}


% \section{Relative frequency}
% \section{Venn diagrams}
\begin{exercises}{}
{
  \begin{enumerate}[itemsep=5pt, label=\textbf{\arabic*}.]
  \item  Laat $S$ die versameling heelgetalle van $1$ tot $16$ aandui, $X$
    die versameling ewe getalle van $1$ tot $16$ en $Y$ die versameling priemgetalle van $1$ tot $16$
    \begin{enumerate}[noitemsep, label=\textbf{(\alph*)} ]
    \item Teken 'n Venn-diagram wat $S$, $X$ and $Y$ akkuraat aandui.
    \item Vind $n\left(S\right)$, $n\left(X\right)$, $n\left(Y\right)$,
      $n\left(X\cup Y\right)$, $n\left(X\cap Y\right)$.
    \end{enumerate}
  \item Daar is $79$ Graad $10$ leerders in die skool. Almal neem Wiskunde, Aardrykskunde of Geskiedenis. $41$ neem Aardrykskunde, $36$ neem Geskiedenis en $30$ neem Wiskunde. $16$ neem Wiskunde en Geskiedenis, $6$ neem Aardrykskunde en Geskiedenis, $8$ neem net Wiskunde en daar is $16$ wat slegs Geskiedenis neem.
    \begin{enumerate}[noitemsep, label=\textbf{(\alph*)} ]
    \item Teken 'n Venn-dagram om hierdie inligting te illustreer.
    \item Hoeveel leerders neem Wiskunde en Aardrykskunde, maar nie Geskiedenis nie?
    \item Hoeveel leerders neem net Aardrykskunde?
    \item Hoeveel leerders neem al drie vakke?
    \end{enumerate}
  \item Papiere word genommer van $1$ tot $12$ en in 'n houer geplaas en die houer geskud. Een papier op 'n slag word getrek en weer teruggeplaas.
    \begin{enumerate}[noitemsep, label=\textbf{(\alph*)} ]
    \item Wat is die monsterruimte, $S$?
    \item Skryf die versameling $A$ neer, wat die gebeurtenis voorstel van dietrek van 'n papier met 'n faktor van $12$.
    \item Skryf die versameling $B$ neer, wat die gebeuertenis voorstel van dietrek 'n papier, gemerk met 'n priemgetal.
    \item Gebruik 'n Venn-diagram om $A$, $B$ and $S$ voor te stel.
    \item Vind
      \begin{enumerate}[noitemsep, label=\textbf{\roman*.} ]
      \item $n\left(S\right)$
      \item $n\left(A\right)$
      \item $n\left(B\right)$

      \end{enumerate}

    \end{enumerate}
 \end{enumerate}
% \practiceinfo
%   \begin{tabularx}{\textwidth}{XXXX}
%     (1.) AAA & (2.) AAA & (3.) AAA \\
%   \end{tabularx}
}
\end{exercises}


 \begin{solutions}{}{
\begin{enumerate}[itemsep=5pt, label=\textbf{\arabic*}. ] 
\item %solution 1
	  \scalebox{0.8} % Change this value to rescale the drawing.
	  {
	  \begin{pspicture}(0,-2.2)(4.5590625,2.2)
	  \pscircle[linewidth=0.04,dimen=outer](2.2,0.0){2.2}
	  \pscircle[linewidth=0.04,dimen=outer](1.45,-0.03){1.29}
	  \pscircle[linewidth=0.04,dimen=outer](2.82,0.48){1.18}
	  \usefont{T1}{ppl}{m}{n}
	  \rput(4.184531,1.89){\LARGE$S$}
	  \usefont{T1}{ppl}{m}{n}
	  \rput(2.1845312,0.27){\LARGE$2$}
	  \usefont{T1}{ppl}{m}{n}
	  \rput(1.2445313,0.89){\LARGE$4$}
	  \usefont{T1}{ppl}{m}{n}
	  \rput(0.64453125,0.51){\LARGE$6$}
	  \usefont{T1}{ppl}{m}{n}
	  \rput(1.3245312,0.19){\LARGE$8$}
	  \usefont{T1}{ppl}{m}{n}
	  \rput(0.6745312,-0.15){\LARGE$10$}
	  \usefont{T1}{ppl}{m}{n}
	  \rput(2.6045313,1.19){\LARGE$3$}
	  \usefont{T1}{ppl}{m}{n}
	  \rput(3.3245313,1.13){\LARGE $5$}
	  \usefont{T1}{ppl}{m}{n}
	  \rput(2.9645312,0.63){\LARGE$7$}
	  \usefont{T1}{ppl}{m}{n}
	  \rput(2.9045312,-1.41){\LARGE$9$}
	  \usefont{T1}{ppl}{m}{n}
	  \rput(1.0045313,1.5){\LARGE$X$}
	  \usefont{T1}{ppl}{m}{n}
	  \rput(2.4945312,1.9){\LARGE$Y$}
	  \usefont{T1}{ppl}{m}{n}
	  \rput(0.75453126,-0.59){\LARGE$12$}
	  \usefont{T1}{ppl}{m}{n}
	  \rput(1.7545313,-0.61){\LARGE$14$}
	  \usefont{T1}{ppl}{m}{n}
	  \rput(1.3945312,-1.03){\LARGE$16$}
	  \usefont{T1}{ppl}{m}{n}
	  \rput(3.5145311,0.35){\LARGE$11$}
	  \usefont{T1}{ppl}{m}{n}
	  \rput(3.2545311,-0.17){\LARGE$13$}
	  \usefont{T1}{ppl}{m}{n}
	  \rput(1.8445313,-1.69){\LARGE$1$}
	  \usefont{T1}{ppl}{m}{n}
	  \rput(3.6745312,-0.87){\LARGE$15$}
	  \end{pspicture} 
	  }
\item%solution 2
    \begin{enumerate}[noitemsep, label=\textbf{(\alph*)} ]
    \item
		      \scalebox{0.8} % Change this value to rescale the drawing.
		      {
		      \begin{pspicture}(0,-2.393125)(4.7790623,2.393125)
		      \pscircle[linewidth=0.04,dimen=outer](1.64,0.7196875){1.18}
		      \pscircle[linewidth=0.04,dimen=outer](1.16,-0.8203125){1.16}
		      \pscircle[linewidth=0.04,dimen=outer](2.64,-0.2603125){1.18}
		      \usefont{T1}{ppl}{m}{n}
		      \rput(4.3,0.6096875){\LARGE$G:41$}
		      \usefont{T1}{ppl}{m}{n}
		      \rput(0.77453125,-2.2){\LARGE$H:36$}
		      \usefont{T1}{ppl}{m}{n}
		      \rput(1.9145312,2.1896875){\LARGE$M:30$}
		      \usefont{T1}{ptm}{m}{n}
		      \rput(0.81453127,-1.1503125){\LARGE$16$}
		      \usefont{T1}{ptm}{m}{n}
		      \rput(1.7445313,1.3096875){\LARGE$8$}
		      \usefont{T1}{ptm}{m}{n}
		      \rput(1.9245312,-0.7503125){\LARGE$6$}
		      \usefont{T1}{ptm}{m}{n}
		      \rput(1.1945312,-0.0303125){\LARGE$16$}
		      \end{pspicture} 
		      }

    \item Each student must do exactly one of the following:
    \begin{itemize}
	\item Take only geography;
	\item Only take maths and/or history;
    \end{itemize}
    There are $30 + 36 -16 =50$ students doing the second one, therefore the must be $79 - 50 = 29$ only doing geography.\\
    \newline
    Each student must do exactly one of:
    \begin{itemize}
	\item Only take geography;
	\item Only take maths;
	\item Take history;
	\item Take geography and maths, but not history;
    \end{itemize}
    There are $29, 8$ and $36$ of the first three, so the answer to B is:\\
    $79 - 29 -8 - 36 = 6$ people

    \item Calculated already: $29$

    \item Each student must do exactly one of:
    \begin{itemize}
	\item Do geography
	\item Only do maths
	\item Only do history
	\item Do maths and history but not geography
    \end{itemize}
    Using the same method as before, the number of people in the last group is:\\
 
    $79 - 41 - 8 - 16 = 14$\\

    But, $16$ people do maths and history, so there must be $16 - 14=2$ people who do all three.
    \end{enumerate}

\item %solution 3
    \begin{enumerate}[noitemsep, label=\textbf{(\alph*)} ]
    \item $S = \{1; 2;\ldots; 12\}$
    
    \item $A = \{1; 2; 3; 4; 6; 12\}$

    \item $B = \{2; 3; 5; 7; 11\}$

    \item

    		  \scalebox{0.8} % Change this value to rescale the drawing.

		  {
		  \begin{pspicture}(0,-2.2)(4.5590625,2.2)
		  \pscircle[linewidth=0.04,dimen=outer](2.2,0.0){2.2}
		  \pscircle[linewidth=0.04,dimen=outer](1.45,-0.03){1.29}
		  \pscircle[linewidth=0.04,dimen=outer](2.82,0.48){1.18}
		  \usefont{T1}{ppl}{m}{n}
		  \rput(4.184531,1.89){\LARGE $S$}
		  \usefont{T1}{ppl}{m}{n}
		  \rput(2.0045311,0.65){\LARGE$2$}
		  \usefont{T1}{ppl}{m}{n}
		  \rput(0.8645313,0.19){\LARGE$4$}
		  \usefont{T1}{ppl}{m}{n}
		  \rput(1.4645313,-0.27){\LARGE$6$}
		  \usefont{T1}{ppl}{m}{n}
		  \rput(1.9245312,-1.67){\LARGE$8$}
		  \usefont{T1}{ppl}{m}{n}
		  \rput(3.5345314,-1.03){\LARGE$10$}
		  \usefont{T1}{ppl}{m}{n}
		  \rput(2.2245312,0.07){\LARGE$3$}
		  \usefont{T1}{ppl}{m}{n}
		  \rput(2.7645311,1.17){\LARGE$5$}
		  \usefont{T1}{ppl}{m}{n}
		  \rput(3.3045313,0.57){\LARGE$7$}
		  \usefont{T1}{ppl}{m}{n}
		  \rput(2.8045313,-1.47){\LARGE$9$}
		  \usefont{T1}{ppl}{m}{n}
		  \rput(0.9945313,1.41){\LARGE$A$}
		  \usefont{T1}{ppl}{m}{n}
		  \rput(2.4945312,1.83){\LARGE$B$}
		  \usefont{T1}{ppl}{m}{n}
		  \rput(1.6145313,-0.95){\LARGE$12$}
		  \usefont{T1}{ppl}{m}{n}
		  \rput(3.1945312,-0.01){\LARGE$11$}
		  \usefont{T1}{ppl}{m}{n}
		  \rput(0.9845312,0.77){\LARGE$1$}
		  \end{pspicture} 
		  }

    \item
	\begin{enumerate}[noitemsep, label=\textbf{\roman*.} ]
	\item $12$
	\item $6$
	\item $5$
	
	\end{enumerate}
    

    \end{enumerate}
\end{enumerate}}
\end{solutions}


% \section{Union and intersection}
% \section{Probability identities}
% \section{Mutually exclusive events}
% \section{Complementary events}
\begin{exercises}{}
{
  \begin{enumerate}[itemsep=3pt, label=\textbf{\arabic*}.]
  \item  'n Kas bevat gekleurde blokkies. Die aantal van elke kleur word in die onderstaande tabel aangedui.

  \begin{center}
      \begin{tabular}{|l|c|c|c|c|}
        \hline
        \textbf{Kleur} & Pers & Oranje & Wit & Pink \\ \hline
       
        \textbf{Aantal blokkies} & $24$ & $32$ & $41$ & $19$ \\ \hline
   
      \end{tabular}
    \end{center}
    'n Blokkie word willekeurig gekies. Wat is die waarskynlikheid dat die blokkie die volgende kleur sal wees:
    \begin{enumerate}[noitemsep, label=\textbf{(\alph*)} ]
    \item pers
    \item pers of wit
    \item pienk en oranje
    \item nie oranje?
    \end{enumerate}

  \item 'n Klein skooltjie het 'n klas met kinders van verskillende ouderdomme. Die tabel gee die aantal kinders van elke ouderdom in die klas.

    \begin{center}
      \begin{tabular}{|l|c|c|c|}
        \hline
               & $3$ jaar oud & $4$ jaar oud & $5$ jaar oud \\\hline
   
        \textbf{manlik}   & $2$ & $7$ & $6$ \\\hline
        \textbf{vroulik} & $6$ & $5$ & $4$ \\\hline
       
      \end{tabular}
    \end{center}

    As 'n leerder lukraak gekies word, wat is die waarskynlikheid dat die leerder die volgende sal wees:
    \begin{enumerate}[noitemsep, label=\textbf{(\alph*)} ]
    \item vroulik
    \item 'n $4$ jaar oud en manlik 
    \item $3$ of $4$ jaar oud
    \item $3$ en $4$ jaar oud
    \item nie $5$
    \item $3$ of vroulik?
    \end{enumerate}

  \item Fiona het $85$ skyfies,genommer van $1$ to
    $85$. Indien 'n skyfie willekeurig gekies word, wat is die waarskynlikheid dat die nommer: 
    \begin{enumerate}\begin{enumerate}[noitemsep, label=\textbf{(\alph*)} ]
    \item eindig met $5$
    \item vermenigvuldig kan word met $3$
    \item vermenigvuldig kan word met  $6$
    \item $65$ is
    \item nie 'n veelvoud van $5$ is nie
    \item 'n veelvoud van $4$ of $3$ is
    \item 'n veelvoud van $2$ en $6$ is
    \item  $1$ is?
    \end{enumerate}
  \end{enumerate}

% \practiceinfo
%   \begin{tabularx}{\textwidth}{XXXX}
%     (1.) AAA & (2.) AAA & (3.) AAA \\
%   \end{tabularx}
}
\end{exercises}


\begin{solutions}{}{
\begin{enumerate}[itemsep=5pt, label=\textbf{\arabic*}. ] 

\item %solution 1
    Before we answer the questions we first work out how many blocks there are in total. This gives us the sample space.\\
    $n(S) = 24 + 32 + 41 + 19$\\
	 $= 116$
    \begin{enumerate}[noitemsep, label=\textbf{(\alph*)} ]

    \item The probability that a block is purple is:\\
    $P(\mbox{purple}) = \frac{n(E)}{n(S)}\\
     P(\mbox{purple}) = \frac{24}{116}\\
     P(\mbox{purple}) = 0,21$

    \item The probability that a block is either purple or white is:\\
    $P(\mbox{purple} \cup \mbox{white}) = P(\mbox{purple}) + P(\mbox{white}) - P(\mbox{purple} \cap \mbox{white})\\
    = \frac{24}{116} + \frac{41}{116} - \frac{24}{116} \times \frac{41}{116}\\
    = 0,64$

    \item Since one block cannot be two colours the probability of this event is $0$.

    \item We first work out the probability that a block is orange:\\
    $P(\mbox{orange}) = \frac{32}{116}
	       = 0,28$\\
    The probability that a block is not orange is:\\
    $P(\mbox{not orange}) = 1 - 0,28\\
		   = 0,72$
    \end{enumerate}
    
\item %solution 2
We calculate the total number of pupils at the school:\\
$6 + 2 + 5 + 7 + 4 + 6 = 30$
    \begin{enumerate}[noitemsep, label=\textbf{(\alph*)} ]
    \item The total number of female children is $6 + 5 + 4 = 15$.\\
    The probability of a randomly selected child being female is:\\
    $P(\mbox{female}) = \frac{n(E)}{n(S)}\\
     P(\mbox{female}) = \frac{15}{30}\\
     P(\mbox{female}) = 0,5$

    \item The probability of a randomly selected child being a 4 year old male is:\\
    $P(\mbox{male}) = \frac{7}{30}\\
	      = 0,23$

    \item There are $6 + 2 + 5 + 7 = 20$ children aged 3 or 4.\\
    The probability of a randomly selected child being either 3 or 4 is:\\
    $\frac{20}{30} = 0,67$

    \item A child cannot be both 3 and 4, so the probability is $0$.

    \item This is the same as a randomly selected child being either 3 or 4 and so is $0,67$.

    \item The probability of a child being either 3 or female is:\\
    $P(3 \cup \mbox{female}) = P(3) + P(\mbox{female}) - P(3 \cap \mbox{female})\\
		      = \frac{10}{30} + 0,5 - \frac{10}{30} \times \frac{15}{30}\\
		      = 0,67$
    \end{enumerate}
    
\item %solution 3
    \begin{enumerate}[noitemsep, label=\textbf{(\alph*)} ]
    \item The set of all discs ending with $5$ is: $\{5$; $15$; $25$; $35$; $45$; $55$; $65$; $75$; $85\}$. This has $9$ elements.\\
 
    The probability of drawing a disc that ends with $5$ is:\\
    $P(5) = \frac{n(E)}{n(S)}\\
     P(5) = \frac{9}{85}\\
     P(5) = 0,11$

    \item The set of all discs that are multiples of $3$ is: $\{3$; $6$; $ 9$; $ 12$; $ 15$; $ 18$; $ 21$; $ 24$; $ 27$; $ 30$; $ 33$; $ 36$; $ 39$; $ 42$; $ 45$; $ 48$; $ 51$; $ 54$; $ 57$; $ 60$; $ 63$; $ 66$; $ 69$; $ 72$; $ 75$; $ 78$; $ 81$; $ 84\}$. This has $28$ elements.\\
 
    The probability of drawing a disc that is a multiple of $3$ is:
    $P(3_m) = \frac{28}{85}
	    = 0,33$

    \item The set of all discs that are multiples of $6$ is: $\{6$; $ 12$; $ 18$; $ 24$; $ 30$; $ 36$; $ 42$; $ 48$; $ 54$; $ 60$; $ 66$; $ 72$; $ 78$; $ 84\}$. This set has $14$ elements.
    The probability of drawing a disc that is a multiple of $6$ is:\\
    $P(6_m) = \frac{14}{85}\\
	    = 0,16$

    \item There is only one element in this set and so the probability of drawing $65$ is:\\
    $P(65) = \frac{1}{85}\\
	   = 0,01$

    \item The set of all discs that is a multiple of $5$ is: $\{5$; $ 10$; $ 15$; $ 20$; $ 25$; $ 30$; $ 35$; $ 40$; $ 45$; $ 50$; $ 55$; $ 60$; $ 65$; $ 70$; $ 75$; $ 80$; $ 85\}$. This set contains $17$ elements. Therefore the number of discs that are not multiples of $5$ is: $85 − 17 = 68$.\\
   
    The probability of drawing a disc that is not a multiple of $5$ is:\\
    $P(\mbox{not }5_m) = \frac{68}{85}\\
	       = 0,80$

    \item In part b, we worked out the probability for a disc that is a multiple of $3$. Now we work out the number of elements in the set of all discs that are multiples of $4$: $\{4$; $ 8$; $ 12$; $ 16$; $ 20$; $ 24$; $ 28$; $ 32$; $ 36$; $ 40$; $ 44$; $ 48$; $ 52$; $ 56$; $ 60$; $ 64$; $ 68$; $ 72$; $ 76$; $ 80$; $ 84\}$. This has $28$ elements.\\
    
    The probability that a disc is a multiple of either $3$ or $4$ is:\\
    $P(3_m \cup 4_m) = P(3_m) + P(4_m) - P(3_m \cap 4_m)\\
    = 0,33 + \frac{28}{85} - 0,33 \times \frac{28}{85}\\
    = 0,55$
    
    \item The set of all discs that are a multiples of $2$ and $6$ is the same as the set of all discs that are a multiple of $6$. Therefore the probability of drawing a disc that is both a multiple of $2$ and $6$ is:
 
    $0,16$
    
    \item There is only $1$ element in this set and so the probability is $0,01$.
    \end{enumerate}

\end{enumerate}}
\end{solutions}


\begin{eocexercises}{}
  \begin{enumerate}[itemsep=5pt, label=\textbf{\arabic*}.]
  \item 'n Groep van $45$ kinders is gevra of hulle  Frosties en/of
    Strawberry Pops eet. $31$ eet beide en $6$ eet net Frosties.  Wat is die waarskynlikheid dat 'n kind wat lukraak gekies word net Strawberry
    Pops eet?
  \item In 'n groep van $42$ leerders het almal behalwe $3$ 'n pakkie skyfies of 'n Fanta of beide gehad. As $23$ 'n pakkie skyfies gehad het, en $7$ hiervan het ook 'n Fanta gehad, wat is die waarskynlikheid dat 'n leerder wat willekeurig gekies word die volgende het:
    \begin{enumerate}[noitemsep, label=\textbf{(\alph*)} ]
    \item beide skyfies en Fanta
    \item net Fanta
    \end{enumerate}
  \item Gebruik 'n Venn-diagram om die volgende waarskynlikhede van 'n dobbelsteen wat gerol word te bereken:
    \begin{enumerate}[noitemsep, label=\textbf{(\alph*)} ]
    \item 'n veelvoud van $5$ en 'n onewe getal
    \item 'n getal wat nie 'n veelvoud van $5$ is nie en ook nie 'n onewe getal nie
    \item 'n getal wat nie 'n veelvoud van $5$ is nie, maar onewe.
    \end{enumerate}
  \item 'n Pakkie het geel en pienk lekkers. Die waarskynlikheid om 'n pienk lekker te kies is $\frac{7}{12}$.
Wat is die waarskynlikheid om 'n geel lekker uit te haal?

  \item In 'n parkeerterrein met $300$ motors is daar $190$ Opels. Wat is die waarskynlikheid dat die eerste motor wat uitry die volgende is:
    \begin{enumerate}[noitemsep, label=\textbf{(\alph*)} ]
    \item 'n Opel
    \item nie 'n Opel
    \end{enumerate}
  \item Tamara het $18$ los sokkies in 'n laai. Agt van hierdie is oranje en twee pienk. Bereken die waarskynlikheid dat die eerste sokkie wat willekeurig gekies word die volgende is:
    \begin{enumerate}[noitemsep, label=\textbf{(\alph*)} ]
    \item Oranje
    \item nie oranje
    \item pienk
    \item nie pienk
    \item oranje of pienk
    \item nie oranje of pienk
    \end{enumerate}
  \item Daar is $9$ brosbrood koekies, $4$ gemmerkoekies,
    $11$ sjokoladekoekies en $18$ Jambos op 'n bord. As 'n koekie willekeurig gekies word, wat is die waaarskynlikheid dat:
    \begin{enumerate}[noitemsep, label=\textbf{(\alph*)} ]
    \item dit 'n gemmerkoekie of 'n Jambo is?
    \item dit nie 'n brosbroodkoekie is nie?
    \end{enumerate}
  \item $280$ kaartjies is verkoop in 'n lotery. Ingrid het $15$ gekoop. Wat is die waarskynlikheid dat Ingrid:
    \begin{enumerate}[noitemsep, label=\textbf{(\alph*)} ]
    \item die prys wen
    \item nie die prys wen nie
    \end{enumerate}
  \item Die kinders in 'n kleuterskool is geklassifiseer per haarkleur en oogkleur. $44$ het rooi hare en nie bruin oe nie, $14$ het bruin oe maar nie rooi hare nie en $40$ het nie bruin oe of rooi hare nie.
    \begin{enumerate}[noitemsep, label=\textbf{(\alph*)} ]
    \item Hoeveel kinders is in die skool?
    \item Wat is die waarskynlikheid dat 'n kind wat willekeurig gekies word die volgende het:
      \begin{enumerate}[noitemsep, label=\roman*. ]
      \item bruin o\"e
      \item rooi hare
      \end{enumerate} 
    \item 'n Kind met bruin\"e word willekeurig gekies. Wat is die waarskynlikheid dat hierdie kind rooi hare het?
    \end{enumerate}
  \item 'n Houer bevat pers, blou en swart lekkers. Die waarskynlikheid dat 'n lekker wat willekeurig gekies word pers is, is $\frac{1}{7}$, en die waarskynlikheid dat dit swart sal wees is $\frac{3}{5}$.
    \begin{enumerate}[noitemsep, label=\textbf{(\alph*)} ]
    \item As ek 'n lekker willekeurig kies, wat is die waarskynlikheid dit die volgende sal wees:
      \begin{enumerate}[noitemsep, label=\roman*. ]
      \item pers of blou
      \item swart
      \item pers
      \end{enumerate}
    \item As daar $70$ lekkers in die houer is, hoeveel perses is daar?
    \item $\frac{2}{5}$ van die pers lekkers in (b) het strepe en die res nie, hoeveel pers lekkers het strepe?
    \end{enumerate}
  \item Vir elk van die volgende, teken 'n Venn-diagram om die situasie voor te stel en vind 'n voorbeeld ter illustrasie.
    \begin{enumerate}[noitemsep, label=\textbf{(\alph*)} ]
    \item 'n monsterruimte waarin daar twee gebeurtenisse is wat nie ondeling uitsluitend is nie.
    \item 'n monsterruimte waarin daar twee komplementere gebeurtenisse is.
    \end{enumerate}
  \item Gebruik 'n Venn-diagram om te bewys dat die waarskynlikheid van voorkoms van gebeurtenis  $A$ of $B$ ($A$ en $B$ is nie onderling uitsluitend nie) gegee word deur:
    \[P(A \cup B) = P(A) + P(B) - P(A \cap B)\]
  \item Al die klawerkaarte word uit 'n pak kaarte gehaal. Die res word dan geskommel en een kaart gekies. Die kaart word dan teruggeplaas voor die volgende een gekies word.
    \begin{enumerate}[noitemsep, label=\textbf{(\alph*)} ]
    \item Wat is die monsterruimte?
    \item Vind 'n versameling om gebeurtenis $P$, die trek van 'n prentkaart, voor te stel.
    \item Vind 'n versameling om gebeurtenis $N$, die trek van 'n nommerkaart, voor te stel.
    \item Vertoon die bostaande gebeurtenisse in 'n Venn-diagram.
    \item Wat is 'n goeie beskrywing van die versamelings $P$ en $N$?
      (Wenk: Vind enige elemente van $P$ in $N$ en van $N$ in $P$.)
    \end{enumerate}

%english questions below
  \item 'n Opname was by Mutende Primêre Skool uitgevoer om vas te stel hoeveel van die $ 650 $ leerders vetkoek, en hoeveel lekkers tydens pouse koop. Die volgende was bevind:
% A survey was conducted at Mutende Secondary school to establish how many of the $650$ learners buy vetkoek and how many buy sweets during break. The following was found:
\begin{itemize}
 \item $50$ leerders het niks gekoop nie
\item $400$ leerders het vetkoek gekoop
\item $300$ leerders het lekkers gekoop
\end{itemize}
\begin{enumerate}[noitemsep, label=\textbf{(\alph*)} ]
 \item Vertoon hierdie inligting met 'n Venn-diagram
\item Indien 'n leerder willekeurig gekies is, bereken die waarskynlikheid dat die leerder koop:
% If a learner is chose randomly, calculate the probability that this learner buys:
\begin{enumerate}[noitemsep, label=\roman*. ]
 \item net lekkers
\item net vetkoek
\item nie vetkoek en nie lekkers nie
\item vetkoek en lekkers
\item vetkoek of lekkers
\end{enumerate}
\end{enumerate}
\item In 'n opname by Lwandani se Sekondêre Skool is $80$ mense gevra of hulle die Sowetan of die Daily Sun of albei koerante lees. Die verslag toondat $45$ mense die Daily Sun lees, $30$ die Sowetan lees en $10$ lees nie een van hulle nie. Gebruik 'n Venn-diagram om die persentasie van mense te vind wat:
\begin{enumerate}[noitemsep, label=\textbf{(\alph*)} ]
 \item net die Daily Sun lees
\item net die Sowetan lees
\item die Daily Sun en die Sowetan lees
\end{enumerate}



  \end{enumerate}
% \practiceinfo
%   \begin{tabularx}{\textwidth}{XXXXX}
%     (1.) AAA & (2.) AAA & (3.) AAA& (4.) AAA& (5.) AAA\\
%     (6.) AAA& (7.) AAA& (8.) AAA& (9.) AAA& (10.) AAA\\
%     (11.) AAA& (12.) AAA& (13.) AAA \\
%   \end{tabularx}
\end{eocexercises}


 \begin{eocsolutions}{}{
\begin{enumerate}[itemsep=5pt, label=\textbf{\arabic*}. ] 


\item %solution 1

$45($All$) - 6($only Frosties$) - 31 ($both$) = 8 ($only Strawberry Pops$)$\\
Therefore $\frac{8}{45}=0,18$

\item %solution 2
    \begin{enumerate}[noitemsep, label=\textbf{(\alph*)} ]
    \item $\frac{7}{42} = \frac{1}{6}$

    \item Since $42 - 3 = 39$ had at least one, and $23 + 7$ had a packet of chips, then $39 - 30 = 9$ only had Fanta.\\
    $\frac{9}{42} = \frac{3}{14}$
    \end{enumerate}

\item %solution 3
	    \scalebox{0.8} % Change this value to rescale the drawing.
	    {
	    \begin{pspicture}(0,-1.8767188)(3.62,1.9167187)
	    \pscircle[linewidth=0.04,dimen=outer](1.81,-0.06671875){1.81}
	    \pscircle[linewidth=0.04,dimen=outer](1.08,-0.03671875){0.94}
	    \pscircle[linewidth=0.04,dimen=outer](2.01,1.2132813){0.45}
	    \usefont{T1}{ppl}{m}{n}
	    \rput(3.1245313,1.7132813){\LARGE$S$}
	    \usefont{T1}{ppl}{m}{n}
	    \rput(1.0045313,0.5132812){\LARGE$2$}
	    \usefont{T1}{ppl}{m}{n}
	    \rput(0.6045312,0.03328125){\LARGE$4$}
	    \usefont{T1}{ppl}{m}{n}
	    \rput(1.3245312,-0.34671876){\LARGE$6$}
	    \usefont{T1}{ppl}{m}{n}
	    \rput(2.8445313,-0.50671875){\LARGE$3$}
	    \usefont{T1}{ppl}{m}{n}
	    \rput(1.9845313,1.1732812){\LARGE$5$}
	    \usefont{T1}{ppl}{m}{n}
	    \rput(2.7645311,0.03328125){\LARGE$1$}
	    \pscircle[linewidth=0.04,dimen=outer](2.81,-0.20671874){0.69}
	    \end{pspicture}
	    }\\
    Multiples of $5 :5$\\
    Odd number: $1, 3, 5$\\
    Neither: $2, 4, 6$\\
    Both: $5$
    \begin{enumerate}[noitemsep, label=\textbf{(\alph*)} ]
    \item $\frac{1}{6}$
    \item $\frac{3}{6} = \frac{1}{2}$
    \item $\frac{2}{6} = \frac{1}{3}$
    \end{enumerate}
\item %solution 4\\
$1 - \frac{7}{12} = \frac{5}{12}$
\item %solution 5
    \begin{enumerate}[noitemsep, label=\textbf{(\alph*)} ]
    \item $\frac{190}{300} = \frac{19}{30}$
    \item $1 - \frac{19}{30} = \frac{11}{30}$
    \end{enumerate}
\item %solution 6
    \begin{enumerate}[noitemsep, label=\textbf{(\alph*)} ]
    \item $\frac{8}{18} = \frac{4}{9}$
    \item $1 - \frac{4}{9} = \frac{5}{9}$
    \item $\frac{2}{18} = \frac{1}{9}$
    \item $1 - \frac{1}{9} = \frac{8}{9}$
    \item $\frac{1}{9} + \frac{4}{9} = \frac{5}{9}$
    \item $1 - \frac{5}{9} = \frac{4}{9}$
    \end{enumerate}
\item %solution 7
    \begin{enumerate}[noitemsep, label=\textbf{(\alph*)} ]
    \item Total number of biscuits is $9 + 4 + 11 + 18 = 42$\\
    $\frac{4}{42} + \frac{18}{42} = \frac{22}{42}$\\
    $\frac{11}{21}$
    \item $1 - \frac{9}{42} = 1 - \frac{3}{14}$\\
    $\frac{11}{14}$
    \end{enumerate}
\item %solution 8
    \begin{enumerate}[noitemsep, label=\textbf{(\alph*)} ]
    \item $\frac{15}{280} = \frac{3}{56}$
    \item $1 - \frac{3}{56} = \frac{53}{56}$
    \end{enumerate}
\item %solution 9
    \begin{enumerate}[noitemsep, label=\textbf{(\alph*)} ]
    \item All $4$ groups are mutually exclusive, so total number of children is $44 + 14 + 5 + 40 = 103$.
    \item
	\begin{enumerate}[itemsep=1pt,  label=\textbf{\roman*}. ]
	\item $\frac{19}{103}$
	\item $\frac{58}{103}$
	\end{enumerate}
    \item $\frac{14}{(14 + 5)} = \frac{14}{19}$
    \end{enumerate}
\item %solution 10
    \begin{enumerate}[noitemsep, label=\textbf{(\alph*)} ]
    \item 
	\begin{enumerate}[itemsep=1pt,  label=\textbf{\roman*}. ]
	\item Same as not black: $1 - \frac{3}{5} = \frac{2}{5}$
	\item $\frac{3}{5}$
	\item $\frac{1}{7}$
	\end{enumerate}
    \item $\frac{1}{7} \times 70 = 10$
    \item $10 \times \frac{2}{5} = 4$
    \end{enumerate}
\item %solution 11
    \begin{enumerate}[noitemsep, label=\textbf{(\alph*)} ]
    \item
    			\scalebox{0.8} % Change this value to rescale the drawing.
			{
			\begin{pspicture}(0,-1.8767188)(3.62,1.9167187)
			\pscircle[linewidth=0.04,dimen=outer](1.81,-0.06671875){1.81}
			\pscircle[linewidth=0.04,dimen=outer](1.2,-0.07671875){0.94}
			\usefont{T1}{ppl}{m}{n}
			\rput(3.1245313,1.7132813){\LARGE$S$}
			\pscircle[linewidth=0.04,dimen=outer](2.45,-0.06671875){0.87}
			\end{pspicture}
			}
    \item
			\scalebox{0.8} % Change this value to rescale the drawing.
			{
			\begin{pspicture}(0,-1.8767188)(3.62,1.9167187)
			\pscircle[linewidth=0.04,dimen=outer](1.81,-0.06671875){1.81}
			\usefont{T1}{ppl}{m}{n}
			\rput(3.1245313,1.7132813){\LARGE$S$}
			\psline[linewidth=0.04cm](2.78,1.4032812)(0.88,-1.5767188)
			\end{pspicture}
			}
    \end{enumerate}

\item %solution 12

\scalebox{0.8} % Change this value to rescale the drawing.
{
    \begin{pspicture}(-2,-2)(3.5,2)
    \usefont{T1}{ppl}{m}{n}
    \rput(0.0,0.0){\LARGE$A$ not $B$}
    \rput(1.0,2.3){\LARGE$A$}
    \rput(3.0,2.3){\LARGE$B$}
    \pscircle[linewidth=0.04,dimen=outer](1.0,0.0){2.0}
    \usefont{T1}{ppl}{m}{n}
    \rput(4.0,0.0){\LARGE$B$ not $A$}
    \pscircle[linewidth=0.04,dimen=outer](3,0.0){2.0}
    \usefont{T1}{ppl}{m}{n}
    \rput(2.0,0.0){\LARGE$A$ and $B$}
    \end{pspicture}
}
\item 
    \begin{enumerate}[noitemsep, label=\textbf{(\alph*)} ]
    \item $\{$deck of cards without clubs$\}$
    \item $P = \{J$; $Q$; $K$ of hearts diamonds or spades$\}$
    \item $N = \{A$; $2$; $ 3$; $ 4$; $ 5$; $ 6$; $ 7$; $ 8$; $ 9$; $ 10$ of hearts, diamonds or spades$\}$
    \item
	      \scalebox{0.7} % Change this value to rescale the drawing.
			{
			\begin{pspicture}(0,-2.4442186)(8.7890625,2.4442186)
			\pscircle[linewidth=0.04,dimen=outer](2.0654688,0.26078126){1.37}
			\pscircle[linewidth=0.04,dimen=outer](5.905469,-0.17921875){2.21}
			\usefont{T1}{ptm}{m}{n}
			\rput(1.4245312,1.1407813){\LARGE$J\diamondsuit$}
			\usefont{T1}{ptm}{m}{n}
			\rput(1.9445312,0.50078124){\LARGE$Q\diamondsuit$}
			\usefont{T1}{ptm}{m}{n}
			\rput(2.5845313,-0.07921875){\LARGE$K\diamondsuit$}
			\usefont{T1}{ptm}{m}{n}
			\rput(2.1845312,-0.61921877){\LARGE$K\heartsuit$}
			\usefont{T1}{ptm}{m}{n}
			\rput(1.4645312,-0.33921874){\LARGE$K\spadesuit$}
			\usefont{T1}{ptm}{m}{n}
			\rput(2.9845314,0.68078125){\LARGE$Q\heartsuit$}
			\usefont{T1}{ptm}{m}{n}
			\rput(2.4,1.2807813){\LARGE$Q\spadesuit$}
			\usefont{T1}{ptm}{m}{n}
			\rput(1.3845313,0.7){\LARGE$J\heartsuit$}
			\usefont{T1}{ptm}{m}{n}
			\rput(1.1045313,0.22078125){\LARGE$J\spadesuit$}
			\usefont{T1}{ptm}{m}{n}
			\rput(4.924531,1.5207813){\LARGE$A\diamondsuit$}
			\usefont{T1}{ptm}{m}{n}
			\rput(6.374531,-1.2992188){\LARGE$2\diamondsuit$}
			\usefont{T1}{ptm}{m}{n}
			\rput(5.6345315,-1.6592188){\LARGE$3\diamondsuit$}
			\usefont{T1}{ptm}{m}{n}
			\rput(4.174531,-0.95921874){\LARGE$4\diamondsuit$}
			\usefont{T1}{ptm}{m}{n}
			\rput(4.494531,0.48078126){\LARGE$5\diamondsuit$}
			\usefont{T1}{ptm}{m}{n}
			\rput(6.7345314,1.4807812){\LARGE$6\diamondsuit$}
			\usefont{T1}{ptm}{m}{n}
			\rput(6.2145314,1.7607813){\LARGE$7\diamondsuit$}
			\usefont{T1}{ptm}{m}{n}
			\rput(5.334531,-0.33921874){\LARGE$8\diamondsuit$}
			\usefont{T1}{ptm}{m}{n}
			\rput(7.214531,0.06078125){\LARGE$9\diamondsuit$}
			\usefont{T1}{ptm}{m}{n}
			\rput(5.9245315,0.30078125){\LARGE$10\diamondsuit$}
			\usefont{T1}{ptm}{m}{n}
			\rput(4.824531,1.1007812){\LARGE$A\heartsuit$}
			\usefont{T1}{ptm}{m}{n}
			\rput(4.684531,0.82078123){\LARGE$A\spadesuit$}
			\usefont{T1}{ptm}{m}{n}
			\rput(5.774531,1.1607813){\LARGE$7\heartsuit$}
			\usefont{T1}{ptm}{m}{n}
			\rput(6.554531,0.98078126){\LARGE$7\spadesuit$}
			\usefont{T1}{ptm}{m}{n}
			\rput(7.394531,0.48078126){\LARGE$6\heartsuit$}
			\usefont{T1}{ptm}{m}{n}
			\rput(6.854531,0.68078125){\LARGE$6\spadesuit$}
			\usefont{T1}{ptm}{m}{n}
			\rput(7.5745316,-0.35921875){\LARGE$9\heartsuit$}
			\usefont{T1}{ptm}{m}{n}
			\rput(7.2345314,-0.67921877){\LARGE$9\spadesuit$}
			\usefont{T1}{ptm}{m}{n}
			\rput(6.124531,-0.11921875){\LARGE$10\heartsuit$}
			\usefont{T1}{ptm}{m}{n}
			\rput(6.284531,-0.63921875){\LARGE$10\spadesuit$}
			\usefont{T1}{ptm}{m}{n}
			\rput(4.6945314,0.12078125){\LARGE$5\heartsuit$}
			\usefont{T1}{ptm}{m}{n}
			\rput(4.134531,-0.23921876){\LARGE$5\spadesuit$}
			\usefont{T1}{ptm}{m}{n}
			\rput(4.774531,-0.63921875){\LARGE$8\heartsuit$}
			\usefont{T1}{ptm}{m}{n}
			\rput(5.454531,-1.0392188){\LARGE$8\spadesuit$}
			\usefont{T1}{ptm}{m}{n}
			\rput(4.7145314,-1.2592187){\LARGE$4\heartsuit$}
			\usefont{T1}{ptm}{m}{n}
			\rput(4.6945314,-1.5592188){\LARGE$4\spadesuit$}
			\usefont{T1}{ptm}{m}{n}
			\rput(5.2745314,-2.0192187){\LARGE$3\heartsuit$}
			\usefont{T1}{ptm}{m}{n}
			\rput(6.014531,-2.0){\LARGE$3\spadesuit$}
			\usefont{T1}{ptm}{m}{n}
			\rput(7.2745314,-1.0792187){\LARGE$2\heartsuit$}
			\usefont{T1}{ptm}{m}{n}
			\rput(6.7745314,-1.7792188){\LARGE$2\spadesuit$}
			\usefont{T1}{ptm}{m}{n}
			\rput(2.0845313,1.9007813){\LARGE$P$}
			\usefont{T1}{ptm}{m}{n}
			\rput(5.974531,2.2407813){\LARGE$N$}
			\end{pspicture} 
			}
\item Mutually exclusive and complementary
 
    \end{enumerate}
\item 
\begin{enumerate}[itemsep=5pt, label=\textbf{(\alph*)} ]
 \item 
\scalebox {0.8} % Change this value to rescale the drawing.
{
\begin{pspicture}(0,-1.97)(5.0,1.97)
\psframe[linewidth=0.04,dimen=outer](5.0,1.97)(0.0,-1.97)
\pscircle[linewidth=0.04,dimen=outer](1.78,0.35){1.08}
\pscircle[linewidth=0.04,dimen=outer](3.0,-0.25){1.0}
\usefont{T1}{ptm}{m}{n}
\rput(4.365,1.34){50}
\usefont{T1}{ptm}{m}{n}
\rput(1.59625,0.2){300}
\usefont{T1}{ptm}{m}{n}
\rput(3.4796875,-0.18){200}
\usefont{T1}{ptm}{m}{n}
\rput(2.3825,-0.04){100}
\usefont{T1}{ptm}{m}{n}
\rput(1.84625,1.12){V}
\usefont{T1}{ptm}{m}{n}
\rput(3.4192188,-0.88){S}
\end{pspicture} 
}
\item %If a learner is chose randomly, calculate the probability that this learner buys:
\begin{enumerate}[noitemsep, label=\textbf{(\roman*)} ]
\item $\frac{200}{650} = 30,8\%$
\item $\frac{300}{650} = 46,2\%$
\item $\frac{50}{650} = 7,7\%$
\item $\frac{100}{650} = 15,4\%$
\item $\frac{600}{650} = 92,3\%$
\end{enumerate}
\end{enumerate}
\item 
\scalebox{0.8} % Change this value to rescale the drawing.
{
\begin{pspicture}(0,-1.97)(5.0,1.97)
\psframe[linewidth=0.04,dimen=outer](5.0,1.97)(0.0,-1.97)
\pscircle[linewidth=0.04,dimen=outer](1.78,0.35){1.08}
\pscircle[linewidth=0.04,dimen=outer](3.0,-0.25){1.0}
\usefont{T1}{ptm}{m}{n}
\rput(4.3525,1.34){10}
\usefont{T1}{ptm}{m}{n}
\rput(1.5107813,0.2){40}
\usefont{T1}{ptm}{m}{n}
\rput(3.3842187,-0.18){25}
\usefont{T1}{ptm}{m}{n}
\rput(2.3895311,0.04){5}
\usefont{T1}{ptm}{m}{n}
\rput(1.7873437,1.02){D/S}
\usefont{T1}{ptm}{m}{n}
\rput(3.1070313,-0.84){D/B}
\end{pspicture} 
}
\begin{enumerate}[noitemsep, label=\textbf{(\alph*)} ]
 \item $\frac{40}{80} = 50\%$
\item $\frac{25}{80} = 31,25\%$
\item $\frac{5}{80} = 6,25\%$
\end{enumerate}
\end{enumerate}}
\end{eocsolutions}


