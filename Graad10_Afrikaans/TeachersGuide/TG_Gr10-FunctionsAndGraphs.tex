\chapter{Funksies}
% \section{Functions in the real world}
\begin{exercises}{}
{
\begin{enumerate}[noitemsep, label=\textbf{\arabic*}. ] 
\item Skryf die volgende in keurdernotasie:
\begin{multicols}{2}
\begin{enumerate}[noitemsep, label=\textbf{(\alph*)} ] 
 \item $(-\infty; 7]$
\item $[13; 4)$
\item $(35; \infty)$
\item $[\frac{3}{4}; 21)$
\item $[-\frac{1}{2}; \frac{1}{2}]$
\item $(-\sqrt{3}; \infty)$
\end{enumerate}
\end{multicols}
\item Skryf die volgende in intervalnotasie:
\begin{multicols}{2}
\begin{enumerate}[noitemsep, label=\textbf{(\alph*)} ] 
 \item $\{p: p \in \mathbb{R},~ p \leq 6\}$
 \item $\{k: k \in \mathbb{R},~ -5 < k < 5\}$
 \item $\{x: x \in \mathbb{R},~ x > \frac{1}{5}\}$
 \item $\{z: z \in \mathbb{R},~ 21 \leq z < 41\}$
\end{enumerate}
\end{multicols}
\end{enumerate}

} 
\end{exercises}


 \begin{solutions}{}{
\begin{enumerate}[itemsep=5pt, label=\textbf{\arabic*}. ] 
\item
\begin{multicols}{2}
\begin{enumerate}[noitemsep, label=\textbf{(\alph*)} ] 
 \item $\{x:x\in\mathbb{R}, x\leq7\}$ %$(- \infty; 7]$
\item $\{y:y\in\mathbb{R}, -13 \leq y<4\}$ %$[13;4)$
\item $\{z:z\in\mathbb{R}, z>35\}$ %$(35; \infty)$
\item $\{t:t\in\mathbb{R}, \frac{3}{4}\leq t <21\}$ %$[\frac{3}{4}; 21)$
\item $\{p:p\in\mathbb{R}, -\frac{1}{2}\leq p \leq \frac{1}{2}\}$ %$[-\frac{1}{2}; \frac{1}{2}]$
\item $\{m:m\in\mathbb{R}, m > -\sqrt{3}\}$  %$(-\sqrt{3}; \infty)$
\end{enumerate}
\end{multicols}
\item %Write the following in interval notation:
\begin{multicols}{2}
\begin{enumerate}[noitemsep, label=\textbf{(\alph*)} ] 
 \item $(-\infty; 6]$ %$\{p: p \in \mathbb{R},~ p \leq 6\}$
 \item $(-5; 5)$ %$\{k: k \in \mathbb{R},~ -5 < k < 5\}$
 \item $(\frac{1}{5}; \infty)$%$\{x: x \in \mathbb{R},~ x > \frac{1}{5}\}$
 \item $[21; 41)$%$\{z: z \in \mathbb{R},~ 21 \leq z < 41\}$
\end{enumerate}
\end{multicols}
\end{enumerate}}
\end{solutions}


\begin{exercises}{}
{
\begin{enumerate}[noitemsep, label=\textbf{\arabic*}. ] 
 \item Gee die $x$- en $y$-afsnitte van die volgende reguitlyngrafieke. Dui aan of die grafiek toeneem of afneem as $x$ toeneem:
\begin{multicols}{2}
      \begin{enumerate}[noitemsep, label=\textbf{(\alph*)} ] 
      \item $y=x+1$
      \item $y=x-1$
      \item $h(x)=2x-1$
      \item $y+3x=1$
      \item $3y-2x=6$
      \item$k(x)=-3$
      \item $x=3y$
      \item $\frac{x}{2} - \frac{y}{3} = 1$
      \end{enumerate}
\end{multicols}
\item Gee die vergelykings van elk van die volgende grafieke:
  \begin{enumerate}[noitemsep, label=\textbf{(\alph*)} ]  
  \item $a(x)$
  \item $b(x)$
  \item $p(x)$
  \item $d(x)$
  \end{enumerate} 
\setcounter{subfigure}{0}
\begin{figure}[H]
\begin{center}
\scalebox{1} % Change this value to rescale the drawing.
{
\begin{pspicture}(0,-4.1467185)(9.493593,4.1867185)
\rput(4.0,-0.1467186){\psaxes[linewidth=0.03,arrowsize=0.05291667cm 2.0,arrowlength=1.4,arrowinset=0.4,tickstyle=bottom,labels=none,ticks=none,ticksize=0.08cm]{<->}(0,0)(-4,-4)(4,4)}
\psline[linewidth=0.04cm](2.74,1.7332813)(7.72,-0.9467186)
\usefont{T1}{ptm}{m}{n}
\rput(4.1768746,3.9832811){$y$}
\usefont{T1}{ptm}{m}{n}
\rput(8.234531,-0.036718614){$x$}
\usefont{T1}{ptm}{m}{n}
\rput(4.4442186,1.3432813){$(0;3)$}
\usefont{T1}{ptm}{m}{n}
\rput(6.9642186,0.043281388){$(4;0)$}
\usefont{T1}{ptm}{m}{n}
\rput(8.274531,-0.9){$a(x)$}
\psline[linewidth=0.04cm](3.2,-3.6332815)(8.42,2.3067186)
\usefont{T1}{ptm}{m}{n}
\rput(4.6742187,-2.8167186){$(0;-6)$}
\usefont{T1}{ptm}{m}{n}
\rput(8.869062,2.4967186){$b(x)$}
\psline[linewidth=0.04cm](0.96,1.0667186)(7.86,1.0467187)
\usefont{T1}{ptm}{m}{n}
\rput(8.274531,1.1567186){$p(x)$}
\psline[linewidth=0.04cm](7.42,-2.0332813)(1.1,1.4667186)
\usefont{T1}{ptm}{m}{n}
\rput(7.8745313,-2.1032813){$d(x)$}
\usefont{T1}{ptm}{m}{n}
\rput(3.7745314,-0.39999998){$0$}
\psline[linewidth=0.04cm,arrowsize=0.113cm 4.0,arrowlength=1.4,arrowinset=0.4]{>>-}(3.16,0.30671862)(2.92,0.4467186)
\psline[linewidth=0.04cm,arrowsize=0.113cm 4.0,arrowlength=1.4,arrowinset=0.4]{>>-}(5.14,0.4067186)(4.9,0.5467186)
\end{pspicture} 
}
\end{center}
\end{figure}  
\item Skets die volgende funksies op dieselfde stel asse deur die afsnitte metode te gebruik. Dui die as-afsnitte duidelik aan asook die ko\"ordinate van die snypunt van die grafieke van $x+2y-5=0$ en $3x-y-1=0$.
\item Trek die grafieke van $f(x)=3-3x$ en $g(x)=\frac{1}{3}x+1$ met die gradi\"ent-afsnit metode.
\end{enumerate}

}
\end{exercises}


 \begin{solutions}{}{
\begin{enumerate}[itemsep=6pt, label=\textbf{\arabic*}. ] 

\item %List the $x$- en $y$-intercepts for the following straight line graphs. Indicate whether the graph is increasing or decreasing:
\begin{multicols}{2}
      \begin{enumerate}[noitemsep, label=\textbf{(\alph*)} ] 
    \item $(0;1)$ en $(-1;0)$; toenemend%$y=x+1$
      \item $(0;-1)$ en $(1;0)$; toenemend%$y=x-1$
      \item $(0;-1)$ en $(\frac{1}{2};0)$; toenemend%$h(x)=2x-1$
      \item $(0;1)$ en $(\frac{1}{3};0)$; afnemend%$y+1=2x$
      \item $(0;2)$ en $(-3;0)$; toenemend%$3y-2x=6$
      \item $(0;3)$; horisontale lyn%$k(x)=-3$
      \item $(0;0)$; toenemend%$x=3y$
      \item $(0;3)$ en $(2;0)$; afnemend%$\frac{x}{2} - \frac{y}{3} = 1$
      \end{enumerate}
\end{multicols}

\item %For the functions in the diagram below, give the equation en domain en range:
\begin{multicols}{2}   
 \begin{enumerate}[noitemsep, label=\textbf{(\alph*)} ]  

\item $a(x)= - \frac{3}{4}x+3$%$a(x)$
\item $b(x)=\frac{3}{2}x-6$%$b(x)$
\item $p(x)=3$%$c(x)$
\item $d(x)=-\frac{3}{4}x$%$d(x)$
\end{multicols}
    \end{enumerate}  
 

\item %Sketch the following functions on the same set of axes, using the dual intercept method. Clearly indicate the intercepts en the point of intersection of the two graphs: $x+2y-5=0$ en $3x-y-1=0$
\scalebox{1}{
\begin{pspicture}(-5,-5)(5,5)
%\psgrid
\psset{yunit=0.6,xunit=0.6}
\psaxes[arrows=<->, labels=none, ticks=none](0,0)(-6,-6)(7,4)
\psplot[plotstyle=curve,arrows=<->]{-4}{6}{x .5  neg mul 2.5 add}
\psplot[linecolor=gray,plotstyle=curve,arrows=<->]{-1}{2}{x 3  mul 1 sub}
\psdots(0,-1)(0.33,0)(5,0)(1,2)(0,2.5)(0,2.5)
\uput[r](1,2.1){\small$P(1;2)$}
\uput[u](5,0){\small$(5;0)$}
\uput[ur](0.33,-0.1){\small$(\frac{1}{3};0)$}
\uput[r](0,-1){\small$(0;-1)$}
\uput[l](0,2.3){\small$(0;\frac{5}{2})$}
\rput(7.4,0.3){\small$x$}
\rput(0.3, 4.2){\small$y$}
\rput(-0.3,-0.4){\small$0$}


\end{pspicture}
}
\item %On the same system of axes, draw the graphs of $f(x)=3-3x$ en $g(x)=\frac{1}{3}x+1$ using the gradient-intercept method.
\scalebox{1.3}{
\begin{pspicture}(-5,-5)(5,5)
%\psgrid
\psset{yunit=0.5,xunit=0.5}
\psaxes[arrows=<->, labels=none, ticks=none](0,0)(-5,-2)(4,7)
\psplot[plotstyle=curve,arrows=<->]{-1.3}{1.3}{x 3 neg mul 3 add}
\psplot[plotstyle=curve,arrows=<->]{-4}{4}{x 0.33 mul 1 add}
\psplot[plotstyle=curve,arrows=<->]{-1.3}{1.3}{x 3  mul 3 add}
\psdots(1,0)(0,1)(3,2)(0,3)(1,6)(-3,0)

\rput(4.2,0.3){\small$x$}
\rput(0.3, 7.2){\small$y$}
\uput[dl](0,0){\small$0$}
\psarc[linewidth=0.04,linecolor=gray](0.5,2){0.25}{0.0}{180.0}
\psarc[linewidth=0.04,linecolor=gray](1.5,2){0.25}{0.0}{180.0}
\psarc[linewidth=0.04,linecolor=gray](2.5,2){0.25}{0.0}{180.0}
\rput{180}(0.5,1){\psarc[linewidth=0.04,linecolor=gray](0,1){0.25}{0.0}{180.0}}
\rput{90}(1,1.5){\psarc[linewidth=0.04,linecolor=gray](0,1){0.25}{0.0}{180.0}}
\rput{90}(1,0.5){\psarc[linewidth=0.04,linecolor=gray](0,1){0.25}{0.0}{180.0}}
\rput{90}(1,2.5){\psarc[linewidth=0.04,linecolor=gray](0,1){0.25}{0.0}{180.0}}
\rput{90}(1,3.5){\psarc[linewidth=0.04,linecolor=gray](0,1){0.25}{0.0}{180.0}}
\rput{90}(1,4.5){\psarc[linewidth=0.04,linecolor=gray](0,1){0.25}{0.0}{180.0}}
\rput{90}(1,5.5){\psarc[linewidth=0.04,linecolor=gray](0,1){0.25}{0.0}{180.0}}
\psarc[linewidth=0.04,linecolor=gray](0.5,6){0.25}{0.0}{180.0}
\uput[r](1.2,-1){\small$f(x)$}
\uput[r](1.1,6.9){\small$h(x)$}
\uput[u](4.5,2){\small$g(x)$}
\uput[ur](0.8,0){\small$(1;0)$}
% \uput[ur](-0.1,1){\small$(0;1)$}
\uput[d](3,2.1){\small$(3;2)$}
\uput[r](-0.1,3){\small$(0;3)$}
\uput[dr](0.9,6){\small$(1;6)$}
\uput[d](-3,0){\small$(-3;0)$}
\end{pspicture}
}

\end{enumerate}}
\end{solutions}


% \section{Quadratic functions of the form $f(x)=a{x}^{2}+q$}
\begin{exercises}{}
{
\begin{enumerate}[noitemsep, label=\textbf{\arabic*}. ] 
\item  Wys dat indien $a<0$ sal die waardeversameling van $f(x)=ax^{2}+q$ $\{f(x):f(x) \leq q \}$ wees.
\item Skets die grafiek van die funksie $y=-x^{2}+4$ en toon al die afsnitte met die asse.
\item Twee parabole is geteken: $g(x)=y=ax^{2}+p$ en $h(x)=y=bx^{2}+q$.
\begin{center}
\begin{pspicture}(-5,-5)(5,1)
\psset{yunit=0.2,xunit=0.5}
\psaxes[arrows=<->,dx=2,Dx=2,dy=2,Dy=2, labels=none, ticks=none](0,0)(-10,-15)(10,28)
\psplot[plotstyle=curve,arrows=<->]{-5.5}{5.5}{x 2 exp  9 sub}
\psplot[plotstyle=curve,arrows=<->]{-5.5}{5.5}{x 2 exp 1 mul neg 23 add}
\rput(0.6,28){$y$}
\rput(0.6,-10){$-9$}
\rput(-5.5,7.5){$(-4;7)$} 
\rput(5.1,7.5){$(4;7)$}
\rput(5.3,-0.85){$3$}
\rput(5.6,17){$g$}
\rput(5.6,-4){$h$}
\rput(0.5, 24){$23$}
\rput (10.4, 0.2){$x$}
\rput(-0.5,-0.85){$0$}
\end{pspicture}
\end{center}
\begin{enumerate}[noitemsep, label=\textbf{(\alph*)} ] 
    \item Vind die waardes van $a$ en $p$.
    \item Vind die waardes van $b$ en $q$.
    \item Vind die waardes van $x$ waarvoor $g({x})\geq h({x})$.
    \item Vir watter waardes van $x$ is $g(x)$ toenemend?
\end{enumerate}
\end{enumerate}

}
\end{exercises}   


 \begin{solutions}{}{
\begin{enumerate}[itemsep=5pt, label=\textbf{\arabic*}. ] 


\item 
Omdat die vierkant van enige nommer altyd positief is, het ons: $x^2\geq0$\\
As ons met $a$ vermenigvuldig, waar $(a < 0)$ verander die rigting van die ongelykheid: $ax^2 \leq 0$\\
Tel $q$ by ann albei kante van die vergelyking gee $ax^2 + q \leq q$\\ 
Dus $f(x) \leq q$\\
Dit gee die waardeversameling $(-\infty; q)$.

\item 
\scalebox{1}{
\begin{pspicture}(-5,-5)(5,1)
%\psgrid
\psset{yunit=0.6,xunit=0.6}
\psaxes[arrows=<->, labels=none, ticks=none](0,0)(-4,-3)(4,5)
\psplot[plotstyle=curve,arrows=<->]{-2.4}{2.4}{x 2 exp 1 neg mul 4 add}
\psdots(0,4)(-1;0)(1;0)
% \uput[r](0,-2.7){$(0;-3)$}
\rput(0.3, 5.3){\small$y$}
\rput (4.2, 0.2){\small$x$}
\rput(-0.3,-0.3){\small$0$}
\rput(3,2.7){\small$y={-x}^{2}-4$}
\rput(0.8,4.2){\small$(0;4)$}
\rput(-2.9,0.35){\small$(-2;0)$}
\rput(2.7,0.35){\small$(2;0)$}
\end{pspicture}}
\item 
\begin{enumerate}[noitemsep, label=\textbf{(\alph*)} ] 
 
   \item $p$ is die $y$-afsnit, dus $p=-9$\\
Om $a$ te vind, gebruik ons een van die punte op die grafiek (b.v. $(4;7)$):\\
$y=ax^2-9\\
7= a(4^2) - 9\\
16a=16\\
\therefore a=1$
    \item 
$q$ is die $y$-afsnit, dus $q = 23$\\ 
Om $b$ te vind, gebruik ons een van die punte op die grafiek (b.v. $(4;7)$): 
$y=bx^2 =23\\
7= b(4^2) +23\\
16b=-16\\
\therefore b=-1$
    \item Dit is die punt waar $g$ bo $h$ lê. 
Van die grafiek, siens ons dat $g$ bo $h$ lê wanneer: $x \leq -4$ or $x \geq 4$ 
    \item $g$ word vermeerder van die draaipunt $(0;-9)$ af, m.a.w. vir $x\geq0$ %For what values of $x$ is $g$ increasing?
  
\end{enumerate}

\end{enumerate}}
\end{solutions}


% \section{Hyperbolic functions of the form $f(x)=\frac{a}{x}+q$}
\begin{exercises}{}
{
\begin{enumerate}[noitemsep, label=\textbf{\arabic*}. ] 
\item Gebruik grafiekpapier en teken die grafiek van  $xy=-6$.
    \begin{enumerate}[noitemsep, label=\textbf{(\alph*)} ] 
    \item Lê die punt $(-2; 3)$  op die grafiek? Gee ’n rede vir jou antwoord.
    \item As die $x$-waarde van ‘n punt op die grafiek gelyk is aan $0,25$ wat is die ooreenstemmende $y$-waarde?
    \item Wat gebeur met die $y$-waardes as die $x$-waardes baie groot word?
   % \item Bereken die kortste afstand van die oorsprong na die grafiek $P$.
    \item Gee die vergelykings van die asimptote.
    \item Met die lyn $y=-x$ as ’n lyn van simmetrie, watter punt is simmetries ten opsigte van $(-2; 3)$?
    \end{enumerate}
\item Skets die grafiek van  $h(x)=\frac{8}{x}$.
    \begin{enumerate}[noitemsep, label=\textbf{(\alph*)} ] 
    \item Hoe sal die grafiek $g(x)=\frac{8}{x}+3$ vergelyk met die grafiek van $h(x)=\frac{8}{x}$? Verduidelik jou antwoord.
    \item Skets die grafiek van $y=\frac{8}{x}+3$ op dieselfde assestelsel. Toon die asimptote, asse van simmetrie en die ko\"ordinate van een punt op die grafiek.
    \end{enumerate}
 \end{enumerate}

}
\end{exercises}


 \begin{solutions}{}{
 \begin{enumerate}[noitemsep, label=\textbf{\arabic*}. ] 
\item %Using graph paper, draw the graph of $xy=-6$.
\scalebox{0.8}{
\begin{pspicture}(-5,-5)(5,1)
%\psgrid
\psset{yunit=0.4,xunit=0.4}
\psaxes[arrows=<->, labels=none, ticks=none](0,0)(-10,-10)(10,10)
\psplot[plotstyle=curve,arrows=<->]{-8}{-0.6}{x -1 exp -6 mul}
\psplot[plotstyle=curve,arrows=<->]{8}{0.6}{x -1 exp -6 mul}
\psdots(-2.45,2.45)(2.45,-2.45)
\rput(0.3, 10.6){$y$}
\rput (10.6, 0.2){$x$}
\rput(-0.8,-0.8){$0$}

% \uput[ul](-2.45,2.45){\small$P$}
% \uput[u](2.45,-2.45){\small$Q$}
\uput[d](5,-2.45){$y=-\frac{6}{x}$}

\end{pspicture}}

    \begin{enumerate}[noitemsep, label=\textbf{(\alph*)} ] 
    \item %Does the point $(-2; 3)$ lie on the graph ? Give a reason for your answer.
$y=\frac{-6}{x}\\
xy = -6$\\
Vervang die waardes van die punt $(-2; 3)$ in die funskie:\\
$xy = (-2)(3) = -6$\\
Dit bevredig die vergelyiking dus  lê die punt op die grafiek.

    \item 
Vervang in die waarde van $x$:\\
$y=\frac{-6}{0,25} \\= -6 \times 4 \\
= -24$
%If the $x$-value of a point on the drawn graph is $0,25$ what is the corresponding $y$-value?    
\item Die $y$-waardes word verminder as die $x$-waardes baie groot word. Hoe grooter die noemer ($x$), hoe kleiner die resultaat van die breuk ($y$).%What happens to the $y$-values as the $x$-values become very large?

    \item Die grafiek is nie horisontale of vertikale geskuif, dus is die asimpototes $y=0$ en $x=0$.%Give the equations of the asymptotes.
    \item Rondom die as van simmetrie  $y=-x$, die punt simmetries met $(-2; 3)$ is $(-3;2)$.%With the line as line of symmetry, what is the point symmetrical to ?

    \end{enumerate}
\item %Draw the graph of $h(x)=\frac{8}{x}$.
\scalebox{0.8}{
\begin{pspicture}(-5,-5)(5,1)
%\psgrid
\psset{yunit=0.4,xunit=0.4}
\psaxes[arrows=<->, labels=none, ticks=none](0,0)(-10,-10)(10,10)
\psplot[plotstyle=curve,arrows=<->]{-8}{-0.8}{x -1 exp 8 mul}
\psplot[plotstyle=curve,arrows=<->]{8}{0.8}{x -1 exp 8 mul}
\psplot[plotstyle=curve,linecolor=gray, arrows=<->]{-8}{-0.8}{x -1 exp 8 mul 3 add}
\psplot[plotstyle=curve,linecolor=gray, arrows=<->]{8}{0.8}{x -1 exp 8 mul 4 add}
\psplot[plotstyle=curve,linecolor=gray, linestyle=dashed]{-8}{8}{x -1  mul 3 add}
\psdot(-4,1)
\rput(0.3, 10.6){$y$}
\rput (10.8, 0.2){$x$}
\rput(-0.8,-0.8){$0$}
\uput[u](-3.7,0.7){$(-4;1)$}
\psline[linewidth=0.5pt,linestyle=dashed](-10,3)(10,3)
% \uput[ul](-2.45,2.45){\small$P$}
% \uput[u](2.45,-2.45){\small$Q$}
\uput[d](3.5,5){$y=\frac{8}{x}$}
\uput[d](4.5,10.5){$y=\frac{8}{x}+3$}
\uput[ul](0,3){$3$}
\end{pspicture}}
    \begin{enumerate}[noitemsep, label=\textbf{(\alph*)} ] 
    \item Di grafiek $g(x)=\frac{8}{x}+3$  is die grafiek van $h(x)=\frac{8}{x}$, vertikale by $3$ eenhede op geskuif. Hulle sou dieselfde vorm gehad het maar die asimptoot van $g(x)$ sou $y=3$ gewees, in plaas van $y=0$ (vir $h(x)$) en die simmetrie-as sou $y=-x+3$ gewees in plaas van $y=-x$ (vir $h(x)$).%How would the graph compare with that of ? Explain your answer fully.
    \item (Grafiek van $y=\frac{8}{x}+3$ is op dieselfde assestelsel hierbo)%GRAPH ON SAME SET OF AXES ABOVE %Draw the graph of $y=\frac{8}{x}+3$ on the same set of axes, showing asymptotes, axes of symmetry en the coordinates of one point on the graph.
    \end{enumerate}
 \end{enumerate}


}
\end{solutions}


\begin{exercises}{ }
 {
 \begin{enumerate}[noitemsep, label=\textbf{\arabic*}. ] 
\item Skets die grafieke van $y=2^{x}$ en $y=(\frac{1}{2})^{x}$ op dieselfde assestelsel.
  \begin{enumerate}[noitemsep, label=\textbf{(\alph*)} ]
\item Is die $x$-as die asimptoot en/of simmetrie-as in albei grafieke? Verduidelik jou antwoord.
\item Watter grafiek word aangedui met die vergelyking $y=2^{-x}$ ? Verduidelik jou antwoord.
\item Los die vergelyking $2^{x}=(\frac{1}{2})^{x}$ met behulp van ’n skets op en kontroleer jou antwoord deur middel van substitusie.
\end{enumerate}
\item Die kurwe van die eksponensi\"ele funksie $f$ in die meegaande diagram sny die y-as by die punt $A(0; 1)$ en $B(2; 9)$ is op $f$.
\begin{center}
\begin{pspicture}(-3,-1)(4,4)
\psset{yunit=0.75,xunit=0.75}
\psaxes[arrows=<->](0,0)(-5,-1)(5,10)
\psplot[plotstyle=curve,arrows=<->]{-2}{2.1}{3 x exp}
\psdots(0,1)(2,9)
\rput(1,1){$A(0;1)$}
\rput(3,9){$B(2;9)$}
\rput(5.2,0.2){$x$}
\rput(0.2,10.2){$y$}
\rput(-0.3,-0.3){$0$}

\end{pspicture}
\end{center}
 \begin{enumerate}[noitemsep, label=\textbf{(\alph*)} ]
\item Bepaal die vergelyking van funksie $f$.
\item Bepaal die vergelyking van $h$, die refleksie van die kurwe van $f$ in die $x$-as.
\item Bepaal die waardeversameling van $h$.
\item Bepaal die vergelyking van  $g$, die refleksie van die kurwe van $f$ in die $y$-as.
\item Bepaal die vergelyking van $j$ indien $j$ 'n vertikale strekking is van $f$ met $+2$ eenhede.
\item Bepaal die vergelyking van $k$ indien $k$ 'n vertikale skuif is van $f$ met $-3$ eenhede.
\end{enumerate}
\end{enumerate}

}
\end{exercises}


 \begin{solutions}{}{
 \begin{enumerate}[noitemsep, label=\textbf{\arabic*}. ] 
\item %Draw the graphs of $y=2^{x}$ en $y=(\frac{1}{2})^{x}$ on the same set of axes.
\scalebox{1}{
\begin{pspicture}(-5,-1)(5,6)
%\psgrid
\psset{yunit=0.6,xunit=0.6}
\psaxes[labels=none,ticks=none,arrows=<->](0,0)(-3.5,-1)(3.5,5)
\psplot[linecolor=gray,plotstyle=curve,arrows=<->]{-2.5}{2}{2 x exp}
\psplot[plotstyle=curve,arrows=<->]{-2}{2.5}{0.5 x exp}
% \psdots(0,4)(1,0)
% \psline[linestyle=dashed](-5,6)(5,6)
% \rput(5.6,6.2){$y=6$}
\rput(3.7,0.2){$x$}
\rput(0.2,5.2){$y$}
\rput(2.7, 2.7){$y= 2^x$}
\rput(-2.7, 2.7){$y= \frac{1}{2}^{x}$}
\rput(-0.3,-0.3){$0$}
\rput(0.25, 1.55){$1$}
% \rput(0.8,4){$(0;4)$}
% \rput(1.5,0.5){$(1;0)$}
\end{pspicture}}
 \begin{enumerate}[noitemsep, label=\textbf{(\alph*)} ]
\item Die $x$-as is 'n asimptoot vir beide grafieke omdat beide na die $x$-as kom sonder om dit aan te raak.%Is the $x$-axis an asymptote or an axis of symmetry to both graphs? Explain your answer.

\item $y= \frac{1}{2}^{x}$ word verteenwoordig deur die vergelyking $y=2^{-x}$ omdat $\frac{1}{2}^{-1} = 2^{-1}$, wat gee $y = 2^{-x}$. %Which graph is represented by the equation  ? Explain your answer.
\item Die grafieke sny mekaar by die punt $(0;1)$. As ond hierdie waardes in beide kante van die vergelyking vervang, kry ons:\\
LK: $ 2^{x}=2^0 = 1$ en\\
RK: $\left(\frac{1}{2}\right)^x =\left(\frac{1}{2}\right)^0 = 1$\\
LK = RK, dus is die antwoord korrek.%S %Solve the equation $$ graphically en check that your answer is correct by using substitution.
\end{enumerate}
\item %The curve of the exponential function $f$ in the accompanying diagram cuts the y-axis at the point $
\begin{enumerate}[noitemsep, label=\textbf{(\alph*)} ]

\item %Determine the equation of the function $f$.
Die algemene vorm van die vergelyking is $f(x)=a^x+q$. \\
Ons is gegee $A(0; 1)$ en $B(2; 9)$.\\
Vervang in die waardes van punt $A$:\\
$1 = a^0 +q\\
1 = 1 +q\\
\therefore q=0$\\
Vervang in die waardes van punt $B$:\\
$9=a^2+0\\
a^2=3^2\\
\therefore a=3$\\
Dus is die vergelyking $f(x) = 3^x$.
\item $h(x)=-3^{x}$%Determine the equation of $h$, the reflection of of $f$ in the $x$-axis.
\item Waardeversameling: $(-\infty;0)$%Determine the range of $h$.
\item $g(x)=3^{-x}$%Determine the equation of $g$, the reflection of $g$ in the $y$-axis.
\item $j(x)=2.3^{x}$%Determine the equation of $j$ if $j$ is a vertical stretch of $f$ by $+2$ units.
\item $k(x)=3^{x}-3$%Determine the equation of $k$ if $k$ is a vertical shift of $f$ by $-3$ units.
\end{enumerate}
\end{enumerate}

}
\end{solutions}


% \section{Trigonometric functions}
% \subsection{Sine functions of the form $f(\theta)=a~sin~\theta+q$}
% \subsection{Cosine functions of the form $f(\theta)=a\cos\theta +q$}
% \subsection{Tangent functions of the form $f(\theta)=a\tan\theta+q$}
\begin{exercises}{}
 {
\begin{enumerate}[noitemsep, label=\textbf{\arabic*}. ] 
\item Deur jou kennis van die invloed van $a$ en $q$ te gebruik, skets elk van die volgende grafieke  sonder om ’n
tabel van waardes te gebruik, vir $\theta \in [{0}^{\circ };{360}^{\circ }]$.
\begin{multicols}{2}
 \begin{enumerate}[noitemsep, label=\textbf{(\alph*)} ]
\item $y=2\sin\theta $
\item $y=-4\cos\theta $
\item $y=-2\cos\theta +1$
\item $y=\sin\theta -3$
\item $y=\tan\theta -2$\item $y=2\cos\theta -1$
\end{enumerate}
\end{multicols}
 \item Gee die vergelykings van elk van die volgende grafieke:
 \begin{enumerate}[noitemsep, label=\textbf{(\alph*)} ]
\item
\begin{pspicture}(-2.5,-2)(5,2)
\psset{yunit=0.25}
\psaxes[Dx=180, dx=2, Dy=2, dy=4]{<->}(0,0)(-0.5,-5.1)(4.5,5.1)
\psplot[xunit=0.0111, plotpoints=500, arrows=->]{0}{360}{x cos -4 mul }
\uput[d](5.4,0.1){$x$ (in degrees)}
\uput[r](0,5.1){$y$}
\rput(-0.2,-0.7){$0$}
\end{pspicture}
\item
\begin{pspicture}(-0.3,-2)(5,2)
\psset{yunit=0.25}
\psaxes[Dx=90, dx=1, Dy=2, dy=4]{<->}(0,0)(-0.5,-5.1)(4.5,5.1)
\psplot[xunit=0.0111, plotpoints=500, arrows=->]{0}{360}{x sin 1 add 2 mul}
\uput[d](5.4,0.1){$x$ (in degrees)}
\uput[r](0,5.1){$y$}
\rput(-0.2,-0.7){$0$}
\end{pspicture}
\end{enumerate}
\end{enumerate}

}
\end{exercises}


 \begin{solutions}{}{
\begin{enumerate}[noitemsep, label=\textbf{\arabic*}. ] 
\item %Using your knowledge of the effects of $a$ en $q$, sketch each of the following graphs, without using a table of values, for $\theta \in [{0}^{\circ };{360}^{\circ }]$
\begin{multicols}{2}
 \begin{enumerate}[noitemsep, label=\textbf{(\alph*)} ]
\item %$y=2~sin~\theta $
\scalebox{0.8}{
\begin{pspicture}(-2.5,-2)(5,2)
\psset{yunit=0.5}
\psaxes[Dx=180, dx=2, Dy=1, dy=1, labels=none, ticks=x]{<->}(0,0)(-0.5,-2.6)(4.5,2.6)
\psplot[xunit=0.0111, plotpoints=500, arrows=->]{0}{360}{x sin 2 mul}
\uput[d](4.7,0){$x$}
\uput[r](0,2.3){$y$}
\uput[l](0,2){$2$}
\rput(-0.2,-0.7){$0$}
\rput(2,-0.7){$180^{\circ}$}
\rput(4,-0.7){$360^{\circ}$}
\psdots(1,2)(3,-2)
\end{pspicture}}

\item %$y=-4~cos~\theta $
\scalebox{0.8}{
\begin{pspicture}(-2.5,-2)(5,2)
\psset{yunit=0.25}
\psaxes[Dx=180, dx=2, Dy=1, dy=1, labels=none, ticks=x]{<->}(0,0)(-0.5,-5.1)(4.5,5.1)
\psplot[xunit=0.0111, plotpoints=500, arrows=->]{0}{360}{x cos -4 mul }
% \psline[linestyle=dotted](0,4)(4,4)
\psdots(2,4)
\uput[l](0,4){$4$}
\uput[d](4.7,0){$x$}
\uput[r](0,5.1){$y$}
\rput(-0.2,-0.7){$0$}
\rput(2,-1.4){$180^{\circ}$}
\rput(4,-1.4){$360^{\circ}$}
\end{pspicture}}

\item %$y=-2~cos~\theta +1$
\scalebox{0.8}{
\begin{pspicture}(-2.5,-2)(5,2)
\psset{yunit=0.25}
\psaxes[Dx=180, dx=2, Dy=1, dy=1, labels=none, ticks=x]{<->}(0,0)(-0.5,-5.1)(4.5,5.1)
\psplot[xunit=0.0111, plotpoints=500, arrows=->]{0}{360}{x cos -2 mul 1 add}
\uput[d](4.7,0){$x$}
\uput[r](0,5.1){$y$}
\rput(-0.2,-0.7){$0$}
\rput(2,-1.4){$180^{\circ}$}
\rput(4,-1.4){$360^{\circ}$}
\psdots(2,3)
% \psline[linestyle=dotted](0,3)(4,3)
\uput[l](0,3){$3$}
\end{pspicture}}

\item %$y=sin~\theta -3$
\scalebox{0.8}{
\begin{pspicture}(-2.5,-2)(5,2)
\psset{yunit=0.5}
\psaxes[Dx=180, dx=2, Dy=1, dy=1, labels=none, ticks=x]{<->}(0,0)(-0.5,-5.1)(4.5,2)
\psplot[xunit=0.0111, plotpoints=500, arrows=->]{0}{360}{x sin 3 sub}
\uput[d](4.7,0){$x$}
\uput[r](0,2.6){$y$}
\rput(-0.2,-0.7){$0$}
\rput(2,-0.7){$180^{\circ}$}
\rput(4,-0.7){$360^{\circ}$}
\uput[l](0,-2){$-2$}
\psdots(1,-2)(3,-4)
\end{pspicture}}

\item %$y=tan~\theta -2$
\scalebox{0.8}{
\begin{pspicture}(0,-2.5)(4,2)
\psset{yunit=0.5}
\psaxes[Dx=180, dx=2, Dy=1, dy=1, labels=none, ticks=x]{<->}(0,0)(-0.5,-5.1)(4.5,2)
\psline[linewidth=0.02,linestyle=dashed](1,-4)(1,2)
\psline[linewidth=0.02,linestyle=dashed](3,-4)(3,2)
% \psline[linewidth=0.02,linestyle=dashed](0,-1)(4.5,-1)
\psplot[xunit=0.0111,yunit=0.5, plotpoints=500, arrows=->]{0}{80}{x sin x cos div 2 sub}
\psplot[xunit=0.0111,yunit=0.5,plotpoints=500, arrows=<->]{100}{260}{x sin x cos div 2 sub}
\psplot[xunit=0.0111,yunit=0.5,plotpoints=500, arrows=<-]{280}{360}{x sin x cos div 2 sub}
\rput(0.2,2.3){$y$}
\rput(4.8,0.2){$x$}
\rput(-0.2,-0.4){$0$}
\rput(-0.5,-1){$-2$}
\rput(2,0.7){$180^{\circ}$}
\rput(4,0.7){$360^{\circ}$}
\psdots(2,-1)(4,-1)
\end{pspicture}}

\item %$y=2~cos~\theta -1$
\scalebox{0.8}{
\begin{pspicture}(-2.5,-2)(5,2)
\psset{yunit=0.25}
\psaxes[Dx=180, dx=2, Dy=1, dy=1, labels=none, ticks=x]{<->}(0,0)(-0.5,-5.1)(4.5,5.1)
\psplot[xunit=0.0111, plotpoints=500, arrows=->]{0}{360}{x cos 2 mul 1 sub}
\uput[d](4.7,0){$x$}
\uput[r](0,5.1){$y$}
\rput(-0.2,-0.7){$0$}
\rput(2,-1.1){$180^{\circ}$}
\rput(4,-1.1){$360^{\circ}$}
\psdots(2,-3)
\uput[l](0,-3){$-3$}
\end{pspicture}}
\end{enumerate}
\end{multicols}
\item %Give the equations of each of the following graphs:
 \begin{enumerate}[noitemsep, label=\textbf{(\alph*)} ]
\item$y=2~cos~\theta$
\item$y=sin~\theta+1$
%\item$y=-tan~\theta+5$
   \end{enumerate}
\end{enumerate}

\end{enumerate}}
\end{solutions}


% \section{Interpretation of graphs}
\begin{eocexercises}{}
  \begin{enumerate}[itemsep=9pt, label=\textbf{\arabic*}. ] 
\item Skets die grafieke van die volgende: 
\begin{multicols}{3}
    \begin{enumerate}[noitemsep, label=\textbf{(\alph*)} ]
    \item $y=2x+4$ 
    \item $y-3x=0$ 
    \item $2y=4-x$
    \end{enumerate}
\end{multicols}
\item Skets die volgende funksies: 
\begin{multicols}{3}
    \begin{enumerate}[noitemsep, label=\textbf{(\alph*)} ] % \setcounter{enumi}{3} 
    \item $y=x^{2}+3$ 
    \item $y=\frac{1}{2}x^{2}+4$
    \item $y=2x^{2}-4$
    \end{enumerate}
\end{multicols}
  \item Skets die volgende funksies en identifiseer die asimptote:
\begin{multicols}{3}
    \begin{enumerate}[noitemsep, label=\textbf{(\alph*)} ]  % \setcounter{enumi}{6} 
    \item $y=3^{x}+2$ 
    \item $y=-4 \times 2^{x}$ 
    \item $y=\left(\dfrac{1}{3}\right)^{x}-2$ 
    \end{enumerate}
\end{multicols}
 \item Skets die volgende funksies en identifiseer die asimptote: 
\begin{multicols}{3}
    \begin{enumerate}[noitemsep, label=\textbf{(\alph*)} ] % \setcounter{enumi}{9} 
    \item $y=\dfrac{3}{x}+4$ 
    \item $y=\dfrac{1}{x}$ 
    \item $y=\dfrac{2}{x}-2$ 
    \end{enumerate}
\end{multicols}
  \item Bepaal of die volgende bewerings waar of vals is. As 'n bewering vals is, gee redes hoekom
    \begin{enumerate}[noitemsep, label=\textbf{(\alph*)} ]
    \item  Die gegewe of gekose y-waarde staan bekend as die afhanklike veranderlike.
    \item 'n Grafiek wat geen onderbrekings het nie, word kongruent genoem
    \item Funksies van die vorm $y=ax+q$ is reguitlyne.
    \item Funksies van die vorm $y=\frac{a}{x}+q$ is eksponensiële funksies.
    \item 'n Asimptoot is 'n reguit of gekromde lyn wat 'n grafiek ten minste een keer sny. 
    \item Gegee die funksie in die vorm $y=ax+q$ ,word die $y$-afsnit gevind deur $x=0$ te stel en op te los vir $y$.
    \end{enumerate}
\item Gegee die funksies $f(x)=-2{x}^{2}-18$ en $g(x)=-2x+6$:
    \begin{enumerate}[noitemsep, label=\textbf{(\alph*)} ]
    \item Skets $f$ en $g$ op dieselfde assestelsel.
    \item Bereken die snypunte van $f$ en $g$.
    \item Gebruik nou die grafieke en hulle snypunte om vir $x$ op te los wanneer:
	\begin{enumerate}[noitemsep, label=\textbf{\roman*}. ] 
	\item $f(x)>0$
	\item $\frac{f(x)}{g(x)}\leq 0$
	\end{enumerate}
    \item Gee die vergelyking van die refleksie van $f$ in die $x$-as.
    \end{enumerate}
\item Nadat ’n bal laat val word, is die hoogte wat die bal terugbons elke keer minder. Die vergelyking $y=5{(0,8)}^{x}$ toon die verwantskap tussen
 $x$, die nommer van die bons, en $y$, die hoogte van die bons vir ’n
spesifieke bal. Wat is die benaderde hoogte van die vyfde bons tot die naaste tiende van ’n eenheid?\newline
\item Mark het $15$ muntstukke in R$~5$ en R$~2$-stukke. Hy het $3$ meer R$~2$-stukke as R$~5$-stukke. Hy het ‘n stelsel van vergelykings opgestel om die situasie te toon, waar $x$  die aantal R$~5$-stukke voorstel en $y$ die aantal R$~2$-stukke. Hy het vervolgens die probleem grafies opgelos.
    \begin{enumerate}[noitemsep, label=\textbf{(\alph*)} ]
    \item Skryf die sisteem van vergelykings neer.
    \item Skets die grafieke op dieselfde assestelsel.
    \item Wat is die oplossing?
    \end{enumerate}
%lots of english
 \item Skets die grafieke van die volgende trigonometriese funksies as
    $\theta \in[~0^{\circ};360^{\circ}]$. Wys die afsnitte en asimptote.
    \begin{enumerate}[noitemsep, label=\textbf{(\alph*)} ]  % \setcounter{enumi}{27} 
    \item $y=-4~cos~\theta$
    \item $y=sin~\theta -2$
    \item $y=-2~sin~\theta +1$
    \item $y=tan~\theta+2$
    \item $y=\dfrac{cos~\theta}{2}$
    \end{enumerate}
  \item As die algemene vergelykings $y=mx+c$, $y=ax^2+q$, $y=\frac{a}{x}+q$, $y=a~sin~x+q$, $y=a^x +q$ en $y=a~tan~x$ gegee is, bereken die spesifieke vergelykings vir elkeen van die volgende grafieke
\vspace{20pt}\\
    \begin{center}
      \begin{table}[H]
        \begin{tabular}{m{6cm}m{6cm}}
%           \hline
          \begin{center}
            \scalebox{0.8}{
              \begin{pspicture}(-5,-5)(5,1)
                %\psgrid
                \psset{yunit=0.5,xunit=0.5}
                \psaxes[arrows=<->, labels=none, ticks=none](0,0)(-6,-7)(6,6)
                \psline[linewidth=0.02, linestyle=dashed](-2,0)(-2,-6)
                \psline[linewidth=0.02, linestyle=dashed](-0,-6)(-2,-6)
                \psplot[plotstyle=curve,arrows=<->]{-2.2}{2}{x 3 mul}
                \rput(-5,5){\textbf{(a)}}
                \psdots(-2,-6)
                \rput(0.3, 6.3){$y$}
                \rput(6.2, 0.2){$x$}
                \rput(-0.37,-0.3){$0$}
                \rput(-3.5,-6){$(-2;-6)$}
              \end{pspicture}
            }
          \end{center}
          &
          \begin{center}
            \scalebox{0.8}{
              \begin{pspicture}(-5,-5)(5,1)
                %\psgrid
                \psset{yunit=0.5,xunit=0.5}
                \psaxes[arrows=<->, labels=none, ticks=none](0,0)(-5,-5)(5,5)
                \psplot[plotstyle=curve,arrows=<->]{-1.7}{1.7}{x 2 exp -2 mul 3 add}
                \rput(-5,5){\textbf{(b)}}
                \psdots(1,1)(0,3)
                \rput(0.3, 5.3){$y$}
                \rput(5.2, 0.2){$x$}
                \rput(-0.37,-0.3){$0$}
                \rput(1.5,1.5){$(1;1)$}

\rput(0.8,3.3){$(0;3)$}
            \end{pspicture}}
          \end{center}
          \\ %\hline
   
        \end{tabular}
      \end{table}
    \end{center}

   \begin{center}
      \begin{table}[H]
        \begin{tabular}{m{6cm}m{6cm}}
     
          \begin{center}
            \scalebox{0.8}{
              \begin{pspicture}(-5,-5)(5,1)
                %\psgrid
                \psset{yunit=0.5,xunit=0.5}
                \psaxes[arrows=<->, labels=none, ticks=none](0,0)(-5,-5)(5,5)
                \psplot[plotstyle=curve,arrows=<->]{-4.5}{-0.6}{x -1 exp -3 mul}
                \psplot[plotstyle=curve,arrows=<->]{0.6}{4.5}{x -1 exp -3 mul }
                \psdots(3,-1)
                \rput(-5,5){\textbf{(c)}}
                \rput(0.3, 5.3){$y$}
                \rput(5.2, 0.2){$x$}
                \rput(-0.37,-0.3){$0$}
                \rput(3.6,-1.5){$(3;-1)$}
            \end{pspicture}}
          \end{center}
          &
          \begin{center}
            \scalebox{0.8}{
              \begin{pspicture}(-5,-5)(5,1)
                %\psgrid
                \psset{yunit=0.5,xunit=0.5}
                \psaxes[arrows=<->, labels=none, ticks=none](0,0)(-6,-2)(6,7)
                \psplot[plotstyle=curve,arrows=<->]{-3.5}{5}{x 2 add}
                \rput(-5,7){\textbf{(d)}}
                \psdots(0,2)(4,6)
                \rput(0.3, 7.3){$y$}
                \rput(6.2, 0.2){$x$}
                \rput(-0.37,-0.3){$0$}
                \rput(1,2){$(0;2)$}
                \rput(5,6){$(4;6)$}
            \end{pspicture}}
          \end{center}
          \\ %\hline
          
          \begin{center}
            \scalebox{0.8}{
              \begin{pspicture}(-6,-5)(5,6)
                \psset{yunit=0.4, xunit=1}
                \psset{xunit=0.01111}
                \psaxes[dx=30,Dx=30, labels=none, ticks=none]{->}(0,0)(-45,-4.5)(400,6.5)
                \psplot[plotstyle=curve, plotpoints=300, linewidth=1pt, arrows=->]{0}{360}{x sin 5 mul 1 add}  
                \rput(415, 0.2){$x$}
                \rput(15, 6.5){$y$}
                \rput(-90,5){\textbf{(e)}}
                \psline[linewidth=0.02, linestyle=dashed](0,6)(360,6)
                \psline[linewidth=0.02, linestyle=dashed](0,1)(360,1)
                \psline[linewidth=0.02, linestyle=dashed](0,-4)(360,-4)
                \rput(-30,6){$6$}
                \rput(-30,1){$1$}
                \rput(-30,-4){$-4$}
                \psdots(195,0)(350,0)
                \rput(160,-0.8){$180^{\circ}$}
                \rput(340,-0.8){$360^{\circ}$}
                \rput(-20,-0.4){$0$}
              \end{pspicture}
            }
          \end{center}
          &
          \begin{center}
            \scalebox{0.8}{
              \begin{pspicture}(-5,-5)(5,1)
                %\psgrid
                \psset{yunit=0.5,xunit=0.5}
                \psaxes[arrows=<->, labels=none, ticks=none](0,0)(-5,-3)(5,5)
                \psplot[plotstyle=curve,arrows=<->]{-4}{0.8}{2 x exp 2 mul 1 add}
                \psdots(0,3)
                \psline[linewidth=0.02,linestyle=dashed](-5,1)(5,1)
                \rput(-5,5){\textbf{(f)}}
                \rput(0.3, 5.3){$y$}
                \rput(5.2, 0.2){$x$}
                \rput(-0.37,-0.3){$0$}
                \rput(6,1){$y=1$}
                \rput(0.8,3){$(0;3)$}
            \end{pspicture}}
          \end{center}
          \\ %\hline
          \begin{center}
            \scalebox{0.8}{
              \begin{pspicture}(0,-2.5)(4,2)
                \psaxes[Dx=180, dx=2, Dy=1, dy=0.5, labels=none, ticks=x]{<->}(0,0)(-1,-2)(4.5,2)
                \rput(-1,3){\textbf{(g)}}
                \psline[linewidth=0.02,linestyle=dashed](1,-2.5)(1,2)
                \psline[linewidth=0.02,linestyle=dashed](3,-2.5)(3,2)
                \psline[linewidth=0.02,linestyle=dashed](0,-1)(4.5,-1)
                \psplot[xunit=0.0111,yunit=0.5, plotpoints=500, arrows=->]{0}{70}{x sin x cos div -1 mul 2 sub}
                \psplot[xunit=0.0111,yunit=0.5,plotpoints=500, arrows=<->]{100}{250}{x sin x cos div -1 mul 2 sub}
                \psplot[xunit=0.0111,yunit=0.5,plotpoints=500, arrows=<-]{280}{360}{x sin x cos div -1 mul 2 sub}
                \rput(0.2,2.3){$y$}
                \rput(4.5,0.2){$x$}
                \rput(-0.5,-1){$-2$}
  \rput(2,-0.4){$180^{\circ}$}
                \rput(4,-0.4){$360^{\circ}$}
                \psdots(1.5,-0.5)
                \rput(1.65,-0.8){\footnotesize$(135^{\circ};-1)$}
              \end{pspicture}
            }
          \end{center}
          
          \\ %\hline
          
        \end{tabular}
      \end{table}
    \end{center}

  \item $y=2^x$ en $y=-2^x$ is hieronder geskets. Antwoord die vrae wat volg:
\begin{center}
    \scalebox{0.9}{
      \begin{pspicture}(-5,-5)(5,1)
        \psset{yunit=1.3,xunit=1.3}
        \psaxes[arrows=<->, labels=none, ticks=none](0,0)(-2.5,-2.5)(2,2.5)
        \psplot[plotstyle=curve,arrows=<->]{-1.7}{0.6}{2 x exp}
        \psplot[plotstyle=curve,arrows=<->]{-1.7}{0.6}{2 x exp -1 mul}
        \psdots(0,1)(0,-1)(-1,0.5)(-1,-0.5)
        \rput(0.2, 2.6){$y$}
        \rput(2.1, 0.2){$x$}
        \rput(-0.2,-0.2){$0$}
        \rput(0.3,1){$M$}
        \rput(0.3,-1){$N$}
        \rput(-1,0.8){$P$}
        \rput(-1,-0.8){$Q$}
        \rput(-1.2,0.15){$R$}
        \psline[linewidth=0.01,linestyle=dashed, dash=0.10cm 0.10cm](-1,0.5)(-1,-0.5)
        \psline[linewidth=0.01](-1,0.2)(-0.8,0.2)
        \psline[linewidth=0.01](-0.8,0.2)(-0.8,0)
    \end{pspicture}}
\end{center}
    \\
    \begin{enumerate}[noitemsep, label=\textbf{(\alph*)} ]
    \item Bereken die ko\"ordinate van $M$ en $N$.
    \item Bereken die lengte $MN$.
    \item Bereken lengte van $PQ$ as $OR$ $1$ eenheid is.
    \item Gee die vergelyking as $y=2^x$ weerspieël word in die $y$-as.
    \item Gee die waardeversameling van elke grafieke.
    \end{enumerate}
\item$f(x)=4^x$ en $g(x)=4x^2+q$ is hieronder geskets. Die punte $A(0;1)$, $B(1;4)$ is $C$ gegee. Antwoord die vrae wat volg:
  \begin{center}  
    \scalebox{1}{
      \begin{pspicture}(-5,-5)(5,1)
        \psset{yunit=0.5,xunit=1}
        \psaxes[arrows=<->, labels=none, ticks=none](0,0)(-2.5,-5)(2,5)
        \psplot[plotstyle=curve,arrows=<->]{-1}{1.2}{4 x exp}
        \psplot[plotstyle=curve,arrows=<->]{-1.2}{1.2}{x 2 exp -4 mul 1 add}
        \psdots(0,1)(1,4)(1,-3)
        \rput(0.2, 5.2){$y$}
        \rput(2.2, 0.1){$x$}
        \rput(-0.3,1.3){$A$}
        \rput(1.2,4){$B$}
        \rput(1.3,-3){$C$}
        \rput(-0.2,-0.4){$0$}
        \rput(2,5){$f(x)=4^x$}
        \rput(2.5,-4){$g(x)=-4x^2+q$}
        \psline[linewidth=0.01,linestyle=dashed, dash=0.10cm 0.10cm](1,4)(1,-3)
        \psline[linewidth=0.02](1,0.4)(1.2,0.4)
        \psline[linewidth=0.02](1.2,0.4)(1.2,0)
    \end{pspicture}}
\end{center}
    \\
    \begin{enumerate}[noitemsep, label=\textbf{(\alph*)} ]
      % \setcounter{enumi}{44}
    \item Bepaal die waarde van $q$.
    \item Bereken die lengte van $BC$.
    \item Gee die vergelyking van $f(x)$ weerspieël in die $x$-as.
    \item Gee die vergelyking van $f(x)$ $1$ eenheid vertikaal opwaarts geskuif.
    \item Gee die vergelyking van die asimpotote van $f(x)$.
    \item Gee die waardeversamelings van $f(x)$ en $g(x)$.
    \end{enumerate}
 \item Skets die grafieke van $h(x)=x^2-4$ en $k(x)=-x^2+4$ op dieselfde assestelsel en antwoord die vrae wat volg: 
    \begin{enumerate}[noitemsep, label=\textbf{(\alph*)} ]
    \item Beskryf die verhouding tussen $h$ en $k$.
    \item Gee die vergelyking van $k(x)$ weerspieël in die lyn $y=4$.
    \item Gee die definisieversameling en die waardeversameling van $h$.
    \end{enumerate}
\item Skets die grafieke van $f(\theta)=2~ sin~\theta$ en $g(\theta)=cos~\theta-1$ op dieselfde assestelsel. Gebruik jou skets om te bepaal:
    \begin{enumerate}[noitemsep, label=\textbf{(\alph*)} ]
    \item $f(180^{\circ})$
    \item $g(180^{\circ})$
    \item $g(270^{\circ}) -f(270^{\circ})$
    \item Die definisieversameling en die waardeversameling van $g$.
    \item Die amplitude en periode van $f$.
    \end{enumerate}
  \end{enumerate}
\end{eocexercises}


 \begin{eocsolutions}{}{
\begin{multicols}{2}
  \begin{enumerate}[itemsep=8pt, label=\textbf{\arabic*}. ] 
\item %Sketch the graphs of the following: 
\scalebox{1}{
\begin{pspicture}(-5,-1)(5,5)
% label the axes
%\psgrid-\infty ;q
\psset{yunit=0.6, xunit=0.6}
\psaxes[arrows=<->, dy=1,Dy=1,dx=1,Dx=1](0,0)(-4,-4)(5,5)

\psplot[linecolor=gray,linestyle=dashed,plotstyle=curve,arrows=<->]{-3}{0.5}{x 2 mul 4 add}
\psplot[linecolor=gray,plotstyle=curve,arrows=<->]{-1}{1.4}{x 3 mul}
\psplot[plotstyle=curve,arrows=<->]{-3}{4.7}{x -0.5 mul 2 add}
% \psdots(-2,-2)(-1,-1)(0,0)(1,1)(2,2)

% \rput(-0.2, -0.3){$0$}
\rput(5.3, 0.1){$x$}
\rput(0.1, 5.3){$y$}
\uput[l](-3,-2){\color{gray}\textbf{(a)}}
\uput[l](-3,3.5){\textbf{(c)}}
\uput[ur](1.3,3.5){\color{gray}\textbf{(b)}}
\uput[dl](0,0){$0$}
\end{pspicture}}
%  \begin{enumerate}[noitemsep, label=\textbf{(\alph*)} ]
%     \item %$y=2x+4$ 
%     \item %$y-3x=0$ 
%     \item %$2y=4-x$
%     \end{enumerate}

\item %Sketch the following functions: 
\scalebox{1}{
\begin{pspicture}(-5,-1)(5,5)
% label the axes
%\psgrid-\infty ;q
\psset{yunit=0.6, xunit=0.6}
\psaxes[arrows=<->, dy=1,Dy=1,dx=1,Dx=1](0,0)(-4,-5)(5,8)

\psplot[linecolor=gray,linestyle=dashed,plotstyle=curve,arrows=<->]{-2}{2}{x 2 exp 3 add}
\psplot[linecolor=gray,plotstyle=curve,arrows=<->]{-2.3}{2.3}{x 2 exp 0.5 mul 4 add}
\psplot[plotstyle=curve,arrows=<->]{-2}{2}{x 2 exp 2 mul 4 sub}
\rput(5.3, 0.1){$x$}
\rput(0.1, 8.3){$y$}
\uput[r](0.4,3.1){\color{gray}\textbf{(a)}}
\uput[r](0.7,-3){\textbf{(c)}}
\uput[r](1.6,5.5){\color{gray}\textbf{(b)}}
\uput[dl](0,0){$0$}
\end{pspicture}}
%  \begin{enumerate}[noitemsep, label=\textbf{(\alph*)} ]
% % \setcounter{enumi}{3} 
%     \item %$y=x^{2}+3$ 
%     \item %$y=\frac{1}{2}x^{2}+4$
%     \item %$y=2x^{2}-4$
%     \end{enumerate}

\item %Sketch the following functions en identify the asymptotes: 
\scalebox{1}{
\begin{pspicture}(-5,-1)(5,5)
% label the axes
%\psgrid-\infty ;q
\psset{yunit=0.6, xunit=0.6}
\psaxes[arrows=<->, dy=1,Dy=1,dx=1,Dx=1](0,0)(-5,-6)(5,8)

\psplot[linecolor=gray,linestyle=dashed, plotstyle=curve,arrows=<->]{-4}{1.5}{3 x exp 2 add}
\psplot[linecolor=gray,plotstyle=curve,arrows=<->]{-4}{0.5}{2 x exp -4 mul}
\psplot[plotstyle=curve,arrows=<->]{-1.5}{3}{0.33 x exp 2 sub}
\rput(5.3, 0.1){$x$}
\rput(0.1, 8.3){$y$}
\uput[r](1.4,6){\color{gray}\textbf{(a)}}
\uput[r](2.8,-2){\textbf{(c)}}
\uput[l](-1.5,-1.5){\color{gray}\textbf{(b)}}
\uput[dr](0,0){$0$}
\end{pspicture}}


%  \begin{enumerate}[noitemsep, label=\textbf{(\alph*)} ] 
% % \setcounter{enumi}{6} 
%     \item %$y=3^{x}+2$ 
%     \item %$y=-4.2^{x}$ 
%     \item %$y=\left(\dfrac{1}{3}\right)^{x}-2$ 
%     \end{enumerate}

\item %Sketch the following functions en identify the asymptotes: 
\scalebox{1}{
\begin{pspicture}(-5,-1)(5,5)
% label the axes
%\psgrid-\infty ;q
\psset{yunit=0.3, xunit=0.3}
\psaxes[arrows=<->, dy=2,Dy=2,dx=2,Dx=2](0,0)(-10,-10)(10,10)

\psplot[linecolor=gray,linestyle=dashed,plotstyle=curve,arrows=<->]{-8}{-0.6}{x -1 exp 3 mul 4 add}
\psplot[linecolor=gray,plotstyle=curve,arrows=<->]{-6}{-0.3}{x -1 exp}
\psplot[plotstyle=curve,arrows=<->]{-8}{-0.5}{x -1 exp 2 mul 2 sub}
\psplot[linecolor=gray,linestyle=dashed,plotstyle=curve,arrows=<->]{0.6}{8}{x -1 exp 3 mul 4 add}
\psplot[linecolor=gray,plotstyle=curve,arrows=<->]{0.3}{5}{x -1 exp}
\psplot[plotstyle=curve,arrows=<->]{0.5}{8}{x -1 exp 2 mul 2 sub}
% \psline[linecolor=lightgray](-8,4)(8,4)
\rput(10.6, 0.1){$x$}
\rput(0.1, 10.6){$y$}
\uput[r](4,5.5){\color{gray}\textbf{(a)}}
\uput[dr](2.8,-2){\textbf{(c)}}
\uput[ur](3,0.4){\color{gray}\textbf{(b)}}
\uput[dr](0,0){$0$}
\end{pspicture}}
%  \begin{enumerate}[noitemsep, label=\textbf{(\alph*)} ]
% % \setcounter{enumi}{9} 
%     \item %$y=\dfrac{3}{x}+4$ 
%     \item %$y=\dfrac{1}{x}$ 
%     \item %$y=\dfrac{2}{x}-2$ 
%     \end{enumerate}
\end{enumerate}
\end{multicols}

\begin{enumerate}[itemsep=8pt, label=\textbf{\arabic*}. ] 
\setcounter{enumi}{4}

\item %Determine whether the following statements are true or false. If the statement is false, give reasons why:
  \begin{enumerate}[noitemsep, label=\textbf{(\alph*)} ]
   \item Vals - die $y$-waarde wat gegee is, is die afhanklike veranderlike omdat sy waarde hang af van die onafhanklike veranderlike $x$.%The given or chosen $y$-value is known as the independent variable.
    \item Vals - 'n grafiek is \textit{deurlopende} as daar geen breke daarin is.%A graph is said to be congruent if there are no breaks in the graph.
    \item Waar.%Functions of the form $y=ax+q$ are straight lines.
    \item Vals - funksies van die vorm $y=\frac{a}{x}+q$ is hiperboliese funskies.
    \item Vals - 'n asimpotoot is 'n reguitlyn wat 'n grafiek \textbf{nooit} nie sal deursny nie.% An asymptote is a straight line which a graph will intersect at least once.
    \item Waar.%Given a function of the form $y=ax+q$, to find the $y$-intercept let $x=0$ en solve for $y$.
    \end{enumerate}
\item %Given the functions $f(x)=-2{x}^{2}-18$ en $g(x)=-2x+6$
 \begin{enumerate}[noitemsep, label=\textbf{(\alph*)} ]

    \item %Draw $f$ en $g$ on the same set of axes.
\scalebox{1}{
\begin{pspicture}(-5,-1)(5,5)
% label the axes
%\psgrid-\infty ;q
\psset{yunit=0.3, xunit=0.5}
\psaxes[arrows=<->, dy=2,Dy=2,dx=2,Dx=2](0,0)(-6,-6)(6,15)

\psplot[linecolor=gray,plotstyle=curve,arrows=<->]{-3}{3}{x 2 exp 2 mul 5 sub}
\psplot[plotstyle=curve,arrows=<->]{-5}{5}{x -2 mul 6 add}
% \psdots(-2,-2)(-1,-1)(0,0)(1,1)(2,2)

% \rput(-0.2, -0.3){$0$}
\rput(6.3, 0.1){$x$}
\rput(0.1, 15.6){$y$}
\uput[r](3,9.5){\color{gray}$f(x)$}
\uput[r](5,-4){$g(x)$}
\uput[dl](0,0){$0$}
\end{pspicture}
}
\item Die $x$-waardes van die kruispunte kan gevind word om deur $f(x)=g(x)$ te sit:\\
$-2x^2 + 18 = -2x +6\\
x^2-9 =x-3\\
x^2 -x - 6=0\\
(x+2)(x-3) = 0\\ 
\therefore x=-2 $ en $x=3$.\\
Die $y$-waardes word gevind  om deur die $x$-waardes in enige van die twee vergelykings te vervang:\\
$g(x) = -2(3)+6 = 0\\
g(x) = -2(-2)+6=10\\$
Dus is die kruispunte $(3;0)$ en $(-2;10)$%Calculate the points of intersection of $f$ en $g$.

\item %Hence use your graphs en the points of intersection to solve for $x$ when:
% \begin{enumerate}[noitemsep, label=\textbf{(\alph*)} ]
\begin{enumerate}[noitemsep, label=\textbf{\roman*}. ] 
\item $
f(x)>0\\
−2x^2+18>0\\
2x^2<18\\
x2<9\\
−3<x<3\\$
%$f(x)>0$

\item $

\frac{−2x^2+18}{−2x+6}\leq0\\
\frac{x^2−9}{x−3} \leq 0\\
\frac{(x−3)(x+3)}{x−3} \leq0 \\
x+3\leq0\\
x\leq−3$

\end{enumerate}
	\item $y=2x^2-18$%Give the equation of the reflection of $f$ in the $x$-axis.
\end{enumerate}
	 \item 

$y=5(0,8)^x\\
= 5(\frac{4}{5})^x\\
$ if $x = 5$ then $y = 5(\frac{4}{5})^5\\
= 5(\frac{1024}{3125})\\
=5(0,38)\\
=1,6$ units%After a ball is dropped, the rebound height of each bounce decreases. The equation $y=5{(0,8)}^{x}$ shows the relationship between the number of bounces ($x$) en the height of the bounce ($y$) for a certain ball. 

% What is the approximate height of the fifth bounce of this ball to the nearest tenth of a unit ?


\item %Mark had $15$ coins in R$~5$ en R$~2$ pieces. He had $3$ more R$~2$ coins than R$~5$ coins. He wrote a system of equations to represent this situation, letting $x$ represent the number of R$~5$ coins en $y$ represent the number of R$~2$ coins. Then he solved the system by graphing.
	 \begin{enumerate}[noitemsep, label=\textbf{(\alph*)} ]
% \setcounter{enumi}{24} 
    \item $x+y=15$; $y=x+3$%Write down the system of equations.
    \item %Draw their graphs on the same set of axes.
\scalebox{1}{
\begin{pspicture}(-5,-1)(5,5)
% label the axes
%\psgrid-\infty ;q
\psset{yunit=0.2, xunit=0.2}
\psaxes[arrows=<->, dy=3,Dy=3,dx=2,Dx=2](0,0)(-10,-10)(20,20)

\psplot[plotstyle=curve,arrows=<->]{-2}{18}{x -1 mul 15 add}
\psplot[linecolor=gray,plotstyle=curve,arrows=<->]{-7}{14}{x 3 add}
% \psdots(-2,-2)(-1,-1)(0,0)(1,1)(2,2)
\psline[linewidth=1pt,linestyle=dotted](0,9)(6,9)
\psline[linewidth=1pt,linestyle=dotted](6,9)(6,0)
% \rput(-0.2, -0.3){$0$}
\rput(20.8, 0.1){$x$}
\rput(0.1, 20.8){$y$}
\uput[r](12,4){\small$y=-x+15$}
\uput[r](12,14){\small\color{gray}$y=x+3$}
% \uput[dl](0,0){$0$}
\end{pspicture}}
 \item 
Vervang waarde van $y=-x+15$ in tweede vergelyking:\\
$-x+15=x+3\\
-2x = -12\\
\therefore x=6$\\
Vervang waarde van $x$ terug in eerste vergelyking:\\
$y = -(6) + 15\\
= 9$\\
Mark het $6$ R $5$ muntstukke en $9$ R~$2$ muntstukke.
    \end{enumerate}

\item %Sketch graphs of the following trigonometric functions for $\theta \in[0;360]$ (show intercepts en asymptotes):
\begin{multicols}{2}
	 \begin{enumerate}[noitemsep, label=\textbf{(\alph*)} ] 
 
    \item %$y=-4~cos~\theta$
\scalebox{0.9}{
\begin{pspicture}(-2.5,-2)(5,2)
\psset{yunit=0.25}
\psaxes[Dx=180, dx=2, Dy=1, dy=1, labels=none, ticks=x]{<->}(0,0)(-0.5,-5.1)(4.5,5.1)
\psplot[xunit=0.0111, plotpoints=500, arrows=->]{0}{360}{x cos -4 mul }
% \psline[linestyle=dotted](0,4)(4,4)
\psdots(2,4)
\uput[l](0,4){$4$}
\uput[d](4.7,0){$x$}
\uput[r](0,5.1){$y$}
\rput(-0.2,-0.7){$0$}
\rput(2,-1.4){$180^{\circ}$}
\rput(4,-1.4){$360^{\circ}$}
\end{pspicture}}
    

\item %$y=sin~\theta -2$
\scalebox{.9}{
\begin{pspicture}(-2.5,-2)(5,2)
\psset{yunit=0.5}
\psaxes[Dx=180, dx=2, Dy=1, dy=1, labels=none, ticks=x]{<->}(0,0)(-0.5,-2.6)(4.5,2.6)
\psplot[xunit=0.0111, plotpoints=500, arrows=->]{0}{360}{x sin 2 sub}
\uput[d](4.7,0){$x$}
\uput[r](0,2.3){$y$}
\uput[l](0,-2){$-2$}
\rput(-0.2,-0.7){$0$}
\rput(2,-0.7){$180^{\circ}$}
\rput(4,-0.7){$360^{\circ}$}
\psdots(1,-1)(3,-3)
\end{pspicture}}

\item %$y=-2~sin~\theta +1$
\scalebox{.9}{
\begin{pspicture}(-2.5,-2)(5,2)
\psset{yunit=0.25}
\psaxes[Dx=180, dx=2, Dy=1, dy=1, labels=none, ticks=x]{<->}(0,0)(-0.5,-5.1)(4.5,5.1)
\psplot[xunit=0.0111, plotpoints=500, arrows=->]{0}{360}{x sin -2 mul 1 add}
\uput[d](4.7,0){$x$}
\uput[r](0,5.1){$y$}
\rput(-0.2,-0.7){$0$}
\rput(2,-1.4){$180^{\circ}$}
\rput(4,-1.4){$360^{\circ}$}
\psdots(1,-1)(3,3)
% \psline[linestyle=dotted](0,3)(4,3)
\uput[l](0,3){$3$}
\end{pspicture}}

\item %$y=tan~\theta+2$
\scalebox{.9}{
\begin{pspicture}(0,-2.5)(4,2)
\psset{yunit=0.5}
\psaxes[Dx=180, dx=2, Dy=1, dy=1, labels=none, ticks=x]{<->}(0,0)(-0.5,-2)(4.5,5)
\psline[linewidth=0.02,linestyle=dashed](1,-2)(1,4)
\psline[linewidth=0.02,linestyle=dashed](3,-2)(3,4)
% \psline[linewidth=0.02,linestyle=dashed](0,-1)(4.5,-1)
\psplot[xunit=0.0111,yunit=0.5, plotpoints=500, arrows=->]{0}{80}{x sin x cos div 2 add}
\psplot[xunit=0.0111,yunit=0.5,plotpoints=500, arrows=<->]{100}{260}{x sin x cos div 2 add}
\psplot[xunit=0.0111,yunit=0.5,plotpoints=500, arrows=<-]{280}{360}{x sin x cos div 2 add}
\rput(0.2,5.3){$y$}
\rput(4.8,0.2){$x$}
\rput(-0.2,-0.4){$0$}
\rput(-0.3,1){$2$}
\rput(2,-0.7){$180^{\circ}$}
\rput(4,-0.7){$360^{\circ}$}
\psdots(2,1)(4,1)
\end{pspicture}}

\item %$y=\dfrac{cos~\theta}{2}$
\scalebox{.9}{
\begin{pspicture}(-2.5,-2)(5,2)
\psset{yunit=1}
\psaxes[Dx=180, dx=2, Dy=1, dy=1, labels=none, ticks=x]{<->}(0,0)(-0.5,-1)(4.5,1)
\psplot[xunit=0.0111, plotpoints=500, arrows=->]{0}{360}{x cos 0.5 mul }
\uput[d](4.7,0){$x$}
\uput[r](0,1.1){$y$}
\rput(-0.2,-0.7){$0$}
\rput(2,-0.3){$180^{\circ}$}
\rput(4,-0.3){$360^{\circ}$}
\psdots(2,-0.5)
\uput[l](0,0.5){$\frac{1}{2}$}
\end{pspicture}}
\end{enumerate}
\end{multicols}
\item %Given the general equations $y=mx+c$, $y=ax^2+q$, $y=\frac{a}{x}+q$, $y=a~sin~x+q$, $y=a^x +q$ en $y=a~tan~x$, determine the
    %specific equations for each of the following graphs:\vspace{20pt}\\
\begin{multicols}{2}
 \begin{enumerate}[noitemsep, label=\textbf{(\alph*)} ] 
    \item $y=3x$
    \item $y=-2x^2 +3$
    \item $y = \frac{-3}{x}$
    \item $y=x+2$
    \item $y=5~sin~x +1$
    \item $y=2.2^x + 1$
    \item $y = -tan~x -2$
    \end{enumerate}
\end{multicols}
\item %$y=2^x$ en $y=-2^x$ are sketched below. Answer the questions that follow:\\

 \begin{enumerate}[noitemsep, label=\textbf{(\alph*)} ]
\item As $M$, $y=2^0\\=1\\$ dus die ko\"ordinate van $M$ is $(0;1)$\\
As $N$, $y= -(2^0)\\ = -1\\$ dus die ko\"ordinate van $N$ is $(0;-1)$%Calculate the coordinates of $M$ en $N$.
\item $MN=1+1=2$ eenhede%Calculate the length of $MN$.
\item By $P$, $x=-1$\\
$\therefore y = 2^{-1} = \frac{1}{2}.\\
$ By $Q$, $y=-(2^{-1}) = -\frac{1}{2}$.\\
Dus lengte $PQ = \frac{1}{2} + \frac{1}{2} =1$ eenheid.%Calculate length $PQ$ if $OR=1$ unit.
\item $y=2^{-x}$%Give the equation of $y=2^x$ reflected about the $y$-axis.
\item Waardeversameling $y=2^x$: $(0;\infty)$;\\
      Waardeversameling $y=-2^x$: $(-\infty;0)$%Give the range of both graphs.
\end{enumerate}

\item %$f(x)=4^x$ en $g(x)=4x^2+q$ en sketched below. The points $A(0;1)$, $B(1;4)$ en $C(1,-3)$ are given. Answer the questions that follow:\\

\begin{multicols}{2}
 \begin{enumerate}[noitemsep, label=\textbf{(\alph*)} ]
\item $q=1$%C
\item $BC=7$ eenhede%.
\item $y=-4^{x}$%
\item $y=4^{x}+1$%
\item $x=0$%Give the range of both graphs.
\item Waardeversameling $f(x)$: $(0;\infty)$\\
      Waardeversameling $g(x)$: $(-\infty;1]$
\end{enumerate}
\end{multicols}
\item %Sketch the graphs $h(x)=x^2-4$ en $k(x)=-x^2+4$ on the same set of axes en answer the questions that follow: 
\scalebox{1}{
\begin{pspicture}(-5,-1)(5,5)
% label the axes
%\psgrid-\infty ;q
\psset{yunit=0.5, xunit=0.5}
\psaxes[arrows=<->, dy=1,Dy=1,dx=1,Dx=1](0,0)(-4,-5)(5,6)
\psplot[linecolor=gray,plotstyle=curve,arrows=<->]{-3}{3}{x 2 exp -1 mul 4 add}
\psplot[plotstyle=curve,arrows=<->]{-3}{3}{x 2 exp 4 sub}
\rput(5.3, 0.1){$x$}
\rput(0.1, 6.3){$y$}

\uput[r](1,-3){$k(x)$}
\uput[r](1,4){\color{gray}$h(x)$}
\uput[dl](0,0){$0$}
\end{pspicture}}
 \begin{enumerate}[noitemsep, label=\textbf{(\alph*)} ]
% \newcounter{enumii_saved}
%       \setcounter{enumii_saved}{\value{enumii}}
% \setcounter{enumii}{1}
 \item $h(x) = x^2 - 4$\\
$k(x) = -x^2+4\\=-(x^2-4)\\ = -h(x)$.\\
$k(x)$ is dus die spieëlbeeld $h(x)$ rondom die $x$-as. %Describe the relationship between $h$ en $k$.%Describe the relationship between $h$ en $k$.
\item $y=x^2+4$%Give the equation of $k(x)$ reflected about the line $y=4$.
\item Definisieversameling $h$: $(-\infty;\infty)$\\
      Waardeversameling $h$: $[-4;\infty)$%Give the domain en range of $h$.
\end{enumerate}

\item %Sketch the graphs $f(\theta)=2~sin~\theta$ en $g(\theta)=cos~\theta-1$ on the same set of axes. Use your sketch to determine:
\scalebox{1}{
\begin{pspicture}(-2.5,-2)(5,2)
\psset{yunit=0.5}
\psaxes[Dx=180, dx=2, Dy=1, dy=1, labels=none, ticks=x]{<->}(0,0)(-0.5,-3)(4.5,3)
\psplot[linecolor=gray,xunit=0.0111, plotpoints=500, arrows=->]{0}{360}{x cos 1 sub}
\psplot[xunit=0.0111, plotpoints=500, arrows=->]{0}{360}{x sin 2 mul}
\uput[d](4.7,0){$x$}
\uput[r](0,2.7){$y$}
\rput(-0.2,-0.7){$0$}
\rput(2,-0.7){$180^{\circ}$}
\rput(4,-0.7){$360^{\circ}$}
\psdots(1,2)(2,-2)(3,-2)
\uput[l](0,2){$2$}
\uput[l](0,-2){$-2$}
\end{pspicture}}
 \begin{enumerate}[noitemsep, label=\textbf{(\alph*)} ]
\item $f(180^{\circ})=0$%$f(180^{\circ})$
\item $g(180^{\circ})=-2$%$g(180^{\circ})$
\item $g(270^{\circ}) -f(270^{\circ})=-1-(02)\\
=-1+2\\=1$%$g(270^{\circ}) -f(270^{\circ})$
\item Definisieversameling: $[0^{\circ}$; $360^{\circ}]$\\
      Waardeversameling: $[-2;0]$%The domain en range of $g$.
\item Amplitude $=2$;\\
      Periode $=360^{\circ}$%The amplitude en period of $f$.
\end{enumerate}


\end{enumerate}}
\end{eocsolutions}



